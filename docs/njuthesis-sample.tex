%%%%%%%%%%%%%%%%%%%%%%%%%%%%%%%%%%%%%%%%%%%%%%%%%%%%%%%%%%%%%%%%%%%%%%
% njuthesis 示例模板 v1.4.2 2024-11-08
% https://github.com/nju-lug/NJUThesis
%
% 贡献者
% Yu XIONG @atxy-blip   Yichen ZHAO @FengChendian
% Song GAO @myandeg     Chang MA @glatavento
% Yilun SUN @HermitSun  Yinfeng LIN @linyinfeng
% Yukai Chou @Muzimuzhi
%
% 许可证
% LaTeX Project Public License(版本 1.3c 或更高)
%%%%%%%%%%%%%%%%%%%%%%%%%%%%%%%%%%%%%%%%%%%%%%%%%%%%%%%%%%%%%%%%%%%%%%

%---------------------------------------------------------------------
% 一些提升使用体验的小技巧:
%   1. 请务必使用 UTF-8 编码编写和保存本文档
%   2. 请务必使用 XeLaTeX 或 LuaLaTeX 引擎进行编译
%   3. 不保证接口稳定,写作前一定要留意版本号
%   4. 以百分号(%)开头的内容为注释,可以随意删除
%---------------------------------------------------------------------

%---------------------------------------------------------------------
% 请先阅读使用手册:
% http://mirrors.ctan.org/macros/unicodetex/latex/njuthesis/njuthesis.pdf
%---------------------------------------------------------------------

\PassOptionsToPackage{%%%%%%%%%%%%%%%%%%%%%%%
	backref=true,
}{biblatex}%%%%%%%%%%%%%%%%%%%%%%%

%\PassOptionsToPackage{%%%%%%%%%%%%%%%%%%%%%%%
%	linkcolor=blue, % normal internal links
%	anchorcolor=black, % anchor (target) text
%	citecolor=magenta, % bibliographical citations
%	filecolor=green, % open local files
%	urlcolor=cyan % linked URLs
%}{hyperref}%%%%%%%%%%%%%%%%%%%%%%%

\documentclass[
    % 模板选项(注意右端逗号):
    %
    type = doctor,
    % type = bachelor|master|doctor|postdoc, % 文档类型,默认为本科生
    % degree = academic|professional,        % 学位类型,默认为学术型
    %
    % nl-cover,   % 是否需要国家图书馆封面,默认关闭
    % decl-page,  % 是否需要诚信承诺书或原创性声明,默认关闭
    %
    %   页面模式,详见手册说明
    % draft,                  % 开启草稿模式
    % anonymous,              % 开启盲审模式
    % minimal,                % 开启最小化模式
    %
    %   单双面模式,默认为适合印刷的双面模式
    % oneside,                % 单面模式,无空白页
    % twoside,                % 双面模式,每一章从奇数页开始
    %
    %   字体设置,不填写则自动调用系统预装字体,详见手册
    % fontset = win|mac|macoffice|fandol|none,
    % latin-font = win|mac|gyre|none,
    % cjk-font   = win|mac|fandol|founder|noto|source|none,
    % math-font  = cambria|newcm|xits, % 完整列表见手册
%    biblatex = false,
%    unicode-math = false,
  ]{njuthesis}

% 模板选项设置,包括个人信息、外观样式等
% 较为冗长且一般不需要反复修改,我们把它放在单独的文件里
\input{njuthesis-setup.def}

\njusetformat{chapter}{\raggedleft\kaishu\zihao{-1}}

% 自行载入所需宏包
% \usepackage{subcaption} % 嵌套小幅图像,比 subfig 和 subfigure 更新更好
% \usepackage{siunitx} % 标准单位符号
\usepackage{physics} % 物理百宝箱
% \usepackage[version=4]{mhchem} % 绘制分子式
% \usepackage{listings} % 展示代码
% \usepackage{algorithm,algorithmic} % 展示算法伪代码

%%%%%%%%%%%%%%%%%%%%%%% my packages
\usepackage{hyperref} % 只读的 njuthesis.cls 中,已经用 RequirePackage 定义了
\usepackage{cases}
\usepackage{accents}
\usepackage{xeCJKfntef}
\usepackage{stmaryrd}
\usepackage{esvect}
\usepackage{fontspec,xunicode-addon} % fontspec 需要 XeLaTeX;xunicode-addon 带圈数字
\usepackage{graphicx}  % resize 表格的宽度 不超过文本宽度
\usepackage{booktabs}  % 经典三线表 \toprule、\midrule、\bottomrule
\usepackage{cancel}

%\usepackage{amsmath,amssymb,amsfonts}%
%\usepackage{amsthm}%
%\usepackage{mathrsfs}%
%\usepackage[title]{appendix}%
\usepackage[table,dvipsnames]{xcolor}% [dvipsnames] for custom .sty;
\usepackage{tikzsetup}
\usepackage{bbm}%
% -------------------- currently these custom packages are coupled together, sharing some public \cmds, so the package below may rely on the one above
\usepackage{secbacklinktoc}% custom .sty
\usepackage{optsection}% custom .sty
%\usepackage{tocformat}% custom .sty
\usepackage{apptoc}% custom .sty
\usepackage{bibbackref}% custom .sty
\usepackage{amseqfloatbackref}% custom .sty
\usepackage{textbackref}% custom .sty
\usepackage{footnotebacklink}% custom .sty

% 在导言区随意定制所需命令
% \DeclareMathOperator{\spn}{span}
% \NewDocumentCommand\mathbi{m}{\textbf{\em #1}}

%%%%%%%%%%%%%%%%%%%%%%% my commands
\makeatletter
\def\ubar{\underaccent{{\underline{\mskip10mu}}}}
\def\uddot{\underaccent{\"}} % 选这个 还是选 \undertilde{} ← 这个类似 underline
\makeatother

\def\sf{\fontspec{Lato-Italic.ttf}} % \text{\sf g}
\setmathtt{Arial Italic} % 需要 \usepackage{fontspec}(现在已经用 isomath 来替代)

\allowdisplaybreaks[4] % 允许 align 环境自动跨页
\ctexset{
	chapter = {
		%		format={\centering\sffamily},
		%		nameformat={\heiti\zihao{3}},
		%		titleformat={\heiti \zihao {3}},
		%		numberformat={\heiti \zihao {3}},
		beforeskip={0pt},afterskip={36pt},
		%		name={第, 章},number={\arabic {chapter}}
	},
	section = {
		%		format={\flushleft \sffamily \heiti \zihao {4}},
		beforeskip={40pt},afterskip={16pt},
	},
	subsection = {
		%		format={\flushleft\sffamily\heiti\zihao{-4}},
		beforeskip={24pt},afterskip={16pt},
	},   
	subsubsection = {
		%		format={\flushleft\sffamily\heiti\zihao{-4}},
		beforeskip={24pt},afterskip={16pt},
}}

%% ------------------------------------------------
% XeLaTeX 下需要把全体带圈数字都设置成 Default 类
% LuaLaTeX 下无须额外设置
\xeCJKDeclareCharClass{Default}{"24EA, "2460->"2473, "3251->"32BF}

% 将中文字体声明为(西文)字体族
\newfontfamily\EnclosedNumbers{Source Han Serif SC}

% 放置钩子,只让带圈字符才需更换字体
\AtBeginUTFCommand[\textcircled]{\begingroup\EnclosedNumbers}
\AtEndUTFCommand[\textcircled]{\endgroup}

%% ------------------------------------------------
% 为设置 无衬线反白(negative circled sans-serif digits) 而设置字体
\newfontfamily{\SHSC}{Source Han Serif SC}
%\def\negacircled#1{\SHSC \symbol{"#1}}
\def\zero{\SHSC \symbol{"24FF}}
\def\one{\SHSC \symbol{"2776}}
\def\two{\SHSC \symbol{"2777}}
\def\three{\SHSC \symbol{"2778}}
\def\four{\SHSC \symbol{"2779}}
\def\five{\SHSC \symbol{"277A}}
\def\six{\SHSC \symbol{"277B}}
\def\seven{\SHSC \symbol{"277C}}
\def\eight{\SHSC \symbol{"277D}}
\def\nine{\SHSC \symbol{"277E}}
\def\ten{\SHSC \symbol{"277F}}

% 开始编写论文
\begin{document}



%---------------------------------------------------------------------
%	封面、摘要、前言和目录
%---------------------------------------------------------------------

% 生成封面页
\maketitle

% 模板默认使用 \flushbottom,即底部平齐
% 效果更好,但可能出现 underfull \vbox 信息
% 以下命令用于抑制这些信息
% \raggedbottom

\begin{abstract}
  中文摘要
\end{abstract}

\begin{abstract*}
  English abstract
\end{abstract*}

% 生成目录
\tableofcontents
% 生成图片清单
% \listoffigures
% 生成表格清单
% \listoftables

%---------------------------------------------------------------------
%	正文部分
%---------------------------------------------------------------------
\mainmatter

% 符号表
% 语法与 description 环境一致
% 两个可选参数依次为说明区域宽度、符号区域宽度
% 带星号的符号表(notation*)不会插入目录
% \begin{notation}[10cm]
%   \item[DFT] 密度泛函理论 (Density functional theory)
%   \item[DMRG] 密度矩阵重正化群 (Density-Matrix Reformation-Group)
% \end{notation}

% 建议将论文内容拆分为多个文件
% 即新建一个 chapters 文件夹
% 把每一章的内容单独放入一个 .tex 文件
% 然后在这里用 \include 导入,例如
%   \include{chapters/introduction}
%   \include{chapters/environments}

% based on 北理工博士毕业论文 的 chapter1 https://tex.nju.edu.cn/project/user/10d0ed00-6f46-42fa-8a3a-18270214cfa3/a0ed1339-764d-4cc7-9ac4-d5d05b33da53

\label{第一章开始}
\chapter{绪论}

\section{本论文研究的目的和意义}

屈原:天问。\cite{berryOpticalSingularitiesBirefringent2003,ossikovskiConstitutiveRelationsOptically2021}。

奖励 = 找出该错误的期望耗时 * 期望时薪。

如果收到独立验证的正面反馈,将过程寄给我,也可获得奖励。

不知道这是不是又一个漂亮的黑洞/温柔的良夜,迷途无返。未选择的路。

\section{国内外研究现状及发展趋势}
%\label{sec:***} 可标注label

\subsection{形状记忆聚氨酯的形状记忆机理}
%\label{sec:features}

形状记忆聚合物(SMP)是继形状记忆合金后在80年代发展起来的一种新型形状记忆材料

分别给磁感应场$\bar{B}^{\;\!\mathcolor{gray}{t}}_{\;\!\mathcolor{gray}{z}}$(而非磁场$\bar{H}^{\;\!\mathcolor{gray}{t}}_{\;\!\mathcolor{gray}{z}}$),自由电流源$\bar{J}^{\;\!\mathcolor{gray}{t}}_{\;\!\textcolor{Maroon}{\text{f}}\mathcolor{gray}{z}}$ 以及电位移场$\bar{D}^{\;\!\mathcolor{gray}{t}}_{\;\!\mathcolor{gray}{z}}$,定义了如下 3 个本构关系(\textcolor{Maroon}{\text{Constitutive Relation}} = \textcolor{Maroon}{\text{CR}})。

其一,磁场 $\bar{H}^{\;\!\mathcolor{gray}{t}}_{\;\!\mathcolor{gray}{z}}$ 的本构关系,考虑$\bar{M}^{\;\!\mathcolor{gray}{t}}_{\;\!\mathcolor{gray}{z}}$\Footnote{磁化强度。其在微观上来源于:分子电流(电子轨道运动)产生的磁矩和电子自旋磁矩的矢量和\cite{nelsonLagrangianTreatmentMagnetic1994},细究还将有电子与原子核、电子的自旋轨道耦合、电子电子间相互作用(多电子产生原子磁矩)、原子与原子间(晶体场)的相互作用\cite{chen-zhuChenZhuxieUndergraduate_courses2024}。磁的起源看上去可以是纯电的\cite{lakhtakiaGenesisPostConstraint2004}。}仅由磁偶极矩\Footnote{即只考虑最低阶磁多极矩 = 不考虑磁四极矩及以上\cite{nelsonLagrangianTreatmentMagnetic1994},因为对于受到电磁场的影响后,反过来产生电磁场的电子而言,其受到的电场力是洛伦兹力的c$\big/v$倍\cite{boydNonlinearOptics2019}。此外,(磁/电)多极矩与(磁/电)非线性是互相独立的——即任何阶的多极矩,都有自己的线性项和非线性项\cite{chen-zhuChenZhuxieUndergraduate_courses2024},这些项与其它阶多极矩的任何项都无关。}贡献时,定义为
\begin{subequations} \label{eq:cr-b}
\begin{align}
	\textcolor{Maroon}{\text{CR for magnetism}}\text{:}&\hspace{0.5em} \bar{B}^{\;\!\mathcolor{gray}{t}}_{\;\!\mathcolor{gray}{z}} \hspace{-0.7em} &&={\symup{\varepsilon}}_0 \left( \bar{H}^{\;\!\mathcolor{gray}{t}}_{\;\!\mathcolor{gray}{z}} + \bar{M}^{\;\!\mathcolor{gray}{t}}_{\;\!\mathcolor{gray}{z}} \right) = {\symup{\varepsilon}}_0 \left\{ \bar{\bar{\delta}}^{\;\!\mathcolor{gray}{t}}~\widetilde *~\bar{H}^{\;\!\mathcolor{gray}{t}}_{\;\!\mathcolor{gray}{z}} + \bar{M}^{\;\!\mathcolor{gray}{t}}_{\;\!\mathcolor{gray}{z}} \right\} \label{cr-b1} \\ 
	& &&\xrightarrow[]{\bar{M}^{\;\!\mathcolor{gray}{t}}_{\;\!\mathcolor{gray}{z}} = \bar{M}^{\;\!\textcolor{Maroon}{\text{(1)}} \mathcolor{gray}{t}}_{\;\!\mathcolor{gray}{z}} + \bar{M}^{\;\!\textcolor{Maroon}{\text{NL}}, \mathcolor{gray}{t}}_{\;\!\mathcolor{gray}{z}}} {\symup{\varepsilon}}_0 \left\{ \left[ \bar{\bar{\delta}}^{\;\!\mathcolor{gray}{t}}~\widetilde *~\bar{H}^{\;\!\mathcolor{gray}{t}}_{\;\!\mathcolor{gray}{z}} + \bar{M}^{\;\!\textcolor{Maroon}{\text{(1)}} \mathcolor{gray}{t}}_{\;\!\mathcolor{gray}{z}} \right] + \bar{M}^{\;\!\textcolor{Maroon}{\text{NL}}, \mathcolor{gray}{t}}_{\;\!\mathcolor{gray}{z}} \right\} \label{cr-b2} \\ 
	& &&\xrightarrow[\displaystyle{ \bar{\bar{\mu}}^{\;\!\textcolor{Maroon}{\text{(1)}} \mathcolor{gray}{t}}_{\;\!\textcolor{Maroon}{\text{r}}\mathcolor{gray}{z}} := \bar{\bar{\delta}}^{\;\!\mathcolor{gray}{t}} + \bar{\bar{\chi}}^{\;\!\textcolor{Maroon}{\text{(1)}}\mathcolor{gray}{t}}_{\;\!\textcolor{Maroon}{\text{m}} \mathcolor{gray}{z}}}]{\displaystyle{\bar{M}^{\;\!\textcolor{Maroon}{\text{(1)}} \mathcolor{gray}{t}}_{\;\!\mathcolor{gray}{z}} := \bar{\bar{\chi}}^{\;\!\textcolor{Maroon}{\text{(1)}}\mathcolor{gray}{t}}_{\;\!\textcolor{Maroon}{\text{m}} \mathcolor{gray}{z}} ~\widetilde *~\bar{H}^{\;\!\mathcolor{gray}{t}}_{\;\!\mathcolor{gray}{z}}}} {\symup{\varepsilon}}_0 \left\{ \bar{\bar{\mu}}^{\;\!\textcolor{Maroon}{\text{(1)}} \mathcolor{gray}{t}}_{\;\!\textcolor{Maroon}{\text{r}}\mathcolor{gray}{z}}~\widetilde *~\bar{H}^{\;\!\mathcolor{gray}{t}}_{\;\!\mathcolor{gray}{z}} + \bar{M}^{\;\!\textcolor{Maroon}{\text{NL}}, \mathcolor{gray}{t}}_{\;\!\mathcolor{gray}{z}} \right\} \label{cr-b3} \\ 
	& &&= \bar{\bar{\mu}}^{\;\!\textcolor{Maroon}{\text{(1)}} \mathcolor{gray}{t}}_{\;\!\mathcolor{gray}{z}}~\widetilde *~\bar{H}^{\;\!\mathcolor{gray}{t}}_{\;\!\mathcolor{gray}{z}} + {\symup{\varepsilon}}_0 \bar{M}^{\;\!\textcolor{Maroon}{\text{NL}}, \mathcolor{gray}{t}}_{\;\!\mathcolor{gray}{z}} =: \bar{B}^{\;\!\textcolor{Maroon}{\text{(1)}} \mathcolor{gray}{t}}_{\;\!\mathcolor{gray}{z}} + \bar{B}^{\;\!\textcolor{Maroon}{\text{NL}}, \mathcolor{gray}{t}}_{\;\!\mathcolor{gray}{z}}~, \label{cr-b4}
\end{align}
\end{subequations}
其中,磁通量密度场 $\bar{B}^{\;\!\mathcolor{gray}{t}}_{\;\!\mathcolor{gray}{z}}$(直接/显示地)关于磁场 $\bar{H}^{\;\!\mathcolor{gray}{t}}_{\;\!\mathcolor{gray}{z}}$\Footnote{磁非线性,如郎之万顺磁性理论\cite{chen-zhuChenZhuxieUndergraduate_courses2024}中 $M \propto$ 郎之万函数 $\mathcal{L} \left( \alpha \right) = \coth \left( \alpha \right) - 1 / \alpha$(其中 $\alpha \propto H_{\textcolor{Maroon}{\text{ex}}}$)或其量子化修正之布里渊函数,铁磁体\cite{chen-zhuChenZhuxieUndergraduate_courses2024}或超导体\cite{wenBriefIntroductionFlux2021}中的磁滞现象等(每个时刻$t$,这些场量都是准静态$\Omega \to 0$的)。}、电场 $\bar{E}^{\;\!\mathcolor{gray}{t}}_{\;\!\mathcolor{gray}{z}}$\Footnote{双各向异性,其中的电$\to$磁耦合(如果 $\bar{B}^{\;\!\mathcolor{gray}{t}}_{\;\!\mathcolor{gray}{z}}$ 中的该部分只是 $\bar{E}^{\;\!\mathcolor{gray}{t}}_{\;\!\mathcolor{gray}{z}}$ 的线性函数,则也可归结到线性项中)。}、应力 $\bar{T}^{\;\!\mathcolor{gray}{t}}_{\;\!\mathcolor{gray}{z}}$\Footnote{正逆压磁/磁致伸缩/磁弹效应(这里未作区分)。}等其他场量(即含空 $\mathcolor{gray}{\bar{r}}$ 的物理量)\Footnote{$\bar{B}^{\;\!\mathcolor{gray}{t}}_{\;\!\mathcolor{gray}{z}},\bar{M}^{\;\!\mathcolor{gray}{t}}_{\;\!\mathcolor{gray}{z}}$ 以及各阶 $\bar{\bar{\mu}}^{\;\!\textcolor{Maroon}{\text{(1)}} \mathcolor{gray}{t}}_{\;\!\mathcolor{gray}{z}},\bar{\bar{\bar{\mu}}}^{\;\!\textcolor{Maroon}{\text{(2)}} \mathcolor{gray}{t}}_{\;\!\mathcolor{gray}{z}},\cdots$ 已经是关于温度$T$、波长$\lambda$(或 角频率$\omega$、时间$t$)等(非)场量的函数。}的非线性函数项,悉数包含在 $\bar{B}^{\;\!\textcolor{Maroon}{\text{NL}}, \mathcolor{gray}{t}}_{\;\!\mathcolor{gray}{z}} = {\symup{\varepsilon}}_0 \bar{M}^{\;\!\textcolor{Maroon}{\text{NL}}, \mathcolor{gray}{t}}_{\;\!\mathcolor{gray}{z}}$ 内;剩余的线性项,放在 $\bar{B}^{\;\!\textcolor{Maroon}{\text{(1)}} \mathcolor{gray}{t}}_{\;\!\mathcolor{gray}{z}} = \bar{\bar{\mu}}^{\;\!\textcolor{Maroon}{\text{(1)}} \mathcolor{gray}{t}}_{\;\!\mathcolor{gray}{z}}~\widetilde *~\bar{H}^{\;\!\mathcolor{gray}{t}}_{\;\!\mathcolor{gray}{z}}$ 中。

其二,自由电流源$\bar{J}^{\;\!\mathcolor{gray}{t}}_{\;\!\textcolor{Maroon}{\text{f}}\mathcolor{gray}{z}}$的本构关系,包含欧姆定律的线性部分(漂移项) $\bar{J}^{\;\!\textcolor{Maroon}{\text{(1)}} \mathcolor{gray}{t}}_{\;\!\textcolor{Maroon}{\text{f}}\mathcolor{gray}{z}} = \bar{\bar{\sigma}}^{\;\!\textcolor{Maroon}{\text{(1)}}\mathcolor{gray}{t}}_{\;\!\mathcolor{gray}{z}}~\widetilde *~\bar{E}^{\;\!\mathcolor{gray}{t}}_{\;\!\mathcolor{gray}{z}}$\Footnote{可以由 Drude 模型描述,定量解释一阶电导率$\bar{\bar{\sigma}}^{\;\!\textcolor{Maroon}{\text{(1)}}\mathcolor{gray}{t}}_{\;\!\mathcolor{gray}{z}}$的起源。},及$\bar{J}^{\;\!\mathcolor{gray}{t}}_{\;\!\textcolor{Maroon}{\text{f}}\mathcolor{gray}{z}}$分别关于电场 $\bar{E}^{\;\!\mathcolor{gray}{t}}_{\;\!\mathcolor{gray}{z}}$\Footnote{欧姆定律中的电非线性部分,比如二/三极管的伏安特性曲线\cite{chen-zhuChenZhuxieUndergraduate_courses2024}(尽管输入/输出or自/因变量,即$\bar{E}^{\;\!\mathcolor{gray}{t}}_{\;\!\mathcolor{gray}{z}}$和$\bar{J}^{\;\!\mathcolor{gray}{t}}_{\;\!\textcolor{Maroon}{\text{f}}\mathcolor{gray}{z}}$,一般均在直流或低频$\Omega$,非交流且不在光波段 opt)。}、磁感应场$\bar{B}^{\;\!\mathcolor{gray}{t}}_{\;\!\mathcolor{gray}{z}}$\Footnote{磁感应场$\bar{B}^{\;\!\mathcolor{gray}{t}}_{\;\!\mathcolor{gray}{z}}$所带来的(电场力以外的)洛伦兹力$\left( \bar{J}^{\;\!\mathcolor{gray}{t}}_{\;\!\textcolor{Maroon}{\text{f}}\mathcolor{gray}{z}} + \dot{\bar{P}}^{\;\!\mathcolor{gray}{t}}_{\;\!\mathcolor{gray}{z}} + \mathcolor{gray}{\bar{\nabla} \times} \bar{M}^{\;\!\mathcolor{gray}{t}}_{\;\!\mathcolor{gray}{z}} \right) \times \bar{B}^{\;\!\mathcolor{gray}{t}}_{\;\!\mathcolor{gray}{z}}$\cite{mackayElectromagneticAnisotropyBianisotropy2019,chen-zhuChenZhuxieUndergraduate_courses2024},会影响导/价带电子的运动(速度)$\bar{v}^{\;\!\mathcolor{gray}{t}}_{\;\!\textcolor{Maroon}{\text{f}}\mathcolor{gray}{z}}$,进而全局地影响自由电流$\bar{J}^{\;\!\mathcolor{gray}{t}}_{\;\!\textcolor{Maroon}{\text{f}}\mathcolor{gray}{z}} = {\rho}^{\;\!\mathcolor{gray}{t}}_{\;\!\textcolor{Maroon}{\text{f}}\mathcolor{gray}{z}} \bar{v}^{\;\!\mathcolor{gray}{t}}_{\;\!\textcolor{Maroon}{\text{f}}\mathcolor{gray}{z}}$和(束缚)电(偶)极化强度$\bar{P}^{\;\!\mathcolor{gray}{t}}_{\;\!\mathcolor{gray}{z}}$,包括它们的线性和非线性项\cite{boydNonlinearOptics2019}。对于强场/超快非线性光学,相对论效应使得电磁场是个统一的整体,动生(而不仅是外加)的$\bar{B}^{\;\!\mathcolor{gray}{t}}_{\;\!\mathcolor{gray}{z}}$还将带来额外的影响。磁场$\bar{H}^{\;\!\mathcolor{gray}{t}}_{\;\!\mathcolor{gray}{z}}$对分子/磁化电流体密度$\mathcolor{gray}{\bar{\nabla} \times} \bar{M}^{\;\!\mathcolor{gray}{t}}_{\;\!\mathcolor{gray}{z}}$产生的影响已包含在$\bar{M}^{\;\!\mathcolor{gray}{t}}_{\;\!\mathcolor{gray}{z}}$中了。}、导带电子浓度(数密度)梯度场$\mathcolor{gray}{\bar{\nabla}} {\rho}^{\;\!\mathcolor{gray}{t}}_{\;\!\textcolor{Maroon}{\text{f}}\mathcolor{gray}{z}}$\Footnote{在光折变效应中,作为$\bar{J}^{\;\!\mathcolor{gray}{t}}_{\;\!\textcolor{Maroon}{\text{f}}\mathcolor{gray}{z}}$中的扩散项\cite{boydNonlinearOptics2019}。${\rho}^{\;\!\mathcolor{gray}{t}}_{\;\!\textcolor{Maroon}{\text{f}}\mathcolor{gray}{z}}, \bar{J}^{\;\!\mathcolor{gray}{t}}_{\;\!\textcolor{Maroon}{\text{f}}\mathcolor{gray}{z}}$之间还应满足\bref{eq:Div-e-f}以及$\bar{J}^{\;\!\mathcolor{gray}{t}}_{\;\!\textcolor{Maroon}{\text{f}}\mathcolor{gray}{z}} = {\rho}^{\;\!\mathcolor{gray}{t}}_{\;\!\textcolor{Maroon}{\text{f}}\mathcolor{gray}{z}} \bar{v}^{\;\!\mathcolor{gray}{t}}_{\;\!\textcolor{Maroon}{\text{f}}\mathcolor{gray}{z}}$\cite{chen-zhuChenZhuxieUndergraduate_courses2024}。}和光伏电流场$\propto \lvert \bar{E}^{\;\!\mathcolor{gray}{t}}_{\;\!\mathcolor{gray}{z}} \rvert^2 \hat{c}$\Footnote{与光电导效应并列,属于内光电效应;也可能在光折变效应的$\bar{J}^{\;\!\mathcolor{gray}{t}}_{\;\!\textcolor{Maroon}{\text{f}}\mathcolor{gray}{z}}$中扮演一份角色,特别是沿着一些各向异性晶体的光轴$\hat{c}$产生电势差和内建电场\cite{boydNonlinearOptics2019}(尽管一般也只影响直流或低频$\Omega$的$\bar{J}^{\;\!\mathcolor{gray}{t}}_{\;\!\textcolor{Maroon}{\text{f}}\mathcolor{gray}{z}}$;但$\bar{J}^{\;\!\mathcolor{gray}{t}}_{\;\!\textcolor{Maroon}{\text{f}}\mathcolor{gray}{z}}$会通过影响光波段的介电常数,进而影响光波段的光强$\lvert \bar{E}^{\;\!\mathcolor{gray}{t}}_{\;\!\mathcolor{gray}{z}} \rvert^2$及$\bar{J}^{\;\!\mathcolor{gray}{t}}_{\;\!\textcolor{Maroon}{\text{f}}\mathcolor{gray}{z}}$自己的重新分布);该二阶的带耦合的非线性,看上去很像非线性极化率$\bar{P}^{\;\!\textcolor{Maroon}{\text{(2)}} \mathcolor{gray}{t}}_{\;\!\mathcolor{gray}{z}}$中的光整流项,但其频率比 THz 低,且只服务于自由电流。—— 以至该项可作为差频合并至$\bar{J}^{\;\!\mathcolor{gray}{t}}_{\;\!\textcolor{Maroon}{\text{f}}\mathcolor{gray}{z}}$关于$\bar{E}^{\;\!\mathcolor{gray}{t}}_{\;\!\mathcolor{gray}{z}}$的二阶非线性$\bar{J}^{\;\!\textcolor{Maroon}{\text{(2)}} \mathcolor{gray}{t}}_{\;\!\textcolor{Maroon}{\text{f}}\mathcolor{gray}{z}}$中去?}等其他场量的非线性项 $\bar{J}^{\;\!\textcolor{Maroon}{\text{NL}}, \mathcolor{gray}{t}}_{\;\!\textcolor{Maroon}{\text{f}}\mathcolor{gray}{z}}$:
\begin{align} \label{eq:cr-j}
	\textcolor{Maroon}{\text{Ohm's law}}\text{:}\hspace{0.5em} \bar{J}^{\;\!\mathcolor{gray}{t}}_{\;\!\textcolor{Maroon}{\text{f}}\mathcolor{gray}{z}} = \bar{\bar{\sigma}}^{\;\!\textcolor{Maroon}{\text{(1)}}\mathcolor{gray}{t}}_{\;\!\mathcolor{gray}{z}}~\widetilde *~\bar{E}^{\;\!\mathcolor{gray}{t}}_{\;\!\mathcolor{gray}{z}} + \bar{J}^{\;\!\textcolor{Maroon}{\text{NL}}, \mathcolor{gray}{t}}_{\;\!\textcolor{Maroon}{\text{f}}\mathcolor{gray}{z}} =: \bar{J}^{\;\!\textcolor{Maroon}{\text{(1)}} \mathcolor{gray}{t}}_{\;\!\textcolor{Maroon}{\text{f}}\mathcolor{gray}{z}} + \bar{J}^{\;\!\textcolor{Maroon}{\text{NL}}, \mathcolor{gray}{t}}_{\;\!\textcolor{Maroon}{\text{f}}\mathcolor{gray}{z}}~,
\end{align}

其三,电位移场$\bar{D}^{\;\!\mathcolor{gray}{t}}_{\;\!\mathcolor{gray}{z}}$的本构关系,当$\bar{P}^{\;\!\mathcolor{gray}{t}}_{\;\!\mathcolor{gray}{z}}$只由电偶极矩\Footnote{不考虑电四极矩及以上。但电四极化强度场$\bar{\bar{Q}}^{\;\!\mathcolor{gray}{t}}_{\;\!\mathcolor{gray}{z}}$(的等效电偶极化强度场$\bar{P}^{\;\!\mathcolor{gray}{t}}_{\;\!\textcolor{Maroon}{\text{Q}}\mathcolor{gray}{z}} = - \mathcolor{gray}{\bar{\nabla} \cdot} \bar{\bar{Q}}^{\;\!\mathcolor{gray}{t}}_{\;\!\mathcolor{gray}{z}}$)\cite{chen-zhuChenZhuxieUndergraduate_courses2024}在有些效应中不可忽视且起关键作用:如其对线性晶体光学中的光学活性的贡献\cite{nelsonDerivingTransmissionReflection1995},以及非线性光学中基于$\bar{\bar{Q}}^{\;\!\mathcolor{gray}{t}}_{\;\!\mathcolor{gray}{z}}$的二阶和频\cite{bethuneOpticalQuadrupoleSumfrequency1976}。电四极子对光与物质相互作用的贡献,还会打破$\bar{D}^{\;\!\mathcolor{gray}{t}}_{\;\!\mathcolor{gray}{z}}$法向连续和$\bar{H}^{\;\!\mathcolor{gray}{t}}_{\;\!\mathcolor{gray}{z}}$切向连续边界条件,并与洛伦兹力的定义、(由唯二的无源齐次\cite{lakhtakiaGenesisPostConstraint2004}微分方程\bref{eq:Curl-EK,eq:Div-Bk}导出的)电磁场标/矢势\cite{chen-zhuChenZhuxieUndergraduate_courses2024}等一起,使$\bar{E}^{\;\!\mathcolor{gray}{t}}_{\;\!\mathcolor{gray}{z}},\bar{B}^{\;\!\mathcolor{gray}{t}}_{\;\!\mathcolor{gray}{z}}$而不是$\bar{E}^{\;\!\mathcolor{gray}{t}}_{\;\!\mathcolor{gray}{z}},\bar{H}^{\;\!\mathcolor{gray}{t}}_{\;\!\mathcolor{gray}{z}}$成为基本场\cite{nelsonDerivingTransmissionReflection1995},对应地,坡印亭矢量也需要修正为$\bar{E}^{\;\!\mathcolor{gray}{t}}_{\;\!\mathcolor{gray}{z}} \times \bar{B}^{\;\!\mathcolor{gray}{t}}_{\;\!\mathcolor{gray}{z}} \big/ {\symup{\varepsilon}}_0$\cite{nelsonGeneralizingPoyntingVector1996,loudonPropagationElectromagneticEnergy1997,richterPoyntingsTheoremEnergy2008}而不是$\bar{E}^{\;\!\mathcolor{gray}{t}}_{\;\!\mathcolor{gray}{z}} \times \bar{H}^{\;\!\mathcolor{gray}{t}}_{\;\!\mathcolor{gray}{z}}$。}构成时,定义为
\begin{subequations} \label{eq:cr-d}
\begin{align}
	\textcolor{Maroon}{\text{CR for electricity}}\text{:}&\hspace{0.5em} \bar{D}^{\;\!\mathcolor{gray}{t}}_{\;\!\mathcolor{gray}{z}} \hspace{-2.0em} &&= {\symup{\varepsilon}}_0 \bar{E}^{\;\!\mathcolor{gray}{t}}_{\;\!\mathcolor{gray}{z}} + \bar{P}^{\;\!\mathcolor{gray}{t}}_{\;\!\mathcolor{gray}{z}} = {\symup{\varepsilon}}_0 \bar{\bar{\delta}}^{\;\!\mathcolor{gray}{t}}~\widetilde *~\bar{E}^{\;\!\mathcolor{gray}{t}}_{\;\!\mathcolor{gray}{z}} + \bar{P}^{\;\!\mathcolor{gray}{t}}_{\;\!\mathcolor{gray}{z}} \label{cr-d1} \\ 
	& &&\xrightarrow[]{\bar{P}^{\;\!\mathcolor{gray}{t}}_{\;\!\mathcolor{gray}{z}} = \bar{P}^{\;\!\textcolor{Maroon}{\text{(1)}} \mathcolor{gray}{t}}_{\;\!\mathcolor{gray}{z}} + \bar{P}^{\;\!\textcolor{Maroon}{\text{NL}}, \mathcolor{gray}{t}}_{\;\!\mathcolor{gray}{z}} + } \left[ {\symup{\varepsilon}}_0 \bar{\bar{\delta}}^{\;\!\mathcolor{gray}{t}}~\widetilde *~\bar{E}^{\;\!\mathcolor{gray}{t}}_{\;\!\mathcolor{gray}{z}} + \bar{P}^{\;\!\textcolor{Maroon}{\text{(1)}} \mathcolor{gray}{t}}_{\;\!\mathcolor{gray}{z}} \right] + \bar{P}^{\;\!\textcolor{Maroon}{\text{NL}}, \mathcolor{gray}{t}}_{\;\!\mathcolor{gray}{z}} \label{cr-d2} \\ 
	& &&\xrightarrow[\displaystyle{ \bar{\bar{\varepsilon}}^{\;\!\textcolor{Maroon}{\text{(1)}} \mathcolor{gray}{t}}_{\;\!\textcolor{Maroon}{\text{r}}\mathcolor{gray}{z}} := \bar{\bar{\delta}}^{\;\!\mathcolor{gray}{t}} + \bar{\bar{\chi}}^{\;\!\textcolor{Maroon}{\text{(1)}}\mathcolor{gray}{t}}_{\;\!\textcolor{Maroon}{\text{f}} \mathcolor{gray}{z}}}]{\displaystyle{\bar{P}^{\;\!\textcolor{Maroon}{\text{(1)}} \mathcolor{gray}{t}}_{\;\!\mathcolor{gray}{z}} := \bar{\bar{\chi}}^{\;\!\textcolor{Maroon}{\text{(1)}}\mathcolor{gray}{t}}_{\;\!\textcolor{Maroon}{\text{f}} \mathcolor{gray}{z}} ~\widetilde *~\bar{E}^{\;\!\mathcolor{gray}{t}}_{\;\!\mathcolor{gray}{z}}}} {\symup{\varepsilon}}_0 \bar{\bar{\varepsilon}}^{\;\!\textcolor{Maroon}{\text{(1)}} \mathcolor{gray}{t}}_{\;\!\textcolor{Maroon}{\text{r}}\mathcolor{gray}{z}}~\widetilde *~\bar{E}^{\;\!\mathcolor{gray}{t}}_{\;\!\mathcolor{gray}{z}} + \bar{P}^{\;\!\textcolor{Maroon}{\text{NL}}, \mathcolor{gray}{t}}_{\;\!\mathcolor{gray}{z}} \label{cr-d3} \\ 
	& &&= \bar{\bar{\varepsilon}}^{\;\!\textcolor{Maroon}{\text{(1)}} \mathcolor{gray}{t}}_{\;\!\mathcolor{gray}{z}}~\widetilde *~\bar{E}^{\;\!\mathcolor{gray}{t}}_{\;\!\mathcolor{gray}{z}} + \bar{P}^{\;\!\textcolor{Maroon}{\text{NL}}, \mathcolor{gray}{t}}_{\;\!\mathcolor{gray}{z}} =: \bar{D}^{\;\!\textcolor{Maroon}{\text{(1)}} \mathcolor{gray}{t}}_{\;\!\mathcolor{gray}{z}} + \bar{D}^{\;\!\textcolor{Maroon}{\text{NL}}, \mathcolor{gray}{t}}_{\;\!\mathcolor{gray}{z}}~, \label{cr-d4}
\end{align}
\end{subequations}
关于其组成成分,电位移场 $\bar{D}^{\;\!\mathcolor{gray}{t}}_{\;\!\mathcolor{gray}{z}}$(直接/显示地)关于电场 $\bar{E}^{\;\!\mathcolor{gray}{t}}_{\;\!\mathcolor{gray}{z}}$\Footnote{电非线性,包括高频段的(非)共振非线性、低频低温\cite{lakhtakiaGenesisPostConstraint2004}段的铁电体/畴的电滞现象等。}、磁场 $\bar{H}^{\;\!\mathcolor{gray}{t}}_{\;\!\mathcolor{gray}{z}}$\Footnote{双各向异性中的磁$\to$电耦合(如果 $\bar{D}^{\;\!\mathcolor{gray}{t}}_{\;\!\mathcolor{gray}{z}}$ 中的该部分只是 $\bar{H}^{\;\!\mathcolor{gray}{t}}_{\;\!\mathcolor{gray}{z}}$ 的线性函数,则也可归结到线性项中)。}、应力 $\bar{T}^{\;\!\mathcolor{gray}{t}}_{\;\!\mathcolor{gray}{z}}$\Footnote{正逆压磁/磁致伸缩/磁弹效应(这里未作区分)。}等其他场量的非线性函数项,均由 $\bar{D}^{\;\!\textcolor{Maroon}{\text{NL}}, \mathcolor{gray}{t}}_{\;\!\mathcolor{gray}{z}} = {\symup{\varepsilon}}_0 \bar{M}^{\;\!\textcolor{Maroon}{\text{NL}}, \mathcolor{gray}{t}}_{\;\!\mathcolor{gray}{z}}$ 贡献;剩余的线性项,由 $\bar{B}^{\;\!\textcolor{Maroon}{\text{(1)}} \mathcolor{gray}{t}}_{\;\!\mathcolor{gray}{z}} = \bar{\bar{\mu}}^{\;\!\textcolor{Maroon}{\text{(1)}} \mathcolor{gray}{t}}_{\;\!\mathcolor{gray}{z}}~\widetilde *~\bar{H}^{\;\!\mathcolor{gray}{t}}_{\;\!\mathcolor{gray}{z}}$ 表示。



\marginLeft{chap:maxwell}\chapter{晶体中的(非)线性光学过程}\label{chap:maxwell}

相对观察者静止\Footnote{否则 $\bar{E}^{\;\!\textcolor{gray}{t}}_{\;\!\textcolor{gray}{z}},\bar{D}^{\;\!\textcolor{gray}{t}}_{\;\!\textcolor{gray}{z}}$ 与 $\bar{B}^{\;\!\textcolor{gray}{t}}_{\;\!\textcolor{gray}{z}}$ 将进一步相互耦合\cite{berryOpticalSingularitiesBianisotropic2005,chen-zhuChenZhuxieUndergraduate_courses2024},以致材料的本构关系将呈现固有双各向异性\cite{mackayElectromagneticAnisotropyBianisotropy2019,mackayModernAnalyticalElectromagnetic2020,lakhtakiaCovariancesInvariancesMaxwell1995};$\bar{J}^{\;\!\textcolor{gray}{t}}_{\;\!\textcolor{Maroon}{\text{e}}\textcolor{gray}{z}}$ 也需要扩展到四维\cite{lakhtakiaCovariancesInvariancesMaxwell1995,chen-zhuChenZhuxieUndergraduate_courses2024}。此外,束缚电荷$- \bar{\nabla} \cdot \bar{P}^{\;\!\textcolor{gray}{t}}_{\;\!\textcolor{gray}{z}}$\cite{mackayElectromagneticAnisotropyBianisotropy2019,chen-zhuChenZhuxieUndergraduate_courses2024}、自由电子${\rho}^{\;\!\textcolor{gray}{t}}_{\;\!\textcolor{Maroon}{\text{e}}\textcolor{gray}{z}}$的(有效)运动质量(或相对论速度)也会变大,并因此同时对线性和非线性极化$\bar{P}^{\;\!\textcolor{gray}{t}}_{\;\!\textcolor{gray}{z}}$、自由电流$\bar{J}^{\;\!\textcolor{gray}{t}}_{\;\!\textcolor{Maroon}{\text{e}}\textcolor{gray}{z}} = {\rho}^{\;\!\textcolor{gray}{t}}_{\;\!\textcolor{Maroon}{\text{e}}\textcolor{gray}{z}} \bar{v}^{\;\!\textcolor{gray}{t}}_{\;\!\textcolor{Maroon}{\text{e}}\textcolor{gray}{z}}$产生额外影响。——上述场景可能发生在超快和强场泵浦(及其通过多光子/隧穿电离产生的等离子体)中\cite{boydNonlinearOptics2019},并且可能需要引入非线性相对论电动力学。} 的三维空间$\textcolor{gray}{\bar{r}}$ 坐标系下,在可能存在非零的自由电荷源和自由电流源${\rho}_{\;\!\textcolor{Maroon}{\text{e}}} \left( \textcolor{gray}{\bar{r}}, \textcolor{gray}{t} \right), \bar{J}_{\;\!\textcolor{Maroon}{\text{e}}} \left( \textcolor{gray}{\bar{r}}, \textcolor{gray}{t} \right)$\Footnote{对于符号约定,比如下标 $\textcolor{Maroon}{\text{e}}$ 的\textcolor{Maroon}{褐红色}及其含义 `\textcolor{Maroon}{electricity}',其定义见\bref{Maroon};此外,在 \bref{1bar} 中还约定:总使用 1 条上短横线 $\bar{~}$(而不是粗体)来表示矢量(如 $\bar{J}_{\;\!\textcolor{Maroon}{\text{e}}}$),以区别于 \bref{0bar} 中定义的无上短横线的标量(如 ${\rho}_{\;\!\textcolor{Maroon}{\text{e}}}$);粗体在本文中另有其含义,不用于表示矢量,见。} 的一般电磁介质内部,4 个空域时变\Footnote{指复矢量场 $\bar{E}^{\;\!\textcolor{gray}{t}}_{\;\!\textcolor{gray}{z}}, \bar{H}^{\;\!\textcolor{gray}{t}}_{\;\!\textcolor{gray}{z}}, \bar{D}^{\;\!\textcolor{gray}{t}}_{\;\!\textcolor{gray}{z}}, \bar{B}^{\;\!\textcolor{gray}{t}}_{\;\!\textcolor{gray}{z}}$ 均是四维时空 $\textcolor{gray}{\bar{r}}, \textcolor{gray}{t}$ 的函数,且因此一般意义上是复色 $\left\{ \omega \in \mathbb{R} \right\}$ 的复场 $\in \mathbb{C}^3 \left( \mathbb{R}^3 \right)$(认为这四者必须为实场\cite{boydNonlinearOptics2019}也没关系:它们对$t$的傅立叶变换所得到的$\pm \omega$单色子波是复共轭的,以至于正负频率的对应复子波求和后会消掉虚部,只剩下实部的余弦$\cos$实子波,因此在正/倒空间中的总/子场均是有物理意义的实场),属于在视觉上占主导的因变量,并用黑色(见 \bref{black})的 \textit{斜体 oblique}(见 \bref{oblique})表示;相对地,自变量用视觉和含义上均更次要的\textcolor{gray}{灰色}来表示,见 \bref{gray}。}复色场 
$\bar{E}^{\;\!\textcolor{gray}{t}}_{\;\!\textcolor{gray}{z}}, \bar{H}^{\;\!\textcolor{gray}{t}}_{\;\!\textcolor{gray}{z}}, \bar{D}^{\;\!\textcolor{gray}{t}}_{\;\!\textcolor{gray}{z}}, \bar{B}^{\;\!\textcolor{gray}{t}}_{\;\!\textcolor{gray}{z}}$\Footnote{由于傅立叶光学一般运行在平行平面间,约定下述表示相互等价:$\bar{E} \left( \textcolor{gray}{\bar{r}}, \textcolor{gray}{t} \right) = \bar{E}^{\;\!\textcolor{gray}{t}}_{\;\!\textcolor{gray}{\bar{r}}} = \bar{E}^{\;\!\textcolor{gray}{t}}_{\;\!\textcolor{gray}{z}} \left( \textcolor{gray}{\bar{\rho}} \right)$,并因此经常省略面内自变量 $\textcolor{gray}{\bar{\rho}}$,以只写作朝 $\textcolor{Maroon}{+\symup{z}}$ 轴传播距离 $\textcolor{gray}{z}$ 的函数 $\bar{E}^{\;\!\textcolor{gray}{t}}_{\;\!\textcolor{gray}{z}} := \bar{E}^{\;\!\textcolor{gray}{t}}_{\;\!\textcolor{gray}{z}} \left( \textcolor{gray}{\bar{\rho}} \right)$。—— 同样的规则也适用于其他场量(如 ${\rho}_{\;\!\textcolor{Maroon}{\text{e}}} \left( \textcolor{gray}{\bar{r}}, \textcolor{gray}{t} \right) \to {\rho}^{\;\!\textcolor{gray}{t}}_{\;\!\textcolor{Maroon}{\text{e}}\textcolor{gray}{z}}$),且适用于空间频率域,见。},满足微分形式\Footnote{尽管是微分形式,仍然处于(相对的)宏观层面:典型的光波长 $1$um 是原子特征尺寸 $1\text{\r{A}} = 0.1$nm 的 $10^4$ 倍,因此\bref{eq:maxwell-eh,eq:maxwell-db,eq:continuity-pj}涉及的所有物理量均是空间平均后的结果\cite{mackayElectromagneticAnisotropyBianisotropy2019};如果要引入(非)线性极化强度/率随考虑区域尺度的缩放,则需要 \textcolor{Maroon}{Clausius-Mossotti equation} 或 \textcolor{Maroon}{Lorentz-Lorenz law} 的局域场修正\cite{boydNonlinearOptics2019}。}的麦氏方程组的 2 个旋度假设:
\begin{subequations} \label{eq:maxwell-eh}
\begin{align}
	\textcolor{Maroon}{\text{Faraday's law of electromagnetic induction}}\text{:}&\hspace{0.5em} \bar{\nabla} \times \bar{E}^{\;\!\textcolor{gray}{t}}_{\;\!\textcolor{gray}{z}} \hspace{-1.2em} &&= - \bar{J}^{\;\!\textcolor{gray}{t}}_{\;\!\textcolor{Maroon}{\text{m}}\textcolor{gray}{z}} - \frac{\partial \bar{B}^{\;\!\textcolor{gray}{t}}_{\;\!\textcolor{gray}{z}}}{\partial t}~, \label{eq:maxwell-e} \\ \textcolor{Maroon}{\text{Jefimenko's}} \to \textcolor{Maroon}{\text{Amp\`{e}re-Maxwell circuital law}}\text{:}&\hspace{0.5em} \bar{\nabla} \times \bar{H}^{\;\!\textcolor{gray}{t}}_{\;\!\textcolor{gray}{z}} \hspace{-1.2em} &&= \bar{J}^{\;\!\textcolor{gray}{t}}_{\;\!\textcolor{Maroon}{\text{e}}\textcolor{gray}{z}} + \frac{\partial \bar{D}^{\;\!\textcolor{gray}{t}}_{\;\!\textcolor{gray}{z}}}{\partial t}~, \label{eq:maxwell-h}
\end{align}
\end{subequations}
以及 2 个散度假设:
\begin{subequations} \label{eq:maxwell-db}
\begin{align}
	\textcolor{Maroon}{\text{Coulomb's}} \to \textcolor{Maroon}{\text{Gauss's law for electricity}}\text{:}&\hspace{0.5em} \bar{\nabla} \cdot \bar{D}^{\;\!\textcolor{gray}{t}}_{\;\!\textcolor{gray}{z}} \hspace{-3.2em} &&= {\rho}^{\;\!\textcolor{gray}{t}}_{\;\!\textcolor{Maroon}{\text{e}}\textcolor{gray}{z}}~, \label{eq:maxwell-d} \\ \textcolor{Maroon}{\text{Biot-Savart}} \to \textcolor{Maroon}{\text{Gauss's law for magnetism}}\text{:}&\hspace{0.5em} \bar{\nabla} \cdot \bar{B}^{\;\!\textcolor{gray}{t}}_{\;\!\textcolor{gray}{z}} \hspace{-3.2em} &&= {\rho}^{\;\!\textcolor{gray}{t}}_{\;\!\textcolor{Maroon}{\text{m}}\textcolor{gray}{z}}~. \label{eq:maxwell-b}
\end{align}
\end{subequations}
其中,为数学形式上的对称(以方便引入狭义相对论效应和检验其洛伦兹协变性\cite{lakhtakiaCovariancesInvariancesMaxwell1995,chen-zhuChenZhuxieUndergraduate_courses2024}),和物理上不排除可能存在的磁单极子,除自由电(荷/流)源${\rho}^{\;\!\textcolor{gray}{t}}_{\;\!\textcolor{Maroon}{\text{e}}\textcolor{gray}{z}}, \bar{J}^{\;\!\textcolor{gray}{t}}_{\;\!\textcolor{Maroon}{\text{e}}\textcolor{gray}{z}}$外,还添加了自由磁源${\rho}^{\;\!\textcolor{gray}{t}}_{\;\!\textcolor{Maroon}{\text{m}}\textcolor{gray}{z}}, \bar{J}^{\;\!\textcolor{gray}{t}}_{\;\!\textcolor{Maroon}{\text{m}}\textcolor{gray}{z}}$\cite{lakhtakiaCovariancesInvariancesMaxwell1995}。这 4 个自由源项,满足2个连续性假设\cite{mackayElectromagneticAnisotropyBianisotropy2019,lakhtakiaCovariancesInvariancesMaxwell1995,chen-zhuChenZhuxieUndergraduate_courses2024}:
\begin{subequations} \label{eq:continuity-pj}
\begin{align}
	\textcolor{Maroon}{\text{Continuity for electric free source}}\text{:}&\hspace{0.5em} \bar{\nabla} \cdot \bar{J}^{\;\!\textcolor{gray}{t}}_{\;\!\textcolor{Maroon}{\text{e}}\textcolor{gray}{z}} + \frac{\partial {\rho}^{\;\!\textcolor{gray}{t}}_{\;\!\textcolor{Maroon}{\text{e}}\textcolor{gray}{z}}}{\partial t} \hspace{-5.2em} &&= 0~, \label{eq:continuity-e} \\ \textcolor{Maroon}{\text{Continuity for magnetic free source}}\text{:}&\hspace{0.5em} \bar{\nabla} \cdot \bar{J}^{\;\!\textcolor{gray}{t}}_{\;\!\textcolor{Maroon}{\text{m}}\textcolor{gray}{z}} + \frac{\partial {\rho}^{\;\!\textcolor{gray}{t}}_{\;\!\textcolor{Maroon}{\text{m}}\textcolor{gray}{z}}}{\partial t} \hspace{-5.2em} &&= 0~. \label{eq:continuity-m}
\end{align}
\end{subequations}
注意,对旋度 \bref{eq:maxwell-eh} 两边取散度($\bar{\nabla} \cdot$),连续性 \bref{eq:continuity-pj} 可导出散度 \bref{eq:maxwell-db},反之亦然\cite{lakhtakiaGenesisPostConstraint2004}。因此,2 条散度方程均不是必需的,可将其视为冗余\Footnote{并且不应简单地仅根据$\bar{D}^{\;\!\textcolor{gray}{t}}_{\;\!\textcolor{gray}{z}}, \bar{B}^{\;\!\textcolor{gray}{t}}_{\;\!\textcolor{gray}{z}}$的横向性,而将二者视为基本场\cite{quesadaPhotonPairsNonlinear2022,berryOpticalSingularitiesBianisotropic2005}。但从场能量体密度变化率$\bar{E}^{\;\!\textcolor{gray}{t}}_{\;\!\textcolor{gray}{z}} \mathbb{d}\bar{D}^{\;\!\textcolor{gray}{t}}_{\;\!\textcolor{gray}{z}} +  \bar{H}^{\;\!\textcolor{gray}{t}}_{\;\!\textcolor{gray}{z}} \mathbb{d}\bar{B}^{\;\!\textcolor{gray}{t}}_{\;\!\textcolor{gray}{z}}$中含有 2 个旋度\bref{eq:maxwell-eh}中对$\bar{D}^{\;\!\textcolor{gray}{t}}_{\;\!\textcolor{gray}{z}}, \bar{B}^{\;\!\textcolor{gray}{t}}_{\;\!\textcolor{gray}{z}}$的微分、方便引入适用于非线性量子光学的正确的哈密顿量\cite{quesadaPhotonPairsNonlinear2022},或者从更便利、自然和优雅地描述天然/法拉第旋光效应的角度\cite{berryOpticalSingularitiesBianisotropic2005},将$\bar{D}^{\;\!\textcolor{gray}{t}}_{\;\!\textcolor{gray}{z}}, \bar{B}^{\;\!\textcolor{gray}{t}}_{\;\!\textcolor{gray}{z}}$视为基本场也有一定道理?}。

现代电磁学/电动力学将$\bar{E}^{\;\!\textcolor{gray}{t}}_{\;\!\textcolor{gray}{z}}, \bar{B}^{\;\!\textcolor{gray}{t}}_{\;\!\textcolor{gray}{z}}$视为基本场\cite{hillionBasicFieldElectromagnetism1996,lakhtakiaGenesisPostConstraint2004,nelsonDerivingTransmissionReflection1995}\Footnote{$\bar{D}^{\;\!\textcolor{gray}{t}}_{\;\!\textcolor{gray}{z}},\bar{H}^{\;\!\textcolor{gray}{t}}_{\;\!\textcolor{gray}{z}}$只是$\bar{E}^{\;\!\textcolor{gray}{t}}_{\;\!\textcolor{gray}{z}},\bar{B}^{\;\!\textcolor{gray}{t}}_{\;\!\textcolor{gray}{z}}$分别加上($\bar{P}^{\;\!\textcolor{gray}{t}}_{\;\!\textcolor{gray}{z}}$)或减去($\bar{M}^{\;\!\textcolor{gray}{t}}_{\;\!\textcolor{gray}{z}}$)束缚源所产生的场后的辅助场,代表自由源所对应的场。$\bar{E}^{\;\!\textcolor{gray}{t}}_{\;\!\textcolor{gray}{z}},\bar{B}^{\;\!\textcolor{gray}{t}}_{\;\!\textcolor{gray}{z}}$是总场和基本场,出于下述原因:其起源是微观且明确的、可直接测量、包含了所有的束缚和自由源产生的场、洛伦兹力公式(普适至相对论情形)、\bref{eq:maxwell-e,eq:maxwell-b}的无源特性及其导出的标矢势和四维势矢量、四维二阶电磁场张量\cite{chen-zhuChenZhuxieUndergraduate_courses2024}、无矛盾地推导和适用 Post 约束\cite{lakhtakiaGenesisPostConstraint2004};同时也方便原子物理中对拉莫尔进动、史特恩—盖拉赫实验、塞曼效应的表述\cite{chen-zhuChenZhuxieUndergraduate_courses2024},以及量子电动力学中对磁光材料的拉氏量的处理\cite{nelsonLagrangianTreatmentMagnetic1994}。——但是,选择$\bar{B}^{\;\!\textcolor{gray}{t}}_{\;\!\textcolor{gray}{z}}$而不是$\bar{H}^{\;\!\textcolor{gray}{t}}_{\;\!\textcolor{gray}{z}}$将不方便(准静)磁学,如铁磁性物质的磁滞回线的表述\cite{hillionBasicFieldElectromagnetism1996}。}。但本文将$\bar{E}^{\;\!\textcolor{gray}{t}}_{\;\!\textcolor{gray}{z}}, \bar{H}^{\;\!\textcolor{gray}{t}}_{\;\!\textcolor{gray}{z}}$视为基本场\Footnote{相对论或手性的情形下,将$\bar{E}^{\;\!\textcolor{gray}{t}}_{\;\!\textcolor{gray}{z}}, \bar{H}^{\;\!\textcolor{gray}{t}}_{\;\!\textcolor{gray}{z}}$而不是$\bar{E}^{\;\!\textcolor{gray}{t}}_{\;\!\textcolor{gray}{z}}, \bar{B}^{\;\!\textcolor{gray}{t}}_{\;\!\textcolor{gray}{z}}$作为本构关系的基本场,可能更有优势\cite{hillionBasicFieldElectromagnetism1996,lakhtakiaGenesisPostConstraint2004};此外,对于边界条件,进可采用$\bar{E}^{\;\!\textcolor{gray}{t}}_{\;\!\textcolor{gray}{z}},\bar{H}^{\;\!\textcolor{gray}{t}}_{\;\!\textcolor{gray}{z}}$切向连续边界条件,退可四维时空傅立叶变换\cite{chenWavevectorspaceMethodWave1993,chenWavePropagationExciton1993,nelsonDerivingTransmissionReflection1995}。还允许不关注微观起源\cite{eimerlQuantumElectrodynamicsOptical1988,nelsonMechanismsDispersionCrystalline1989,boydNonlinearOptics2019,loudonPropagationElectromagneticEnergy1997,laxLinearNonlinearElectrodynamics1971}。但可能没法处理材料表面积累电荷(如铁电体的$\textcolor{Maroon}{+\symup{c}}$ 面)、表面电流\cite{chen-zhuChenZhuxieUndergraduate_courses2024}、表面光学活性\cite{nelsonMechanismsDispersionCrystalline1989},尽管没有使用到任何散度方程/横向约束,已经很有吸引力了\cite{eimerlQuantumElectrodynamicsOptical1988,berryOpticalSingularitiesBirefringent2003,berryOpticalSingularitiesBianisotropic2005}。},并分别给磁感应场$\bar{B}^{\;\!\textcolor{gray}{t}}_{\;\!\textcolor{gray}{z}}$(而非磁场$\bar{H}^{\;\!\textcolor{gray}{t}}_{\;\!\textcolor{gray}{z}}$),自由电流源$\bar{J}^{\;\!\textcolor{gray}{t}}_{\;\!\textcolor{Maroon}{\text{e}}\textcolor{gray}{z}}$ 以及电位移场$\bar{D}^{\;\!\textcolor{gray}{t}}_{\;\!\textcolor{gray}{z}}$,定义了如下 3 个本构关系(\textcolor{Maroon}{\text{Constitutive Relation}} = \textcolor{Maroon}{\text{CR}})。

其一,磁场 $\bar{H}^{\;\!\textcolor{gray}{t}}_{\;\!\textcolor{gray}{z}}$ 的本构关系,考虑$\bar{M}^{\;\!\textcolor{gray}{t}}_{\;\!\textcolor{gray}{z}}$\Footnote{磁化强度。其在微观上来源于:分子电流(电子轨道运动)产生的磁矩和电子自旋磁矩的矢量和\cite{nelsonLagrangianTreatmentMagnetic1994},细究还将有电子与原子核、电子的自旋轨道耦合、电子电子间相互作用(多电子产生原子磁矩)、原子与原子间(晶体场)的相互作用\cite{chen-zhuChenZhuxieUndergraduate_courses2024}。磁的起源看上去可以是纯电的\cite{lakhtakiaGenesisPostConstraint2004}。}仅由磁偶极矩\Footnote{即只考虑最低阶磁多极矩 = 不考虑磁四极矩及以上\cite{nelsonLagrangianTreatmentMagnetic1994},因为对于受到电磁场的影响后,反过来产生电磁场的电子而言,其受到的电场力是洛伦兹力的c$\big/v$倍\cite{boydNonlinearOptics2019}。此外,(磁/电)多极矩与(磁/电)非线性是互相独立的——即任何阶的多极矩,都有自己的线性项和非线性项\cite{chen-zhuChenZhuxieUndergraduate_courses2024},这些项与其它阶多极矩的任何项都无关。}贡献时,定义为
\begin{subequations} \label{eq:cr-b}
\begin{align}
	\textcolor{Maroon}{\text{CR for magnetism}}\text{:}&\hspace{0.5em} \bar{B}^{\;\!\textcolor{gray}{t}}_{\;\!\textcolor{gray}{z}} \hspace{-0.7em} &&={\symup{\mu}}_0 \left( \bar{H}^{\;\!\textcolor{gray}{t}}_{\;\!\textcolor{gray}{z}} + \bar{M}^{\;\!\textcolor{gray}{t}}_{\;\!\textcolor{gray}{z}} \right) = {\symup{\mu}}_0 \left\{ \bar{\bar{\delta}}^{\;\!\textcolor{gray}{t}}~\widetilde *~\bar{H}^{\;\!\textcolor{gray}{t}}_{\;\!\textcolor{gray}{z}} + \bar{M}^{\;\!\textcolor{gray}{t}}_{\;\!\textcolor{gray}{z}} \right\} \label{cr-b1} \\ & &&\xrightarrow[]{\bar{M}^{\;\!\textcolor{gray}{t}}_{\;\!\textcolor{gray}{z}} = \bar{M}^{\;\!\textcolor{Maroon}{\text{(1)}} \textcolor{gray}{t}}_{\;\!\textcolor{gray}{z}} + \bar{M}^{\;\!\textcolor{Maroon}{\text{NL}}, \textcolor{gray}{t}}_{\;\!\textcolor{gray}{z}}} {\symup{\mu}}_0 \left\{ \left[ \bar{\bar{\delta}}^{\;\!\textcolor{gray}{t}}~\widetilde *~\bar{H}^{\;\!\textcolor{gray}{t}}_{\;\!\textcolor{gray}{z}} + \bar{M}^{\;\!\textcolor{Maroon}{\text{(1)}} \textcolor{gray}{t}}_{\;\!\textcolor{gray}{z}} \right] + \bar{M}^{\;\!\textcolor{Maroon}{\text{NL}}, \textcolor{gray}{t}}_{\;\!\textcolor{gray}{z}} \right\} \label{cr-b2} \\ & &&\xrightarrow[\displaystyle{ \bar{\bar{\mu}}^{\;\!\textcolor{Maroon}{\text{(1)}} \textcolor{gray}{t}}_{\;\!\textcolor{Maroon}{\text{r}}\textcolor{gray}{z}} := \bar{\bar{\delta}}^{\;\!\textcolor{gray}{t}} + \bar{\bar{\chi}}^{\;\!\textcolor{Maroon}{\text{(1)}}\textcolor{gray}{t}}_{\;\!\textcolor{Maroon}{\text{m}} \textcolor{gray}{z}}}]{\displaystyle{\bar{M}^{\;\!\textcolor{Maroon}{\text{(1)}} \textcolor{gray}{t}}_{\;\!\textcolor{gray}{z}} := \bar{\bar{\chi}}^{\;\!\textcolor{Maroon}{\text{(1)}}\textcolor{gray}{t}}_{\;\!\textcolor{Maroon}{\text{m}} \textcolor{gray}{z}} ~\widetilde *~\bar{H}^{\;\!\textcolor{gray}{t}}_{\;\!\textcolor{gray}{z}}}} {\symup{\mu}}_0 \left\{ \bar{\bar{\mu}}^{\;\!\textcolor{Maroon}{\text{(1)}} \textcolor{gray}{t}}_{\;\!\textcolor{Maroon}{\text{r}}\textcolor{gray}{z}}~\widetilde *~\bar{H}^{\;\!\textcolor{gray}{t}}_{\;\!\textcolor{gray}{z}} + \bar{M}^{\;\!\textcolor{Maroon}{\text{NL}}, \textcolor{gray}{t}}_{\;\!\textcolor{gray}{z}} \right\} \label{cr-b3} \\ & &&= \bar{\bar{\mu}}^{\;\!\textcolor{Maroon}{\text{(1)}} \textcolor{gray}{t}}_{\;\!\textcolor{gray}{z}}~\widetilde *~\bar{H}^{\;\!\textcolor{gray}{t}}_{\;\!\textcolor{gray}{z}} + {\symup{\mu}}_0 \bar{M}^{\;\!\textcolor{Maroon}{\text{NL}}, \textcolor{gray}{t}}_{\;\!\textcolor{gray}{z}} =: \bar{B}^{\;\!\textcolor{Maroon}{\text{(1)}} \textcolor{gray}{t}}_{\;\!\textcolor{gray}{z}} + \bar{B}^{\;\!\textcolor{Maroon}{\text{NL}}, \textcolor{gray}{t}}_{\;\!\textcolor{gray}{z}}~, \label{cr-b4}
\end{align}
\end{subequations}
其中,磁通量密度场 $\bar{B}^{\;\!\textcolor{gray}{t}}_{\;\!\textcolor{gray}{z}}$(直接/显示地)关于磁场 $\bar{H}^{\;\!\textcolor{gray}{t}}_{\;\!\textcolor{gray}{z}}$\Footnote{磁非线性,如郎之万顺磁性理论\cite{chen-zhuChenZhuxieUndergraduate_courses2024}中 $M \propto$ 郎之万函数 $\mathcal{L} \left( \alpha \right) = \coth \left( \alpha \right) - 1 / \alpha$(其中 $\alpha \propto H_{\textcolor{Maroon}{\text{ex}}}$)或其量子化修正之布里渊函数,铁磁体\cite{chen-zhuChenZhuxieUndergraduate_courses2024}或超导体\cite{wenBriefIntroductionFlux2021}中的磁滞现象等(每个时刻$t$,这些场量都是准静态$\Omega \to 0$的)。}、电场 $\bar{E}^{\;\!\textcolor{gray}{t}}_{\;\!\textcolor{gray}{z}}$\Footnote{双各向异性,其中的电$\to$磁耦合(如果 $\bar{B}^{\;\!\textcolor{gray}{t}}_{\;\!\textcolor{gray}{z}}$ 中的该部分只是 $\bar{E}^{\;\!\textcolor{gray}{t}}_{\;\!\textcolor{gray}{z}}$ 的线性函数,则也可归结到线性项中)。}、应力 $\bar{T}^{\;\!\textcolor{gray}{t}}_{\;\!\textcolor{gray}{z}}$\Footnote{正逆压磁/磁致伸缩/磁弹效应(这里未作区分)。}等其他场量(即含空 $\textcolor{gray}{\bar{r}}$ 的物理量)\Footnote{$\bar{B}^{\;\!\textcolor{gray}{t}}_{\;\!\textcolor{gray}{z}},\bar{M}^{\;\!\textcolor{gray}{t}}_{\;\!\textcolor{gray}{z}}$ 以及各阶 $\bar{\bar{\mu}}^{\;\!\textcolor{Maroon}{\text{(1)}} \textcolor{gray}{t}}_{\;\!\textcolor{gray}{z}},\bar{\bar{\bar{\mu}}}^{\;\!\textcolor{Maroon}{\text{(2)}} \textcolor{gray}{t}}_{\;\!\textcolor{gray}{z}},\cdots$ 已经是关于温度$T$、波长$\lambda$(或 角频率$\omega$、时间$t$)等(非)场量的函数。}的非线性函数项,悉数包含在 $\bar{B}^{\;\!\textcolor{Maroon}{\text{NL}}, \textcolor{gray}{t}}_{\;\!\textcolor{gray}{z}} = {\symup{\mu}}_0 \bar{M}^{\;\!\textcolor{Maroon}{\text{NL}}, \textcolor{gray}{t}}_{\;\!\textcolor{gray}{z}}$ 内;剩余的线性项,放在 $\bar{B}^{\;\!\textcolor{Maroon}{\text{(1)}} \textcolor{gray}{t}}_{\;\!\textcolor{gray}{z}} = \bar{\bar{\mu}}^{\;\!\textcolor{Maroon}{\text{(1)}} \textcolor{gray}{t}}_{\;\!\textcolor{gray}{z}}~\widetilde *~\bar{H}^{\;\!\textcolor{gray}{t}}_{\;\!\textcolor{gray}{z}}$ 中。

其二,自由电流源$\bar{J}^{\;\!\textcolor{gray}{t}}_{\;\!\textcolor{Maroon}{\text{e}}\textcolor{gray}{z}}$的本构关系,包含欧姆定律的线性部分(漂移项) $\bar{J}^{\;\!\textcolor{Maroon}{\text{(1)}} \textcolor{gray}{t}}_{\;\!\textcolor{Maroon}{\text{e}}\textcolor{gray}{z}} = \bar{\bar{\sigma}}^{\;\!\textcolor{Maroon}{\text{(1)}}\textcolor{gray}{t}}_{\;\!\textcolor{gray}{z}}~\widetilde *~\bar{E}^{\;\!\textcolor{gray}{t}}_{\;\!\textcolor{gray}{z}}$\Footnote{可以由 Drude 模型描述,定量解释一阶电导率$\bar{\bar{\sigma}}^{\;\!\textcolor{Maroon}{\text{(1)}}\textcolor{gray}{t}}_{\;\!\textcolor{gray}{z}}$的起源。},及$\bar{J}^{\;\!\textcolor{gray}{t}}_{\;\!\textcolor{Maroon}{\text{e}}\textcolor{gray}{z}}$分别关于电场 $\bar{E}^{\;\!\textcolor{gray}{t}}_{\;\!\textcolor{gray}{z}}$\Footnote{欧姆定律中的电非线性部分,比如二/三极管的伏安特性曲线\cite{chen-zhuChenZhuxieUndergraduate_courses2024}(尽管输入/输出or自/因变量,即$\bar{E}^{\;\!\textcolor{gray}{t}}_{\;\!\textcolor{gray}{z}}$和$\bar{J}^{\;\!\textcolor{gray}{t}}_{\;\!\textcolor{Maroon}{\text{e}}\textcolor{gray}{z}}$,一般均在直流或低频$\Omega$,非交流且不在光波段 opt)。}、磁感应场$\bar{B}^{\;\!\textcolor{gray}{t}}_{\;\!\textcolor{gray}{z}}$\Footnote{磁感应场$\bar{B}^{\;\!\textcolor{gray}{t}}_{\;\!\textcolor{gray}{z}}$所带来的(电场力以外的)洛伦兹力$\left( \bar{J}^{\;\!\textcolor{gray}{t}}_{\;\!\textcolor{Maroon}{\text{e}}\textcolor{gray}{z}} + \dot{\bar{P}}^{\;\!\textcolor{gray}{t}}_{\;\!\textcolor{gray}{z}} + \bar{\nabla} \times \bar{M}^{\;\!\textcolor{gray}{t}}_{\;\!\textcolor{gray}{z}} \right) \times \bar{B}^{\;\!\textcolor{gray}{t}}_{\;\!\textcolor{gray}{z}}$\cite{mackayElectromagneticAnisotropyBianisotropy2019,chen-zhuChenZhuxieUndergraduate_courses2024},会影响导/价带电子的运动(速度)$\bar{v}^{\;\!\textcolor{gray}{t}}_{\;\!\textcolor{Maroon}{\text{e}}\textcolor{gray}{z}}$,进而全局地影响自由电流$\bar{J}^{\;\!\textcolor{gray}{t}}_{\;\!\textcolor{Maroon}{\text{e}}\textcolor{gray}{z}} = {\rho}^{\;\!\textcolor{gray}{t}}_{\;\!\textcolor{Maroon}{\text{e}}\textcolor{gray}{z}} \bar{v}^{\;\!\textcolor{gray}{t}}_{\;\!\textcolor{Maroon}{\text{e}}\textcolor{gray}{z}}$和(束缚)电(偶)极化强度$\bar{P}^{\;\!\textcolor{gray}{t}}_{\;\!\textcolor{gray}{z}}$,包括它们的线性和非线性项\cite{boydNonlinearOptics2019}。对于强场/超快非线性光学,相对论效应使得电磁场是个统一的整体,动生(而不仅是外加)的$\bar{B}^{\;\!\textcolor{gray}{t}}_{\;\!\textcolor{gray}{z}}$还将带来额外的影响。磁场$\bar{H}^{\;\!\textcolor{gray}{t}}_{\;\!\textcolor{gray}{z}}$对分子/磁化电流体密度$\bar{\nabla} \times \bar{M}^{\;\!\textcolor{gray}{t}}_{\;\!\textcolor{gray}{z}}$产生的影响已包含在$\bar{M}^{\;\!\textcolor{gray}{t}}_{\;\!\textcolor{gray}{z}}$中了。}、导带电子浓度(数密度)梯度场$\bar{\nabla} {\rho}^{\;\!\textcolor{gray}{t}}_{\;\!\textcolor{Maroon}{\text{e}}\textcolor{gray}{z}}$\Footnote{在光折变效应中,作为$\bar{J}^{\;\!\textcolor{gray}{t}}_{\;\!\textcolor{Maroon}{\text{e}}\textcolor{gray}{z}}$中的扩散项\cite{boydNonlinearOptics2019}。${\rho}^{\;\!\textcolor{gray}{t}}_{\;\!\textcolor{Maroon}{\text{e}}\textcolor{gray}{z}}, \bar{J}^{\;\!\textcolor{gray}{t}}_{\;\!\textcolor{Maroon}{\text{e}}\textcolor{gray}{z}}$之间还应满足\bref{eq:continuity-e}以及$\bar{J}^{\;\!\textcolor{gray}{t}}_{\;\!\textcolor{Maroon}{\text{e}}\textcolor{gray}{z}} = {\rho}^{\;\!\textcolor{gray}{t}}_{\;\!\textcolor{Maroon}{\text{e}}\textcolor{gray}{z}} \bar{v}^{\;\!\textcolor{gray}{t}}_{\;\!\textcolor{Maroon}{\text{e}}\textcolor{gray}{z}}$\cite{chen-zhuChenZhuxieUndergraduate_courses2024}。}和光伏电流场$\propto \lvert \bar{E}^{\;\!\textcolor{gray}{t}}_{\;\!\textcolor{gray}{z}} \rvert^2 \hat{c}$\Footnote{与光电导效应并列,属于内光电效应;也可能在光折变效应的$\bar{J}^{\;\!\textcolor{gray}{t}}_{\;\!\textcolor{Maroon}{\text{e}}\textcolor{gray}{z}}$中扮演一份角色,特别是沿着一些各向异性晶体的光轴$\hat{c}$产生电势差和内建电场\cite{boydNonlinearOptics2019}(尽管一般也只影响直流或低频$\Omega$的$\bar{J}^{\;\!\textcolor{gray}{t}}_{\;\!\textcolor{Maroon}{\text{e}}\textcolor{gray}{z}}$;但$\bar{J}^{\;\!\textcolor{gray}{t}}_{\;\!\textcolor{Maroon}{\text{e}}\textcolor{gray}{z}}$会通过影响光波段的介电常数,进而影响光波段的光强$\lvert \bar{E}^{\;\!\textcolor{gray}{t}}_{\;\!\textcolor{gray}{z}} \rvert^2$及$\bar{J}^{\;\!\textcolor{gray}{t}}_{\;\!\textcolor{Maroon}{\text{e}}\textcolor{gray}{z}}$自己的重新分布);该二阶的带耦合的非线性,看上去很像非线性极化率$\bar{P}^{\;\!\textcolor{Maroon}{\text{(2)}} \textcolor{gray}{t}}_{\;\!\textcolor{gray}{z}}$中的光整流项,但其频率比 THz 低,且只服务于自由电流。—— 以至该项可作为差频合并至$\bar{J}^{\;\!\textcolor{gray}{t}}_{\;\!\textcolor{Maroon}{\text{e}}\textcolor{gray}{z}}$关于$\bar{E}^{\;\!\textcolor{gray}{t}}_{\;\!\textcolor{gray}{z}}$的二阶非线性$\bar{J}^{\;\!\textcolor{Maroon}{\text{(2)}} \textcolor{gray}{t}}_{\;\!\textcolor{Maroon}{\text{e}}\textcolor{gray}{z}}$中去?}等其他场量的非线性项 $\bar{J}^{\;\!\textcolor{Maroon}{\text{NL}}, \textcolor{gray}{t}}_{\;\!\textcolor{Maroon}{\text{e}}\textcolor{gray}{z}}$:
\begin{equation} \label{eq:cr-j}
	\textcolor{Maroon}{\text{Ohm's law}}\text{:}\hspace{0.5em} \bar{J}^{\;\!\textcolor{gray}{t}}_{\;\!\textcolor{Maroon}{\text{e}}\textcolor{gray}{z}} = \bar{\bar{\sigma}}^{\;\!\textcolor{Maroon}{\text{(1)}}\textcolor{gray}{t}}_{\;\!\textcolor{gray}{z}}~\widetilde *~\bar{E}^{\;\!\textcolor{gray}{t}}_{\;\!\textcolor{gray}{z}} + \bar{J}^{\;\!\textcolor{Maroon}{\text{NL}}, \textcolor{gray}{t}}_{\;\!\textcolor{Maroon}{\text{e}}\textcolor{gray}{z}} =: \bar{J}^{\;\!\textcolor{Maroon}{\text{(1)}} \textcolor{gray}{t}}_{\;\!\textcolor{Maroon}{\text{e}}\textcolor{gray}{z}} + \bar{J}^{\;\!\textcolor{Maroon}{\text{NL}}, \textcolor{gray}{t}}_{\;\!\textcolor{Maroon}{\text{e}}\textcolor{gray}{z}}~,
\end{equation}

其三,电位移场$\bar{D}^{\;\!\textcolor{gray}{t}}_{\;\!\textcolor{gray}{z}}$的本构关系,当$\bar{P}^{\;\!\textcolor{gray}{t}}_{\;\!\textcolor{gray}{z}}$只由电偶极矩\Footnote{不考虑电四极矩及以上。但电四极化强度场$\bar{\bar{Q}}^{\;\!\textcolor{gray}{t}}_{\;\!\textcolor{gray}{z}}$(的等效电偶极化强度场$\bar{P}^{\;\!\textcolor{gray}{t}}_{\;\!\textcolor{Maroon}{\text{Q}}\textcolor{gray}{z}} = - \bar{\nabla} \cdot \bar{\bar{Q}}^{\;\!\textcolor{gray}{t}}_{\;\!\textcolor{gray}{z}}$)\cite{chen-zhuChenZhuxieUndergraduate_courses2024}在有些效应中不可忽视且起关键作用:如其对线性晶体光学中的光学活性的贡献\cite{nelsonDerivingTransmissionReflection1995},以及非线性光学中基于$\bar{\bar{Q}}^{\;\!\textcolor{gray}{t}}_{\;\!\textcolor{gray}{z}}$的二阶和频\cite{bethuneOpticalQuadrupoleSumfrequency1976}。电四极子对光与物质相互作用的贡献,还会打破$\bar{D}^{\;\!\textcolor{gray}{t}}_{\;\!\textcolor{gray}{z}}$法向连续和$\bar{H}^{\;\!\textcolor{gray}{t}}_{\;\!\textcolor{gray}{z}}$切向连续边界条件,并与洛伦兹力的定义、(由唯二的无源齐次\cite{lakhtakiaGenesisPostConstraint2004}微分方程\bref{eq:maxwell-e,eq:maxwell-b}导出的)电磁场标/矢势\cite{chen-zhuChenZhuxieUndergraduate_courses2024}等一起,使$\bar{E}^{\;\!\textcolor{gray}{t}}_{\;\!\textcolor{gray}{z}},\bar{B}^{\;\!\textcolor{gray}{t}}_{\;\!\textcolor{gray}{z}}$而不是$\bar{E}^{\;\!\textcolor{gray}{t}}_{\;\!\textcolor{gray}{z}},\bar{H}^{\;\!\textcolor{gray}{t}}_{\;\!\textcolor{gray}{z}}$成为基本场\cite{nelsonDerivingTransmissionReflection1995},对应地,坡印亭矢量也需要修正为$\bar{E}^{\;\!\textcolor{gray}{t}}_{\;\!\textcolor{gray}{z}} \times \bar{B}^{\;\!\textcolor{gray}{t}}_{\;\!\textcolor{gray}{z}} \big/ {\symup{\mu}}_0$\cite{nelsonGeneralizingPoyntingVector1996,loudonPropagationElectromagneticEnergy1997}而不是$\bar{E}^{\;\!\textcolor{gray}{t}}_{\;\!\textcolor{gray}{z}} \times \bar{H}^{\;\!\textcolor{gray}{t}}_{\;\!\textcolor{gray}{z}}$。}构成时,定义为
\begin{subequations} \label{eq:cr-d}
\begin{align}
	\textcolor{Maroon}{\text{CR for electricity}}\text{:}&\hspace{0.5em} \bar{D}^{\;\!\textcolor{gray}{t}}_{\;\!\textcolor{gray}{z}} \hspace{-2.0em} &&= {\symup{\varepsilon}}_0 \bar{E}^{\;\!\textcolor{gray}{t}}_{\;\!\textcolor{gray}{z}} + \bar{P}^{\;\!\textcolor{gray}{t}}_{\;\!\textcolor{gray}{z}} = {\symup{\varepsilon}}_0 \bar{\bar{\delta}}^{\;\!\textcolor{gray}{t}}~\widetilde *~\bar{E}^{\;\!\textcolor{gray}{t}}_{\;\!\textcolor{gray}{z}} + \bar{P}^{\;\!\textcolor{gray}{t}}_{\;\!\textcolor{gray}{z}} \label{cr-d1} \\ & &&\xrightarrow[]{\bar{P}^{\;\!\textcolor{gray}{t}}_{\;\!\textcolor{gray}{z}} = \bar{P}^{\;\!\textcolor{Maroon}{\text{(1)}} \textcolor{gray}{t}}_{\;\!\textcolor{gray}{z}} + \bar{P}^{\;\!\textcolor{Maroon}{\text{NL}}, \textcolor{gray}{t}}_{\;\!\textcolor{gray}{z}} + } \left[ {\symup{\varepsilon}}_0 \bar{\bar{\delta}}^{\;\!\textcolor{gray}{t}}~\widetilde *~\bar{E}^{\;\!\textcolor{gray}{t}}_{\;\!\textcolor{gray}{z}} + \bar{P}^{\;\!\textcolor{Maroon}{\text{(1)}} \textcolor{gray}{t}}_{\;\!\textcolor{gray}{z}} \right] + \bar{P}^{\;\!\textcolor{Maroon}{\text{NL}}, \textcolor{gray}{t}}_{\;\!\textcolor{gray}{z}} \label{cr-d2} \\ & &&\xrightarrow[\displaystyle{ \bar{\bar{\varepsilon}}^{\;\!\textcolor{Maroon}{\text{(1)}} \textcolor{gray}{t}}_{\;\!\textcolor{Maroon}{\text{r}}\textcolor{gray}{z}} := \bar{\bar{\delta}}^{\;\!\textcolor{gray}{t}} + \bar{\bar{\chi}}^{\;\!\textcolor{Maroon}{\text{(1)}}\textcolor{gray}{t}}_{\;\!\textcolor{Maroon}{\text{e}} \textcolor{gray}{z}}}]{\displaystyle{\bar{P}^{\;\!\textcolor{Maroon}{\text{(1)}} \textcolor{gray}{t}}_{\;\!\textcolor{gray}{z}} := \bar{\bar{\chi}}^{\;\!\textcolor{Maroon}{\text{(1)}}\textcolor{gray}{t}}_{\;\!\textcolor{Maroon}{\text{e}} \textcolor{gray}{z}} ~\widetilde *~\bar{E}^{\;\!\textcolor{gray}{t}}_{\;\!\textcolor{gray}{z}}}} {\symup{\varepsilon}}_0 \bar{\bar{\varepsilon}}^{\;\!\textcolor{Maroon}{\text{(1)}} \textcolor{gray}{t}}_{\;\!\textcolor{Maroon}{\text{r}}\textcolor{gray}{z}}~\widetilde *~\bar{E}^{\;\!\textcolor{gray}{t}}_{\;\!\textcolor{gray}{z}} + \bar{P}^{\;\!\textcolor{Maroon}{\text{NL}}, \textcolor{gray}{t}}_{\;\!\textcolor{gray}{z}} \label{cr-d3} \\ & &&= \bar{\bar{\varepsilon}}^{\;\!\textcolor{Maroon}{\text{(1)}} \textcolor{gray}{t}}_{\;\!\textcolor{gray}{z}}~\widetilde *~\bar{E}^{\;\!\textcolor{gray}{t}}_{\;\!\textcolor{gray}{z}} + \bar{P}^{\;\!\textcolor{Maroon}{\text{NL}}, \textcolor{gray}{t}}_{\;\!\textcolor{gray}{z}} =: \bar{D}^{\;\!\textcolor{Maroon}{\text{(1)}} \textcolor{gray}{t}}_{\;\!\textcolor{gray}{z}} + \bar{D}^{\;\!\textcolor{Maroon}{\text{NL}}, \textcolor{gray}{t}}_{\;\!\textcolor{gray}{z}}~, \label{cr-d4}
\end{align}
\end{subequations}
关于其组成成分,电位移场 $\bar{D}^{\;\!\textcolor{gray}{t}}_{\;\!\textcolor{gray}{z}}$(直接/显示地)关于电场 $\bar{E}^{\;\!\textcolor{gray}{t}}_{\;\!\textcolor{gray}{z}}$\Footnote{电非线性,包括高频段的(非)共振非线性、低频低温\cite{lakhtakiaGenesisPostConstraint2004}段的铁电体/畴的电滞现象等。}、磁场 $\bar{H}^{\;\!\textcolor{gray}{t}}_{\;\!\textcolor{gray}{z}}$\Footnote{双各向异性中的磁$\to$电耦合(如果 $\bar{D}^{\;\!\textcolor{gray}{t}}_{\;\!\textcolor{gray}{z}}$ 中的该部分只是 $\bar{H}^{\;\!\textcolor{gray}{t}}_{\;\!\textcolor{gray}{z}}$ 的线性函数,则也可归结到线性项中)。}、应力 $\bar{T}^{\;\!\textcolor{gray}{t}}_{\;\!\textcolor{gray}{z}}$\Footnote{正逆压磁/磁致伸缩/磁弹效应(这里未作区分)。}等其他场量的非线性函数项,均由 $\bar{D}^{\;\!\textcolor{Maroon}{\text{NL}}, \textcolor{gray}{t}}_{\;\!\textcolor{gray}{z}} = {\symup{\mu}}_0 \bar{M}^{\;\!\textcolor{Maroon}{\text{NL}}, \textcolor{gray}{t}}_{\;\!\textcolor{gray}{z}}$ 贡献;剩余的线性项,由 $\bar{B}^{\;\!\textcolor{Maroon}{\text{(1)}} \textcolor{gray}{t}}_{\;\!\textcolor{gray}{z}} = \bar{\bar{\mu}}^{\;\!\textcolor{Maroon}{\text{(1)}} \textcolor{gray}{t}}_{\;\!\textcolor{gray}{z}}~\widetilde *~\bar{H}^{\;\!\textcolor{gray}{t}}_{\;\!\textcolor{gray}{z}}$ 表示。




%---------------------------------------------------------------------
%	参考文献
%---------------------------------------------------------------------

% 生成参考文献页
\printbibliography

%---------------------------------------------------------------------
%	致谢
%---------------------------------------------------------------------

\begin{acknowledgement}
  感谢 \href{https://git.nju.edu.cn/nju-lug/lug-introduction}{LUG@NJU}。
\end{acknowledgement}

%---------------------------------------------------------------------
%	学术简历
%---------------------------------------------------------------------

% 详见手册中“成果列表”一节
% \njuchapter{学术成果}
% \njupaperlist[攻读博士学位期间发表的学术论文]{preskill2018}

%---------------------------------------------------------------------
%	附录部分
%---------------------------------------------------------------------

% 附录部分使用单独的字母序号
\appendix
\let\appthechapter\thechapter% for secbacklinktoc .sty
\let\appthesection\thesection% for secbacklinktoc .sty
\let\appthesubsection\thesubsection% for secbacklinktoc .sty
\let\appthesubsubsection\thesubsubsection% for secbacklinktoc .sty
\counterwithin{figure}{section} % for custom .sty
\counterwithin{table}{section}% for custom .sty

% 可以在这里插入补充材料
\appchapter{正文中涉及的数据及源代码}

\appsection{类 Class - 实例 Instance}

\begin{table}[h!]
	\caption{\label{tab:font} 字体 —— 用于区分变量与常量,以及不同类变量,的含义。}
	\resizebox{1.0\linewidth}{!}{  % 宽度不超文本宽度
		\begin{tabular}{c|c|c|c|c}
			\toprule[2pt]
			类 Class & 对象 Object & 含义 Meaning & 实例 Instance & 来源 Origin \\ \midrule[1.2pt]
			字体 font & 直体 upright & 常量(非变量非场量非函数)、文字(说明) & $\upmu_0$ 中的 $\upmu$ 和 $0$ & \blabel{Maroon} \\
			& \textit{斜体 oblique} & 变量(自变量或因变量)、场量、函数 & $\bar{E}$ 中的 $E$ & \blabel{oblique} \\ \midrule
			粗细 width & 细体 thin\hphantom{k} & \hphantom{列向量 =} 张量:2 阶张量 & $\bar{\bar{\chi}},\bar{\bar{\varepsilon}}$ & \blabel{Maroon} \\
			& \textbf{粗体 thick} & \hphantom{列向量 =} 张量:3 阶张量 & $\bar{\bar{\bar{\chi}}}$ & \blabel{Maroon} \\ \midrule
			下线 down & 0 根下短线 \hphantom{$\underline{~}$} & $\in$ 实验(室)坐标系 = $\mathcal{Z}$ 系(下的物理量) & $\bar{\symbf{g}}$ & \blabel{Maroon} \\
			& 1 根下短线 $\underline{~}$ & $\in$ 晶体(学)坐标系 = $\mathcal{C}$ 系(下的物理量) & $\bar{\underline{\symbf{g}}}$ & \blabel{Maroon} \\ \midrule
			长线 long & 1 根上长线 $\overline{~~}$ & 广义列向量,其中每个元素可能都是列向量 & $\symbf{\underline{\bar{g}}}$ & \blabel{Maroon} \\
			& 1 根下长线 $\underline{~~}$ & 未定义;保留的关键字符 &  & \blabel{Maroon} \\ 
			\bottomrule[2pt]
		\end{tabular}
	}
\end{table}

\begin{table}[h!]
	\caption{\label{tab:line} 上/下 \& 长/短 划线 —— 作为变量的装饰,的含义。}
	\resizebox{1.0\linewidth}{!}{  % 宽度不超文本宽度
		\begin{tabular}{c|c|c|c|c}
			\toprule[2pt]
			类 Class & 对象 Object & 含义 Meaning & 实例 Instance & 来源 Origin \\ \midrule[1.2pt]
			划线 line & 0 条上短线 \hphantom{$\bar{~}$} & \hphantom{列向量 =} 标量 = 0 阶张量 & ${\rho}_{\;\!\textcolor{Maroon}{\symup{f}}}$ 中的 $\rho$ & \blabel{0bar} \\
			& 1 条上短线 $\bar{~}$ & 列向量 = 矢量 = 1 阶张量 & $\bar{J}_{\;\!\textcolor{Maroon}{\symup{f}}}$ 中的 $\bar{J}$ & \blabel{1bar} \\
			上线 up & 2 条上短线 $\bar{\bar{~}}$ & \hphantom{列向量 =} 张量:2 阶张量 & $\bar{J}_{\;\!\textcolor{Maroon}{\symup{f}}}$ 中的 $\bar{J}$ & \blabel{Maroon} \\
			& 3 条上短线 $\bar{\bar{\bar{~}}}$ & \hphantom{列向量 =} 张量:3 阶张量 & $\bar{\bar{\bar{\chi}}}$ & \blabel{Maroon} \\ \midrule
			长线 long & 1 条上长线 $\overline{~~}$ & 广义列向量,其中每个元素可能都是列向量 & $\symbf{\underline{\bar{g}}}$ & \blabel{Maroon} \\
			& 2 条上长线 $\overline{\overline{~~}}$ & 广义张量:对角 2 阶张量 &  & \blabel{Maroon} \\ \midrule
			下线 down & 0 条下短线 \hphantom{$\underline{~}$} & $\in$ 实验(室)坐标系 = $\mathcal{Z}$ 系(下的物理量) & $\bar{\symbf{g}}$ & \blabel{Maroon} \\
			& 1 条下短线 $\underline{~}$ & $\in$ 晶体(学)坐标系 = $\mathcal{C}$ 系(下的物理量) & $\bar{\underline{\symbf{g}}}$ & \blabel{Maroon} \\ 
			\bottomrule[2pt]
		\end{tabular}
	}
\end{table}

\begin{table}[h!]
	\caption{\label{tab:color} $J \in \left[ 1, 10 \right]$ 的和弦级数的 $a_0, \left\{ a_j, b_j \right\}$ 划线 参数。}
	\resizebox{1.0\linewidth}{!}{  % 宽度不超文本宽度
		\begin{tabular}{c|c|c|c|c}
			\toprule[2pt]
			类 Class & 对象 Object & 含义 Meaning & 实例 Instance & 来源 Origin \\ \midrule[1.2pt]
			颜色 color & 黑色 black & 非自变量,包括因变量和(非函数=无自变量的)独立变量 & $\bar{E}^{\;\!\textcolor{gray}{t}}_{\;\!\textcolor{gray}{z}}$ 中的 $\bar{E}$ & \blabel{black} \\
			& {\color{gray} 灰色 gray} & (附属于某个因变量的)自变量,在视觉和含义上均更次要 & $\bar{E}^{\;\!\textcolor{gray}{t}}_{\;\!\textcolor{gray}{z}}$ 中的 $\textcolor{gray}{t,z}$ & \blabel{gray} \\
			& {\color{Maroon} 褐红 Maroon} & 起补充说明作用的直体文字脚标 & ${\rho}_{\;\!\textcolor{Maroon}{\symup{f}}}, \bar{J}_{\;\!\textcolor{Maroon}{\symup{f}}}$ 中的 $\textcolor{Maroon}{\symup{f}}$ & \blabel{Maroon} \\
			$b_3$  &        &        & 2.7821 & 1.7670 \\
			$b_5$  &        &        &        &        \\
			$b_6$  &        &        &        &        \\
			$b_7$  &        &        &        &        \\
			$b_8$  &        &        &        &        \\
			$b_9$  &        &        &        &        \\
			$b_{10}$ &        &        &        &      \\ \bottomrule[2pt]
		\end{tabular}
	}
\end{table}

\appsubsection{颜色 Color - 褐红 Maroon}

	

\appsubsection{ Color}

\dots

% 完工
\end{document}
