%!TEX root = ..\njuthesis-sample.tex

\chapter{任意 \texorpdfstring{$\chi ( \bar{r} )$}{$\text{χ} ( \bar{r} )$} 材料中的(标量)非线性角谱理论}\label{chap:NLAST}\marginLeft[5.8em]{chap:NLAST}

在\textcolor{Maroon}{非线性角谱}之前,求解\textcolor{Plum}{非线性}光学衍射过程所\textbf{面临的问题}有二。

{\one} 如果不把\textcolor{Plum}{微-积分方程} \bref{eq:up-scalar-g-EE-312-discrete-convolution4,eq:Born_approx-scalar-g-E-12-convolution4} 沿 $\mathcolor{gray}{z}$ \textcolor{Plum}{定积分},那么在\textcolor{gray}{纵向}上,一般进行\textcolor{Maroon}{分步傅立叶变换}(\textcolor{Maroon}{split-step Fourier transform}, \textcolor{Maroon}{SSF})\cite{ellenbogenNonlinearGenerationManipulation2009,saltielCerenkovTypeSecondHarmonicGeneration2009, barsiImagingNonlinearMedia2009, trajtenberg-millsSimulatingCorrelationsStructured2020, rozenbergInverseDesignSpontaneous2022, yesharimObservationAllopticalStern2022}、$\mathcolor{gray}{\bar{k}_{\symup{\rho}}}$ \textcolor{gray}{空间}\textcolor{Maroon}{龙格库塔法}(\textcolor{Maroon}{Runge-Kutta methods}, \textcolor{Maroon}{RK})\cite{zhangFullyVectorialSimulation2016,zhongKdomainMethodFast2020}迭代求解\textcolor{Plum}{非齐次}\textcolor{NavyBlue}{有源}偏微分方程。—— 这些方法的\textcolor{Plum}{缺点}\Footnote{此外,它还有一个缺点,就是没有统一的\textcolor{Plum}{迭代格式},\textcolor{Plum}{具体算法}在实施时会\textbf{因人而异}(比如,\textcolor{NavyBlue}{场衍射传播}和\textcolor{NavyBlue}{源驱动的场增量}分别各自应实施在\textcolor{gray}{倒空间}还是在\textcolor{gray}{正空间}?)。相应地,\textcolor{Plum}{误差分析}也没有统一的标准。}是:为了保证\textcolor{Plum}{精度},同时受\textcolor{gray}{正空间}\textcolor{NavyBlue}{调制结构}的\textcolor{Plum}{最小线宽}、算法在\textcolor{gray}{倒空间}的\textcolor{Plum}{低速度精度积}限制,\textcolor{Maroon}{SSF} 以及 $\mathcolor{gray}{\bar{k}_{\symup{\rho}}}$\textcolor{gray}{-space} \textcolor{Maroon}{RK} 的每一步的\textcolor{Plum}{步长}必须足够小\Footnote{此外,\textcolor{Plum}{步长}可取到的\textcolor{Plum}{最大值}的\textcolor{Plum}{数学定义}较为模糊:即缺乏\textcolor{Plum}{不显著牺牲精度}的\textcolor{Plum}{最大步长}的\textcolor{Plum}{显式表达式}。},从而导致对于\textcolor{NavyBlue}{长调制区域},需要进行\textcolor{Plum}{较多步数}的\textbf{迭代计算}。

{\two} 如果沿 $\mathcolor{gray}{z}$ \textcolor{Plum}{定积分}了,那么最终只能到达\textcolor{Plum}{非线性卷积}解 \bref{eq:up-scalar-g-EE-312-discrete-since2,eq:down-scalar-g-EE-132-discrete-Since,eq:Born_approx-scalar-g-E-12-since5},而无法继续深入下去。—— 然而,对\textcolor{Plum}{非线性卷积}的运算,要求对 2 维 $\mathcolor{gray}{\bar{k}_{\symup{\rho}}}$ 阵列中的每一个点,都进行一次\textcolor{gray}{横向}\textcolor{Plum}{高维线性卷积}积分(维度 = 2 $\times$ \textcolor{NavyBlue}{驱动源}中的\textcolor{NavyBlue}{混频场}数量),\textcolor{Plum}{for 循环}层数 = 2 $\times$ 参与\textcolor{NavyBlue}{混频}的电场总数。这对于\textcolor{Maroon}{三波混频}过程而言,即使是单个光场的单个\textcolor{gray}{空间频率},\textcolor{Plum}{求和}层数是 $2\times 3=6$,每层索引数 = 该维度\textcolor{Plum}{采样数} $\sim 500$,总的计算量约为 $500^6=1.5625 \times 10^{16}$ 的天文数字。

综上,前、后两个旧方案\Footnote{还有一些其他算法,比如 \textcolor{Maroon}{Green 函数法}\cite{yaoWavefrontPhasemodulationControl2013, chenPhaseMatchingControlledOrbital2020}、\textcolor{Maroon}{转移矩阵法}(\textcolor{Maroon}{transfer-matrix method}, \textcolor{Maroon}{TTM})\cite{liSecondHarmonicGeneration2007, liNonlinearFrequencyConversion2008}等。它们在起源上与\textcolor{Maroon}{非线性角谱}不严格共处于同一条谱系上,没法与\textcolor{Maroon}{非线性角谱}进行直接对比。},分别在横、纵两个方向上,计算负担较大。

\textcolor{Maroon}{非线性角谱}立足于\textcolor{NavyBlue}{无耦合}\textcolor{Plum}{非线性}过程的\textcolor{Plum}{非线性卷积}解 \bref{eq:up-scalar-g-EE-312-discrete-since2,eq:down-scalar-g-EE-132-discrete-Since,eq:Born_approx-scalar-g-E-12-since5}。在\textcolor{Plum}{非线性卷积}解的基础上,尝试将其转化为\textcolor{Plum}{线性卷积},进而将其过渡到\textcolor{Plum}{傅立叶变换}。它既在\textcolor{NavyBlue}{物理}上\textcolor{Plum}{解析}了该\textcolor{Plum}{非线性卷积}过程的\textcolor{Plum}{数学本质};又在\textcolor{NavyBlue}{工程}上,结合现代\textcolor{Maroon}{快速傅立叶变换}(\textcolor{Maroon}{Fast Fourier Transform}, \textcolor{Maroon}{FFT})算法,在尽量不牺牲\textcolor{Plum}{精度}的前提下,大幅加快对上述\textcolor{Plum}{非线性卷积}过程的实际计算\textcolor{Plum}{速度}。

\textcolor{Maroon}{非线性角谱}的\textbf{出发点}很简单:以 \bref{eq:up-scalar-g-EE-312-discrete-convolution4,eq:up-scalar-g-EE-312-discrete-since2} 为例,同样作为\textcolor{Plum}{非线性卷积核},$\mathbb{e}^{\mathbb{i} \Delta \xint{\begin{smallmatrix} ~ \\ {}^{}_{\mathcolor{gray}{-}} \\ ~ \end{smallmatrix}}{15}{k}_{\symup{z}}^{\;\! \textcolor{PineGreen}{\hat{1} \hat{2} \hat{3}} } \mathcolor{gray}{z} }$ 可以被化简为\textcolor{Plum}{线性卷积},但它的\textcolor{Plum}{定积分}结果 $\text{since} \left( \frac{ \Delta \xint{\begin{smallmatrix} ~ \\ {}^{}_{\mathcolor{gray}{-}} \\ ~ \end{smallmatrix}}{15}{k}_{\symup{z}}^{\;\! \textcolor{PineGreen}{\hat{1} \hat{2} \hat{3}} } \mathcolor{gray}{z} }{ 2 } \right) \mathcolor{gray}{z} = \text{sinc} \left( \frac{ \Delta \xint{\begin{smallmatrix} ~ \\ {}^{}_{\mathcolor{gray}{-}} \\ ~ \end{smallmatrix}}{15}{k}_{\symup{z}}^{\;\! \textcolor{PineGreen}{\hat{1} \hat{2} \hat{3}} } \mathcolor{gray}{z} }{ 2 } \right) \mathbb{e}^{\mathbb{i} \frac{ \Delta \xint{\begin{smallmatrix} ~ \\ {}^{}_{\mathcolor{gray}{-}} \\ ~ \end{smallmatrix}}{15}{k}_{\symup{z}}^{\;\! \textcolor{PineGreen}{\hat{1} \hat{2} \hat{3}} } \mathcolor{gray}{z} }{ 2 } } \mathcolor{gray}{z}$ 却因其中的 $\text{sinc} \left( \frac{ \Delta \xint{\begin{smallmatrix} ~ \\ {}^{}_{\mathcolor{gray}{-}} \\ ~ \end{smallmatrix}}{15}{k}_{\symup{z}}^{\;\! \textcolor{PineGreen}{\hat{1} \hat{2} \hat{3}} } \mathcolor{gray}{z} }{ 2 } \right)$ 无法化简为\textcolor{Plum}{线性卷积},而无法直接继承 \textcolor{Maroon}{FFT} 的加速。然而,如果不对 $\mathbb{e}^{\mathbb{i} \Delta \xint{\begin{smallmatrix} ~ \\ {}^{}_{\mathcolor{gray}{-}} \\ ~ \end{smallmatrix}}{15}{k}_{\symup{z}}^{\;\! \textcolor{PineGreen}{\hat{1} \hat{2} \hat{3}} } \mathcolor{gray}{z} }$ \textcolor{Plum}{定积分},又只能退回 \textcolor{Maroon}{SSF} 或 $\mathcolor{gray}{\bar{k}_{\symup{\rho}}}$\textcolor{gray}{-domain} \textcolor{Maroon}{RK} 法。

\textcolor{Maroon}{非线性角谱}的\textbf{核心技巧}即“先入世后出世再入世”般“二渡赤水”(\textcolor{Maroon}{ZigZag}?):既然 $\text{sinc} \left( x \right)$ 本身无法分解为 $\mathbb{e}$ 指数之和及 $\mathbb{e}$ 指数之积的排列组合\Footnote{进而分配与对应的初始电场标量\textcolor{Maroon}{时空谱} $\xint{{}^{}_{\mathcolor{gray}{-}}}{10}{g}^{\;\! \textcolor{PineGreen}{\hat{1}}}_{\;\! \hat{1} \mathcolor{gray}{0}} \left( \mathcolor{gray}{\bar{k}_{1\symup{\rho}}} \right), \xint{{}^{}_{\mathcolor{gray}{-}}}{10}{g}^{\;\! \textcolor{PineGreen}{\hat{2}}}_{\;\! \hat{2} \mathcolor{gray}{0}} \left( \mathcolor{gray}{\bar{k}_{2\symup{\rho}}} \right)$ 分别相乘,以最终可化简为\textcolor{Plum}{线性卷积}。},那 $\text{sinc} \left( x \right)$ 函数的各种近似呢?它们有无可能分解为 $\mathbb{e}$ 指数之和及 $\mathbb{e}$ 指数之积的排列组合?

%\marginLeft[-2.4em]{sec:NLCOV_chord}\section{\textcolor{Maroon}{Chord} 级数替换非线性卷积核 \textcolor{Maroon}{Sinc}}\label{sec:NLCOV_chord}

\marginLeft[-2.4em]{sec:NLAST_match}\section{\textcolor{Maroon}{Matched} 匹配型非线性角谱 \textcolor{Maroon}{NLAST}}\label{sec:NLAST_match}

\vspace*{-2.0em}

由于大多数\textcolor{Plum}{非线性}过程被当作\textcolor{gray}{频率转换}的工具,自然期望该过程的\textcolor{NavyBlue}{转换效率}较高;因此相比\textcolor{PineGreen}{失配}\textcolor{Plum}{解},人们更关注\textcolor{NavyBlue}{波矢}/\textcolor{PineGreen}{相位匹配}情况下的\textcolor{Plum}{解析解}。

在得到该\textcolor{Plum}{解析解}(见 \bref{sec:NLAST_match})之前,需要先找到一些数学\textcolor{Plum}{函数}或\textcolor{Plum}{级数},以在\textcolor{Plum}{有限区间}内\textcolor{Plum}{无限精度}地近似 Sinc 函数。

\marginLeft[-2.4em]{ssec:NLAST_sinc_series}\subsection{Sinc 函数的 Maclaurin 和 Padé 级数}\label{ssec:NLAST_sinc_series}

\begin{figure}[htbp!]
	\centering
	\renewcommand{\mypath}{\Desktop/article_fig/phd_thesis_fig/chapter-05/fig:sinc_approx.pdf}
	\includegraphics[width=0.6\textwidth]{"\mypath"}
	\biackcaption[\textbf{Comparison of sinc($x$) approximations.}]{-0.7em}{\textbf{sinc($x$) 对比其近似\textcolor{Plum}{函数}们。}}{fig:sinc_approx}
\end{figure}

存在一些函数,在\textcolor{Plum}{原点}的\textcolor{Plum}{邻域}上,近似 $\text{sinc} \left( x \right)$ 函数。

它们是:{\one} $\cos \left( {\dfrac{x}{{\sqrt 3 }}} \right)$ {\two} $\exp \left( { - \dfrac{x^2}{6}} \right)$ {\three} $\dfrac{{1 - \frac{7}{{60}}{x^2}}}{{1 + \frac{3}{{60}}{x^2}}}$ {\four} $\mathbb{e}^{3.2\left(\sqrt{1-{\left( \dfrac{x}{\symup{\pi}} \right)}^2}-1\right)}$ {\five} $\dfrac{1}{2} \left[ \mathbb{e}^{7.2\left(\sqrt{1-{\left( \dfrac{x}{\symup{\pi}} \right)}^2}-1\right)} + \mathbb{e}^{7.2\left(\sqrt[4]{1-{\left( \dfrac{x}{\symup{\pi}} \right)}^4}-1\right)} \right]
$。这些函数的对比,见\bref{fig:sinc_approx,tab:taylor-sinc_alternatives}。

可以看出,\textcolor{Plum}{余弦} $\cos \left( {\dfrac{x}{{\sqrt 3 }}} \right)$ 和 \textcolor{Plum}{ES 核}\cite{barnettParallelNonuniformFast2019,barnettAliasingErrorExpvsqrt1z22020} $\mathbb{e}^{3.2\left(\sqrt{1-{\left( \dfrac{x}{\symup{\pi}} \right)}^2}-1\right)}$ 从上方逼近 $\text{sinc} \left( x \right)$。\textcolor{Plum}{高斯} $\exp \left( { - \dfrac{x^2}{6}} \right)$ 和 \textcolor{Plum}{Padé 分式} $\dfrac{{1 - \frac{7}{{60}}{x^2}}}{{1 + \frac{3}{{60}}{x^2}}}$ 从下方逼近 $\text{sinc} \left( x \right)$。\textcolor{Plum}{加权 ES 核} $\dfrac{1}{2} \left[ \mathbb{e}^{7.2\left(\sqrt{1-{\left( \dfrac{x}{\symup{\pi}} \right)}^2}-1\right)} + \mathbb{e}^{7.2\left(\sqrt[4]{1-{\left( \dfrac{x}{\symup{\pi}} \right)}^4}-1\right)} \right]
$ 和 \textcolor{Plum}{Padé 分式} $\dfrac{{1 - \frac{7}{{60}}{x^2}}}{{1 + \frac{3}{{60}}{x^2}}}$ 几乎精确命中 $\text{sinc} \left( x \right)$ 的第一主峰/瓣。

然而,上述 5 种近似中,只有\textcolor{Plum}{余弦} $\cos \left( {\dfrac{x}{{\sqrt 3 }}} \right)$ (通过\textcolor{Plum}{欧拉公式}化为 $\mathbb{e}$ 指数之和)和\textcolor{Plum}{高斯} $\exp \left( { - \dfrac{x^2}{6}} \right)$(本身即 $\mathbb{e}$ 指数)可以化为\textcolor{Plum}{线性卷积}。但是,这两者的\textcolor{Plum}{精度}不够(特别是在 $\pm \symup{\pi}$ 以外),迫使我们找寻另一些在\textcolor{Plum}{原点}附近的\textcolor{Plum}{大范围}\textcolor{Plum}{邻域}上,均\textcolor{Plum}{高精度}接近 $\text{sinc} \left( x \right)$ 函数的新\textcolor{Plum}{函数}们,甚至\textcolor{Plum}{级数}们。
\begin{table}[htbp]
	\centering
	% 增加行高,使分数显示更舒展
	\renewcommand{\arraystretch}{2.2}
	\biackcaption[\textbf{Maclaurin expansion coefficients of sinc(x) alternatives.}]{-0.7em}{\textbf{近似 sinc(x) 的一些函数的\textcolor{Maroon}{麦克劳林}级数:$f(x) = c_0 + c_2 x^2 + c_4 x^4 + c_6 x^6 + O(x^8)$}}{tab:taylor-sinc_alternatives}
	\label{tab:coefficients}
	\begin{tabular}{lcccc}
		\toprule
		\textbf{Function} $f(x)$ & $\boldsymbol{c_0}$ & $\boldsymbol{c_2}$ & $\boldsymbol{c_4}$ & $\boldsymbol{c_6}$ \\
		\midrule
		$\text{sinc}(x) = \dfrac{\sin(x)}{x}$ 
		& $1$ & $-\dfrac{1}{6}$ & $\dfrac{1}{120}$ & $-\dfrac{1}{5040}$ \\
		
		$\cos\left(\dfrac{x}{\sqrt{3}}\right)$ 
		& $1$ & $-\dfrac{1}{6}$ & $\dfrac{1}{216}$ & $-\dfrac{1}{19440}$ \\
		
		$\exp\left(-\dfrac{x^2}{6}\right)$
		& $1$ & $-\dfrac{1}{6}$ & $\dfrac{1}{72}$ & $-\dfrac{1}{1296}$ \\
		
		$\left( 1-\dfrac{7}{60}x^2 \right) \big/ \left( 1+\dfrac{3}{60}x^2 \right)$
		& $1$ & $-\dfrac{1}{6}$ & $\dfrac{1}{120}$ & $-\dfrac{1}{2400}$ \\
		
		$\exp\left[3.2\left(\sqrt{1-\left( \dfrac{x}{\symup{\pi}} \right)^2}-1\right)\right]$
		& $1$ & $-\dfrac{8}{5\symup{\pi}^2}$ & $\dfrac{22}{25\symup{\pi}^4}$ & $-\dfrac{91}{375\symup{\pi}^6}$ \\
		
		% 使用 aligned 环境处理长公式换行,保持左对齐
		$\begin{aligned}
			&\dfrac{1}{2}\exp\left[7.2\left(\sqrt{1-\left( \dfrac{x}{\symup{\pi}} \right)^2}-1\right)\right] \\ 
			+&\dfrac{1}{2}\exp\left[7.2\left(\sqrt[4]{1-\left( \dfrac{x}{\symup{\pi}} \right)^4}-1\right)\right]
		\end{aligned}$
		& $1$ & $-\dfrac{9}{5\symup{\pi}^2}$ & $\dfrac{189}{100\symup{\pi}^4}$ & $-\dfrac{2493}{1000\symup{\pi}^6}$ \\
		\bottomrule
	\end{tabular}
\end{table}

上\bref{tab:taylor-sinc_alternatives} 展示了这 5 个函数的\textcolor{Maroon}{麦克劳林}级数。几乎否定了所有截断的\textcolor{Maroon}{麦克劳林}(\textcolor{Plum}{原点}处的\textcolor{Maroon}{泰勒})和\textcolor{Maroon}{帕德}(\textcolor{Maroon}{Padé})级数,作为可分解的 $\text{sinc} \left( x \right)$ 函数在\textcolor{Plum}{原点}附近大面积(可拓展地)\textcolor{Plum}{高精度}近似的可能。还能有什么其他的\textcolor{Plum}{函数}或\textcolor{Plum}{级数}虚位以待?有的。

\marginLeft[-2.4em]{ssec:NLAST_sinc_expand}\subsection{Sinc 函数的 Viète 无穷乘积 展开}\label{ssec:NLAST_sinc_expand}

\begin{figure}[htbp!]
	\centering
	\renewcommand{\mypath}{\Desktop/article_fig/phd_thesis_fig/chapter-05/fig:sinc_series.pdf}
	\includegraphics[width=0.6\textwidth]{"\mypath"}
	\biackcaption[\textbf{Comparison of sinc($x$) series.}]{-0.7em}{\textbf{sinc($x$) 对比其近似\textcolor{Plum}{级数}们。}}{fig:sinc_series}
\end{figure}

存在一些在较大范围区间内都接近,且该区间可进一步延长的函数,在\textcolor{Plum}{原点}的\textcolor{Plum}{大面积}\textcolor{Plum}{邻域}上,近似 $\text{sinc} \left( x \right)$ 函数。它们是:{\one} \textcolor{Maroon}{KB 窗/核}(\textcolor{Maroon}{Kaiser–Bessel Window/Kernel}) \cite{barnettParallelNonuniformFast2019}KBW$\left( 3.1, x, \symup{\pi} \right) := \frac{I_0 \left( 3.1 \sqrt{1 - \left( \dfrac{x}{\symup{\pi}} \right)^2} \right)}{I_0 \left( 3.1 \right)}$ {\two} $\text{sinc} \left( x \right)$ 的\textcolor{Plum}{无穷乘积}(而不是\textcolor{Plum}{求和})展开:$\text{sinc} \left( x \right) = \text{sinc} \left(\dfrac{x}{2^{\textcolor{NavyBlue}{J}}}\right) \prod\limits_{\textcolor{NavyBlue}{j}=\textcolor{NavyBlue}{1}}^{\textcolor{NavyBlue}{J}} \cos\left(\dfrac{x}{2^{\textcolor{NavyBlue}{j}}}\right) = \prod\limits_{\textcolor{NavyBlue}{j}=\textcolor{NavyBlue}{1}}^{{\textcolor{NavyBlue}{\infty}}} \cos \left( \dfrac{x}{2^{\textcolor{NavyBlue}{j}}} \right)
$ {\three} $\text{sinc} \left( x \right)$ 的\textcolor{Plum}{分数阶}\textcolor{Plum}{非正交余弦基}\textcolor{Maroon}{和弦级数}展开:$\text{sinc}\left( x \right) = \sum\limits^{\textcolor{NavyBlue}{\infty}}_{\textcolor{NavyBlue}{j}=\textcolor{NavyBlue}{1}} {{a_{\textcolor{NavyBlue}{j}}}\cos \left( \displaystyle{ {\dfrac{x}{{{b_{\textcolor{NavyBlue}{j}}}}}} } \right)}$。

可以看出,\textcolor{Maroon}{KB 窗/核} 本身计算就较复杂,且不可分解为 $\mathbb{e}$ 指数之和或 $\mathbb{e}$ 指数之积,因此暂不考虑使用它来近似 $\text{sinc} \left( x \right)$ 函数。以至于,目前只剩下后 2 个选项:\textcolor{Plum}{无穷乘积}、\textcolor{Maroon}{无穷级数}。

对 $\text{sinc} \left( x \right)$ 的 \textcolor{Plum}{连乘}(\textcolor{Maroon}{Viète})\textcolor{Plum}{级数} 和 \textcolor{Maroon}{和弦级数} 截断(取有限项),结果分别为:$\prod\limits_{\textcolor{NavyBlue}{j}=\textcolor{NavyBlue}{1}}^{\textcolor{NavyBlue}{J}} \cos \left( \dfrac{x}{2^{\textcolor{NavyBlue}{j}}} \right)$ 以及 $a_{\textcolor{NavyBlue}{0}} + \sum\limits^{\textcolor{NavyBlue}{J}-1}_{\textcolor{NavyBlue}{j}=\textcolor{NavyBlue}{1}} {{a_{\textcolor{NavyBlue}{j}}}\cos \left( \displaystyle{ {\dfrac{x}{{{b_{\textcolor{NavyBlue}{j}}}}}} } \right)}$,前者在 $2^{\textcolor{NavyBlue}{J}}$ 个 $\mathbb{e}$ 指数项的求和下,在 $\left| x \right| \leq \left( 2^{\textcolor{NavyBlue}{J}-1}-1 \right) \symup{\pi}$ 区间内逼近 $\text{sinc} \left( x \right)$,后者在 $2 {\textcolor{NavyBlue}{J}}$ 个 $\mathbb{e}$ 指数项的求和下,在 $\left| x \right| \leq {\textcolor{NavyBlue}{J}} \symup{\pi}$ 区间内(见)逼近 $\text{sinc} \left( x \right)$。

可以看出,为准确逼近 $\text{sinc} \left( x \right)$,\textcolor{Maroon}{和弦级数} $a_{\textcolor{NavyBlue}{0}} + \sum\limits^{\textcolor{NavyBlue}{J}-1}_{\textcolor{NavyBlue}{j}=\textcolor{NavyBlue}{1}} {{a_{\textcolor{NavyBlue}{j}}}\cos \left( \displaystyle{ {\dfrac{x}{{{b_{\textcolor{NavyBlue}{j}}}}}} } \right)}$ 每新增 2 个 $\mathbb{e}$ 指数项,\textcolor{Plum}{适用范围}就单边扩大 $\symup{\pi}$,效率为 $\dfrac{{\textcolor{NavyBlue}{J}}\symup{\pi}}{2{\textcolor{NavyBlue}{J}}} = \dfrac{\symup{\pi}}{2}$;而\textcolor{Plum}{连乘级数} $\prod\limits_{\textcolor{NavyBlue}{j}=\textcolor{NavyBlue}{1}}^{\textcolor{NavyBlue}{J}} \cos \left( \dfrac{x}{2^{\textcolor{NavyBlue}{j}}} \right)$ 的效率为 $\dfrac{\left( 2^{\textcolor{NavyBlue}{J}-1}-1 \right) \symup{\pi}}{2^{\textcolor{NavyBlue}{J}}} = \dfrac{\symup{\pi}}{2}-\dfrac{\symup{\pi}}{2^{\textcolor{NavyBlue}{J}}} < \dfrac{\symup{\pi}}{2}$,\textcolor{Plum}{单位成本}一直高于\textcolor{Maroon}{和弦级数}的 $\dfrac{2}{\symup{\pi}}$(总有额外的开销)。这意味着,对于相同的 \textcolor{Maroon}{approaching} 范围,\textcolor{Maroon}{和弦级数} 总是比 \textcolor{Plum}{连乘级数} 更节省算力。

此外,\textcolor{Maroon}{和弦级数}允许\textcolor{Plum}{线性步进},而\textcolor{Plum}{连乘级数}只允许\textcolor{Plum}{倍增步进},即后者的区间只能按 $2$ 的\textcolor{Plum}{幂级数}阶梯式跳跃,这可能会造成严重的计算资源浪费。

总的来说,对比 2 个核心指标后,发现无论是从“单位区间的计算成本”,还是从“颗粒度与可控性”上,\textcolor{Maroon}{和弦级数}的绝对效率、工程适应性和(可线性)扩展性即灵活度,都比\textcolor{Plum}{连乘级数}更强。

因此,\textcolor{PineGreen}{匹配型}\textcolor{Maroon}{非线性角谱}将使用 \textcolor{Maroon}{和弦级数} $\sum\limits^{\textcolor{NavyBlue}{\infty}}_{\textcolor{NavyBlue}{j}=\textcolor{NavyBlue}{1}} {{a_{\textcolor{NavyBlue}{j}}}\cos \left( \displaystyle{ {\dfrac{x}{{{b_{\textcolor{NavyBlue}{j}}}}}} } \right)}$ 来近似 $\text{sinc}\left( x \right)$ 卷积核。\textcolor{Maroon}{和弦级数}将在下一小节即 \bref{ssec:NLAST_sinc_chord} 中,得到更进一步的介绍。

\vspace*{-2.5em}

\marginLeft[-2.4em]{ssec:NLAST_sinc_chord}\subsection{Sinc 函数的 Chord 和弦级数 展开}\label{ssec:NLAST_sinc_chord}

数学上,总存在一共\textcolor{Plum}{正整数} ${\textcolor{NavyBlue}{J}} \in \mathbb{N}^+$ \textcolor{Plum}{对}合适的\textcolor{Plum}{系数对} $a_{\textcolor{NavyBlue}{j}}, b_{\textcolor{NavyBlue}{j}}$ 所构成的\textcolor{Plum}{集合} $\left\{ a_{\textcolor{NavyBlue}{j}}, b_{\textcolor{NavyBlue}{j}} \right\}$,加上\textcolor{Plum}{直流偏置} $a_{\textcolor{NavyBlue}{0}}$,一共 $2{\textcolor{NavyBlue}{J}} + 1$ 个\textcolor{Plum}{待定系数},使得可以将 $\text{sinc} \left( x \right), \text{since} \left( x \right)$ 函数在该\textcolor{Plum}{自变量范围}内
\begin{equation} \label{eq:sinc(e)-chord-x_interval}
	\left| x \right| \leq \left( {\textcolor{NavyBlue}{J}}+1 \right) \symup{\pi} ~,
\end{equation}
展开为“\textcolor{Plum}{分数阶}\textcolor{Plum}{非正交余弦基}\textcolor{Maroon}{和弦级数}”\Footnote{注意,隐式使用了\textcolor{Maroon}{爱因斯坦求和约定},以遍历角/哑标 $\textcolor{NavyBlue}{j}$,甚至遍历 $\pm$,并求和。}:
\begin{subequations} \label{eq:sinc(e)-chord}
	\begin{align}
		\hspace{-1em} \text{sinc}\left( x \right) = a_{\textcolor{NavyBlue}{0}} + \sum\limits^{\textcolor{NavyBlue}{J}}_{\textcolor{NavyBlue}{j}=\textcolor{NavyBlue}{1}} {{a_{\textcolor{NavyBlue}{j}}}\cos \left( \displaystyle{ {\frac{x}{{{b_{\textcolor{NavyBlue}{j}}}}}} } \right)} &= a_{\textcolor{NavyBlue}{0}} + \frac{ a^{\textcolor{NavyBlue}{j}} }{ 2 } \left( \mathbb{e}^{\mathbb{i}{\frac{x}{b_{\textcolor{NavyBlue}{j}}}}} + \mathbb{e}^{-\mathbb{i}{\frac{x}{b_{\textcolor{NavyBlue}{j}}}}} \right) = a_{\textcolor{NavyBlue}{0}} + \frac{ a^{\textcolor{NavyBlue}{j}} }{ 2 } \mathbb{e}^{\pm \mathbb{i} {\frac{x}{b_{\textcolor{NavyBlue}{j}}}}} \label{eq:sinc-chord}~, \\ \hspace{-1em} \text{since}\left( x \right) \xrightarrow[]{\text{\bref{eq:up-scalar-g-EE-312-discrete-since3}}} \text{sinc}\left( x \right) \mathbb{e}^{\mathbb{i}{x}} &= \left( a_{\textcolor{NavyBlue}{0}} + \frac{ a^{\textcolor{NavyBlue}{j}} }{ 2 } \mathbb{e}^{\pm \mathbb{i} {\frac{x}{b_{\textcolor{NavyBlue}{j}}}}} \right) \mathbb{e}^{\mathbb{i}{x}} = a_{\textcolor{NavyBlue}{0}} \mathbb{e}^{\mathbb{i}{x}} + \frac{ a^{\textcolor{NavyBlue}{j}} }{ 2 } \mathbb{e}^{\mathbb{i} {\frac{b_{\textcolor{NavyBlue}{j}} \pm 1}{b_{\textcolor{NavyBlue}{j}}}} x} \label{eq:since-chord}~,
	\end{align}
\end{subequations}
其中,\textcolor{Plum}{直流偏置}/背景 $a_{\textcolor{NavyBlue}{0}}$ 与\textcolor{Plum}{交流系数对} $a_{\textcolor{NavyBlue}{j}}, b_{\textcolor{NavyBlue}{j}}$ 们,满足如下关系:
\begin{subequations} \label{eq:chord:abj-relations}
	\begin{align}
		&a_{\textcolor{NavyBlue}{0}} + \sum\limits^{\textcolor{NavyBlue}{J}}_{\textcolor{NavyBlue}{j}=\textcolor{NavyBlue}{1}} a_{\textcolor{NavyBlue}{j}} = \sum\limits^{\textcolor{NavyBlue}{J}}_{\textcolor{NavyBlue}{j}=\textcolor{NavyBlue}{0}} a_{\textcolor{NavyBlue}{j}} = 1 \label{eq:chord:abj-relations-a}~, \\
		&\begin{cases}
			a_{\textcolor{NavyBlue}{0}} = 0 ~, &\quad \text{for} \ \ \ {\textcolor{NavyBlue}{J}} = \textcolor{NavyBlue}{1} \\
			a_{\textcolor{NavyBlue}{0}} \neq 0 ~, &\quad \text{for} \ \ \ {\textcolor{NavyBlue}{J}} \geq \textcolor{NavyBlue}{2} \\
		\end{cases} \label{eq:chord:abj-relations-b}~, \\
		&\begin{cases}
			a_{\textcolor{NavyBlue}{j_2}} > a_{\textcolor{NavyBlue}{j_1}} ~, &\quad \text{for} \ \ \ \textcolor{NavyBlue}{j_2} > \textcolor{NavyBlue}{j_1} \\
			b_{\textcolor{NavyBlue}{j_2}} > b_{\textcolor{NavyBlue}{j_1}} ~, &\quad \text{for} \ \ \ \textcolor{NavyBlue}{j_2} > \textcolor{NavyBlue}{j_1} \\
		\end{cases} \label{eq:chord:abj-relations-c}~, \\
		&\begin{cases}
			a_{\textcolor{NavyBlue}{j}} < 10^{-4} ~, &\quad \text{for} \ \ \ {\textcolor{NavyBlue}{J}} \geq \textcolor{NavyBlue}{5} \ \ \ \text{and} \ \ \ \textcolor{NavyBlue}{1} \leq \textcolor{NavyBlue}{j} 	\leq \text{\textcolor{Plum}{discard}}({\textcolor{NavyBlue}{J}}) \\
			b_{\textcolor{NavyBlue}{j}} > 1 ~, &\quad \text{for} \ \ \ a_{\textcolor{NavyBlue}{j}} \geq 10^{-4} \\
		\end{cases} \label{eq:chord:abj-relations-d}~,
	\end{align}
\end{subequations}
由于有截断的\textcolor{Plum}{离散}多个 $b_{\textcolor{NavyBlue}{j}} > 1$,以及它们在\textcolor{Plum}{分母}上这一点,具有“\textcolor{NavyBlue}{波长}”或“\textcolor{NavyBlue}{弦长}”的性质,因此也称该级数 \bref{eq:sinc-chord} 为“\textcolor{Maroon}{和弦级数}”;此外,求和阶数 ${\textcolor{NavyBlue}{J}}$ 对应的,包含\textcolor{Plum}{欧拉分解}后的 $\mathbb{e}$ 指数项和 $a_{\textcolor{NavyBlue}{0}}$ 项的,总的有效($ a_{\textcolor{NavyBlue}{j}} \geq 10^{-4} $)求和项数 $N$ 为:
\begin{equation} \label{eq:chord-N=2J+2_effective}
	N_{\textcolor{NavyBlue}{J}} := 2 \left[ {\textcolor{NavyBlue}{J}} - \text{\textcolor{Plum}{discard}} \left( {\textcolor{NavyBlue}{J}} \right) + \text{\textcolor{Plum}{int}} \left( {\textcolor{NavyBlue}{J}} \geq 2 \right) \right] \in \left( {\textcolor{NavyBlue}{J}}, 2 {\textcolor{NavyBlue}{J}} + 2 \right] ~,
\end{equation}
其中,\textcolor{Plum}{discard} 表示“因幅值 $a_{\textcolor{NavyBlue}{j}}$ 太小而可忽略(的\textcolor{Plum}{余弦} $\cos$ 项)”;$\text{\textcolor{Plum}{discard}}({\textcolor{NavyBlue}{J}})$ 表示共 ${\textcolor{NavyBlue}{J}}$ 项对 $\cos$ 的求和中,这样的、可忽略的\textcolor{Plum}{余弦} $\cos$ 项的个数;并且,$\text{\textcolor{Plum}{int}}$ 表示将\textcolor{Plum}{布尔值}转化为\textcolor{Plum}{整型}。

指定\textcolor{gray}{横向阶数} ${\textcolor{NavyBlue}{J}} = \textcolor{NavyBlue}{4}$ 的\textcolor{Plum}{分数阶}\textcolor{Plum}{非正交余弦基}\textcolor{Maroon}{和弦级数},如下图 \ref{fig:sinc+chord_spectrum} 所示:
\begin{figure}[htbp!]
	\centering
	\renewcommand{\mypath}{\Desktop/article_fig/phd_thesis_fig/chapter-05/fig:chord-3.1.png}
	\includegraphics[width=1\textwidth]{"\mypath"}
	\biackcaption[\textbf{$\text{sinc}$ function, the chord series with ${\textcolor{NavyBlue}{J}} = \textcolor{NavyBlue}{4}$, and its cosine spectrum.}]{-0.7em}{\textbf{目标 $\text{sinc}$ 函数、${\textcolor{NavyBlue}{J}} = \textcolor{NavyBlue}{4}$ 的\textcolor{Maroon}{和弦级数},及其\textcolor{Plum}{余弦谱}}}{fig:sinc+chord_spectrum}
\end{figure}

${\textcolor{NavyBlue}{J}} = \textcolor{NavyBlue}{4}$ 及其他\textcolor{gray}{阶数} $\textcolor{NavyBlue}{J}$ 的\textcolor{Maroon}{和弦级数}的 $a_{\textcolor{NavyBlue}{0}}, \left\{ a_{\textcolor{NavyBlue}{j}}, b_{\textcolor{NavyBlue}{j}} \right\}$ \textcolor{Plum}{系}/\textcolor{Plum}{参数表},见 \bref{tab:chord-abj}。不难看出,\bref{fig:sinc+chord_spectrum,tab:chord-abj} 同时满足 \bref{eq:chord:abj-relations,eq:chord-N=2J+2_effective} 的\textcolor{Plum}{约束}/“\textcolor{Plum}{线性规划}”。
\begin{table}[htbp]
	\centering
	\biackcaption[\textbf{$a_{\textcolor{NavyBlue}{0}}$
		and $\left\{ a_{\textcolor{NavyBlue}{j}}, b_{\textcolor{NavyBlue}{j}} \right\}$ for the chord series with $\textcolor{NavyBlue}{J} \in \left[ \textcolor{NavyBlue}{1}, \textcolor{NavyBlue}{10} \right]$.}]{-0.7em}{\textbf{$\textcolor{NavyBlue}{J} \in \left[ \textcolor{NavyBlue}{1}, \textcolor{NavyBlue}{10} \right]$ 的\textcolor{Maroon}{和弦级数}的 $a_{\textcolor{NavyBlue}{0}}, \left\{ a_{\textcolor{NavyBlue}{j}}, b_{\textcolor{NavyBlue}{j}} \right\}$ \textcolor{Plum}{参数表}}}{tab:chord-abj}
	\resizebox{1.0\linewidth}{!}{  % 宽度不超文本宽度
		\begin{tabular}{c|cccccccccc}
			\toprule[2pt]
			$\textcolor{NavyBlue}{J}$  & \textcolor{NavyBlue}{1}      & \textcolor{NavyBlue}{2}      & \textcolor{NavyBlue}{3}      & \textcolor{NavyBlue}{4}      & \textcolor{NavyBlue}{5}      & \textcolor{NavyBlue}{6}      & \textcolor{NavyBlue}{7}      & \textcolor{NavyBlue}{8}      & \textcolor{NavyBlue}{9}      & \textcolor{NavyBlue}{10}     \\ \midrule[1.2pt]
			$a_{\textcolor{NavyBlue}{0}}$  & 0 & 0.2436 & 0.1819 & 0.1466 & 0.1349 & 0.1137 & 0.0970 & 0.0921 & 0.0817 & 0.0798 \\ \midrule
			$a_{\textcolor{NavyBlue}{1}}$  & 1 & 0.3010 & 0.1695 & 0.1021 & $\cancel{< 10^{-4}}$      & $\cancel{< 10^{-4}}$      & $\cancel{< 10^{-4}}$      & $\cancel{< 10^{-4}}$      & $\cancel{< 10^{-4}}$      & $\cancel{< 10^{-4}}$      \\
			$a_{\textcolor{NavyBlue}{2}}$  &        & 0.4554 & 0.2984 & 0.2048 & 0.1300 & 0.0866 & 0.0648 & $\cancel{< 10^{-4}}$      & $\cancel{< 10^{-4}}$      & $\cancel{< 10^{-4}}$      \\
			$a_{\textcolor{NavyBlue}{3}}$  &        &        & 0.3503 & 0.2606 & 0.2172 & 0.1608 & 0.1245 & 0.0764 & 0.0572 & $\cancel{< 10^{-4}}$      \\
			$a_{\textcolor{NavyBlue}{4}}$  &        &        &        & 0.2859 & 0.2519 & 0.1979 & 0.1578 & 0.1353 & 0.1058 & 0.0550 \\
			$a_{\textcolor{NavyBlue}{5}}$  &        &        &        &        & 0.2660 & 0.2162 & 0.1765 & 0.1604 & 0.1318 & 0.1107 \\
			$a_{\textcolor{NavyBlue}{6}}$  &        &        &        &        &        & 0.2248 & 0.1870 & 0.1730 & 0.1463 & 0.1361 \\
			$a_{\textcolor{NavyBlue}{7}}$  &        &        &        &        &        &        & 0.1924 & 0.1797 & 0.1549 & 0.1479 \\
			$a_{\textcolor{NavyBlue}{8}}$  &        &        &        &        &        &        &        & 0.1831 & 0.1599 & 0.1541 \\
			$a_{\textcolor{NavyBlue}{9}}$  &        &        &        &        &        &        &        &        & 0.1625 & 0.1574 \\
			$a_{\textcolor{NavyBlue}{10}}$ &        &        &        &        &        &        &        &        &        & 0.1591 \\ \midrule[1.2pt]
			$b_{\textcolor{NavyBlue}{1}}$  & 1.7321 & 1.1486 & 1.0745 & 1.0422 & $\cancel{0.6083}$ & $\cancel{0.5747}$ & $\cancel{0.3598}$ & $\cancel{0.1991}$ & $\cancel{0.1242}$ & $\cancel{0.0299}$ \\
			$b_{\textcolor{NavyBlue}{2}}$  &        & 2.0964 & 1.4534 & 1.2470 & 1.0572 & 1.0363 & 1.0267 & $\cancel{0.7408}$ & $\cancel{0.3171}$ & $\cancel{0.3042}$ \\
			$b_{\textcolor{NavyBlue}{3}}$  &        &        & 2.7821 & 1.7670 & 1.3061 & 1.1943 & 1.1408 & 1.0315 & 1.0237 & $\cancel{0.5960}$ \\
			$b_{\textcolor{NavyBlue}{4}}$  &        &        &        & 3.4388 & 1.8920 & 1.5249 & 1.3628 & 1.1643 & 1.1201 & 1.0217 \\
			$b_{\textcolor{NavyBlue}{5}}$  &        &        &        &        & 3.7238 & 2.2339 & 1.7676 & 1.4094 & 1.2942 & 1.1201 \\
			$b_{\textcolor{NavyBlue}{6}}$  &        &        &        &        &        & 4.4146 & 2.6077 & 1.8443 & 1.5799 & 1.3027 \\
			$b_{\textcolor{NavyBlue}{7}}$  &        &        &        &        &        &        & 5.1690 & 2.7360 & 2.0748 & 1.6001 \\
			$b_{\textcolor{NavyBlue}{8}}$  &        &        &        &        &        &        &        & 5.4395 & 3.0829 & 2.1112 \\
			$b_{\textcolor{NavyBlue}{9}}$  &        &        &        &        &        &        &        &        & 6.1337 & 3.1469 \\
			$b_{\textcolor{NavyBlue}{10}}$ &        &        &        &        &        &        &        &        &        & 6.2726 \\ \bottomrule[2pt]
		\end{tabular}
	}
\end{table}

接下来将用上述 \textcolor{Maroon}{chord 级数} 替换各式(即 \bref{eq:up-scalar-g-EE-312-discrete-since2,eq:down-scalar-g-EE-132-discrete-Since,eq:Born_approx-scalar-g-E-12-since5})中的 $\text{sinc} \left( x \right)$ \textcolor{Plum}{卷积核},并对\textcolor{Plum}{非线性} \textcolor{Maroon}{Chord 级数} \textcolor{Plum}{卷积核} 进行 \textcolor{Plum}{线性化},以解析\textcolor{Plum}{非线性}光学\textcolor{NavyBlue}{动力学过程}底层运行的,大量此类\textcolor{Plum}{非线性卷积}积分。

\marginLeft[-2.4em]{ssec:NLAST_chord}\subsection{非线性 Chord 级数 卷积核 的 线性化}\label{ssec:NLAST_chord}

即使有了 \textcolor{Maroon}{Chord 级数} 来作为 \textcolor{Plum}{非线性} \textcolor{Plum}{卷积核},整个\textcolor{Plum}{卷积积分}仍然是\textcolor{Plum}{非线性}的。但 \textcolor{Maroon}{Chord 级数} 的 \textcolor{Plum}{可欧拉分解} 的 \textcolor{Plum}{数学特征},使得对 \textcolor{Maroon}{Chord 级数} \textcolor{Plum}{卷积核}的\textcolor{Plum}{线性化}非常容易。接下来,以三种代表性\textcolor{Plum}{非线性}过程为例,将\textcolor{Plum}{卷积核}替换为了\textcolor{Maroon}{Chord 级数}的\textcolor{Plum}{非线性卷积}积分转换为\textcolor{Plum}{线性卷积},并最终落脚于\textcolor{Plum}{线性傅立叶变换}。

\marginLeft[-2.4em]{sssec:NLAST_match_SFG}\subsubsection{SFG 和频过程的匹配型 NLAST}\label{sssec:NLAST_match_SFG}

一方面,将 \bref{eq:since-chord} 的\textcolor{Plum}{级数}/\textcolor{NavyBlue}{交流}/\textcolor{Plum}{余弦}部分 $\dfrac{a^{\textcolor{NavyBlue}{j}}}{2} \cdot \mathbb{e}^{\mathbb{i} {\frac{b_{\textcolor{NavyBlue}{j}} \pm 1}{b_{\textcolor{NavyBlue}{j}}}} x}$,代入\textcolor{Maroon}{和频}标量\textcolor{Maroon}{时空谱}耦合波方程的\textcolor{Plum}{非线性卷积解} \bref{eq:up-scalar-g-EE-312-discrete-since2},得到 $\xint{\begin{smallmatrix} ~ \\ {}^{}_{\mathcolor{gray}{-}} \\ ~ \end{smallmatrix}}{09}{\mathtt{g}}^{\;\! \textcolor{PineGreen}{\hat{3}}}_{\;\! \mathcolor{gray}{z}}$ 的\textcolor{NavyBlue}{交流项} $\xint{\begin{smallmatrix} ~ \\ {}^{}_{\mathcolor{gray}{-}} \\ ~ \end{smallmatrix}}{09}{\mathtt{g}}^{\;\! \textcolor{PineGreen}{\hat{3}}}_{\;\! \mathcolor{gray}{z};\text{\textcolor{NavyBlue}{AC}}}$:
\begin{subequations} \label{eq:up-scalar-g-EE-312-discrete-AC}
	\begin{align}
		&\hspace*{-1.6em} \xint{\begin{smallmatrix} ~ \\ {}^{}_{\mathcolor{gray}{-}} \\ ~ \end{smallmatrix}}{09}{\mathtt{g}}^{\;\! \textcolor{PineGreen}{\hat{3}}}_{\;\! \mathcolor{gray}{z}} = \Upsilon^{\;\! \hat{3} \textcolor{PineGreen}{\hat{3}} \hat{1} \hat{2} }_{\;\! \textcolor{Maroon}{(2)} \textcolor{PineGreen}{\hat{1} \hat{2}} } \mathcolor{gray}{\iiint} \xint{\mathcolor{gray}{-}}{18}{M}^{\;\! \mathcolor{gray}{3} \hat{1} \hat{2} }_{\;\! \hat{3} \textcolor{Maroon}{(2)} \mathcolor{gray}{\bar{q}} } \mathcolor{gray}{\iint} \xint{{}^{}_{\mathcolor{gray}{-}}}{10}{g}^{\;\! \textcolor{PineGreen}{\hat{1}}}_{\;\! \hat{1} \mathcolor{gray}{0}} \xint{{}^{}_{\mathcolor{gray}{-}}}{10}{g}^{\;\! \textcolor{PineGreen}{\hat{2}}}_{\;\! \hat{2} \mathcolor{gray}{0}} \cdot \text{since} \left( \frac{ \Delta \xint{\begin{smallmatrix} ~ \\ {}^{}_{\mathcolor{gray}{-}} \\ ~ \end{smallmatrix}}{15}{k}_{\symup{z}}^{\;\! \textcolor{PineGreen}{\hat{1} \hat{2} \hat{3}} } \mathcolor{gray}{z} }{ 2 } \right) \mathcolor{gray}{z} ~ \mathbb{d} \mathcolor{gray}{\bar{k}_{1\symup{\rho}}} \mathbb{d} \mathcolor{gray}{\bar{q}} ~, \label{eq:up-scalar-g-EE-312-discrete-SINCE} \\ 
		&\hspace*{-1.6em}  \xrightarrow[ \xint{{}^{}_{\mathcolor{gray}{-}}}{10}{g}^{\;\! \textcolor{PineGreen}{\hat{\jmath}}}_{\;\! \hat{\jmath} \mathcolor{gray}{0}}, \xint{\begin{smallmatrix} ~ \\ {}^{}_{\mathcolor{gray}{-}} \\ ~ \end{smallmatrix}}{15}{k}_{\symup{z}}^{\;\! \textcolor{PineGreen}{\hat{\jmath}} } := \xint{{}^{}_{\mathcolor{gray}{-}}}{10}{g}^{\;\! \textcolor{PineGreen}{\hat{\jmath}}}_{\;\! \hat{\jmath} \mathcolor{gray}{0}} \left( \mathcolor{gray}{\bar{k}_{\jmath\symup{\rho}}} \right), \xint{\begin{smallmatrix} ~ \\ {}^{}_{\mathcolor{gray}{-}} \\ ~ \end{smallmatrix}}{15}{k}_{\symup{z}}^{\;\! \textcolor{PineGreen}{\hat{\jmath}} } \left( \mathcolor{gray}{\bar{k}_{\jmath\symup{\rho}}} \right) ]{\text{\textcolor{NavyBlue}{AC}~part of~\bref{eq:since-chord}}} \Upsilon^{\;\! \hat{3} \textcolor{PineGreen}{\hat{3}} \hat{1} \hat{2} }_{\;\! \textcolor{Maroon}{(2)} \textcolor{PineGreen}{\hat{1} \hat{2}} } \dfrac{a^{\textcolor{NavyBlue}{j}}}{2} \mathcolor{gray}{z} \mathcolor{gray}{\iiint} \xint{\mathcolor{gray}{-}}{18}{M}^{\;\! \mathcolor{gray}{3} \hat{1} \hat{2} }_{\;\! \hat{3} \textcolor{Maroon}{(2)} \mathcolor{gray}{\bar{q}} } \mathcolor{gray}{\iint} \xint{{}^{}_{\mathcolor{gray}{-}}}{10}{g}^{\;\! \textcolor{PineGreen}{\hat{1}}}_{\;\! \hat{1} \mathcolor{gray}{0}} \xint{{}^{}_{\mathcolor{gray}{-}}}{10}{g}^{\;\! \textcolor{PineGreen}{\hat{2}}}_{\;\! \hat{2} \mathcolor{gray}{0}} \cdot \mathbb{e}^{\mathbb{i} \frac{b_{\textcolor{NavyBlue}{j}} \pm 1}{b_{\textcolor{NavyBlue}{j}}} \frac{ \Delta \xint{\begin{smallmatrix} ~ \\ {}^{}_{\mathcolor{gray}{-}} \\ ~ \end{smallmatrix}}{15}{k}_{\symup{z}}^{\;\! \textcolor{PineGreen}{\hat{1} \hat{2} \hat{3}} } \mathcolor{gray}{z} }{ 2 }} ~ \mathbb{d} \mathcolor{gray}{\bar{k}_{1\symup{\rho}}} \mathbb{d} \mathcolor{gray}{\bar{q}} \label{eq:up-scalar-g-EE-312-discrete-chord-AC} \\ &\hspace*{-1.6em}  \xrightarrow[\text{\bref{eq:up-scalar-g-EE-312-discrete-convolution6}}]{\text{\bref{eq:up-scalar-g-EE-312-discrete-convolution5}}} \Upsilon^{\;\! \hat{3} \textcolor{PineGreen}{\hat{3}} \hat{1} \hat{2} \textcolor{NavyBlue}{j} }_{\;\! \textcolor{Maroon}{(2)} \textcolor{PineGreen}{\hat{1} \hat{2}} \mathcolor{gray}{z} } \mathcolor{gray}{\iiint} \xint{\mathcolor{gray}{-}}{18}{M}^{\;\! \mathcolor{gray}{3} \hat{1} \hat{2} }_{\;\! \hat{3} \textcolor{Maroon}{(2)} \mathcolor{gray}{\bar{q}} } \mathbb{e}^{\mathbb{i} \mathcolor{gray}{q_{\symup{z}}} \frac{b_{\textcolor{NavyBlue}{j}} \pm 1}{b_{\textcolor{NavyBlue}{j}}} \frac{ \mathcolor{gray}{z} }{ 2 }} \mathcolor{gray}{\iint} \xint{{}^{}_{\mathcolor{gray}{-}}}{10}{g}^{\;\! \textcolor{PineGreen}{\hat{1}}}_{\;\! \hat{1} \mathcolor{gray}{0}} \mathbb{e}^{\mathbb{i} \xint{\begin{smallmatrix} ~ \\ {}^{}_{\mathcolor{gray}{-}} \\ ~ \end{smallmatrix}}{15}{k}_{\symup{z}}^{\;\! \textcolor{PineGreen}{\hat{1}} } \frac{b_{\textcolor{NavyBlue}{j}} \pm 1}{b_{\textcolor{NavyBlue}{j}}} \frac{ \mathcolor{gray}{z} }{ 2 }} \xint{{}^{}_{\mathcolor{gray}{-}}}{10}{g}^{\;\! \textcolor{PineGreen}{\hat{2}}}_{\;\! \hat{2} \mathcolor{gray}{0}} \mathbb{e}^{\mathbb{i} \xint{\begin{smallmatrix} ~ \\ {}^{}_{\mathcolor{gray}{-}} \\ ~ \end{smallmatrix}}{15}{k}_{\symup{z}}^{\;\! \textcolor{PineGreen}{\hat{2}} } \frac{b_{\textcolor{NavyBlue}{j}} \pm 1}{b_{\textcolor{NavyBlue}{j}}} \frac{ \mathcolor{gray}{z} }{ 2 }} ~ \mathbb{d} \mathcolor{gray}{\bar{k}_{1\symup{\rho}}} \mathbb{d} \mathcolor{gray}{\bar{q}} \label{eq:up-scalar-g-EE-312-discrete-AC-Delta_k}
		\\ &\hspace*{-1.6em}  \xrightarrow[]{\text{\bref{eq:components-eigenwave}}} \Upsilon^{\;\! \hat{3} \textcolor{PineGreen}{\hat{3}} \hat{1} \hat{2} \textcolor{NavyBlue}{j} }_{\;\! \textcolor{Maroon}{(2)} \textcolor{PineGreen}{\hat{1} \hat{2}} \mathcolor{gray}{z} } \mathcolor{gray}{\int} \mathbb{e}^{\mathbb{i} \mathcolor{gray}{q_{\symup{z}}} \frac{b_{\textcolor{NavyBlue}{j}} \pm 1}{b_{\textcolor{NavyBlue}{j}}} \frac{ \mathcolor{gray}{z} }{ 2 }} ~\mathbb{d} \mathcolor{gray}{q_{\symup{z}}} \mathcolor{gray}{\iint} \xint{\mathcolor{gray}{-}}{18}{M}^{\;\! \mathcolor{gray}{3} \hat{1} \hat{2} }_{\;\! \hat{3} \textcolor{Maroon}{(2)} \mathcolor{gray}{\bar{q}} } \mathcolor{gray}{\iint} \xint{\mathcolor{gray}{-}}{30}{\bar{E}}^{\;\! \textcolor{PineGreen}{\hat{1}}}_{\;\! \hat{1}, \frac{b_{\textcolor{NavyBlue}{j}} \pm 1}{b_{\textcolor{NavyBlue}{j}}} \frac{ \mathcolor{gray}{z} }{ 2 }} \xint{\mathcolor{gray}{-}}{30}{\bar{E}}^{\;\! \textcolor{PineGreen}{\hat{2}}}_{\;\! \hat{2}, \frac{b_{\textcolor{NavyBlue}{j}} \pm 1}{b_{\textcolor{NavyBlue}{j}}} \frac{ \mathcolor{gray}{z} }{ 2 }}~ \mathbb{d} \mathcolor{gray}{\bar{k}_{1\symup{\rho}}} \mathbb{d} \mathcolor{gray}{\bar{q}_{\symup{\rho}}} \label{eq:up-scalar-g-EE-312-discrete-AC-G1G2}
		\\ &\hspace*{-1.6em}  
		\xrightarrow[\text{\textcolor{Plum}{convolution}}]{\text{linear}} \Upsilon^{\;\! \hat{3} \textcolor{PineGreen}{\hat{3}} \hat{1} \hat{2} \textcolor{NavyBlue}{j} }_{\;\! \textcolor{Maroon}{(2)} \textcolor{PineGreen}{\hat{1} \hat{2}} \mathcolor{gray}{z} } \mathcolor{gray}{\int}  \xint{\mathcolor{gray}{-}}{18}{M}^{\;\! \mathcolor{gray}{3} \hat{1} \hat{2} }_{\;\! \hat{3} \textcolor{Maroon}{(2)} \mathcolor{gray}{q_{\symup{z}}} } \mathcolor{gray}{*} \xint{\mathcolor{gray}{-}}{30}{\bar{E}}^{\;\! \textcolor{PineGreen}{\hat{1}}}_{\;\! \hat{1}, \frac{b_{\textcolor{NavyBlue}{j}} \pm 1}{b_{\textcolor{NavyBlue}{j}}} \frac{ \mathcolor{gray}{z} }{ 2 }} \mathcolor{gray}{*} \xint{\mathcolor{gray}{-}}{30}{\bar{E}}^{\;\! \textcolor{PineGreen}{\hat{2}}}_{\;\! \hat{2}, \frac{b_{\textcolor{NavyBlue}{j}} \pm 1}{b_{\textcolor{NavyBlue}{j}}} \frac{ \mathcolor{gray}{z} }{ 2 }} \mathbb{e}^{\mathbb{i} \mathcolor{gray}{q_{\symup{z}}} \frac{b_{\textcolor{NavyBlue}{j}} \pm 1}{b_{\textcolor{NavyBlue}{j}}} \frac{ \mathcolor{gray}{z} }{ 2 }} ~\mathbb{d} \mathcolor{gray}{q_{\symup{z}}} \label{eq:up-scalar-g-EE-312-discrete-AC-convolution}
		\\ &\hspace*{-1.6em}  \xrightarrow[\text{\bref{eq:FT-krho}}]{\text{linear \textcolor{Plum}{FT}}} \Upsilon^{\;\! \hat{3} \textcolor{PineGreen}{\hat{3}} \hat{1} \hat{2} \textcolor{NavyBlue}{j} }_{\;\! \textcolor{Maroon}{(2)} \textcolor{PineGreen}{\hat{1} \hat{2}} \mathcolor{gray}{z} } \mathcolor{gray}{\int} \mathcolor{gray}{\mathcal F} \left[ {M}^{\;\! \mathcolor{gray}{3} \hat{1} \hat{2} }_{\;\! \hat{3} \textcolor{Maroon}{(2)} \mathcolor{gray}{q_{\symup{z}}} } {\bar{E}}^{\;\! \textcolor{PineGreen}{\hat{1}}}_{\;\! \hat{1}, \frac{b_{\textcolor{NavyBlue}{j}} \pm 1}{b_{\textcolor{NavyBlue}{j}}} \frac{ \mathcolor{gray}{z} }{ 2 }} {\bar{E}}^{\;\! \textcolor{PineGreen}{\hat{2}}}_{\;\! \hat{2}, \frac{b_{\textcolor{NavyBlue}{j}} \pm 1}{b_{\textcolor{NavyBlue}{j}}} \frac{ \mathcolor{gray}{z} }{ 2 }} \right] \mathbb{e}^{\mathbb{i} \mathcolor{gray}{q_{\symup{z}}} \frac{b_{\textcolor{NavyBlue}{j}} \pm 1}{b_{\textcolor{NavyBlue}{j}}} \frac{ \mathcolor{gray}{z} }{ 2 }} ~\mathbb{d} \mathcolor{gray}{q_{\symup{z}}} \label{eq:up-scalar-g-EE-312-discrete-AC-FT}
		\\ &\hspace*{-1.6em} \xrightarrow[]{\text{\bref{eq:FT-kz}}} \Upsilon^{\;\! \hat{3} \textcolor{PineGreen}{\hat{3}} \hat{1} \hat{2} \textcolor{NavyBlue}{j} }_{\;\! \textcolor{Maroon}{(2)} \textcolor{PineGreen}{\hat{1} \hat{2}} \mathcolor{gray}{z} } \mathcolor{gray}{\int} \mathcolor{gray}{\mathcal F} \left[ \mathcolor{gray}{\mathcal F_{z}} \left[ {M}^{\;\! \mathcolor{gray}{3} \hat{1} \hat{2} }_{\;\! \hat{3} \textcolor{Maroon}{(2)} \mathcolor{gray}{z} } \right] {\bar{E}}^{\;\! \textcolor{PineGreen}{\hat{1}}}_{\;\! \hat{1}, \frac{b_{\textcolor{NavyBlue}{j}} \pm 1}{b_{\textcolor{NavyBlue}{j}}} \frac{ \mathcolor{gray}{z} }{ 2 }} {\bar{E}}^{\;\! \textcolor{PineGreen}{\hat{2}}}_{\;\! \hat{2}, \frac{b_{\textcolor{NavyBlue}{j}} \pm 1}{b_{\textcolor{NavyBlue}{j}}} \frac{ \mathcolor{gray}{z} }{ 2 }} \right] \mathbb{e}^{\mathbb{i} \mathcolor{gray}{q_{\symup{z}}} \frac{b_{\textcolor{NavyBlue}{j}} \pm 1}{b_{\textcolor{NavyBlue}{j}}} \frac{ \mathcolor{gray}{z} }{ 2 }} ~\mathbb{d} \mathcolor{gray}{q_{\symup{z}}} \label{eq:up-scalar-g-EE-312-discrete-AC-Mz}
		\\ &\hspace*{-1.6em} =: \Upsilon^{\;\! \hat{3} \textcolor{PineGreen}{\hat{3}} \hat{1} \hat{2} \textcolor{NavyBlue}{j} }_{\;\! \textcolor{Maroon}{(2)} \textcolor{PineGreen}{\hat{1} \hat{2}} \mathcolor{gray}{z} } \xint{\begin{smallmatrix} ~ \\ {}^{}_{\mathcolor{gray}{-}} \\ ~ \end{smallmatrix}}{09}{\mathtt{g}}^{\;\! \textcolor{PineGreen}{\hat{3}}; \text{\textcolor{Maroon}{NLAST}}}_{\;\! \frac{b_{\textcolor{NavyBlue}{j}} \pm 1}{b_{\textcolor{NavyBlue}{j}}} \frac{ \mathcolor{gray}{z} }{ 2 }} := \xint{\begin{smallmatrix} ~ \\ {}^{}_{\mathcolor{gray}{-}} \\ ~ \end{smallmatrix}}{09}{\mathtt{g}}^{\;\! \textcolor{PineGreen}{\hat{3}}}_{\;\! \mathcolor{gray}{z};\text{\textcolor{NavyBlue}{AC}}}~, ~~\text{where}~~ \Upsilon^{\;\! \hat{3} \textcolor{PineGreen}{\hat{3}} \hat{1} \hat{2} \textcolor{NavyBlue}{j} }_{\;\! \textcolor{Maroon}{(2)} \textcolor{PineGreen}{\hat{1} \hat{2}} \mathcolor{gray}{z} } := \Upsilon^{\;\! \hat{3} \textcolor{PineGreen}{\hat{3}} \hat{1} \hat{2} }_{\;\! \textcolor{Maroon}{(2)} \textcolor{PineGreen}{\hat{1} \hat{2}} } a^{\textcolor{NavyBlue}{j}} \dfrac{\mathcolor{gray}{z}}{2} \mathbb{e}^{- \mathbb{i} \xint{\begin{smallmatrix} ~ \\ {}^{}_{\mathcolor{gray}{-}} \\ ~ \end{smallmatrix}}{15}{k}_{\symup{z}}^{\;\! \textcolor{PineGreen}{\hat{3}} } \frac{b_{\textcolor{NavyBlue}{j}} \pm 1}{b_{\textcolor{NavyBlue}{j}}} \frac{ \mathcolor{gray}{z} }{ 2 }} ~, \label{eq:up-scalar-g-EE-312-discrete-AC-NLAST}
	\end{align}
\end{subequations}

另一方面,将 \bref{eq:since-chord} 的\textcolor{NavyBlue}{直流部分} $a_{\textcolor{NavyBlue}{0}} \mathbb{e}^{\mathbb{i}{x}}$ 代入 \bref{eq:up-scalar-g-EE-312-discrete-SINCE},也可类似地得到 $\xint{\begin{smallmatrix} ~ \\ {}^{}_{\mathcolor{gray}{-}} \\ ~ \end{smallmatrix}}{09}{\mathtt{g}}^{\;\! \textcolor{PineGreen}{\hat{3}}}_{\;\! \mathcolor{gray}{z}}$ 的\textcolor{NavyBlue}{直流项} $\xint{\begin{smallmatrix} ~ \\ {}^{}_{\mathcolor{gray}{-}} \\ ~ \end{smallmatrix}}{09}{\mathtt{g}}^{\;\! \textcolor{PineGreen}{\hat{3}}}_{\;\! \mathcolor{gray}{z};\text{\textcolor{NavyBlue}{DC}}}$:
\begin{subequations} \label{eq:up-scalar-g-EE-312-discrete-DC}
	\begin{align}
		\xint{\begin{smallmatrix} ~ \\ {}^{}_{\mathcolor{gray}{-}} \\ ~ \end{smallmatrix}}{09}{\mathtt{g}}^{\;\! \textcolor{PineGreen}{\hat{3}}}_{\;\! \mathcolor{gray}{z};\text{\textcolor{NavyBlue}{DC}}} &\xrightarrow[\text{\bref{eq:up-scalar-g-EE-312-discrete-SINCE}}]{\text{\textcolor{NavyBlue}{DC}~part of}} \Upsilon^{\;\! \hat{3} \textcolor{PineGreen}{\hat{3}} \hat{1} \hat{2} }_{\;\! \textcolor{Maroon}{(2)} \textcolor{PineGreen}{\hat{1} \hat{2}} } a_{\textcolor{NavyBlue}{0}} \mathcolor{gray}{\iiint} \xint{\mathcolor{gray}{-}}{18}{M}^{\;\! \mathcolor{gray}{3} \hat{1} \hat{2} }_{\;\! \hat{3} \textcolor{Maroon}{(2)} \mathcolor{gray}{\bar{q}} } \mathcolor{gray}{\iint} \xint{{}^{}_{\mathcolor{gray}{-}}}{10}{g}^{\;\! \textcolor{PineGreen}{\hat{1}}}_{\;\! \hat{1} \mathcolor{gray}{0}} \xint{{}^{}_{\mathcolor{gray}{-}}}{10}{g}^{\;\! \textcolor{PineGreen}{\hat{2}}}_{\;\! \hat{2} \mathcolor{gray}{0}} \cdot \mathbb{e}^{\mathbb{i} \frac{ \Delta \xint{\begin{smallmatrix} ~ \\ {}^{}_{\mathcolor{gray}{-}} \\ ~ \end{smallmatrix}}{15}{k}_{\symup{z}}^{\;\! \textcolor{PineGreen}{\hat{1} \hat{2} \hat{3}} } \mathcolor{gray}{z} }{ 2 } } \mathcolor{gray}{z} ~ \mathbb{d} \mathcolor{gray}{\bar{k}_{1\symup{\rho}}} \mathbb{d} \mathcolor{gray}{\bar{q}} \label{eq:up-scalar-g-EE-312-discrete-chord-DC} \\ & \xrightarrow[\text{\textcolor{Plum}{rearrange}}]{\text{\bref{eq:up-scalar-g-EE-312-discrete-AC-convolution}}} \Upsilon^{\;\! \hat{3} \textcolor{PineGreen}{\hat{3}} \hat{1} \hat{2} \textcolor{NavyBlue}{0} }_{\;\! \textcolor{Maroon}{(2)} \textcolor{PineGreen}{\hat{1} \hat{2}} \mathcolor{gray}{z} } \mathcolor{gray}{\int}  \xint{\mathcolor{gray}{-}}{18}{M}^{\;\! \mathcolor{gray}{3} \hat{1} \hat{2} }_{\;\! \hat{3} \textcolor{Maroon}{(2)} \mathcolor{gray}{q_{\symup{z}}} } \mathcolor{gray}{*} \xint{\mathcolor{gray}{-}}{30}{\bar{E}}^{\;\! \textcolor{PineGreen}{\hat{1}}}_{\;\! \hat{1}, \frac{ \mathcolor{gray}{z} }{ 2 }} \mathcolor{gray}{*} \xint{\mathcolor{gray}{-}}{30}{\bar{E}}^{\;\! \textcolor{PineGreen}{\hat{2}}}_{\;\! \hat{2}, \frac{ \mathcolor{gray}{z} }{ 2 }} \mathbb{e}^{\mathbb{i} \mathcolor{gray}{q_{\symup{z}}} \frac{ \mathcolor{gray}{z} }{ 2 }} ~\mathbb{d} \mathcolor{gray}{q_{\symup{z}}} \label{eq:up-scalar-g-EE-312-discrete-DC-convolution} \\ & \xrightarrow[\text{\bref{eq:FT-kz}}]{\text{linear \textcolor{Plum}{FT}}} \Upsilon^{\;\! \hat{3} \textcolor{PineGreen}{\hat{3}} \hat{1} \hat{2} \textcolor{NavyBlue}{0} }_{\;\! \textcolor{Maroon}{(2)} \textcolor{PineGreen}{\hat{1} \hat{2}} \mathcolor{gray}{z} } \mathcolor{gray}{\int} \mathcolor{gray}{\mathcal F} \left[ \mathcolor{gray}{\mathcal F_{z}} \left[ {M}^{\;\! \mathcolor{gray}{3} \hat{1} \hat{2} }_{\;\! \hat{3} \textcolor{Maroon}{(2)} \mathcolor{gray}{z} } \right] {\bar{E}}^{\;\! \textcolor{PineGreen}{\hat{1}}}_{\;\! \hat{1}, \frac{ \mathcolor{gray}{z} }{ 2 }} {\bar{E}}^{\;\! \textcolor{PineGreen}{\hat{2}}}_{\;\! \hat{2}, \frac{ \mathcolor{gray}{z} }{ 2 }} \right] \mathbb{e}^{\mathbb{i} \mathcolor{gray}{q_{\symup{z}}} \frac{ \mathcolor{gray}{z} }{ 2 }} ~\mathbb{d} \mathcolor{gray}{q_{\symup{z}}} \label{eq:up-scalar-g-EE-312-discrete-DC-Mz} \\ &= \Upsilon^{\;\! \hat{3} \textcolor{PineGreen}{\hat{3}} \hat{1} \hat{2} \textcolor{NavyBlue}{0} }_{\;\! \textcolor{Maroon}{(2)} \textcolor{PineGreen}{\hat{1} \hat{2}} \mathcolor{gray}{z} } \xint{\begin{smallmatrix} ~ \\ {}^{}_{\mathcolor{gray}{-}} \\ ~ \end{smallmatrix}}{09}{\mathtt{g}}^{\;\! \textcolor{PineGreen}{\hat{3}}; \text{\textcolor{Maroon}{NLAST}}}_{\;\! \frac{ \mathcolor{gray}{z} }{ 2 }} ~,
		~~\text{where}~~ \Upsilon^{\;\! \hat{3} \textcolor{PineGreen}{\hat{3}} \hat{1} \hat{2} \textcolor{NavyBlue}{0} }_{\;\! \textcolor{Maroon}{(2)} \textcolor{PineGreen}{\hat{1} \hat{2}} \mathcolor{gray}{z} } := \Upsilon^{\;\! \hat{3} \textcolor{PineGreen}{\hat{3}} \hat{1} \hat{2} }_{\;\! \textcolor{Maroon}{(2)} \textcolor{PineGreen}{\hat{1} \hat{2}} } a^{\textcolor{NavyBlue}{0}} \mathcolor{gray}{z} \mathbb{e}^{- \mathbb{i} \xint{\begin{smallmatrix} ~ \\ {}^{}_{\mathcolor{gray}{-}} \\ ~ \end{smallmatrix}}{15}{k}_{\symup{z}}^{\;\! \textcolor{PineGreen}{\hat{3}} } \frac{ \mathcolor{gray}{z} }{ 2 } } ~, \label{eq:up-scalar-g-EE-312-discrete-DC-NLAST}
	\end{align}
\end{subequations}
接着,结合\textcolor{NavyBlue}{直}/\textcolor{NavyBlue}{交流项},得到\textcolor{NavyBlue}{双泵浦} ${\mathtt{G}}^{\symup{p}_{1}}_{z}, {\mathtt{G}}^{\symup{p}_{2}}_{z}$ \textcolor{Maroon}{未耗尽近似}条件下的,\textcolor{Maroon}{和频}\textcolor{Maroon}{时空谱}(\textcolor{NavyBlue}{无衍射}\textcolor{PineGreen}{复振幅})\textcolor{PineGreen}{匹配型}\textcolor{Maroon}{非线性角谱}\textcolor{Plum}{解}
\begin{subequations} \label{eq:up-scalar-g-EE-312-discrete-chord}
	\begin{align}
		&\hspace*{-3.7em} \xint{\begin{smallmatrix} ~ \\ {}^{}_{\mathcolor{gray}{-}} \\ ~ \end{smallmatrix}}{09}{\mathtt{g}}^{\;\! \textcolor{PineGreen}{\hat{3}}}_{\;\! \mathcolor{gray}{z}} = \xint{\begin{smallmatrix} ~ \\ {}^{}_{\mathcolor{gray}{-}} \\ ~ \end{smallmatrix}}{09}{\mathtt{g}}^{\;\! \textcolor{PineGreen}{\hat{3}}}_{\;\! \mathcolor{gray}{z};\text{\textcolor{NavyBlue}{DC}}} + \xint{\begin{smallmatrix} ~ \\ {}^{}_{\mathcolor{gray}{-}} \\ ~ \end{smallmatrix}}{09}{\mathtt{g}}^{\;\! \textcolor{PineGreen}{\hat{3}}}_{\;\! \mathcolor{gray}{z};\text{\textcolor{NavyBlue}{AC}}} = \Upsilon^{\;\! \hat{3} \textcolor{PineGreen}{\hat{3}} \hat{1} \hat{2} \textcolor{NavyBlue}{j} }_{\;\! \textcolor{Maroon}{(2)} \textcolor{PineGreen}{\hat{1} \hat{2}} \mathcolor{gray}{z} } \xint{\begin{smallmatrix} ~ \\ {}^{}_{\mathcolor{gray}{-}} \\ ~ \end{smallmatrix}}{09}{\mathtt{g}}^{\;\! \textcolor{PineGreen}{\hat{3}}; \text{\textcolor{Maroon}{NLAST}}}_{\;\! \frac{b_{\textcolor{NavyBlue}{j}} \pm 1}{b_{\textcolor{NavyBlue}{j}}} \frac{ \mathcolor{gray}{z} }{ 2 }} + \Upsilon^{\;\! \hat{3} \textcolor{PineGreen}{\hat{3}} \hat{1} \hat{2} \textcolor{NavyBlue}{0} }_{\;\! \textcolor{Maroon}{(2)} \textcolor{PineGreen}{\hat{1} \hat{2}} \mathcolor{gray}{z} } \xint{\begin{smallmatrix} ~ \\ {}^{}_{\mathcolor{gray}{-}} \\ ~ \end{smallmatrix}}{09}{\mathtt{g}}^{\;\! \textcolor{PineGreen}{\hat{3}}; \text{\textcolor{Maroon}{NLAST}}}_{\;\! \frac{ \mathcolor{gray}{z} }{ 2 }} \label{eq:up-scalar-g-EE-312-discrete-chord_a} \\ 
		&\hspace*{-3.7em} \xrightarrow[\text{\bref{eq:up-scalar-g-EE-312-discrete-AC-Mz}}]{\text{\bref{eq:up-scalar-g-EE-312-discrete-DC-Mz}}} \Upsilon^{\;\! \hat{3} \textcolor{PineGreen}{\hat{3}} \hat{1} \hat{2} }_{\;\! \textcolor{Maroon}{(2)} \textcolor{PineGreen}{\hat{1} \hat{2}} } \mathcolor{gray}{z} \mathcolor{gray}{\int} \begin{pmatrix} a^{\textcolor{NavyBlue}{0}} ~ \mathcolor{gray}{\mathcal F} \left[ \mathcolor{gray}{\mathcal F_{z}} \left[ {M}^{\;\! \mathcolor{gray}{3} \hat{1} \hat{2} }_{\;\! \hat{3} \textcolor{Maroon}{(2)} \mathcolor{gray}{z} } \right] {\bar{E}}^{\;\! \textcolor{PineGreen}{\hat{1}}}_{\;\! \hat{1}, \frac{ \mathcolor{gray}{z} }{ 2 }} {\bar{E}}^{\;\! \textcolor{PineGreen}{\hat{2}}}_{\;\! \hat{2}, \frac{ \mathcolor{gray}{z} }{ 2 }} \right] \mathbb{e}^{\mathbb{i} \left( \mathcolor{gray}{q_{\symup{z}}} - \xint{\begin{smallmatrix} ~ \\ {}^{}_{\mathcolor{gray}{-}} \\ ~ \end{smallmatrix}}{15}{k}_{\symup{z}}^{\;\! \textcolor{PineGreen}{\hat{3}} } \right) \frac{ \mathcolor{gray}{z} }{ 2 }} \\ + \dfrac{a^{\textcolor{NavyBlue}{j}}}{2} ~ \mathcolor{gray}{\mathcal F} \left[ \mathcolor{gray}{\mathcal F_{z}} \left[ {M}^{\;\! \mathcolor{gray}{3} \hat{1} \hat{2} }_{\;\! \hat{3} \textcolor{Maroon}{(2)} \mathcolor{gray}{z} } \right] {\bar{E}}^{\;\! \textcolor{PineGreen}{\hat{1}}}_{\;\! \hat{1}, \frac{b_{\textcolor{NavyBlue}{j}} \pm 1}{b_{\textcolor{NavyBlue}{j}}} \frac{ \mathcolor{gray}{z} }{ 2 }} {\bar{E}}^{\;\! \textcolor{PineGreen}{\hat{2}}}_{\;\! \hat{2}, \frac{b_{\textcolor{NavyBlue}{j}} \pm 1}{b_{\textcolor{NavyBlue}{j}}} \frac{ \mathcolor{gray}{z} }{ 2 }} \right] \mathbb{e}^{\mathbb{i} \left( \mathcolor{gray}{q_{\symup{z}}} - \xint{\begin{smallmatrix} ~ \\ {}^{}_{\mathcolor{gray}{-}} \\ ~ \end{smallmatrix}}{15}{k}_{\symup{z}}^{\;\! \textcolor{PineGreen}{\hat{3}} } \right) \frac{b_{\textcolor{NavyBlue}{j}} \pm 1}{b_{\textcolor{NavyBlue}{j}}} \frac{ \mathcolor{gray}{z} }{ 2 }} \end{pmatrix} \mathbb{d} \mathcolor{gray}{q_{\symup{z}}} \label{eq:up-scalar-g-EE-312-discrete-chord_b}~,
	\end{align}
\end{subequations}
最后,得完整的、\textcolor{NavyBlue}{带衍射项} $ \mathbb{e}^{ \mathbb{i} \xint{\begin{smallmatrix} ~ \\ {}^{}_{\mathcolor{gray}{-}} \\ ~ \end{smallmatrix}}{15}{k}_{\symup{z}}^{\;\! \textcolor{PineGreen}{\hat{3}} } \mathcolor{gray}{z} }$ 的\textcolor{Maroon}{和频时空谱}\textcolor{PineGreen}{匹配型}\textcolor{NavyBlue}{含衍射}\textcolor{PineGreen}{复振幅} \textcolor{Maroon}{NLAST} \textcolor{Plum}{解}
\begin{subequations} \label{eq:up-scalar-G-EE-312-discrete-chord}
	\begin{align}
		&\hspace*{-3.7em} \xint{\mathcolor{gray}{-}}{20}{\mathtt{G}}^{\;\! \textcolor{PineGreen}{\hat{3}}}_{\;\! \mathcolor{gray}{z}} \xrightarrow[]{\text{\bref{eq:amp_phase}}} \xint{\begin{smallmatrix} ~ \\ {}^{}_{\mathcolor{gray}{-}} \\ ~ \end{smallmatrix}}{09}{\mathtt{g}}^{\;\! \textcolor{PineGreen}{\hat{3}}}_{\;\! \mathcolor{gray}{z}} \mathbb{e}^{ \mathbb{i} \xint{\begin{smallmatrix} ~ \\ {}^{}_{\mathcolor{gray}{-}} \\ ~ \end{smallmatrix}}{15}{k}_{\symup{z}}^{\;\! \textcolor{PineGreen}{\hat{3}} } \mathcolor{gray}{z} } \label{eq:up-scalar-G-EE-312-discrete-chord_a} \\ 
		&\hspace*{-3.7em} \xrightarrow[]{\text{\bref{eq:up-scalar-g-EE-312-discrete-chord_b}}} \Upsilon^{\;\! \hat{3} \textcolor{PineGreen}{\hat{3}} \hat{1} \hat{2} }_{\;\! \textcolor{Maroon}{(2)} \textcolor{PineGreen}{\hat{1} \hat{2}} } \mathcolor{gray}{z} \mathcolor{gray}{\int} \begin{pmatrix} a^{\textcolor{NavyBlue}{0}} ~ \mathcolor{gray}{\mathcal F} \left[ \mathcolor{gray}{\mathcal F_{z}} \left[ {M}^{\;\! \mathcolor{gray}{3} \hat{1} \hat{2} }_{\;\! \hat{3} \textcolor{Maroon}{(2)} \mathcolor{gray}{z} } \right] {\bar{E}}^{\;\! \textcolor{PineGreen}{\hat{1}}}_{\;\! \hat{1}, \frac{ \mathcolor{gray}{z} }{ 2 }} {\bar{E}}^{\;\! \textcolor{PineGreen}{\hat{2}}}_{\;\! \hat{2}, \frac{ \mathcolor{gray}{z} }{ 2 }} \right] \mathbb{e}^{\mathbb{i} \left( \mathcolor{gray}{q_{\symup{z}}} + \xint{\begin{smallmatrix} ~ \\ {}^{}_{\mathcolor{gray}{-}} \\ ~ \end{smallmatrix}}{15}{k}_{\symup{z}}^{\;\! \textcolor{PineGreen}{\hat{3}} } \right) \frac{ \mathcolor{gray}{z} }{ 2 }} \\ + \dfrac{a^{\textcolor{NavyBlue}{j}}}{2} ~ \mathcolor{gray}{\mathcal F} \left[ \mathcolor{gray}{\mathcal F_{z}} \left[ {M}^{\;\! \mathcolor{gray}{3} \hat{1} \hat{2} }_{\;\! \hat{3} \textcolor{Maroon}{(2)} \mathcolor{gray}{z} } \right] {\bar{E}}^{\;\! \textcolor{PineGreen}{\hat{1}}}_{\;\! \hat{1}, \frac{b_{\textcolor{NavyBlue}{j}} \pm 1}{b_{\textcolor{NavyBlue}{j}}} \frac{ \mathcolor{gray}{z} }{ 2 }} {\bar{E}}^{\;\! \textcolor{PineGreen}{\hat{2}}}_{\;\! \hat{2}, \frac{b_{\textcolor{NavyBlue}{j}} \pm 1}{b_{\textcolor{NavyBlue}{j}}} \frac{ \mathcolor{gray}{z} }{ 2 }} \right] \mathbb{e}^{\mathbb{i} \left( \mathcolor{gray}{q_{\symup{z}}} + \xint{\begin{smallmatrix} ~ \\ {}^{}_{\mathcolor{gray}{-}} \\ ~ \end{smallmatrix}}{15}{k}_{\symup{z}}^{\;\! \textcolor{PineGreen}{\hat{3}} } \right) \frac{b_{\textcolor{NavyBlue}{j}} \mp 1}{b_{\textcolor{NavyBlue}{j}}} \frac{ \mathcolor{gray}{z} }{ 2 }} \end{pmatrix} \mathbb{d} \mathcolor{gray}{q_{\symup{z}}} \label{eq:up-scalar-G-EE-312-discrete-chord_b}~,
	\end{align}
\end{subequations}

\marginLeft[-2.4em]{sssec:NLAST_match_DFG}\subsubsection{DFG 差频过程的匹配型 NLAST}\label{sssec:NLAST_match_DFG}


\marginLeft[-2.4em]{sec:NLAST_iterate}\section{\textcolor{Maroon}{Iterative} 迭代型非线性角谱 \textcolor{Maroon}{NLAST}}\label{sec:NLAST_iterate}



\marginLeft[-2.4em]{sec:NLAST_dismatch}\section{\textcolor{Maroon}{Mismatched} 失配型非线性角谱 \textcolor{Maroon}{NLAST}}\label{sec:NLAST_dismatch}


那么仍先以和频为例,一步步给出其匹配型\textcolor{Maroon}{非线性角谱}解。