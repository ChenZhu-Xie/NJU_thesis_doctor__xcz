\marginLeft[-2.4em]{chap:NFCO}\chapter{任意 \texorpdfstring{$\bar{\bar{\bar{\chi}}}$}{$\bar{\bar{\bar{\text{χ}}}}$} 材料里的矢量非线性傅立叶晶体光学}\label{chap:NFCO}

\bref{ssec:E-waveq-nonlinear} 末得到了\textcolor{PineGreen}{纯电(非磁)各向异性}介质中的\textcolor{Plum}{非线性}矢量电场波动方程 \bref{eq:simplify7-LE0-SVA-V_1singular-nokxky-zeta-g},其右侧的二阶\textcolor{Plum}{局域}\textcolor{Plum}{非线性}\textcolor{NavyBlue}{电偶-$(\text{电偶}\otimes\text{电偶})$}极\textcolor{NavyBlue}{波源}(的\textcolor{Plum}{横向}部分) $\Xint{\mathcolor{gray}{-}}{25}{\bar{P}}^{\;\! \mathcolor{gray}{\omega} \textcolor{PineGreen}{\imath}}_{\;\! \symup{\rho} \mathcolor{gray}{z} \textcolor{Maroon}{(2)}}$(\bref{eq:vec-DP^(2)-p_pp,eq:vec-DP^(2)-p_pp,eq:vec-DP^(2)-plane_wave_basis-p_pp}),可以按需更改为三阶、四阶\textcolor{Plum}{非线性} $\Xint{\mathcolor{gray}{-}}{25}{\bar{P}}^{\;\! \mathcolor{gray}{\omega} \textcolor{PineGreen}{\imath}}_{\;\! \symup{\rho} \mathcolor{gray}{z} \textcolor{Maroon}{(3)}}, \Xint{\mathcolor{gray}{-}}{25}{\bar{P}}^{\;\! \mathcolor{gray}{\omega} \textcolor{PineGreen}{\imath}}_{\;\! \symup{\rho} \mathcolor{gray}{z} \textcolor{Maroon}{(4)}}$,或者一阶\textcolor{NavyBlue}{势散射}源 $\Xint{\mathcolor{gray}{-}}{25}{\bar{P}}^{\;\! \mathcolor{gray}{\omega} \textcolor{PineGreen}{\imath}}_{\;\! \symup{\rho} \mathcolor{gray}{z} \textcolor{Maroon}{(1)}}$(\bref{eq:nonlinear(2)-wave_wkrho-simplify4-2}),甚至他们的\textcolor{Plum}{线性组合}/\textcolor{Plum}{叠加} $\Xint{\mathcolor{gray}{-}}{25}{\bar{P}}^{\;\! \mathcolor{gray}{\omega} \textcolor{PineGreen}{\imath}}_{\;\! \symup{\rho} \mathcolor{gray}{z}}$。

将每一个参与相互作用的\textcolor{gray}{时间频率组分} $\mathcolor{gray}{\omega}$ 所对应的 \bref{eq:simplify7-LE0-SVA-V_1singular-nokxky-zeta-g} 组合起来,即得到\textcolor{PineGreen}{实验室}\textcolor{Plum}{坐标系} \textcolor{PineGreen}{$\mathcal{Z}$ 系}下的矢量(电场)\textcolor{Maroon}{时空谱}耦合波方程组 --- 它既可以由相干\textcolor{NavyBlue}{脉冲光连续谱} $\{ \mathcolor{gray}{\omega} \}$ 对应的\textcolor{Plum}{无穷个}矢量\textcolor{Maroon}{时空谱}耦合波方程组成,也可以由\textcolor{Plum}{有限个}\textcolor{NavyBlue}{连续光} $\{ \mathcolor{gray}{\omega}_{\textcolor{Maroon}{i}} \}$ 对应的有限个矢量\textcolor{Maroon}{时空谱}耦合波方程组成;也可以只是\textcolor{gray}{单一频率} $\mathcolor{gray}{\omega}$ 的光,其自己参与的如\textcolor{Maroon}{三}/\textcolor{Maroon}{四波混频}、\textcolor{Maroon}{一次电光效应}或\textcolor{PineGreen}{折射率微调制}引起的\textcolor{Maroon}{(势)散射过程}等,所对应的\textcolor{Plum}{单个}矢量\textcolor{Maroon}{时空谱}的传播/波动方程。

%\vspace*{-7.5em}

\marginLeft[-2.4em]{sec:up_convert}\section{\textcolor{Maroon}{Up conversion} 上转换 - 电场本征复振幅 \textcolor{Maroon}{equation}}\label{sec:up_convert}

利用 \bref{eq:vec-amp_polar} 即 $\Xint{{}^{}_{\mathcolor{gray}{-}}}{10}{\bar{g}}^{\;\!\mathcolor{gray}{\omega} \textcolor{PineGreen}{\jmath}}_{\;\! \textcolor{Maroon}{\Yup} \mathcolor{gray}{z}} := \Xint{\begin{smallmatrix} ~ \\ {}^{}_{\mathcolor{gray}{-}} \\ ~ \end{smallmatrix}}{09}{\mathtt{g}}^{\;\!\mathcolor{gray}{\omega} \textcolor{PineGreen}{\jmath}}_{\;\! \mathcolor{gray}{z}} \Xint{{}^{}_{\mathcolor{gray}{-}}}{10}{\bar{g}}^{\;\!\mathcolor{gray}{\omega} \textcolor{PineGreen}{\jmath}}_{\;\! \textcolor{Maroon}{\Yup}}$ 在\textcolor{PineGreen}{纯电(非磁)各向异性}介质中的\textcolor{Plum}{横向}形式,将\textcolor{Plum}{复}矢量 $\Xint{{}^{}_{\mathcolor{gray}{-}}}{10}{\bar{g}}^{\;\!\mathcolor{gray}{\omega} \textcolor{PineGreen}{\imath}}_{\;\! \textcolor{Maroon}{\Yup}}$ 三分量\textcolor{Plum}{复归一化}\Footnote{注意,不像 \bref{eq:transition_matrix-transverse_input},\textcolor{PineGreen}{本征偏振态} $\Xint{{}^{}_{\mathcolor{gray}{-}}}{10}{\bar{g}}^{\;\!\mathcolor{gray}{\omega} \textcolor{PineGreen}{\imath}}_{\;\! \textcolor{Maroon}{\Yup}}, \Xint{{}^{}_{\mathcolor{gray}{-}}}{10}{\bar{g}}^{\;\!\mathcolor{gray}{\omega} \textcolor{PineGreen}{\imath}}_{\;\! \textcolor{Maroon}{\symup{\rho}}}$ 在\textcolor{Plum}{线性}\textcolor{PineGreen}{晶体光学}中无需\textcolor{Plum}{(三维)复归一化},但在\textcolor{Plum}{非线性}\textcolor{PineGreen}{晶体光学}中最好\textcolor{Plum}{三维复归一化},这既有助于\textcolor{Plum}{标准化}后续引入的\textcolor{NavyBlue}{有效非线性系数},又赋予和明确了\textcolor{PineGreen}{本征复振幅} $\Xint{\begin{smallmatrix} ~ \\ {}^{}_{\mathcolor{gray}{-}} \\ ~ \end{smallmatrix}}{09}{\mathtt{g}}^{\;\!\mathcolor{gray}{\omega} \textcolor{PineGreen}{\imath}}_{\;\! \mathcolor{gray}{z}}$、\textcolor{PineGreen}{本征偏振态} $\Xint{{}^{}_{\mathcolor{gray}{-}}}{10}{\bar{g}}^{\;\!\mathcolor{gray}{\omega} \textcolor{PineGreen}{\imath}}_{\;\! \textcolor{Maroon}{\Yup}}$ 各自的物理意义。但实际上,\textcolor{Plum}{非线性}\textcolor{PineGreen}{晶体光学}也允许\textcolor{Plum}{不归一化} $\Xint{{}^{}_{\mathcolor{gray}{-}}}{10}{\bar{g}}^{\;\!\mathcolor{gray}{\omega} \textcolor{PineGreen}{\imath}}_{\;\! \textcolor{Maroon}{\Yup}}$ 或 $\Xint{{}^{}_{\mathcolor{gray}{-}}}{10}{\bar{g}}^{\;\!\mathcolor{gray}{\omega} \textcolor{PineGreen}{\imath}}_{\;\! \textcolor{Maroon}{\symup{\rho}}}$,见下文。}后的\textcolor{Plum}{二维}\textcolor{PineGreen}{本征偏振态} $\Xint{{}^{}_{\mathcolor{gray}{-}}}{10}{\bar{g}}^{\;\!\mathcolor{gray}{\omega} \textcolor{PineGreen}{\imath}}_{\;\! \textcolor{Maroon}{\symup{\rho}} \mathcolor{gray}{z}} := \Xint{\begin{smallmatrix} ~ \\ {}^{}_{\mathcolor{gray}{-}} \\ ~ \end{smallmatrix}}{09}{\mathtt{g}}^{\;\!\mathcolor{gray}{\omega} \textcolor{PineGreen}{\imath}}_{\;\! \mathcolor{gray}{z}} \Xint{{}^{}_{\mathcolor{gray}{-}}}{10}{\hat{g}}^{\;\!\mathcolor{gray}{\omega} \textcolor{PineGreen}{\imath}}_{\;\! \textcolor{Maroon}{\symup{\rho}}}$代入矢量\textcolor{Plum}{非线性}波动方程 \bref{eq:simplify7-LE0-SVA-V_1singular-nokxky-zeta-g} 的\textcolor{NavyBlue}{左侧场}中,并两侧\textcolor{Plum}{左点乘}不含 $\mathcolor{gray}{z}$ 的\textcolor{Plum}{二维横向}\textcolor{PineGreen}{本征偏振态} $\Xint{{}^{}_{\mathcolor{gray}{-}}}{10}{\hat{g}}^{\;\!\mathcolor{gray}{\omega} \textcolor{PineGreen}{\imath}}_{\;\! \textcolor{Maroon}{\symup{\rho}}}$ 的\textcolor{Plum}{共轭转置} $\Xint{{}^{}_{\mathcolor{gray}{-}}}{10}{\hat{g}}^{\;\!\mathcolor{gray}{\omega} \textcolor{PineGreen}{\imath} \textcolor{Plum}{\dag}}_{\;\! \textcolor{Maroon}{\symup{\rho}}}$,可得
\begin{subequations} \label{eq:simplify7-scalar}
	\begin{align}
		\Xint{{}^{}_{\mathcolor{gray}{-}}}{10}{\hat{g}}^{\;\!\mathcolor{gray}{\omega} \textcolor{PineGreen}{\imath}}_{\;\! \textcolor{Maroon}{\symup{\rho}}} \mathcolor{gray}{\nabla_z} \Xint{\begin{smallmatrix} ~ \\ {}^{}_{\mathcolor{gray}{-}} \\ ~ \end{smallmatrix}}{09}{\mathtt{g}}^{\;\!\mathcolor{gray}{\omega} \textcolor{PineGreen}{\imath}}_{\;\! \mathcolor{gray}{z}} &\xrightarrow[\text{\bref{eq:simplify7-LE0-SVA-V_1singular-nokxky-zeta-g}}]{\text{$\textcolor{Maroon}{\Yup} \to \textcolor{Maroon}{\symup{\rho}} \left( \text{\bref{eq:vec-amp_polar}} \right)$}} \mathbb{i} k_{\textcolor{Maroon}{\mathsf{o}} \mathcolor{gray}{\omega}}^{\;\! 2} \frac{\Xint{\mathcolor{gray}{-}}{25}{\bar{P}}^{\;\! \mathcolor{gray}{\omega} \textcolor{PineGreen}{\imath}}_{\;\! \textcolor{Maroon}{\symup{\rho}} \mathcolor{gray}{z}}}{2 \Xint{\begin{smallmatrix} ~ \\ {}^{}_{\mathcolor{gray}{-}} \\ ~ \end{smallmatrix}}{15}{k}_{\;\! \symup{z}}^{\;\! \mathcolor{gray}{\omega} \textcolor{PineGreen}{\imath}} \mathbb{e}^{\mathbb{i} \Xint{\begin{smallmatrix} ~ \\ {}^{}_{\mathcolor{gray}{-}} \\ ~ \end{smallmatrix}}{15}{k}_{\symup{z}}^{\;\! \mathcolor{gray}{\omega} \textcolor{PineGreen}{\imath}} \mathcolor{gray}{z}}} \label{eq:simplify7-scalar-g} \\
		\mathcolor{gray}{\nabla_z} \Xint{\begin{smallmatrix} ~ \\ {}^{}_{\mathcolor{gray}{-}} \\ ~ \end{smallmatrix}}{09}{\mathtt{g}}^{\;\!\mathcolor{gray}{\omega} \textcolor{PineGreen}{\imath}}_{\;\! \mathcolor{gray}{z}} &\xrightarrow[\text{$\text{■} $\bref{eq:simplify7-scalar-g}}]{\text{$\left( \Xint{{}^{}_{\mathcolor{gray}{-}}}{10}{\hat{g}}^{\;\!\mathcolor{gray}{\omega} \textcolor{PineGreen}{\imath} \textcolor{Plum}{\dag}}_{\;\! \textcolor{Maroon}{\symup{\rho}}} \cdot \text{■} \right) \big/ \left( \Xint{{}^{}_{\mathcolor{gray}{-}}}{10}{\hat{g}}^{\;\!\mathcolor{gray}{\omega} \textcolor{PineGreen}{\imath} \textcolor{Plum}{\dag}}_{\;\! \textcolor{Maroon}{\symup{\rho}}} \cdot \Xint{{}^{}_{\mathcolor{gray}{-}}}{10}{\hat{g}}^{\;\!\mathcolor{gray}{\omega} \textcolor{PineGreen}{\imath}}_{\;\! \textcolor{Maroon}{\symup{\rho}}} \right)$}} \mathbb{i} k_{\textcolor{Maroon}{\mathsf{o}} \mathcolor{gray}{\omega}}^{\;\! 2} \frac{\Xint{{}^{}_{\mathcolor{gray}{-}}}{10}{\hat{g}}^{\;\!\mathcolor{gray}{\omega} \textcolor{PineGreen}{\imath} \textcolor{Plum}{\dag}}_{\;\! \textcolor{Maroon}{\symup{\rho}}} \cdot \Xint{\mathcolor{gray}{-}}{25}{\bar{P}}^{\;\! \mathcolor{gray}{\omega} \textcolor{PineGreen}{\imath}}_{\;\! \textcolor{Maroon}{\symup{\rho}} \mathcolor{gray}{z}}}{\Xint{{}^{}_{\mathcolor{gray}{-}}}{10}{\hat{g}}^{\;\!\mathcolor{gray}{\omega} \textcolor{PineGreen}{\imath} \textcolor{Plum}{\dag}}_{\;\! \textcolor{Maroon}{\symup{\rho}}} \cdot \Xint{{}^{}_{\mathcolor{gray}{-}}}{10}{\hat{g}}^{\;\!\mathcolor{gray}{\omega} \textcolor{PineGreen}{\imath}}_{\;\! \textcolor{Maroon}{\symup{\rho}}} 2 \Xint{\begin{smallmatrix} ~ \\ {}^{}_{\mathcolor{gray}{-}} \\ ~ \end{smallmatrix}}{15}{k}_{\;\! \symup{z}}^{\;\! \mathcolor{gray}{\omega} \textcolor{PineGreen}{\imath}} \mathbb{e}^{\mathbb{i} \Xint{\begin{smallmatrix} ~ \\ {}^{}_{\mathcolor{gray}{-}} \\ ~ \end{smallmatrix}}{15}{k}_{\symup{z}}^{\;\! \mathcolor{gray}{\omega} \textcolor{PineGreen}{\imath}} \mathcolor{gray}{z}}} ~,  \label{eq:simplify7-scalar-g-conjugate}
	\end{align}
\end{subequations}
其中 \bref{eq:simplify7-scalar-g-conjugate} 即为标量\textcolor{Maroon}{时空谱},即\textcolor{PineGreen}{本征复振幅} $\Xint{\begin{smallmatrix} ~ \\ {}^{}_{\mathcolor{gray}{-}} \\ ~ \end{smallmatrix}}{09}{\mathtt{g}}^{\;\!\mathcolor{gray}{\omega} \textcolor{PineGreen}{\imath}}_{\;\! \mathcolor{gray}{z}}$ 满足的矢量\textcolor{Plum}{非线性}波动方程。分母中的 $\Xint{{}^{}_{\mathcolor{gray}{-}}}{10}{\hat{g}}^{\;\!\mathcolor{gray}{\omega} \textcolor{PineGreen}{\imath} \textcolor{Plum}{\dag}}_{\;\! \textcolor{Maroon}{\symup{\rho}}} \cdot \Xint{{}^{}_{\mathcolor{gray}{-}}}{10}{\hat{g}}^{\;\!\mathcolor{gray}{\omega} \textcolor{PineGreen}{\imath}}_{\;\! \textcolor{Maroon}{\symup{\rho}}}$,其实在暗示 $\Xint{{}^{}_{\mathcolor{gray}{-}}}{10}{\hat{g}}^{\;\!\mathcolor{gray}{\omega} \textcolor{PineGreen}{\imath}}_{\;\! \textcolor{Maroon}{\symup{\rho}}}$ 既可以是\textcolor{Plum}{二维复归一化},也可以是\textcolor{Plum}{三维复归一化}后的。这对应 $\Xint{{}^{}_{\mathcolor{gray}{-}}}{10}{\hat{g}}^{\;\!\mathcolor{gray}{\omega} \textcolor{PineGreen}{\imath} \textcolor{Plum}{\dag}}_{\;\! \textcolor{Maroon}{\symup{\rho}}} \cdot \Xint{{}^{}_{\mathcolor{gray}{-}}}{10}{\hat{g}}^{\;\!\mathcolor{gray}{\omega} \textcolor{PineGreen}{\imath}}_{\;\! \textcolor{Maroon}{\symup{\rho}}}$ 的值,既可以是也可以不是 $1$。并且甚至可以对 \bref{eq:simplify7-scalar-g} 左侧随便\textcolor{Plum}{点乘}一个\textcolor{Plum}{二维}复向量(场) $\Xint{{}^{}_{\mathcolor{gray}{-}}}{04}{\bar{c}}^{\;\!\mathcolor{gray}{\omega} \textcolor{PineGreen}{\imath}}_{\;\! \textcolor{Maroon}{\symup{\rho}}}$(不一定非得是\textcolor{Plum}{横向}\textcolor{PineGreen}{本征偏振态}的\textcolor{Plum}{共轭转置} $\Xint{{}^{}_{\mathcolor{gray}{-}}}{10}{\hat{g}}^{\;\!\mathcolor{gray}{\omega} \textcolor{PineGreen}{\imath} \textcolor{Plum}{\dag}}_{\;\! \textcolor{Maroon}{\symup{\rho}}}$)都行,只需保证 \bref{eq:simplify7-scalar-g-conjugate} 分母中的 $\Xint{{}^{}_{\mathcolor{gray}{-}}}{04}{\bar{c}}^{\;\!\mathcolor{gray}{\omega} \textcolor{PineGreen}{\imath}}_{\;\! \textcolor{Maroon}{\symup{\rho}}} \cdot \Xint{{}^{}_{\mathcolor{gray}{-}}}{10}{\hat{g}}^{\;\!\mathcolor{gray}{\omega} \textcolor{PineGreen}{\imath}}_{\;\! \textcolor{Maroon}{\symup{\rho}}} \neq 0$。

\bref{eq:simplify7-scalar} 中所有\textcolor{Plum}{横向} $\textcolor{Maroon}{\symup{\rho}}$ 场,somehow\Footnote{出于物理学家的直觉和对形式美的追求。类似数学家的“注意到”,但没有他们的“注意到”那么严谨。}可进一步写做笛卡尔\textcolor{Plum}{三分量} $\textcolor{Maroon}{\Yup}$ 形式
\begin{subequations} \label{eq:simplify8-scalar}
\begin{align}
	\Xint{{}^{}_{\mathcolor{gray}{-}}}{10}{\hat{g}}^{\;\!\mathcolor{gray}{\omega} \textcolor{PineGreen}{\imath}}_{\;\! \textcolor{Maroon}{\Yup}} \mathcolor{gray}{\nabla_z} \Xint{\begin{smallmatrix} ~ \\ {}^{}_{\mathcolor{gray}{-}} \\ ~ \end{smallmatrix}}{09}{\mathtt{g}}^{\;\!\mathcolor{gray}{\omega} \textcolor{PineGreen}{\imath}}_{\;\! \mathcolor{gray}{z}} &\xrightarrow[\text{\bref{eq:simplify7-scalar-g}}]{\text{$\textcolor{Maroon}{\symup{\rho}} \to \textcolor{Maroon}{\Yup}$}} \mathbb{i} k_{\textcolor{Maroon}{\mathsf{o}} \mathcolor{gray}{\omega}}^{\;\! 2} \frac{\Xint{\mathcolor{gray}{-}}{25}{\bar{P}}^{\;\! \mathcolor{gray}{\omega} \textcolor{PineGreen}{\imath}}_{\;\! \textcolor{Maroon}{\Yup} \mathcolor{gray}{z}}}{2 \Xint{\begin{smallmatrix} ~ \\ {}^{}_{\mathcolor{gray}{-}} \\ ~ \end{smallmatrix}}{15}{k}_{\;\! \symup{z}}^{\;\! \mathcolor{gray}{\omega} \textcolor{PineGreen}{\imath}} \mathbb{e}^{\mathbb{i} \Xint{\begin{smallmatrix} ~ \\ {}^{}_{\mathcolor{gray}{-}} \\ ~ \end{smallmatrix}}{15}{k}_{\symup{z}}^{\;\! \mathcolor{gray}{\omega} \textcolor{PineGreen}{\imath}} \mathcolor{gray}{z}}} \label{eq:simplify8-scalar-g} \\
	\mathcolor{gray}{\nabla_z} \Xint{\begin{smallmatrix} ~ \\ {}^{}_{\mathcolor{gray}{-}} \\ ~ \end{smallmatrix}}{09}{\mathtt{g}}^{\;\!\mathcolor{gray}{\omega} \textcolor{PineGreen}{\imath}}_{\;\! \mathcolor{gray}{z}} &\xrightarrow[\text{\bref{eq:simplify7-scalar-g-conjugate}}]{\text{$\textcolor{Maroon}{\symup{\rho}} \to \textcolor{Maroon}{\Yup}$}} \mathbb{i} k_{\textcolor{Maroon}{\mathsf{o}} \mathcolor{gray}{\omega}}^{\;\! 2} \frac{\Xint{{}^{}_{\mathcolor{gray}{-}}}{10}{\hat{g}}^{\;\!\mathcolor{gray}{\omega} \textcolor{PineGreen}{\imath} \textcolor{Plum}{\dag}}_{\;\! \textcolor{Maroon}{\Yup}} \cdot \Xint{\mathcolor{gray}{-}}{25}{\bar{P}}^{\;\! \mathcolor{gray}{\omega} \textcolor{PineGreen}{\imath}}_{\;\! \textcolor{Maroon}{\Yup} \mathcolor{gray}{z}}}{\Xint{{}^{}_{\mathcolor{gray}{-}}}{10}{\hat{g}}^{\;\!\mathcolor{gray}{\omega} \textcolor{PineGreen}{\imath} \textcolor{Plum}{\dag}}_{\;\! \textcolor{Maroon}{\Yup}} \cdot \Xint{{}^{}_{\mathcolor{gray}{-}}}{10}{\hat{g}}^{\;\!\mathcolor{gray}{\omega} \textcolor{PineGreen}{\imath}}_{\;\! \textcolor{Maroon}{\Yup}} 2 \Xint{\begin{smallmatrix} ~ \\ {}^{}_{\mathcolor{gray}{-}} \\ ~ \end{smallmatrix}}{15}{k}_{\;\! \symup{z}}^{\;\! \mathcolor{gray}{\omega} \textcolor{PineGreen}{\imath}} \mathbb{e}^{\mathbb{i} \Xint{\begin{smallmatrix} ~ \\ {}^{}_{\mathcolor{gray}{-}} \\ ~ \end{smallmatrix}}{15}{k}_{\symup{z}}^{\;\! \mathcolor{gray}{\omega} \textcolor{PineGreen}{\imath}} \mathcolor{gray}{z}}} ~,  \label{eq:simplify8-scalar-g-conjugate}
\end{align}
\end{subequations}
注,\bref{eq:simplify8-scalar-g-conjugate} 仍属于矢量\textcolor{Plum}{非线性}波动方程 \bypertarget{waveq-scalar-2-vector},尽管方程左侧是对\textcolor{PineGreen}{本征复振幅}标量场 $\Xint{\begin{smallmatrix} ~ \\ {}^{}_{\mathcolor{gray}{-}} \\ ~ \end{smallmatrix}}{09}{\mathtt{g}}^{\;\!\mathcolor{gray}{\omega} \textcolor{PineGreen}{\imath}}_{\;\! \mathcolor{gray}{z}}$ 的 $\mathcolor{gray}{\nabla_z}$:因为 {\one} 右侧的 $\Xint{\mathcolor{gray}{-}}{25}{\bar{P}}^{\;\! \mathcolor{gray}{\omega} \textcolor{PineGreen}{\imath}}_{\;\! \textcolor{Maroon}{\Yup} \mathcolor{gray}{z}}$ 是矢量的;{\two} \textcolor{PineGreen}{本征复振幅}标量场 $\Xint{\begin{smallmatrix} ~ \\ {}^{}_{\mathcolor{gray}{-}} \\ ~ \end{smallmatrix}}{09}{\mathtt{g}}^{\;\!\mathcolor{gray}{\omega} \textcolor{PineGreen}{\imath}}_{\;\! \mathcolor{gray}{z}}$ 一旦已知,再乘以\textcolor{PineGreen}{本征偏振态} $\Xint{{}^{}_{\mathcolor{gray}{-}}}{10}{\hat{g}}^{\;\!\mathcolor{gray}{\omega} \textcolor{PineGreen}{\imath}}_{\;\! \textcolor{Maroon}{\Yup}}$ 后,可直接转换为矢量\textcolor{Maroon}{时空谱}三分量 $\Xint{{}^{}_{\mathcolor{gray}{-}}}{10}{\bar{g}}^{\;\!\mathcolor{gray}{\omega} \textcolor{PineGreen}{\imath}}_{\;\! \textcolor{Maroon}{\Yup} \mathcolor{gray}{z}} := \Xint{\begin{smallmatrix} ~ \\ {}^{}_{\mathcolor{gray}{-}} \\ ~ \end{smallmatrix}}{09}{\mathtt{g}}^{\;\!\mathcolor{gray}{\omega} \textcolor{PineGreen}{\imath}}_{\;\! \mathcolor{gray}{z}} \Xint{{}^{}_{\mathcolor{gray}{-}}}{10}{\hat{g}}^{\;\!\mathcolor{gray}{\omega} \textcolor{PineGreen}{\imath}}_{\;\! \textcolor{Maroon}{\Yup}}$,或通过 \bref{eq:vec-eigenmode_amp-matrix} 一步到位至电矢量场\textcolor{Maroon}{傅立叶谱} $\Xint{\mathcolor{gray}{-}}{25}{\bar{E}}^{\;\!\mathcolor{gray}{\omega}}_{\;\! \mathcolor{gray}{z}}$。

\marginLeft[-2.4em]{ssec:SHG_spectrum}\subsection{脉冲光倍频 - 电场本征复振幅方程}\label{ssec:SHG_spectrum}

对于以\textcolor{NavyBlue}{脉冲光}\textcolor{Maroon}{倍频}\cite{boydNonlinearOptics2019}为主\Footnote{若 $\mathcolor{gray}{\omega}_{\textcolor{Maroon}{\text{P}}}, \mathcolor{gray}{\omega}$ 分别在单个\textcolor{NavyBlue}{泵浦光}脉冲\textcolor{gray}{中心频率} $\mathcolor{gray}{\Omega}_{\textcolor{Maroon}{\text{P}}} = \mathcolor{gray}{\Omega} \big/ 2$ 及其产生的 $2\mathcolor{gray}{\omega}_{\textcolor{Maroon}{\text{P}}}$ \textcolor{Maroon}{倍频}光脉冲\textcolor{gray}{中心频率} $\mathcolor{gray}{\Omega} = 2\mathcolor{gray}{\Omega}_{\textcolor{Maroon}{\text{P}}}$ 附近,且 $\mathcolor{gray}{\omega} = 2\mathcolor{gray}{\omega}_{\textcolor{Maroon}{\text{P}}} > \textcolor{gray}{0}$,则该式代表单\textcolor{NavyBlue}{脉冲光}\textcolor{Maroon}{倍频}过程。}、以\textcolor{NavyBlue}{脉冲}\textcolor{Maroon}{光整流}后续级联\textcolor{Maroon}{电光效应}\cite{jangMulticycleTerahertzPulse2020}为辅\Footnote{若 $\mathcolor{gray}{\omega}, \mathcolor{gray}{\omega}_{\textcolor{Maroon}{\text{THz}}}$ 分别在单个\textcolor{NavyBlue}{泵浦}光脉冲\textcolor{gray}{中心频率} $\mathcolor{gray}{\Omega} \gg \mathcolor{gray}{\Omega}_{\textcolor{Maroon}{\text{THz}}}$ 及其产生的 \textcolor{Maroon}{THz} 脉冲的\textcolor{gray}{中心频率} $\mathcolor{gray}{\Omega}_{\textcolor{Maroon}{\text{THz}}} \ll \mathcolor{gray}{\Omega}$ 附近,且 $\mathcolor{gray}{\omega} \gg \mathcolor{gray}{\omega}_{\textcolor{Maroon}{\text{THz}}} > \textcolor{gray}{0}$,则该式代表\textcolor{NavyBlue}{脉冲}\textcolor{Maroon}{光整流}后续级联\textcolor{Maroon}{电光效应}\cite{jangMulticycleTerahertzPulse2020}过程。对应\textcolor{NavyBlue}{脉冲电}(\textcolor{Maroon}{THz})与\textcolor{NavyBlue}{脉冲光}的\textcolor{Maroon}{和频}。}的 $\mathcolor{gray}{\omega'} + \left( \mathcolor{gray}{\omega}-\mathcolor{gray}{\omega'} \right) \to \mathcolor{gray}{\omega} > \textcolor{gray}{0}$\Footnote{在映射到\textcolor{NavyBlue}{物理过程}时,默认\textcolor{gray}{各频率}(对应 cos 余弦)为正;但在\textcolor{Plum}{数学积分}(对应 $\mathbb{e}$ 指数)中可为负。}二阶\textcolor{Plum}{非线性}\textcolor{gray}{频率}\textcolor{Maroon}{上转换}过程,波动方程 \bref{eq:simplify7-scalar-g} 与 \bref{eq:simplify8-scalar-g} \textcolor{Plum}{非线性}\textcolor{NavyBlue}{波源}项 $\Xint{\mathcolor{gray}{-}}{25}{\bar{P}}^{\;\! \mathcolor{gray}{\omega} \textcolor{PineGreen}{\imath}}_{\;\! \textcolor{Maroon}{\Yup} \mathcolor{gray}{z}} = \mathcolor{gray}{\mathcal F} \left[ \bar{P}^{\;\! \mathcolor{gray}{\omega} \textcolor{PineGreen}{\imath}}_{\;\! \textcolor{Maroon}{\Yup} \mathcolor{gray}{z}} \right]$ 进一步限定为,\textcolor{PineGreen}{本征模}\textcolor{Plum}{符号替换} $\textcolor{PineGreen}{\hat{3}},\textcolor{PineGreen}{\hat{2}},\textcolor{PineGreen}{\hat{1}} = \textcolor{PineGreen}{\imath},\textcolor{PineGreen}{\jmath},\textcolor{PineGreen}{l}$ 后的,\textcolor{PineGreen}{平面波基}下的二阶\textcolor{Plum}{局域}\textcolor{Plum}{非线性}\textcolor{NavyBlue}{电偶-$(\text{电偶}\otimes\text{电偶})$}极矩场 \bref{eq:vec-DP^(2)-plane_wave_basis-p_pp}:
\begin{subequations} \label{eq:DP^(2)-3_12-spectrum}
\begin{align}
	\Xint{\mathcolor{gray}{-}}{30}{\bar{P}}^{\;\! \mathcolor{gray}{\omega} \textcolor{Maroon}{(2)} }_{\;\! \mathcolor{gray}{z} \textcolor{PineGreen}{\hat{3}}} &\xrightarrow[\text{\bref{eq:vec-DP^(2)-plane_wave_basis-p_pp}}]{\text{$\textcolor{PineGreen}{\imath},\textcolor{PineGreen}{\jmath},\textcolor{PineGreen}{l} \to \textcolor{PineGreen}{\hat{3}},\textcolor{PineGreen}{\hat{2}},\textcolor{PineGreen}{\hat{1}}$}} \Xint{{}^{}_{\mathcolor{gray}{-}}}{23}{\bar{\bar{\bar{\chi}}}}^{\;\! \mathcolor{gray}{\omega} \textcolor{Maroon}{(2)}}_{\mathcolor{gray}{z} \textcolor{PineGreen}{\hat{3} \hat{1} \hat{2}} } ~{}^{\mathcolor{gray}{*}}_{\mathcolor{gray}{*}} \left( \Xint{\mathcolor{gray}{-}}{295}{\bar{E}}^{\;\! \mathcolor{gray}{\omega} \textcolor{PineGreen}{\hat{1}} }_{\;\! \mathcolor{gray}{z} } ~\mathcolor{gray}{\widetilde \circledast}~ \Xint{\mathcolor{gray}{-}}{295}{\bar{E}}^{\;\! \mathcolor{gray}{\omega} \textcolor{PineGreen}{\hat{2}} }_{\;\! \mathcolor{gray}{z} } \right) \label{eq:vec-DP^(2)-3_12-spectrum} \\
	\Xint{\mathcolor{gray}{-}}{30}{P}^{\;\! \textcolor{PineGreen}{\hat{3}} \mathcolor{gray}{\omega} }_{\;\! \hat{3}\mathcolor{gray}{z} \textcolor{Maroon}{(2)} } &\xrightarrow[\text{\bref{eq:components-DP^(2)-plane_wave_basis-p_pp}}]{\text{$\textcolor{PineGreen}{\imath},\textcolor{PineGreen}{\jmath},\textcolor{PineGreen}{l} \to \textcolor{PineGreen}{\hat{3}},\textcolor{PineGreen}{\hat{2}},\textcolor{PineGreen}{\hat{1}}$}} \Xint{{}^{}_{\mathcolor{gray}{-}}}{23}{\chi}^{\;\! \textcolor{PineGreen}{\hat{3}} \mathcolor{gray}{\omega} \hat{1} \hat{2} }_{\;\! \hat{3} \mathcolor{gray}{z} \textcolor{PineGreen}{\hat{1} \hat{2}} \textcolor{Maroon}{(2)}} \mathcolor{gray}{*} \left( \Xint{\mathcolor{gray}{-}}{295}{E}^{\;\! \textcolor{PineGreen}{\hat{1}} \mathcolor{gray}{\omega} }_{\;\! \hat{1} \mathcolor{gray}{z}} ~\mathcolor{gray}{\widetilde \circledast}~ \Xint{\mathcolor{gray}{-}}{295}{E}^{\;\! \textcolor{PineGreen}{\hat{2}} \mathcolor{gray}{\omega} }_{\;\! \hat{2} \mathcolor{gray}{z}} \right) ~, \label{eq:components-DP^(2)-3_12-spectrum}
\end{align}
\end{subequations}
同时,矢量\textcolor{Plum}{非线性}波动方程 \bref{eq:simplify8-scalar-g-conjugate} 也简写作
\begin{align} \label{eq:simplify8-scalar-g-modulus}
	\mathcolor{gray}{\nabla_z} \Xint{\begin{smallmatrix} ~ \\ {}^{}_{\mathcolor{gray}{-}} \\ ~ \end{smallmatrix}}{09}{\mathtt{g}}^{\;\!\mathcolor{gray}{\omega} \textcolor{PineGreen}{\hat{3}}}_{\;\! \mathcolor{gray}{z}} &\xrightarrow[\text{\bref{eq:simplify8-scalar-g-conjugate}}]{\text{$\textcolor{PineGreen}{\imath} \to \textcolor{PineGreen}{\hat{3}}$}} \mathbb{i} k_{\textcolor{Maroon}{\mathsf{o}} \mathcolor{gray}{\omega}}^{\;\! 2} \frac{\Xint{{}^{}_{\mathcolor{gray}{-}}}{10}{\hat{g}}^{\;\! \textcolor{PineGreen}{\hat{3}} \textcolor{Plum}{\dag}}_{\;\! \mathcolor{gray}{\omega}} \cdot \Xint{\mathcolor{gray}{-}}{25}{\bar{P}}^{\;\! \mathcolor{gray}{\omega} \textcolor{PineGreen}{\hat{3}} }_{\;\! \mathcolor{gray}{z}  \textcolor{Maroon}{(2)}}}{ 2 \lvert \Xint{{}^{}_{\mathcolor{gray}{-}}}{10}{\hat{g}}^{\;\! \textcolor{PineGreen}{\hat{3}}}_{\;\! \mathcolor{gray}{\omega}} \rvert^2 \Xint{\begin{smallmatrix} ~ \\ {}^{}_{\mathcolor{gray}{-}} \\ ~ \end{smallmatrix}}{15}{k}_{\;\! \symup{z}}^{\;\! \mathcolor{gray}{\omega} \textcolor{PineGreen}{\hat{3}}} \mathbb{e}^{\mathbb{i} \Xint{\begin{smallmatrix} ~ \\ {}^{}_{\mathcolor{gray}{-}} \\ ~ \end{smallmatrix}}{15}{k}_{\symup{z}}^{\;\! \mathcolor{gray}{\omega} \textcolor{PineGreen}{\hat{3}}} \mathcolor{gray}{z}}} ~.
\end{align}

二阶\textcolor{Plum}{非线性}系数 $\Xint{{}^{}_{\mathcolor{gray}{-}}}{23}{\chi}^{\;\! \mathcolor{gray}{\omega} \hat{1} \hat{2} }_{\;\! \hat{3} \mathcolor{gray}{z} \textcolor{Maroon}{(2)} }$ 总可分解为\textcolor{Plum}{均匀背景} ${\chi}^{\;\! \mathcolor{gray}{\omega} \hat{1} \hat{2} }_{\;\! \hat{3} \textcolor{Maroon}{(2)} }$ 与\textcolor{Plum}{调制函数} $\Xint{\mathcolor{gray}{-}}{18}{M}^{\;\! \mathcolor{gray}{\omega} \hat{1} \hat{2} }_{\;\! \hat{3} \mathcolor{gray}{z} \textcolor{Maroon}{(2)} }$ 之积
%\Footnote{当不存在“\textcolor{PineGreen}{模式}”\textcolor{Plum}{角标},以\textcolor{Plum}{推断}\textcolor{NavyBlue}{场量}的\textcolor{Plum}{自变量}时,不能省略 $\mathcolor{gray}{\omega}$ \textcolor{Plum}{角标}。}
\begin{subequations} \label{eq:chi2-modulate}
\begin{align}
	\Xint{{}^{}_{\mathcolor{gray}{-}}}{23}{\bar{\bar{\bar{\chi}}}}^{\;\! \mathcolor{gray}{\omega} }_{\;\! \mathcolor{gray}{z} \textcolor{Maroon}{(2)} } &= \bar{\bar{\bar{\chi}}}^{\;\! \mathcolor{gray}{\omega} }_{\;\! \textcolor{Maroon}{(2)} } \odot \Xint{\mathcolor{gray}{-}}{18}{\bar{\bar{\bar{M}}}}^{\;\! \mathcolor{gray}{\omega} }_{\;\! \mathcolor{gray}{z} \textcolor{Maroon}{(2)} } \label{eq:vec-chi2-modulate} \\
	\Xint{{}^{}_{\mathcolor{gray}{-}}}{23}{\chi}^{\;\! \mathcolor{gray}{\omega} \hat{1} \hat{2} }_{\;\! \hat{3} \mathcolor{gray}{z} \textcolor{Maroon}{(2)} } &= {\chi}^{\;\! \mathcolor{gray}{\omega} \hat{1} \hat{2} }_{\;\! \hat{3} \textcolor{Maroon}{(2)} } \Xint{\mathcolor{gray}{-}}{18}{M}^{\;\! \mathcolor{gray}{\omega} \hat{1} \hat{2} }_{\;\! \hat{3} \mathcolor{gray}{z} \textcolor{Maroon}{(2)} } ~, \label{eq:components-chi2-modulate}
\end{align}
\end{subequations}
这里隐式地定义了\textcolor{Plum}{哈达马积}/\textcolor{Plum}{对应元素积} $\odot$ 或 $"{.\cdot}"$,类似于 matlab 的 $"{.*}"$ 语法(矩阵对应元素相乘)。注意,三阶\textcolor{Plum}{调制张量}\textcolor{NavyBlue}{场} $\Xint{\mathcolor{gray}{-}}{18}{\bar{\bar{\bar{M}}}}^{\;\! \mathcolor{gray}{\omega} }_{\;\! \mathcolor{gray}{z} \textcolor{Maroon}{(2)} }$ 像 ${\chi}^{\;\! \mathcolor{gray}{\omega} }_{\;\! \textcolor{Maroon}{(2)} }$(\textcolor{NavyBlue}{非场},没有\textcolor{gray}{“$-$”标志})一样,仍是\textcolor{gray}{波长} $\mathcolor{gray}{\lambda}$ 的函数,即仍是 $\mathcolor{gray}{\omega}$ \textcolor{NavyBlue}{色散}(\textcolor{Plum}{各向异性})的。但分离出的\textcolor{Plum}{定常}\textcolor{Plum}{均匀背景}张量 ${\chi}^{\;\! \mathcolor{gray}{\omega} }_{\;\! \textcolor{Maroon}{(2)} }$ 因子,不再是 $\mathcolor{gray}{\bar{r}}$ 的函数,并且可以从许多\textcolor{NavyBlue}{实验主导}的文献中获得\cite{nyePhysicalPropertiesCrystals2012,zuOpticalSecondHarmonic2024,zuAnalyticalNumericalModeling2022,gananyQuasiphaseMatchingLiNbO32006,segondsLinearNonlinearOptical2004,dolevLinearNonlinearOptical2009,kaschkeCalculationNonlinearOptical1989,itoGeneralizedStudyAngular1975}。注意,$\Xint{{}^{}_{\mathcolor{gray}{-}}}{23}{\bar{\bar{\bar{\chi}}}}^{\;\! \mathcolor{gray}{\omega} }_{\;\! \mathcolor{gray}{z} \textcolor{Maroon}{(2)} }$ 不是\textcolor{PineGreen}{本征模} $\textcolor{PineGreen}{\hat{3}},\textcolor{PineGreen}{\hat{2}},\textcolor{PineGreen}{\hat{1}}$ 的函数 \bypertarget{chi2-free-of-eigenmodes}。

将 $\mathcolor{gray}{\bar{r}}$ 域上\textcolor{Plum}{被调制}的二阶\textcolor{Plum}{非线性}系数 \bref{eq:components-chi2-modulate} 代入\textcolor{Plum}{非线性}\textcolor{NavyBlue}{波源} \bref{eq:components-DP^(2)-3_12-spectrum} 得
\begin{subequations} \label{eq:DP^(2)-3_12-spectrum-SFG}
\begin{align}
	\Xint{\mathcolor{gray}{-}}{30}{P}^{\;\! \textcolor{PineGreen}{\hat{3}} \mathcolor{gray}{\omega} }_{\;\! \hat{3}\mathcolor{gray}{z} \textcolor{Maroon}{(2)} } &\xrightarrow[]{\text{\bref{eq:components-DP^(2)-3_12-spectrum}}} \Xint{{}^{}_{\mathcolor{gray}{-}}}{23}{\chi}^{\;\! \textcolor{PineGreen}{\hat{3}} \mathcolor{gray}{\omega} \hat{1} \hat{2} }_{\;\! \hat{3} \mathcolor{gray}{z} \textcolor{PineGreen}{\hat{1} \hat{2}} \textcolor{Maroon}{(2)}} \mathcolor{gray}{*} \left( \Xint{\mathcolor{gray}{-}}{295}{E}^{\;\!\textcolor{PineGreen}{\hat{1}} \mathcolor{gray}{\omega}}_{\;\! \hat{1} \mathcolor{gray}{z}} ~\mathcolor{gray}{\widetilde \circledast}~ \Xint{\mathcolor{gray}{-}}{295}{E}^{\;\!\textcolor{PineGreen}{\hat{2}} \mathcolor{gray}{\omega}}_{\;\! \hat{2} \mathcolor{gray}{z}} \right) \label{eq:DP^(2)-3_12-spectrum-SFG1} \\
	&\xrightarrow[]{\text{\bref{eq:components-chi2-modulate}}} {\chi}^{\;\! \textcolor{PineGreen}{\hat{3}} \mathcolor{gray}{\omega} \hat{1} \hat{2} }_{\;\! \hat{3} \textcolor{Maroon}{(2)} \textcolor{PineGreen}{\hat{1} \hat{2}}} \Xint{\mathcolor{gray}{-}}{18}{M}^{\;\! \mathcolor{gray}{\omega} \hat{1} \hat{2} }_{\;\! \hat{3} \mathcolor{gray}{z} \textcolor{Maroon}{(2)} } \mathcolor{gray}{*} \left( \Xint{\mathcolor{gray}{-}}{295}{E}^{\;\!\textcolor{PineGreen}{\hat{1}} \mathcolor{gray}{\omega}}_{\;\! \hat{1} \mathcolor{gray}{z}} ~\mathcolor{gray}{\widetilde \circledast}~ \Xint{\mathcolor{gray}{-}}{295}{E}^{\;\!\textcolor{PineGreen}{\hat{2}} \mathcolor{gray}{\omega}}_{\;\! \hat{2} \mathcolor{gray}{z}} \right) \label{eq:DP^(2)-3_12-spectrum-SFG2} \\
	&\xrightarrow[]{\text{\bref{eq:FT-krho}}} {\chi}^{\;\! \textcolor{PineGreen}{\hat{3}} \mathcolor{gray}{\omega} \hat{1} \hat{2} }_{\;\! \hat{3} \textcolor{Maroon}{(2)} \textcolor{PineGreen}{\hat{1} \hat{2}}} \mathcolor{gray}{\mathcal F} \left[ M^{\;\! \mathcolor{gray}{\omega} \hat{1} \hat{2} }_{\;\! \hat{3} \mathcolor{gray}{z} \textcolor{Maroon}{(2)} } \right] \mathcolor{gray}{*} \left( \Xint{\mathcolor{gray}{-}}{295}{E}^{\;\!\textcolor{PineGreen}{\hat{1}} \mathcolor{gray}{\omega}}_{\;\! \hat{1} \mathcolor{gray}{z}} ~\mathcolor{gray}{\widetilde \circledast}~ \Xint{\mathcolor{gray}{-}}{295}{E}^{\;\!\textcolor{PineGreen}{\hat{2}} \mathcolor{gray}{\omega}}_{\;\! \hat{2} \mathcolor{gray}{z}} \right) \label{eq:DP^(2)-3_12-spectrum-SFG3} \\
	&\xrightarrow[]{\text{\bref{eq:IFT-z}}} {\chi}^{\;\! \textcolor{PineGreen}{\hat{3}} \mathcolor{gray}{\omega} \hat{1} \hat{2} }_{\;\! \hat{3} \textcolor{Maroon}{(2)} \textcolor{PineGreen}{\hat{1} \hat{2}}} \mathcolor{gray}{\mathcal F_{z}^{-1}} \left[ \mathcolor{gray}{\mathcal F_{\bar{k}}} \left[ M^{\;\! \mathcolor{gray}{\omega} \hat{1} \hat{2} }_{\;\! \hat{3} \mathcolor{gray}{z} \textcolor{Maroon}{(2)} } \right] \right] \mathcolor{gray}{*} \left( \Xint{\mathcolor{gray}{-}}{295}{E}^{\;\!\textcolor{PineGreen}{\hat{1}} \mathcolor{gray}{\omega}}_{\;\! \hat{1} \mathcolor{gray}{z}} ~\mathcolor{gray}{\widetilde \circledast}~ \Xint{\mathcolor{gray}{-}}{295}{E}^{\;\!\textcolor{PineGreen}{\hat{2}} \mathcolor{gray}{\omega}}_{\;\! \hat{2} \mathcolor{gray}{z}} \right) \label{eq:DP^(2)-3_12-spectrum-SFG4} \\
	&= {\chi}^{\;\! \textcolor{PineGreen}{\hat{3}} \mathcolor{gray}{\omega} \hat{1} \hat{2} }_{\;\! \hat{3} \textcolor{Maroon}{(2)} \textcolor{PineGreen}{\hat{1} \hat{2}}} \mathcolor{gray}{\mathcal F_{z}^{-1}} \left[ \mathcolor{gray}{\mathcal F_{\bar{k}}} \left[ M^{\;\! \mathcolor{gray}{\omega} \hat{1} \hat{2} }_{\;\! \hat{3} \mathcolor{gray}{z} \textcolor{Maroon}{(2)} } \right] \mathcolor{gray}{*} \left( \Xint{\mathcolor{gray}{-}}{295}{E}^{\;\!\textcolor{PineGreen}{\hat{1}} \mathcolor{gray}{\omega}}_{\;\! \hat{1} \mathcolor{gray}{z}} ~\mathcolor{gray}{\widetilde \circledast}~ \Xint{\mathcolor{gray}{-}}{295}{E}^{\;\!\textcolor{PineGreen}{\hat{2}} \mathcolor{gray}{\omega}}_{\;\! \hat{2} \mathcolor{gray}{z}} \right) \right] \label{eq:DP^(2)-3_12-spectrum-SFG5} \\
	&\xrightarrow[]{\text{\bref{eq:components-C}}}: {\chi}^{\;\! \textcolor{PineGreen}{\hat{3}} \mathcolor{gray}{\omega} \hat{1} \hat{2} }_{\;\! \hat{3} \textcolor{Maroon}{(2)} \textcolor{PineGreen}{\hat{1} \hat{2}}} \mathcolor{gray}{\mathcal F_{z}^{-1}} \left[ \Xint{\mathcolor{gray}{-}}{18}{M}^{\;\! \mathcolor{gray}{\omega} \hat{1} \hat{2} }_{\;\! \hat{3} \mathcolor{gray}{k_{\symup{z}}} \textcolor{Maroon}{(2)} } \mathcolor{gray}{*} \left( \Xint{\mathcolor{gray}{-}}{295}{E}^{\;\!\textcolor{PineGreen}{\hat{1}} \mathcolor{gray}{\omega}}_{\;\! \hat{1} \mathcolor{gray}{z}} ~\mathcolor{gray}{\widetilde \circledast}~ \Xint{\mathcolor{gray}{-}}{295}{E}^{\;\!\textcolor{PineGreen}{\hat{2}} \mathcolor{gray}{\omega}}_{\;\! \hat{2} \mathcolor{gray}{z}} \right) \right] ~, \label{eq:DP^(2)-3_12-spectrum-SFG6}
\end{align}
\end{subequations}
其中,$\Xint{{}^{}_{\mathcolor{gray}{-}}}{23}{\chi}^{\;\! \textcolor{PineGreen}{\hat{3}} \mathcolor{gray}{\omega} \hat{1} \hat{2} }_{\;\! \hat{3} \mathcolor{gray}{z} \textcolor{PineGreen}{\hat{1} \hat{2}} \textcolor{Maroon}{(2)}}$ 的值,与\textcolor{PineGreen}{本征模} $\textcolor{PineGreen}{\hat{3}},\textcolor{PineGreen}{\hat{2}},\textcolor{PineGreen}{\hat{1}}$ 无关。$\textcolor{PineGreen}{\hat{3}},\textcolor{PineGreen}{\hat{2}},\textcolor{PineGreen}{\hat{1}}$ 在其中,只是帮助 $\Xint{{}^{}_{\mathcolor{gray}{-}}}{23}{\chi}^{\;\! \textcolor{PineGreen}{\hat{3}} \mathcolor{gray}{\omega} \hat{1} \hat{2} }_{\;\! \hat{3} \mathcolor{gray}{z} \textcolor{PineGreen}{\hat{1} \hat{2}} \textcolor{Maroon}{(2)}}$ 与 $\Xint{\mathcolor{gray}{-}}{25}{E}^{\;\!\textcolor{PineGreen}{\hat{1}} \mathcolor{gray}{\omega}}_{\;\! \hat{1} \mathcolor{gray}{z}}, \Xint{\mathcolor{gray}{-}}{25}{E}^{\;\!\textcolor{PineGreen}{\hat{2}} \mathcolor{gray}{\omega}}_{\;\! \hat{2} \mathcolor{gray}{z}}$ 一起,起到\textcolor{Plum}{爱因斯坦求和}的作用。

此外,\bref{eq:DP^(2)-3_12-spectrum-SFG6} 中,定义了三维 $\mathcolor{gray}{\bar{k}}$ \textcolor{gray}{空间}的\textcolor{NavyBlue}{倒格波系数}(关于 $\mathcolor{gray}{\bar{k}} \asymp \left( \mathcolor{gray}{\bar{k}_{\symup{\rho}}}, \mathcolor{gray}{k_{\symup{z}}} \right)$ 的三阶\textcolor{Plum}{张量}\textcolor{NavyBlue}{场})
\begin{subequations} \label{eq:C}
\begin{align}
	\Xint{\mathcolor{gray}{-}}{18}{M}^{\;\! \mathcolor{gray}{\omega} \hat{1} \hat{2} }_{\;\! \hat{3} \mathcolor{gray}{k_{\symup{z}}} \textcolor{Maroon}{(2)} } &:\xleftarrow[]{\text{\bref{eq:FT-k}}} \mathcolor{gray}{\mathcal F_{\bar{k}}} \left[ M^{\;\! \mathcolor{gray}{\omega} \hat{1} \hat{2} }_{\;\! \hat{3} \mathcolor{gray}{z} \textcolor{Maroon}{(2)} } \right] \label{eq:components-C} \\
	\Xint{\mathcolor{gray}{-}}{18}{\bar{\bar{\bar{M}}}}^{\;\! \mathcolor{gray}{\omega} }_{\;\! \mathcolor{gray}{k_{\symup{z}}} \textcolor{Maroon}{(2)} } &:\xleftarrow[]{\text{\bref{eq:FT-k}}} \mathcolor{gray}{\mathcal F_{\bar{k}}} \left[ \bar{\bar{\bar{M}}}^{\;\! \mathcolor{gray}{\omega} }_{\;\! \mathcolor{gray}{z} \textcolor{Maroon}{(2)} } \right] ~, \label{eq:vec-C}
\end{align}
\end{subequations}
其中,3 维空域 $\mathcolor{gray}{\bar{r}} \in \mathcolor{gray}{\bar{\mathbb{R}}_{\textcolor{Plum}{3}}}$ 中的\textcolor{Plum}{傅立叶正变换} $\mathcolor{gray}{\mathcal F_{\bar{k}}}$ 来自 \bref{eq:FT-k}。

在 \bref{eq:DP^(2)-3_12-spectrum-SFG4} 中,还定义了空域 $\mathcolor{gray}{z} \in \mathcolor{gray}{\bar{\mathbb{R}}_{\textcolor{Plum}{1}}}$ 向 1 维\textcolor{Plum}{傅立叶正} $\mathcolor{gray}{\mathcal F_{z}}$、\textcolor{Plum}{逆} $\mathcolor{gray}{\mathcal F_{z}^{-1}}$ \textcolor{Plum}{变换对}
\begin{subequations} \label{eq:FT-z_kz}
\begin{align}
	\mathcolor{gray}{\mathcal F_{z}} \left[ \cdot \right] &:= \frac{ 1 }{ 2\symup{\pi} } \mathcolor{gray}{\int_{-\infty}^{+\infty}} \cdot~ \mathbb{e}^{-\mathbb{i}\mathcolor{gray}{k_{\symup{z}}} \mathcolor{gray}{z}} \mathbb{d}\mathcolor{gray}{z} ~, \label{eq:FT-kz} \\
	\mathcolor{gray}{\mathcal F_{z}^{-1}} \left[ \cdot \right] &:= \hphantom{\frac{ 1 }{ 2\symup{\pi} }} \mathcolor{gray}{\int_{-\infty}^{+\infty}} \cdot~ \mathbb{e}^{\mathbb{i}\mathcolor{gray}{k_{\symup{z}}} \mathcolor{gray}{z}} \hphantom{^-} \mathbb{d}\mathcolor{gray}{k_{\symup{z}}} ~. \label{eq:IFT-z}
\end{align}
\end{subequations}

利用\textcolor{PineGreen}{本征波}的第 4 种定义 \bref{eq:vec-eigenwave'} 的\textcolor{PineGreen}{本征偏振态} $\Xint{{}^{}_{\mathcolor{gray}{-}}}{10}{\bar{g}}^{\;\! \mathcolor{gray}{\omega} \textcolor{PineGreen}{\hat{\jmath}}}$ \textcolor{Plum}{复归一化}版 $\Xint{\mathcolor{gray}{-}}{25}{\bar{E}}^{\;\! \mathcolor{gray}{\omega} \textcolor{PineGreen}{\hat{\jmath}}}_{\;\! \mathcolor{gray}{z}} := \Xint{\mathcolor{gray}{-}}{16}{\mathtt{G}}^{\;\! \mathcolor{gray}{\omega} \textcolor{PineGreen}{\hat{\jmath}}}_{\;\! \mathcolor{gray}{z}} \Xint{{}^{}_{\mathcolor{gray}{-}}}{10}{\hat{g}}^{\;\! \mathcolor{gray}{\omega} \textcolor{PineGreen}{\hat{\jmath}} }$,将 \bref{eq:DP^(2)-3_12-spectrum-SFG6} 分离出\textcolor{PineGreen}{含衍射本征复振幅} $\Xint{\mathcolor{gray}{-}}{16}{\mathtt{G}}^{\;\!\mathcolor{gray}{\omega} \textcolor{PineGreen}{\hat{\jmath}}}_{\;\! \mathcolor{gray}{z}}$(\bref{eq:amp_phase})和\textcolor{Plum}{复归一化}\textcolor{PineGreen}{本征偏振态} $\Xint{{}^{}_{\mathcolor{gray}{-}}}{10}{\hat{g}}^{\;\!\mathcolor{gray}{\omega}}_{\;\! \textcolor{PineGreen}{\hat{\jmath}}}$,并将 \bref{eq:DP^(2)-3_12-spectrum-SFG6}(的系数张量 $\Xint{\mathcolor{gray}{-}}{18}{M}^{\;\! \mathcolor{gray}{\omega} \hat{1} \hat{2} }_{\;\! \hat{3} \mathcolor{gray}{k_{\symup{z}}} \textcolor{Maroon}{(2)} }$)升级为“\textcolor{Plum}{半张量式}”$\Xint{\mathcolor{gray}{-}}{18}{\bar{M}}^{\;\! \mathcolor{gray}{\omega} \hat{1} \hat{2} }_{\;\! \mathcolor{gray}{k_{\symup{z}}} \textcolor{Maroon}{(2)} }$\Footnote{注意,{\one} $\Xint{{}^{}_{\mathcolor{gray}{-}}}{10}{\hat{g}}^{\;\! \mathcolor{gray}{\omega} \textcolor{PineGreen}{\hat{1}}}_{\;\! \hat{1}}$ 是矢量 $\Xint{{}^{}_{\mathcolor{gray}{-}}}{10}{\hat{g}}^{\;\! \mathcolor{gray}{\omega} }_{\;\! \textcolor{PineGreen}{\hat{1}}}$ 在 $\hat{1}$ 方向的分量,即标量;{\two} 对 $\mathcolor{gray}{\widetilde \circledast}$ 的运算\textcolor{Plum}{优先级},整体来说,既不高于,也不低于,也不等于对 $\mathcolor{gray}{*}$ 的\textcolor{Plum}{优先级}:需要拆分后,才能谈\textcolor{Plum}{优先级},见下文 \bref{ssec:scalar} 中 \bref{eq:scalar_nonlinear_drive2} 的下一段。}
\begin{subequations} \label{eq:DP^(2)-3_12-spectrum-G}
\begin{align}
	\Xint{\mathcolor{gray}{-}}{30}{\bar{P}}^{\;\! \mathcolor{gray}{\omega} \textcolor{PineGreen}{\hat{3}} }_{\;\! \mathcolor{gray}{z} \textcolor{Maroon}{(2)} } &\xrightarrow[]{\text{\bref{eq:DP^(2)-3_12-spectrum-SFG6}}} \bar{\chi}^{\;\! \mathcolor{gray}{\omega} \textcolor{PineGreen}{\hat{3}} \hat{1} \hat{2} }_{\;\! \textcolor{Maroon}{(2)} \textcolor{PineGreen}{\hat{1} \hat{2}}} \odot \mathcolor{gray}{\mathcal F_{z}^{-1}} \left[ \Xint{\mathcolor{gray}{-}}{18}{\bar{M}}^{\;\! \mathcolor{gray}{\omega} \hat{1} \hat{2} }_{\;\! \mathcolor{gray}{k_{\symup{z}}} \textcolor{Maroon}{(2)} } \mathcolor{gray}{*} \Xint{\mathcolor{gray}{-}}{295}{E}^{\;\! \mathcolor{gray}{\omega} \textcolor{PineGreen}{\hat{1}}}_{\;\! \hat{1} \mathcolor{gray}{z}} ~\mathcolor{gray}{\widetilde \circledast}~ \Xint{\mathcolor{gray}{-}}{295}{E}^{\;\! \mathcolor{gray}{\omega} \textcolor{PineGreen}{\hat{2}}}_{\;\! \hat{2} \mathcolor{gray}{z}} \right] \label{eq:DP^(2)-3_12-spectrum-G1} \\
	&\xrightarrow[]{\text{$\sim$ \bref{eq:components-eigenwave'}}} \bar{\chi}^{\;\! \mathcolor{gray}{\omega} \textcolor{PineGreen}{\hat{3}} \hat{1} \hat{2} }_{\;\! \textcolor{Maroon}{(2)} \textcolor{PineGreen}{\hat{1} \hat{2}}} \odot \mathcolor{gray}{\mathcal F_{z}^{-1}} \left[ \Xint{\mathcolor{gray}{-}}{18}{\bar{M}}^{\;\! \mathcolor{gray}{\omega} \hat{1} \hat{2} }_{\;\! \mathcolor{gray}{k_{\symup{z}}} \textcolor{Maroon}{(2)} } \mathcolor{gray}{*} \left( \Xint{\mathcolor{gray}{-}}{20}{\mathtt{G}}^{\;\! \mathcolor{gray}{\omega} \textcolor{PineGreen}{\hat{1}}}_{\;\! \mathcolor{gray}{z}} \Xint{{}^{}_{\mathcolor{gray}{-}}}{10}{\hat{g}}^{\;\! \mathcolor{gray}{\omega} \textcolor{PineGreen}{\hat{1}}}_{\;\! \hat{1}} \right) \mathcolor{gray}{\widetilde \circledast} \left( \Xint{\mathcolor{gray}{-}}{20}{\mathtt{G}}^{\;\! \mathcolor{gray}{\omega} \textcolor{PineGreen}{\hat{2}}}_{\;\! \mathcolor{gray}{z}} \Xint{{}^{}_{\mathcolor{gray}{-}}}{10}{\hat{g}}^{\;\! \mathcolor{gray}{\omega} \textcolor{PineGreen}{\hat{2}}}_{\;\! \hat{2}} \right) \right] ~, \label{eq:DP^(2)-3_12-spectrum-G2}
\end{align}
\end{subequations}
该版本的\textcolor{Plum}{非线性}\textcolor{NavyBlue}{波源},保留了 \bref{eq:vec-DP^(2)-3_12-spectrum} 左侧的 $\Xint{\mathcolor{gray}{-}}{25}{\bar{P}}^{\;\! \mathcolor{gray}{\omega} \textcolor{PineGreen}{\hat{3}}}_{\;\! \textcolor{Maroon}{\Yup} \mathcolor{gray}{z}}$ 的矢量形式,以便直接代入 \bref{eq:simplify8-scalar-g-modulus};又丢弃了 \bref{eq:vec-DP^(2)-3_12-spectrum} 右侧的 ${}^{\mathcolor{gray}{*}}_{\mathcolor{gray}{*}}$,并采用了 \bref{eq:components-DP^(2)-3_12-spectrum} 右侧的 $\mathcolor{gray}{*}$;--- 这样做的代价便是 \bref{eq:DP^(2)-3_12-spectrum-G} 中的二阶\textcolor{Plum}{非线性}系数张量,既从完整的三阶张量 $\Xint{{}^{}_{\mathcolor{gray}{-}}}{23}{\bar{\bar{\bar{\chi}}}}^{\;\! \mathcolor{gray}{\omega} }_{\;\! \mathcolor{gray}{z} \textcolor{Maroon}{(2)} }$ 降阶(\bref{eq:vec-chi2-modulate}),又从零阶张量元 $\Xint{{}^{}_{\mathcolor{gray}{-}}}{23}{\chi}^{\;\! \mathcolor{gray}{\omega} \hat{1} \hat{2} }_{\;\! \hat{3} \mathcolor{gray}{z} \textcolor{Maroon}{(2)} }$ 升阶(\bref{eq:components-chi2-modulate}),至同阶于 \bref{eq:DP^(2)-3_12-spectrum-G} 左侧 $\Xint{\mathcolor{gray}{-}}{25}{\bar{P}}^{\;\! \mathcolor{gray}{\omega} \textcolor{PineGreen}{\pm}}_{\;\! \textcolor{Maroon}{\Yup} \mathcolor{gray}{z}}$ 的一阶张量 $\Xint{{}^{}_{\mathcolor{gray}{-}}}{23}{\bar{\chi}}^{\;\! \mathcolor{gray}{\omega} \hat{1} \hat{2} }_{\;\! \mathcolor{gray}{z} \textcolor{Maroon}{(2)} }$ 的矢量形式。--- 这便体现了 \bref{hook:0bar,hook:1bar,hook:2bar,hook:3bar} 中对“划上线 line up”制度的前置顶层设计,以表达不同阶张量的优势。算符的运算顺序见 \byperref{OperatorSequence}{后处}。

将\textcolor{Plum}{替换后的}矢量\textcolor{Plum}{非线性}\textcolor{NavyBlue}{波源}项 $\Xint{\mathcolor{gray}{-}}{25}{\bar{P}}^{\;\! \mathcolor{gray}{\omega} \textcolor{PineGreen}{\hat{3}} }_{\;\! \mathcolor{gray}{z} \textcolor{Maroon}{(2)} }$ 整体,即 \bref{eq:DP^(2)-3_12-spectrum-G1},代入 \bref{eq:simplify8-scalar-g-modulus},即得以\textcolor{NavyBlue}{脉冲光}\textcolor{Maroon}{倍频}\cite{boydNonlinearOptics2019}、\textcolor{NavyBlue}{脉冲}\textcolor{Maroon}{光整流}后的级联\textcolor{Maroon}{电光效应}\cite{jangMulticycleTerahertzPulse2020}等过程为代表的电场\textcolor{PineGreen}{本征复振幅}方程
\begin{align} \label{eq:simplify8-scalar-g-modulus-P-spectrum}
	\mathcolor{gray}{\nabla_z} \Xint{\begin{smallmatrix} ~ \\ {}^{}_{\mathcolor{gray}{-}} \\ ~ \end{smallmatrix}}{09}{\mathtt{g}}^{\;\!\mathcolor{gray}{\omega} \textcolor{PineGreen}{\hat{3}}}_{\;\! \mathcolor{gray}{z}} &\xrightarrow[\text{\bref{eq:simplify8-scalar-g-modulus}}]{\text{\bref{eq:DP^(2)-3_12-spectrum-G1}}} \mathbb{i} k_{\textcolor{Maroon}{\mathsf{o}} \mathcolor{gray}{\omega}}^{\;\! 2} \frac{\Xint{{}^{}_{\mathcolor{gray}{-}}}{10}{\hat{g}}^{\;\! \textcolor{PineGreen}{\hat{3}} \textcolor{Plum}{\dag}}_{\;\! \mathcolor{gray}{\omega}} \cdot \bar{\chi}^{\;\! \mathcolor{gray}{\omega} \textcolor{PineGreen}{\hat{3}} \hat{1} \hat{2} }_{\;\! \textcolor{Maroon}{(2)} \textcolor{PineGreen}{\hat{1} \hat{2}}} \odot \mathcolor{gray}{\mathcal F_{z}^{-1}} \left[ \Xint{\mathcolor{gray}{-}}{18}{\bar{M}}^{\;\! \mathcolor{gray}{\omega} \hat{1} \hat{2} }_{\;\! \mathcolor{gray}{k_{\symup{z}}} \textcolor{Maroon}{(2)} } \mathcolor{gray}{*} \Xint{\mathcolor{gray}{-}}{25}{E}^{\;\! \mathcolor{gray}{\omega} \textcolor{PineGreen}{\hat{1}}}_{\;\! \hat{1} \mathcolor{gray}{z}} ~\mathcolor{gray}{\widetilde \circledast}~ \Xint{\mathcolor{gray}{-}}{25}{E}^{\;\! \mathcolor{gray}{\omega} \textcolor{PineGreen}{\hat{2}}}_{\;\! \hat{2} \mathcolor{gray}{z}} \right]}{ 2 \lvert \Xint{{}^{}_{\mathcolor{gray}{-}}}{10}{\hat{g}}^{\;\! \textcolor{PineGreen}{\hat{3}}}_{\;\! \mathcolor{gray}{\omega}} \rvert^2 \Xint{\begin{smallmatrix} ~ \\ {}^{}_{\mathcolor{gray}{-}} \\ ~ \end{smallmatrix}}{15}{k}_{\;\! \symup{z}}^{\;\! \mathcolor{gray}{\omega} \textcolor{PineGreen}{\hat{3}}} \mathbb{e}^{\mathbb{i} \Xint{\begin{smallmatrix} ~ \\ {}^{}_{\mathcolor{gray}{-}} \\ ~ \end{smallmatrix}}{15}{k}_{\symup{z}}^{\;\! \mathcolor{gray}{\omega} \textcolor{PineGreen}{\hat{3}}} \mathcolor{gray}{z}}} ~, 
\end{align}
注意,\textcolor{Plum}{哈达马积} $\odot$ 的运算\textcolor{Plum}{优先级}恒高于\textcolor{Plum}{点积} $\cdot$(以省略一对\textcolor{Plum}{小括号})。

\marginLeft[-2.4em]{ssec:SFG_discrete}\subsection{连续光和频 - 电场本征复振幅方程}\label{ssec:SFG_discrete}

对于\textcolor{NavyBlue}{非脉冲}/\textcolor{NavyBlue}{非连续谱},而是两个\textcolor{Plum}{独立}、\textcolor{Plum}{离散}、\textcolor{NavyBlue}{单色}\textcolor{gray}{波长}的\textcolor{Maroon}{和频}或\textcolor{Maroon}{上转换}\Footnote{尽管\textcolor{NavyBlue}{双泵浦}的\textcolor{NavyBlue}{光强}可能不大,这里仍不说“\textcolor{Maroon}{上转换}”:因为在本文的语境中,“\textcolor{Maroon}{上转换}”过程一般是“\textcolor{NavyBlue}{一强一弱}”\textcolor{NavyBlue}{双泵浦}生成\textcolor{NavyBlue}{弱} $\mathcolor{gray}{\omega}_{\textcolor{gray}{3}}$,以至于参与\textcolor{gray}{混频}的三波中,有两束\textcolor{NavyBlue}{弱光}(一入一出)不满足\textcolor{Maroon}{泵浦未耗尽近似}条件,因此只要有“\textcolor{Maroon}{上转换}”则必有“\textcolor{Maroon}{下转换}”过程发生(\textcolor{NavyBlue}{能量}从 $\mathcolor{gray}{\omega}_{\textcolor{gray}{3}}$ \textcolor{NavyBlue}{回流}到其中一个\textcolor{NavyBlue}{弱泵浦}中),于是不可避免地涉及\textcolor{NavyBlue}{三波混频}\textcolor{Maroon}{时空谱}耦合波方程组中的至少 2 个方程,然而这里只给出了 1 个“\textcolor{Maroon}{上转换}”过程的方程,因此这里只能代表/指\textcolor{Maroon}{和频}过程。}出\textcolor{gray}{第三个波长}的 $\mathcolor{gray}{\omega}_{\textcolor{gray}{1}} + \mathcolor{gray}{\omega}_{\textcolor{gray}{2}} \to \mathcolor{gray}{\omega}_{\textcolor{gray}{3}}$ 过程,即纯\textcolor{NavyBlue}{(准)连续光}\textcolor{gray}{混频}的特例,\bref{eq:DP^(2)-3_12-spectrum} 变为
\begin{subequations} \label{eq:DP^(2)-3_12-discrete}
	\begin{align}
		\Xint{\mathcolor{gray}{-}}{30}{\bar{P}}^{\;\! \textcolor{Maroon}{(2)} }_{\;\! \mathcolor{gray}{z} \textcolor{PineGreen}{\hat{3}}} &= \Xint{{}^{}_{\mathcolor{gray}{-}}}{23}{\bar{\bar{\bar{\chi}}}}^{\;\!  \textcolor{Maroon}{(2)}}_{\mathcolor{gray}{z} \textcolor{PineGreen}{\hat{3} \hat{1} \hat{2}} } ~{}^{\mathcolor{gray}{*}}_{\mathcolor{gray}{*}} \left( \Xint{\mathcolor{gray}{-}}{295}{\bar{E}}^{\;\! \textcolor{PineGreen}{\hat{1}} }_{\;\! \mathcolor{gray}{z} } \mathcolor{gray}{*} \Xint{\mathcolor{gray}{-}}{295}{\bar{E}}^{\;\! \textcolor{PineGreen}{\hat{2}} }_{\;\! \mathcolor{gray}{z} } \right) \label{eq:vec-DP^(2)-3_12-discrete} \\
		\Xint{\mathcolor{gray}{-}}{30}{P}^{\;\! \textcolor{PineGreen}{\hat{3}} \textcolor{Maroon}{(2)} }_{\;\! \hat{3}\mathcolor{gray}{z}} &= \Xint{{}^{}_{\mathcolor{gray}{-}}}{23}{\chi}^{\;\! \textcolor{PineGreen}{\hat{3}} \textcolor{Maroon}{(2)} \hat{1} \hat{2} }_{\;\! \hat{3} \mathcolor{gray}{z} \textcolor{PineGreen}{\hat{1} \hat{2}}} \mathcolor{gray}{*} \left( \Xint{\mathcolor{gray}{-}}{295}{E}^{\;\!\textcolor{PineGreen}{\hat{1}}}_{\;\! \hat{1} \mathcolor{gray}{z}} \mathcolor{gray}{*} \Xint{\mathcolor{gray}{-}}{295}{E}^{\;\!\textcolor{PineGreen}{\hat{2}}}_{\;\! \hat{2} \mathcolor{gray}{z}} \right) ~. \label{eq:components-DP^(2)-3_12-discrete}
	\end{align}
\end{subequations}
其中,每个场量都\textcolor{Plum}{未显含}\textcolor{gray}{角频率} $\mathcolor{gray}{\omega}$,但可以\textcolor{Plum}{推断}出来它们运行在 $\mathcolor{gray}{\omega}$ 域:因为\textcolor{PineGreen}{模式} $\textcolor{PineGreen}{\hat{3}},\textcolor{PineGreen}{\hat{2}},\textcolor{PineGreen}{\hat{1}}$ 只存在于 $\mathcolor{gray}{\omega}~ (, \mathcolor{gray}{\bar{k}_{\symup{\rho}}})$ 域,在时间 $\mathcolor{gray}{t}~ (, \mathcolor{gray}{\bar{k}_{\symup{\rho}}})$ 域内没有“\textcolor{PineGreen}{模式}”这一说法。这样表示,是在最大程度\textcolor{Plum}{省略符号}的同时\textcolor{Plum}{保留全信息}。

将 \bref{eq:components-chi2-modulate} 的\textcolor{NavyBlue}{(准)连续光}/\textcolor{NavyBlue}{离散谱}版本,代入\textcolor{Plum}{非线性}\textcolor{NavyBlue}{波源} \bref{eq:components-DP^(2)-3_12-discrete} 得
\begin{subequations} \label{eq:DP^(2)-3_12-discrete-SFG}
\begin{align}
	\Xint{\mathcolor{gray}{-}}{30}{P}^{\;\! \textcolor{PineGreen}{\hat{3}} \textcolor{Maroon}{(2)} }_{\;\! \hat{3}\mathcolor{gray}{z}} &\xrightarrow[]{\text{\bref{eq:components-DP^(2)-3_12-discrete}}} \Xint{{}^{}_{\mathcolor{gray}{-}}}{23}{\chi}^{\;\! \textcolor{PineGreen}{\hat{3}} \textcolor{Maroon}{(2)} \hat{1} \hat{2} }_{\;\! \hat{3} \mathcolor{gray}{z} \textcolor{PineGreen}{\hat{1} \hat{2}}} \mathcolor{gray}{*} \left( \Xint{\mathcolor{gray}{-}}{295}{E}^{\;\! \textcolor{PineGreen}{\hat{1}}}_{\;\! \hat{1} \mathcolor{gray}{z}} \mathcolor{gray}{*} \Xint{\mathcolor{gray}{-}}{295}{E}^{\;\! \textcolor{PineGreen}{\hat{2}}}_{\;\! \hat{2} \mathcolor{gray}{z}} \right) \label{eq:DP^(2)-3_12-discrete-SFG1} \\
	&\xrightarrow[]{\text{$\sim$\bref{eq:components-chi2-modulate}}} {\chi}^{\;\! \textcolor{PineGreen}{\hat{3}} \hat{1} \hat{2} }_{\;\! \hat{3} \textcolor{PineGreen}{\hat{1} \hat{2}} \textcolor{Maroon}{(2)}} \Xint{\mathcolor{gray}{-}}{18}{M}^{\;\! \mathcolor{gray}{3} \hat{1} \hat{2} }_{\;\! \hat{3} \mathcolor{gray}{z} \textcolor{Maroon}{(2)} } \mathcolor{gray}{*} \left( \Xint{\mathcolor{gray}{-}}{295}{E}^{\;\! \textcolor{PineGreen}{\hat{1}}}_{\;\! \hat{1} \mathcolor{gray}{z}} \mathcolor{gray}{*} \Xint{\mathcolor{gray}{-}}{295}{E}^{\;\! \textcolor{PineGreen}{\hat{2}}}_{\;\! \hat{2} \mathcolor{gray}{z}} \right) \label{eq:DP^(2)-3_12-discrete-SFG2} \\
	&\xrightarrow[]{\text{\bref{eq:FT-krho}}} {\chi}^{\;\! \textcolor{PineGreen}{\hat{3}} \hat{1} \hat{2} }_{\;\! \hat{3} \textcolor{PineGreen}{\hat{1} \hat{2}} \textcolor{Maroon}{(2)}} \mathcolor{gray}{\mathcal F} \left[ M^{\;\! \mathcolor{gray}{3} \hat{1} \hat{2} }_{\;\! \hat{3} \mathcolor{gray}{z} \textcolor{Maroon}{(2)} } \right] \mathcolor{gray}{*} \left( \Xint{\mathcolor{gray}{-}}{295}{E}^{\;\! \textcolor{PineGreen}{\hat{1}}}_{\;\! \hat{1} \mathcolor{gray}{z}} \mathcolor{gray}{*} \Xint{\mathcolor{gray}{-}}{295}{E}^{\;\! \textcolor{PineGreen}{\hat{2}}}_{\;\! \hat{2} \mathcolor{gray}{z}} \right) \label{eq:DP^(2)-3_12-discrete-SFG3} \\
	&\xrightarrow[]{\text{\bref{eq:IFT-z}}} {\chi}^{\;\! \textcolor{PineGreen}{\hat{3}} \hat{1} \hat{2} }_{\;\! \hat{3} \textcolor{PineGreen}{\hat{1} \hat{2}} \textcolor{Maroon}{(2)}} \mathcolor{gray}{\mathcal F_{z}^{-1}} \left[ \mathcolor{gray}{\mathcal F_{\bar{k}}} \left[ M^{\;\! \mathcolor{gray}{3} \hat{1} \hat{2} }_{\;\! \hat{3} \mathcolor{gray}{z} \textcolor{Maroon}{(2)} } \right] \right] \mathcolor{gray}{*} \left( \Xint{\mathcolor{gray}{-}}{295}{E}^{\;\! \textcolor{PineGreen}{\hat{1}}}_{\;\! \hat{1} \mathcolor{gray}{z}} \mathcolor{gray}{*} \Xint{\mathcolor{gray}{-}}{295}{E}^{\;\! \textcolor{PineGreen}{\hat{2}}}_{\;\! \hat{2} \mathcolor{gray}{z}} \right) \label{eq:DP^(2)-3_12-discrete-SFG4} \\
	&= {\chi}^{\;\! \textcolor{PineGreen}{\hat{3}} \hat{1} \hat{2} }_{\;\! \hat{3} \textcolor{PineGreen}{\hat{1} \hat{2}} \textcolor{Maroon}{(2)}} \mathcolor{gray}{\mathcal F_{z}^{-1}} \left[ \mathcolor{gray}{\mathcal F_{\bar{k}}} \left[ M^{\;\! \mathcolor{gray}{3} \hat{1} \hat{2} }_{\;\! \hat{3} \mathcolor{gray}{z} \textcolor{Maroon}{(2)} } \right] \mathcolor{gray}{*} \left( \Xint{\mathcolor{gray}{-}}{295}{E}^{\;\! \textcolor{PineGreen}{\hat{1}}}_{\;\! \hat{1} \mathcolor{gray}{z}} \mathcolor{gray}{*} \Xint{\mathcolor{gray}{-}}{295}{E}^{\;\! \textcolor{PineGreen}{\hat{2}}}_{\;\! \hat{2} \mathcolor{gray}{z}} \right) \right] \label{eq:DP^(2)-3_12-discrete-SFG5} \\
	&\xrightarrow[]{\text{$\sim$\bref{eq:components-C}}} {\chi}^{\;\! \textcolor{PineGreen}{\hat{3}} \hat{1} \hat{2} }_{\;\! \hat{3} \textcolor{PineGreen}{\hat{1} \hat{2}} \textcolor{Maroon}{(2)}} \mathcolor{gray}{\mathcal F_{z}^{-1}} \left[ \Xint{\mathcolor{gray}{-}}{18}{M}^{\;\! \mathcolor{gray}{3} \hat{1} \hat{2} }_{\;\! \hat{3} \mathcolor{gray}{k_{\symup{z}}} \textcolor{Maroon}{(2)} } \mathcolor{gray}{*} \left( \Xint{\mathcolor{gray}{-}}{295}{E}^{\;\! \textcolor{PineGreen}{\hat{1}}}_{\;\! \hat{1} \mathcolor{gray}{z}} \mathcolor{gray}{*} \Xint{\mathcolor{gray}{-}}{295}{E}^{\;\! \textcolor{PineGreen}{\hat{2}}}_{\;\! \hat{2} \mathcolor{gray}{z}} \right) \right] ~, \label{eq:DP^(2)-3_12-discrete-SFG6}
\end{align}
\end{subequations}
其中,$\Xint{\mathcolor{gray}{-}}{18}{M}^{\;\! \mathcolor{gray}{3} \hat{1} \hat{2} }_{\;\! \hat{3} \mathcolor{gray}{z} \textcolor{Maroon}{(2)} }$ 中的 \textcolor{gray}{灰色数字 3} 表示 $\mathcolor{gray}{\omega}_{\textcolor{gray}{3}}$。

将 \bref{eq:DP^(2)-3_12-discrete-SFG1} 中,由\textcolor{Plum}{分量形式}的 $\Xint{\mathcolor{gray}{-}}{18}{M}^{\;\! \mathcolor{gray}{3} \hat{1} \hat{2} }_{\;\! \hat{3} \mathcolor{gray}{k_{\symup{z}}} \textcolor{Maroon}{(2)} }$ 表示的\textcolor{Plum}{标量形式}的\textcolor{Plum}{非线性}\textcolor{NavyBlue}{波源}项 $\Xint{\mathcolor{gray}{-}}{25}{P}^{\;\! \textcolor{PineGreen}{\hat{3}} \textcolor{Maroon}{(2)} }_{\;\! \hat{3}\mathcolor{gray}{z}}$,升级为由\textcolor{Plum}{半张量形式}的 $\Xint{\mathcolor{gray}{-}}{18}{\bar{M}}^{\;\! \mathcolor{gray}{3} \hat{1} \hat{2} }_{\;\! \mathcolor{gray}{k_{\symup{z}}} \textcolor{Maroon}{(2)} }$ 表示的\textcolor{Plum}{矢量形式} $\Xint{\mathcolor{gray}{-}}{25}{\bar{P}}^{\;\! \textcolor{PineGreen}{\hat{3}} }_{\;\! \mathcolor{gray}{z} \textcolor{Maroon}{(2)} }$,即
\begin{subequations} \label{eq:DP^(2)-3_12-discrete-G}
\begin{align}
	\Xint{\mathcolor{gray}{-}}{30}{\bar{P}}^{\;\! \textcolor{PineGreen}{\hat{3}} }_{\;\! \mathcolor{gray}{z}  \textcolor{Maroon}{(2)}} &\xrightarrow[]{\text{\bref{eq:DP^(2)-3_12-discrete-SFG6}}} \bar{\chi}^{\;\! \textcolor{PineGreen}{\hat{3}} \hat{1} \hat{2} }_{\;\! \textcolor{Maroon}{(2)} \textcolor{PineGreen}{\hat{1} \hat{2}}} \odot \mathcolor{gray}{\mathcal F_{z}^{-1}} \left[ \Xint{\mathcolor{gray}{-}}{18}{\bar{M}}^{\;\! \mathcolor{gray}{3} \hat{1} \hat{2} }_{\;\! \mathcolor{gray}{k_{\symup{z}}} \textcolor{Maroon}{(2)} } \mathcolor{gray}{*} \Xint{\mathcolor{gray}{-}}{295}{E}^{\;\! \textcolor{PineGreen}{\hat{1}}}_{\;\! \hat{1} \mathcolor{gray}{z}} \mathcolor{gray}{*} \Xint{\mathcolor{gray}{-}}{295}{E}^{\;\! \textcolor{PineGreen}{\hat{2}}}_{\;\! \hat{2} \mathcolor{gray}{z}} \right] \label{eq:DP^(2)-3_12-discrete-G1} \\
	&\xrightarrow[]{\text{$\sim$ \bref{eq:components-eigenwave'}}} \bar{\chi}^{\;\! \textcolor{PineGreen}{\hat{3}} \hat{1} \hat{2} }_{\;\! \textcolor{Maroon}{(2)} \textcolor{PineGreen}{\hat{1} \hat{2}}} \odot \mathcolor{gray}{\mathcal F_{z}^{-1}} \left[ \Xint{\mathcolor{gray}{-}}{18}{\bar{M}}^{\;\! \mathcolor{gray}{3} \hat{1} \hat{2} }_{\;\! \mathcolor{gray}{k_{\symup{z}}} \textcolor{Maroon}{(2)} } \mathcolor{gray}{*} \left( \Xint{\mathcolor{gray}{-}}{20}{\mathtt{G}}^{\;\! \textcolor{PineGreen}{\hat{1}}}_{\;\! \mathcolor{gray}{z}} \Xint{{}^{}_{\mathcolor{gray}{-}}}{10}{\hat{g}}^{\;\! \textcolor{PineGreen}{\hat{1}}}_{\;\! \hat{1}} \right) \mathcolor{gray}{*} \left( \Xint{\mathcolor{gray}{-}}{20}{\mathtt{G}}^{\;\! \textcolor{PineGreen}{\hat{2}}}_{\;\! \mathcolor{gray}{z}} \Xint{{}^{}_{\mathcolor{gray}{-}}}{10}{\hat{g}}^{\;\! \textcolor{PineGreen}{\hat{2}}}_{\;\! \hat{2}} \right) \right] ~, \label{eq:DP^(2)-3_12-discrete-G2}
\end{align}
\end{subequations}
对应地,\bref{eq:simplify8-scalar-g-modulus-P-spectrum} 降为\textcolor{Plum}{离散}个\textcolor{gray}{波长}的\textcolor{NavyBlue}{(准)连续光}\textcolor{Maroon}{和频}或\textcolor{Maroon}{上转换}的\textcolor{NavyBlue}{非超快}版本
\begin{align} \label{eq:simplify8-scalar-g-modulus-P-discrete}
	\mathcolor{gray}{\nabla_z} \Xint{\begin{smallmatrix} ~ \\ {}^{}_{\mathcolor{gray}{-}} \\ ~ \end{smallmatrix}}{09}{\mathtt{g}}^{\;\! \textcolor{PineGreen}{\hat{3}}}_{\;\! \mathcolor{gray}{z}} &\xrightarrow[\text{$\sim$\bref{eq:simplify8-scalar-g-modulus}}]{\text{\bref{eq:DP^(2)-3_12-discrete-G1}}} \mathbb{i} k_{\textcolor{Maroon}{\mathsf{o}} \mathcolor{gray}{3}}^{\;\! 2} \frac{\Xint{{}^{}_{\mathcolor{gray}{-}}}{10}{\hat{g}}^{\;\! \textcolor{PineGreen}{\hat{3}} \textcolor{Plum}{\dag}}_{\;\! } \cdot \bar{\chi}^{\;\! \textcolor{PineGreen}{\hat{3}} \hat{1} \hat{2} }_{\;\! \textcolor{Maroon}{(2)} \textcolor{PineGreen}{\hat{1} \hat{2}}} \odot \mathcolor{gray}{\mathcal F_{z}^{-1}} \left[ \Xint{\mathcolor{gray}{-}}{18}{\bar{M}}^{\;\! \mathcolor{gray}{3} \hat{1} \hat{2} }_{\;\! \mathcolor{gray}{k_{\symup{z}}} \textcolor{Maroon}{(2)} } \mathcolor{gray}{*} \Xint{\mathcolor{gray}{-}}{25}{E}^{\;\! \textcolor{PineGreen}{\hat{1}}}_{\;\! \hat{1} \mathcolor{gray}{z}} \mathcolor{gray}{*} \Xint{\mathcolor{gray}{-}}{25}{E}^{\;\! \textcolor{PineGreen}{\hat{2}}}_{\;\! \hat{2} \mathcolor{gray}{z}} \right]}{ 2 \lvert \Xint{{}^{}_{\mathcolor{gray}{-}}}{10}{\hat{g}}^{\;\! \textcolor{PineGreen}{\hat{3}}} \rvert^2 \Xint{\begin{smallmatrix} ~ \\ {}^{}_{\mathcolor{gray}{-}} \\ ~ \end{smallmatrix}}{15}{k}_{\;\! \symup{z}}^{\;\!  \textcolor{PineGreen}{\hat{3}}} \mathbb{e}^{\mathbb{i} \Xint{\begin{smallmatrix} ~ \\ {}^{}_{\mathcolor{gray}{-}} \\ ~ \end{smallmatrix}}{15}{k}_{\symup{z}}^{\;\!  \textcolor{PineGreen}{\hat{3}}} \mathcolor{gray}{z}}} ~, 
\end{align}
同样注意,\textcolor{Plum}{哈达马积} $\odot$ 的运算\textcolor{Plum}{优先级}恒高于\textcolor{Plum}{点积} $\cdot$(以省略一对\textcolor{Plum}{小括号})。

\vspace*{-2.7em}

\marginLeft[-2.4em]{ssec:scalar}\subsection{标量非线性波源、标量调制场条件}\label{ssec:scalar}

如果\textcolor{Plum}{非线性}\textcolor{NavyBlue}{驱动源}中参与相互作用的\textbf{每一个}\textcolor{NavyBlue}{泵浦} $\Xint{\mathcolor{gray}{-}}{25}{E}^{\;\! \mathcolor{gray}{\omega} \textcolor{PineGreen}{\hat{\jmath}}}_{\;\! \hat{\jmath} \mathcolor{gray}{z}}$ 的\textcolor{PineGreen}{本征偏振态} $\Xint{{}^{}_{\mathcolor{gray}{-}}}{10}{\hat{g}}^{\;\! \mathcolor{gray}{\omega} \textcolor{PineGreen}{\hat{\jmath}}}_{\;\! \hat{\jmath}}$ 固定为\textcolor{Maroon}{倒空间}中的\textcolor{Plum}{定常}矢量 ${\hat{g}}^{\;\! \mathcolor{gray}{\omega} \textcolor{PineGreen}{\hat{\jmath}}}_{\;\! \hat{\jmath}}$,不是\textcolor{gray}{横向空间频率} $\mathcolor{gray}{\bar{k}_{\symup{\rho}}}$ 的函数,不随\textcolor{PineGreen}{波矢}方向变化而改变,则在该
\begin{subequations} \label{eq:scalar_nonlinear_drive}
\begin{align}
	&\text{\textbf{标量\textcolor{Plum}{非线性}\textcolor{NavyBlue}{波源}}条件(\textcolor{NavyBlue}{脉冲}):} \hspace{0.2em} \Xint{{}^{}_{\mathcolor{gray}{-}}}{10}{\hat{g}}^{\;\! \mathcolor{gray}{\omega} \textcolor{PineGreen}{\hat{\jmath}}}_{\;\! \hat{\jmath}} \hspace{-7.5em}&&\equiv~ {\hat{g}}^{\;\! \mathcolor{gray}{\omega} \textcolor{PineGreen}{\hat{\jmath}}}_{\;\! \hat{\jmath}} ~, \label{eq:scalar_nonlinear_drive-spectrum} \\
	&\text{\textbf{标量\textcolor{Plum}{非线性}\textcolor{NavyBlue}{波源}}条件(\textcolor{NavyBlue}{连续}):} \hspace{0.7em} \Xint{{}^{}_{\mathcolor{gray}{-}}}{10}{\hat{g}}^{\;\! \textcolor{PineGreen}{\hat{\jmath}}}_{\;\! \hat{\jmath}} \hspace{-7.5em}&&\equiv~ {\hat{g}}^{\;\! \textcolor{PineGreen}{\hat{\jmath}}}_{\;\! \hat{\jmath}} ~, \label{eq:scalar_nonlinear_drive-discrete}
\end{align}
\end{subequations}
下,\textcolor{NavyBlue}{脉冲光}\textcolor{Maroon}{倍频}、\textcolor{NavyBlue}{连续光}\textcolor{Maroon}{和频}过程,分别所对应的\bref{eq:DP^(2)-3_12-spectrum-G2,eq:DP^(2)-3_12-discrete-G2} \textcolor{Plum}{非线性}\textcolor{NavyBlue}{波源},可进一步\textcolor{Plum}{退化}为\Footnote{注,其中 $\odot$ 运算/相互作用,只作用于数据结构/对象 $\bar{\chi}^{\;\! \mathcolor{gray}{\omega} \textcolor{PineGreen}{\hat{3}} \hat{1} \hat{2} }_{\;\! \textcolor{Maroon}{(2)} \textcolor{PineGreen}{\hat{1} \hat{2}}}$ 与 $\Xint{\mathcolor{gray}{-}}{18}{\bar{M}}^{\;\! \mathcolor{gray}{\omega} \hat{1} \hat{2} }_{\;\! \mathcolor{gray}{k_{\symup{z}}} \textcolor{Maroon}{(2)} }$ 两者,与标量(场)${\hat{g}}^{\;\! \mathcolor{gray}{\omega} \textcolor{PineGreen}{\hat{1}}}_{\;\! \hat{1}}, {\hat{g}}^{\;\! \mathcolor{gray}{\omega} \textcolor{PineGreen}{\hat{2}}}_{\;\! \hat{2}}$ 无关。}
\begin{subequations} \label{eq:DP^(2)-3_12-chieff-G}
\begin{align}
	\Xint{\mathcolor{gray}{-}}{30}{\bar{P}}^{\;\! \mathcolor{gray}{\omega} \textcolor{PineGreen}{\hat{3}} }_{\;\! \mathcolor{gray}{z} \textcolor{Maroon}{(2)} } &\xrightarrow[\text{\bref{eq:DP^(2)-3_12-spectrum-G2}}]{\text{\bref{eq:scalar_nonlinear_drive-spectrum}}} \bar{\chi}^{\;\! \mathcolor{gray}{\omega} \textcolor{PineGreen}{\hat{3}} \hat{1} \hat{2} }_{\;\! \textcolor{Maroon}{(2)} \textcolor{PineGreen}{\hat{1} \hat{2}}} ~ {\hat{g}}^{\;\! \mathcolor{gray}{\omega} \textcolor{PineGreen}{\hat{1}}}_{\;\! \hat{1}} ~\mathcolor{gray}{\widetilde *}~ {\hat{g}}^{\;\! \mathcolor{gray}{\omega} \textcolor{PineGreen}{\hat{2}}}_{\;\! \hat{2}} \odot \mathcolor{gray}{\mathcal F_{z}^{-1}} \left[ \Xint{\mathcolor{gray}{-}}{18}{\bar{M}}^{\;\! \mathcolor{gray}{\omega} \hat{1} \hat{2} }_{\;\! \mathcolor{gray}{k_{\symup{z}}} \textcolor{Maroon}{(2)} } \mathcolor{gray}{*} \Xint{\mathcolor{gray}{-}}{20}{\mathtt{G}}^{\;\! \mathcolor{gray}{\omega} \textcolor{PineGreen}{\hat{1}}}_{\;\! \mathcolor{gray}{z}} ~\mathcolor{gray}{\widetilde \circledast}~ \Xint{\mathcolor{gray}{-}}{20}{\mathtt{G}}^{\;\! \mathcolor{gray}{\omega} \textcolor{PineGreen}{\hat{2}}}_{\;\! \mathcolor{gray}{z}} \right] ~, \label{eq:DP^(2)-3_12-spectrum-chieff-G} \\
	\Xint{\mathcolor{gray}{-}}{30}{\bar{P}}^{\;\! \textcolor{PineGreen}{\hat{3}} }_{\;\! \mathcolor{gray}{z} \textcolor{Maroon}{(2)} } &\xrightarrow[\text{\bref{eq:DP^(2)-3_12-discrete-G2}}]{\text{\bref{eq:scalar_nonlinear_drive-discrete}}} \bar{\chi}^{\;\! \textcolor{PineGreen}{\hat{3}} \hat{1} \hat{2} }_{\;\! \textcolor{Maroon}{(2)} \textcolor{PineGreen}{\hat{1} \hat{2}}} ~ {\hat{g}}^{\;\! \textcolor{PineGreen}{\hat{1}}}_{\;\! \hat{1}}  {\hat{g}}^{\;\! \textcolor{PineGreen}{\hat{2}}}_{\;\! \hat{2}} \odot \mathcolor{gray}{\mathcal F_{z}^{-1}} \left[ \Xint{\mathcolor{gray}{-}}{18}{\bar{M}}^{\;\! \mathcolor{gray}{3} \hat{1} \hat{2} }_{\;\! \mathcolor{gray}{k_{\symup{z}}} \textcolor{Maroon}{(2)} } \mathcolor{gray}{*} \Xint{\mathcolor{gray}{-}}{20}{\mathtt{G}}^{\;\! \textcolor{PineGreen}{\hat{1}}}_{\;\! \mathcolor{gray}{z}} \mathcolor{gray}{*} \Xint{\mathcolor{gray}{-}}{20}{\mathtt{G}}^{\;\! \textcolor{PineGreen}{\hat{2}}}_{\;\! \mathcolor{gray}{z}} \right] ~, \label{eq:DP^(2)-3_12-discrete-chieff-G}
\end{align}
\end{subequations}
但这一般是不成立的:因为 \cref{chap:LFCO} 中的\textcolor{Plum}{线性}(\textcolor{Maroon}{傅立叶})\textcolor{PineGreen}{晶体光学}已经解析出结论:在非各向同性材料里,电磁波的\textcolor{PineGreen}{本征偏振态}是 $\mathcolor{gray}{\bar{k}_{\symup{\rho}}}$ 的函数;尽管如此,为了不止步于 \bref{eq:DP^(2)-3_12-spectrum-G},以及为了得到后续的标量\textcolor{Plum}{非线性}\textcolor{Maroon}{时空谱}耦合波方程,我们在
%\bref{chap:LFCO}
\begin{align} \label{eq:scalar_nonlinear_drive2}
	\text{\textbf{参与构成\textcolor{Plum}{非线性}\textcolor{NavyBlue}{波源}的所有行波,均为\textcolor{PineGreen}{本征偏振态}\textcolor{Plum}{固定}的标量场}}
\end{align}
即“\textbf{标量\textcolor{Plum}{非线性}\textcolor{NavyBlue}{波源}}”的假设 \bref{eq:scalar_nonlinear_drive} 下,从 \bref{eq:DP^(2)-3_12-chieff-G} 开始继续向后推导。

注意,不论正上标带 “$\mathcolor{gray}{\sim}$” 的符号有多少个(\bref{eq:DP^(2)-3_12-spectrum-chieff-G} 中有两个:“~$\mathcolor{gray}{\widetilde *},~ \mathcolor{gray}{\widetilde \circledast}$~”),只对这些符号所作用的最左(这里即 $\Xint{{}^{}_{\mathcolor{gray}{-}}}{10}{\hat{g}}^{\;\! \mathcolor{gray}{\omega} \textcolor{PineGreen}{\hat{1}}}_{\;\! \hat{1}}$)到最右(这里即 $\Xint{\mathcolor{gray}{-}}{16}{\mathtt{G}}^{\;\! \mathcolor{gray}{\omega} \textcolor{PineGreen}{\hat{2}}}_{\;\! \mathcolor{gray}{z}}$)之间的部分作为\textcolor{Plum}{被积函数}/\textcolor{Plum}{表达式},在\textcolor{gray}{时间频率}维度做一次(而非多次)一维\textcolor{Plum}{卷积积分}。此外,\textbf{需\textcolor{Plum}{按以下顺序执行积分}:“$\mathcolor{gray}{\widetilde \circledast}$” 的 $\mathcolor{gray}{\bar{k}_{\symup{\rho}}}$ 域 $\to$ $\mathcolor{gray}{\bar{k}_{\symup{\rho}}}$ 域的 “$\mathcolor{gray}{*}$” $\to$ $\mathcolor{gray}{k_{\symup{z}}}$ 域的 $\mathcolor{gray}{\mathcal F^{-1}_z} \left[ \cdot \right]$ $\to$ “$\mathcolor{gray}{\widetilde \circledast}$” 的 $\mathcolor{gray}{\omega}$ 域(即 $\mathcolor{gray}{\omega}$ 域的 ``~$\mathcolor{gray}{\widetilde *}$~'')}\bypertarget{OperatorSequence}。这也是相应程序中 \textbf{for 循环从内到外层的计算顺序}。

在 \bref{eq:scalar_nonlinear_drive,eq:scalar_nonlinear_drive2} 的“\textbf{标量\textcolor{Plum}{非线性}\textcolor{NavyBlue}{波源}}”条件下,\textcolor{NavyBlue}{脉冲光}\textcolor{Maroon}{倍频}、\textcolor{NavyBlue}{连续光}\textcolor{Maroon}{和频}过程的电场\textcolor{PineGreen}{本征复振幅} \bref{eq:simplify8-scalar-g-modulus-P-spectrum,eq:simplify8-scalar-g-modulus-P-discrete} 退化为
\begin{subequations} \label{eq:scalar-g-modulus-P-chieff}
\begin{align}
	\mathcolor{gray}{\nabla_z} \Xint{\begin{smallmatrix} ~ \\ {}^{}_{\mathcolor{gray}{-}} \\ ~ \end{smallmatrix}}{09}{\mathtt{g}}^{\;\!\mathcolor{gray}{\omega} \textcolor{PineGreen}{\hat{3}}}_{\;\! \mathcolor{gray}{z}} &\xrightarrow[\text{\bref{eq:simplify8-scalar-g-modulus-P-spectrum}}]{\text{\bref{eq:scalar_nonlinear_drive-spectrum}}} \mathbb{i} k_{\textcolor{Maroon}{\mathsf{o}} \mathcolor{gray}{\omega}}^{\;\! 2} \frac{\textcolor{gray}{\widetilde{\textcolor{black}{\chi}}}^{ \hat{3} \textcolor{PineGreen}{\hat{3}} \textcolor{Maroon}{(2)} \mathcolor{gray}{\omega} }_{ \textcolor{NavyBlue}{\text{eff}} \hat{1} \hat{2} \textcolor{PineGreen}{\hat{1} \hat{2}} } ~ \mathcolor{gray}{\mathcal F_{z}^{-1}} \left[ \Xint{\mathcolor{gray}{-}}{18}{M}^{\;\! \mathcolor{gray}{\omega} \hat{1} \hat{2} }_{\;\! \hat{3} \mathcolor{gray}{k_{\symup{z}}} \textcolor{Maroon}{(2)} } \mathcolor{gray}{*} \Xint{\mathcolor{gray}{-}}{15}{\mathtt{G}}^{\;\! \mathcolor{gray}{\omega} \textcolor{PineGreen}{\hat{1}}}_{\;\! \mathcolor{gray}{z}} ~\mathcolor{gray}{\widetilde \circledast}~ \Xint{\mathcolor{gray}{-}}{15}{\mathtt{G}}^{\;\! \mathcolor{gray}{\omega} \textcolor{PineGreen}{\hat{2}}}_{\;\! \mathcolor{gray}{z}} \right]}{ 2 \lvert \Xint{{}^{}_{\mathcolor{gray}{-}}}{10}{\hat{g}}^{\;\! \textcolor{PineGreen}{\hat{3}}}_{\;\! \mathcolor{gray}{\omega}} \rvert^2 \Xint{\begin{smallmatrix} ~ \\ {}^{}_{\mathcolor{gray}{-}} \\ ~ \end{smallmatrix}}{15}{k}_{\;\! \symup{z}}^{\;\! \mathcolor{gray}{\omega} \textcolor{PineGreen}{\hat{3}}} \mathbb{e}^{\mathbb{i} \Xint{\begin{smallmatrix} ~ \\ {}^{}_{\mathcolor{gray}{-}} \\ ~ \end{smallmatrix}}{15}{k}_{\symup{z}}^{\;\! \mathcolor{gray}{\omega} \textcolor{PineGreen}{\hat{3}}} \mathcolor{gray}{z}}} ~, \label{eq:scalar-g-modulus-P-chieff-spectrum} \\
	\mathcolor{gray}{\nabla_z} \Xint{\begin{smallmatrix} ~ \\ {}^{}_{\mathcolor{gray}{-}} \\ ~ \end{smallmatrix}}{09}{\mathtt{g}}^{\;\! \textcolor{PineGreen}{\hat{3}}}_{\;\! \mathcolor{gray}{z}} &\xrightarrow[\text{\bref{eq:simplify8-scalar-g-modulus-P-discrete}}]{\text{\bref{eq:scalar_nonlinear_drive-discrete}}} \mathbb{i} k_{\textcolor{Maroon}{\mathsf{o}} \mathcolor{gray}{3}}^{\;\! 2} \frac{{\chi}^{\hat{3} \textcolor{PineGreen}{\hat{3}} \textcolor{Maroon}{(2)} }_{\textcolor{NavyBlue}{\text{eff}} \hat{1} \textcolor{PineGreen}{\hat{1}} \hat{2} \textcolor{PineGreen}{\hat{2}} } ~ \mathcolor{gray}{\mathcal F_{z}^{-1}} \left[ \Xint{\mathcolor{gray}{-}}{18}{M}^{\;\! \mathcolor{gray}{3} \hat{1} \hat{2} }_{\;\! \hat{3} \mathcolor{gray}{k_{\symup{z}}} \textcolor{Maroon}{(2)} } \mathcolor{gray}{*} \Xint{\mathcolor{gray}{-}}{15}{\mathtt{G}}^{\;\! \textcolor{PineGreen}{\hat{1}}}_{\;\! \mathcolor{gray}{z}} \mathcolor{gray}{*} \Xint{\mathcolor{gray}{-}}{15}{\mathtt{G}}^{\;\! \textcolor{PineGreen}{\hat{2}}}_{\;\! \mathcolor{gray}{z}} \right]}{ 2 \lvert \Xint{{}^{}_{\mathcolor{gray}{-}}}{10}{\hat{g}}^{\;\! \textcolor{PineGreen}{\hat{3}}} \rvert^2 \Xint{\begin{smallmatrix} ~ \\ {}^{}_{\mathcolor{gray}{-}} \\ ~ \end{smallmatrix}}{15}{k}_{\;\! \symup{z}}^{\;\!  \textcolor{PineGreen}{\hat{3}}} \mathbb{e}^{\mathbb{i} \Xint{\begin{smallmatrix} ~ \\ {}^{}_{\mathcolor{gray}{-}} \\ ~ \end{smallmatrix}}{15}{k}_{\symup{z}}^{\;\!  \textcolor{PineGreen}{\hat{3}}} \mathcolor{gray}{z}}} ~, \label{eq:scalar-g-modulus-P-chieff-discrete}
\end{align}
\end{subequations}
其中,定义了\textcolor{NavyBlue}{脉冲光}\textcolor{Maroon}{倍频}、\textcolor{NavyBlue}{连续光}\textcolor{Maroon}{和频}过程的\textcolor{NavyBlue}{有效非线性系数}(三阶)张量
\begin{subequations} \label{eq:chieff}
\begin{align}
	\textcolor{gray}{\widetilde{\textcolor{black}{\chi}}}^{ \hat{3} \textcolor{PineGreen}{\hat{3}} \textcolor{Maroon}{(2)} \mathcolor{gray}{\omega} }_{ \textcolor{NavyBlue}{\text{eff}} \hat{1} \hat{2} \textcolor{PineGreen}{\hat{1} \hat{2}} } &= \Xint{{}^{}_{\mathcolor{gray}{-}}}{10}{\hat{g}}^{\;\! \hat{3} \textcolor{PineGreen}{\hat{3}} \textcolor{Plum}{*}}_{\;\! \mathcolor{gray}{\omega}} {\chi}^{\;\! \hat{3} \textcolor{PineGreen}{\hat{3}} \textcolor{Maroon}{(2)} }_{\;\! \mathcolor{gray}{\omega} \hat{1} \hat{2} \textcolor{PineGreen}{\hat{1} \hat{2}} } ~ {\hat{g}}^{\;\! \mathcolor{gray}{\omega} }_{\;\! \hat{1} \textcolor{PineGreen}{\hat{1}} } ~\mathcolor{gray}{\widetilde *}~ {\hat{g}}^{\;\! \mathcolor{gray}{\omega} }_{\;\! \hat{2} \textcolor{PineGreen}{\hat{2}} } ~, \label{eq:chieff-spectrum} \\
	{\chi}^{\hat{3} \textcolor{PineGreen}{\hat{3}} \textcolor{Maroon}{(2)} }_{\textcolor{NavyBlue}{\text{eff}} \hat{1} \textcolor{PineGreen}{\hat{1}} \hat{2} \textcolor{PineGreen}{\hat{2}} } &= \Xint{{}^{}_{\mathcolor{gray}{-}}}{10}{\hat{g}}^{\;\! \hat{3} \textcolor{PineGreen}{\hat{3}} \textcolor{Plum}{*}}_{\;\! } {\chi}^{\;\! \hat{3} \textcolor{PineGreen}{\hat{3}} }_{\;\! \textcolor{Maroon}{(2)} \hat{1} \textcolor{PineGreen}{\hat{1}} \hat{2} \textcolor{PineGreen}{\hat{2}}} ~ {\hat{g}}_{\;\! \hat{1} \textcolor{PineGreen}{\hat{1}} } ~ {\hat{g}}_{\;\! \hat{2} \textcolor{PineGreen}{\hat{2}}} ~, \label{eq:chieff-discrete}
\end{align}
\end{subequations}
其中,$\chi$ 头上的一个\textcolor{gray}{灰色波浪}符号 `$\mathcolor{gray}{\sim}$',表示 $\textcolor{gray}{\widetilde{\textcolor{black}{\chi}}}^{ \hat{3} \textcolor{PineGreen}{\hat{3}} \textcolor{Maroon}{(2)} \mathcolor{gray}{\omega} }_{ \textcolor{NavyBlue}{\text{eff}} \hat{1} \hat{2} \textcolor{PineGreen}{\hat{1} \hat{2}} }~, {\chi}^{\hat{3} \textcolor{PineGreen}{\hat{3}} \textcolor{Maroon}{(2)} }_{\textcolor{NavyBlue}{\text{eff}} \hat{1} \textcolor{PineGreen}{\hat{1}} \hat{2} \textcolor{PineGreen}{\hat{2}} }$ 整体,作为\textcolor{Plum}{被积函数},处在\textcolor{gray}{$\omega$ 域}的一维\textcolor{Plum}{卷积积分}内。注意,虽然 \bref{eq:chieff-spectrum} 左侧的 $\chi$ 头上有一 `$\mathcolor{gray}{\sim}$',这并不代表相应的 $\textcolor{gray}{\widetilde{\textcolor{black}{\chi}}}$ 是矢量(见 \bref{hook:1bar})。此外,\bref{eq:scalar-g-modulus-P-chieff} 的分子,即 $\textcolor{gray}{\widetilde{\textcolor{black}{\chi}}}^{ \hat{3} \textcolor{PineGreen}{\hat{3}} \textcolor{Maroon}{(2)} \mathcolor{gray}{\omega} }_{ \textcolor{NavyBlue}{\text{eff}} \hat{1} \hat{2} \textcolor{PineGreen}{\hat{1} \hat{2}} }~, {\chi}^{\hat{3} \textcolor{PineGreen}{\hat{3}} \textcolor{Maroon}{(2)} }_{\textcolor{NavyBlue}{\text{eff}} \hat{1} \textcolor{PineGreen}{\hat{1}} \hat{2} \textcolor{PineGreen}{\hat{2}} }$ 中的 $\Xint{{}^{}_{\mathcolor{gray}{-}}}{10}{\hat{g}}^{\;\! \hat{3} \textcolor{PineGreen}{\hat{3}} \textcolor{Plum}{*}}_{\;\! \mathcolor{gray}{\omega}}, \Xint{{}^{}_{\mathcolor{gray}{-}}}{10}{\hat{g}}^{\;\! \hat{3} \textcolor{PineGreen}{\hat{3}} \textcolor{Plum}{*}}_{\;\! }$ 是(\textcolor{Plum}{协变})矢量\Footnote{这里对\textcolor{Plum}{协变} $\bra{\text{bra}}$、\textcolor{Plum}{逆变} $\ket{\text{ket}}$ 指标 $\hat{3} \textcolor{PineGreen}{\hat{3}}$ 上下位置的定义,相反于 \byperref{another-bra-example}{前处}。可以看出,在这方面,是灵活的。}的分量,即标量;而分母中的 $\Xint{{}^{}_{\mathcolor{gray}{-}}}{10}{\hat{g}}^{\;\! \textcolor{PineGreen}{\hat{3}}}_{\;\! \mathcolor{gray}{\omega}}, \Xint{{}^{}_{\mathcolor{gray}{-}}}{10}{\hat{g}}^{\;\! \textcolor{PineGreen}{\hat{3}}}$ 是矢量。

若三阶张量\textcolor{NavyBlue}{调制场} $\Xint{\mathcolor{gray}{-}}{18}{M}^{\;\! \mathcolor{gray}{\omega} \hat{1} \hat{2} }_{\;\! \hat{3} \mathcolor{gray}{z} \textcolor{Maroon}{(2)} }, \Xint{\mathcolor{gray}{-}}{18}{\bar{M}}^{\;\! \mathcolor{gray}{\omega} \hat{1} \hat{2} }_{\;\! \mathcolor{gray}{z} \textcolor{Maroon}{(2)} }$ 退化为标量\textcolor{NavyBlue}{场} $\Xint{\mathcolor{gray}{-}}{18}{M}^{\;\! \mathcolor{gray}{\omega} }_{\;\! \mathcolor{gray}{z} \textcolor{Maroon}{(2)} }$,即若再施加
\begin{align} \label{eq:scalar_chi2_modulation}
	&\text{\textbf{标量场 $\chi^{\;\! \mathcolor{gray}{\omega} }_{\;\! \mathcolor{gray}{z} \textcolor{Maroon}{(2)}}$ \textcolor{NavyBlue}{调制}}条件:} \hspace{0.2em} \Xint{\mathcolor{gray}{-}}{18}{\bar{\bar{\bar{M}}}}^{\;\! \mathcolor{gray}{\omega} }_{\;\! \mathcolor{gray}{z} \textcolor{Maroon}{(2)} } \equiv \Xint{\mathcolor{gray}{-}}{18}{M}^{\;\! \mathcolor{gray}{\omega} }_{\;\! \mathcolor{gray}{z} \textcolor{Maroon}{(2)} } ~,
\end{align}
条件,则\textcolor{NavyBlue}{脉冲光}\textcolor{Maroon}{倍频}、\textcolor{NavyBlue}{连续光}\textcolor{Maroon}{和频}过程的电场\textcolor{PineGreen}{本征复振幅}方程 \bref{eq:scalar-g-modulus-P-chieff} 退化为
\begin{subequations} \label{eq:scalar-g-modulus-P-chieff-scalar}
\begin{align}
	\mathcolor{gray}{\nabla_z} \Xint{\begin{smallmatrix} ~ \\ {}^{}_{\mathcolor{gray}{-}} \\ ~ \end{smallmatrix}}{09}{\mathtt{g}}^{\;\!\mathcolor{gray}{\omega} \textcolor{PineGreen}{\hat{3}}}_{\;\! \mathcolor{gray}{z}} &\xrightarrow[\text{\bref{eq:scalar-g-modulus-P-chieff-spectrum}}]{\text{\bref{eq:scalar_chi2_modulation}}} \mathbb{i} k_{\textcolor{Maroon}{\mathsf{o}} \mathcolor{gray}{\omega}}^{\;\! 2} \frac{\textcolor{gray}{\widetilde{\textcolor{black}{\chi}}}^{ \textcolor{PineGreen}{\hat{3}} \textcolor{Maroon}{(2)} \mathcolor{gray}{\omega} }_{ \textcolor{NavyBlue}{\text{eff}} \textcolor{PineGreen}{\hat{1} \hat{2}} } ~ \mathcolor{gray}{\mathcal F_{z}^{-1}} \left[ \Xint{\mathcolor{gray}{-}}{18}{M}^{\;\! \mathcolor{gray}{\omega} }_{\;\! \mathcolor{gray}{k_{\symup{z}}} \textcolor{Maroon}{(2)} } \mathcolor{gray}{*} \Xint{\mathcolor{gray}{-}}{15}{\mathtt{G}}^{\;\! \mathcolor{gray}{\omega} \textcolor{PineGreen}{\hat{1}}}_{\;\! \mathcolor{gray}{z}} ~\mathcolor{gray}{\widetilde \circledast}~ \Xint{\mathcolor{gray}{-}}{15}{\mathtt{G}}^{\;\! \mathcolor{gray}{\omega} \textcolor{PineGreen}{\hat{2}}}_{\;\! \mathcolor{gray}{z}} \right]}{ 2 \lvert \Xint{{}^{}_{\mathcolor{gray}{-}}}{10}{\hat{g}}^{\;\! \textcolor{PineGreen}{\hat{3}}}_{\;\! \mathcolor{gray}{\omega}} \rvert^2 \Xint{\begin{smallmatrix} ~ \\ {}^{}_{\mathcolor{gray}{-}} \\ ~ \end{smallmatrix}}{15}{k}_{\;\! \symup{z}}^{\;\! \mathcolor{gray}{\omega} \textcolor{PineGreen}{\hat{3}}} \mathbb{e}^{\mathbb{i} \Xint{\begin{smallmatrix} ~ \\ {}^{}_{\mathcolor{gray}{-}} \\ ~ \end{smallmatrix}}{15}{k}_{\symup{z}}^{\;\! \mathcolor{gray}{\omega} \textcolor{PineGreen}{\hat{3}}} \mathcolor{gray}{z}}} ~, \label{eq:scalar-g-modulus-P-chieff-scalar-spectrum} \\
	\mathcolor{gray}{\nabla_z} \Xint{\begin{smallmatrix} ~ \\ {}^{}_{\mathcolor{gray}{-}} \\ ~ \end{smallmatrix}}{09}{\mathtt{g}}^{\;\! \textcolor{PineGreen}{\hat{3}}}_{\;\! \mathcolor{gray}{z}} &\xrightarrow[\text{\bref{eq:scalar-g-modulus-P-chieff-discrete}}]{\text{\bref{eq:scalar_chi2_modulation}}} \mathbb{i} k_{\textcolor{Maroon}{\mathsf{o}} \mathcolor{gray}{3}}^{\;\! 2} \frac{{\chi}^{\textcolor{PineGreen}{\hat{3}} \textcolor{Maroon}{(2)} }_{\textcolor{NavyBlue}{\text{eff}} \textcolor{PineGreen}{\hat{1}} \textcolor{PineGreen}{\hat{2}} } ~ \mathcolor{gray}{\mathcal F_{z}^{-1}} \left[ \Xint{\mathcolor{gray}{-}}{18}{M}^{\;\! \mathcolor{gray}{3}}_{\;\! \mathcolor{gray}{k_{\symup{z}}} \textcolor{Maroon}{(2)} } \mathcolor{gray}{*} \Xint{\mathcolor{gray}{-}}{15}{\mathtt{G}}^{\;\! \textcolor{PineGreen}{\hat{1}}}_{\;\! \mathcolor{gray}{z}} \mathcolor{gray}{*} \Xint{\mathcolor{gray}{-}}{15}{\mathtt{G}}^{\;\! \textcolor{PineGreen}{\hat{2}}}_{\;\! \mathcolor{gray}{z}} \right]}{ 2 \lvert \Xint{{}^{}_{\mathcolor{gray}{-}}}{10}{\hat{g}}^{\;\! \textcolor{PineGreen}{\hat{3}}} \rvert^2 \Xint{\begin{smallmatrix} ~ \\ {}^{}_{\mathcolor{gray}{-}} \\ ~ \end{smallmatrix}}{15}{k}_{\;\! \symup{z}}^{\;\!  \textcolor{PineGreen}{\hat{3}}} \mathbb{e}^{\mathbb{i} \Xint{\begin{smallmatrix} ~ \\ {}^{}_{\mathcolor{gray}{-}} \\ ~ \end{smallmatrix}}{15}{k}_{\symup{z}}^{\;\!  \textcolor{PineGreen}{\hat{3}}} \mathcolor{gray}{z}}} ~, \label{eq:scalar-g-modulus-P-chieff-scalar-discrete}
\end{align}
\end{subequations}
同时\textcolor{NavyBlue}{脉冲光}\textcolor{Maroon}{倍频}、\textcolor{NavyBlue}{连续光}\textcolor{Maroon}{和频}过程的\textcolor{NavyBlue}{有效非线性系数}张量 \bref{eq:chieff} 退化为标量
\begin{subequations} \label{eq:chieff-scalar}
\begin{align}
	\textcolor{gray}{\widetilde{\textcolor{black}{\chi}}}^{ \textcolor{PineGreen}{\hat{3}} \textcolor{Maroon}{(2)} \mathcolor{gray}{\omega} }_{ \textcolor{NavyBlue}{\text{eff}} \textcolor{PineGreen}{\hat{1} \hat{2}} } &\xrightarrow[\text{\bref{eq:chieff-spectrum}}]{\text{\bref{eq:scalar_chi2_modulation}}} \Xint{{}^{}_{\mathcolor{gray}{-}}}{10}{\hat{g}}^{\;\! \hat{3} \textcolor{PineGreen}{\hat{3}} \textcolor{Plum}{*}}_{\;\! \mathcolor{gray}{\omega}} {\chi}^{\;\! \textcolor{PineGreen}{\hat{3}} \mathcolor{gray}{\omega} \hat{1} \hat{2} }_{\;\! \hat{3} \textcolor{Maroon}{(2)} \textcolor{PineGreen}{\hat{1} \hat{2}}} ~ {\hat{g}}^{\;\! \mathcolor{gray}{\omega} }_{\;\! \hat{1} \textcolor{PineGreen}{\hat{1}} } ~\mathcolor{gray}{\widetilde *}~ {\hat{g}}^{\;\! \mathcolor{gray}{\omega} }_{\;\! \hat{2} \textcolor{PineGreen}{\hat{2}} } ~, \label{eq:chieff-scalar-spectrum} \\
	{\chi}^{\textcolor{PineGreen}{\hat{3}} \textcolor{Maroon}{(2)} }_{\textcolor{NavyBlue}{\text{eff}} \textcolor{PineGreen}{\hat{1}} \textcolor{PineGreen}{\hat{2}} } &\xrightarrow[\text{\bref{eq:chieff-discrete}}]{\text{\bref{eq:scalar_chi2_modulation}}} \Xint{{}^{}_{\mathcolor{gray}{-}}}{10}{\hat{g}}^{\;\! \hat{3} \textcolor{PineGreen}{\hat{3}} \textcolor{Plum}{*}}_{\;\! } {\chi}^{\;\! \textcolor{PineGreen}{\hat{3}} \hat{1} \hat{2} }_{\;\! \hat{3} \textcolor{Maroon}{(2)} \textcolor{PineGreen}{\hat{1} \hat{2}}} ~ {\hat{g}}_{\;\! \hat{1} \textcolor{PineGreen}{\hat{1}} } ~ {\hat{g}}_{\;\! \hat{2} \textcolor{PineGreen}{\hat{2}} } ~, \label{eq:chieff-scalar-discrete}
\end{align}
\end{subequations}
注意,\bref{eq:chieff} 中的 ${\chi}^{\;\! \hat{3} \textcolor{PineGreen}{\hat{3}} }_{\;\! \textcolor{Maroon}{(2)} \hat{1} \hat{2} \textcolor{PineGreen}{\hat{1} \hat{2}} \mathcolor{gray}{\omega}}, {\chi}^{\;\! \hat{3} \textcolor{PineGreen}{\hat{3}} }_{\;\! \textcolor{Maroon}{(2)} \hat{1} \textcolor{PineGreen}{\hat{1}} \hat{2} \textcolor{PineGreen}{\hat{2}}}$,以及 \bref{eq:chieff-scalar} 中的 ${\chi}^{\;\! \textcolor{PineGreen}{\hat{3}} \mathcolor{gray}{\omega} \hat{1} \hat{2} }_{\;\! \hat{3} \textcolor{Maroon}{(2)} \textcolor{PineGreen}{\hat{1} \hat{2}}}, {\chi}^{\;\! \textcolor{PineGreen}{\hat{3}} \hat{1} \hat{2} }_{\;\! \hat{3} \textcolor{Maroon}{(2)} \textcolor{PineGreen}{\hat{1} \hat{2}}}$,如 \bref{eq:chi2-modulate} 下方的 \byperref{chi2-free-of-eigenmodes}{说明文字} 所提,二阶\textcolor{Plum}{非线性}系数 $\chi$ 的角标 $\textcolor{PineGreen}{\hat{1}}, \textcolor{PineGreen}{\hat{2}}$ 不从属于任何主体(包括它自己 $\chi$ 和其他 $g$),只服务于爱因斯坦求和。

\marginLeft[-2.4em]{ssec:undepleted-pump-approximation}\subsection{泵浦未耗尽近似条件下的非线性卷积解}\label{ssec:undepleted-pump-approximation}

\vspace*{-1.0em}

\marginLeft[-2.4em]{sec:down_convert}\section{\textcolor{Maroon}{Down conversion} 下转换 - 电场本征复振幅 \textcolor{Maroon}{equation}}\label{sec:down_convert}

\bref{eq:scalar-g-modulus-P-chieff-spectrum} 即为\textcolor{Plum}{各向异性}材料中,\textcolor{gray}{光波段单色 $\omega$} \textcolor{Maroon}{傅立叶}标量\textcolor{NavyBlue}{脉冲光}\textcolor{Maroon}{倍频},或\textcolor{Maroon}{光整流}后续级联\textcolor{Maroon}{电光效应}的,标量\textcolor{Maroon}{时空谱}耦合波方程。其配合 \bref{eq:chieff-spectrum} 和 \textcolor{PineGreen}{本征偏振态} $\Xint{{}^{}_{\mathcolor{gray}{-}}}{10}{\hat{g}}^{\;\!\mathcolor{gray}{\omega} \textcolor{PineGreen}{\hat{3}}}_{\;\! \textcolor{Maroon}{\Yup}}$ 即变成右侧标量、左侧矢量的\text{\textbf{标量\textcolor{Plum}{非线性}\textcolor{NavyBlue}{波源}}条件(\textcolor{NavyBlue}{脉冲})}条件(见 \bref{eq:simplify8-scalar-g-conjugate} 下方的 \byperref{waveq-scalar-2-vector}{段落})下的标量\textcolor{Maroon}{时空谱}耦合波方程。若波动方程进一步回退到 \bref{eq:simplify8-scalar-g-modulus-P-spectrum}(特别是分子回溯至 \bref{eq:DP^(2)-3_12-spectrum-G2}),则进化至不受\textbf{\textcolor{NavyBlue}{混频源}\textcolor{PineGreen}{本征偏振态}固定(非场)}的近似条件(\bref{eq:scalar_nonlinear_drive})约束的最广义情形。

以相干\textcolor{NavyBlue}{脉冲光连续谱}为例(\textcolor{Plum}{离散}\textcolor{gray}{波长}的情况类似),上一段以及上一节,所考察的全都是\textcolor{Maroon}{上转换}过程,忽略了二阶\textcolor{Plum}{非线性}\textcolor{gray}{频率转换} $=$ \textcolor{NavyBlue}{三波混频}过程中的另一个重要方面,即\textcolor{Maroon}{下转换}过程、\textcolor{NavyBlue}{能量回流}效应、\textcolor{NavyBlue}{效率饱和}问题。

对于\textcolor{NavyBlue}{脉冲光}\textcolor{Maroon}{倍频}过程,(其逆过程)还存在\textcolor{NavyBlue}{倍频光}能量回流到基波的\textcolor{Maroon}{下转换}通道(尚未涉及)。对于 \textcolor{Maroon}{THz} 波段 $\textcolor{gray}{\omega}$ 的\textcolor{Maroon}{电光效应}过程,(其逆过程)还有 1 个\textcolor{NavyBlue}{脉冲}\textcolor{Maroon}{光整流}标/矢量\textcolor{Maroon}{时空谱}耦合波方程(尚未处理)。

前者\textcolor{NavyBlue}{脉冲光}\textcolor{Maroon}{倍频}与其缺失的\textcolor{Maroon}{下转换}过程一起,后者\textcolor{Maroon}{电光效应}与其缺失的\textcolor{Maroon}{光整流}过程一起;二者共同构成\textcolor{NavyBlue}{脉冲光}\textcolor{gray}{谱内}(\textcolor{gray}{自})\textcolor{gray}{混频}的标/矢量\textcolor{Maroon}{时空谱}耦合波方程组,以完整描述单个\textcolor{NavyBlue}{脉冲光}\textcolor{gray}{谱内混频}\textcolor{Maroon}{上转换}出另一个\textcolor{NavyBlue}{光脉冲},或\textcolor{Maroon}{下转换}出另一个\textcolor{Maroon}{THz} \textcolor{NavyBlue}{脉冲}(的过程)。

\marginLeft[-2.4em]{ssec:cross-correlation}\subsection{上转换 $\to$ 下转换,卷积 $\to$ 互相关}\label{ssec:cross-correlation}

所有的\textcolor{Maroon}{上转换}过程,涉及\textcolor{gray}{傅立叶域}的\textcolor{Plum}{卷积}。所有的\textcolor{Maroon}{下转换}过程,涉及\textcolor{gray}{傅立叶域}的\textcolor{Plum}{互相关}。因此在起笔\textcolor{Maroon}{下转换}过程,以及\textcolor{NavyBlue}{三波混频}耦合波方程组之前,需要针对\textcolor{Plum}{互相关},引入\textcolor{Plum}{数学铺垫}。首先,可以验证,下述 2 条纯粹的\textcolor{Plum}{数学关系}成立:
\begin{subequations} \label{eq:F[B*]F[AB*]}
\begin{align}
	\hspace{-1em} \mathcolor{gray}{\mathcal F} \left[ B_{\;\! \mathcolor{gray}{z}}^{\textcolor{Plum}{*}} \right] &= \left\{ \mathcolor{gray}{\mathcal F} \left[ B_{\;\! \mathcolor{gray}{z}} \right] \mathcolor{gray}{\Big|}_{\mathcolor{gray}{- \bar{k}_{\symup{\rho}}}} \right\}^{\textcolor{Plum}{*}} = \mathcolor{gray}{\mathcal F}^{\textcolor{Plum}{*}} \left[ B_{\;\! \mathcolor{gray}{z}} \right] \mathcolor{gray}{\Big|}_{\mathcolor{gray}{- \bar{k}_{\symup{\rho}}}}
	= \Xint{\mathcolor{gray}{-}}{18}{B}_{\;\! \mathcolor{gray}{z}}^{\textcolor{Plum}{*}} \left( \mathcolor{gray}{- \bar{k}_{\symup{\rho}}} \right) ~,  \label{eq:F[B*]} \\ 
	\hspace{-1em} \mathcolor{gray}{\mathcal F} \left[ A_{\;\! \mathcolor{gray}{z}} \cdot B_{\;\! \mathcolor{gray}{z}}^{\textcolor{Plum}{*}} \right] &= \mathcolor{gray}{\mathcal F} \left[ A_{\;\! \mathcolor{gray}{z}} \right] \textcolor{gray}{*}~ \mathcolor{gray}{\mathcal F} \left[ B_{\;\! \mathcolor{gray}{z}}^{\textcolor{Plum}{*}} \right] = \mathcolor{gray}{\mathcal F}^{\textcolor{Plum}{*}} \left[ B_{\;\! \mathcolor{gray}{z}} \right] \mathcolor{gray}{\Big|}_{\mathcolor{gray}{- \bar{k}_{\symup{\rho}}}} \textcolor{gray}{*}~ \mathcolor{gray}{\mathcal F} \left[ A_{\;\! \mathcolor{gray}{z}} \right] =: \mathcolor{gray}{\mathcal F} \left[ B_{\;\! \mathcolor{gray}{z}} \right] \textcolor{gray}{\circ}~ \mathcolor{gray}{\mathcal F} \left[ A_{\;\! \mathcolor{gray}{z}} \right] \label{eq:F[AB*]} \\
	&= \Xint{\mathcolor{gray}{-}}{18}{A}_{\;\! \mathcolor{gray}{z}} ~\textcolor{gray}{*}~ \mathcolor{gray}{\mathcal F} \left[ B_{\;\! \mathcolor{gray}{z}}^{\textcolor{Plum}{*}} \right] \xrightarrow[]{\text{\bref{eq:F[B*]}}} \Xint{\mathcolor{gray}{-}}{18}{B}_{\;\! \mathcolor{gray}{z}}^{\textcolor{Plum}{*}} \left( \mathcolor{gray}{- \bar{k}_{\symup{\rho}}} \right) \textcolor{gray}{*}~ \Xint{\mathcolor{gray}{-}}{18}{A}_{\;\! \mathcolor{gray}{z}} \xrightarrow[]{\text{\bref{eq:AcircB-a}}} \Xint{\mathcolor{gray}{-}}{18}{B}_{\;\! \mathcolor{gray}{z}} ~\textcolor{gray}{\circ}~ \Xint{\mathcolor{gray}{-}}{18}{A}_{\;\! \mathcolor{gray}{z}} ~,
\end{align}
\end{subequations}
其中,以 $\mathcolor{gray}{\bar{k}_{\symup{\rho}}}$ 域为例,定义了\textcolor{Plum}{互相关} `$\textcolor{gray}{\circ}$' 这个二元/双目算符
\begin{subequations} \label{eq:AcircB}
\begin{align}
	\Xint{\mathcolor{gray}{-}}{18}{A}_{\;\! \mathcolor{gray}{z}} ~\textcolor{gray}{\circ}~ \Xint{\mathcolor{gray}{-}}{18}{B}_{\;\! \mathcolor{gray}{z}} := \Xint{\mathcolor{gray}{-}}{18}{A}_{\;\! \mathcolor{gray}{z}}^{\textcolor{Plum}{*}} \left( \mathcolor{gray}{- \bar{k}_{\symup{\rho}}} \right) \textcolor{gray}{*}~ \Xint{\mathcolor{gray}{-}}{18}{B}_{\;\! \mathcolor{gray}{z}} &= \mathcolor{gray}{\iint_{-\infty}^{+\infty}} \Xint{\mathcolor{gray}{-}}{18}{A}_{\;\! \mathcolor{gray}{z}}^{\textcolor{Plum}{*}} \left( \mathcolor{gray}{\bar{k}'_{\symup{\rho}}} \mathcolor{gray}{- \bar{k}_{\symup{\rho}}} \right) \cdot \Xint{\mathcolor{gray}{-}}{18}{B}_{\;\! \mathcolor{gray}{z}} \left( \mathcolor{gray}{\bar{k}'_{\symup{\rho}}} \right) \mathbb{d} \mathcolor{gray}{\bar{k}'_{\symup{\rho}}} \label{eq:AcircB-a} \\ &= \mathcolor{gray}{\iint_{-\infty}^{+\infty}} \Xint{\mathcolor{gray}{-}}{18}{A}_{\;\! \mathcolor{gray}{z}}^{\textcolor{Plum}{*}} \left( \mathcolor{gray}{\bar{k}'_{\symup{\rho}}} \right) \cdot \Xint{\mathcolor{gray}{-}}{18}{B}_{\;\! \mathcolor{gray}{z}} \left( \mathcolor{gray}{\bar{k}'_{\symup{\rho}}} + \mathcolor{gray}{\bar{k}_{\symup{\rho}}} \right) \mathbb{d} \mathcolor{gray}{\bar{k}'_{\symup{\rho}}} \label{eq:AcircB-b}~, 
\end{align}
\end{subequations}
不像\textcolor{Plum}{卷积},\textcolor{Plum}{互相关}不满足\textcolor{Plum}{交换律}:
\begin{subequations} \label{eq:AcircB=BcircA*(-k_rho)}
\begin{align}
	\Xint{\mathcolor{gray}{-}}{18}{A}_{\;\! \mathcolor{gray}{z}} ~\textcolor{gray}{\circ}~ \Xint{\mathcolor{gray}{-}}{18}{B}_{\;\! \mathcolor{gray}{z}} &\xrightarrow[]{\text{\bref{eq:AcircB-a}}} \Xint{\mathcolor{gray}{-}}{18}{A}_{\;\! \mathcolor{gray}{z}}^{\textcolor{Plum}{*}} \left( \mathcolor{gray}{- \bar{k}_{\symup{\rho}}} \right) \textcolor{gray}{*}~ \Xint{\mathcolor{gray}{-}}{18}{B}_{\;\! \mathcolor{gray}{z}} \label{eq:AcircB=BcircA*(-k_rho)-a} \\ &\xrightarrow[]{\left[ \left[ \cdot \right]^{\textcolor{Plum}{*}} \left( \mathcolor{gray}{- \bar{k}_{\symup{\rho}}} \right) \right]^{\textcolor{Plum}{*}} \left( \mathcolor{gray}{- \bar{k}_{\symup{\rho}}} \right)} \left[ \Xint{\mathcolor{gray}{-}}{18}{A}_{\;\! \mathcolor{gray}{z}} ~\textcolor{gray}{*}~ \Xint{\mathcolor{gray}{-}}{18}{B}_{\;\! \mathcolor{gray}{z}}^{\textcolor{Plum}{*}} \left( \mathcolor{gray}{- \bar{k}_{\symup{\rho}}} \right) \right]^{\textcolor{Plum}{*}} \left( \mathcolor{gray}{- \bar{k}_{\symup{\rho}}} \right) \label{eq:AcircB=BcircA*(-k_rho)-b} \\ &= \left[ \Xint{\mathcolor{gray}{-}}{18}{B}_{\;\! \mathcolor{gray}{z}}^{\textcolor{Plum}{*}} \left( \mathcolor{gray}{- \bar{k}_{\symup{\rho}}} \right) \textcolor{gray}{*}~ \Xint{\mathcolor{gray}{-}}{18}{A}_{\;\! \mathcolor{gray}{z}} \right]^{\textcolor{Plum}{*}} \left( \mathcolor{gray}{- \bar{k}_{\symup{\rho}}} \right) \xrightarrow[]{\text{\bref{eq:AcircB-a}}} \left[ \Xint{\mathcolor{gray}{-}}{18}{B}_{\;\! \mathcolor{gray}{z}} ~\textcolor{gray}{\circ}~ \Xint{\mathcolor{gray}{-}}{18}{A}_{\;\! \mathcolor{gray}{z}} \right]^{\textcolor{Plum}{*}} \left( \mathcolor{gray}{- \bar{k}_{\symup{\rho}}} \right) \label{eq:AcircB=BcircA*(-k_rho)-c}~, 
\end{align}
\end{subequations}
\textcolor{Plum}{互相关}也不满足\textcolor{Plum}{结合律}
\begin{subequations} \label{eq:(AcircB)circC!=Acirc(BcircC)}
\begin{align}
	\hspace{-0.9em} \left( \Xint{\mathcolor{gray}{-}}{18}{A}_{\;\! \mathcolor{gray}{z}} ~\textcolor{gray}{\circ}~ \Xint{\mathcolor{gray}{-}}{18}{B}_{\;\! \mathcolor{gray}{z}} \right) \textcolor{gray}{\circ}~ \Xint{\mathcolor{gray}{-}}{18}{C}_{\;\! \mathcolor{gray}{z}} &\xrightarrow[]{\text{\bref{eq:AcircB-a}}} \left( \Xint{\mathcolor{gray}{-}}{18}{A}_{\;\! \mathcolor{gray}{z}} ~\textcolor{gray}{\circ}~ \Xint{\mathcolor{gray}{-}}{18}{B}_{\;\! \mathcolor{gray}{z}} \right)^{\textcolor{Plum}{*}} \left( \mathcolor{gray}{- \bar{k}_{\symup{\rho}}} \right) \textcolor{gray}{*}~ \Xint{\mathcolor{gray}{-}}{18}{C}_{\;\! \mathcolor{gray}{z}} \label{eq:(AcircB)circC!=Acirc(BcircC)a} \\ 
	\hspace{-0.9em} &\xrightarrow[]{\text{\bref{eq:AcircB=BcircA*(-k_rho)-c}}} \Xint{\mathcolor{gray}{-}}{18}{B}_{\;\! \mathcolor{gray}{z}} ~\textcolor{gray}{\circ}~ \Xint{\mathcolor{gray}{-}}{18}{A}_{\;\! \mathcolor{gray}{z}} ~\textcolor{gray}{*}~ \Xint{\mathcolor{gray}{-}}{18}{C}_{\;\! \mathcolor{gray}{z}} \label{eq:(AcircB)circC!=Acirc(BcircC)b} \\ 
	\hspace{-0.9em} &\xrightarrow[]{\text{\bref{eq:AcircB-a}}} \Xint{\mathcolor{gray}{-}}{18}{B}_{\;\! \mathcolor{gray}{z}}^{\textcolor{Plum}{*}} \left( \mathcolor{gray}{- \bar{k}_{\symup{\rho}}} \right) \textcolor{gray}{*}~ \Xint{\mathcolor{gray}{-}}{18}{A}_{\;\! \mathcolor{gray}{z}} ~\textcolor{gray}{*}~ \Xint{\mathcolor{gray}{-}}{18}{C}_{\;\! \mathcolor{gray}{z}} \label{eq:(AcircB)circC!=Acirc(BcircC)c} \\ 
	\hspace{-0.9em} &= \Xint{\mathcolor{gray}{-}}{18}{A}_{\;\! \mathcolor{gray}{z}} ~\textcolor{gray}{*}~ \Xint{\mathcolor{gray}{-}}{18}{B}_{\;\! \mathcolor{gray}{z}}^{\textcolor{Plum}{*}} \left( \mathcolor{gray}{- \bar{k}_{\symup{\rho}}} \right) \textcolor{gray}{*}~ \Xint{\mathcolor{gray}{-}}{18}{C}_{\;\! \mathcolor{gray}{z}} \label{eq:(AcircB)circC!=Acirc(BcircC)d} \\ 
	\hspace{-0.9em} &\neq \Xint{\mathcolor{gray}{-}}{18}{A}_{\;\! \mathcolor{gray}{z}}^{\textcolor{Plum}{*}} \left( \mathcolor{gray}{- \bar{k}_{\symup{\rho}}} \right) \textcolor{gray}{*}~ \Xint{\mathcolor{gray}{-}}{18}{B}_{\;\! \mathcolor{gray}{z}}^{\textcolor{Plum}{*}} \left( \mathcolor{gray}{- \bar{k}_{\symup{\rho}}} \right) \textcolor{gray}{*}~ \Xint{\mathcolor{gray}{-}}{18}{C}_{\;\! \mathcolor{gray}{z}} \xrightarrow[]{\text{\bref{eq:AcircB-a}}} \left( \Xint{\mathcolor{gray}{-}}{18}{A}_{\;\! \mathcolor{gray}{z}} ~\textcolor{gray}{*}~ \Xint{\mathcolor{gray}{-}}{18}{B}_{\;\! \mathcolor{gray}{z}} \right) \textcolor{gray}{\circ}~ \Xint{\mathcolor{gray}{-}}{18}{C}_{\;\! \mathcolor{gray}{z}} \label{eq:(AcircB)circC!=Acirc(BcircC)e} \\ 
	\hspace{-0.9em} &\xrightarrow[]{\text{\bref{eq:AcircB-a}}} \Xint{\mathcolor{gray}{-}}{18}{A}_{\;\! \mathcolor{gray}{z}}^{\textcolor{Plum}{*}} \left( \mathcolor{gray}{- \bar{k}_{\symup{\rho}}} \right) \textcolor{gray}{*} \left( \Xint{\mathcolor{gray}{-}}{18}{B}_{\;\! \mathcolor{gray}{z}} ~\textcolor{gray}{\circ}~ \Xint{\mathcolor{gray}{-}}{18}{C}_{\;\! \mathcolor{gray}{z}} \right) \label{eq:(AcircB)circC!=Acirc(BcircC)f} \\
	\hspace{-0.9em} &\xrightarrow[]{\text{\bref{eq:AcircB-a}}} \Xint{\mathcolor{gray}{-}}{18}{A}_{\;\! \mathcolor{gray}{z}} ~\textcolor{gray}{\circ} \left( \Xint{\mathcolor{gray}{-}}{18}{B}_{\;\! \mathcolor{gray}{z}} ~\textcolor{gray}{\circ}~ \Xint{\mathcolor{gray}{-}}{18}{C}_{\;\! \mathcolor{gray}{z}} \right) \label{eq:(AcircB)circC!=Acirc(BcircC)g}~. 
\end{align}
\end{subequations}

\marginLeft[-2.4em]{ssec:OR_spectrum+DFG_discrete}\subsection{下转换过程:脉冲光整流、连续光差频}\label{ssec:OR_spectrum+DFG_discrete}

考虑\textcolor{NavyBlue}{脉冲光}谱内(\textcolor{Maroon}{自})\textcolor{Maroon}{差频},即\textcolor{Maroon}{光整流}过程,若不纳入其后续级联的\textcolor{Maroon}{电光效应},则该二阶\textcolor{Plum}{非线性}过程的\textcolor{gray}{频率}\textcolor{Maroon}{守恒}方程\Footnote{也可写成\textcolor{Plum}{加}/\textcolor{Plum}{和}的形式 $\left( \textcolor{gray}{\omega'} + \textcolor{gray}{\omega} \right) - \textcolor{gray}{\omega'} \to \textcolor{gray}{\omega} > \textcolor{gray}{0}$,但有时\textcolor{Plum}{减}/\textcolor{Plum}{差}更贴近\textcolor{Plum}{卷积}或\textcolor{Plum}{相关运算}的数学定义。}为 $ \textcolor{gray}{\omega'} - \left( \textcolor{gray}{\omega'}-\textcolor{gray}{\omega} \right) \to \textcolor{gray}{\omega} > \textcolor{gray}{0}$;对于该\textcolor{Maroon}{下转换}过程,波动方程 \bref{eq:simplify8-scalar-g-modulus} 右侧\textcolor{Plum}{非线性}\textcolor{NavyBlue}{波源}项 $\Xint{\mathcolor{gray}{-}}{25}{\bar{P}}^{\;\! \mathcolor{gray}{\omega} \textcolor{PineGreen}{\hat{1}} }_{\;\! \mathcolor{gray}{z}  \textcolor{Maroon}{(2)}} = \mathcolor{gray}{\mathcal F} \left[ {\bar{P}}^{\;\! \mathcolor{gray}{\omega} \textcolor{PineGreen}{\hat{1}} }_{\;\! \mathcolor{gray}{z}  \textcolor{Maroon}{(2)}} \right]$ 变为
\begin{subequations} \label{eq:DP^(2)-1_32-spectrum-DFG}
\begin{align}
	\Xint{\mathcolor{gray}{-}}{30}{P}^{\;\! \textcolor{PineGreen}{\hat{1}} \mathcolor{gray}{\omega} }_{\;\! \hat{1}\mathcolor{gray}{z} \textcolor{Maroon}{(2)} } &\xrightarrow[]{\text{$\sim$\bref{eq:D_wkrho}}} \mathcolor{gray}{\mathcal F} \left[ {\chi}^{\;\! \textcolor{PineGreen}{\hat{1}} \mathcolor{gray}{\omega} \hat{3} \hat{2} \textcolor{Plum}{*} }_{\;\! \hat{1} \mathcolor{gray}{z} \textcolor{PineGreen}{\hat{3} \hat{2}} \textcolor{Maroon}{(2)}} \right] \mathcolor{gray}{*} \mathcolor{gray}{\mathcal F} \left[ E^{\;\!\textcolor{PineGreen}{\hat{3}} \mathcolor{gray}{\omega}}_{\;\! \hat{3} \mathcolor{gray}{z}} ~\mathcolor{gray}{\widetilde *}~ E^{\;\!\textcolor{PineGreen}{\hat{2}} \mathcolor{gray}{- \omega} \textcolor{Plum}{*}}_{\;\! \hat{2} \mathcolor{gray}{z}} \right] \label{eq:DP^(2)-1_32-spectrum-DFG1} \\
	&\xrightarrow[\text{$\sim$\bref{eq:components-chi2-modulate}}]{\text{\bref{eq:F[B*]}}} \mathcolor{gray}{\mathcal F} \left[ {\chi}^{\;\! \textcolor{PineGreen}{\hat{1}} \mathcolor{gray}{\omega} \hat{3} \hat{2} \textcolor{Plum}{*} }_{\;\! \hat{1} \textcolor{Maroon}{(2)} \textcolor{PineGreen}{\hat{3} \hat{2}}} {M}^{\;\! \mathcolor{gray}{\omega} \hat{3} \hat{2} \textcolor{Plum}{*} }_{\;\! \hat{1} \mathcolor{gray}{z} \textcolor{Maroon}{(2)} } \right] \mathcolor{gray}{*} \left[ \Xint{\mathcolor{gray}{-}}{295}{E}^{\;\!\textcolor{PineGreen}{\hat{3}} \mathcolor{gray}{\omega}}_{\;\! \hat{3} \mathcolor{gray}{z}} ~\mathcolor{gray}{\widetilde \circledast}~ \Xint{\mathcolor{gray}{-}}{295}{E}^{\;\!\textcolor{PineGreen}{\hat{2}} \mathcolor{gray}{- \omega} \textcolor{Plum}{*}}_{\;\! \hat{2} \mathcolor{gray}{z}} \left( \mathcolor{gray}{- \bar{k}_{\symup{\rho}}} \right) \right] \label{eq:DP^(2)-1_32-spectrum-DFG2} \\
	&\xrightarrow[]{\text{\bref{eq:AcircB-a}}} {\chi}^{\;\! \textcolor{PineGreen}{\hat{1}} \mathcolor{gray}{\omega} \hat{3} \hat{2} \textcolor{Plum}{*} }_{\;\! \hat{1} \textcolor{Maroon}{(2)} \textcolor{PineGreen}{\hat{3} \hat{2}}} ~\mathcolor{gray}{\mathcal F} \left[ M^{\;\! \mathcolor{gray}{\omega} \hat{3} \hat{2} \textcolor{Plum}{*} }_{\;\! \hat{1} \mathcolor{gray}{z} \textcolor{Maroon}{(2)} } \right] \mathcolor{gray}{*} \left( \Xint{\mathcolor{gray}{-}}{295}{E}^{\;\!\textcolor{PineGreen}{\hat{2}} \mathcolor{gray}{\omega} }_{\;\! \hat{2} \mathcolor{gray}{z}} ~\mathcolor{gray}{\widetilde \circledcirc}~ \Xint{\mathcolor{gray}{-}}{295}{E}^{\;\!\textcolor{PineGreen}{\hat{3}} \mathcolor{gray}{\omega}}_{\;\! \hat{3} \mathcolor{gray}{z}} \right) \label{eq:DP^(2)-1_32-spectrum-DFG3} \\
	&\xrightarrow[]{\text{\bref{eq:IFT-z}}} {\chi}^{\;\! \textcolor{PineGreen}{\hat{1}} \mathcolor{gray}{\omega} \hat{3} \hat{2} \textcolor{Plum}{*} }_{\;\! \hat{1} \textcolor{Maroon}{(2)} \textcolor{PineGreen}{\hat{3} \hat{2}}} ~\mathcolor{gray}{\mathcal F_{z}^{-1}} \left[ \mathcolor{gray}{\mathcal F_{\bar{k}}} \left[ M^{\;\! \mathcolor{gray}{\omega} \hat{3} \hat{2} \textcolor{Plum}{*} }_{\;\! \hat{1} \mathcolor{gray}{z} \textcolor{Maroon}{(2)} } \right] \right] \mathcolor{gray}{*} \left( \Xint{\mathcolor{gray}{-}}{295}{E}^{\;\!\textcolor{PineGreen}{\hat{2}} \mathcolor{gray}{\omega} }_{\;\! \hat{2} \mathcolor{gray}{z}} ~\mathcolor{gray}{\widetilde \circledcirc}~ \Xint{\mathcolor{gray}{-}}{295}{E}^{\;\!\textcolor{PineGreen}{\hat{3}} \mathcolor{gray}{\omega}}_{\;\! \hat{3} \mathcolor{gray}{z}} \right) \label{eq:DP^(2)-1_32-spectrum-DFG4} \\
	&= {\chi}^{\;\! \textcolor{PineGreen}{\hat{1}} \mathcolor{gray}{\omega} \hat{3} \hat{2} \textcolor{Plum}{*} }_{\;\! \hat{1} \textcolor{Maroon}{(2)} \textcolor{PineGreen}{\hat{3} \hat{2}}} ~\mathcolor{gray}{\mathcal F_{z}^{-1}} \left[ \mathcolor{gray}{\mathcal F_{\bar{k}}} \left[ M^{\;\! \mathcolor{gray}{\omega} \hat{3} \hat{2} \textcolor{Plum}{*} }_{\;\! \hat{1} \mathcolor{gray}{z} \textcolor{Maroon}{(2)} } \right] \mathcolor{gray}{*} \left( \Xint{\mathcolor{gray}{-}}{295}{E}^{\;\!\textcolor{PineGreen}{\hat{2}} \mathcolor{gray}{\omega} }_{\;\! \hat{2} \mathcolor{gray}{z}} ~\mathcolor{gray}{\widetilde \circledcirc}~ \Xint{\mathcolor{gray}{-}}{295}{E}^{\;\!\textcolor{PineGreen}{\hat{3}} \mathcolor{gray}{\omega}}_{\;\! \hat{3} \mathcolor{gray}{z}} \right) \right] \label{eq:DP^(2)-1_32-spectrum-DFG5} \\
	&\xrightarrow[]{\text{\bref{eq:F[B*]}}} {\chi}^{\;\! \textcolor{PineGreen}{\hat{1}} \mathcolor{gray}{\omega} \hat{3} \hat{2} \textcolor{Plum}{*} }_{\;\! \hat{1} \textcolor{Maroon}{(2)} \textcolor{PineGreen}{\hat{3} \hat{2}}} ~\mathcolor{gray}{\mathcal F_{z}^{-1}} \left[ \Xint{\mathcolor{gray}{-}}{18}{M}^{\;\! \mathcolor{gray}{\omega} \hat{3} \hat{2} \textcolor{Plum}{*} }_{\;\! \hat{1} \mathcolor{gray}{- k_{\symup{z}}} \textcolor{Maroon}{(2)} } \left( \mathcolor{gray}{- \bar{k}_{\symup{\rho}}} \right) \mathcolor{gray}{*} \left( \Xint{\mathcolor{gray}{-}}{295}{E}^{\;\!\textcolor{PineGreen}{\hat{2}} \mathcolor{gray}{\omega} }_{\;\! \hat{2} \mathcolor{gray}{z}} ~\mathcolor{gray}{\widetilde \circledcirc}~ \Xint{\mathcolor{gray}{-}}{295}{E}^{\;\!\textcolor{PineGreen}{\hat{3}} \mathcolor{gray}{\omega}}_{\;\! \hat{3} \mathcolor{gray}{z}} \right) \right] \label{eq:DP^(2)-1_32-spectrum-DFG6} \\
	&\xrightarrow[\text{\bref{eq:IFT*-z}}]{\text{\bref{eq:AcircB-a}}} {\chi}^{\;\! \textcolor{PineGreen}{\hat{1}} \mathcolor{gray}{\omega} \hat{3} \hat{2} \textcolor{Plum}{*} }_{\;\! \hat{1} \textcolor{Maroon}{(2)} \textcolor{PineGreen}{\hat{3} \hat{2}}} ~\mathcolor{gray}{\mathcal F_{z}^{-\textcolor{Plum}{*}}} \left[ \Xint{\mathcolor{gray}{-}}{18}{M}^{\;\! \mathcolor{gray}{\omega} \hat{3} \hat{2} }_{\;\! \hat{1} \mathcolor{gray}{k_{\symup{z}}} \textcolor{Maroon}{(2)} } \mathcolor{gray}{\circ} \left( \Xint{\mathcolor{gray}{-}}{295}{E}^{\;\!\textcolor{PineGreen}{\hat{2}} \mathcolor{gray}{\omega} }_{\;\! \hat{2} \mathcolor{gray}{z}} ~\mathcolor{gray}{\widetilde \circledcirc}~ \Xint{\mathcolor{gray}{-}}{295}{E}^{\;\!\textcolor{PineGreen}{\hat{3}} \mathcolor{gray}{\omega}}_{\;\! \hat{3} \mathcolor{gray}{z}} \right) \right] ~, \label{eq:DP^(2)-1_32-spectrum-DFG7}
\end{align}
\end{subequations}
其中,类似 $"\mathcolor{gray}{*}"$ 之于 $"\mathcolor{gray}{\widetilde \circledast}"$ 地,在 \bref{eq:DP^(2)-1_32-spectrum-DFG3} 中定义了 $\mathcolor{gray}{\bar{k}_{\symup{\rho}}}$ 域\textcolor{Plum}{互相关}算符 $"\mathcolor{gray}{\circ}"$ 所对应的 $\mathcolor{gray}{\omega}, \mathcolor{gray}{\bar{k}_{\symup{\rho}}}$ 域的\textcolor{Plum}{互相关}算符 $"\mathcolor{gray}{\widetilde \circledcirc}"$;为了省略对\textcolor{Plum}{互相关}运算及其\textcolor{NavyBlue}{对象}的\textcolor{Plum}{括号},规定 ``$\mathcolor{gray}{*},\mathcolor{gray}{\circ}$'' 二者地位平等,且 ``$\mathcolor{gray}{\widetilde \circledast},\mathcolor{gray}{\widetilde \circledcirc}$'' 二者也地位平等,对应\textbf{须遵从类似 \byperref{OperatorSequence}{前处} 的\textcolor{Plum}{积分顺序}:“$\mathcolor{gray}{\widetilde \circledcirc}$” 的 $\mathcolor{gray}{\bar{k}_{\symup{\rho}}}$ 域 $\to$ $\mathcolor{gray}{\bar{k}_{\symup{\rho}}}$ 域的 “$\mathcolor{gray}{\circ}$” $\to$ $\mathcolor{gray}{k_{\symup{z}}}$ 域的 $\mathcolor{gray}{\mathcal F^{-1}_z} \left[ \cdot \right]$ $\to$ “$\mathcolor{gray}{\widetilde \circledcirc}$” 的 $\mathcolor{gray}{\omega}$ 域(即 $\mathcolor{gray}{\omega}$ 域的 ``~$\mathcolor{gray}{\widetilde \circ}$~'')}。这使得有些括号可以省略,如 \bref{eq:DP^(2)-1_32-spectrum-DFG7} 中的小括号,至 \bref{eq:DP^(2)-1_32-spectrum-G1}。

然而,有些\textcolor{Plum}{括号}不能省略,见 \bref{eq:(AcircB)circC!=Acirc(BcircC)e}。这是因为,当\textcolor{Plum}{互相关}与\textcolor{Plum}{卷积}同时出现时,对于同为 $\mathcolor{gray}{\bar{k}_{\symup{\rho}}}$ 域和/或 $\mathcolor{gray}{\omega}$ 域的、同层次的\textcolor{Plum}{互相关}和\textcolor{Plum}{卷积},不要求\textcolor{Plum}{互相关}的\textcolor{Plum}{优先级}高于\textcolor{Plum}{卷积}。注意,从 \bref{eq:DP^(2)-1_32-spectrum-DFG2} 到 \bref{eq:DP^(2)-1_32-spectrum-DFG3} 可以发现,\textcolor{Plum}{互相关}的\textbf{解读顺序},相对于\textcolor{Plum}{卷积}而言,是\textbf{相反的}。

之后会提到,\bref{eq:DP^(2)-1_32-spectrum-DFG7} 中的 $"\mathcolor{gray}{\widetilde \circledcirc}"$ 中的 $"~\mathcolor{gray}{\widetilde \circ}~"$ 和 $"\mathcolor{gray}{\circ}"$,二者均会用到 2 条\textcolor{Plum}{互相关}定义/运算规则 \bref{eq:AcircB-a} 和 \bref{eq:AcircB-b},并且在不同的情况下使用不同的规则,这一点对于理解下转换版本的\textcolor{Plum}{非线性}\textcolor{Plum}{卷积}过程的\textcolor{gray}{横向波矢守恒}、\textcolor{PineGreen}{纵向相位匹配}以及\textcolor{gray}{频率守恒方程}至关重要。

此外,\bref{eq:DP^(2)-1_32-spectrum-DFG7} 中定义了 $\mathcolor{gray}{\mathcal F_{z}^{-\mathcolor{Plum}{*}}}$ 的核函数 $\mathbb{e}^{-\mathbb{i}\mathcolor{gray}{k_{\symup{z}}} \mathcolor{gray}{z}}$ 共轭于 $\mathcolor{gray}{\mathcal F_{z}^{-1}}$ 的核函数 $\mathbb{e}^{\mathbb{i}\mathcolor{gray}{k_{\symup{z}}} \mathcolor{gray}{z}}$ 的空域 $\mathcolor{gray}{z} \in \mathcolor{gray}{\bar{\mathbb{R}}_{\textcolor{Plum}{1}}}$ 向 1 维\textcolor{Plum}{傅立叶正} $\mathcolor{gray}{\mathcal F_{z}^{\mathcolor{Plum}{*}}}$、\textcolor{Plum}{逆} $\mathcolor{gray}{\mathcal F_{z}^{-\mathcolor{Plum}{*}}}$ \textcolor{Plum}{变换对}:
\begin{subequations} \label{eq:FT*-z_kz}
\begin{align}
	\mathcolor{gray}{\mathcal F_{z}^{\mathcolor{Plum}{*}}} \left[ \cdot \right] &:= \frac{ 1 }{ 2\symup{\pi} } \mathcolor{gray}{\int_{-\infty}^{+\infty}} \cdot~ \mathbb{e}^{\mathbb{i}\mathcolor{gray}{k_{\symup{z}}} \mathcolor{gray}{z}} \hphantom{^-} \mathbb{d}\mathcolor{gray}{z} ~, \label{eq:FT*-kz} \\
	\mathcolor{gray}{\mathcal F_{z}^{-\mathcolor{Plum}{*}}} \left[ \cdot \right] &:= \hphantom{\frac{ 1 }{ 2\symup{\pi} }} \mathcolor{gray}{\int_{-\infty}^{+\infty}} \cdot~ \mathbb{e}^{-\mathbb{i}\mathcolor{gray}{k_{\symup{z}}} \mathcolor{gray}{z}} \mathbb{d}\mathcolor{gray}{k_{\symup{z}}} ~. \label{eq:IFT*-z}
\end{align}
\end{subequations}

将 \bref{eq:DP^(2)-1_32-spectrum-DFG7} 写成左侧\textcolor{Plum}{矢量}、右侧\textcolor{Plum}{半张量}的形式即
\begin{subequations} \label{eq:DP^(2)-1_32-spectrum-G}
\begin{align}
	\Xint{\mathcolor{gray}{-}}{30}{\bar{P}}^{\;\! \mathcolor{gray}{\omega} \textcolor{PineGreen}{\hat{1}} }_{\;\! \mathcolor{gray}{z} \textcolor{Maroon}{(2)} } &\xrightarrow[]{\text{\bref{eq:DP^(2)-1_32-spectrum-DFG7}}} \bar{\chi}^{\;\! \textcolor{PineGreen}{\hat{1}} \mathcolor{gray}{\omega} \hat{3} \hat{2} \textcolor{Plum}{*} }_{\;\! \textcolor{Maroon}{(2)} \textcolor{PineGreen}{\hat{3} \hat{2}}} \odot \mathcolor{gray}{\mathcal F_{z}^{-\textcolor{Plum}{*}}} \left[ \Xint{\mathcolor{gray}{-}}{18}{\bar{M}}^{\;\! \mathcolor{gray}{\omega} \hat{3} \hat{2} }_{\;\! \mathcolor{gray}{k_{\symup{z}}} \textcolor{Maroon}{(2)} } \mathcolor{gray}{\circ} \Xint{\mathcolor{gray}{-}}{295}{E}^{\;\!\textcolor{PineGreen}{\hat{2}} \mathcolor{gray}{\omega} }_{\;\! \hat{2} \mathcolor{gray}{z}} ~\mathcolor{gray}{\widetilde \circledcirc}~ \Xint{\mathcolor{gray}{-}}{295}{E}^{\;\!\textcolor{PineGreen}{\hat{3}} \mathcolor{gray}{\omega}}_{\;\! \hat{3} \mathcolor{gray}{z}} \right] \label{eq:DP^(2)-1_32-spectrum-G1} \\
	&\xrightarrow[]{\text{$\sim$ \bref{eq:components-eigenwave'}}} \bar{\chi}^{\;\! \textcolor{PineGreen}{\hat{1}} \mathcolor{gray}{\omega} \hat{3} \hat{2} \textcolor{Plum}{*} }_{\;\! \textcolor{Maroon}{(2)} \textcolor{PineGreen}{\hat{3} \hat{2}}} \odot \mathcolor{gray}{\mathcal F_{z}^{-\textcolor{Plum}{*}}} \left[ \Xint{\mathcolor{gray}{-}}{18}{\bar{M}}^{\;\! \mathcolor{gray}{\omega} \hat{3} \hat{2} }_{\;\! \mathcolor{gray}{k_{\symup{z}}} \textcolor{Maroon}{(2)} } \mathcolor{gray}{\circ} \left( \Xint{\mathcolor{gray}{-}}{20}{\mathtt{G}}^{\;\! \textcolor{PineGreen}{\hat{2}} \mathcolor{gray}{\omega} }_{\;\! \mathcolor{gray}{z}} \Xint{{}^{}_{\mathcolor{gray}{-}}}{10}{\hat{g}}^{\;\! \textcolor{PineGreen}{\hat{2}} \mathcolor{gray}{\omega} }_{\;\! \hat{2}} \right) ~\mathcolor{gray}{\widetilde \circledcirc}~ \left( \Xint{\mathcolor{gray}{-}}{20}{\mathtt{G}}^{\;\! \textcolor{PineGreen}{\hat{3}} \mathcolor{gray}{\omega} }_{\;\! \mathcolor{gray}{z}} \Xint{{}^{}_{\mathcolor{gray}{-}}}{10}{\hat{g}}^{\;\! \textcolor{PineGreen}{\hat{3}} \mathcolor{gray}{\omega} }_{\;\! \hat{3}} \right) \right] \label{eq:DP^(2)-1_32-spectrum-G2} \\ 
	&\xrightarrow[]{\text{\bref{eq:scalar_nonlinear_drive-spectrum}}} \bar{\chi}^{\;\! \textcolor{Maroon}{(2)} \textcolor{PineGreen}{\hat{1}} \textcolor{Plum}{*} }_{\;\! \mathcolor{gray}{\omega} \hat{3} \hat{2} \textcolor{PineGreen}{\hat{3} \hat{2}} } ~ {\hat{g}}^{\;\! \mathcolor{gray}{\omega} }_{\;\! \hat{2} \textcolor{PineGreen}{\hat{2}}} ~\mathcolor{gray}{\widetilde \circ}~ {\hat{g}}^{\;\! \mathcolor{gray}{\omega} }_{\;\! \hat{3} \textcolor{PineGreen}{\hat{3}}} \odot \mathcolor{gray}{\mathcal F_{z}^{-\textcolor{Plum}{*}}} \left[ \Xint{\mathcolor{gray}{-}}{18}{\bar{M}}^{\;\! \mathcolor{gray}{\omega} \hat{3} \hat{2} }_{\;\! \mathcolor{gray}{k_{\symup{z}}} \textcolor{Maroon}{(2)} } \mathcolor{gray}{\circ} \Xint{\mathcolor{gray}{-}}{20}{\mathtt{G}}^{\;\! \textcolor{PineGreen}{\hat{2}} \mathcolor{gray}{\omega} }_{\;\! \mathcolor{gray}{z}} ~\mathcolor{gray}{\widetilde \circledcirc}~ \Xint{\mathcolor{gray}{-}}{20}{\mathtt{G}}^{\;\! \textcolor{PineGreen}{\hat{3}} \mathcolor{gray}{\omega} }_{\;\! \mathcolor{gray}{z}} \right] \label{eq:DP^(2)-1_32-spectrum-G3} \\
	&\xrightarrow[]{\text{\bref{eq:scalar_chi2_modulation}}} \bar{\chi}^{\;\! \mathcolor{gray}{\omega} \textcolor{PineGreen}{\hat{1}} \hat{3} \hat{2} \textcolor{Plum}{*} }_{\;\! \textcolor{Maroon}{(2)} \textcolor{PineGreen}{\hat{3}} \textcolor{PineGreen}{\hat{2}} } ~ {\hat{g}}^{\;\! \mathcolor{gray}{\omega} }_{\;\! \hat{2} \textcolor{PineGreen}{\hat{2}}} ~\mathcolor{gray}{\widetilde \circ}~ {\hat{g}}^{\;\! \mathcolor{gray}{\omega} }_{\;\! \hat{3} \textcolor{PineGreen}{\hat{3}}} \odot \mathcolor{gray}{\mathcal F_{z}^{-\textcolor{Plum}{*}}} \left[ M^{\;\! \mathcolor{gray}{\omega} }_{\;\! \mathcolor{gray}{k_{\symup{z}}} \textcolor{Maroon}{(2)} } \mathcolor{gray}{\circ} \Xint{\mathcolor{gray}{-}}{20}{\mathtt{G}}^{\;\! \textcolor{PineGreen}{\hat{2}} \mathcolor{gray}{\omega} }_{\;\! \mathcolor{gray}{z}} ~\mathcolor{gray}{\widetilde \circledcirc}~ \Xint{\mathcolor{gray}{-}}{20}{\mathtt{G}}^{\;\! \textcolor{PineGreen}{\hat{3}} \mathcolor{gray}{\omega} }_{\;\! \mathcolor{gray}{z}} \right] ~, \label{eq:DP^(2)-1_32-spectrum-G4}
\end{align}
\end{subequations}
其中,在“\textbf{标量\textcolor{Plum}{非线性}\textcolor{NavyBlue}{波源}}”条件 \bref{eq:scalar_nonlinear_drive} 下,通过 \bref{eq:DP^(2)-1_32-spectrum-G3} 定义了以\textcolor{NavyBlue}{脉冲}\textcolor{Maroon}{光整流}、\textcolor{NavyBlue}{连续光}\textcolor{Maroon}{差频}为代表的\textcolor{Maroon}{下转换}过程的\textcolor{NavyBlue}{有效非线性系数}(三阶)张量
\begin{subequations} \label{eq:chieff*}
\begin{align}
	\widetilde{\chi}^{ \textcolor{Maroon}{(2)} \textcolor{PineGreen}{\hat{1}} \mathcolor{gray}{\omega} }_{ \textcolor{NavyBlue}{\text{eff}} \hat{3} \hat{2} \textcolor{PineGreen}{\hat{3} \hat{2}} } &\xrightarrow[\text{$\sim$ \bref{eq:simplify8-scalar-g-modulus}}]{\text{\bref{eq:DP^(2)-1_32-spectrum-G3}}} \Xint{{}^{}_{\mathcolor{gray}{-}}}{10}{\hat{g}}^{\;\! \textcolor{PineGreen}{\hat{1}} \textcolor{Plum}{*}}_{\;\! \mathcolor{gray}{\omega}} \odot \bar{\chi}^{\;\! \textcolor{Maroon}{(2)} \textcolor{PineGreen}{\hat{1}} \textcolor{Plum}{*} }_{\;\! \mathcolor{gray}{\omega} \hat{3} \hat{2} \textcolor{PineGreen}{\hat{3} \hat{2}} } ~ {\hat{g}}^{\;\! \mathcolor{gray}{\omega} }_{\;\! \hat{2} \textcolor{PineGreen}{\hat{2}} } ~\mathcolor{gray}{\widetilde \circ}~ {\hat{g}}^{\;\! \mathcolor{gray}{\omega} }_{\;\! \hat{3} \textcolor{PineGreen}{\hat{3}} } ~, \label{eq:chieff*-spectrum} \\
	\bar{\chi}^{ \textcolor{Maroon}{(2)} \textcolor{PineGreen}{\hat{1}} }_{\textcolor{NavyBlue}{\text{eff}} \hat{3} \textcolor{PineGreen}{\hat{3}} \hat{2} \textcolor{PineGreen}{\hat{2}} } &\xrightarrow[\text{$\sim$ \bref{eq:simplify8-scalar-g-modulus}}]{\text{$\sim$ \bref{eq:DP^(2)-1_32-spectrum-G3}}} \Xint{{}^{}_{\mathcolor{gray}{-}}}{10}{\hat{g}}^{\;\! \textcolor{PineGreen}{\hat{1}} \textcolor{Plum}{*}}_{\;\! } \odot \bar{\chi}^{\;\! \textcolor{PineGreen}{\hat{1}} \textcolor{Maroon}{(2)} \textcolor{Plum}{*} }_{\;\! \hat{3} \textcolor{PineGreen}{\hat{3}} \hat{2} \textcolor{PineGreen}{\hat{2}}} ~ {\hat{g}}_{\;\! \hat{3} \textcolor{PineGreen}{\hat{3}}} ~ {\hat{g}}^{\;\! \textcolor{Plum}{*}}_{\;\! \hat{2} \textcolor{PineGreen}{\hat{2}} } ~, \label{eq:chieff*-discrete}
\end{align}
\end{subequations}
注意,其中的 $\chi$ 头上的波浪符号是\textbf{黑色的} `$\sim$' 而不是\textbf{\textcolor{gray}{灰色的}} `$\mathcolor{gray}{\sim}$',意味着 $\widetilde{\chi}$ 既是矢量,又参与 \textcolor{gray}{$\omega$ 域}的一维\textcolor{Plum}{卷积积分}。这源于,将 \bref{eq:DP^(2)-1_32-spectrum-G3} 代入 \bref{eq:simplify8-scalar-g-modulus},所得的以\textcolor{NavyBlue}{脉冲}\textcolor{Maroon}{光整流}、\textcolor{NavyBlue}{连续光}\textcolor{Maroon}{差频}为代表的\textcolor{Maroon}{下转换}电场\textcolor{PineGreen}{本征复振幅}方程
\begin{subequations} \label{eq:scalar-g-modulus-P-chieff*}
\begin{align}
	\mathcolor{gray}{\nabla_z} \Xint{\begin{smallmatrix} ~ \\ {}^{}_{\mathcolor{gray}{-}} \\ ~ \end{smallmatrix}}{09}{\mathtt{g}}^{\;\!\mathcolor{gray}{\omega} \textcolor{PineGreen}{\hat{1}}}_{\;\! \mathcolor{gray}{z}} &\xrightarrow[\text{$\sim$ \bref{eq:simplify8-scalar-g-modulus}}]{\text{\bref{eq:DP^(2)-1_32-spectrum-G3}}} \mathbb{i} k_{\textcolor{Maroon}{\mathsf{o}} \mathcolor{gray}{\omega}}^{\;\! 2} \frac{ \widetilde{\chi}^{ \textcolor{Maroon}{(2)} \textcolor{PineGreen}{\hat{1}} \mathcolor{gray}{\omega} \mathsf{\textcolor{Plum}{T}} }_{ \textcolor{NavyBlue}{\text{eff}} \hat{3} \hat{2} \textcolor{PineGreen}{\hat{3} \hat{2}} } \cdot \mathcolor{gray}{\mathcal F_{z}^{-\textcolor{Plum}{*}}} \left[ \Xint{\mathcolor{gray}{-}}{18}{\bar{M}}^{\;\! \mathcolor{gray}{\omega} \hat{3} \hat{2} }_{\;\! \mathcolor{gray}{k_{\symup{z}}} \textcolor{Maroon}{(2)} } \mathcolor{gray}{\circ} \Xint{\mathcolor{gray}{-}}{15}{\mathtt{G}}^{\;\! \textcolor{PineGreen}{\hat{2}} \mathcolor{gray}{\omega} }_{\;\! \mathcolor{gray}{z}} ~\mathcolor{gray}{\widetilde \circledcirc}~ \Xint{\mathcolor{gray}{-}}{15}{\mathtt{G}}^{\;\! \textcolor{PineGreen}{\hat{3}} \mathcolor{gray}{\omega} }_{\;\! \mathcolor{gray}{z}} \right] }{ 2 \lvert \Xint{{}^{}_{\mathcolor{gray}{-}}}{10}{\hat{g}}^{\;\! \textcolor{PineGreen}{\hat{1}}}_{\;\! \mathcolor{gray}{\omega}} \rvert^2 \Xint{\begin{smallmatrix} ~ \\ {}^{}_{\mathcolor{gray}{-}} \\ ~ \end{smallmatrix}}{15}{k}_{\;\! \symup{z}}^{\;\! \mathcolor{gray}{\omega} \textcolor{PineGreen}{\hat{1}}} \mathbb{e}^{\mathbb{i} \Xint{\begin{smallmatrix} ~ \\ {}^{}_{\mathcolor{gray}{-}} \\ ~ \end{smallmatrix}}{15}{k}_{\symup{z}}^{\;\! \mathcolor{gray}{\omega} \textcolor{PineGreen}{\hat{1}}} \mathcolor{gray}{z}}} ~, \label{eq:scalar-g-modulus-P-chieff*-spectrum} \\
	\mathcolor{gray}{\nabla_z} \Xint{\begin{smallmatrix} ~ \\ {}^{}_{\mathcolor{gray}{-}} \\ ~ \end{smallmatrix}}{09}{\mathtt{g}}^{\;\! \textcolor{PineGreen}{\hat{1}}}_{\;\! \mathcolor{gray}{z}} &\xrightarrow[\text{$\sim$ \bref{eq:simplify8-scalar-g-modulus}}]{\text{$\sim$ \bref{eq:DP^(2)-1_32-spectrum-G3}}} \mathbb{i} k_{\textcolor{Maroon}{\mathsf{o}} \mathcolor{gray}{1}}^{\;\! 2} \frac{ \bar{\chi}^{ \textcolor{Maroon}{(2)} \textcolor{PineGreen}{\hat{1}}  \mathsf{\textcolor{Plum}{T}} }_{\textcolor{NavyBlue}{\text{eff}} \hat{3} \textcolor{PineGreen}{\hat{3}} \hat{2} \textcolor{PineGreen}{\hat{2}} } \cdot \mathcolor{gray}{\mathcal F_{z}^{-\textcolor{Plum}{*}}} \left[ \Xint{\mathcolor{gray}{-}}{18}{\bar{M}}^{\;\! \mathcolor{gray}{1} \hat{3} \hat{2} }_{\;\! \mathcolor{gray}{k_{\symup{z}}} \textcolor{Maroon}{(2)} } \mathcolor{gray}{\circ} \left( \Xint{\mathcolor{gray}{-}}{15}{\mathtt{G}}^{\;\! \textcolor{PineGreen}{\hat{2}} }_{\;\! \mathcolor{gray}{z}} \mathcolor{gray}{\circ} \Xint{\mathcolor{gray}{-}}{15}{\mathtt{G}}^{\;\! \textcolor{PineGreen}{\hat{3}} }_{\;\! \mathcolor{gray}{z}} \right) \right]}{ 2 \lvert \Xint{{}^{}_{\mathcolor{gray}{-}}}{10}{\hat{g}}^{\;\! \textcolor{PineGreen}{\hat{1}}} \rvert^2 \Xint{\begin{smallmatrix} ~ \\ {}^{}_{\mathcolor{gray}{-}} \\ ~ \end{smallmatrix}}{15}{k}_{\;\! \symup{z}}^{\;\!  \textcolor{PineGreen}{\hat{1}}} \mathbb{e}^{\mathbb{i} \Xint{\begin{smallmatrix} ~ \\ {}^{}_{\mathcolor{gray}{-}} \\ ~ \end{smallmatrix}}{15}{k}_{\symup{z}}^{\;\!  \textcolor{PineGreen}{\hat{1}}} \mathcolor{gray}{z}}} ~, \label{eq:scalar-g-modulus-P-chieff*-discrete}
\end{align}
\end{subequations}
注,\bref{eq:scalar-g-modulus-P-chieff*} 中涉及\textcolor{Plum}{逆变}矢量 $\widetilde{\chi}^{ \textcolor{Maroon}{(2)} \textcolor{PineGreen}{\hat{1}} \mathcolor{gray}{\omega} \mathsf{\textcolor{Plum}{T}} }_{ \textcolor{NavyBlue}{\text{eff}} \hat{3} \hat{2} \textcolor{PineGreen}{\hat{3} \hat{2}} }, \bar{\chi}^{ \textcolor{Maroon}{(2)} \textcolor{PineGreen}{\hat{1}}  \mathsf{\textcolor{Plum}{T}} }_{\textcolor{NavyBlue}{\text{eff}} \hat{3} \textcolor{PineGreen}{\hat{3}} \hat{2} \textcolor{PineGreen}{\hat{2}} }$ 与\textcolor{Plum}{协变}矢量 $\Xint{\mathcolor{gray}{-}}{18}{\bar{M}}^{\;\! \mathcolor{gray}{\omega} \hat{3} \hat{2} }_{\;\! \mathcolor{gray}{k_{\symup{z}}} \textcolor{Maroon}{(2)} }, \Xint{\mathcolor{gray}{-}}{18}{\bar{M}}^{\;\! \mathcolor{gray}{1} \hat{3} \hat{2} }_{\;\! \mathcolor{gray}{k_{\symup{z}}} \textcolor{Maroon}{(2)} }$ 的点积。

在“\textbf{标量\textcolor{Plum}{非线性}\textcolor{NavyBlue}{波源}}” \bref{eq:scalar_nonlinear_drive} 和“\textbf{标量场 $\chi^{\;\! \mathcolor{gray}{\omega} }_{\;\! \mathcolor{gray}{z} \textcolor{Maroon}{(2)}}$ \textcolor{NavyBlue}{调制}}” \bref{eq:scalar_chi2_modulation} 这 2 个条件的共同作用下,\bref{eq:DP^(2)-1_32-spectrum-G4,eq:simplify8-scalar-g-modulus} 所共同辅助定义的,以\textcolor{NavyBlue}{脉冲}\textcolor{Maroon}{光整流}、\textcolor{NavyBlue}{连续光}\textcolor{Maroon}{差频}为代表的\textcolor{Maroon}{下转换}过程的\textcolor{NavyBlue}{有效非线性系数}张量 \bref{eq:chieff*} 退化为标量
\begin{subequations} \label{eq:chieff*-scalar}
\begin{align}
	\textcolor{gray}{\widetilde{\textcolor{black}{\chi}}}^{ \mathcolor{gray}{\omega} \textcolor{PineGreen}{\hat{1}} \textcolor{Maroon}{(2)} }_{ \textcolor{NavyBlue}{\text{eff}} \textcolor{PineGreen}{\hat{3} \hat{2}} } &\xrightarrow[\text{\bref{eq:scalar-g-modulus-P-chieff*-spectrum}}]{\text{\bref{eq:scalar_chi2_modulation}}} \Xint{{}^{}_{\mathcolor{gray}{-}}}{10}{\hat{g}}^{\;\! \textcolor{PineGreen}{\hat{1}} \textcolor{Plum}{\dag}}_{\;\! \mathcolor{gray}{\omega}} \cdot \bar{\chi}^{\;\! \textcolor{PineGreen}{\hat{1}} \mathcolor{gray}{\omega} \hat{3} \hat{2} \textcolor{Plum}{*} }_{\;\!  \textcolor{Maroon}{(2)} \textcolor{PineGreen}{\hat{3} \hat{2}} } ~ {\hat{g}}^{\;\! \mathcolor{gray}{\omega} }_{\;\! \hat{2} \textcolor{PineGreen}{\hat{2}} } ~\mathcolor{gray}{\widetilde \circ}~ {\hat{g}}^{\;\! \mathcolor{gray}{\omega} }_{\;\! \hat{3} \textcolor{PineGreen}{\hat{3}} } ~, \label{eq:chieff*-scalar-spectrum} \\
	\chi^{ \textcolor{PineGreen}{\hat{1}} \textcolor{Maroon}{(2)} }_{\textcolor{NavyBlue}{\text{eff}} \textcolor{PineGreen}{\hat{3}} \textcolor{PineGreen}{\hat{2}} } &\xrightarrow[\text{\bref{eq:scalar-g-modulus-P-chieff*-discrete}}]{\text{\bref{eq:scalar_chi2_modulation}}} \Xint{{}^{}_{\mathcolor{gray}{-}}}{10}{\hat{g}}^{\;\! \textcolor{PineGreen}{\hat{1}} \textcolor{Plum}{\dag}}_{\;\! } \cdot \bar{\chi}^{\;\! \textcolor{PineGreen}{\hat{1}} \hat{3} \hat{2} \textcolor{Plum}{*} }_{\;\! \textcolor{Maroon}{(2)} \textcolor{PineGreen}{\hat{3} \hat{2}}} ~ {\hat{g}}_{\;\! \hat{3} \textcolor{PineGreen}{\hat{3}}} ~ {\hat{g}}^{\;\! \textcolor{Plum}{*}}_{\;\! \hat{2} \textcolor{PineGreen}{\hat{2}} } ~, \label{eq:chieff*-scalar-discrete}
\end{align}
\end{subequations}
自此,\textcolor{NavyBlue}{脉冲}\textcolor{Maroon}{光整流}、\textcolor{NavyBlue}{连续光}\textcolor{Maroon}{差频}的电场\textcolor{PineGreen}{本征复振幅}方程 \bref{eq:scalar-g-modulus-P-chieff*} 变为
\begin{subequations} \label{eq:scalar-g-modulus-P-chieff*-scalar}
\begin{align}
	\mathcolor{gray}{\nabla_z} \Xint{\begin{smallmatrix} ~ \\ {}^{}_{\mathcolor{gray}{-}} \\ ~ \end{smallmatrix}}{09}{\mathtt{g}}^{\;\!\mathcolor{gray}{\omega} \textcolor{PineGreen}{\hat{1}}}_{\;\! \mathcolor{gray}{z}} &\xrightarrow[\text{\bref{eq:scalar-g-modulus-P-chieff*-spectrum}}]{\text{\bref{eq:scalar_chi2_modulation}}} \mathbb{i} k_{\textcolor{Maroon}{\mathsf{o}} \mathcolor{gray}{\omega}}^{\;\! 2} \frac{ \textcolor{gray}{\widetilde{\textcolor{black}{\chi}}}^{ \mathcolor{gray}{\omega} \textcolor{PineGreen}{\hat{1}} \textcolor{Maroon}{(2)} }_{ \textcolor{NavyBlue}{\text{eff}} \textcolor{PineGreen}{\hat{3} \hat{2}} } ~ \mathcolor{gray}{\mathcal F_{z}^{-\textcolor{Plum}{*}}} \left[ \Xint{\mathcolor{gray}{-}}{18}{M}^{\;\! \mathcolor{gray}{\omega} }_{\;\! \mathcolor{gray}{k_{\symup{z}}} \textcolor{Maroon}{(2)} } \mathcolor{gray}{\circ} \Xint{\mathcolor{gray}{-}}{15}{\mathtt{G}}^{\;\! \textcolor{PineGreen}{\hat{2}} \mathcolor{gray}{\omega} }_{\;\! \mathcolor{gray}{z}} ~\mathcolor{gray}{\widetilde \circledcirc}~ \Xint{\mathcolor{gray}{-}}{15}{\mathtt{G}}^{\;\! \textcolor{PineGreen}{\hat{3}} \mathcolor{gray}{\omega} }_{\;\! \mathcolor{gray}{z}} \right] }{ 2 \lvert \Xint{{}^{}_{\mathcolor{gray}{-}}}{10}{\hat{g}}^{\;\! \textcolor{PineGreen}{\hat{1}}}_{\;\! \mathcolor{gray}{\omega}} \rvert^2 \Xint{\begin{smallmatrix} ~ \\ {}^{}_{\mathcolor{gray}{-}} \\ ~ \end{smallmatrix}}{15}{k}_{\;\! \symup{z}}^{\;\! \mathcolor{gray}{\omega} \textcolor{PineGreen}{\hat{1}}} \mathbb{e}^{\mathbb{i} \Xint{\begin{smallmatrix} ~ \\ {}^{}_{\mathcolor{gray}{-}} \\ ~ \end{smallmatrix}}{15}{k}_{\symup{z}}^{\;\! \mathcolor{gray}{\omega} \textcolor{PineGreen}{\hat{1}}} \mathcolor{gray}{z}}} ~, \label{eq:scalar-g-modulus-P-chieff*-scalar-spectrum} \\
	\mathcolor{gray}{\nabla_z} \Xint{\begin{smallmatrix} ~ \\ {}^{}_{\mathcolor{gray}{-}} \\ ~ \end{smallmatrix}}{09}{\mathtt{g}}^{\;\! \textcolor{PineGreen}{\hat{1}}}_{\;\! \mathcolor{gray}{z}} &\xrightarrow[\text{$\sim$ \bref{eq:scalar-g-modulus-P-chieff*-discrete}}]{\text{\bref{eq:scalar_chi2_modulation}}} \mathbb{i} k_{\textcolor{Maroon}{\mathsf{o}} \mathcolor{gray}{1}}^{\;\! 2} \frac{ \chi^{ \textcolor{PineGreen}{\hat{1}} \textcolor{Maroon}{(2)} }_{\textcolor{NavyBlue}{\text{eff}} \textcolor{PineGreen}{\hat{3}} \textcolor{PineGreen}{\hat{2}} } ~ \mathcolor{gray}{\mathcal F_{z}^{-\textcolor{Plum}{*}}} \left[ \Xint{\mathcolor{gray}{-}}{18}{M}^{\;\! \mathcolor{gray}{1} }_{\;\! \mathcolor{gray}{k_{\symup{z}}} \textcolor{Maroon}{(2)} } \mathcolor{gray}{\circ} \left( \Xint{\mathcolor{gray}{-}}{15}{\mathtt{G}}^{\;\! \textcolor{PineGreen}{\hat{2}} }_{\;\! \mathcolor{gray}{z}} \mathcolor{gray}{\circ} \Xint{\mathcolor{gray}{-}}{15}{\mathtt{G}}^{\;\! \textcolor{PineGreen}{\hat{3}} }_{\;\! \mathcolor{gray}{z}} \right) \right]}{ 2 \lvert \Xint{{}^{}_{\mathcolor{gray}{-}}}{10}{\hat{g}}^{\;\! \textcolor{PineGreen}{\hat{1}}} \rvert^2 \Xint{\begin{smallmatrix} ~ \\ {}^{}_{\mathcolor{gray}{-}} \\ ~ \end{smallmatrix}}{15}{k}_{\;\! \symup{z}}^{\;\!  \textcolor{PineGreen}{\hat{1}}} \mathbb{e}^{\mathbb{i} \Xint{\begin{smallmatrix} ~ \\ {}^{}_{\mathcolor{gray}{-}} \\ ~ \end{smallmatrix}}{15}{k}_{\symup{z}}^{\;\!  \textcolor{PineGreen}{\hat{1}}} \mathcolor{gray}{z}}} ~, \label{eq:scalar-g-modulus-P-chieff*-scalar-discrete}
\end{align}
\end{subequations}
注意,对 \bref{eq:scalar-g-modulus-P-chieff*-scalar-discrete,eq:scalar-g-modulus-P-chieff*-discrete} 添加了类似 \bref{eq:(AcircB)circC!=Acirc(BcircC)g} 的括号,以保证从右往左计算两个\textcolor{Plum}{互相关}。

%\marginLeft[-2.4em]{ssec:3wavemix}\subsection{三波混频、光整流级联电光效应 - 电场本征复振幅方程}\label{ssec:3wavemix}
%\marginLeft[-2.4em]{ssec:3wavemix}\subsection{脉冲光、连续光三波混频 - 电场本征复振幅方程}\label{ssec:3wavemix}
\marginLeft[-2.4em]{ssec:pulse-3wavemix}\subsection{脉冲光三波混频 - 电场本征复振幅方程}\label{ssec:pulse-3wavemix}

结合 \bref{ssec:SHG_spectrum} 中,以\textcolor{NavyBlue}{脉冲光}\textcolor{Maroon}{倍频}、\textcolor{NavyBlue}{脉冲}\textcolor{Maroon}{光整流}后的级联\textcolor{Maroon}{电光效应}为代表的\textcolor{NavyBlue}{超快}\textcolor{Maroon}{上转换}过程的电场\textcolor{PineGreen}{本征复振幅}方程 \bref{eq:simplify8-scalar-g-modulus-P-spectrum} 及其(由 \bref{eq:simplify8-scalar-g-modulus,eq:DP^(2)-1_32-spectrum-DFG7} 构建的)\textcolor{Maroon}{下转换}版本,可以得到仅电子和光子间\textcolor{NavyBlue}{瞬态}相互作用的\textcolor{Maroon}{三波混频}方程
\begin{subequations} \label{eq:3wavemix-scalar-g-EE-spectrum}
\begin{align}
	\mathcolor{gray}{\nabla_z} \Xint{\begin{smallmatrix} ~ \\ {}^{}_{\mathcolor{gray}{-}} \\ ~ \end{smallmatrix}}{09}{\mathtt{g}}^{\;\!\mathcolor{gray}{\omega} \textcolor{PineGreen}{\hat{3}}}_{\;\! \mathcolor{gray}{z}} &\xrightarrow[]{\text{\bref{eq:simplify8-scalar-g-modulus-P-spectrum}}} \mathbb{i} k_{\textcolor{Maroon}{\mathsf{o}} \mathcolor{gray}{\omega}}^{\;\! 2} \frac{\Xint{{}^{}_{\mathcolor{gray}{-}}}{10}{\hat{g}}^{\;\! \hat{3} \textcolor{PineGreen}{\hat{3}} \textcolor{Plum}{*}}_{\;\! \mathcolor{gray}{\omega}} {\chi}^{\;\! \textcolor{PineGreen}{\hat{3}} \mathcolor{gray}{\omega} \hat{1} \hat{2} }_{\;\! \hat{3} \textcolor{Maroon}{(2)} \textcolor{PineGreen}{\hat{1} \hat{2}}} ~ \mathcolor{gray}{\mathcal F_{z}^{-1}} \left[ \Xint{\mathcolor{gray}{-}}{18}{M}^{\;\! \mathcolor{gray}{\omega} \hat{1} \hat{2} }_{\;\! \hat{3} \mathcolor{gray}{k_{\symup{z}}} \textcolor{Maroon}{(2)} } \mathcolor{gray}{*} \Xint{\mathcolor{gray}{-}}{25}{E}^{\;\! \textcolor{PineGreen}{\hat{1}} \mathcolor{gray}{\omega} }_{\;\! \hat{1} \mathcolor{gray}{z}} ~\mathcolor{gray}{\widetilde \circledast}~ \Xint{\mathcolor{gray}{-}}{25}{E}^{\;\! \textcolor{PineGreen}{\hat{2}} \mathcolor{gray}{\omega} }_{\;\! \hat{2} \mathcolor{gray}{z}} \right]}{ 2 \lvert \Xint{{}^{}_{\mathcolor{gray}{-}}}{10}{\hat{g}}^{\;\! \textcolor{PineGreen}{\hat{3}}}_{\;\! \mathcolor{gray}{\omega}} \rvert^2 \Xint{\begin{smallmatrix} ~ \\ {}^{}_{\mathcolor{gray}{-}} \\ ~ \end{smallmatrix}}{15}{k}_{\;\! \symup{z}}^{\;\! \mathcolor{gray}{\omega} \textcolor{PineGreen}{\hat{3}}} \mathbb{e}^{\mathbb{i} \Xint{\begin{smallmatrix} ~ \\ {}^{}_{\mathcolor{gray}{-}} \\ ~ \end{smallmatrix}}{15}{k}_{\symup{z}}^{\;\! \mathcolor{gray}{\omega} \textcolor{PineGreen}{\hat{3}}} \mathcolor{gray}{z}}} ~, \label{eq:up-scalar-g-EE-312-spectrum} \\
	\mathcolor{gray}{\nabla_z} \Xint{\begin{smallmatrix} ~ \\ {}^{}_{\mathcolor{gray}{-}} \\ ~ \end{smallmatrix}}{09}{\mathtt{g}}^{\;\!\mathcolor{gray}{\omega} \textcolor{PineGreen}{\hat{1}}}_{\;\! \mathcolor{gray}{z}} &\xrightarrow[\text{$\sim$ \bref{eq:simplify8-scalar-g-modulus}}]{\text{\bref{eq:DP^(2)-1_32-spectrum-DFG7}}} \mathbb{i} k_{\textcolor{Maroon}{\mathsf{o}} \mathcolor{gray}{\omega}}^{\;\! 2} \frac{\Xint{{}^{}_{\mathcolor{gray}{-}}}{10}{\hat{g}}^{\;\! \hat{1} \textcolor{PineGreen}{\hat{1}} \textcolor{Plum}{*}}_{\;\! \mathcolor{gray}{\omega}} {\chi}^{\;\! \textcolor{PineGreen}{\hat{1}} \mathcolor{gray}{\omega} \hat{3} \hat{2} \textcolor{Plum}{*} }_{\;\! \hat{1} \textcolor{Maroon}{(2)} \textcolor{PineGreen}{\hat{3} \hat{2}}} ~\mathcolor{gray}{\mathcal F_{z}^{-\textcolor{Plum}{*}}} \left[ \Xint{\mathcolor{gray}{-}}{18}{M}^{\;\! \mathcolor{gray}{\omega} \hat{3} \hat{2} }_{\;\! \hat{1} \mathcolor{gray}{k_{\symup{z}}} \textcolor{Maroon}{(2)} } \mathcolor{gray}{\circ} \Xint{\mathcolor{gray}{-}}{255}{E}^{\;\!\textcolor{PineGreen}{\hat{2}} \mathcolor{gray}{\omega} }_{\;\! \hat{2} \mathcolor{gray}{z}} ~\mathcolor{gray}{\widetilde \circledcirc}~ \Xint{\mathcolor{gray}{-}}{255}{E}^{\;\! \textcolor{PineGreen}{\hat{3}} \mathcolor{gray}{\omega}}_{\;\! \hat{3} \mathcolor{gray}{z}} \right]}{ 2 \lvert \Xint{{}^{}_{\mathcolor{gray}{-}}}{10}{\hat{g}}^{\;\! \textcolor{PineGreen}{\hat{1}}}_{\;\! \mathcolor{gray}{\omega}} \rvert^2 \Xint{\begin{smallmatrix} ~ \\ {}^{}_{\mathcolor{gray}{-}} \\ ~ \end{smallmatrix}}{15}{k}_{\;\! \symup{z}}^{\;\! \mathcolor{gray}{\omega} \textcolor{PineGreen}{\hat{1}}} \mathbb{e}^{\mathbb{i} \Xint{\begin{smallmatrix} ~ \\ {}^{}_{\mathcolor{gray}{-}} \\ ~ \end{smallmatrix}}{15}{k}_{\symup{z}}^{\;\! \mathcolor{gray}{\omega} \textcolor{PineGreen}{\hat{1}}} \mathcolor{gray}{z}}} ~, \label{eq:down-scalar-g-EE-132-spectrum} \\
	\mathcolor{gray}{\nabla_z} \Xint{\begin{smallmatrix} ~ \\ {}^{}_{\mathcolor{gray}{-}} \\ ~ \end{smallmatrix}}{09}{\mathtt{g}}^{\;\!\mathcolor{gray}{\omega} \textcolor{PineGreen}{\hat{2}}}_{\;\! \mathcolor{gray}{z}} &\xrightarrow[\text{$\sim$ \bref{eq:simplify8-scalar-g-modulus}}]{\text{$\sim$ \bref{eq:DP^(2)-1_32-spectrum-DFG7}}} \mathbb{i} k_{\textcolor{Maroon}{\mathsf{o}} \mathcolor{gray}{\omega}}^{\;\! 2} \frac{\Xint{{}^{}_{\mathcolor{gray}{-}}}{10}{\hat{g}}^{\;\! \hat{2} \textcolor{PineGreen}{\hat{2}} \textcolor{Plum}{*}}_{\;\! \mathcolor{gray}{\omega}} {\chi}^{\;\! \textcolor{PineGreen}{\hat{2}} \mathcolor{gray}{\omega} \hat{3} \hat{1} \textcolor{Plum}{*} }_{\;\! \hat{2} \textcolor{Maroon}{(2)} \textcolor{PineGreen}{\hat{3} \hat{1}}} ~\mathcolor{gray}{\mathcal F_{z}^{-\textcolor{Plum}{*}}} \left[ \Xint{\mathcolor{gray}{-}}{18}{M}^{\;\! \mathcolor{gray}{\omega} \hat{3} \hat{1} }_{\;\! \hat{2} \mathcolor{gray}{k_{\symup{z}}} \textcolor{Maroon}{(2)} } \mathcolor{gray}{\circ} \Xint{\mathcolor{gray}{-}}{255}{E}^{\;\!\textcolor{PineGreen}{\hat{1}} \mathcolor{gray}{\omega} }_{\;\! \hat{1} \mathcolor{gray}{z}} ~\mathcolor{gray}{\widetilde \circledcirc}~ \Xint{\mathcolor{gray}{-}}{255}{E}^{\;\!\textcolor{PineGreen}{\hat{3}} \mathcolor{gray}{\omega}}_{\;\! \hat{3} \mathcolor{gray}{z}} \right]}{ 2 \lvert \Xint{{}^{}_{\mathcolor{gray}{-}}}{10}{\hat{g}}^{\;\! \textcolor{PineGreen}{\hat{2}}}_{\;\! \mathcolor{gray}{\omega}} \rvert^2 \Xint{\begin{smallmatrix} ~ \\ {}^{}_{\mathcolor{gray}{-}} \\ ~ \end{smallmatrix}}{15}{k}_{\;\! \symup{z}}^{\;\! \mathcolor{gray}{\omega} \textcolor{PineGreen}{\hat{2}}} \mathbb{e}^{\mathbb{i} \Xint{\begin{smallmatrix} ~ \\ {}^{}_{\mathcolor{gray}{-}} \\ ~ \end{smallmatrix}}{15}{k}_{\symup{z}}^{\;\! \mathcolor{gray}{\omega} \textcolor{PineGreen}{\hat{2}}} \mathcolor{gray}{z}}} ~. \label{eq:down-scalar-g-EE-231-spectrum}
\end{align}
\end{subequations}

在“\textbf{标量\textcolor{Plum}{非线性}\textcolor{NavyBlue}{波源}}(\textcolor{NavyBlue}{脉冲})”条件 \bref{eq:scalar_nonlinear_drive-spectrum} 下,上述描述电子和光子\textcolor{NavyBlue}{宽带}相互作用的\textcolor{Maroon}{三波混频}方程,从 \bref{eq:3wavemix-scalar-g-EE-spectrum} 变为
\begin{subequations} \label{eq:3wavemix-scalar-g-EE-chieff-spectrum}
\begin{align}
	\mathcolor{gray}{\nabla_z} \Xint{\begin{smallmatrix} ~ \\ {}^{}_{\mathcolor{gray}{-}} \\ ~ \end{smallmatrix}}{09}{\mathtt{g}}^{\;\!\mathcolor{gray}{\omega} \textcolor{PineGreen}{\hat{3}}}_{\;\! \mathcolor{gray}{z}} &\xrightarrow[\text{\bref{eq:up-scalar-g-EE-312-spectrum}}]{\text{\bref{eq:scalar_nonlinear_drive-spectrum}}} \mathbb{i} k_{\textcolor{Maroon}{\mathsf{o}} \mathcolor{gray}{\omega}}^{\;\! 2} \frac{\textcolor{gray}{\widetilde{\textcolor{black}{\chi}}}^{ \hat{3} \textcolor{PineGreen}{\hat{3}} \textcolor{Maroon}{(2)} \mathcolor{gray}{\omega} }_{ \textcolor{NavyBlue}{\text{eff}} \hat{1} \hat{2} \textcolor{PineGreen}{\hat{1} \hat{2}} } ~ \mathcolor{gray}{\mathcal F_{z}^{-1}} \left[ \Xint{\mathcolor{gray}{-}}{18}{M}^{\;\! \mathcolor{gray}{\omega} \hat{1} \hat{2} }_{\;\! \hat{3} \mathcolor{gray}{k_{\symup{z}}} \textcolor{Maroon}{(2)} } \mathcolor{gray}{*} \Xint{\mathcolor{gray}{-}}{15}{\mathtt{G}}^{\;\! \textcolor{PineGreen}{\hat{1}} \mathcolor{gray}{\omega} }_{\;\! \mathcolor{gray}{z}} ~\mathcolor{gray}{\widetilde \circledast}~ \Xint{\mathcolor{gray}{-}}{15}{\mathtt{G}}^{\;\! \textcolor{PineGreen}{\hat{2}} \mathcolor{gray}{\omega} }_{\;\! \mathcolor{gray}{z}} \right]}{ 2 \lvert \Xint{{}^{}_{\mathcolor{gray}{-}}}{10}{\hat{g}}^{\;\! \textcolor{PineGreen}{\hat{3}}}_{\;\! \mathcolor{gray}{\omega}} \rvert^2 \Xint{\begin{smallmatrix} ~ \\ {}^{}_{\mathcolor{gray}{-}} \\ ~ \end{smallmatrix}}{15}{k}_{\;\! \symup{z}}^{\;\! \mathcolor{gray}{\omega} \textcolor{PineGreen}{\hat{3}}} \mathbb{e}^{\mathbb{i} \Xint{\begin{smallmatrix} ~ \\ {}^{}_{\mathcolor{gray}{-}} \\ ~ \end{smallmatrix}}{15}{k}_{\symup{z}}^{\;\! \mathcolor{gray}{\omega} \textcolor{PineGreen}{\hat{3}}} \mathcolor{gray}{z}}} ~, \label{eq:up-scalar-g-EE-312-chieff-spectrum} \\
	\mathcolor{gray}{\nabla_z} \Xint{\begin{smallmatrix} ~ \\ {}^{}_{\mathcolor{gray}{-}} \\ ~ \end{smallmatrix}}{09}{\mathtt{g}}^{\;\!\mathcolor{gray}{\omega} \textcolor{PineGreen}{\hat{1}}}_{\;\! \mathcolor{gray}{z}} &\xrightarrow[\text{\bref{eq:down-scalar-g-EE-132-spectrum}}]{\text{\bref{eq:scalar_nonlinear_drive-spectrum}}} \mathbb{i} k_{\textcolor{Maroon}{\mathsf{o}} \mathcolor{gray}{\omega}}^{\;\! 2} \frac{ \textcolor{gray}{\widetilde{\textcolor{black}{\chi}}}^{ \hat{1} \textcolor{PineGreen}{\hat{1}} \textcolor{Maroon}{(2)} \mathcolor{gray}{\omega} }_{ \textcolor{NavyBlue}{\text{eff}} \hat{3} \hat{2} \textcolor{PineGreen}{\hat{3} \hat{2}} } \mathcolor{gray}{\mathcal F_{z}^{-\textcolor{Plum}{*}}} \left[ \Xint{\mathcolor{gray}{-}}{18}{M}^{\;\! \mathcolor{gray}{\omega} \hat{3} \hat{2} }_{\;\! \hat{1} \mathcolor{gray}{k_{\symup{z}}} \textcolor{Maroon}{(2)} } \mathcolor{gray}{\circ} \Xint{\mathcolor{gray}{-}}{15}{\mathtt{G}}^{\;\! \textcolor{PineGreen}{\hat{2}} \mathcolor{gray}{\omega} }_{\;\! \mathcolor{gray}{z}} ~\mathcolor{gray}{\widetilde \circledcirc}~ \Xint{\mathcolor{gray}{-}}{15}{\mathtt{G}}^{\;\! \textcolor{PineGreen}{\hat{3}} \mathcolor{gray}{\omega} }_{\;\! \mathcolor{gray}{z}} \right] }{ 2 \lvert \Xint{{}^{}_{\mathcolor{gray}{-}}}{10}{\hat{g}}^{\;\! \textcolor{PineGreen}{\hat{1}}}_{\;\! \mathcolor{gray}{\omega}} \rvert^2 \Xint{\begin{smallmatrix} ~ \\ {}^{}_{\mathcolor{gray}{-}} \\ ~ \end{smallmatrix}}{15}{k}_{\;\! \symup{z}}^{\;\! \mathcolor{gray}{\omega} \textcolor{PineGreen}{\hat{1}}} \mathbb{e}^{\mathbb{i} \Xint{\begin{smallmatrix} ~ \\ {}^{}_{\mathcolor{gray}{-}} \\ ~ \end{smallmatrix}}{15}{k}_{\symup{z}}^{\;\! \mathcolor{gray}{\omega} \textcolor{PineGreen}{\hat{1}}} \mathcolor{gray}{z}}} ~, \label{eq:down-scalar-g-EE-132-chieff*-spectrum} \\
	\mathcolor{gray}{\nabla_z} \Xint{\begin{smallmatrix} ~ \\ {}^{}_{\mathcolor{gray}{-}} \\ ~ \end{smallmatrix}}{09}{\mathtt{g}}^{\;\!\mathcolor{gray}{\omega} \textcolor{PineGreen}{\hat{2}}}_{\;\! \mathcolor{gray}{z}} &\xrightarrow[\text{\bref{eq:down-scalar-g-EE-231-spectrum}}]{\text{\bref{eq:scalar_nonlinear_drive-spectrum}}} \mathbb{i} k_{\textcolor{Maroon}{\mathsf{o}} \mathcolor{gray}{\omega}}^{\;\! 2} \frac{ \textcolor{gray}{\widetilde{\textcolor{black}{\chi}}}^{ \hat{2} \textcolor{PineGreen}{\hat{2}} \textcolor{Maroon}{(2)} \mathcolor{gray}{\omega} }_{ \textcolor{NavyBlue}{\text{eff}} \hat{3} \hat{1} \textcolor{PineGreen}{\hat{3} \hat{1}} } \mathcolor{gray}{\mathcal F_{z}^{-\textcolor{Plum}{*}}} \left[ \Xint{\mathcolor{gray}{-}}{18}{M}^{\;\! \mathcolor{gray}{\omega} \hat{3} \hat{1} }_{\;\! \hat{2} \mathcolor{gray}{k_{\symup{z}}} \textcolor{Maroon}{(2)} } \mathcolor{gray}{\circ} \Xint{\mathcolor{gray}{-}}{15}{\mathtt{G}}^{\;\! \textcolor{PineGreen}{\hat{1}} \mathcolor{gray}{\omega} }_{\;\! \mathcolor{gray}{z}} ~\mathcolor{gray}{\widetilde \circledcirc}~ \Xint{\mathcolor{gray}{-}}{15}{\mathtt{G}}^{\;\! \textcolor{PineGreen}{\hat{3}} \mathcolor{gray}{\omega} }_{\;\! \mathcolor{gray}{z}} \right] }{ 2 \lvert \Xint{{}^{}_{\mathcolor{gray}{-}}}{10}{\hat{g}}^{\;\! \textcolor{PineGreen}{\hat{2}}}_{\;\! \mathcolor{gray}{\omega}} \rvert^2 \Xint{\begin{smallmatrix} ~ \\ {}^{}_{\mathcolor{gray}{-}} \\ ~ \end{smallmatrix}}{15}{k}_{\;\! \symup{z}}^{\;\! \mathcolor{gray}{\omega} \textcolor{PineGreen}{\hat{2}}} \mathbb{e}^{\mathbb{i} \Xint{\begin{smallmatrix} ~ \\ {}^{}_{\mathcolor{gray}{-}} \\ ~ \end{smallmatrix}}{15}{k}_{\symup{z}}^{\;\! \mathcolor{gray}{\omega} \textcolor{PineGreen}{\hat{2}}} \mathcolor{gray}{z}}} ~, \label{eq:down-scalar-g-EE-231-chieff*-spectrum}
\end{align}
\end{subequations}
定义了“\textbf{标量\textcolor{Plum}{非线性}\textcolor{NavyBlue}{波源}}”条件下,该\textcolor{NavyBlue}{超快}过程的\textcolor{NavyBlue}{有效非线性系数}(三阶)张量
\begin{subequations} \label{eq:3wavemix-chieff-spectrum}
\begin{align}
	\textcolor{gray}{\widetilde{\textcolor{black}{\chi}}}^{ \hat{3} \textcolor{PineGreen}{\hat{3}} \textcolor{Maroon}{(2)} \mathcolor{gray}{\omega} }_{ \textcolor{NavyBlue}{\text{eff}} \hat{1} \hat{2} \textcolor{PineGreen}{\hat{1} \hat{2}} } &\xrightarrow[]{\text{\bref{eq:chieff-spectrum}}} \Xint{{}^{}_{\mathcolor{gray}{-}}}{10}{\hat{g}}^{\;\! \hat{3} \textcolor{PineGreen}{\hat{3}} \textcolor{Plum}{*}}_{\;\! \mathcolor{gray}{\omega}} {\chi}^{\;\! \hat{3} \textcolor{PineGreen}{\hat{3}} \textcolor{Maroon}{(2)} }_{\;\! \mathcolor{gray}{\omega} \hat{1} \hat{2} \textcolor{PineGreen}{\hat{1} \hat{2}} } ~ {\hat{g}}^{\;\! \mathcolor{gray}{\omega} }_{\;\! \hat{1} \textcolor{PineGreen}{\hat{1}} } ~\mathcolor{gray}{\widetilde *}~ {\hat{g}}^{\;\! \mathcolor{gray}{\omega} }_{\;\! \hat{2} \textcolor{PineGreen}{\hat{2}} } ~, \label{eq:up-312-chieff*-spectrum} \\
	\textcolor{gray}{\widetilde{\textcolor{black}{\chi}}}^{ \hat{1} \textcolor{PineGreen}{\hat{1}} \textcolor{Maroon}{(2)} \mathcolor{gray}{\omega} }_{ \textcolor{NavyBlue}{\text{eff}} \hat{3} \hat{2} \textcolor{PineGreen}{\hat{3} \hat{2}} } &\xrightarrow[]{\text{\bref{eq:chieff*-spectrum}}} \Xint{{}^{}_{\mathcolor{gray}{-}}}{10}{\hat{g}}^{\;\! \hat{1} \textcolor{PineGreen}{\hat{1}} \textcolor{Plum}{*}}_{\;\! \mathcolor{gray}{\omega}} {\chi}^{\;\! \hat{1} \textcolor{PineGreen}{\hat{1}} \textcolor{Maroon}{(2)} \textcolor{Plum}{*} }_{\;\! \mathcolor{gray}{\omega} \hat{3} \hat{2} \textcolor{PineGreen}{\hat{3} \hat{2}} } ~ {\hat{g}}^{\;\! \mathcolor{gray}{\omega} }_{\;\! \hat{2} \textcolor{PineGreen}{\hat{2}} } ~\mathcolor{gray}{\widetilde \circ}~ {\hat{g}}^{\;\! \mathcolor{gray}{\omega} }_{\;\! \hat{3} \textcolor{PineGreen}{\hat{3}} } ~, \label{eq:down-132-chieff*-spectrum} \\
	\textcolor{gray}{\widetilde{\textcolor{black}{\chi}}}^{ \hat{2} \textcolor{PineGreen}{\hat{2}} \textcolor{Maroon}{(2)} \mathcolor{gray}{\omega} }_{ \textcolor{NavyBlue}{\text{eff}} \hat{3} \hat{1} \textcolor{PineGreen}{\hat{3} \hat{1}} } &\xrightarrow[]{\text{$\sim$ \bref{eq:chieff*-spectrum}}} \Xint{{}^{}_{\mathcolor{gray}{-}}}{10}{\hat{g}}^{\;\! \hat{2} \textcolor{PineGreen}{\hat{2}} \textcolor{Plum}{*}}_{\;\! \mathcolor{gray}{\omega}} {\chi}^{\;\! \hat{2} \textcolor{PineGreen}{\hat{2}} \textcolor{Maroon}{(2)} \textcolor{Plum}{*} }_{\;\! \mathcolor{gray}{\omega} \hat{3} \hat{1} \textcolor{PineGreen}{\hat{3} \hat{1}} } ~ {\hat{g}}^{\;\! \mathcolor{gray}{\omega} }_{\;\! \hat{1} \textcolor{PineGreen}{\hat{1}} } ~\mathcolor{gray}{\widetilde \circ}~ {\hat{g}}^{\;\! \mathcolor{gray}{\omega} }_{\;\! \hat{3} \textcolor{PineGreen}{\hat{3}} } ~. \label{eq:down-231-chieff*-spectrum}
\end{align}
\end{subequations}

进一步,再在“\textbf{标量场 $\chi^{\;\! \mathcolor{gray}{\omega} }_{\;\! \mathcolor{gray}{z} \textcolor{Maroon}{(2)}}$ \textcolor{NavyBlue}{调制}}” \bref{eq:scalar_chi2_modulation} 的额外条件下,描述电子和光子\textcolor{NavyBlue}{宽带}相互作用的\textcolor{Maroon}{三波混频}方程从 \bref{eq:3wavemix-scalar-g-EE-chieff-spectrum} 变为
\begin{subequations} \label{eq:3wavemix-scalar-g-EE-chieff-scalar-spectrum}
\begin{align}
	\mathcolor{gray}{\nabla_z} \Xint{\begin{smallmatrix} ~ \\ {}^{}_{\mathcolor{gray}{-}} \\ ~ \end{smallmatrix}}{09}{\mathtt{g}}^{\;\!\mathcolor{gray}{\omega} \textcolor{PineGreen}{\hat{3}}}_{\;\! \mathcolor{gray}{z}} &\xrightarrow[\text{\bref{eq:up-scalar-g-EE-312-chieff-spectrum}}]{\text{\bref{eq:scalar_chi2_modulation}}} \mathbb{i} k_{\textcolor{Maroon}{\mathsf{o}} \mathcolor{gray}{\omega}}^{\;\! 2} \frac{\textcolor{gray}{\widetilde{\textcolor{black}{\chi}}}^{ \mathcolor{gray}{\omega} \textcolor{PineGreen}{\hat{3}} \textcolor{Maroon}{(2)} }_{ \textcolor{NavyBlue}{\text{eff}} \textcolor{PineGreen}{\hat{1} \hat{2}} } ~ \mathcolor{gray}{\mathcal F_{z}^{-1}} \left[ \Xint{\mathcolor{gray}{-}}{18}{M}^{\;\! \mathcolor{gray}{\omega} }_{\;\! \mathcolor{gray}{k_{\symup{z}}} \textcolor{Maroon}{(2)} } \mathcolor{gray}{*} \Xint{\mathcolor{gray}{-}}{15}{\mathtt{G}}^{\;\! \textcolor{PineGreen}{\hat{1}} \mathcolor{gray}{\omega} }_{\;\! \mathcolor{gray}{z}} ~\mathcolor{gray}{\widetilde \circledast}~ \Xint{\mathcolor{gray}{-}}{15}{\mathtt{G}}^{\;\! \textcolor{PineGreen}{\hat{2}} \mathcolor{gray}{\omega} }_{\;\! \mathcolor{gray}{z}} \right]}{ 2 \lvert \Xint{{}^{}_{\mathcolor{gray}{-}}}{10}{\hat{g}}^{\;\! \textcolor{PineGreen}{\hat{3}}}_{\;\! \mathcolor{gray}{\omega}} \rvert^2 \Xint{\begin{smallmatrix} ~ \\ {}^{}_{\mathcolor{gray}{-}} \\ ~ \end{smallmatrix}}{15}{k}_{\;\! \symup{z}}^{\;\! \mathcolor{gray}{\omega} \textcolor{PineGreen}{\hat{3}}} \mathbb{e}^{\mathbb{i} \Xint{\begin{smallmatrix} ~ \\ {}^{}_{\mathcolor{gray}{-}} \\ ~ \end{smallmatrix}}{15}{k}_{\symup{z}}^{\;\! \mathcolor{gray}{\omega} \textcolor{PineGreen}{\hat{3}}} \mathcolor{gray}{z}}} ~, \label{eq:up-scalar-g-EE-132-chieff-scalar-spectrum} \\
	\mathcolor{gray}{\nabla_z} \Xint{\begin{smallmatrix} ~ \\ {}^{}_{\mathcolor{gray}{-}} \\ ~ \end{smallmatrix}}{09}{\mathtt{g}}^{\;\!\mathcolor{gray}{\omega} \textcolor{PineGreen}{\hat{1}}}_{\;\! \mathcolor{gray}{z}} &\xrightarrow[\text{\bref{eq:down-scalar-g-EE-132-chieff*-spectrum}}]{\text{\bref{eq:scalar_chi2_modulation}}} \mathbb{i} k_{\textcolor{Maroon}{\mathsf{o}} \mathcolor{gray}{\omega}}^{\;\! 2} \frac{ \textcolor{gray}{\widetilde{\textcolor{black}{\chi}}}^{ \mathcolor{gray}{\omega} \textcolor{PineGreen}{\hat{1}} \textcolor{Maroon}{(2)} }_{ \textcolor{NavyBlue}{\text{eff}} \textcolor{PineGreen}{\hat{3} \hat{2}} } ~ \mathcolor{gray}{\mathcal F_{z}^{-\textcolor{Plum}{*}}} \left[ \Xint{\mathcolor{gray}{-}}{18}{M}^{\;\! \mathcolor{gray}{\omega} }_{\;\! \mathcolor{gray}{k_{\symup{z}}} \textcolor{Maroon}{(2)} } \mathcolor{gray}{\circ} \Xint{\mathcolor{gray}{-}}{15}{\mathtt{G}}^{\;\! \textcolor{PineGreen}{\hat{2}} \mathcolor{gray}{\omega} }_{\;\! \mathcolor{gray}{z}} ~\mathcolor{gray}{\widetilde \circledcirc}~ \Xint{\mathcolor{gray}{-}}{15}{\mathtt{G}}^{\;\! \textcolor{PineGreen}{\hat{3}} \mathcolor{gray}{\omega} }_{\;\! \mathcolor{gray}{z}} \right] }{ 2 \lvert \Xint{{}^{}_{\mathcolor{gray}{-}}}{10}{\hat{g}}^{\;\! \textcolor{PineGreen}{\hat{1}}}_{\;\! \mathcolor{gray}{\omega}} \rvert^2 \Xint{\begin{smallmatrix} ~ \\ {}^{}_{\mathcolor{gray}{-}} \\ ~ \end{smallmatrix}}{15}{k}_{\;\! \symup{z}}^{\;\! \mathcolor{gray}{\omega} \textcolor{PineGreen}{\hat{1}}} \mathbb{e}^{\mathbb{i} \Xint{\begin{smallmatrix} ~ \\ {}^{}_{\mathcolor{gray}{-}} \\ ~ \end{smallmatrix}}{15}{k}_{\symup{z}}^{\;\! \mathcolor{gray}{\omega} \textcolor{PineGreen}{\hat{1}}} \mathcolor{gray}{z}}} ~, \label{eq:down-scalar-g-EE-132-chieff*-scalar-spectrum} \\
	\mathcolor{gray}{\nabla_z} \Xint{\begin{smallmatrix} ~ \\ {}^{}_{\mathcolor{gray}{-}} \\ ~ \end{smallmatrix}}{09}{\mathtt{g}}^{\;\!\mathcolor{gray}{\omega} \textcolor{PineGreen}{\hat{2}}}_{\;\! \mathcolor{gray}{z}} &\xrightarrow[\text{\bref{eq:down-scalar-g-EE-231-chieff*-spectrum}}]{\text{\bref{eq:scalar_chi2_modulation}}} \mathbb{i} k_{\textcolor{Maroon}{\mathsf{o}} \mathcolor{gray}{\omega}}^{\;\! 2} \frac{ \textcolor{gray}{\widetilde{\textcolor{black}{\chi}}}^{ \mathcolor{gray}{\omega} \textcolor{PineGreen}{\hat{2}} \textcolor{Maroon}{(2)} }_{ \textcolor{NavyBlue}{\text{eff}} \textcolor{PineGreen}{\hat{3} \hat{1}} } ~ \mathcolor{gray}{\mathcal F_{z}^{-\textcolor{Plum}{*}}} \left[ \Xint{\mathcolor{gray}{-}}{18}{M}^{\;\! \mathcolor{gray}{\omega} }_{\;\! \mathcolor{gray}{k_{\symup{z}}} \textcolor{Maroon}{(2)} } \mathcolor{gray}{\circ} \Xint{\mathcolor{gray}{-}}{15}{\mathtt{G}}^{\;\! \textcolor{PineGreen}{\hat{1}} \mathcolor{gray}{\omega} }_{\;\! \mathcolor{gray}{z}} ~\mathcolor{gray}{\widetilde \circledcirc}~ \Xint{\mathcolor{gray}{-}}{15}{\mathtt{G}}^{\;\! \textcolor{PineGreen}{\hat{3}} \mathcolor{gray}{\omega} }_{\;\! \mathcolor{gray}{z}} \right] }{ 2 \lvert \Xint{{}^{}_{\mathcolor{gray}{-}}}{10}{\hat{g}}^{\;\! \textcolor{PineGreen}{\hat{2}}}_{\;\! \mathcolor{gray}{\omega}} \rvert^2 \Xint{\begin{smallmatrix} ~ \\ {}^{}_{\mathcolor{gray}{-}} \\ ~ \end{smallmatrix}}{15}{k}_{\;\! \symup{z}}^{\;\! \mathcolor{gray}{\omega} \textcolor{PineGreen}{\hat{2}}} \mathbb{e}^{\mathbb{i} \Xint{\begin{smallmatrix} ~ \\ {}^{}_{\mathcolor{gray}{-}} \\ ~ \end{smallmatrix}}{15}{k}_{\symup{z}}^{\;\! \mathcolor{gray}{\omega} \textcolor{PineGreen}{\hat{2}}} \mathcolor{gray}{z}}} ~, \label{eq:down-scalar-g-EE-231-chieff*-scalar-spectrum} \\
\end{align}
\end{subequations}
其中,在“\textbf{标量\textcolor{Plum}{非线性}\textcolor{NavyBlue}{波源}}” \bref{eq:scalar_nonlinear_drive} 和“\textbf{标量场 $\chi^{\;\! \mathcolor{gray}{\omega} }_{\;\! \mathcolor{gray}{z} \textcolor{Maroon}{(2)}}$ \textcolor{NavyBlue}{调制}}” \bref{eq:scalar_chi2_modulation} 这 2 个条件下的\textcolor{NavyBlue}{有效非线性系数}从张量降为标量
\begin{subequations} \label{eq:3wavemix-chieff-scalar-spectrum}
\begin{align}
	\textcolor{gray}{\widetilde{\textcolor{black}{\chi}}}^{ \mathcolor{gray}{\omega} \textcolor{PineGreen}{\hat{3}} \textcolor{Maroon}{(2)} }_{ \textcolor{NavyBlue}{\text{eff}} \textcolor{PineGreen}{\hat{1} \hat{2}} } &\xrightarrow[]{\text{\bref{eq:chieff-scalar-spectrum}}} \Xint{{}^{}_{\mathcolor{gray}{-}}}{10}{\hat{g}}^{\;\! \hat{3} \textcolor{PineGreen}{\hat{3}} \textcolor{Plum}{*}}_{\;\! \mathcolor{gray}{\omega}} {\chi}^{\;\! \textcolor{PineGreen}{\hat{3}} \mathcolor{gray}{\omega} \hat{1} \hat{2} }_{\;\! \hat{3} \textcolor{Maroon}{(2)} \textcolor{PineGreen}{\hat{1} \hat{2}}} ~ {\hat{g}}^{\;\! \mathcolor{gray}{\omega} }_{\;\! \hat{1} \textcolor{PineGreen}{\hat{1}} } ~\mathcolor{gray}{\widetilde *}~ {\hat{g}}^{\;\! \mathcolor{gray}{\omega} }_{\;\! \hat{2} \textcolor{PineGreen}{\hat{2}} } ~, \label{eq:up-312-chieff-scalar-spectrum} \\
	\textcolor{gray}{\widetilde{\textcolor{black}{\chi}}}^{ \mathcolor{gray}{\omega} \textcolor{PineGreen}{\hat{1}} \textcolor{Maroon}{(2)} }_{ \textcolor{NavyBlue}{\text{eff}} \textcolor{PineGreen}{\hat{3} \hat{2}} } &\xrightarrow[]{\text{\bref{eq:chieff*-scalar-spectrum}}} \Xint{{}^{}_{\mathcolor{gray}{-}}}{10}{\hat{g}}^{\;\! \hat{1} \textcolor{PineGreen}{\hat{1}} \textcolor{Plum}{*}}_{\;\! \mathcolor{gray}{\omega}} {\chi}^{\;\! \textcolor{PineGreen}{\hat{1}} \mathcolor{gray}{\omega} \hat{3} \hat{2} \textcolor{Plum}{*} }_{\;\! \hat{1} \textcolor{Maroon}{(2)} \textcolor{PineGreen}{\hat{3} \hat{2}} } ~ {\hat{g}}^{\;\! \mathcolor{gray}{\omega} }_{\;\! \hat{2} \textcolor{PineGreen}{\hat{2}} } ~\mathcolor{gray}{\widetilde \circ}~ {\hat{g}}^{\;\! \mathcolor{gray}{\omega} }_{\;\! \hat{3} \textcolor{PineGreen}{\hat{3}} } ~, \label{eq:down-132-chieff*-scalar-spectrum} \\
	\textcolor{gray}{\widetilde{\textcolor{black}{\chi}}}^{ \mathcolor{gray}{\omega} \textcolor{PineGreen}{\hat{2}} \textcolor{Maroon}{(2)} }_{ \textcolor{NavyBlue}{\text{eff}} \textcolor{PineGreen}{\hat{3} \hat{1}} } &\xrightarrow[]{\text{$\sim$ \bref{eq:chieff*-scalar-spectrum}}} \Xint{{}^{}_{\mathcolor{gray}{-}}}{10}{\hat{g}}^{\;\! \hat{2} \textcolor{PineGreen}{\hat{2}} \textcolor{Plum}{*}}_{\;\! \mathcolor{gray}{\omega}} {\chi}^{\;\! \textcolor{PineGreen}{\hat{2}} \mathcolor{gray}{\omega} \hat{3} \hat{1} \textcolor{Plum}{*} }_{\;\! \hat{2} \textcolor{Maroon}{(2)} \textcolor{PineGreen}{\hat{3} \hat{1}} } ~ {\hat{g}}^{\;\! \mathcolor{gray}{\omega} }_{\;\! \hat{1} \textcolor{PineGreen}{\hat{1}} } ~\mathcolor{gray}{\widetilde \circ}~ {\hat{g}}^{\;\! \mathcolor{gray}{\omega} }_{\;\! \hat{3} \textcolor{PineGreen}{\hat{3}} } ~. \label{eq:down-231-chieff*-scalar-spectrum}
\end{align}
\end{subequations}

\marginLeft[-2.4em]{ssec:CW-3wavemix}\subsection{连续光三波混频 - 电场本征复振幅方程}\label{ssec:CW-3wavemix}

结合 \bref{ssec:SFG_discrete} 中,以\textcolor{Plum}{离散}个\textcolor{gray}{波长}的\textcolor{NavyBlue}{(准)连续光}\textcolor{Maroon}{和频}为代表的\textcolor{NavyBlue}{窄带}\textcolor{Maroon}{上转换}过程的电场\textcolor{PineGreen}{本征复振幅}方程 \bref{eq:simplify8-scalar-g-modulus-P-discrete} 及其\textcolor{Maroon}{下转换}版本,可以得到仅电子和光子间\textcolor{NavyBlue}{稳态}相互作用的\textcolor{Maroon}{三波混频}方程
\begin{subequations} \label{eq:3wavemix-scalar-g-EE-discrete}
\begin{align}
	\mathcolor{gray}{\nabla_z} \Xint{\begin{smallmatrix} ~ \\ {}^{}_{\mathcolor{gray}{-}} \\ ~ \end{smallmatrix}}{09}{\mathtt{g}}^{\;\! \textcolor{PineGreen}{\hat{3}}}_{\;\! \mathcolor{gray}{z}} &\xrightarrow[]{\text{\bref{eq:simplify8-scalar-g-modulus-P-discrete}}} \mathbb{i} k_{\textcolor{Maroon}{\mathsf{o}} \mathcolor{gray}{3}}^{\;\! 2} \frac{\Xint{{}^{}_{\mathcolor{gray}{-}}}{10}{\hat{g}}^{\;\! \hat{3} \textcolor{PineGreen}{\hat{3}} \textcolor{Plum}{*}}_{\;\! } {\chi}^{\;\! \textcolor{PineGreen}{\hat{3}}  \hat{1} \hat{2} }_{\;\! \hat{3} \textcolor{Maroon}{(2)} \textcolor{PineGreen}{\hat{1} \hat{2}}} ~ \mathcolor{gray}{\mathcal F_{z}^{-1}} \left[ \Xint{\mathcolor{gray}{-}}{18}{M}^{\;\! \mathcolor{gray}{3} \hat{1} \hat{2} }_{\;\! \hat{3} \mathcolor{gray}{k_{\symup{z}}} \textcolor{Maroon}{(2)} } \mathcolor{gray}{*} \Xint{\mathcolor{gray}{-}}{25}{E}^{\;\! \textcolor{PineGreen}{\hat{1}}  }_{\;\! \hat{1} \mathcolor{gray}{z}} \mathcolor{gray}{*} \Xint{\mathcolor{gray}{-}}{25}{E}^{\;\! \textcolor{PineGreen}{\hat{2}}  }_{\;\! \hat{2} \mathcolor{gray}{z}} \right]}{ 2 \lvert \Xint{{}^{}_{\mathcolor{gray}{-}}}{10}{\hat{g}}^{\;\! \textcolor{PineGreen}{\hat{3}}} \rvert^2 \Xint{\begin{smallmatrix} ~ \\ {}^{}_{\mathcolor{gray}{-}} \\ ~ \end{smallmatrix}}{15}{k}_{\;\! \symup{z}}^{\;\!  \textcolor{PineGreen}{\hat{3}}} \mathbb{e}^{\mathbb{i} \Xint{\begin{smallmatrix} ~ \\ {}^{}_{\mathcolor{gray}{-}} \\ ~ \end{smallmatrix}}{15}{k}_{\symup{z}}^{\;\!  \textcolor{PineGreen}{\hat{3}}} \mathcolor{gray}{z}}} ~, \label{eq:up-scalar-g-EE-312-discrete} \\
	\mathcolor{gray}{\nabla_z} \Xint{\begin{smallmatrix} ~ \\ {}^{}_{\mathcolor{gray}{-}} \\ ~ \end{smallmatrix}}{09}{\mathtt{g}}^{\;\! \textcolor{PineGreen}{\hat{1}}}_{\;\! \mathcolor{gray}{z}} &\xrightarrow[\text{$\sim$ \bref{eq:simplify8-scalar-g-modulus}}]{\text{$\sim$ \bref{eq:DP^(2)-1_32-spectrum-DFG7}}} \mathbb{i} k_{\textcolor{Maroon}{\mathsf{o}} \mathcolor{gray}{1}}^{\;\! 2} \frac{\Xint{{}^{}_{\mathcolor{gray}{-}}}{10}{\hat{g}}^{\;\! \hat{1} \textcolor{PineGreen}{\hat{1}} \textcolor{Plum}{*}} {\chi}^{\;\! \textcolor{PineGreen}{\hat{1}}  \hat{3} \hat{2} \textcolor{Plum}{*} }_{\;\! \hat{1} \textcolor{Maroon}{(2)} \textcolor{PineGreen}{\hat{3} \hat{2}}} ~\mathcolor{gray}{\mathcal F_{z}^{-\textcolor{Plum}{*}}} \left[ \Xint{\mathcolor{gray}{-}}{18}{M}^{\;\! \mathcolor{gray}{1} \hat{3} \hat{2} }_{\;\! \hat{1} \mathcolor{gray}{k_{\symup{z}}} \textcolor{Maroon}{(2)} } \mathcolor{gray}{\circ} \Xint{\mathcolor{gray}{-}}{255}{E}^{\;\!\textcolor{PineGreen}{\hat{2}}  }_{\;\! \hat{2} \mathcolor{gray}{z}} \mathcolor{gray}{\circ} \Xint{\mathcolor{gray}{-}}{255}{E}^{\;\! \textcolor{PineGreen}{\hat{3}} }_{\;\! \hat{3} \mathcolor{gray}{z}} \right]}{ 2 \lvert \Xint{{}^{}_{\mathcolor{gray}{-}}}{10}{\hat{g}}^{\;\! \textcolor{PineGreen}{\hat{1}}} \rvert^2 \Xint{\begin{smallmatrix} ~ \\ {}^{}_{\mathcolor{gray}{-}} \\ ~ \end{smallmatrix}}{15}{k}_{\;\! \symup{z}}^{\;\!  \textcolor{PineGreen}{\hat{1}}} \mathbb{e}^{\mathbb{i} \Xint{\begin{smallmatrix} ~ \\ {}^{}_{\mathcolor{gray}{-}} \\ ~ \end{smallmatrix}}{15}{k}_{\symup{z}}^{\;\!  \textcolor{PineGreen}{\hat{1}}} \mathcolor{gray}{z}}} ~, \label{eq:down-scalar-g-EE-132-discrete} \\
	\mathcolor{gray}{\nabla_z} \Xint{\begin{smallmatrix} ~ \\ {}^{}_{\mathcolor{gray}{-}} \\ ~ \end{smallmatrix}}{09}{\mathtt{g}}^{\;\! \textcolor{PineGreen}{\hat{2}}}_{\;\! \mathcolor{gray}{z}} &\xrightarrow[\text{$\sim$ \bref{eq:simplify8-scalar-g-modulus}}]{\text{$\sim$ \bref{eq:DP^(2)-1_32-spectrum-DFG7}}} \mathbb{i} k_{\textcolor{Maroon}{\mathsf{o}} \mathcolor{gray}{2}}^{\;\! 2} \frac{\Xint{{}^{}_{\mathcolor{gray}{-}}}{10}{\hat{g}}^{\;\! \hat{2} \textcolor{PineGreen}{\hat{2}} \textcolor{Plum}{*}} {\chi}^{\;\! \textcolor{PineGreen}{\hat{2}}  \hat{3} \hat{1} \textcolor{Plum}{*} }_{\;\! \hat{2} \textcolor{Maroon}{(2)} \textcolor{PineGreen}{\hat{3} \hat{1}}} ~\mathcolor{gray}{\mathcal F_{z}^{-\textcolor{Plum}{*}}} \left[ \Xint{\mathcolor{gray}{-}}{18}{M}^{\;\! \mathcolor{gray}{2} \hat{3} \hat{1} }_{\;\! \hat{2} \mathcolor{gray}{k_{\symup{z}}} \textcolor{Maroon}{(2)} } \mathcolor{gray}{\circ} \Xint{\mathcolor{gray}{-}}{255}{E}^{\;\!\textcolor{PineGreen}{\hat{1}}  }_{\;\! \hat{1} \mathcolor{gray}{z}} \mathcolor{gray}{\circ} \Xint{\mathcolor{gray}{-}}{255}{E}^{\;\!\textcolor{PineGreen}{\hat{3}} }_{\;\! \hat{3} \mathcolor{gray}{z}} \right]}{ 2 \lvert \Xint{{}^{}_{\mathcolor{gray}{-}}}{10}{\hat{g}}^{\;\! \textcolor{PineGreen}{\hat{2}}} \rvert^2 \Xint{\begin{smallmatrix} ~ \\ {}^{}_{\mathcolor{gray}{-}} \\ ~ \end{smallmatrix}}{15}{k}_{\;\! \symup{z}}^{\;\!  \textcolor{PineGreen}{\hat{2}}} \mathbb{e}^{\mathbb{i} \Xint{\begin{smallmatrix} ~ \\ {}^{}_{\mathcolor{gray}{-}} \\ ~ \end{smallmatrix}}{15}{k}_{\symup{z}}^{\;\!  \textcolor{PineGreen}{\hat{2}}} \mathcolor{gray}{z}}} ~, \label{eq:down-scalar-g-EE-231-discrete}
\end{align}
\end{subequations}
注意,在 $\mathcolor{gray}{\bar{k}_{\symup{\rho}}}$ 域,默认先计算\textcolor{NavyBlue}{光场}间的\textcolor{Plum}{互相关},再计算与\textcolor{NavyBlue}{倒格波系数}的\textcolor{Plum}{互相关},即从右到左计算 \bref{eq:down-scalar-g-EE-231-discrete,eq:down-scalar-g-EE-231-discrete} 中的两个\textcolor{Plum}{互相关},以省略一对小括号。

在“\textbf{标量\textcolor{Plum}{非线性}\textcolor{NavyBlue}{波源}}(\textcolor{NavyBlue}{连续})”条件 \bref{eq:scalar_nonlinear_drive-discrete} 下,上述描述电子和光子\textcolor{NavyBlue}{窄带}相互作用的\textcolor{Maroon}{三波混频}方程,从 \bref{eq:3wavemix-scalar-g-EE-discrete} 变为
\begin{subequations} \label{eq:3wavemix-scalar-g-EE-chieff-discrete}
\begin{align}
	\mathcolor{gray}{\nabla_z} \Xint{\begin{smallmatrix} ~ \\ {}^{}_{\mathcolor{gray}{-}} \\ ~ \end{smallmatrix}}{09}{\mathtt{g}}^{\;\! \textcolor{PineGreen}{\hat{3}}}_{\;\! \mathcolor{gray}{z}} &\xrightarrow[]{\text{\bref{eq:scalar-g-modulus-P-chieff-discrete}}} \mathbb{i} k_{\textcolor{Maroon}{\mathsf{o}} \mathcolor{gray}{3}}^{\;\! 2} \frac{ {\chi}^{ \hat{3} \textcolor{PineGreen}{\hat{3}} \textcolor{Maroon}{(2)} }_{ \textcolor{NavyBlue}{\text{eff}} \hat{1} \textcolor{PineGreen}{\hat{1}} \hat{2} \textcolor{PineGreen}{\hat{2}} } ~ \mathcolor{gray}{\mathcal F_{z}^{-1}} \left[ \Xint{\mathcolor{gray}{-}}{18}{M}^{\;\! \mathcolor{gray}{3} \hat{1} \hat{2} }_{\;\! \hat{3} \mathcolor{gray}{k_{\symup{z}}} \textcolor{Maroon}{(2)} } \mathcolor{gray}{*} \Xint{\mathcolor{gray}{-}}{15}{\mathtt{G}}^{\;\! \textcolor{PineGreen}{\hat{1}} }_{\;\! \mathcolor{gray}{z}} \mathcolor{gray}{*} \Xint{\mathcolor{gray}{-}}{15}{\mathtt{G}}^{\;\! \textcolor{PineGreen}{\hat{2}} }_{\;\! \mathcolor{gray}{z}} \right]}{ 2 \lvert \Xint{{}^{}_{\mathcolor{gray}{-}}}{10}{\hat{g}}^{\;\! \textcolor{PineGreen}{\hat{3}}} \rvert^2 \Xint{\begin{smallmatrix} ~ \\ {}^{}_{\mathcolor{gray}{-}} \\ ~ \end{smallmatrix}}{15}{k}_{\;\! \symup{z}}^{\;\!  \textcolor{PineGreen}{\hat{3}}} \mathbb{e}^{\mathbb{i} \Xint{\begin{smallmatrix} ~ \\ {}^{}_{\mathcolor{gray}{-}} \\ ~ \end{smallmatrix}}{15}{k}_{\symup{z}}^{\;\!  \textcolor{PineGreen}{\hat{3}}} \mathcolor{gray}{z}}} ~, \label{eq:up-scalar-g-EE-312-chieff-discrete} \\
	\mathcolor{gray}{\nabla_z} \Xint{\begin{smallmatrix} ~ \\ {}^{}_{\mathcolor{gray}{-}} \\ ~ \end{smallmatrix}}{09}{\mathtt{g}}^{\;\! \textcolor{PineGreen}{\hat{1}}}_{\;\! \mathcolor{gray}{z}} &\xrightarrow[]{\text{\bref{eq:scalar-g-modulus-P-chieff*-discrete}}} \mathbb{i} k_{\textcolor{Maroon}{\mathsf{o}} \mathcolor{gray}{1}}^{\;\! 2} \frac{ {\chi}^{ \hat{1} \textcolor{PineGreen}{\hat{1}} \textcolor{Maroon}{(2)} }_{ \textcolor{NavyBlue}{\text{eff}} \hat{3} \textcolor{PineGreen}{\hat{3}} \hat{2} \textcolor{PineGreen}{\hat{2}} } ~\mathcolor{gray}{\mathcal F_{z}^{-\textcolor{Plum}{*}}} \left[ \Xint{\mathcolor{gray}{-}}{18}{M}^{\;\! \mathcolor{gray}{1} \hat{3} \hat{2} }_{\;\! \hat{1} \mathcolor{gray}{k_{\symup{z}}} \textcolor{Maroon}{(2)} } \mathcolor{gray}{\circ} \Xint{\mathcolor{gray}{-}}{15}{\mathtt{G}}^{\;\! \textcolor{PineGreen}{\hat{2}} }_{\;\! \mathcolor{gray}{z}} \mathcolor{gray}{\circ} \Xint{\mathcolor{gray}{-}}{15}{\mathtt{G}}^{\;\! \textcolor{PineGreen}{\hat{3}} }_{\;\! \mathcolor{gray}{z}} \right]}{ 2 \lvert \Xint{{}^{}_{\mathcolor{gray}{-}}}{10}{\hat{g}}^{\;\! \textcolor{PineGreen}{\hat{1}}} \rvert^2 \Xint{\begin{smallmatrix} ~ \\ {}^{}_{\mathcolor{gray}{-}} \\ ~ \end{smallmatrix}}{15}{k}_{\;\! \symup{z}}^{\;\!  \textcolor{PineGreen}{\hat{1}}} \mathbb{e}^{\mathbb{i} \Xint{\begin{smallmatrix} ~ \\ {}^{}_{\mathcolor{gray}{-}} \\ ~ \end{smallmatrix}}{15}{k}_{\symup{z}}^{\;\!  \textcolor{PineGreen}{\hat{1}}} \mathcolor{gray}{z}}} ~, \label{eq:down-scalar-g-EE-132-chieff*-discrete} \\
	\mathcolor{gray}{\nabla_z} \Xint{\begin{smallmatrix} ~ \\ {}^{}_{\mathcolor{gray}{-}} \\ ~ \end{smallmatrix}}{09}{\mathtt{g}}^{\;\! \textcolor{PineGreen}{\hat{2}}}_{\;\! \mathcolor{gray}{z}} &\xrightarrow[]{\text{$\sim$ \bref{eq:scalar-g-modulus-P-chieff*-discrete}}} \mathbb{i} k_{\textcolor{Maroon}{\mathsf{o}} \mathcolor{gray}{2}}^{\;\! 2} \frac{ {\chi}^{ \hat{2} \textcolor{PineGreen}{\hat{2}} \textcolor{Maroon}{(2)} }_{ \textcolor{NavyBlue}{\text{eff}} \hat{3} \textcolor{PineGreen}{\hat{3}} \hat{1} \textcolor{PineGreen}{\hat{1}} } ~\mathcolor{gray}{\mathcal F_{z}^{-\textcolor{Plum}{*}}} \left[ \Xint{\mathcolor{gray}{-}}{18}{M}^{\;\! \mathcolor{gray}{2} \hat{3} \hat{1} }_{\;\! \hat{2} \mathcolor{gray}{k_{\symup{z}}} \textcolor{Maroon}{(2)} } \mathcolor{gray}{\circ} \Xint{\mathcolor{gray}{-}}{15}{\mathtt{G}}^{\;\! \textcolor{PineGreen}{\hat{1}} }_{\;\! \mathcolor{gray}{z}} \mathcolor{gray}{\circ} \Xint{\mathcolor{gray}{-}}{15}{\mathtt{G}}^{\;\! \textcolor{PineGreen}{\hat{3}} }_{\;\! \mathcolor{gray}{z}} \right]}{ 2 \lvert \Xint{{}^{}_{\mathcolor{gray}{-}}}{10}{\hat{g}}^{\;\! \textcolor{PineGreen}{\hat{2}}} \rvert^2 \Xint{\begin{smallmatrix} ~ \\ {}^{}_{\mathcolor{gray}{-}} \\ ~ \end{smallmatrix}}{15}{k}_{\;\! \symup{z}}^{\;\!  \textcolor{PineGreen}{\hat{2}}} \mathbb{e}^{\mathbb{i} \Xint{\begin{smallmatrix} ~ \\ {}^{}_{\mathcolor{gray}{-}} \\ ~ \end{smallmatrix}}{15}{k}_{\symup{z}}^{\;\!  \textcolor{PineGreen}{\hat{2}}} \mathcolor{gray}{z}}} ~, \label{eq:down-scalar-g-EE-231-chieff*-discrete}
\end{align}
\end{subequations}
定义了“\textbf{标量\textcolor{Plum}{非线性}\textcolor{NavyBlue}{波源}}”条件下,该\textcolor{NavyBlue}{稳态}过程的\textcolor{NavyBlue}{有效非线性系数}(三阶)张量
\begin{subequations} \label{eq:3wavemix-chieff-discrete}
\begin{align}
	{\chi}^{ \hat{3} \textcolor{PineGreen}{\hat{3}} \textcolor{Maroon}{(2)} }_{ \textcolor{NavyBlue}{\text{eff}} \hat{1} \textcolor{PineGreen}{\hat{1}} \hat{2} \textcolor{PineGreen}{\hat{2}} } &\xrightarrow[]{\text{\bref{eq:chieff-discrete}}} \Xint{{}^{}_{\mathcolor{gray}{-}}}{10}{\hat{g}}^{\;\! \hat{3} \textcolor{PineGreen}{\hat{3}} \textcolor{Plum}{*}} {\chi}^{\;\! \hat{3} \textcolor{PineGreen}{\hat{3}} }_{\;\! \textcolor{Maroon}{(2)} \hat{1} \textcolor{PineGreen}{\hat{1}} \hat{2} \textcolor{PineGreen}{\hat{2}} } ~ {\hat{g}}_{\;\! \hat{1} \textcolor{PineGreen}{\hat{1}} } ~ {\hat{g}}_{\;\! \hat{2} \textcolor{PineGreen}{\hat{2}} } ~, \label{eq:up-312-chieff*-discrete} \\
	{\chi}^{ \hat{1} \textcolor{PineGreen}{\hat{1}} \textcolor{Maroon}{(2)} }_{ \textcolor{NavyBlue}{\text{eff}} \hat{3} \hat{2} \textcolor{PineGreen}{\hat{3} \hat{2}} } &\xrightarrow[]{\text{\bref{eq:chieff*-discrete}}} \Xint{{}^{}_{\mathcolor{gray}{-}}}{10}{\hat{g}}^{\;\! \hat{1} \textcolor{PineGreen}{\hat{1}} \textcolor{Plum}{*}} {\chi}^{\;\! \hat{1} \textcolor{PineGreen}{\hat{1}} \textcolor{Plum}{*} }_{\;\! \textcolor{Maroon}{(2)} \hat{3} \textcolor{PineGreen}{\hat{3}} \hat{2} \textcolor{PineGreen}{\hat{2}} } ~ {\hat{g}}_{\;\! \hat{3} \textcolor{PineGreen}{\hat{3}}} ~ {\hat{g}}^{\;\! \textcolor{Plum}{*}}_{\;\! \hat{2} \textcolor{PineGreen}{\hat{2}} } ~, \label{eq:down-132-chieff*-discrete} \\
	{\chi}^{ \hat{2} \textcolor{PineGreen}{\hat{2}} \textcolor{Maroon}{(2)} }_{ \textcolor{NavyBlue}{\text{eff}} \hat{3} \textcolor{PineGreen}{\hat{3}} \hat{1} \textcolor{PineGreen}{\hat{1}} } &\xrightarrow[]{\text{$\sim$ \bref{eq:chieff*-discrete}}} \Xint{{}^{}_{\mathcolor{gray}{-}}}{10}{\hat{g}}^{\;\! \hat{2} \textcolor{PineGreen}{\hat{2}} \textcolor{Plum}{*}} {\chi}^{\;\! \hat{2} \textcolor{PineGreen}{\hat{2}} \textcolor{Plum}{*} }_{\;\! \textcolor{Maroon}{(2)} \hat{3} \hat{1} \textcolor{PineGreen}{\hat{3} \hat{1}} } ~ {\hat{g}}_{\;\! \hat{3} \textcolor{PineGreen}{\hat{3}}} ~ {\hat{g}}^{\;\! \textcolor{Plum}{*}}_{\;\! \hat{1} \textcolor{PineGreen}{\hat{1}} } ~. \label{eq:down-231-chieff*-discrete}
\end{align}
\end{subequations}

进一步,再在“\textbf{标量场 $\chi^{\;\! \mathcolor{gray}{\omega} }_{\;\! \mathcolor{gray}{z} \textcolor{Maroon}{(2)}}$ \textcolor{NavyBlue}{调制}}” \bref{eq:scalar_chi2_modulation} 的额外条件下,描述电子和光子\textcolor{NavyBlue}{窄带}相互作用的\textcolor{Maroon}{三波混频}方程从 \bref{eq:3wavemix-scalar-g-EE-chieff-discrete} 变为

\marginLeft[-2.4em]{sec:summary-chapter5}\section{\textcolor{Maroon}{Summary} 小结 \textcolor{Maroon}{of chapter 5}}\label{sec:summary-chapter5}

\cite{dregerSecondharmonicGenerationNonlinear1990,zubairyAnalyticApproachSecondharmonic1985}

too many diffs, cannot push to gitee, but github is ok

\cite{katoSecondharmonicGeneration20481986,katoTemperaturetuned90Phasematching1994,brunerTemperaturedependentSellmeierEquation2003,jundtTemperaturedependentSellmeierEquation1997,katoSellmeierThermoopticDispersion2002}

