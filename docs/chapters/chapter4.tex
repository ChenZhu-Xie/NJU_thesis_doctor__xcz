%!TEX root = ..\njuthesis-sample.tex

\chapter{任意 \texorpdfstring{$\bar{\bar{\bar{\chi}}}$}{$\bar{\bar{\bar{\text{χ}}}}$} 材料里的矢量非线性傅立叶晶体光学}\label{chap:NFCO}
\noindent\marginLeft[14.5em]{chap:NFCO}
\vspace{-\baselineskip} % 向上回退一行的高度

\bref{ssec:E-waveq-nonlinear} 末得到了\textcolor{PineGreen}{纯电(非磁)各向异性}介质中的\textcolor{Plum}{非线性}矢量电场波动方程 \bref{eq:simplify7-LE0-SVA-V_1singular-nokxky-zeta-g},其右侧的二阶\textcolor{Plum}{局域}\textcolor{Plum}{非线性}\textcolor{NavyBlue}{电偶-$(\text{电偶}\otimes\text{电偶})$}极\textcolor{NavyBlue}{波源}(的\textcolor{gray}{横向}部分) $\xint{\mathcolor{gray}{-}}{25}{\bar{P}}^{\;\! \mathcolor{gray}{\omega} \textcolor{PineGreen}{\imath}}_{\;\! \symup{\rho} \mathcolor{gray}{z} \textcolor{Maroon}{(2)}}$(\bref{eq:vec-DP^(2)-p_pp,eq:vec-DP^(2)-p_pp,eq:vec-DP^(2)-plane_wave_basis-p_pp}),可以按需更改为三/四阶\textcolor{Plum}{非线性} $\xint{\mathcolor{gray}{-}}{25}{\bar{P}}^{\;\! \mathcolor{gray}{\omega} \textcolor{PineGreen}{\imath}}_{\;\! \symup{\rho} \mathcolor{gray}{z} \textcolor{Maroon}{(3)}}, \xint{\mathcolor{gray}{-}}{25}{\bar{P}}^{\;\! \mathcolor{gray}{\omega} \textcolor{PineGreen}{\imath}}_{\;\! \symup{\rho} \mathcolor{gray}{z} \textcolor{Maroon}{(4)}}$,或者一阶\textcolor{NavyBlue}{势散射源} $\xint{\mathcolor{gray}{-}}{25}{\bar{P}}^{\;\! \mathcolor{gray}{\omega} \textcolor{PineGreen}{\imath}}_{\;\! \symup{\rho} \mathcolor{gray}{z} \textcolor{Maroon}{(1)}}$(\bref{eq:nonlinear(2)-wave_wkrho-simplify4-2}),甚至他们的\textcolor{Plum}{线性组合}/\textcolor{Plum}{叠加} $\xint{\mathcolor{gray}{-}}{25}{\bar{P}}^{\;\! \mathcolor{gray}{\omega} \textcolor{PineGreen}{\imath}}_{\;\! \symup{\rho} \mathcolor{gray}{z}}$。

将每一个参与相互作用的\textcolor{gray}{时间频率组分} $\mathcolor{gray}{\omega}$ 所对应的 \bref{eq:simplify7-LE0-SVA-V_1singular-nokxky-zeta-g} 组合起来,即得到\textcolor{PineGreen}{实验室}\textcolor{Plum}{坐标系} \textcolor{PineGreen}{$\mathcal{Z}$ 系}下的矢量(电场)\textcolor{Maroon}{时空谱}耦合波方程组 --- 它既可以由相干\textcolor{NavyBlue}{脉冲光连续谱} $\left\{ \mathcolor{gray}{\omega} \right\}$ 对应的\textcolor{Plum}{无穷个}矢量\textcolor{Maroon}{时空谱}耦合波方程组成,也可以由\textcolor{Plum}{有限个}\textcolor{NavyBlue}{连续光} $\left\{ \mathcolor{gray}{\omega}_{\textcolor{Maroon}{i}} \right\}$ 对应的有限个矢量\textcolor{Maroon}{时空谱}耦合波方程组成;也可以只是\textcolor{gray}{单一频率} $\mathcolor{gray}{\omega}$ 的光,其自己参与的如\textcolor{Maroon}{三}/\textcolor{Maroon}{四波混频}、\textcolor{Maroon}{一次电光效应}或\textcolor{PineGreen}{折射率微调制}引起的\textcolor{Maroon}{(势)散射过程}等,所对应的\textcolor{Plum}{单个}矢量\textcolor{Maroon}{时空谱}的传播/波动方程。

%\vspace*{-7.5em}
\bref{ssec:E-waveq-nonlinear} 末得到了\textcolor{PineGreen}{纯电(非磁)各向异性}介质中的\textcolor{Plum}{非线性}矢量电场波动方程 \bref{eq:simplify7-LE0-SVA-V_1singular-nokxky-zeta-g},其右侧的二阶\textcolor{Plum}{局域}\textcolor{Plum}{非线性}\textcolor{NavyBlue}{电偶-$(\text{电偶}\otimes\text{电偶})$}极\textcolor{NavyBlue}{波源}(的\textcolor{gray}{横向}部分) $\xint{\mathcolor{gray}{-}}{25}{\bar{P}}^{\;\! \mathcolor{gray}{\omega} \textcolor{PineGreen}{\imath}}_{\;\! \symup{\rho} \mathcolor{gray}{z} \textcolor{Maroon}{(2)}}$(\bref{eq:vec-DP^(2)-p_pp,eq:vec-DP^(2)-p_pp,eq:vec-DP^(2)-plane_wave_basis-p_pp}),可以按需更改为三/四阶\textcolor{Plum}{非线性} $\xint{\mathcolor{gray}{-}}{25}{\bar{P}}^{\;\! \mathcolor{gray}{\omega} \textcolor{PineGreen}{\imath}}_{\;\! \symup{\rho} \mathcolor{gray}{z} \textcolor{Maroon}{(3)}}, \xint{\mathcolor{gray}{-}}{25}{\bar{P}}^{\;\! \mathcolor{gray}{\omega} \textcolor{PineGreen}{\imath}}_{\;\! \symup{\rho} \mathcolor{gray}{z} \textcolor{Maroon}{(4)}}$,或者一阶\textcolor{NavyBlue}{势散射源} $\xint{\mathcolor{gray}{-}}{25}{\bar{P}}^{\;\! \mathcolor{gray}{\omega} \textcolor{PineGreen}{\imath}}_{\;\! \symup{\rho} \mathcolor{gray}{z} \textcolor{Maroon}{(1)}}$(\bref{eq:nonlinear(2)-wave_wkrho-simplify4-2}),甚至他们的\textcolor{Plum}{线性组合}/\textcolor{Plum}{叠加} $\xint{\mathcolor{gray}{-}}{25}{\bar{P}}^{\;\! \mathcolor{gray}{\omega} \textcolor{PineGreen}{\imath}}_{\;\! \symup{\rho} \mathcolor{gray}{z}}$。

\marginLeft[-2.4em]{sec:up_convert}\section{\textcolor{Maroon}{Up conversion} 上转换 - 电场本征复振幅 \textcolor{Maroon}{equation}}\label{sec:up_convert}

利用 \bref{eq:vec-amp_polar} 即 $\xint{{}^{}_{\mathcolor{gray}{-}}}{10}{\bar{g}}^{\;\!\mathcolor{gray}{\omega} \textcolor{PineGreen}{\jmath}}_{\;\! \textcolor{Maroon}{\Yup} \mathcolor{gray}{z}} := \xint{\begin{smallmatrix} ~ \\ {}^{}_{\mathcolor{gray}{-}} \\ ~ \end{smallmatrix}}{09}{\mathtt{g}}^{\;\!\mathcolor{gray}{\omega} \textcolor{PineGreen}{\jmath}}_{\;\! \mathcolor{gray}{z}} \xint{{}^{}_{\mathcolor{gray}{-}}}{10}{\bar{g}}^{\;\!\mathcolor{gray}{\omega} \textcolor{PineGreen}{\jmath}}_{\;\! \textcolor{Maroon}{\Yup}}$ 在\textcolor{PineGreen}{纯电(非磁)各向异性}介质中的\textcolor{gray}{横向}形式,将\textcolor{Plum}{复}矢量 $\xint{{}^{}_{\mathcolor{gray}{-}}}{10}{\bar{g}}^{\;\!\mathcolor{gray}{\omega} \textcolor{PineGreen}{\imath}}_{\;\! \textcolor{Maroon}{\Yup}}$ 三分量\textcolor{Plum}{复归一化}\Footnote{注意,不像 \bref{eq:transition_matrix-transverse_input},\textcolor{PineGreen}{本征偏振态} $\xint{{}^{}_{\mathcolor{gray}{-}}}{10}{\bar{g}}^{\;\!\mathcolor{gray}{\omega} \textcolor{PineGreen}{\imath}}_{\;\! \textcolor{Maroon}{\Yup}}, \xint{{}^{}_{\mathcolor{gray}{-}}}{10}{\bar{g}}^{\;\!\mathcolor{gray}{\omega} \textcolor{PineGreen}{\imath}}_{\;\! \textcolor{Maroon}{\symup{\rho}}}$ 在\textcolor{Plum}{线性}\textcolor{PineGreen}{晶体光学}中无需\textcolor{Plum}{(三维)复归一化},但在\textcolor{Plum}{非线性}\textcolor{PineGreen}{晶体光学}中最好\textcolor{Plum}{三维复归一化},这既有助于\textcolor{Plum}{标准化}后续引入的\textcolor{NavyBlue}{有效非线性系数},又赋予和明确了\textcolor{PineGreen}{本征复振幅} $\xint{\begin{smallmatrix} ~ \\ {}^{}_{\mathcolor{gray}{-}} \\ ~ \end{smallmatrix}}{09}{\mathtt{g}}^{\;\!\mathcolor{gray}{\omega} \textcolor{PineGreen}{\imath}}_{\;\! \mathcolor{gray}{z}}$、\textcolor{PineGreen}{本征偏振态} $\xint{{}^{}_{\mathcolor{gray}{-}}}{10}{\bar{g}}^{\;\!\mathcolor{gray}{\omega} \textcolor{PineGreen}{\imath}}_{\;\! \textcolor{Maroon}{\Yup}}$ 各自的物理意义。但实际上,\textcolor{Plum}{非线性}\textcolor{PineGreen}{晶体光学}也允许\textcolor{Plum}{不归一化} $\xint{{}^{}_{\mathcolor{gray}{-}}}{10}{\bar{g}}^{\;\!\mathcolor{gray}{\omega} \textcolor{PineGreen}{\imath}}_{\;\! \textcolor{Maroon}{\Yup}}$ 或 $\xint{{}^{}_{\mathcolor{gray}{-}}}{10}{\bar{g}}^{\;\!\mathcolor{gray}{\omega} \textcolor{PineGreen}{\imath}}_{\;\! \textcolor{Maroon}{\symup{\rho}}}$,见下文。}后的\textcolor{Plum}{二维}\textcolor{PineGreen}{本征偏振态} $\xint{{}^{}_{\mathcolor{gray}{-}}}{10}{\bar{g}}^{\;\!\mathcolor{gray}{\omega} \textcolor{PineGreen}{\imath}}_{\;\! \textcolor{Maroon}{\symup{\rho}} \mathcolor{gray}{z}} := \xint{\begin{smallmatrix} ~ \\ {}^{}_{\mathcolor{gray}{-}} \\ ~ \end{smallmatrix}}{09}{\mathtt{g}}^{\;\!\mathcolor{gray}{\omega} \textcolor{PineGreen}{\imath}}_{\;\! \mathcolor{gray}{z}} \xint{{}^{}_{\mathcolor{gray}{-}}}{10}{\hat{g}}^{\;\!\mathcolor{gray}{\omega} \textcolor{PineGreen}{\imath}}_{\;\! \textcolor{Maroon}{\symup{\rho}}}$代入矢量\textcolor{Plum}{非线性}波动方程 \bref{eq:simplify7-LE0-SVA-V_1singular-nokxky-zeta-g} 的\textcolor{NavyBlue}{左侧场}中,并两侧\textcolor{Plum}{左点乘}不含 $\mathcolor{gray}{z}$ 的\textcolor{Plum}{二维横向}\textcolor{PineGreen}{本征偏振态} $\xint{{}^{}_{\mathcolor{gray}{-}}}{10}{\hat{g}}^{\;\!\mathcolor{gray}{\omega} \textcolor{PineGreen}{\imath}}_{\;\! \textcolor{Maroon}{\symup{\rho}}}$ 的\textcolor{Plum}{共轭转置} $\xint{{}^{}_{\mathcolor{gray}{-}}}{10}{\hat{g}}^{\;\!\mathcolor{gray}{\omega} \textcolor{PineGreen}{\imath} \textcolor{Plum}{\dag}}_{\;\! \textcolor{Maroon}{\symup{\rho}}}$,可得
\begin{subequations} \label{eq:simplify7-scalar}
	\begin{align}
		\xint{{}^{}_{\mathcolor{gray}{-}}}{10}{\hat{g}}^{\;\!\mathcolor{gray}{\omega} \textcolor{PineGreen}{\imath}}_{\;\! \textcolor{Maroon}{\symup{\rho}}} \mathcolor{gray}{\nabla_z} \xint{\begin{smallmatrix} ~ \\ {}^{}_{\mathcolor{gray}{-}} \\ ~ \end{smallmatrix}}{09}{\mathtt{g}}^{\;\!\mathcolor{gray}{\omega} \textcolor{PineGreen}{\imath}}_{\;\! \mathcolor{gray}{z}} &\xrightarrow[\text{\bref{eq:simplify7-LE0-SVA-V_1singular-nokxky-zeta-g}}]{\text{$\textcolor{Maroon}{\Yup} \to \textcolor{Maroon}{\symup{\rho}} \left( \text{\bref{eq:vec-amp_polar}} \right)$}} \mathbb{i} k_{\textcolor{Maroon}{\mathsf{o}} \mathcolor{gray}{\omega}}^{\;\! 2} \frac{\xint{\mathcolor{gray}{-}}{25}{\bar{P}}^{\;\! \mathcolor{gray}{\omega} \textcolor{PineGreen}{\imath}}_{\;\! \textcolor{Maroon}{\symup{\rho}} \mathcolor{gray}{z}}}{2 \xint{\begin{smallmatrix} ~ \\ {}^{}_{\mathcolor{gray}{-}} \\ ~ \end{smallmatrix}}{15}{k}_{\;\! \symup{z}}^{\;\! \mathcolor{gray}{\omega} \textcolor{PineGreen}{\imath}} \mathbb{e}^{\mathbb{i} \xint{\begin{smallmatrix} ~ \\ {}^{}_{\mathcolor{gray}{-}} \\ ~ \end{smallmatrix}}{15}{k}_{\symup{z}}^{\;\! \mathcolor{gray}{\omega} \textcolor{PineGreen}{\imath}} \mathcolor{gray}{z}}} \label{eq:simplify7-scalar-g} \\
		\mathcolor{gray}{\nabla_z} \xint{\begin{smallmatrix} ~ \\ {}^{}_{\mathcolor{gray}{-}} \\ ~ \end{smallmatrix}}{09}{\mathtt{g}}^{\;\!\mathcolor{gray}{\omega} \textcolor{PineGreen}{\imath}}_{\;\! \mathcolor{gray}{z}} &\xrightarrow[\text{$\text{■} $\bref{eq:simplify7-scalar-g}}]{\text{$\left( \xint{{}^{}_{\mathcolor{gray}{-}}}{10}{\hat{g}}^{\;\!\mathcolor{gray}{\omega} \textcolor{PineGreen}{\imath} \textcolor{Plum}{\dag}}_{\;\! \textcolor{Maroon}{\symup{\rho}}} \cdot \text{■} \right) \big/ \left( \xint{{}^{}_{\mathcolor{gray}{-}}}{10}{\hat{g}}^{\;\!\mathcolor{gray}{\omega} \textcolor{PineGreen}{\imath} \textcolor{Plum}{\dag}}_{\;\! \textcolor{Maroon}{\symup{\rho}}} \cdot \xint{{}^{}_{\mathcolor{gray}{-}}}{10}{\hat{g}}^{\;\!\mathcolor{gray}{\omega} \textcolor{PineGreen}{\imath}}_{\;\! \textcolor{Maroon}{\symup{\rho}}} \right)$}} \mathbb{i} k_{\textcolor{Maroon}{\mathsf{o}} \mathcolor{gray}{\omega}}^{\;\! 2} \frac{\xint{{}^{}_{\mathcolor{gray}{-}}}{10}{\hat{g}}^{\;\!\mathcolor{gray}{\omega} \textcolor{PineGreen}{\imath} \textcolor{Plum}{\dag}}_{\;\! \textcolor{Maroon}{\symup{\rho}}} \cdot \xint{\mathcolor{gray}{-}}{25}{\bar{P}}^{\;\! \mathcolor{gray}{\omega} \textcolor{PineGreen}{\imath}}_{\;\! \textcolor{Maroon}{\symup{\rho}} \mathcolor{gray}{z}}}{\xint{{}^{}_{\mathcolor{gray}{-}}}{10}{\hat{g}}^{\;\!\mathcolor{gray}{\omega} \textcolor{PineGreen}{\imath} \textcolor{Plum}{\dag}}_{\;\! \textcolor{Maroon}{\symup{\rho}}} \cdot \xint{{}^{}_{\mathcolor{gray}{-}}}{10}{\hat{g}}^{\;\!\mathcolor{gray}{\omega} \textcolor{PineGreen}{\imath}}_{\;\! \textcolor{Maroon}{\symup{\rho}}} 2 \xint{\begin{smallmatrix} ~ \\ {}^{}_{\mathcolor{gray}{-}} \\ ~ \end{smallmatrix}}{15}{k}_{\;\! \symup{z}}^{\;\! \mathcolor{gray}{\omega} \textcolor{PineGreen}{\imath}} \mathbb{e}^{\mathbb{i} \xint{\begin{smallmatrix} ~ \\ {}^{}_{\mathcolor{gray}{-}} \\ ~ \end{smallmatrix}}{15}{k}_{\symup{z}}^{\;\! \mathcolor{gray}{\omega} \textcolor{PineGreen}{\imath}} \mathcolor{gray}{z}}} ~, \label{eq:simplify7-scalar-g-conjugate}
	\end{align}
\end{subequations}
其中 \bref{eq:simplify7-scalar-g-conjugate} 即为标量\textcolor{Maroon}{时空谱},即\textcolor{PineGreen}{本征复振幅} $\xint{\begin{smallmatrix} ~ \\ {}^{}_{\mathcolor{gray}{-}} \\ ~ \end{smallmatrix}}{09}{\mathtt{g}}^{\;\!\mathcolor{gray}{\omega} \textcolor{PineGreen}{\imath}}_{\;\! \mathcolor{gray}{z}}$ 满足的矢量\textcolor{Plum}{非线性}波动方程。分母中的 $\xint{{}^{}_{\mathcolor{gray}{-}}}{10}{\hat{g}}^{\;\!\mathcolor{gray}{\omega} \textcolor{PineGreen}{\imath} \textcolor{Plum}{\dag}}_{\;\! \textcolor{Maroon}{\symup{\rho}}} \cdot \xint{{}^{}_{\mathcolor{gray}{-}}}{10}{\hat{g}}^{\;\!\mathcolor{gray}{\omega} \textcolor{PineGreen}{\imath}}_{\;\! \textcolor{Maroon}{\symup{\rho}}}$,其实在暗示 $\xint{{}^{}_{\mathcolor{gray}{-}}}{10}{\hat{g}}^{\;\!\mathcolor{gray}{\omega} \textcolor{PineGreen}{\imath}}_{\;\! \textcolor{Maroon}{\symup{\rho}}}$ 既可以是\textcolor{Plum}{二维复归一化},也可以是\textcolor{Plum}{三维复归一化}后的。这对应 $\xint{{}^{}_{\mathcolor{gray}{-}}}{10}{\hat{g}}^{\;\!\mathcolor{gray}{\omega} \textcolor{PineGreen}{\imath} \textcolor{Plum}{\dag}}_{\;\! \textcolor{Maroon}{\symup{\rho}}} \cdot \xint{{}^{}_{\mathcolor{gray}{-}}}{10}{\hat{g}}^{\;\!\mathcolor{gray}{\omega} \textcolor{PineGreen}{\imath}}_{\;\! \textcolor{Maroon}{\symup{\rho}}}$ 的值,既可以是也可以不是 $1$。并且甚至可以对 \bref{eq:simplify7-scalar-g} 左侧随便\textcolor{Plum}{点乘}一个\textcolor{Plum}{二维}复向量(场) $\xint{{}^{}_{\mathcolor{gray}{-}}}{04}{\bar{c}}^{\;\!\mathcolor{gray}{\omega} \textcolor{PineGreen}{\imath}}_{\;\! \textcolor{Maroon}{\symup{\rho}}}$(不一定非得是\textcolor{gray}{横向}\textcolor{PineGreen}{本征偏振态}的\textcolor{Plum}{共轭转置} $\xint{{}^{}_{\mathcolor{gray}{-}}}{10}{\hat{g}}^{\;\!\mathcolor{gray}{\omega} \textcolor{PineGreen}{\imath} \textcolor{Plum}{\dag}}_{\;\! \textcolor{Maroon}{\symup{\rho}}}$)都行,只需保证 \bref{eq:simplify7-scalar-g-conjugate} 分母中的 $\xint{{}^{}_{\mathcolor{gray}{-}}}{04}{\bar{c}}^{\;\!\mathcolor{gray}{\omega} \textcolor{PineGreen}{\imath}}_{\;\! \textcolor{Maroon}{\symup{\rho}}} \cdot \xint{{}^{}_{\mathcolor{gray}{-}}}{10}{\hat{g}}^{\;\!\mathcolor{gray}{\omega} \textcolor{PineGreen}{\imath}}_{\;\! \textcolor{Maroon}{\symup{\rho}}} \neq 0$。

\bref{eq:simplify7-scalar} 中所有\textcolor{gray}{横向} $\textcolor{Maroon}{\symup{\rho}}$ 场,somehow\Footnote{出于\textcolor{NavyBlue}{物理学家}的直觉和对形式美的追求。类似\textcolor{Plum}{数学家}的“注意到”,但没有他们的“注意到”那么严谨。}可进而写做笛卡尔\textcolor{Plum}{三分量} $\textcolor{Maroon}{\Yup}$ 形式
\begin{subequations} \label{eq:simplify8-scalar}
\begin{align}
	\xint{{}^{}_{\mathcolor{gray}{-}}}{10}{\hat{g}}^{\;\!\mathcolor{gray}{\omega} \textcolor{PineGreen}{\imath}}_{\;\! \textcolor{Maroon}{\Yup}} \mathcolor{gray}{\nabla_z} \xint{\begin{smallmatrix} ~ \\ {}^{}_{\mathcolor{gray}{-}} \\ ~ \end{smallmatrix}}{09}{\mathtt{g}}^{\;\!\mathcolor{gray}{\omega} \textcolor{PineGreen}{\imath}}_{\;\! \mathcolor{gray}{z}} &\xrightarrow[\text{\bref{eq:simplify7-scalar-g}}]{\text{$\textcolor{Maroon}{\symup{\rho}} \to \textcolor{Maroon}{\Yup}$}} \mathbb{i} k_{\textcolor{Maroon}{\mathsf{o}} \mathcolor{gray}{\omega}}^{\;\! 2} \frac{\xint{\mathcolor{gray}{-}}{25}{\bar{P}}^{\;\! \mathcolor{gray}{\omega} \textcolor{PineGreen}{\imath}}_{\;\! \textcolor{Maroon}{\Yup} \mathcolor{gray}{z}}}{2 \xint{\begin{smallmatrix} ~ \\ {}^{}_{\mathcolor{gray}{-}} \\ ~ \end{smallmatrix}}{15}{k}_{\;\! \symup{z}}^{\;\! \mathcolor{gray}{\omega} \textcolor{PineGreen}{\imath}} \mathbb{e}^{\mathbb{i} \xint{\begin{smallmatrix} ~ \\ {}^{}_{\mathcolor{gray}{-}} \\ ~ \end{smallmatrix}}{15}{k}_{\symup{z}}^{\;\! \mathcolor{gray}{\omega} \textcolor{PineGreen}{\imath}} \mathcolor{gray}{z}}} \label{eq:simplify8-scalar-g} \\
	\mathcolor{gray}{\nabla_z} \xint{\begin{smallmatrix} ~ \\ {}^{}_{\mathcolor{gray}{-}} \\ ~ \end{smallmatrix}}{09}{\mathtt{g}}^{\;\!\mathcolor{gray}{\omega} \textcolor{PineGreen}{\imath}}_{\;\! \mathcolor{gray}{z}} &\xrightarrow[\text{\bref{eq:simplify7-scalar-g-conjugate}}]{\text{$\textcolor{Maroon}{\symup{\rho}} \to \textcolor{Maroon}{\Yup}$}} \mathbb{i} k_{\textcolor{Maroon}{\mathsf{o}} \mathcolor{gray}{\omega}}^{\;\! 2} \frac{\xint{{}^{}_{\mathcolor{gray}{-}}}{10}{\hat{g}}^{\;\!\mathcolor{gray}{\omega} \textcolor{PineGreen}{\imath} \textcolor{Plum}{\dag}}_{\;\! \textcolor{Maroon}{\Yup}} \cdot \xint{\mathcolor{gray}{-}}{25}{\bar{P}}^{\;\! \mathcolor{gray}{\omega} \textcolor{PineGreen}{\imath}}_{\;\! \textcolor{Maroon}{\Yup} \mathcolor{gray}{z}}}{\xint{{}^{}_{\mathcolor{gray}{-}}}{10}{\hat{g}}^{\;\!\mathcolor{gray}{\omega} \textcolor{PineGreen}{\imath} \textcolor{Plum}{\dag}}_{\;\! \textcolor{Maroon}{\Yup}} \cdot \xint{{}^{}_{\mathcolor{gray}{-}}}{10}{\hat{g}}^{\;\!\mathcolor{gray}{\omega} \textcolor{PineGreen}{\imath}}_{\;\! \textcolor{Maroon}{\Yup}} 2 \xint{\begin{smallmatrix} ~ \\ {}^{}_{\mathcolor{gray}{-}} \\ ~ \end{smallmatrix}}{15}{k}_{\;\! \symup{z}}^{\;\! \mathcolor{gray}{\omega} \textcolor{PineGreen}{\imath}} \mathbb{e}^{\mathbb{i} \xint{\begin{smallmatrix} ~ \\ {}^{}_{\mathcolor{gray}{-}} \\ ~ \end{smallmatrix}}{15}{k}_{\symup{z}}^{\;\! \mathcolor{gray}{\omega} \textcolor{PineGreen}{\imath}} \mathcolor{gray}{z}}} ~,  \label{eq:simplify8-scalar-g-conjugate}
\end{align}
\end{subequations}
注,\bref{eq:simplify8-scalar-g-conjugate} 仍属于矢量\textcolor{Plum}{非线性}波动方程 \bypertarget{waveq-scalar-2-vector},尽管左侧是对\textcolor{PineGreen}{本征复振幅}标量场 $\xint{\begin{smallmatrix} ~ \\ {}^{}_{\mathcolor{gray}{-}} \\ ~ \end{smallmatrix}}{09}{\mathtt{g}}^{\;\!\mathcolor{gray}{\omega} \textcolor{PineGreen}{\imath}}_{\;\! \mathcolor{gray}{z}}$ 的 $\mathcolor{gray}{\nabla_z}$:因为 {\one} 右侧的 $\xint{\mathcolor{gray}{-}}{25}{\bar{P}}^{\;\! \mathcolor{gray}{\omega} \textcolor{PineGreen}{\imath}}_{\;\! \textcolor{Maroon}{\Yup} \mathcolor{gray}{z}}$ 是矢量的;此外,{\two} \textcolor{PineGreen}{本征复振幅}标量场 $\xint{\begin{smallmatrix} ~ \\ {}^{}_{\mathcolor{gray}{-}} \\ ~ \end{smallmatrix}}{09}{\mathtt{g}}^{\;\!\mathcolor{gray}{\omega} \textcolor{PineGreen}{\imath}}_{\;\! \mathcolor{gray}{z}}$ 一旦已知,再乘以\textcolor{PineGreen}{本征偏振态} $\xint{{}^{}_{\mathcolor{gray}{-}}}{10}{\hat{g}}^{\;\!\mathcolor{gray}{\omega} \textcolor{PineGreen}{\imath}}_{\;\! \textcolor{Maroon}{\Yup}}$ 后,可直接转换为矢量\textcolor{Maroon}{时空谱}三分量 $\xint{{}^{}_{\mathcolor{gray}{-}}}{10}{\bar{g}}^{\;\!\mathcolor{gray}{\omega} \textcolor{PineGreen}{\imath}}_{\;\! \textcolor{Maroon}{\Yup} \mathcolor{gray}{z}} := \xint{\begin{smallmatrix} ~ \\ {}^{}_{\mathcolor{gray}{-}} \\ ~ \end{smallmatrix}}{09}{\mathtt{g}}^{\;\!\mathcolor{gray}{\omega} \textcolor{PineGreen}{\imath}}_{\;\! \mathcolor{gray}{z}} \xint{{}^{}_{\mathcolor{gray}{-}}}{10}{\hat{g}}^{\;\!\mathcolor{gray}{\omega} \textcolor{PineGreen}{\imath}}_{\;\! \textcolor{Maroon}{\Yup}}$,或通过 \bref{eq:vec-eigenmode_amp-matrix} 一步到位至电矢量场\textcolor{Maroon}{傅立叶谱} $\xint{\mathcolor{gray}{-}}{25}{\bar{E}}^{\;\!\mathcolor{gray}{\omega}}_{\;\! \mathcolor{gray}{z}}$。

\marginLeft[-2.4em]{ssec:SHG_spectrum}\subsection{脉冲光倍频 - 电场本征复振幅方程}\label{ssec:SHG_spectrum}

对于以\textcolor{NavyBlue}{脉冲光}\textcolor{Maroon}{倍频}\cite{boydNonlinearOptics2019}为主\Footnote{若 $\mathcolor{gray}{\omega}_{\textcolor{Maroon}{\text{P}}}, \mathcolor{gray}{\omega}$ 分别在\textcolor{NavyBlue}{泵浦光}脉冲\textcolor{gray}{中心频率} $\mathcolor{gray}{\Omega}_{\textcolor{Maroon}{\text{P}}} = \mathcolor{gray}{\Omega} \big/ 2$ 及其产生的 $2\mathcolor{gray}{\omega}_{\textcolor{Maroon}{\text{P}}}$ \textcolor{Maroon}{倍频}\textcolor{NavyBlue}{光脉冲}\textcolor{gray}{中心频率} $\mathcolor{gray}{\Omega} = 2\mathcolor{gray}{\Omega}_{\textcolor{Maroon}{\text{P}}}$ 附近,且 $\mathcolor{gray}{\omega} = 2\mathcolor{gray}{\omega}_{\textcolor{Maroon}{\text{P}}} > \textcolor{gray}{0}$,则该式代表单\textcolor{NavyBlue}{脉冲光}\textcolor{Maroon}{倍频}过程。}、以\textcolor{NavyBlue}{脉冲}\textcolor{Maroon}{光整流}后续级联\textcolor{Maroon}{电光效应}\cite{jangMulticycleTerahertzPulse2020}为辅\Footnote{若 $\mathcolor{gray}{\omega}, \mathcolor{gray}{\omega}_{\textcolor{Maroon}{\text{THz}}}$ 分别在\textcolor{NavyBlue}{泵浦}\textcolor{NavyBlue}{光脉冲}\textcolor{gray}{中心频率} $\mathcolor{gray}{\Omega} \gg \mathcolor{gray}{\Omega}_{\textcolor{Maroon}{\text{THz}}}$ 及其产生的 \textcolor{Maroon}{THz} \textcolor{NavyBlue}{脉冲}的\textcolor{gray}{中心频率} $\mathcolor{gray}{\Omega}_{\textcolor{Maroon}{\text{THz}}} \ll \mathcolor{gray}{\Omega}$ 附近,且 $\mathcolor{gray}{\omega} \gg \mathcolor{gray}{\omega}_{\textcolor{Maroon}{\text{THz}}} > \textcolor{gray}{0}$,则该式代表\textcolor{NavyBlue}{脉冲}\textcolor{Maroon}{光整流}后续级联\textcolor{Maroon}{电光效应}\cite{jangMulticycleTerahertzPulse2020}过程。对应\textcolor{NavyBlue}{脉冲电}(\textcolor{Maroon}{THz})与\textcolor{NavyBlue}{脉冲光}的\textcolor{Maroon}{和频}。}的 $\mathcolor{gray}{\omega'} + \left( \mathcolor{gray}{\omega}-\mathcolor{gray}{\omega'} \right) \to \mathcolor{gray}{\omega} > \textcolor{gray}{0}$\Footnote{在映射到\textcolor{NavyBlue}{物理过程}时,默认\textcolor{gray}{各频率}(对应 cos 余弦)为正;但在\textcolor{Plum}{数学积分}(对应 $\mathbb{e}$ 指数)中可为负。}二阶\textcolor{Plum}{非线性}\textcolor{gray}{频率}\textcolor{Maroon}{上转换}过程,波动方程 \bref{eq:simplify7-scalar-g} 与 \bref{eq:simplify8-scalar-g} \textcolor{Plum}{非线性}\textcolor{NavyBlue}{波源}项 $\xint{\mathcolor{gray}{-}}{25}{\bar{P}}^{\;\! \mathcolor{gray}{\omega} \textcolor{PineGreen}{\imath}}_{\;\! \textcolor{Maroon}{\Yup} \mathcolor{gray}{z}} = \mathcolor{gray}{\mathcal F} \left[ \bar{P}^{\;\! \mathcolor{gray}{\omega} \textcolor{PineGreen}{\imath}}_{\;\! \textcolor{Maroon}{\Yup} \mathcolor{gray}{z}} \right]$ 进一步限定为,\textcolor{PineGreen}{本征模}\textcolor{Plum}{符号替换} $\textcolor{PineGreen}{\hat{3}},\textcolor{PineGreen}{\hat{2}},\textcolor{PineGreen}{\hat{1}} = \textcolor{PineGreen}{\imath},\textcolor{PineGreen}{\jmath},\textcolor{PineGreen}{l}$ 后的,\textcolor{PineGreen}{平面波基}下的二阶\textcolor{Plum}{局域}\textcolor{Plum}{非线性}\textcolor{NavyBlue}{电偶-$(\text{电偶}\otimes\text{电偶})$}极矩场 \bref{eq:vec-DP^(2)-plane_wave_basis-p_pp}:
\begin{subequations} \label{eq:DP^(2)-3_12-spectrum}
\begin{align}
	\xint{\mathcolor{gray}{-}}{30}{\bar{P}}^{\;\! \mathcolor{gray}{\omega} \textcolor{Maroon}{(2)} }_{\;\! \mathcolor{gray}{z} \textcolor{PineGreen}{\hat{3}}} &\xrightarrow[\text{\bref{eq:vec-DP^(2)-plane_wave_basis-p_pp}}]{\text{$\textcolor{PineGreen}{\imath},\textcolor{PineGreen}{\jmath},\textcolor{PineGreen}{l} \to \textcolor{PineGreen}{\hat{3}},\textcolor{PineGreen}{\hat{2}},\textcolor{PineGreen}{\hat{1}}$}} \xint{{}^{}_{\mathcolor{gray}{-}}}{23}{\bar{\bar{\bar{\chi}}}}^{\;\! \mathcolor{gray}{\omega} \textcolor{Maroon}{(2)}}_{\mathcolor{gray}{z} \textcolor{PineGreen}{\hat{3} \hat{1} \hat{2}} } ~{}^{\mathcolor{gray}{*}}_{\mathcolor{gray}{*}} \left( \xint{\mathcolor{gray}{-}}{295}{\bar{E}}^{\;\! \mathcolor{gray}{\omega} \textcolor{PineGreen}{\hat{1}} }_{\;\! \mathcolor{gray}{z} } ~\mathcolor{gray}{\widetilde \circledast}~ \xint{\mathcolor{gray}{-}}{295}{\bar{E}}^{\;\! \mathcolor{gray}{\omega} \textcolor{PineGreen}{\hat{2}} }_{\;\! \mathcolor{gray}{z} } \right) \label{eq:vec-DP^(2)-3_12-spectrum} \\
	\xint{\mathcolor{gray}{-}}{30}{P}^{\;\! \textcolor{PineGreen}{\hat{3}} \mathcolor{gray}{\omega} }_{\;\! \hat{3}\mathcolor{gray}{z} \textcolor{Maroon}{(2)} } &\xrightarrow[\text{\bref{eq:components-DP^(2)-plane_wave_basis-p_pp}}]{\text{$\textcolor{PineGreen}{\imath},\textcolor{PineGreen}{\jmath},\textcolor{PineGreen}{l} \to \textcolor{PineGreen}{\hat{3}},\textcolor{PineGreen}{\hat{2}},\textcolor{PineGreen}{\hat{1}}$}} \xint{{}^{}_{\mathcolor{gray}{-}}}{23}{\chi}^{\;\! \textcolor{PineGreen}{\hat{3}} \mathcolor{gray}{\omega} \hat{1} \hat{2} }_{\;\! \hat{3} \mathcolor{gray}{z} \textcolor{PineGreen}{\hat{1} \hat{2}} \textcolor{Maroon}{(2)}} \mathcolor{gray}{*} \left( \xint{\mathcolor{gray}{-}}{295}{E}^{\;\! \textcolor{PineGreen}{\hat{1}} \mathcolor{gray}{\omega} }_{\;\! \hat{1} \mathcolor{gray}{z}} ~\mathcolor{gray}{\widetilde \circledast}~ \xint{\mathcolor{gray}{-}}{295}{E}^{\;\! \textcolor{PineGreen}{\hat{2}} \mathcolor{gray}{\omega} }_{\;\! \hat{2} \mathcolor{gray}{z}} \right) ~, \label{eq:components-DP^(2)-3_12-spectrum}
\end{align}
\end{subequations}
同时,矢量\textcolor{Plum}{非线性}波动方程 \bref{eq:simplify8-scalar-g-conjugate} 也简写作
\begin{align} \label{eq:simplify8-scalar-g-modulus}
	\mathcolor{gray}{\nabla_z} \xint{\begin{smallmatrix} ~ \\ {}^{}_{\mathcolor{gray}{-}} \\ ~ \end{smallmatrix}}{09}{\mathtt{g}}^{\;\!\mathcolor{gray}{\omega} \textcolor{PineGreen}{\hat{3}}}_{\;\! \mathcolor{gray}{z}} &\xrightarrow[\text{\bref{eq:simplify8-scalar-g-conjugate}}]{\text{$\textcolor{PineGreen}{\imath} \to \textcolor{PineGreen}{\hat{3}}$}} \mathbb{i} k_{\textcolor{Maroon}{\mathsf{o}} \mathcolor{gray}{\omega}}^{\;\! 2} \frac{\xint{{}^{}_{\mathcolor{gray}{-}}}{10}{\hat{g}}^{\;\! \textcolor{PineGreen}{\hat{3}} \textcolor{Plum}{\dag}}_{\;\! \mathcolor{gray}{\omega}} \cdot \xint{\mathcolor{gray}{-}}{25}{\bar{P}}^{\;\! \mathcolor{gray}{\omega} \textcolor{PineGreen}{\hat{3}} }_{\;\! \mathcolor{gray}{z}  \textcolor{Maroon}{(2)}}}{ 2 \lvert \xint{{}^{}_{\mathcolor{gray}{-}}}{10}{\hat{g}}^{\;\! \textcolor{PineGreen}{\hat{3}}}_{\;\! \mathcolor{gray}{\omega}} \rvert^2 \xint{\begin{smallmatrix} ~ \\ {}^{}_{\mathcolor{gray}{-}} \\ ~ \end{smallmatrix}}{15}{k}_{\;\! \symup{z}}^{\;\! \mathcolor{gray}{\omega} \textcolor{PineGreen}{\hat{3}}} \mathbb{e}^{\mathbb{i} \xint{\begin{smallmatrix} ~ \\ {}^{}_{\mathcolor{gray}{-}} \\ ~ \end{smallmatrix}}{15}{k}_{\symup{z}}^{\;\! \mathcolor{gray}{\omega} \textcolor{PineGreen}{\hat{3}}} \mathcolor{gray}{z}}} ~.
\end{align}

二阶\textcolor{Plum}{非线性}系数 $\xint{{}^{}_{\mathcolor{gray}{-}}}{23}{\chi}^{\;\! \mathcolor{gray}{\omega} \hat{1} \hat{2} }_{\;\! \hat{3} \mathcolor{gray}{z} \textcolor{Maroon}{(2)} }$ 总可分解为\textcolor{Plum}{均匀背景} ${\chi}^{\;\! \mathcolor{gray}{\omega} \hat{1} \hat{2} }_{\;\! \hat{3} \textcolor{Maroon}{(2)} }$ 与\textcolor{Plum}{调制函数} $\xint{\mathcolor{gray}{-}}{18}{M}^{\;\! \mathcolor{gray}{\omega} \hat{1} \hat{2} }_{\;\! \hat{3} \mathcolor{gray}{z} \textcolor{Maroon}{(2)} }$ 之积
%\Footnote{当不存在“\textcolor{PineGreen}{模式}”\textcolor{Plum}{角标},以\textcolor{Plum}{推断}\textcolor{NavyBlue}{场量}的\textcolor{Plum}{自变量}时,不能省略 $\mathcolor{gray}{\omega}$ \textcolor{Plum}{角标}。}
\begin{subequations} \label{eq:chi2-modulate}
\begin{align}
	\xint{{}^{}_{\mathcolor{gray}{-}}}{23}{\bar{\bar{\bar{\chi}}}} ^{\;\! \mathcolor{gray}{\omega} } _{\;\! \mathcolor{gray}{z} \textcolor{Maroon}{(2)} } &= \bar{\bar{\bar{\chi}}}^{\;\! \mathcolor{gray}{\omega} }_{\;\! \textcolor{Maroon}{(2)} } \odot \xint{\mathcolor{gray}{-}}{18}{\bar{\bar{\bar{M}}}}^{\;\! \mathcolor{gray}{\omega} }_{\;\! \mathcolor{gray}{z} \textcolor{Maroon}{(2)} } \label{eq:vec-chi2-modulate} \\
	\xint{{}^{}_{\mathcolor{gray}{-}}}{23}{\chi}^{\;\! \mathcolor{gray}{\omega} \hat{1} \hat{2} }_{\;\! \hat{3} \mathcolor{gray}{z} \textcolor{Maroon}{(2)} } &= {\chi}^{\;\! \mathcolor{gray}{\omega} \hat{1} \hat{2} }_{\;\! \hat{3} \textcolor{Maroon}{(2)} } \xint{\mathcolor{gray}{-}}{18}{M}^{\;\! \mathcolor{gray}{\omega} \hat{1} \hat{2} }_{\;\! \hat{3} \mathcolor{gray}{z} \textcolor{Maroon}{(2)} } ~, \label{eq:components-chi2-modulate}
\end{align}
\end{subequations}
这里隐式地定义了\textcolor{Plum}{哈达马积}/\textcolor{Plum}{对应元素积} $\odot$ 或 $"{.\cdot}"$,类似于 matlab 的 $"{.*}"$ 语法(矩阵对应元素相乘)。注意,三阶\textcolor{Plum}{调制张量}\textcolor{NavyBlue}{场} $\xint{\mathcolor{gray}{-}}{18}{\bar{\bar{\bar{M}}}}^{\;\! \mathcolor{gray}{\omega} }_{\;\! \mathcolor{gray}{z} \textcolor{Maroon}{(2)} }$ 像 ${\chi}^{\;\! \mathcolor{gray}{\omega} }_{\;\! \textcolor{Maroon}{(2)} }$(\textcolor{NavyBlue}{非场},没有\textcolor{gray}{“$-$”标志})一样,仍是\textcolor{gray}{波长} $\mathcolor{gray}{\lambda}$ 的函数,即仍是 $\mathcolor{gray}{\omega}$ \textcolor{NavyBlue}{色散}(\textcolor{Plum}{各向异性})的。但分离出的\textcolor{Plum}{定常}\textcolor{Plum}{均匀背景}张量 ${\chi}^{\;\! \mathcolor{gray}{\omega} }_{\;\! \textcolor{Maroon}{(2)} }$ 因子,不再是 $\mathcolor{gray}{\bar{r}}$ 的函数,并且可以从许多\textcolor{NavyBlue}{实验主导}的文献中获得\cite{nyePhysicalPropertiesCrystals2012,zuOpticalSecondHarmonic2024,zuAnalyticalNumericalModeling2022,gananyQuasiphaseMatchingLiNbO32006,segondsLinearNonlinearOptical2004,dolevLinearNonlinearOptical2009,kaschkeCalculationNonlinearOptical1989,itoGeneralizedStudyAngular1975}。注意,$\xint{{}^{}_{\mathcolor{gray}{-}}}{23}{\bar{\bar{\bar{\chi}}}}^{\;\! \mathcolor{gray}{\omega} }_{\;\! \mathcolor{gray}{z} \textcolor{Maroon}{(2)} }$ 不是\textcolor{PineGreen}{本征模} $\textcolor{PineGreen}{\hat{3}},\textcolor{PineGreen}{\hat{2}},\textcolor{PineGreen}{\hat{1}}$ 的函数 \bypertarget{chi2-free-of-eigenmodes}。

将 $\mathcolor{gray}{\bar{r}}$ 域上\textcolor{Plum}{被调制}的 $\xint{{}^{}_{\mathcolor{gray}{-}}}{23}{\chi}^{\;\! \mathcolor{gray}{\omega} \hat{1} \hat{2} }_{\;\! \hat{3} \mathcolor{gray}{z} \textcolor{Maroon}{(2)} }$ \bref{eq:components-chi2-modulate} 代入\textcolor{Plum}{非线性}\textcolor{NavyBlue}{波源} \bref{eq:components-DP^(2)-3_12-spectrum} 得
\begin{subequations} \label{eq:DP^(2)-3_12-spectrum-SFG}
\begin{align}
	\xint{\mathcolor{gray}{-}}{30}{P}^{\;\! \textcolor{PineGreen}{\hat{3}} \mathcolor{gray}{\omega} }_{\;\! \hat{3}\mathcolor{gray}{z} \textcolor{Maroon}{(2)} } &\xrightarrow[]{\text{\bref{eq:components-DP^(2)-3_12-spectrum}}} \xint{{}^{}_{\mathcolor{gray}{-}}}{23}{\chi}^{\;\! \textcolor{PineGreen}{\hat{3}} \mathcolor{gray}{\omega} \hat{1} \hat{2} }_{\;\! \hat{3} \mathcolor{gray}{z} \textcolor{PineGreen}{\hat{1} \hat{2}} \textcolor{Maroon}{(2)}} \mathcolor{gray}{*} \left( \xint{\mathcolor{gray}{-}}{295}{E}^{\;\!\textcolor{PineGreen}{\hat{1}} \mathcolor{gray}{\omega}}_{\;\! \hat{1} \mathcolor{gray}{z}} ~\mathcolor{gray}{\widetilde \circledast}~ \xint{\mathcolor{gray}{-}}{295}{E}^{\;\!\textcolor{PineGreen}{\hat{2}} \mathcolor{gray}{\omega}}_{\;\! \hat{2} \mathcolor{gray}{z}} \right) \label{eq:DP^(2)-3_12-spectrum-SFG1} \\
	&\xrightarrow[]{\text{\bref{eq:components-chi2-modulate}}} {\chi}^{\;\! \textcolor{PineGreen}{\hat{3}} \mathcolor{gray}{\omega} \hat{1} \hat{2} }_{\;\! \hat{3} \textcolor{Maroon}{(2)} \textcolor{PineGreen}{\hat{1} \hat{2}}} \xint{\mathcolor{gray}{-}}{18}{M}^{\;\! \mathcolor{gray}{\omega} \hat{1} \hat{2} }_{\;\! \hat{3} \mathcolor{gray}{z} \textcolor{Maroon}{(2)} } \mathcolor{gray}{*} \left( \xint{\mathcolor{gray}{-}}{295}{E}^{\;\!\textcolor{PineGreen}{\hat{1}} \mathcolor{gray}{\omega}}_{\;\! \hat{1} \mathcolor{gray}{z}} ~\mathcolor{gray}{\widetilde \circledast}~ \xint{\mathcolor{gray}{-}}{295}{E}^{\;\!\textcolor{PineGreen}{\hat{2}} \mathcolor{gray}{\omega}}_{\;\! \hat{2} \mathcolor{gray}{z}} \right) \label{eq:DP^(2)-3_12-spectrum-SFG2} \\
	&\xrightarrow[]{\text{\bref{eq:FT-krho}}} {\chi}^{\;\! \textcolor{PineGreen}{\hat{3}} \mathcolor{gray}{\omega} \hat{1} \hat{2} }_{\;\! \hat{3} \textcolor{Maroon}{(2)} \textcolor{PineGreen}{\hat{1} \hat{2}}} \mathcolor{gray}{\mathcal F} \left[ M^{\;\! \mathcolor{gray}{\omega} \hat{1} \hat{2} }_{\;\! \hat{3} \mathcolor{gray}{z} \textcolor{Maroon}{(2)} } \right] \mathcolor{gray}{*} \left( \xint{\mathcolor{gray}{-}}{295}{E}^{\;\!\textcolor{PineGreen}{\hat{1}} \mathcolor{gray}{\omega}}_{\;\! \hat{1} \mathcolor{gray}{z}} ~\mathcolor{gray}{\widetilde \circledast}~ \xint{\mathcolor{gray}{-}}{295}{E}^{\;\!\textcolor{PineGreen}{\hat{2}} \mathcolor{gray}{\omega}}_{\;\! \hat{2} \mathcolor{gray}{z}} \right) \label{eq:DP^(2)-3_12-spectrum-SFG3} \\
	&\xrightarrow[]{\text{\bref{eq:IFT-z}}} {\chi}^{\;\! \textcolor{PineGreen}{\hat{3}} \mathcolor{gray}{\omega} \hat{1} \hat{2} }_{\;\! \hat{3} \textcolor{Maroon}{(2)} \textcolor{PineGreen}{\hat{1} \hat{2}}} \mathcolor{gray}{\mathcal F_{z}^{-1}} \left[ \mathcolor{gray}{\mathcal F_{\bar{k}}} \left[ M^{\;\! \mathcolor{gray}{\omega} \hat{1} \hat{2} }_{\;\! \hat{3} \mathcolor{gray}{z} \textcolor{Maroon}{(2)} } \right] \right] \mathcolor{gray}{*} \left( \xint{\mathcolor{gray}{-}}{295}{E}^{\;\!\textcolor{PineGreen}{\hat{1}} \mathcolor{gray}{\omega}}_{\;\! \hat{1} \mathcolor{gray}{z}} ~\mathcolor{gray}{\widetilde \circledast}~ \xint{\mathcolor{gray}{-}}{295}{E}^{\;\!\textcolor{PineGreen}{\hat{2}} \mathcolor{gray}{\omega}}_{\;\! \hat{2} \mathcolor{gray}{z}} \right) \label{eq:DP^(2)-3_12-spectrum-SFG4} \\
	&= {\chi}^{\;\! \textcolor{PineGreen}{\hat{3}} \mathcolor{gray}{\omega} \hat{1} \hat{2} }_{\;\! \hat{3} \textcolor{Maroon}{(2)} \textcolor{PineGreen}{\hat{1} \hat{2}}} \mathcolor{gray}{\mathcal F_{z}^{-1}} \left[ \mathcolor{gray}{\mathcal F_{\bar{k}}} \left[ M^{\;\! \mathcolor{gray}{\omega} \hat{1} \hat{2} }_{\;\! \hat{3} \mathcolor{gray}{z} \textcolor{Maroon}{(2)} } \right] \mathcolor{gray}{*} \left( \xint{\mathcolor{gray}{-}}{295}{E}^{\;\!\textcolor{PineGreen}{\hat{1}} \mathcolor{gray}{\omega}}_{\;\! \hat{1} \mathcolor{gray}{z}} ~\mathcolor{gray}{\widetilde \circledast}~ \xint{\mathcolor{gray}{-}}{295}{E}^{\;\!\textcolor{PineGreen}{\hat{2}} \mathcolor{gray}{\omega}}_{\;\! \hat{2} \mathcolor{gray}{z}} \right) \right] \label{eq:DP^(2)-3_12-spectrum-SFG5} \\
	&\xrightarrow[]{\text{\bref{eq:components-C}}}: {\chi}^{\;\! \textcolor{PineGreen}{\hat{3}} \mathcolor{gray}{\omega} \hat{1} \hat{2} }_{\;\! \hat{3} \textcolor{Maroon}{(2)} \textcolor{PineGreen}{\hat{1} \hat{2}}} \mathcolor{gray}{\mathcal F_{z}^{-1}} \left[ \xint{\mathcolor{gray}{-}}{18}{M}^{\;\! \mathcolor{gray}{\omega} \hat{1} \hat{2} }_{\;\! \hat{3} \mathcolor{gray}{k_{\symup{z}}} \textcolor{Maroon}{(2)} } \mathcolor{gray}{*} \left( \xint{\mathcolor{gray}{-}}{295}{E}^{\;\!\textcolor{PineGreen}{\hat{1}} \mathcolor{gray}{\omega}}_{\;\! \hat{1} \mathcolor{gray}{z}} ~\mathcolor{gray}{\widetilde \circledast}~ \xint{\mathcolor{gray}{-}}{295}{E}^{\;\!\textcolor{PineGreen}{\hat{2}} \mathcolor{gray}{\omega}}_{\;\! \hat{2} \mathcolor{gray}{z}} \right) \right] ~, \label{eq:DP^(2)-3_12-spectrum-SFG6}
\end{align}
\end{subequations}
其中,$\xint{{}^{}_{\mathcolor{gray}{-}}}{23}{\chi}^{\;\! \textcolor{PineGreen}{\hat{3}} \mathcolor{gray}{\omega} \hat{1} \hat{2} }_{\;\! \hat{3} \mathcolor{gray}{z} \textcolor{PineGreen}{\hat{1} \hat{2}} \textcolor{Maroon}{(2)}}$ 的值,与\textcolor{PineGreen}{本征模} $\textcolor{PineGreen}{\hat{3}},\textcolor{PineGreen}{\hat{2}},\textcolor{PineGreen}{\hat{1}}$ 无关。$\textcolor{PineGreen}{\hat{3}},\textcolor{PineGreen}{\hat{2}},\textcolor{PineGreen}{\hat{1}}$ 在其中,只是帮助 $\xint{{}^{}_{\mathcolor{gray}{-}}}{23}{\chi}^{\;\! \textcolor{PineGreen}{\hat{3}} \mathcolor{gray}{\omega} \hat{1} \hat{2} }_{\;\! \hat{3} \mathcolor{gray}{z} \textcolor{PineGreen}{\hat{1} \hat{2}} \textcolor{Maroon}{(2)}}$ 与 $\xint{\mathcolor{gray}{-}}{25}{E}^{\;\!\textcolor{PineGreen}{\hat{1}} \mathcolor{gray}{\omega}}_{\;\! \hat{1} \mathcolor{gray}{z}}, \xint{\mathcolor{gray}{-}}{25}{E}^{\;\!\textcolor{PineGreen}{\hat{2}} \mathcolor{gray}{\omega}}_{\;\! \hat{2} \mathcolor{gray}{z}}$ 一起,起到\textcolor{Plum}{爱因斯坦求和}的作用。

此外,\bref{eq:DP^(2)-3_12-spectrum-SFG6} 中,定义了\textcolor{NavyBlue}{调制场}在三维 $\mathcolor{gray}{\bar{k}}$ \textcolor{gray}{空间}的\textcolor{NavyBlue}{倒格波系数}(关于 $\mathcolor{gray}{\bar{k}} \asymp \left( \mathcolor{gray}{\bar{k}_{\symup{\rho}}}, \mathcolor{gray}{k_{\symup{z}}} \right)$ 的三阶\textcolor{Plum}{张量}\textcolor{NavyBlue}{场})
\begin{subequations} \label{eq:C}
\begin{align}
	\xint{\mathcolor{gray}{-}}{18}{M}^{\;\! \mathcolor{gray}{\omega} \hat{1} \hat{2} }_{\;\! \hat{3} \mathcolor{gray}{k_{\symup{z}}} \textcolor{Maroon}{(2)} } &:\xleftarrow[]{\text{\bref{eq:FT-k}}} \mathcolor{gray}{\mathcal F_{\bar{k}}} \left[ M^{\;\! \mathcolor{gray}{\omega} \hat{1} \hat{2} }_{\;\! \hat{3} \mathcolor{gray}{z} \textcolor{Maroon}{(2)} } \right] \label{eq:components-C} \\
	\xint{\mathcolor{gray}{-}}{18}{\bar{\bar{\bar{M}}}}^{\;\! \mathcolor{gray}{\omega} }_{\;\! \mathcolor{gray}{k_{\symup{z}}} \textcolor{Maroon}{(2)} } &:\xleftarrow[]{\text{\bref{eq:FT-k}}} \mathcolor{gray}{\mathcal F_{\bar{k}}} \left[ \bar{\bar{\bar{M}}}^{\;\! \mathcolor{gray}{\omega} }_{\;\! \mathcolor{gray}{z} \textcolor{Maroon}{(2)} } \right] ~, \label{eq:vec-C}
\end{align}
\end{subequations}
其中,3 维空域 $\mathcolor{gray}{\bar{r}} \in \mathcolor{gray}{\bar{\mathbb{R}}_{\textcolor{Plum}{3}}}$ 中的\textcolor{Plum}{傅立叶正变换} $\mathcolor{gray}{\mathcal F_{\bar{k}}}$ 来自 \bref{eq:FT-k}。

\bref{eq:DP^(2)-3_12-spectrum-SFG4} 中,还定义了空域 $\mathcolor{gray}{z} \in \mathcolor{gray}{\bar{\mathbb{R}}_{\textcolor{Plum}{1}}}$ 向 1 维\textcolor{Plum}{傅立叶正} $\mathcolor{gray}{\mathcal F_{z}}$、\textcolor{Plum}{逆} $\mathcolor{gray}{\mathcal F_{z}^{-1}}$ \textcolor{Plum}{变换对}
\begin{subequations} \label{eq:FT-z_kz}
\begin{align}
	\mathcolor{gray}{\mathcal F_{z}} \left[ \cdot \right] &:= \frac{ 1 }{ 2\symup{\pi} } \mathcolor{gray}{\int_{-\infty}^{+\infty}} \cdot~ \mathbb{e}^{-\mathbb{i}\mathcolor{gray}{k_{\symup{z}}} \mathcolor{gray}{z}} \mathbb{d}\mathcolor{gray}{z} ~, \label{eq:FT-kz} \\
	\mathcolor{gray}{\mathcal F_{z}^{-1}} \left[ \cdot \right] &:= \hphantom{\frac{ 1 }{ 2\symup{\pi} }} \mathcolor{gray}{\int_{-\infty}^{+\infty}} \cdot~ \mathbb{e}^{\mathbb{i}\mathcolor{gray}{k_{\symup{z}}} \mathcolor{gray}{z}} \hphantom{^-} \mathbb{d}\mathcolor{gray}{k_{\symup{z}}} ~. \label{eq:IFT-z}
\end{align}
\end{subequations}

利用\textcolor{PineGreen}{本征波}的第 4 种定义 \bref{eq:vec-eigenwave'} 的\textcolor{PineGreen}{本征偏振态} $\xint{{}^{}_{\mathcolor{gray}{-}}}{10}{\bar{g}}^{\;\! \mathcolor{gray}{\omega} \textcolor{PineGreen}{\hat{\jmath}}}$ \textcolor{Plum}{复归一化}版 $\xint{\mathcolor{gray}{-}}{25}{\bar{E}}^{\;\! \mathcolor{gray}{\omega} \textcolor{PineGreen}{\hat{\jmath}}}_{\;\! \mathcolor{gray}{z}} := \xint{\mathcolor{gray}{-}}{16}{\mathtt{G}}^{\;\! \mathcolor{gray}{\omega} \textcolor{PineGreen}{\hat{\jmath}}}_{\;\! \mathcolor{gray}{z}} \xint{{}^{}_{\mathcolor{gray}{-}}}{10}{\hat{g}}^{\;\! \mathcolor{gray}{\omega} \textcolor{PineGreen}{\hat{\jmath}} }$,将 \bref{eq:DP^(2)-3_12-spectrum-SFG6} 分离出\textcolor{NavyBlue}{含衍射}\textcolor{PineGreen}{本征复振幅} $\xint{\mathcolor{gray}{-}}{16}{\mathtt{G}}^{\;\!\mathcolor{gray}{\omega} \textcolor{PineGreen}{\hat{\jmath}}}_{\;\! \mathcolor{gray}{z}}$(\bref{eq:amp_phase})和\textcolor{Plum}{复归一化}\textcolor{PineGreen}{本征偏振态} $\xint{{}^{}_{\mathcolor{gray}{-}}}{10}{\hat{g}}^{\;\!\mathcolor{gray}{\omega}}_{\;\! \textcolor{PineGreen}{\hat{\jmath}}}$,并将 \bref{eq:DP^(2)-3_12-spectrum-SFG6}(的系数张量 $\xint{\mathcolor{gray}{-}}{18}{M}^{\;\! \mathcolor{gray}{\omega} \hat{1} \hat{2} }_{\;\! \hat{3} \mathcolor{gray}{k_{\symup{z}}} \textcolor{Maroon}{(2)} }$)升级为“\textcolor{Plum}{半张量式}”$\xint{\mathcolor{gray}{-}}{18}{\bar{M}}^{\;\! \mathcolor{gray}{\omega} \hat{1} \hat{2} }_{\;\! \mathcolor{gray}{k_{\symup{z}}} \textcolor{Maroon}{(2)} }$\Footnote{注意,{\one} $\xint{{}^{}_{\mathcolor{gray}{-}}}{10}{\hat{g}}^{\;\! \mathcolor{gray}{\omega} \textcolor{PineGreen}{\hat{1}}}_{\;\! \hat{1}}$ 是矢量 $\xint{{}^{}_{\mathcolor{gray}{-}}}{10}{\hat{g}}^{\;\! \mathcolor{gray}{\omega} }_{\;\! \textcolor{PineGreen}{\hat{1}}}$ 在 $\hat{1}$ 方向的分量,即标量;{\two} 对 $\mathcolor{gray}{\widetilde \circledast}$ 的运算\textcolor{Plum}{优先级},整体来说,不高于、低于,或等于对 $\mathcolor{gray}{*}$ 的\textcolor{Plum}{优先级}:需要拆分后,才能谈\textcolor{Plum}{优先级},见下文 \bref{ssec:scalar} 中 \bref{eq:scalar_nonlinear_drive2} 的下一段。}
\begin{subequations} \label{eq:DP^(2)-3_12-spectrum-G}
\begin{align}
	\xint{\mathcolor{gray}{-}}{30}{\bar{P}}^{\;\! \mathcolor{gray}{\omega} \textcolor{PineGreen}{\hat{3}} }_{\;\! \mathcolor{gray}{z} \textcolor{Maroon}{(2)} } &\xrightarrow[]{\text{\bref{eq:DP^(2)-3_12-spectrum-SFG6}}} \bar{\chi}^{\;\! \mathcolor{gray}{\omega} \textcolor{PineGreen}{\hat{3}} \hat{1} \hat{2} }_{\;\! \textcolor{Maroon}{(2)} \textcolor{PineGreen}{\hat{1} \hat{2}}} \odot \mathcolor{gray}{\mathcal F_{z}^{-1}} \left[ \xint{\mathcolor{gray}{-}}{18}{\bar{M}}^{\;\! \mathcolor{gray}{\omega} \hat{1} \hat{2} }_{\;\! \mathcolor{gray}{k_{\symup{z}}} \textcolor{Maroon}{(2)} } \mathcolor{gray}{*} \xint{\mathcolor{gray}{-}}{295}{E}^{\;\! \mathcolor{gray}{\omega} \textcolor{PineGreen}{\hat{1}}}_{\;\! \hat{1} \mathcolor{gray}{z}} ~\mathcolor{gray}{\widetilde \circledast}~ \xint{\mathcolor{gray}{-}}{295}{E}^{\;\! \mathcolor{gray}{\omega} \textcolor{PineGreen}{\hat{2}}}_{\;\! \hat{2} \mathcolor{gray}{z}} \right] \label{eq:DP^(2)-3_12-spectrum-G1} \\
	&\xrightarrow[]{\text{$\sim$ \bref{eq:components-eigenwave'}}} \bar{\chi}^{\;\! \mathcolor{gray}{\omega} \textcolor{PineGreen}{\hat{3}} \hat{1} \hat{2} }_{\;\! \textcolor{Maroon}{(2)} \textcolor{PineGreen}{\hat{1} \hat{2}}} \odot \mathcolor{gray}{\mathcal F_{z}^{-1}} \left[ \xint{\mathcolor{gray}{-}}{18}{\bar{M}}^{\;\! \mathcolor{gray}{\omega} \hat{1} \hat{2} }_{\;\! \mathcolor{gray}{k_{\symup{z}}} \textcolor{Maroon}{(2)} } \mathcolor{gray}{*} \left( \xint{\mathcolor{gray}{-}}{20}{\mathtt{G}}^{\;\! \mathcolor{gray}{\omega} \textcolor{PineGreen}{\hat{1}}}_{\;\! \mathcolor{gray}{z}} \xint{{}^{}_{\mathcolor{gray}{-}}}{10}{\hat{g}}^{\;\! \mathcolor{gray}{\omega} \textcolor{PineGreen}{\hat{1}}}_{\;\! \hat{1}} \right) \mathcolor{gray}{\widetilde \circledast} \left( \xint{\mathcolor{gray}{-}}{20}{\mathtt{G}}^{\;\! \mathcolor{gray}{\omega} \textcolor{PineGreen}{\hat{2}}}_{\;\! \mathcolor{gray}{z}} \xint{{}^{}_{\mathcolor{gray}{-}}}{10}{\hat{g}}^{\;\! \mathcolor{gray}{\omega} \textcolor{PineGreen}{\hat{2}}}_{\;\! \hat{2}} \right) \right] ~, \label{eq:DP^(2)-3_12-spectrum-G2}
\end{align}
\end{subequations}
该版本的\textcolor{Plum}{非线性}\textcolor{NavyBlue}{波源},保留了 \bref{eq:vec-DP^(2)-3_12-spectrum} 左侧的 $\xint{\mathcolor{gray}{-}}{25}{\bar{P}}^{\;\! \mathcolor{gray}{\omega} \textcolor{PineGreen}{\hat{3}}}_{\;\! \textcolor{Maroon}{\Yup} \mathcolor{gray}{z}}$ 的矢量形式,以便直接代入 \bref{eq:simplify8-scalar-g-modulus};又丢弃了 \bref{eq:vec-DP^(2)-3_12-spectrum} 右侧的 ${}^{\mathcolor{gray}{*}}_{\mathcolor{gray}{*}}$,并采用了 \bref{eq:components-DP^(2)-3_12-spectrum} 右侧的 $\mathcolor{gray}{*}$;--- 这样做的代价便是 \bref{eq:DP^(2)-3_12-spectrum-G} 中的二阶\textcolor{Plum}{非线性}系数张量,既从完整的三阶张量 $\xint{{}^{}_{\mathcolor{gray}{-}}}{23}{\bar{\bar{\bar{\chi}}}}^{\;\! \mathcolor{gray}{\omega} }_{\;\! \mathcolor{gray}{z} \textcolor{Maroon}{(2)} }$ 降阶(\bref{eq:vec-chi2-modulate}),又从零阶张量元 $\xint{{}^{}_{\mathcolor{gray}{-}}}{23}{\chi}^{\;\! \mathcolor{gray}{\omega} \hat{1} \hat{2} }_{\;\! \hat{3} \mathcolor{gray}{z} \textcolor{Maroon}{(2)} }$ 升阶(\bref{eq:components-chi2-modulate}),至同阶于 \bref{eq:DP^(2)-3_12-spectrum-G} 左侧 $\xint{\mathcolor{gray}{-}}{25}{\bar{P}}^{\;\! \mathcolor{gray}{\omega} \textcolor{PineGreen}{\pm}}_{\;\! \textcolor{Maroon}{\Yup} \mathcolor{gray}{z}}$ 的一阶张量 $\xint{{}^{}_{\mathcolor{gray}{-}}}{23}{\bar{\chi}}^{\;\! \mathcolor{gray}{\omega} \hat{1} \hat{2} }_{\;\! \mathcolor{gray}{z} \textcolor{Maroon}{(2)} }$ 的矢量形式。--- 这便体现了 \bref{hook:0bar,hook:1bar,hook:2bar,hook:3bar} 中对“划上线 line up”制度的前置顶层设计,以表达不同阶张量的优势。算符的运算顺序见 \byperref{OperatorSequence}{后处}。

将\textcolor{Plum}{替换后的}矢量\textcolor{Plum}{非线性}\textcolor{NavyBlue}{波源}项 $\xint{\mathcolor{gray}{-}}{25}{\bar{P}}^{\;\! \mathcolor{gray}{\omega} \textcolor{PineGreen}{\hat{3}} }_{\;\! \mathcolor{gray}{z} \textcolor{Maroon}{(2)} }$ 整体,即 \bref{eq:DP^(2)-3_12-spectrum-G1},代入 \bref{eq:simplify8-scalar-g-modulus},即得以\textcolor{NavyBlue}{脉冲光}\textcolor{Maroon}{倍频}\cite{boydNonlinearOptics2019}、\textcolor{NavyBlue}{脉冲}\textcolor{Maroon}{光整流}后的级联\textcolor{Maroon}{电光效应}\cite{jangMulticycleTerahertzPulse2020}等过程为代表的电场\textcolor{PineGreen}{本征复振幅}方程
\begin{align} \label{eq:simplify8-scalar-g-modulus-P-spectrum}
	\mathcolor{gray}{\nabla_z} \xint{\begin{smallmatrix} ~ \\ {}^{}_{\mathcolor{gray}{-}} \\ ~ \end{smallmatrix}}{09}{\mathtt{g}}^{\;\!\mathcolor{gray}{\omega} \textcolor{PineGreen}{\hat{3}}}_{\;\! \mathcolor{gray}{z}} &\xrightarrow[\text{\bref{eq:simplify8-scalar-g-modulus}}]{\text{\bref{eq:DP^(2)-3_12-spectrum-G1}}} \mathbb{i} k_{\textcolor{Maroon}{\mathsf{o}} \mathcolor{gray}{\omega}}^{\;\! 2} \frac{\xint{{}^{}_{\mathcolor{gray}{-}}}{10}{\hat{g}}^{\;\! \textcolor{PineGreen}{\hat{3}} \textcolor{Plum}{\dag}}_{\;\! \mathcolor{gray}{\omega}} \cdot \bar{\chi}^{\;\! \mathcolor{gray}{\omega} \textcolor{PineGreen}{\hat{3}} \hat{1} \hat{2} }_{\;\! \textcolor{Maroon}{(2)} \textcolor{PineGreen}{\hat{1} \hat{2}}} \odot \mathcolor{gray}{\mathcal F_{z}^{-1}} \left[ \xint{\mathcolor{gray}{-}}{18}{\bar{M}}^{\;\! \mathcolor{gray}{\omega} \hat{1} \hat{2} }_{\;\! \mathcolor{gray}{k_{\symup{z}}} \textcolor{Maroon}{(2)} } \mathcolor{gray}{*} \xint{\mathcolor{gray}{-}}{25}{E}^{\;\! \mathcolor{gray}{\omega} \textcolor{PineGreen}{\hat{1}}}_{\;\! \hat{1} \mathcolor{gray}{z}} ~\mathcolor{gray}{\widetilde \circledast}~ \xint{\mathcolor{gray}{-}}{25}{E}^{\;\! \mathcolor{gray}{\omega} \textcolor{PineGreen}{\hat{2}}}_{\;\! \hat{2} \mathcolor{gray}{z}} \right]}{ 2 \lvert \xint{{}^{}_{\mathcolor{gray}{-}}}{10}{\hat{g}}^{\;\! \textcolor{PineGreen}{\hat{3}}}_{\;\! \mathcolor{gray}{\omega}} \rvert^2 \xint{\begin{smallmatrix} ~ \\ {}^{}_{\mathcolor{gray}{-}} \\ ~ \end{smallmatrix}}{15}{k}_{\;\! \symup{z}}^{\;\! \mathcolor{gray}{\omega} \textcolor{PineGreen}{\hat{3}}} \mathbb{e}^{\mathbb{i} \xint{\begin{smallmatrix} ~ \\ {}^{}_{\mathcolor{gray}{-}} \\ ~ \end{smallmatrix}}{15}{k}_{\symup{z}}^{\;\! \mathcolor{gray}{\omega} \textcolor{PineGreen}{\hat{3}}} \mathcolor{gray}{z}}} ~, 
\end{align}
注意,\textcolor{Plum}{哈达马积} $\odot$ 的运算\textcolor{Plum}{优先级}恒高于\textcolor{Plum}{点积} $\cdot$(以省略一对\textcolor{Plum}{小括号})。

\marginLeft[-2.4em]{ssec:SFG_discrete}\subsection{连续光和频 - 电场本征复振幅方程}\label{ssec:SFG_discrete}

对于\textcolor{NavyBlue}{非脉冲}/\textcolor{NavyBlue}{非连续谱},而是两个\textcolor{Plum}{独立}、\textcolor{Plum}{离散}、\textcolor{NavyBlue}{单色}\textcolor{gray}{波长}的\textcolor{Maroon}{和频}或\textcolor{Maroon}{上转换}\Footnote{尽管\textcolor{NavyBlue}{双泵浦}的\textcolor{NavyBlue}{光强}可能不大,这里仍不说“\textcolor{Maroon}{上转换}”:因为在本文的语境中,“\textcolor{Maroon}{上转换}”过程一般是“\textcolor{NavyBlue}{一强一弱}”\textcolor{NavyBlue}{双泵浦}生成\textcolor{NavyBlue}{弱} $\mathcolor{gray}{\omega}_{\textcolor{gray}{3}}$,以至于参与\textcolor{gray}{混频}的三波中,有两束\textcolor{NavyBlue}{弱光}(一入一出)不满足\textcolor{Maroon}{泵浦未耗尽近似}条件,因此只要有“\textcolor{Maroon}{上转换}”则必有“\textcolor{Maroon}{下转换}”过程发生(\textcolor{NavyBlue}{能量}从 $\mathcolor{gray}{\omega}_{\textcolor{gray}{3}}$ \textcolor{NavyBlue}{回流}到其中一个\textcolor{NavyBlue}{弱泵浦}中),于是不可避免地涉及\textcolor{Maroon}{三波混频}\textcolor{Maroon}{时空谱}耦合波方程组中的至少 2 个方程,然而这里只给出了 1 个“\textcolor{Maroon}{上转换}”过程的方程,因此这里只能代表/指\textcolor{Maroon}{和频}过程。}出\textcolor{gray}{第三个波长}的 $\mathcolor{gray}{\omega}_{\textcolor{gray}{1}} + \mathcolor{gray}{\omega}_{\textcolor{gray}{2}} \to \mathcolor{gray}{\omega}_{\textcolor{gray}{3}}$ 过程,即纯\textcolor{NavyBlue}{(准)连续光}\textcolor{gray}{混频}的特例,\bref{eq:DP^(2)-3_12-spectrum} 变为
\begin{subequations} \label{eq:DP^(2)-3_12-discrete}
	\begin{align}
		\xint{\mathcolor{gray}{-}}{30}{\bar{P}}^{\;\! \textcolor{Maroon}{(2)} }_{\;\! \mathcolor{gray}{z} \textcolor{PineGreen}{\hat{3}}} &= \xint{{}^{}_{\mathcolor{gray}{-}}}{23}{\bar{\bar{\bar{\chi}}}}^{\;\!  \textcolor{Maroon}{(2)}}_{\mathcolor{gray}{z} \textcolor{PineGreen}{\hat{3} \hat{1} \hat{2}} } ~{}^{\mathcolor{gray}{*}}_{\mathcolor{gray}{*}} \left( \xint{\mathcolor{gray}{-}}{295}{\bar{E}}^{\;\! \textcolor{PineGreen}{\hat{1}} }_{\;\! \mathcolor{gray}{z} } \mathcolor{gray}{*} \xint{\mathcolor{gray}{-}}{295}{\bar{E}}^{\;\! \textcolor{PineGreen}{\hat{2}} }_{\;\! \mathcolor{gray}{z} } \right) \label{eq:vec-DP^(2)-3_12-discrete} \\
		\xint{\mathcolor{gray}{-}}{30}{P}^{\;\! \textcolor{PineGreen}{\hat{3}} \textcolor{Maroon}{(2)} }_{\;\! \hat{3}\mathcolor{gray}{z}} &= \xint{{}^{}_{\mathcolor{gray}{-}}}{23}{\chi}^{\;\! \textcolor{PineGreen}{\hat{3}} \textcolor{Maroon}{(2)} \hat{1} \hat{2} }_{\;\! \hat{3} \mathcolor{gray}{z} \textcolor{PineGreen}{\hat{1} \hat{2}}} \mathcolor{gray}{*} \left( \xint{\mathcolor{gray}{-}}{295}{E}^{\;\!\textcolor{PineGreen}{\hat{1}}}_{\;\! \hat{1} \mathcolor{gray}{z}} \mathcolor{gray}{*} \xint{\mathcolor{gray}{-}}{295}{E}^{\;\!\textcolor{PineGreen}{\hat{2}}}_{\;\! \hat{2} \mathcolor{gray}{z}} \right) ~. \label{eq:components-DP^(2)-3_12-discrete}
	\end{align}
\end{subequations}
其中,每个\textcolor{NavyBlue}{场量}都\textcolor{Plum}{未显含}\textcolor{gray}{角频率} $\mathcolor{gray}{\omega}$,但可以\textcolor{Plum}{推断}出来它们运行在 $\mathcolor{gray}{\omega}$ 域:因为\textcolor{PineGreen}{模式} $\textcolor{PineGreen}{\hat{3}},\textcolor{PineGreen}{\hat{2}},\textcolor{PineGreen}{\hat{1}}$ 只存在于 $\mathcolor{gray}{\omega}~ (, \mathcolor{gray}{\bar{k}_{\symup{\rho}}})$ 域,在时间 $\mathcolor{gray}{t}~ (, \mathcolor{gray}{\bar{k}_{\symup{\rho}}})$ 域内没有“\textcolor{PineGreen}{模式}”这一说法。这样表示,是在最大程度\textcolor{Plum}{省略符号}的同时\textcolor{Plum}{保留全信息}。

将 \bref{eq:components-chi2-modulate} 的\textcolor{NavyBlue}{(准)连续光}/\textcolor{NavyBlue}{离散谱}版本,代入\textcolor{Plum}{非线性}\textcolor{NavyBlue}{波源} \bref{eq:components-DP^(2)-3_12-discrete} 得
\begin{subequations} \label{eq:DP^(2)-3_12-discrete-SFG}
\begin{align}
	\xint{\mathcolor{gray}{-}}{30}{P}^{\;\! \textcolor{PineGreen}{\hat{3}} \textcolor{Maroon}{(2)} }_{\;\! \hat{3}\mathcolor{gray}{z}} &\xrightarrow[]{\text{\bref{eq:components-DP^(2)-3_12-discrete}}} \xint{{}^{}_{\mathcolor{gray}{-}}}{23}{\chi}^{\;\! \textcolor{PineGreen}{\hat{3}} \textcolor{Maroon}{(2)} \hat{1} \hat{2} }_{\;\! \hat{3} \mathcolor{gray}{z} \textcolor{PineGreen}{\hat{1} \hat{2}}} \mathcolor{gray}{*} \left( \xint{\mathcolor{gray}{-}}{295}{E}^{\;\! \textcolor{PineGreen}{\hat{1}}}_{\;\! \hat{1} \mathcolor{gray}{z}} \mathcolor{gray}{*} \xint{\mathcolor{gray}{-}}{295}{E}^{\;\! \textcolor{PineGreen}{\hat{2}}}_{\;\! \hat{2} \mathcolor{gray}{z}} \right) \label{eq:DP^(2)-3_12-discrete-SFG1} \\
	&\xrightarrow[]{\text{$\sim$\bref{eq:components-chi2-modulate}}} {\chi}^{\;\! \textcolor{PineGreen}{\hat{3}} \hat{1} \hat{2} }_{\;\! \hat{3} \textcolor{PineGreen}{\hat{1} \hat{2}} \textcolor{Maroon}{(2)}} \xint{\mathcolor{gray}{-}}{18}{M}^{\;\! \mathcolor{gray}{3} \hat{1} \hat{2} }_{\;\! \hat{3} \mathcolor{gray}{z} \textcolor{Maroon}{(2)} } \mathcolor{gray}{*} \left( \xint{\mathcolor{gray}{-}}{295}{E}^{\;\! \textcolor{PineGreen}{\hat{1}}}_{\;\! \hat{1} \mathcolor{gray}{z}} \mathcolor{gray}{*} \xint{\mathcolor{gray}{-}}{295}{E}^{\;\! \textcolor{PineGreen}{\hat{2}}}_{\;\! \hat{2} \mathcolor{gray}{z}} \right) \label{eq:DP^(2)-3_12-discrete-SFG2} \\
	&\xrightarrow[]{\text{\bref{eq:FT-krho}}} {\chi}^{\;\! \textcolor{PineGreen}{\hat{3}} \hat{1} \hat{2} }_{\;\! \hat{3} \textcolor{PineGreen}{\hat{1} \hat{2}} \textcolor{Maroon}{(2)}} \mathcolor{gray}{\mathcal F} \left[ M^{\;\! \mathcolor{gray}{3} \hat{1} \hat{2} }_{\;\! \hat{3} \mathcolor{gray}{z} \textcolor{Maroon}{(2)} } \right] \mathcolor{gray}{*} \left( \xint{\mathcolor{gray}{-}}{295}{E}^{\;\! \textcolor{PineGreen}{\hat{1}}}_{\;\! \hat{1} \mathcolor{gray}{z}} \mathcolor{gray}{*} \xint{\mathcolor{gray}{-}}{295}{E}^{\;\! \textcolor{PineGreen}{\hat{2}}}_{\;\! \hat{2} \mathcolor{gray}{z}} \right) \label{eq:DP^(2)-3_12-discrete-SFG3} \\
	&\xrightarrow[]{\text{\bref{eq:IFT-z}}} {\chi}^{\;\! \textcolor{PineGreen}{\hat{3}} \hat{1} \hat{2} }_{\;\! \hat{3} \textcolor{PineGreen}{\hat{1} \hat{2}} \textcolor{Maroon}{(2)}} \mathcolor{gray}{\mathcal F_{z}^{-1}} \left[ \mathcolor{gray}{\mathcal F_{\bar{k}}} \left[ M^{\;\! \mathcolor{gray}{3} \hat{1} \hat{2} }_{\;\! \hat{3} \mathcolor{gray}{z} \textcolor{Maroon}{(2)} } \right] \right] \mathcolor{gray}{*} \left( \xint{\mathcolor{gray}{-}}{295}{E}^{\;\! \textcolor{PineGreen}{\hat{1}}}_{\;\! \hat{1} \mathcolor{gray}{z}} \mathcolor{gray}{*} \xint{\mathcolor{gray}{-}}{295}{E}^{\;\! \textcolor{PineGreen}{\hat{2}}}_{\;\! \hat{2} \mathcolor{gray}{z}} \right) \label{eq:DP^(2)-3_12-discrete-SFG4} \\
	&= {\chi}^{\;\! \textcolor{PineGreen}{\hat{3}} \hat{1} \hat{2} }_{\;\! \hat{3} \textcolor{PineGreen}{\hat{1} \hat{2}} \textcolor{Maroon}{(2)}} \mathcolor{gray}{\mathcal F_{z}^{-1}} \left[ \mathcolor{gray}{\mathcal F_{\bar{k}}} \left[ M^{\;\! \mathcolor{gray}{3} \hat{1} \hat{2} }_{\;\! \hat{3} \mathcolor{gray}{z} \textcolor{Maroon}{(2)} } \right] \mathcolor{gray}{*} \left( \xint{\mathcolor{gray}{-}}{295}{E}^{\;\! \textcolor{PineGreen}{\hat{1}}}_{\;\! \hat{1} \mathcolor{gray}{z}} \mathcolor{gray}{*} \xint{\mathcolor{gray}{-}}{295}{E}^{\;\! \textcolor{PineGreen}{\hat{2}}}_{\;\! \hat{2} \mathcolor{gray}{z}} \right) \right] \label{eq:DP^(2)-3_12-discrete-SFG5} \\
	&\xrightarrow[]{\text{$\sim$\bref{eq:components-C}}} {\chi}^{\;\! \textcolor{PineGreen}{\hat{3}} \hat{1} \hat{2} }_{\;\! \hat{3} \textcolor{PineGreen}{\hat{1} \hat{2}} \textcolor{Maroon}{(2)}} \mathcolor{gray}{\mathcal F_{z}^{-1}} \left[ \xint{\mathcolor{gray}{-}}{18}{M}^{\;\! \mathcolor{gray}{3} \hat{1} \hat{2} }_{\;\! \hat{3} \mathcolor{gray}{k_{\symup{z}}} \textcolor{Maroon}{(2)} } \mathcolor{gray}{*} \left( \xint{\mathcolor{gray}{-}}{295}{E}^{\;\! \textcolor{PineGreen}{\hat{1}}}_{\;\! \hat{1} \mathcolor{gray}{z}} \mathcolor{gray}{*} \xint{\mathcolor{gray}{-}}{295}{E}^{\;\! \textcolor{PineGreen}{\hat{2}}}_{\;\! \hat{2} \mathcolor{gray}{z}} \right) \right] ~, \label{eq:DP^(2)-3_12-discrete-SFG6}
\end{align}
\end{subequations}
其中,$\xint{\mathcolor{gray}{-}}{18}{M}^{\;\! \mathcolor{gray}{3} \hat{1} \hat{2} }_{\;\! \hat{3} \mathcolor{gray}{z} \textcolor{Maroon}{(2)} }$ 中的 \textcolor{gray}{灰色数字 3} 表示 $\mathcolor{gray}{\omega}_{\textcolor{gray}{3}}$。

将 \bref{eq:DP^(2)-3_12-discrete-SFG1} 中,由\textcolor{Plum}{分量形式}的 $\xint{\mathcolor{gray}{-}}{18}{M}^{\;\! \mathcolor{gray}{3} \hat{1} \hat{2} }_{\;\! \hat{3} \mathcolor{gray}{k_{\symup{z}}} \textcolor{Maroon}{(2)} }$ 表示的\textcolor{Plum}{标量形式}的\textcolor{Plum}{非线性}\textcolor{NavyBlue}{波源}项 $\xint{\mathcolor{gray}{-}}{25}{P}^{\;\! \textcolor{PineGreen}{\hat{3}} \textcolor{Maroon}{(2)} }_{\;\! \hat{3}\mathcolor{gray}{z}}$,升级为由\textcolor{Plum}{半张量形式}的 $\xint{\mathcolor{gray}{-}}{18}{\bar{M}}^{\;\! \mathcolor{gray}{3} \hat{1} \hat{2} }_{\;\! \mathcolor{gray}{k_{\symup{z}}} \textcolor{Maroon}{(2)} }$ 表示的\textcolor{Plum}{矢量形式} $\xint{\mathcolor{gray}{-}}{25}{\bar{P}}^{\;\! \textcolor{PineGreen}{\hat{3}} }_{\;\! \mathcolor{gray}{z} \textcolor{Maroon}{(2)} }$,即
\begin{subequations} \label{eq:DP^(2)-3_12-discrete-G}
\begin{align}
	\xint{\mathcolor{gray}{-}}{30}{\bar{P}}^{\;\! \textcolor{PineGreen}{\hat{3}} }_{\;\! \mathcolor{gray}{z}  \textcolor{Maroon}{(2)}} &\xrightarrow[]{\text{\bref{eq:DP^(2)-3_12-discrete-SFG6}}} \bar{\chi}^{\;\! \textcolor{PineGreen}{\hat{3}} \hat{1} \hat{2} }_{\;\! \textcolor{Maroon}{(2)} \textcolor{PineGreen}{\hat{1} \hat{2}}} \odot \mathcolor{gray}{\mathcal F_{z}^{-1}} \left[ \xint{\mathcolor{gray}{-}}{18}{\bar{M}}^{\;\! \mathcolor{gray}{3} \hat{1} \hat{2} }_{\;\! \mathcolor{gray}{k_{\symup{z}}} \textcolor{Maroon}{(2)} } \mathcolor{gray}{*} \xint{\mathcolor{gray}{-}}{295}{E}^{\;\! \textcolor{PineGreen}{\hat{1}}}_{\;\! \hat{1} \mathcolor{gray}{z}} \mathcolor{gray}{*} \xint{\mathcolor{gray}{-}}{295}{E}^{\;\! \textcolor{PineGreen}{\hat{2}}}_{\;\! \hat{2} \mathcolor{gray}{z}} \right] \label{eq:DP^(2)-3_12-discrete-G1} \\
	&\xrightarrow[]{\text{$\sim$ \bref{eq:components-eigenwave'}}} \bar{\chi}^{\;\! \textcolor{PineGreen}{\hat{3}} \hat{1} \hat{2} }_{\;\! \textcolor{Maroon}{(2)} \textcolor{PineGreen}{\hat{1} \hat{2}}} \odot \mathcolor{gray}{\mathcal F_{z}^{-1}} \left[ \xint{\mathcolor{gray}{-}}{18}{\bar{M}}^{\;\! \mathcolor{gray}{3} \hat{1} \hat{2} }_{\;\! \mathcolor{gray}{k_{\symup{z}}} \textcolor{Maroon}{(2)} } \mathcolor{gray}{*} \left( \xint{\mathcolor{gray}{-}}{20}{\mathtt{G}}^{\;\! \textcolor{PineGreen}{\hat{1}}}_{\;\! \mathcolor{gray}{z}} \xint{{}^{}_{\mathcolor{gray}{-}}}{10}{\hat{g}}^{\;\! \textcolor{PineGreen}{\hat{1}}}_{\;\! \hat{1}} \right) \mathcolor{gray}{*} \left( \xint{\mathcolor{gray}{-}}{20}{\mathtt{G}}^{\;\! \textcolor{PineGreen}{\hat{2}}}_{\;\! \mathcolor{gray}{z}} \xint{{}^{}_{\mathcolor{gray}{-}}}{10}{\hat{g}}^{\;\! \textcolor{PineGreen}{\hat{2}}}_{\;\! \hat{2}} \right) \right] ~, \label{eq:DP^(2)-3_12-discrete-G2}
\end{align}
\end{subequations}
对应地,\bref{eq:simplify8-scalar-g-modulus-P-spectrum} 降为\textcolor{Plum}{离散}个\textcolor{gray}{波长}的\textcolor{NavyBlue}{(准)连续光}\textcolor{Maroon}{和频}或\textcolor{Maroon}{上转换}的\textcolor{NavyBlue}{非超快}版本
\begin{align} \label{eq:simplify8-scalar-g-modulus-P-discrete}
	\mathcolor{gray}{\nabla_z} \xint{\begin{smallmatrix} ~ \\ {}^{}_{\mathcolor{gray}{-}} \\ ~ \end{smallmatrix}}{09}{\mathtt{g}}^{\;\! \textcolor{PineGreen}{\hat{3}}}_{\;\! \mathcolor{gray}{z}} &\xrightarrow[\text{$\sim$\bref{eq:simplify8-scalar-g-modulus}}]{\text{\bref{eq:DP^(2)-3_12-discrete-G1}}} \mathbb{i} k_{\textcolor{Maroon}{\mathsf{o}} \mathcolor{gray}{3}}^{\;\! 2} \frac{\xint{{}^{}_{\mathcolor{gray}{-}}}{10}{\hat{g}}^{\;\! \textcolor{PineGreen}{\hat{3}} \textcolor{Plum}{\dag}}_{\;\! } \cdot \bar{\chi}^{\;\! \textcolor{PineGreen}{\hat{3}} \hat{1} \hat{2} }_{\;\! \textcolor{Maroon}{(2)} \textcolor{PineGreen}{\hat{1} \hat{2}}} \odot \mathcolor{gray}{\mathcal F_{z}^{-1}} \left[ \xint{\mathcolor{gray}{-}}{18}{\bar{M}}^{\;\! \mathcolor{gray}{3} \hat{1} \hat{2} }_{\;\! \mathcolor{gray}{k_{\symup{z}}} \textcolor{Maroon}{(2)} } \mathcolor{gray}{*} \xint{\mathcolor{gray}{-}}{25}{E}^{\;\! \textcolor{PineGreen}{\hat{1}}}_{\;\! \hat{1} \mathcolor{gray}{z}} \mathcolor{gray}{*} \xint{\mathcolor{gray}{-}}{25}{E}^{\;\! \textcolor{PineGreen}{\hat{2}}}_{\;\! \hat{2} \mathcolor{gray}{z}} \right]}{ 2 \lvert \xint{{}^{}_{\mathcolor{gray}{-}}}{10}{\hat{g}}^{\;\! \textcolor{PineGreen}{\hat{3}}} \rvert^2 \xint{\begin{smallmatrix} ~ \\ {}^{}_{\mathcolor{gray}{-}} \\ ~ \end{smallmatrix}}{15}{k}_{\;\! \symup{z}}^{\;\!  \textcolor{PineGreen}{\hat{3}}} \mathbb{e}^{\mathbb{i} \xint{\begin{smallmatrix} ~ \\ {}^{}_{\mathcolor{gray}{-}} \\ ~ \end{smallmatrix}}{15}{k}_{\symup{z}}^{\;\!  \textcolor{PineGreen}{\hat{3}}} \mathcolor{gray}{z}}} ~, 
\end{align}
同样注意,\textcolor{Plum}{哈达马积} $\odot$ 的运算\textcolor{Plum}{优先级}恒高于\textcolor{Plum}{点积} $\cdot$(以省略一对\textcolor{Plum}{小括号})。

\vspace*{-2.7em}

\marginLeft[-2.4em]{ssec:scalar}\subsection{标量非线性波源、标量调制场条件}\label{ssec:scalar}

如果\textcolor{Plum}{非线性}\textcolor{NavyBlue}{驱动源}中参与相互作用的\textbf{每一个}\textcolor{NavyBlue}{泵浦} $\xint{\mathcolor{gray}{-}}{25}{E}^{\;\! \mathcolor{gray}{\omega} \textcolor{PineGreen}{\hat{\jmath}}}_{\;\! \hat{\jmath} \mathcolor{gray}{z}}$ 的\textcolor{PineGreen}{本征偏振态} $\xint{{}^{}_{\mathcolor{gray}{-}}}{10}{\hat{g}}^{\;\! \mathcolor{gray}{\omega} \textcolor{PineGreen}{\hat{\jmath}}}_{\;\! \hat{\jmath}}$ 固定为\textcolor{Maroon}{倒空间}中的\textcolor{Plum}{定常}矢量 ${\hat{g}}^{\;\! \mathcolor{gray}{\omega} \textcolor{PineGreen}{\hat{\jmath}}}_{\;\! \hat{\jmath}}$,不是\textcolor{gray}{横向空间频率} $\mathcolor{gray}{\bar{k}_{\symup{\rho}}}$ 的函数,不随\textcolor{PineGreen}{波矢}方向变化而改变,则在该
\begin{subequations} \label{eq:scalar_nonlinear_drive}
\begin{align}
	&\text{\textbf{标量\textcolor{Plum}{非线性}\textcolor{NavyBlue}{波源}}条件(\textcolor{NavyBlue}{脉冲}):} \hspace{0.2em} \xint{{}^{}_{\mathcolor{gray}{-}}}{10}{\hat{g}}^{\;\! \mathcolor{gray}{\omega} \textcolor{PineGreen}{\hat{\jmath}}}_{\;\! \hat{\jmath}} \hspace{-7.5em}&&\equiv~ {\hat{g}}^{\;\! \mathcolor{gray}{\omega} \textcolor{PineGreen}{\hat{\jmath}}}_{\;\! \hat{\jmath}} ~, \label{eq:scalar_nonlinear_drive-spectrum} \\
	&\text{\textbf{标量\textcolor{Plum}{非线性}\textcolor{NavyBlue}{波源}}条件(\textcolor{NavyBlue}{连续}):} \hspace{0.7em} \xint{{}^{}_{\mathcolor{gray}{-}}}{10}{\hat{g}}^{\;\! \textcolor{PineGreen}{\hat{\jmath}}}_{\;\! \hat{\jmath}} \hspace{-7.5em}&&\equiv~ {\hat{g}}^{\;\! \textcolor{PineGreen}{\hat{\jmath}}}_{\;\! \hat{\jmath}} ~, \label{eq:scalar_nonlinear_drive-discrete}
\end{align}
\end{subequations}
下,\textcolor{NavyBlue}{脉冲光}\textcolor{Maroon}{倍频}、\textcolor{NavyBlue}{连续光}\textcolor{Maroon}{和频}过程,分别所对应的\bref{eq:DP^(2)-3_12-spectrum-G2,eq:DP^(2)-3_12-discrete-G2} \textcolor{Plum}{非线性}\textcolor{NavyBlue}{波源},可进一步\textcolor{Plum}{退化}为\Footnote{注,其中 $\odot$ 运算/相互作用,只作用于数据结构/对象 $\bar{\chi}^{\;\! \mathcolor{gray}{\omega} \textcolor{PineGreen}{\hat{3}} \hat{1} \hat{2} }_{\;\! \textcolor{Maroon}{(2)} \textcolor{PineGreen}{\hat{1} \hat{2}}}$ 与 $\xint{\mathcolor{gray}{-}}{18}{\bar{M}}^{\;\! \mathcolor{gray}{\omega} \hat{1} \hat{2} }_{\;\! \mathcolor{gray}{k_{\symup{z}}} \textcolor{Maroon}{(2)} }$ 两者,与标量(场)${\hat{g}}^{\;\! \mathcolor{gray}{\omega} \textcolor{PineGreen}{\hat{1}}}_{\;\! \hat{1}}, {\hat{g}}^{\;\! \mathcolor{gray}{\omega} \textcolor{PineGreen}{\hat{2}}}_{\;\! \hat{2}}$ 无关。}
\begin{subequations} \label{eq:DP^(2)-3_12-chieff-G}
\begin{align}
	\xint{\mathcolor{gray}{-}}{30}{\bar{P}}^{\;\! \mathcolor{gray}{\omega} \textcolor{PineGreen}{\hat{3}} }_{\;\! \mathcolor{gray}{z} \textcolor{Maroon}{(2)} } &\xrightarrow[\text{\bref{eq:DP^(2)-3_12-spectrum-G2}}]{\text{\bref{eq:scalar_nonlinear_drive-spectrum}}} \bar{\chi}^{\;\! \mathcolor{gray}{\omega} \textcolor{PineGreen}{\hat{3}} \hat{1} \hat{2} }_{\;\! \textcolor{Maroon}{(2)} \textcolor{PineGreen}{\hat{1} \hat{2}}} ~ {\hat{g}}^{\;\! \mathcolor{gray}{\omega} \textcolor{PineGreen}{\hat{1}}}_{\;\! \hat{1}} ~\mathcolor{gray}{\widetilde *}~ {\hat{g}}^{\;\! \mathcolor{gray}{\omega} \textcolor{PineGreen}{\hat{2}}}_{\;\! \hat{2}} \odot \mathcolor{gray}{\mathcal F_{z}^{-1}} \left[ \xint{\mathcolor{gray}{-}}{18}{\bar{M}}^{\;\! \mathcolor{gray}{\omega} \hat{1} \hat{2} }_{\;\! \mathcolor{gray}{k_{\symup{z}}} \textcolor{Maroon}{(2)} } \mathcolor{gray}{*} \xint{\mathcolor{gray}{-}}{20}{\mathtt{G}}^{\;\! \mathcolor{gray}{\omega} \textcolor{PineGreen}{\hat{1}}}_{\;\! \mathcolor{gray}{z}} ~\mathcolor{gray}{\widetilde \circledast}~ \xint{\mathcolor{gray}{-}}{20}{\mathtt{G}}^{\;\! \mathcolor{gray}{\omega} \textcolor{PineGreen}{\hat{2}}}_{\;\! \mathcolor{gray}{z}} \right] ~, \label{eq:DP^(2)-3_12-spectrum-chieff-G} \\
	\xint{\mathcolor{gray}{-}}{30}{\bar{P}}^{\;\! \textcolor{PineGreen}{\hat{3}} }_{\;\! \mathcolor{gray}{z} \textcolor{Maroon}{(2)} } &\xrightarrow[\text{\bref{eq:DP^(2)-3_12-discrete-G2}}]{\text{\bref{eq:scalar_nonlinear_drive-discrete}}} \bar{\chi}^{\;\! \textcolor{PineGreen}{\hat{3}} \hat{1} \hat{2} }_{\;\! \textcolor{Maroon}{(2)} \textcolor{PineGreen}{\hat{1} \hat{2}}} ~ {\hat{g}}^{\;\! \textcolor{PineGreen}{\hat{1}}}_{\;\! \hat{1}}  {\hat{g}}^{\;\! \textcolor{PineGreen}{\hat{2}}}_{\;\! \hat{2}} \odot \mathcolor{gray}{\mathcal F_{z}^{-1}} \left[ \xint{\mathcolor{gray}{-}}{18}{\bar{M}}^{\;\! \mathcolor{gray}{3} \hat{1} \hat{2} }_{\;\! \mathcolor{gray}{k_{\symup{z}}} \textcolor{Maroon}{(2)} } \mathcolor{gray}{*} \xint{\mathcolor{gray}{-}}{20}{\mathtt{G}}^{\;\! \textcolor{PineGreen}{\hat{1}}}_{\;\! \mathcolor{gray}{z}} \mathcolor{gray}{*} \xint{\mathcolor{gray}{-}}{20}{\mathtt{G}}^{\;\! \textcolor{PineGreen}{\hat{2}}}_{\;\! \mathcolor{gray}{z}} \right] ~, \label{eq:DP^(2)-3_12-discrete-chieff-G}
\end{align}
\end{subequations}
但这一般是不成立的:因为 \bref{chap:LFCO} 中的\textcolor{Plum}{线性}(\textcolor{Maroon}{傅立叶})\textcolor{PineGreen}{晶体光学}已经解析出结论:在非各向同性材料里,电磁波的\textcolor{PineGreen}{本征偏振态}是 $\mathcolor{gray}{\bar{k}_{\symup{\rho}}}$ 的函数;尽管如此,为了不止步于 \bref{eq:DP^(2)-3_12-spectrum-G},以及为了得到后续的标量\textcolor{Plum}{非线性}\textcolor{Maroon}{时空谱}耦合波方程,在
\begin{align} \label{eq:scalar_nonlinear_drive2}
	\text{\textbf{参与构成\textcolor{Plum}{非线性}\textcolor{NavyBlue}{波源}的所有行波,均为\textcolor{PineGreen}{本征偏振态}\textcolor{Plum}{固定}的标量场}}
\end{align}
即“\textbf{标量\textcolor{Plum}{非线性}\textcolor{NavyBlue}{波源}}”的假设 \bref{eq:scalar_nonlinear_drive} 下,从 \bref{eq:DP^(2)-3_12-chieff-G} 开始继续向后推导。

注意,不论正上标带 “$\mathcolor{gray}{\sim}$” 的符号有多少个(\bref{eq:DP^(2)-3_12-spectrum-chieff-G} 中有两个:“~$\mathcolor{gray}{\widetilde *},~ \mathcolor{gray}{\widetilde \circledast}$~”),只对这些符号所作用的最左(这里即 $\xint{{}^{}_{\mathcolor{gray}{-}}}{10}{\hat{g}}^{\;\! \mathcolor{gray}{\omega} \textcolor{PineGreen}{\hat{1}}}_{\;\! \hat{1}}$)到最右(这里即 $\xint{\mathcolor{gray}{-}}{16}{\mathtt{G}}^{\;\! \mathcolor{gray}{\omega} \textcolor{PineGreen}{\hat{2}}}_{\;\! \mathcolor{gray}{z}}$)之间的部分作为\textcolor{Plum}{被积函数}/\textcolor{Plum}{表达式},在\textcolor{gray}{时间频率}维度做一次(而非多次)一维\textcolor{Plum}{卷积积分}。此外,\textbf{需\textcolor{Plum}{按以下顺序执行积分}:“$\mathcolor{gray}{\widetilde \circledast}$” 的 $\mathcolor{gray}{\bar{k}_{\symup{\rho}}}$ 域 $\to$ $\mathcolor{gray}{\bar{k}_{\symup{\rho}}}$ 域的 “$\mathcolor{gray}{*}$” $\to$ $\mathcolor{gray}{k_{\symup{z}}}$ 域的 $\mathcolor{gray}{\mathcal F^{-1}_z} \left[ \cdot \right]$ $\to$ “$\mathcolor{gray}{\widetilde \circledast}$” 的 $\mathcolor{gray}{\omega}$ 域(即 $\mathcolor{gray}{\omega}$ 域的 ``~$\mathcolor{gray}{\widetilde *}$~'')}\bypertarget{OperatorSequence}。这也是相应程序中 \textbf{for 循环从内到外层的计算顺序}。

在 \bref{eq:scalar_nonlinear_drive,eq:scalar_nonlinear_drive2} 的“\textbf{标量\textcolor{Plum}{非线性}\textcolor{NavyBlue}{波源}}”条件下,\textcolor{NavyBlue}{脉冲光}\textcolor{Maroon}{倍频}、\textcolor{NavyBlue}{连续光}\textcolor{Maroon}{和频}过程的电场\textcolor{PineGreen}{本征复振幅} \bref{eq:simplify8-scalar-g-modulus-P-spectrum,eq:simplify8-scalar-g-modulus-P-discrete} 退化为
\begin{subequations} \label{eq:scalar-g-modulus-P-chieff}
\begin{align}
	\mathcolor{gray}{\nabla_z} \xint{\begin{smallmatrix} ~ \\ {}^{}_{\mathcolor{gray}{-}} \\ ~ \end{smallmatrix}}{09}{\mathtt{g}}^{\;\!\mathcolor{gray}{\omega} \textcolor{PineGreen}{\hat{3}}}_{\;\! \mathcolor{gray}{z}} &\xrightarrow[\text{\bref{eq:simplify8-scalar-g-modulus-P-spectrum}}]{\text{\bref{eq:scalar_nonlinear_drive-spectrum}}} \mathbb{i} k_{\textcolor{Maroon}{\mathsf{o}} \mathcolor{gray}{\omega}}^{\;\! 2} \frac{\textcolor{gray}{\xint{{}^{}_{\mathcolor{gray}{-}}}{23}{\widetilde{\textcolor{black}{\chi}}}}^{ \hat{3} \textcolor{PineGreen}{\hat{3}} \textcolor{Maroon}{(2)} \mathcolor{gray}{\omega} }_{ \textcolor{NavyBlue}{\text{eff}} \hat{1} \hat{2} \textcolor{PineGreen}{\hat{1} \hat{2}} } ~ \mathcolor{gray}{\mathcal F_{z}^{-1}} \left[ \xint{\mathcolor{gray}{-}}{18}{M}^{\;\! \mathcolor{gray}{\omega} \hat{1} \hat{2} }_{\;\! \hat{3} \mathcolor{gray}{k_{\symup{z}}} \textcolor{Maroon}{(2)} } \mathcolor{gray}{*} \xint{\mathcolor{gray}{-}}{15}{\mathtt{G}}^{\;\! \mathcolor{gray}{\omega} \textcolor{PineGreen}{\hat{1}}}_{\;\! \mathcolor{gray}{z}} ~\mathcolor{gray}{\widetilde \circledast}~ \xint{\mathcolor{gray}{-}}{15}{\mathtt{G}}^{\;\! \mathcolor{gray}{\omega} \textcolor{PineGreen}{\hat{2}}}_{\;\! \mathcolor{gray}{z}} \right]}{ 2 \lvert \xint{{}^{}_{\mathcolor{gray}{-}}}{10}{\hat{g}}^{\;\! \textcolor{PineGreen}{\hat{3}}}_{\;\! \mathcolor{gray}{\omega}} \rvert^2 \xint{\begin{smallmatrix} ~ \\ {}^{}_{\mathcolor{gray}{-}} \\ ~ \end{smallmatrix}}{15}{k}_{\;\! \symup{z}}^{\;\! \mathcolor{gray}{\omega} \textcolor{PineGreen}{\hat{3}}} \mathbb{e}^{\mathbb{i} \xint{\begin{smallmatrix} ~ \\ {}^{}_{\mathcolor{gray}{-}} \\ ~ \end{smallmatrix}}{15}{k}_{\symup{z}}^{\;\! \mathcolor{gray}{\omega} \textcolor{PineGreen}{\hat{3}}} \mathcolor{gray}{z}}} ~, \label{eq:scalar-g-modulus-P-chieff-spectrum} \\
	\mathcolor{gray}{\nabla_z} \xint{\begin{smallmatrix} ~ \\ {}^{}_{\mathcolor{gray}{-}} \\ ~ \end{smallmatrix}}{09}{\mathtt{g}}^{\;\! \textcolor{PineGreen}{\hat{3}}}_{\;\! \mathcolor{gray}{z}} &\xrightarrow[\text{\bref{eq:simplify8-scalar-g-modulus-P-discrete}}]{\text{\bref{eq:scalar_nonlinear_drive-discrete}}} \mathbb{i} k_{\textcolor{Maroon}{\mathsf{o}} \mathcolor{gray}{3}}^{\;\! 2} \frac{\xint{{}^{}_{\mathcolor{gray}{-}}}{23}{\chi}^{\hat{3} \textcolor{PineGreen}{\hat{3}} \textcolor{Maroon}{(2)} }_{\textcolor{NavyBlue}{\text{eff}} \hat{1} \textcolor{PineGreen}{\hat{1}} \hat{2} \textcolor{PineGreen}{\hat{2}} } ~ \mathcolor{gray}{\mathcal F_{z}^{-1}} \left[ \xint{\mathcolor{gray}{-}}{18}{M}^{\;\! \mathcolor{gray}{3} \hat{1} \hat{2} }_{\;\! \hat{3} \mathcolor{gray}{k_{\symup{z}}} \textcolor{Maroon}{(2)} } \mathcolor{gray}{*} \xint{\mathcolor{gray}{-}}{15}{\mathtt{G}}^{\;\! \textcolor{PineGreen}{\hat{1}}}_{\;\! \mathcolor{gray}{z}} \mathcolor{gray}{*} \xint{\mathcolor{gray}{-}}{15}{\mathtt{G}}^{\;\! \textcolor{PineGreen}{\hat{2}}}_{\;\! \mathcolor{gray}{z}} \right]}{ 2 \lvert \xint{{}^{}_{\mathcolor{gray}{-}}}{10}{\hat{g}}^{\;\! \textcolor{PineGreen}{\hat{3}}} \rvert^2 \xint{\begin{smallmatrix} ~ \\ {}^{}_{\mathcolor{gray}{-}} \\ ~ \end{smallmatrix}}{15}{k}_{\;\! \symup{z}}^{\;\!  \textcolor{PineGreen}{\hat{3}}} \mathbb{e}^{\mathbb{i} \xint{\begin{smallmatrix} ~ \\ {}^{}_{\mathcolor{gray}{-}} \\ ~ \end{smallmatrix}}{15}{k}_{\symup{z}}^{\;\!  \textcolor{PineGreen}{\hat{3}}} \mathcolor{gray}{z}}} ~, \label{eq:scalar-g-modulus-P-chieff-discrete}
\end{align}
\end{subequations}
其中,定义了\textcolor{NavyBlue}{脉冲光}\textcolor{Maroon}{倍频}、\textcolor{NavyBlue}{连续光}\textcolor{Maroon}{和频}过程的\textcolor{NavyBlue}{有效非线性系数}(三阶)张量(场)
\begin{subequations} \label{eq:chieff}
\begin{align}
	\textcolor{gray}{\xint{{}^{}_{\mathcolor{gray}{-}}}{23}{\widetilde{\textcolor{black}{\chi}}}}^{ \hat{3} \textcolor{PineGreen}{\hat{3}} \textcolor{Maroon}{(2)} \mathcolor{gray}{\omega} }_{ \textcolor{NavyBlue}{\text{eff}} \hat{1} \hat{2} \textcolor{PineGreen}{\hat{1} \hat{2}} } &= \xint{{}^{}_{\mathcolor{gray}{-}}}{10}{\hat{g}}^{\;\! \hat{3} \textcolor{PineGreen}{\hat{3}} \textcolor{Plum}{*}}_{\;\! \mathcolor{gray}{\omega}} {\chi}^{\;\! \hat{3} \textcolor{PineGreen}{\hat{3}} \textcolor{Maroon}{(2)} }_{\;\! \mathcolor{gray}{\omega} \hat{1} \hat{2} \textcolor{PineGreen}{\hat{1} \hat{2}} } ~ {\hat{g}}^{\;\! \mathcolor{gray}{\omega} }_{\;\! \hat{1} \textcolor{PineGreen}{\hat{1}} } ~\mathcolor{gray}{\widetilde *}~ {\hat{g}}^{\;\! \mathcolor{gray}{\omega} }_{\;\! \hat{2} \textcolor{PineGreen}{\hat{2}} } ~, \label{eq:chieff-spectrum} \\
	\xint{{}^{}_{\mathcolor{gray}{-}}}{23}{\chi}^{\hat{3} \textcolor{PineGreen}{\hat{3}} \textcolor{Maroon}{(2)} }_{\textcolor{NavyBlue}{\text{eff}} \hat{1} \textcolor{PineGreen}{\hat{1}} \hat{2} \textcolor{PineGreen}{\hat{2}} } &= \xint{{}^{}_{\mathcolor{gray}{-}}}{10}{\hat{g}}^{\;\! \hat{3} \textcolor{PineGreen}{\hat{3}} \textcolor{Plum}{*}}_{\;\! } {\chi}^{\;\! \hat{3} \textcolor{PineGreen}{\hat{3}} }_{\;\! \textcolor{Maroon}{(2)} \hat{1} \textcolor{PineGreen}{\hat{1}} \hat{2} \textcolor{PineGreen}{\hat{2}}} ~ {\hat{g}}_{\;\! \hat{1} \textcolor{PineGreen}{\hat{1}} } ~ {\hat{g}}_{\;\! \hat{2} \textcolor{PineGreen}{\hat{2}}} ~, \label{eq:chieff-discrete}
\end{align}
\end{subequations}
其中,$\chi$ 头上的一个\textcolor{gray}{灰色宽波浪}符号 `$\mathcolor{gray}{\sim}$',表示 $\textcolor{gray}{\xint{{}^{}_{\mathcolor{gray}{-}}}{23}{\widetilde{\textcolor{black}{\chi}}}}^{ \hat{3} \textcolor{PineGreen}{\hat{3}} \textcolor{Maroon}{(2)} \mathcolor{gray}{\omega} }_{ \textcolor{NavyBlue}{\text{eff}} \hat{1} \hat{2} \textcolor{PineGreen}{\hat{1} \hat{2}} }$ 整体,作为\textcolor{Plum}{被积函数},处在 \textcolor{gray}{$\omega$ 域}的一维\textcolor{Plum}{卷积积分}内。注意,虽然 \bref{eq:chieff-spectrum} 左侧的 $\chi$ 头上有一 `$\mathcolor{gray}{\sim}$',这并不代表相应的 $\textcolor{gray}{\xint{{}^{}_{\mathcolor{gray}{-}}}{23}{\widetilde{\textcolor{black}{\chi}}}}$ 是矢量(见 \bref{hook:1bar})。此外,\bref{eq:scalar-g-modulus-P-chieff} 的分子,即 $\textcolor{gray}{\xint{{}^{}_{\mathcolor{gray}{-}}}{23}{\widetilde{\textcolor{black}{\chi}}}}^{ \hat{3} \textcolor{PineGreen}{\hat{3}} \textcolor{Maroon}{(2)} \mathcolor{gray}{\omega} }_{ \textcolor{NavyBlue}{\text{eff}} \hat{1} \hat{2} \textcolor{PineGreen}{\hat{1} \hat{2}} }~, \xint{{}^{}_{\mathcolor{gray}{-}}}{23}{\chi}^{\hat{3} \textcolor{PineGreen}{\hat{3}} \textcolor{Maroon}{(2)} }_{\textcolor{NavyBlue}{\text{eff}} \hat{1} \textcolor{PineGreen}{\hat{1}} \hat{2} \textcolor{PineGreen}{\hat{2}} }$ 中的 $\xint{{}^{}_{\mathcolor{gray}{-}}}{10}{\hat{g}}^{\;\! \hat{3} \textcolor{PineGreen}{\hat{3}} \textcolor{Plum}{*}}_{\;\! \mathcolor{gray}{\omega}}, \xint{{}^{}_{\mathcolor{gray}{-}}}{10}{\hat{g}}^{\;\! \hat{3} \textcolor{PineGreen}{\hat{3}} \textcolor{Plum}{*}}_{\;\! }$ 是(\textcolor{Plum}{协变})矢量\Footnote{这里对\textcolor{Plum}{协变} $\bra{\text{bra}}$、\textcolor{Plum}{逆变} $\ket{\text{ket}}$ 指标 $\hat{3} \textcolor{PineGreen}{\hat{3}}$ 上下位置的定义,相反于 \byperref{another-bra-example}{前处}。可以看出,在这方面是灵活的。}的分量,即标量(场);而分母中的 $\xint{{}^{}_{\mathcolor{gray}{-}}}{10}{\hat{g}}^{\;\! \textcolor{PineGreen}{\hat{3}}}_{\;\! \mathcolor{gray}{\omega}}, \xint{{}^{}_{\mathcolor{gray}{-}}}{10}{\hat{g}}^{\;\! \textcolor{PineGreen}{\hat{3}}}$ 是矢量(场)。

若三阶张量\textcolor{NavyBlue}{调制场} $\xint{\mathcolor{gray}{-}}{18}{M}^{\;\! \mathcolor{gray}{\omega} \hat{1} \hat{2} }_{\;\! \hat{3} \mathcolor{gray}{z} \textcolor{Maroon}{(2)} }, \xint{\mathcolor{gray}{-}}{18}{\bar{M}}^{\;\! \mathcolor{gray}{\omega} \hat{1} \hat{2} }_{\;\! \mathcolor{gray}{z} \textcolor{Maroon}{(2)} }$ 退化为标量\textcolor{NavyBlue}{场} $\xint{\mathcolor{gray}{-}}{18}{M}^{\;\! \mathcolor{gray}{\omega} }_{\;\! \mathcolor{gray}{z} \textcolor{Maroon}{(2)} }$,即若再施加
\begin{align} \label{eq:scalar_chi2_modulation}
	&\text{\textbf{标量场 $\chi^{\;\! \mathcolor{gray}{\omega} }_{\;\! \mathcolor{gray}{z} \textcolor{Maroon}{(2)}}$ \textcolor{NavyBlue}{调制}}条件:} \hspace{0.2em} \xint{\mathcolor{gray}{-}}{18}{\bar{\bar{\bar{M}}}}^{\;\! \mathcolor{gray}{\omega} }_{\;\! \mathcolor{gray}{z} \textcolor{Maroon}{(2)} } \equiv \xint{\mathcolor{gray}{-}}{18}{M}^{\;\! \mathcolor{gray}{\omega} }_{\;\! \mathcolor{gray}{z} \textcolor{Maroon}{(2)} } ~,
\end{align}
条件,则\textcolor{NavyBlue}{脉冲光}\textcolor{Maroon}{倍频}、\textcolor{NavyBlue}{连续光}\textcolor{Maroon}{和频}过程的电场\textcolor{PineGreen}{本征复振幅}方程 \bref{eq:scalar-g-modulus-P-chieff} 退为
\begin{subequations} \label{eq:scalar-g-modulus-P-chieff-scalar}
\begin{align}
	\mathcolor{gray}{\nabla_z} \xint{\begin{smallmatrix} ~ \\ {}^{}_{\mathcolor{gray}{-}} \\ ~ \end{smallmatrix}}{09}{\mathtt{g}}^{\;\!\mathcolor{gray}{\omega} \textcolor{PineGreen}{\hat{3}}}_{\;\! \mathcolor{gray}{z}} &\xrightarrow[\text{\bref{eq:scalar-g-modulus-P-chieff-spectrum}}]{\text{\bref{eq:scalar_chi2_modulation}}} \mathbb{i} k_{\textcolor{Maroon}{\mathsf{o}} \mathcolor{gray}{\omega}}^{\;\! 2} \frac{\textcolor{gray}{\xint{{}^{}_{\mathcolor{gray}{-}}}{23}{\widetilde{\textcolor{black}{\chi}}}}^{ \textcolor{PineGreen}{\hat{3}} \textcolor{Maroon}{(2)} \mathcolor{gray}{\omega} }_{ \textcolor{NavyBlue}{\text{eff}} \textcolor{PineGreen}{\hat{1} \hat{2}} } ~ \mathcolor{gray}{\mathcal F_{z}^{-1}} \left[ \xint{\mathcolor{gray}{-}}{18}{M}^{\;\! \mathcolor{gray}{\omega} }_{\;\! \mathcolor{gray}{k_{\symup{z}}} \textcolor{Maroon}{(2)} } \mathcolor{gray}{*} \xint{\mathcolor{gray}{-}}{15}{\mathtt{G}}^{\;\! \mathcolor{gray}{\omega} \textcolor{PineGreen}{\hat{1}}}_{\;\! \mathcolor{gray}{z}} ~\mathcolor{gray}{\widetilde \circledast}~ \xint{\mathcolor{gray}{-}}{15}{\mathtt{G}}^{\;\! \mathcolor{gray}{\omega} \textcolor{PineGreen}{\hat{2}}}_{\;\! \mathcolor{gray}{z}} \right]}{ 2 \lvert \xint{{}^{}_{\mathcolor{gray}{-}}}{10}{\hat{g}}^{\;\! \textcolor{PineGreen}{\hat{3}}}_{\;\! \mathcolor{gray}{\omega}} \rvert^2 \xint{\begin{smallmatrix} ~ \\ {}^{}_{\mathcolor{gray}{-}} \\ ~ \end{smallmatrix}}{15}{k}_{\;\! \symup{z}}^{\;\! \mathcolor{gray}{\omega} \textcolor{PineGreen}{\hat{3}}} \mathbb{e}^{\mathbb{i} \xint{\begin{smallmatrix} ~ \\ {}^{}_{\mathcolor{gray}{-}} \\ ~ \end{smallmatrix}}{15}{k}_{\symup{z}}^{\;\! \mathcolor{gray}{\omega} \textcolor{PineGreen}{\hat{3}}} \mathcolor{gray}{z}}} ~, \label{eq:scalar-g-modulus-P-chieff-scalar-spectrum} \\
	\mathcolor{gray}{\nabla_z} \xint{\begin{smallmatrix} ~ \\ {}^{}_{\mathcolor{gray}{-}} \\ ~ \end{smallmatrix}}{09}{\mathtt{g}}^{\;\! \textcolor{PineGreen}{\hat{3}}}_{\;\! \mathcolor{gray}{z}} &\xrightarrow[\text{\bref{eq:scalar-g-modulus-P-chieff-discrete}}]{\text{\bref{eq:scalar_chi2_modulation}}} \mathbb{i} k_{\textcolor{Maroon}{\mathsf{o}} \mathcolor{gray}{3}}^{\;\! 2} \frac{\xint{{}^{}_{\mathcolor{gray}{-}}}{23}{\chi}^{\textcolor{PineGreen}{\hat{3}} \textcolor{Maroon}{(2)} }_{\textcolor{NavyBlue}{\text{eff}} \textcolor{PineGreen}{\hat{1}} \textcolor{PineGreen}{\hat{2}} } ~ \mathcolor{gray}{\mathcal F_{z}^{-1}} \left[ \xint{\mathcolor{gray}{-}}{18}{M}^{\;\! \mathcolor{gray}{3}}_{\;\! \mathcolor{gray}{k_{\symup{z}}} \textcolor{Maroon}{(2)} } \mathcolor{gray}{*} \xint{\mathcolor{gray}{-}}{15}{\mathtt{G}}^{\;\! \textcolor{PineGreen}{\hat{1}}}_{\;\! \mathcolor{gray}{z}} \mathcolor{gray}{*} \xint{\mathcolor{gray}{-}}{15}{\mathtt{G}}^{\;\! \textcolor{PineGreen}{\hat{2}}}_{\;\! \mathcolor{gray}{z}} \right]}{ 2 \lvert \xint{{}^{}_{\mathcolor{gray}{-}}}{10}{\hat{g}}^{\;\! \textcolor{PineGreen}{\hat{3}}} \rvert^2 \xint{\begin{smallmatrix} ~ \\ {}^{}_{\mathcolor{gray}{-}} \\ ~ \end{smallmatrix}}{15}{k}_{\;\! \symup{z}}^{\;\!  \textcolor{PineGreen}{\hat{3}}} \mathbb{e}^{\mathbb{i} \xint{\begin{smallmatrix} ~ \\ {}^{}_{\mathcolor{gray}{-}} \\ ~ \end{smallmatrix}}{15}{k}_{\symup{z}}^{\;\!  \textcolor{PineGreen}{\hat{3}}} \mathcolor{gray}{z}}} ~, \label{eq:scalar-g-modulus-P-chieff-scalar-discrete}
\end{align}
\end{subequations}
同时\textcolor{NavyBlue}{脉冲光}\textcolor{Maroon}{倍频}、\textcolor{NavyBlue}{连续光}\textcolor{Maroon}{和频}过程的\textcolor{NavyBlue}{有效非线性系数}张量 \bref{eq:chieff} 退为标量
\begin{subequations} \label{eq:chieff-scalar}
\begin{align}
	\textcolor{gray}{\xint{{}^{}_{\mathcolor{gray}{-}}}{23}{\widetilde{\textcolor{black}{\chi}}}}^{ \textcolor{PineGreen}{\hat{3}} \textcolor{Maroon}{(2)} \mathcolor{gray}{\omega} }_{ \textcolor{NavyBlue}{\text{eff}} \textcolor{PineGreen}{\hat{1} \hat{2}} } &\xrightarrow[\text{\bref{eq:chieff-spectrum}}]{\text{\bref{eq:scalar_chi2_modulation}}} \xint{{}^{}_{\mathcolor{gray}{-}}}{10}{\hat{g}}^{\;\! \hat{3} \textcolor{PineGreen}{\hat{3}} \textcolor{Plum}{*}}_{\;\! \mathcolor{gray}{\omega}} {\chi}^{\;\! \textcolor{PineGreen}{\hat{3}} \mathcolor{gray}{\omega} \hat{1} \hat{2} }_{\;\! \hat{3} \textcolor{Maroon}{(2)} \textcolor{PineGreen}{\hat{1} \hat{2}}} ~ {\hat{g}}^{\;\! \mathcolor{gray}{\omega} }_{\;\! \hat{1} \textcolor{PineGreen}{\hat{1}} } ~\mathcolor{gray}{\widetilde *}~ {\hat{g}}^{\;\! \mathcolor{gray}{\omega} }_{\;\! \hat{2} \textcolor{PineGreen}{\hat{2}} } ~, \label{eq:chieff-scalar-spectrum} \\
	\xint{{}^{}_{\mathcolor{gray}{-}}}{23}{\chi}^{\textcolor{PineGreen}{\hat{3}} \textcolor{Maroon}{(2)} }_{\textcolor{NavyBlue}{\text{eff}} \textcolor{PineGreen}{\hat{1}} \textcolor{PineGreen}{\hat{2}} } &\xrightarrow[\text{\bref{eq:chieff-discrete}}]{\text{\bref{eq:scalar_chi2_modulation}}} \xint{{}^{}_{\mathcolor{gray}{-}}}{10}{\hat{g}}^{\;\! \hat{3} \textcolor{PineGreen}{\hat{3}} \textcolor{Plum}{*}}_{\;\! } {\chi}^{\;\! \textcolor{PineGreen}{\hat{3}} \hat{1} \hat{2} }_{\;\! \hat{3} \textcolor{Maroon}{(2)} \textcolor{PineGreen}{\hat{1} \hat{2}}} ~ {\hat{g}}_{\;\! \hat{1} \textcolor{PineGreen}{\hat{1}} } ~ {\hat{g}}_{\;\! \hat{2} \textcolor{PineGreen}{\hat{2}} } ~, \label{eq:chieff-scalar-discrete}
\end{align}
\end{subequations}
注意,\bref{eq:chieff} 中的 ${\chi}^{\;\! \hat{3} \textcolor{PineGreen}{\hat{3}} }_{\;\! \textcolor{Maroon}{(2)} \hat{1} \hat{2} \textcolor{PineGreen}{\hat{1} \hat{2}} \mathcolor{gray}{\omega}}, {\chi}^{\;\! \hat{3} \textcolor{PineGreen}{\hat{3}} }_{\;\! \textcolor{Maroon}{(2)} \hat{1} \textcolor{PineGreen}{\hat{1}} \hat{2} \textcolor{PineGreen}{\hat{2}}}$,以及 \bref{eq:chieff-scalar} 中的 ${\chi}^{\;\! \textcolor{PineGreen}{\hat{3}} \mathcolor{gray}{\omega} \hat{1} \hat{2} }_{\;\! \hat{3} \textcolor{Maroon}{(2)} \textcolor{PineGreen}{\hat{1} \hat{2}}}, {\chi}^{\;\! \textcolor{PineGreen}{\hat{3}} \hat{1} \hat{2} }_{\;\! \hat{3} \textcolor{Maroon}{(2)} \textcolor{PineGreen}{\hat{1} \hat{2}}}$,如 \bref{eq:chi2-modulate} 下方的 \byperref{chi2-free-of-eigenmodes}{说明文字} 所提,二阶\textcolor{Plum}{非线性}系数 $\chi$ 的角标 $\textcolor{PineGreen}{\hat{1}}, \textcolor{PineGreen}{\hat{2}}$ 不从属于任何主体(包括它自己 $\chi$ 和其他 $g$),只服务于爱因斯坦求和。

\vspace*{-4.0em}

\marginLeft[-2.4em]{ssec:undepleted-pump-approximation}\subsection{泵浦未耗尽近似条件下的非线性卷积解}\label{ssec:undepleted-pump-approximation}

在得到了 \bref{ssec:SHG_spectrum,ssec:SFG_discrete} 中\textcolor{Maroon}{上转换}过程的电场\textcolor{PineGreen}{本征复振幅} \bref{eq:simplify8-scalar-g-modulus-P-spectrum,eq:simplify8-scalar-g-modulus-P-discrete} 后,现尝试着对它们进行\textbf{\textcolor{Plum}{解析求解}}。为简便起见,暂以三个\textcolor{NavyBlue}{连续光}参与\textcolor{gray}{混频}所对应的 \bref{eq:simplify8-scalar-g-modulus-P-discrete} 为例;\textcolor{NavyBlue}{脉冲光}对应的 \bref{eq:simplify8-scalar-g-modulus-P-spectrum} 同理。
\begin{subequations} \label{eq:up-scalar-g-EE-312-discrete-convolution}
\begin{align}
	\hspace{-0.2em} \mathcolor{gray}{\nabla_z} \xint{\begin{smallmatrix} ~ \\ {}^{}_{\mathcolor{gray}{-}} \\ ~ \end{smallmatrix}}{09}{\mathtt{g}}^{\;\! \textcolor{PineGreen}{\hat{3}}}_{\;\! \mathcolor{gray}{z}} \xrightarrow[]{\text{\bref{eq:simplify8-scalar-g-modulus-P-discrete}}} \mathbb{i} k_{\textcolor{Maroon}{\mathsf{o}} \mathcolor{gray}{3}}^{\;\! 2} \frac{\xint{{}^{}_{\mathcolor{gray}{-}}}{10}{\hat{g}}^{\;\! \hat{3} \textcolor{PineGreen}{\hat{3}} \textcolor{Plum}{*}}_{\;\! } {\chi}^{\;\! \textcolor{PineGreen}{\hat{3}}  \hat{1} \hat{2} }_{\;\! \hat{3} \textcolor{Maroon}{(2)} \textcolor{PineGreen}{\hat{1} \hat{2}}} }{ 2 \lvert \xint{{}^{}_{\mathcolor{gray}{-}}}{10}{\hat{g}}^{\;\! \textcolor{PineGreen}{\hat{3}}} \rvert^2 \xint{\begin{smallmatrix} ~ \\ {}^{}_{\mathcolor{gray}{-}} \\ ~ \end{smallmatrix}}{15}{k}_{\;\! \symup{z}}^{\;\!  \textcolor{PineGreen}{\hat{3}}} }~ \mathcolor{gray}{\mathcal F_{z}^{-1}} & \left[ \xint{\mathcolor{gray}{-}}{18}{M}^{\;\! \mathcolor{gray}{3} \hat{1} \hat{2} }_{\;\! \hat{3} \mathcolor{gray}{k_{\symup{z}}} \textcolor{Maroon}{(2)} } \mathcolor{gray}{*} \xint{\mathcolor{gray}{-}}{25}{E}^{\;\! \textcolor{PineGreen}{\hat{1}}  }_{\;\! \hat{1} \mathcolor{gray}{z}} \mathcolor{gray}{*} \xint{\mathcolor{gray}{-}}{25}{E}^{\;\! \textcolor{PineGreen}{\hat{2}}  }_{\;\! \hat{2} \mathcolor{gray}{z}} \right] \big/ \mathbb{e}^{\mathbb{i} \xint{\begin{smallmatrix} ~ \\ {}^{}_{\mathcolor{gray}{-}} \\ ~ \end{smallmatrix}}{15}{k}_{\symup{z}}^{\;\!  \textcolor{PineGreen}{\hat{3}}} \mathcolor{gray}{z}} \label{eq:up-scalar-g-EE-312-discrete-convolution1} \\
	\xrightarrow[\text{\bref{eq:coupling-coeff}}]{\text{\bref{eq:components-eigenwave}}}: \Upsilon^{\;\! \hat{3} \textcolor{PineGreen}{\hat{3}} \hat{1} \hat{2} }_{\;\! \textcolor{Maroon}{(2)} \textcolor{PineGreen}{\hat{1} \hat{2}}}~ \mathcolor{gray}{\mathcal F_{z}^{-1}} \left[ \xint{\mathcolor{gray}{-}}{18}{M}^{\;\! \mathcolor{gray}{3} \hat{1} \hat{2} }_{\;\! \hat{3} \mathcolor{gray}{k_{\symup{z}}} \textcolor{Maroon}{(2)} } \mathcolor{gray}{*} \right.&\left. \! \big( \xint{{}^{}_{\mathcolor{gray}{-}}}{10}{g}^{\;\! \textcolor{PineGreen}{\hat{1}}}_{\;\! \hat{1} \mathcolor{gray}{z}} \mathbb{e}^{\mathbb{i} \xint{\begin{smallmatrix} ~ \\ {}^{}_{\mathcolor{gray}{-}} \\ ~ \end{smallmatrix}}{15}{k}_{\symup{z}}^{\;\! \textcolor{PineGreen}{\hat{1}} } \mathcolor{gray}{z}} \big) \mathcolor{gray}{*} \big( \xint{{}^{}_{\mathcolor{gray}{-}}}{10}{g}^{\;\! \textcolor{PineGreen}{\hat{2}}}_{\;\! \hat{2} \mathcolor{gray}{z}} \mathbb{e}^{\mathbb{i} \xint{\begin{smallmatrix} ~ \\ {}^{}_{\mathcolor{gray}{-}} \\ ~ \end{smallmatrix}}{15}{k}_{\symup{z}}^{\;\! \textcolor{PineGreen}{\hat{2}} } \mathcolor{gray}{z} } \big) \right] \big/ \mathbb{e}^{\mathbb{i} \xint{\begin{smallmatrix} ~ \\ {}^{}_{\mathcolor{gray}{-}} \\ ~ \end{smallmatrix}}{15}{k}_{\symup{z}}^{\;\!  \textcolor{PineGreen}{\hat{3}}} \mathcolor{gray}{z}} \label{eq:up-scalar-g-EE-312-discrete-convolution2} \\
	\xrightarrow[]{\text{\bref{eq:Nondepleted-Pump-Approximation-discrete}}} \Upsilon^{\;\! \hat{3} \textcolor{PineGreen}{\hat{3}} \hat{1} \hat{2} }_{\;\! \textcolor{Maroon}{(2)} \textcolor{PineGreen}{\hat{1} \hat{2}}}~ \mathcolor{gray}{\mathcal F_{z}^{-1}} \left[ \xint{\mathcolor{gray}{-}}{18}{M}^{\;\! \mathcolor{gray}{3} \hat{1} \hat{2} }_{\;\! \hat{3} \mathcolor{gray}{k_{\symup{z}}} \textcolor{Maroon}{(2)} } \mathcolor{gray}{*} \right.&\left. \! \big( \xint{{}^{}_{\mathcolor{gray}{-}}}{10}{g}^{\;\! \textcolor{PineGreen}{\hat{1}}}_{\;\! \hat{1} \mathcolor{gray}{0}} \mathbb{e}^{\mathbb{i} \xint{\begin{smallmatrix} ~ \\ {}^{}_{\mathcolor{gray}{-}} \\ ~ \end{smallmatrix}}{15}{k}_{\symup{z}}^{\;\! \textcolor{PineGreen}{\hat{1}} \mathcolor{gray}{z}} } \big) \mathcolor{gray}{*} \big( \xint{{}^{}_{\mathcolor{gray}{-}}}{10}{g}^{\;\! \textcolor{PineGreen}{\hat{2}}}_{\;\! \hat{2} \mathcolor{gray}{0}} \mathbb{e}^{\mathbb{i} \xint{\begin{smallmatrix} ~ \\ {}^{}_{\mathcolor{gray}{-}} \\ ~ \end{smallmatrix}}{15}{k}_{\symup{z}}^{\;\! \textcolor{PineGreen}{\hat{2}} \mathcolor{gray}{z}} } \big) \right] \big/ \mathbb{e}^{\mathbb{i} \xint{\begin{smallmatrix} ~ \\ {}^{}_{\mathcolor{gray}{-}} \\ ~ \end{smallmatrix}}{15}{k}_{\symup{z}}^{\;\!  \textcolor{PineGreen}{\hat{3}}} \mathcolor{gray}{z}} \label{eq:up-scalar-g-EE-312-discrete-convolution3} \\
	= \Upsilon^{\;\! \hat{3} \textcolor{PineGreen}{\hat{3}} \hat{1} \hat{2} }_{\;\! \textcolor{Maroon}{(2)} \textcolor{PineGreen}{\hat{1} \hat{2}}} \mathcolor{gray}{\iiint} \xint{\mathcolor{gray}{-}}{18}{M}^{\;\! \mathcolor{gray}{3} \hat{1} \hat{2} }_{\;\! \hat{3} \textcolor{Maroon}{(2)} } \left( \mathcolor{gray}{\bar{q}} \right) \mathcolor{gray}{\iint} \xint{{}^{}_{\mathcolor{gray}{-}}}{10}{g}^{\;\! \textcolor{PineGreen}{\hat{1}}}_{\;\! \hat{1} \mathcolor{gray}{0}} & \left( \mathcolor{gray}{\bar{k}_{1\symup{\rho}}} \right) \xint{{}^{}_{\mathcolor{gray}{-}}}{10}{g}^{\;\! \textcolor{PineGreen}{\hat{2}}}_{\;\! \hat{2} \mathcolor{gray}{0}} \left( \mathcolor{gray}{\bar{k}_{2\symup{\rho}}} \right) \mathbb{e}^{\mathbb{i} \Delta \xint{\begin{smallmatrix} ~ \\ {}^{}_{\mathcolor{gray}{-}} \\ ~ \end{smallmatrix}}{15}{k}_{\symup{z}}^{\;\! \textcolor{PineGreen}{\hat{1} \hat{2} \hat{3}} } \mathcolor{gray}{z} } ~ \mathbb{d} \mathcolor{gray}{\bar{k}_{1\symup{\rho}}} \mathbb{d} \mathcolor{gray}{\bar{q}} ~, \label{eq:up-scalar-g-EE-312-discrete-convolution4} \\
	\text{where}\ \ \ \ \text{\textcolor{gray}{6D}} \ \ \ \ \ \Delta \xint{\begin{smallmatrix} ~ \\ {}^{}_{\mathcolor{gray}{-}} \\ ~ \end{smallmatrix}}{15}{k}_{\symup{z}}^{\;\! \textcolor{PineGreen}{\hat{1} \hat{2} \hat{3}} } :=\; & \xint{\begin{smallmatrix} ~ \\ {}^{}_{\mathcolor{gray}{-}} \\ ~ \end{smallmatrix}}{15}{k}_{\symup{z}}^{\;\! \textcolor{PineGreen}{\hat{1} \hat{2}} } - \xint{\begin{smallmatrix} ~ \\ {}^{}_{\mathcolor{gray}{-}} \\ ~ \end{smallmatrix}}{15}{k}_{\symup{z}}^{\;\! \textcolor{PineGreen}{\hat{3}} } ~~~ \text{(\textcolor{gray}{纵向})\textcolor{PineGreen}{波矢失配量}} ~, \label{eq:up-scalar-g-EE-312-discrete-convolution5} \\
	\text{in which}\ \ \ \ \text{\textcolor{gray}{6D}} \ \ \ \! \ \ \ \ \ \;\! \xint{\begin{smallmatrix} ~ \\ {}^{}_{\mathcolor{gray}{-}} \\ ~ \end{smallmatrix}}{15}{k}_{\symup{z}}^{\;\! \textcolor{PineGreen}{\hat{1} \hat{2}} } :=\; & \xint{\begin{smallmatrix} ~ \\ {}^{}_{\mathcolor{gray}{-}} \\ ~ \end{smallmatrix}}{15}{k}_{\symup{z}}^{\;\! \textcolor{PineGreen}{\hat{1}} } \left( \mathcolor{gray}{\bar{k}_{1\symup{\rho}}} \right) + \xint{\begin{smallmatrix} ~ \\ {}^{}_{\mathcolor{gray}{-}} \\ ~ \end{smallmatrix}}{15}{k}_{\symup{z}}^{\;\! \textcolor{PineGreen}{\hat{2}} } \left( \mathcolor{gray}{\bar{k}_{2\symup{\rho}}} \right) + \mathcolor{gray}{q_{\symup{z}}} \label{eq:up-scalar-g-EE-312-discrete-convolution6}~, \\
	\text{and}\ \ \ \ \text{\textcolor{gray}{6D}} \ \ \ \ \ \ \ \ \ \;\! \mathcolor{gray}{\bar{k}_{2\symup{\rho}}} :=\; & \mathcolor{gray}{\bar{k}_{\symup{\rho}}} - \mathcolor{gray}{\bar{k}_{1\symup{\rho}}} - \mathcolor{gray}{\bar{q}_{\symup{\rho}}} \label{eq:up-scalar-g-EE-312-discrete-convolution7}~,
\end{align}
\end{subequations}
其中,定义了该过程的 z 向\textcolor{PineGreen}{波矢失配量} $\Delta \xint{\begin{smallmatrix} ~ \\ {}^{}_{\mathcolor{gray}{-}} \\ ~ \end{smallmatrix}}{15}{k}$ \Footnote{需要特别注意的是,在未经后续\textcolor{Plum}{非线性}\textcolor{Maroon}{角谱}理论进一步\textcolor{Plum}{解析}之前,\textcolor{Plum}{全矢量}和\textcolor{Plum}{纯标量}\textcolor{Plum}{非线性}\textcolor{Maroon}{傅立叶光学}显示/指出:{\one} \textcolor{PineGreen}{波矢失配量}是 \textcolor{gray}{6 维}的,而不是\textcolor{Plum}{半矢量}\textcolor{Plum}{非线性}\textcolor{Maroon}{傅立叶光学}所给出的,和传统意义上广为人知的 \textcolor{gray}{2 维}的。{\two} \textcolor{PineGreen}{波矢失配量} $\Delta \xint{\begin{smallmatrix} ~ \\ {}^{}_{\mathcolor{gray}{-}} \\ ~ \end{smallmatrix}}{15}{k}$ 是个标量(场),不是矢量(场),且只需、只须考虑\textcolor{PineGreen}{失配波矢}的 z 分量。} \bref{eq:up-scalar-g-EE-312-discrete-convolution5} 和 \textcolor{NavyBlue}{耦合(强度)系数}
\begin{align} \label{eq:coupling-coeff}
	\Upsilon^{\;\! \hat{3} \textcolor{PineGreen}{\hat{3}} \hat{1} \hat{2} }_{\;\! \textcolor{Maroon}{(2)} \textcolor{PineGreen}{\hat{1} \hat{2}}} = \mathbb{i} k_{\textcolor{Maroon}{\mathsf{o}} \mathcolor{gray}{3}}^{\;\! 2} \frac{\xint{{}^{}_{\mathcolor{gray}{-}}}{10}{\hat{g}}^{\;\! \hat{3} \textcolor{PineGreen}{\hat{3}} \textcolor{Plum}{*}}_{\;\! } {\chi}^{\;\! \hat{3} \textcolor{PineGreen}{\hat{3}} \hat{1} \hat{2} }_{\;\! \textcolor{Maroon}{(2)} \textcolor{PineGreen}{\hat{1} \hat{2}}} }{ 2 \lvert \xint{{}^{}_{\mathcolor{gray}{-}}}{10}{\hat{g}}^{\;\! \textcolor{PineGreen}{\hat{3}}} \rvert^2 \xint{\begin{smallmatrix} ~ \\ {}^{}_{\mathcolor{gray}{-}} \\ ~ \end{smallmatrix}}{15}{k}_{\;\! \symup{z}}^{\;\!  \textcolor{PineGreen}{\hat{3}}} } ~,
\end{align}
并且,使用到了\textbf{\textcolor{NavyBlue}{泵浦未耗尽}近似条件}(即\textcolor{NavyBlue}{泵浦们}的\textcolor{PineGreen}{本征复振幅} $\left\{ \xint{\begin{smallmatrix} ~ \\ {}^{}_{\mathcolor{gray}{-}} \\ ~ \end{smallmatrix}}{09}{\mathtt{g}}^{\;\! \mathcolor{gray}{\omega} \textcolor{PineGreen}{\hat{\jmath}}}_{\;\! \mathcolor{gray}{0}} \right\}$ 与 $\mathcolor{gray}{z}$ 无关,亦即只在晶体内\textcolor{Plum}{线性}衍射;注意,这类似但不同于\textbf{标量\textcolor{Plum}{非线性}\textcolor{NavyBlue}{波源}} \bref{eq:scalar_nonlinear_drive})
\begin{subequations} \label{eq:Nondepleted-Pump-Approximation}
\begin{align}
	\hspace{-7em} \text{\textbf{for \textcolor{NavyBlue}{pump}} (\textcolor{NavyBlue}{Pulse}):} \hspace{0.65em} \xint{\begin{smallmatrix} ~ \\ {}^{}_{\mathcolor{gray}{-}} \\ ~ \end{smallmatrix}}{09}{\mathtt{g}}^{\;\! \mathcolor{gray}{\omega} \textcolor{PineGreen}{\hat{\jmath}}}_{\;\! \mathcolor{gray}{z}} &\equiv \xint{\begin{smallmatrix} ~ \\ {}^{}_{\mathcolor{gray}{-}} \\ ~ \end{smallmatrix}}{09}{\mathtt{g}}^{\;\!  \mathcolor{gray}{\omega} \textcolor{PineGreen}{\hat{\jmath}}}_{\;\! \mathcolor{gray}{0}} ~, \label{eq:Nondepleted-Pump-Approximation-spectrum} \\
	\hspace{-7em} \text{\textbf{for \textcolor{NavyBlue}{pump}} (\textcolor{NavyBlue}{Continuous Wave}):} \hspace{1em} \xint{\begin{smallmatrix} ~ \\ {}^{}_{\mathcolor{gray}{-}} \\ ~ \end{smallmatrix}}{09}{\mathtt{g}}^{\;\! \textcolor{PineGreen}{\hat{\jmath}}}_{\;\! \mathcolor{gray}{z}} &\equiv \xint{\begin{smallmatrix} ~ \\ {}^{}_{\mathcolor{gray}{-}} \\ ~ \end{smallmatrix}}{09}{\mathtt{g}}^{\;\! \textcolor{PineGreen}{\hat{\jmath}}}_{\;\! \mathcolor{gray}{0}} ~, \label{eq:Nondepleted-Pump-Approximation-discrete}
\end{align}
\end{subequations}
注意,\bref{eq:up-scalar-g-EE-312-discrete-convolution4} 的 4 维\textcolor{Plum}{非线性卷积}中,对 $\mathcolor{gray}{\bar{k}_{1\symup{\rho}}}$ 的积分,也可换作是对 $\mathcolor{gray}{\bar{k}_{2\symup{\rho}}}$ 的积分,只需将上述 \bref{eq:up-scalar-g-EE-312-discrete-convolution7} 更改为 $\mathcolor{gray}{\bar{k}_{1\symup{\rho}}} := \mathcolor{gray}{\bar{k}_{\symup{\rho}}} - \mathcolor{gray}{\bar{k}_{2\symup{\rho}}} - \mathcolor{gray}{\bar{q}_{\symup{\rho}}}$ 即可,其他不必作改动。此外,\bref{eq:up-scalar-g-EE-312-discrete-convolution5,eq:up-scalar-g-EE-312-discrete-convolution6,eq:up-scalar-g-EE-312-discrete-convolution7} 中的 \textcolor{gray}{6D},表示相应\textcolor{NavyBlue}{场量}的\textcolor{Plum}{自变量}是 \textcolor{gray}{6 维}的。

可以看出,在\textcolor{Plum}{非线性}\textcolor{Maroon}{傅立叶光学}中,由于该方法预先内置的\textcolor{Maroon}{谱分析}模板,默认对\textcolor{gray}{空域}三维波动方程 \bref{eq:wave_trho} 执行了 2D 横向\textcolor{Plum}{傅立叶变换} \bref{eq:FT-krho},这使得\textbf{横向($\mathcolor{gray}{\symup{\rho}}$ 向或 $\perp$ 向)\textcolor{gray}{波矢总是守恒、匹配的}(见 \bref{eq:up-scalar-g-EE-312-discrete-convolution7}),\textcolor{PineGreen}{波矢失配}只发生在纵向($\mathcolor{gray}{z}$ 向)上(见 \bref{eq:up-scalar-g-EE-312-discrete-convolution6})}。这使得看上去只能计算 \textcolor{Maroon}{Bragg} 或 \textcolor{Maroon}{Raman-Nath} 衍射型\textcolor{Plum}{非线性}过程,而无法计算优先保证\textcolor{PineGreen}{纵向匹配}的 \textcolor{Maroon}{Cherenkov-type} \textcolor{Plum}{非线性}过程?不是的。后面我们会证明,该方法也能计算\textcolor{Plum}{非线性} \textcolor{Maroon}{Cherenkov} 过程,并且会打破众所周知的对 \textcolor{Maroon}{Cherenkov-type} \textcolor{Plum}{非线性}过程的固/旧有认知。

将 \bref{eq:up-scalar-g-EE-312-discrete-convolution4} 对 $\mathcolor{gray}{z}$ 沿 $\mathcolor{gray}{0} \to \mathcolor{gray}{z}$ \textcolor{Plum}{定积分},得\textcolor{NavyBlue}{连续光}\textcolor{Maroon}{和频}标量\textcolor{Maroon}{时空谱}耦合波方程 \bref{eq:up-scalar-g-EE-312-discrete-convolution3} 的 4 维(横向)\textcolor{Plum}{非线性卷积}解:
\begin{subequations} \label{eq:up-scalar-g-EE-312-discrete-since}
\begin{align}
	&\mathllap{ \xint{\begin{smallmatrix} ~ \\ {}^{}_{\mathcolor{gray}{-}} \\ ~ \end{smallmatrix}}{09}{\mathtt{g}}^{\;\! \textcolor{PineGreen}{\hat{3}}}_{\;\! \mathcolor{gray}{z}} \xrightarrow[\text{\bref{eq:up-scalar-g-EE-312-discrete-convolution4}}]{\text{$\mathcolor{gray}{\int}_{\mathcolor{gray}{0}}^{\mathcolor{gray}{z}} \cdot \mathbb{d} \mathcolor{gray}{z}$}} } \Upsilon^{\;\! \hat{3} \textcolor{PineGreen}{\hat{3}} \hat{1} \hat{2} }_{\;\! \textcolor{Maroon}{(2)} \textcolor{PineGreen}{\hat{1} \hat{2}} } \mathcolor{gray}{\iiint} \! \xint{\mathcolor{gray}{-}}{18}{M}^{\;\! \mathcolor{gray}{3} \hat{1} \hat{2} }_{\;\! \hat{3} \textcolor{Maroon}{(2)} } \left( \mathcolor{gray}{\bar{q}} \right) \mathcolor{gray}{\iint} \! \xint{{}^{}_{\mathcolor{gray}{-}}}{10}{g}^{\;\! \textcolor{PineGreen}{\hat{1}}}_{\;\! \hat{1} \mathcolor{gray}{0}} \! \left( \mathcolor{gray}{\bar{k}_{1\symup{\rho}}} \right) \xint{{}^{}_{\mathcolor{gray}{-}}}{10}{g}^{\;\! \textcolor{PineGreen}{\hat{2}}}_{\;\! \hat{2} \mathcolor{gray}{0}} \! \left( \mathcolor{gray}{\bar{k}_{2\symup{\rho}}} \right) \frac{ \mathbb{e}^{\mathbb{i} \Delta \xint{\begin{smallmatrix} ~ \\ {}^{}_{\mathcolor{gray}{-}} \\ ~ \end{smallmatrix}}{15}{k}_{\symup{z}}^{\;\! \textcolor{PineGreen}{\hat{1} \hat{2} \hat{3}} } \mathcolor{gray}{z} } - 1 }{ \mathbb{i} \Delta \xint{\begin{smallmatrix} ~ \\ {}^{}_{\mathcolor{gray}{-}} \\ ~ \end{smallmatrix}}{15}{k}_{\symup{z}}^{\;\! \textcolor{PineGreen}{\hat{1} \hat{2} \hat{3}} } } ~ \mathbb{d} \mathcolor{gray}{\bar{k}_{1\symup{\rho}}} \mathbb{d} \mathcolor{gray}{\bar{q}} \label{eq:up-scalar-g-EE-312-discrete-since1} \\
	&\mathllap{=} \Upsilon^{\;\! \hat{3} \textcolor{PineGreen}{\hat{3}} \hat{1} \hat{2} }_{\;\! \textcolor{Maroon}{(2)} \textcolor{PineGreen}{\hat{1} \hat{2}} } \mathcolor{gray}{\iiint} \xint{\mathcolor{gray}{-}}{18}{M}^{\;\! \mathcolor{gray}{3} \hat{1} \hat{2} }_{\;\! \hat{3} \textcolor{Maroon}{(2)} } \left( \mathcolor{gray}{\bar{q}} \right) \mathcolor{gray}{\iint} \xint{{}^{}_{\mathcolor{gray}{-}}}{10}{g}^{\;\! \textcolor{PineGreen}{\hat{1}}}_{\;\! \hat{1} \mathcolor{gray}{0}} \left( \mathcolor{gray}{\bar{k}_{1\symup{\rho}}} \right) \xint{{}^{}_{\mathcolor{gray}{-}}}{10}{g}^{\;\! \textcolor{PineGreen}{\hat{2}}}_{\;\! \hat{2} \mathcolor{gray}{0}} \left( \mathcolor{gray}{\bar{k}_{2\symup{\rho}}} \right) \text{since} \left( \frac{ \Delta \xint{\begin{smallmatrix} ~ \\ {}^{}_{\mathcolor{gray}{-}} \\ ~ \end{smallmatrix}}{15}{k}_{\symup{z}}^{\;\! \textcolor{PineGreen}{\hat{1} \hat{2} \hat{3}} } \mathcolor{gray}{z} }{ 2 } \right) \mathcolor{gray}{z} ~ \mathbb{d} \mathcolor{gray}{\bar{k}_{1\symup{\rho}}} \mathbb{d} \mathcolor{gray}{\bar{q}} ~, \label{eq:up-scalar-g-EE-312-discrete-since2} \\
	&\hspace{8em} \text{where} ~~~~~~ \text{since} \left( x \right) := \text{sinc} \left( x \right) \mathbb{e}^{\mathbb{i} x } ~. \label{eq:up-scalar-g-EE-312-discrete-since3} \\
	&\hspace{8em} \text{in which} ~~~~~~ \text{sinc} \left( x \right) = \text{sin} \left( x \right) \big/ x ~. \label{eq:up-scalar-g-EE-312-discrete-since4}
\end{align}
\end{subequations}

从 \bref{eq:up-scalar-g-EE-312-discrete-convolution2} 中可看出,该 \bref{ssec:undepleted-pump-approximation} 的解,在形式上均为\textcolor{Plum}{前向解}。实际上,由于\textcolor{PineGreen}{模式} $\textcolor{PineGreen}{\hat{1}}, \textcolor{PineGreen}{\hat{2}},\textcolor{PineGreen}{\hat{3}}$ 均未确定(见 \bref{ssec:eigenmodes-compamp}),因此该小节也包含了\textcolor{Plum}{反向传播解}(\textcolor{NavyBlue}{驱动源}或生成的电磁场中\textcolor{gray}{任何组分},都可能是\textcolor{Plum}{反向传播}的),以及所有可能的\textcolor{Plum}{反向传播}\textcolor{PineGreen}{模式}的组合$\left\{ \textcolor{PineGreen}{\hat{1}}, \textcolor{PineGreen}{\hat{2}},\textcolor{PineGreen}{\hat{3}} \right\}$。除此之外,也存在\textcolor{Plum}{复共轭解}(表面上对\textcolor{Plum}{前向解}两侧\textcolor{Plum}{取复共轭};本质上需对\textcolor{Maroon}{正空间}波动方程\textcolor{Plum}{取复共轭}),留给读者自行尝试。

\vspace*{-1.0em}

\marginLeft[-2.4em]{sec:down_convert}\section{\textcolor{Maroon}{Down conversion} 下转换 - 电场本征复振幅 \textcolor{Maroon}{equation}}\label{sec:down_convert}

\bref{eq:scalar-g-modulus-P-chieff-spectrum} 即为\textcolor{Plum}{各向异性}材料中,\textcolor{gray}{光波段单色 $\omega$} \textcolor{Maroon}{傅立叶}标量\textcolor{NavyBlue}{脉冲光}\textcolor{Maroon}{倍频},或\textcolor{Maroon}{光整流}后续级联\textcolor{Maroon}{电光效应}的,标量\textcolor{Maroon}{时空谱}耦合波方程。其配合 \bref{eq:chieff-spectrum} 和 \textcolor{PineGreen}{本征偏振态} $\xint{{}^{}_{\mathcolor{gray}{-}}}{10}{\hat{g}}^{\;\!\mathcolor{gray}{\omega} \textcolor{PineGreen}{\hat{3}}}_{\;\! \textcolor{Maroon}{\Yup}}$ 即变成右侧标量、左侧矢量的\text{\textbf{标量\textcolor{Plum}{非线性}\textcolor{NavyBlue}{波源}}条件(\textcolor{NavyBlue}{脉冲})}条件(见 \bref{eq:simplify8-scalar-g-conjugate} 下方的 \byperref{waveq-scalar-2-vector}{段落})下的标量\textcolor{Maroon}{时空谱}耦合波方程。若波动方程进一步回退到 \bref{eq:simplify8-scalar-g-modulus-P-spectrum}(特别是分子回溯至 \bref{eq:DP^(2)-3_12-spectrum-G2}),则进化至不受\textbf{\textcolor{NavyBlue}{混频源}\textcolor{PineGreen}{本征偏振态}固定(非场)}的近似条件(\bref{eq:scalar_nonlinear_drive})约束的最广义情形。

以相干\textcolor{NavyBlue}{脉冲光连续谱}为例(\textcolor{Plum}{离散}\textcolor{gray}{波长}的情况类似),上一段以及上一节,所考察的全都是\textcolor{Maroon}{上转换}过程,忽略了二阶\textcolor{Plum}{非线性}\textcolor{gray}{频率转换} $=$ \textcolor{Maroon}{三波混频}过程中的另一个重要方面,即\textcolor{Maroon}{下转换}过程、\textcolor{NavyBlue}{能量回流}效应、\textcolor{NavyBlue}{效率饱和}问题。

对于\textcolor{NavyBlue}{脉冲光}\textcolor{Maroon}{倍频}过程,(其逆过程)还存在\textcolor{NavyBlue}{倍频光}能量回流到\textcolor{NavyBlue}{基波}的\textcolor{Maroon}{下转换}通道(尚未涉及)。对于 \textcolor{Maroon}{THz} 波段 $\textcolor{gray}{\omega}$ 的\textcolor{Maroon}{电光效应}过程,(其逆过程)还有 1 个\textcolor{NavyBlue}{脉冲}\textcolor{Maroon}{光整流}(“反/逆电光效应”)标/矢量\textcolor{Maroon}{时空谱}耦合波方程(尚未处理)。

前者\textcolor{NavyBlue}{脉冲光}\textcolor{Maroon}{倍频}与其缺失的\textcolor{Maroon}{下转换}过程一起,后者\textcolor{Maroon}{电光效应}与其缺失的\textcolor{Maroon}{光整流}过程一起;二者共同构成\textcolor{NavyBlue}{脉冲光}\textcolor{gray}{谱内}(\textcolor{gray}{自})\textcolor{gray}{混频}的标/矢量\textcolor{Maroon}{时空谱}耦合波方程组,以完整描述单个\textcolor{NavyBlue}{脉冲光}\textcolor{gray}{谱内混频}\textcolor{Maroon}{上转换}出另一个\textcolor{NavyBlue}{光脉冲},或\textcolor{Maroon}{下转换}出另一个\textcolor{Maroon}{THz} \textcolor{NavyBlue}{脉冲}(的过程)。

\marginLeft[-2.4em]{ssec:cross-correlation}\subsection{上转换 $\to$ 下转换,卷积 $\to$ 互相关}\label{ssec:cross-correlation}

所有的\textcolor{Maroon}{上转换}过程,涉及\textcolor{gray}{傅立叶域}的\textcolor{Plum}{卷积}。所有的\textcolor{Maroon}{下转换}过程,涉及\textcolor{gray}{傅立叶域}的\textcolor{Plum}{互相关}。因此在起笔\textcolor{Maroon}{下转换}过程,以及\textcolor{Maroon}{三波混频}耦合波方程组之前,需要针对\textcolor{Plum}{互相关},引入\textcolor{Plum}{数学铺垫}。首先,可以验证,下述 2 条纯粹的\textcolor{Plum}{数学关系}成立:
\begin{subequations} \label{eq:F[B*]F[AB*]}
\begin{align}
	\hspace{-1.3em} \mathcolor{gray}{\mathcal F} \left[ B_{\;\! \mathcolor{gray}{z}}^{\textcolor{Plum}{*}} \right] &= \left\{ \mathcolor{gray}{\mathcal F} \left[ B_{\;\! \mathcolor{gray}{z}} \right] \mathcolor{gray}{\Big|}_{\mathcolor{gray}{- \bar{k}_{\symup{\rho}}}} \right\}^{\textcolor{Plum}{*}} = \mathcolor{gray}{\mathcal F}^{\textcolor{Plum}{*}} \left[ B_{\;\! \mathcolor{gray}{z}} \right] \mathcolor{gray}{\Big|}_{\mathcolor{gray}{- \bar{k}_{\symup{\rho}}}}
	= \xint{\mathcolor{gray}{-}}{18}{B}_{\;\! \mathcolor{gray}{z}}^{\textcolor{Plum}{*}} \left( \mathcolor{gray}{- \bar{k}_{\symup{\rho}}} \right) ~,  \label{eq:F[B*]} \\ 
	\hspace{-1.3em} \mathcolor{gray}{\mathcal F} \left[ A_{\;\! \mathcolor{gray}{z}} \cdot B_{\;\! \mathcolor{gray}{z}}^{\textcolor{Plum}{*}} \right] &= \mathcolor{gray}{\mathcal F} \left[ A_{\;\! \mathcolor{gray}{z}} \right] \textcolor{gray}{*}~ \mathcolor{gray}{\mathcal F} \left[ B_{\;\! \mathcolor{gray}{z}}^{\textcolor{Plum}{*}} \right] = \mathcolor{gray}{\mathcal F}^{\textcolor{Plum}{*}} \left[ B_{\;\! \mathcolor{gray}{z}} \right] \mathcolor{gray}{\Big|}_{\mathcolor{gray}{- \bar{k}_{\symup{\rho}}}} \textcolor{gray}{*}~ \mathcolor{gray}{\mathcal F} \left[ A_{\;\! \mathcolor{gray}{z}} \right] =: \mathcolor{gray}{\mathcal F} \left[ B_{\;\! \mathcolor{gray}{z}} \right] \textcolor{gray}{\circ}~ \mathcolor{gray}{\mathcal F} \left[ A_{\;\! \mathcolor{gray}{z}} \right] \label{eq:F[AB*]} \\
	&\mathllap{= \xint{\mathcolor{gray}{-}}{18}{A}_{\;\! \mathcolor{gray}{z}}} ~\textcolor{gray}{*}~ \mathcolor{gray}{\mathcal F} \left[ B_{\;\! \mathcolor{gray}{z}}^{\textcolor{Plum}{*}} \right] \xrightarrow[]{\text{\bref{eq:F[B*]}}} \xint{\mathcolor{gray}{-}}{18}{B}_{\;\! \mathcolor{gray}{z}}^{\textcolor{Plum}{*}} \left( \mathcolor{gray}{- \bar{k}_{\symup{\rho}}} \right) \textcolor{gray}{*}~ \xint{\mathcolor{gray}{-}}{18}{A}_{\;\! \mathcolor{gray}{z}} \xrightarrow[]{\text{\bref{eq:AcircB-a}}} \xint{\mathcolor{gray}{-}}{18}{B}_{\;\! \mathcolor{gray}{z}} ~\textcolor{gray}{\circ}~ \xint{\mathcolor{gray}{-}}{18}{A}_{\;\! \mathcolor{gray}{z}} ~,
\end{align}
\end{subequations}
其中,以 $\mathcolor{gray}{\bar{k}_{\symup{\rho}}}$ 域为例,定义了\textcolor{Plum}{互相关} `$\textcolor{gray}{\circ}$' 这个二元/双目算符
\begin{subequations} \label{eq:AcircB}
\begin{align}
	\xint{\mathcolor{gray}{-}}{18}{A}_{\;\! \mathcolor{gray}{z}} ~\textcolor{gray}{\circ}~ \xint{\mathcolor{gray}{-}}{18}{B}_{\;\! \mathcolor{gray}{z}} := \xint{\mathcolor{gray}{-}}{18}{A}_{\;\! \mathcolor{gray}{z}}^{\textcolor{Plum}{*}} \left( \mathcolor{gray}{- \bar{k}_{\symup{\rho}}} \right) \textcolor{gray}{*}~ \xint{\mathcolor{gray}{-}}{18}{B}_{\;\! \mathcolor{gray}{z}} &= \mathcolor{gray}{\iint_{-\infty}^{+\infty}} \xint{\mathcolor{gray}{-}}{18}{A}_{\;\! \mathcolor{gray}{z}}^{\textcolor{Plum}{*}} \left( \mathcolor{gray}{\bar{k}'_{\symup{\rho}}} \mathcolor{gray}{- \bar{k}_{\symup{\rho}}} \right) \cdot \xint{\mathcolor{gray}{-}}{18}{B}_{\;\! \mathcolor{gray}{z}} \left( \mathcolor{gray}{\bar{k}'_{\symup{\rho}}} \right) \mathbb{d} \mathcolor{gray}{\bar{k}'_{\symup{\rho}}} \label{eq:AcircB-a} \\ &= \mathcolor{gray}{\iint_{-\infty}^{+\infty}} \xint{\mathcolor{gray}{-}}{18}{A}_{\;\! \mathcolor{gray}{z}}^{\textcolor{Plum}{*}} \left( \mathcolor{gray}{\bar{k}'_{\symup{\rho}}} \right) \cdot \xint{\mathcolor{gray}{-}}{18}{B}_{\;\! \mathcolor{gray}{z}} \left( \mathcolor{gray}{\bar{k}'_{\symup{\rho}}} + \mathcolor{gray}{\bar{k}_{\symup{\rho}}} \right) \mathbb{d} \mathcolor{gray}{\bar{k}'_{\symup{\rho}}} \label{eq:AcircB-b}~, 
\end{align}
\end{subequations}
不像\textcolor{Plum}{卷积},\textcolor{Plum}{互相关}不满足\textcolor{Plum}{交换律}:
\begin{subequations} \label{eq:AcircB=BcircA*(-k_rho)}
\begin{align}
	\xint{\mathcolor{gray}{-}}{18}{A}_{\;\! \mathcolor{gray}{z}} ~\textcolor{gray}{\circ}~ \xint{\mathcolor{gray}{-}}{18}{B}_{\;\! \mathcolor{gray}{z}} &\xrightarrow[]{\text{\bref{eq:AcircB-a}}} \xint{\mathcolor{gray}{-}}{18}{A}_{\;\! \mathcolor{gray}{z}}^{\textcolor{Plum}{*}} \left( \mathcolor{gray}{- \bar{k}_{\symup{\rho}}} \right) \textcolor{gray}{*}~ \xint{\mathcolor{gray}{-}}{18}{B}_{\;\! \mathcolor{gray}{z}} \label{eq:AcircB=BcircA*(-k_rho)-a} \\ &\xrightarrow[]{\left[ \left[ \cdot \right]^{\textcolor{Plum}{*}} \left( \mathcolor{gray}{- \bar{k}_{\symup{\rho}}} \right) \right]^{\textcolor{Plum}{*}} \left( \mathcolor{gray}{- \bar{k}_{\symup{\rho}}} \right)} \left[ \xint{\mathcolor{gray}{-}}{18}{A}_{\;\! \mathcolor{gray}{z}} ~\textcolor{gray}{*}~ \xint{\mathcolor{gray}{-}}{18}{B}_{\;\! \mathcolor{gray}{z}}^{\textcolor{Plum}{*}} \left( \mathcolor{gray}{- \bar{k}_{\symup{\rho}}} \right) \right]^{\textcolor{Plum}{*}} \left( \mathcolor{gray}{- \bar{k}_{\symup{\rho}}} \right) \label{eq:AcircB=BcircA*(-k_rho)-b} \\ &= \left[ \xint{\mathcolor{gray}{-}}{18}{B}_{\;\! \mathcolor{gray}{z}}^{\textcolor{Plum}{*}} \left( \mathcolor{gray}{- \bar{k}_{\symup{\rho}}} \right) \textcolor{gray}{*}~ \xint{\mathcolor{gray}{-}}{18}{A}_{\;\! \mathcolor{gray}{z}} \right]^{\textcolor{Plum}{*}} \left( \mathcolor{gray}{- \bar{k}_{\symup{\rho}}} \right) \xrightarrow[]{\text{\bref{eq:AcircB-a}}} \left[ \xint{\mathcolor{gray}{-}}{18}{B}_{\;\! \mathcolor{gray}{z}} ~\textcolor{gray}{\circ}~ \xint{\mathcolor{gray}{-}}{18}{A}_{\;\! \mathcolor{gray}{z}} \right]^{\textcolor{Plum}{*}} \left( \mathcolor{gray}{- \bar{k}_{\symup{\rho}}} \right) \label{eq:AcircB=BcircA*(-k_rho)-c}~, 
\end{align}
\end{subequations}
\textcolor{Plum}{互相关}也不满足\textcolor{Plum}{结合律}
\begin{subequations} \label{eq:(AcircB)circC!=Acirc(BcircC)}
\begin{align}
	\left( \xint{\mathcolor{gray}{-}}{18}{A}_{\;\! \mathcolor{gray}{z}} ~\textcolor{gray}{\circ}~ \xint{\mathcolor{gray}{-}}{18}{B}_{\;\! \mathcolor{gray}{z}} \right) \textcolor{gray}{\circ}~ \xint{\mathcolor{gray}{-}}{18}{C}_{\;\! \mathcolor{gray}{z}} &\xrightarrow[]{\text{\bref{eq:AcircB-a}}} \left( \xint{\mathcolor{gray}{-}}{18}{A}_{\;\! \mathcolor{gray}{z}} ~\textcolor{gray}{\circ}~ \xint{\mathcolor{gray}{-}}{18}{B}_{\;\! \mathcolor{gray}{z}} \right)^{\textcolor{Plum}{*}} \left( \mathcolor{gray}{- \bar{k}_{\symup{\rho}}} \right) \textcolor{gray}{*}~ \xint{\mathcolor{gray}{-}}{18}{C}_{\;\! \mathcolor{gray}{z}} \label{eq:(AcircB)circC!=Acirc(BcircC)a} \\ 
	&\xrightarrow[]{\text{\bref{eq:AcircB=BcircA*(-k_rho)-c}}} \xint{\mathcolor{gray}{-}}{18}{B}_{\;\! \mathcolor{gray}{z}} ~\textcolor{gray}{\circ}~ \xint{\mathcolor{gray}{-}}{18}{A}_{\;\! \mathcolor{gray}{z}} ~\textcolor{gray}{*}~ \xint{\mathcolor{gray}{-}}{18}{C}_{\;\! \mathcolor{gray}{z}} \label{eq:(AcircB)circC!=Acirc(BcircC)b} \\
	&\xrightarrow[]{\text{\bref{eq:AcircB-a}}} \xint{\mathcolor{gray}{-}}{18}{B}_{\;\! \mathcolor{gray}{z}}^{\textcolor{Plum}{*}} \left( \mathcolor{gray}{- \bar{k}_{\symup{\rho}}} \right) \textcolor{gray}{*}~ \xint{\mathcolor{gray}{-}}{18}{A}_{\;\! \mathcolor{gray}{z}} ~\textcolor{gray}{*}~ \xint{\mathcolor{gray}{-}}{18}{C}_{\;\! \mathcolor{gray}{z}} \label{eq:(AcircB)circC!=Acirc(BcircC)c} \\ 
	&= \xint{\mathcolor{gray}{-}}{18}{A}_{\;\! \mathcolor{gray}{z}} ~\textcolor{gray}{*}~ \xint{\mathcolor{gray}{-}}{18}{B}_{\;\! \mathcolor{gray}{z}}^{\textcolor{Plum}{*}} \left( \mathcolor{gray}{- \bar{k}_{\symup{\rho}}} \right) \textcolor{gray}{*}~ \xint{\mathcolor{gray}{-}}{18}{C}_{\;\! \mathcolor{gray}{z}} \label{eq:(AcircB)circC!=Acirc(BcircC)d} \\ 
	&\mathllap{\neq \xint{\mathcolor{gray}{-}}{18}{A}_{\;\! \mathcolor{gray}{z}}^{\textcolor{Plum}{*}} \left( \mathcolor{gray}{- \bar{k}_{\symup{\rho}}} \right)} ~\textcolor{gray}{*}~ \xint{\mathcolor{gray}{-}}{18}{B}_{\;\! \mathcolor{gray}{z}}^{\textcolor{Plum}{*}} \left( \mathcolor{gray}{- \bar{k}_{\symup{\rho}}} \right) \textcolor{gray}{*}~ \xint{\mathcolor{gray}{-}}{18}{C}_{\;\! \mathcolor{gray}{z}} \xrightarrow[]{\text{\bref{eq:AcircB-a}}} \left( \xint{\mathcolor{gray}{-}}{18}{A}_{\;\! \mathcolor{gray}{z}} ~\textcolor{gray}{*}~ \xint{\mathcolor{gray}{-}}{18}{B}_{\;\! \mathcolor{gray}{z}} \right) \textcolor{gray}{\circ}~ \xint{\mathcolor{gray}{-}}{18}{C}_{\;\! \mathcolor{gray}{z}} \label{eq:(AcircB)circC!=Acirc(BcircC)e} \\
	&\xrightarrow[]{\text{\bref{eq:AcircB-a}}} \xint{\mathcolor{gray}{-}}{18}{A}_{\;\! \mathcolor{gray}{z}}^{\textcolor{Plum}{*}} \left( \mathcolor{gray}{- \bar{k}_{\symup{\rho}}} \right) \textcolor{gray}{*} \left( \xint{\mathcolor{gray}{-}}{18}{B}_{\;\! \mathcolor{gray}{z}} ~\textcolor{gray}{\circ}~ \xint{\mathcolor{gray}{-}}{18}{C}_{\;\! \mathcolor{gray}{z}} \right) \label{eq:(AcircB)circC!=Acirc(BcircC)f} \\
	&\xrightarrow[]{\text{\bref{eq:AcircB-a}}} \xint{\mathcolor{gray}{-}}{18}{A}_{\;\! \mathcolor{gray}{z}} ~\textcolor{gray}{\circ} \left( \xint{\mathcolor{gray}{-}}{18}{B}_{\;\! \mathcolor{gray}{z}} ~\textcolor{gray}{\circ}~ \xint{\mathcolor{gray}{-}}{18}{C}_{\;\! \mathcolor{gray}{z}} \right) \label{eq:(AcircB)circC!=Acirc(BcircC)g}~. 
\end{align}
\end{subequations}

\marginLeft[-2.4em]{ssec:OR_spectrum+DFG_discrete}\subsection{下转换过程:脉冲光整流、连续光差频}\label{ssec:OR_spectrum+DFG_discrete}

考虑\textcolor{NavyBlue}{脉冲光}谱内(\textcolor{Maroon}{自})\textcolor{Maroon}{差频},即\textcolor{Maroon}{光整流}过程,若不纳入其后续级联的\textcolor{Maroon}{电光效应},则该二阶\textcolor{Plum}{非线性}过程的\textcolor{gray}{频率}\textcolor{Maroon}{守恒}方程\Footnote{也可写成\textcolor{Plum}{加}/\textcolor{Plum}{和}的形式 $\left( \textcolor{gray}{\omega'} + \textcolor{gray}{\omega} \right) - \textcolor{gray}{\omega'} \to \textcolor{gray}{\omega} > \textcolor{gray}{0}$,但有时\textcolor{Plum}{减}/\textcolor{Plum}{差}更贴近\textcolor{Plum}{卷积}或\textcolor{Plum}{相关运算}的数学定义。}为 $ \textcolor{gray}{\omega'} - \left( \textcolor{gray}{\omega'}-\textcolor{gray}{\omega} \right) \to \textcolor{gray}{\omega} > \textcolor{gray}{0}$;对于该\textcolor{Maroon}{下转换}过程,波动方程 \bref{eq:simplify8-scalar-g-modulus} 右侧\textcolor{Plum}{非线性}\textcolor{NavyBlue}{波源}项 $\xint{\mathcolor{gray}{-}}{25}{\bar{P}}^{\;\! \mathcolor{gray}{\omega} \textcolor{PineGreen}{\hat{1}} }_{\;\! \mathcolor{gray}{z}  \textcolor{Maroon}{(2)}} = \mathcolor{gray}{\mathcal F} \left[ {\bar{P}}^{\;\! \mathcolor{gray}{\omega} \textcolor{PineGreen}{\hat{1}} }_{\;\! \mathcolor{gray}{z}  \textcolor{Maroon}{(2)}} \right]$ 变为
\begin{subequations} \label{eq:DP^(2)-1_32-spectrum-DFG}
\begin{align}
	\xint{\mathcolor{gray}{-}}{30}{P}^{\;\! \textcolor{PineGreen}{\hat{1}} \mathcolor{gray}{\omega} }_{\;\! \hat{1}\mathcolor{gray}{z} \textcolor{Maroon}{(2)} } &\xrightarrow[]{\text{$\sim$\bref{eq:D_wkrho}}} \mathcolor{gray}{\mathcal F} \left[ {\chi}^{\;\! \textcolor{PineGreen}{\hat{1}} \mathcolor{gray}{\omega} \hat{3} \hat{2} \textcolor{Plum}{*} }_{\;\! \hat{1} \mathcolor{gray}{z} \textcolor{PineGreen}{\hat{3} \hat{2}} \textcolor{Maroon}{(2)}} \right] \mathcolor{gray}{*} \mathcolor{gray}{\mathcal F} \left[ E^{\;\!\textcolor{PineGreen}{\hat{3}} \mathcolor{gray}{\omega}}_{\;\! \hat{3} \mathcolor{gray}{z}} ~\mathcolor{gray}{\widetilde *}~ E^{\;\!\textcolor{PineGreen}{\hat{2}} \mathcolor{gray}{- \omega} \textcolor{Plum}{*}}_{\;\! \hat{2} \mathcolor{gray}{z}} \right] \label{eq:DP^(2)-1_32-spectrum-DFG1} \\
	&\xrightarrow[\text{$\sim$\bref{eq:components-chi2-modulate}}]{\text{\bref{eq:F[B*]}}} \mathcolor{gray}{\mathcal F} \left[ {\chi}^{\;\! \textcolor{PineGreen}{\hat{1}} \mathcolor{gray}{\omega} \hat{3} \hat{2} \textcolor{Plum}{*} }_{\;\! \hat{1} \textcolor{Maroon}{(2)} \textcolor{PineGreen}{\hat{3} \hat{2}}} {M}^{\;\! \mathcolor{gray}{\omega} \hat{3} \hat{2} \textcolor{Plum}{*} }_{\;\! \hat{1} \mathcolor{gray}{z} \textcolor{Maroon}{(2)} } \right] \mathcolor{gray}{*} \left[ \xint{\mathcolor{gray}{-}}{295}{E}^{\;\!\textcolor{PineGreen}{\hat{3}} \mathcolor{gray}{\omega}}_{\;\! \hat{3} \mathcolor{gray}{z}} ~\mathcolor{gray}{\widetilde \circledast}~ \xint{\mathcolor{gray}{-}}{295}{E}^{\;\!\textcolor{PineGreen}{\hat{2}} \mathcolor{gray}{- \omega} \textcolor{Plum}{*}}_{\;\! \hat{2} \mathcolor{gray}{z}} \left( \mathcolor{gray}{- \bar{k}_{\symup{\rho}}} \right) \right] \label{eq:DP^(2)-1_32-spectrum-DFG2} \\
	&\xrightarrow[]{\text{\bref{eq:AcircB-a}}} {\chi}^{\;\! \textcolor{PineGreen}{\hat{1}} \mathcolor{gray}{\omega} \hat{3} \hat{2} \textcolor{Plum}{*} }_{\;\! \hat{1} \textcolor{Maroon}{(2)} \textcolor{PineGreen}{\hat{3} \hat{2}}} ~\mathcolor{gray}{\mathcal F} \left[ M^{\;\! \mathcolor{gray}{\omega} \hat{3} \hat{2} \textcolor{Plum}{*} }_{\;\! \hat{1} \mathcolor{gray}{z} \textcolor{Maroon}{(2)} } \right] \mathcolor{gray}{*} \left( \xint{\mathcolor{gray}{-}}{295}{E}^{\;\!\textcolor{PineGreen}{\hat{2}} \mathcolor{gray}{\omega} }_{\;\! \hat{2} \mathcolor{gray}{z}} ~\mathcolor{gray}{\widetilde \circledcirc}~ \xint{\mathcolor{gray}{-}}{295}{E}^{\;\!\textcolor{PineGreen}{\hat{3}} \mathcolor{gray}{\omega}}_{\;\! \hat{3} \mathcolor{gray}{z}} \right) \label{eq:DP^(2)-1_32-spectrum-DFG3} \\
	&\xrightarrow[]{\text{\bref{eq:IFT-z}}} {\chi}^{\;\! \textcolor{PineGreen}{\hat{1}} \mathcolor{gray}{\omega} \hat{3} \hat{2} \textcolor{Plum}{*} }_{\;\! \hat{1} \textcolor{Maroon}{(2)} \textcolor{PineGreen}{\hat{3} \hat{2}}} ~\mathcolor{gray}{\mathcal F_{z}^{-1}} \left[ \mathcolor{gray}{\mathcal F_{\bar{k}}} \left[ M^{\;\! \mathcolor{gray}{\omega} \hat{3} \hat{2} \textcolor{Plum}{*} }_{\;\! \hat{1} \mathcolor{gray}{z} \textcolor{Maroon}{(2)} } \right] \right] \mathcolor{gray}{*} \left( \xint{\mathcolor{gray}{-}}{295}{E}^{\;\!\textcolor{PineGreen}{\hat{2}} \mathcolor{gray}{\omega} }_{\;\! \hat{2} \mathcolor{gray}{z}} ~\mathcolor{gray}{\widetilde \circledcirc}~ \xint{\mathcolor{gray}{-}}{295}{E}^{\;\!\textcolor{PineGreen}{\hat{3}} \mathcolor{gray}{\omega}}_{\;\! \hat{3} \mathcolor{gray}{z}} \right) \label{eq:DP^(2)-1_32-spectrum-DFG4} \\
	&= {\chi}^{\;\! \textcolor{PineGreen}{\hat{1}} \mathcolor{gray}{\omega} \hat{3} \hat{2} \textcolor{Plum}{*} }_{\;\! \hat{1} \textcolor{Maroon}{(2)} \textcolor{PineGreen}{\hat{3} \hat{2}}} ~\mathcolor{gray}{\mathcal F_{z}^{-1}} \left[ \mathcolor{gray}{\mathcal F_{\bar{k}}} \left[ M^{\;\! \mathcolor{gray}{\omega} \hat{3} \hat{2} \textcolor{Plum}{*} }_{\;\! \hat{1} \mathcolor{gray}{z} \textcolor{Maroon}{(2)} } \right] \mathcolor{gray}{*} \left( \xint{\mathcolor{gray}{-}}{295}{E}^{\;\!\textcolor{PineGreen}{\hat{2}} \mathcolor{gray}{\omega} }_{\;\! \hat{2} \mathcolor{gray}{z}} ~\mathcolor{gray}{\widetilde \circledcirc}~ \xint{\mathcolor{gray}{-}}{295}{E}^{\;\!\textcolor{PineGreen}{\hat{3}} \mathcolor{gray}{\omega}}_{\;\! \hat{3} \mathcolor{gray}{z}} \right) \right] \label{eq:DP^(2)-1_32-spectrum-DFG5} \\
	&\xrightarrow[]{\text{\bref{eq:F[B*]}}} {\chi}^{\;\! \textcolor{PineGreen}{\hat{1}} \mathcolor{gray}{\omega} \hat{3} \hat{2} \textcolor{Plum}{*} }_{\;\! \hat{1} \textcolor{Maroon}{(2)} \textcolor{PineGreen}{\hat{3} \hat{2}}} ~\mathcolor{gray}{\mathcal F_{z}^{-1}} \left[ \xint{\mathcolor{gray}{-}}{18}{M}^{\;\! \mathcolor{gray}{\omega} \hat{3} \hat{2} \textcolor{Plum}{*} }_{\;\! \hat{1} \mathcolor{gray}{- k_{\symup{z}}} \textcolor{Maroon}{(2)} } \left( \mathcolor{gray}{- \bar{k}_{\symup{\rho}}} \right) \mathcolor{gray}{*} \left( \xint{\mathcolor{gray}{-}}{295}{E}^{\;\!\textcolor{PineGreen}{\hat{2}} \mathcolor{gray}{\omega} }_{\;\! \hat{2} \mathcolor{gray}{z}} ~\mathcolor{gray}{\widetilde \circledcirc}~ \xint{\mathcolor{gray}{-}}{295}{E}^{\;\!\textcolor{PineGreen}{\hat{3}} \mathcolor{gray}{\omega}}_{\;\! \hat{3} \mathcolor{gray}{z}} \right) \right] \label{eq:DP^(2)-1_32-spectrum-DFG6} \\
	&\xrightarrow[\text{\bref{eq:IFT*-z}}]{\text{\bref{eq:AcircB-a}}} {\chi}^{\;\! \textcolor{PineGreen}{\hat{1}} \mathcolor{gray}{\omega} \hat{3} \hat{2} \textcolor{Plum}{*} }_{\;\! \hat{1} \textcolor{Maroon}{(2)} \textcolor{PineGreen}{\hat{3} \hat{2}}} ~\mathcolor{gray}{\mathcal F_{z}^{-\textcolor{Plum}{*}}} \left[ \xint{\mathcolor{gray}{-}}{18}{M}^{\;\! \mathcolor{gray}{\omega} \hat{3} \hat{2} }_{\;\! \hat{1} \mathcolor{gray}{k_{\symup{z}}} \textcolor{Maroon}{(2)} } \mathcolor{gray}{\circ} \left( \xint{\mathcolor{gray}{-}}{295}{E}^{\;\!\textcolor{PineGreen}{\hat{2}} \mathcolor{gray}{\omega} }_{\;\! \hat{2} \mathcolor{gray}{z}} ~\mathcolor{gray}{\widetilde \circledcirc}~ \xint{\mathcolor{gray}{-}}{295}{E}^{\;\!\textcolor{PineGreen}{\hat{3}} \mathcolor{gray}{\omega}}_{\;\! \hat{3} \mathcolor{gray}{z}} \right) \right] ~, \label{eq:DP^(2)-1_32-spectrum-DFG7}
\end{align}
\end{subequations}
其中,类似 $"\mathcolor{gray}{*}"$ 之于 $"\mathcolor{gray}{\widetilde \circledast}"$ 地,在 \bref{eq:DP^(2)-1_32-spectrum-DFG3} 中定义了 $\mathcolor{gray}{\bar{k}_{\symup{\rho}}}$ 域\textcolor{Plum}{互相关}算符 $"\mathcolor{gray}{\circ}"$ 所对应的 $\mathcolor{gray}{\omega}, \mathcolor{gray}{\bar{k}_{\symup{\rho}}}$ 域的\textcolor{Plum}{互相关}算符 $"\mathcolor{gray}{\widetilde \circledcirc}"$;为了省略对\textcolor{Plum}{互相关}运算及其\textcolor{NavyBlue}{对象}的\textcolor{Plum}{括号},规定 ``$\mathcolor{gray}{*},\mathcolor{gray}{\circ}$'' 二者地位平等,且 ``$\mathcolor{gray}{\widetilde \circledast},\mathcolor{gray}{\widetilde \circledcirc}$'' 二者也地位平等,对应\textbf{须遵从类似 \byperref{OperatorSequence}{前处} 的\textcolor{Plum}{积分顺序}:“$\mathcolor{gray}{\widetilde \circledcirc}$” 的 $\mathcolor{gray}{\bar{k}_{\symup{\rho}}}$ 域 $\to$ $\mathcolor{gray}{\bar{k}_{\symup{\rho}}}$ 域的 “$\mathcolor{gray}{\circ}$” $\to$ $\mathcolor{gray}{k_{\symup{z}}}$ 域的 $\mathcolor{gray}{\mathcal F^{-1}_z} \left[ \cdot \right]$ $\to$ “$\mathcolor{gray}{\widetilde \circledcirc}$” 的 $\mathcolor{gray}{\omega}$ 域(即 $\mathcolor{gray}{\omega}$ 域的 ``~$\mathcolor{gray}{\widetilde \circ}$~'')}。这使得有些括号可以省略,如 \bref{eq:DP^(2)-1_32-spectrum-DFG7} 中的小括号,至 \bref{eq:DP^(2)-1_32-spectrum-G1}。

然而,有些\textcolor{Plum}{括号}不能省略,见 \bref{eq:(AcircB)circC!=Acirc(BcircC)e}。这是因为,当\textcolor{Plum}{互相关}与\textcolor{Plum}{卷积}同时出现时,对于同为 $\mathcolor{gray}{\bar{k}_{\symup{\rho}}}$ 域和/或 $\mathcolor{gray}{\omega}$ 域的、同层次的\textcolor{Plum}{互相关}和\textcolor{Plum}{卷积},不要求\textcolor{Plum}{互相关}的\textcolor{Plum}{优先级}高于\textcolor{Plum}{卷积}。注意,从 \bref{eq:DP^(2)-1_32-spectrum-DFG2} 到 \bref{eq:DP^(2)-1_32-spectrum-DFG3} 可以发现,\textcolor{Plum}{互相关}的\textbf{解读顺序},相对于\textcolor{Plum}{卷积}而言,是\textbf{相反的}。

之后会提到,\bref{eq:DP^(2)-1_32-spectrum-DFG7} 中的 $"\mathcolor{gray}{\widetilde \circledcirc}"$ 中的 $"~\mathcolor{gray}{\widetilde \circ}~"$ 和 $"\mathcolor{gray}{\circ}"$,二者均会用到 2 条\textcolor{Plum}{互相关}定义/运算规则 \bref{eq:AcircB-a} 和 \bref{eq:AcircB-b},并且在不同的情况下使用不同的规则,这一点对于理解下转换版本的\textcolor{Plum}{非线性}\textcolor{Plum}{卷积}过程的\textcolor{gray}{横向波矢守恒}、\textcolor{PineGreen}{纵向相位匹配}以及\textcolor{gray}{频率守恒方程}至关重要。

此外,\bref{eq:DP^(2)-1_32-spectrum-DFG7} 中定义了 $\mathcolor{gray}{\mathcal F_{z}^{-\mathcolor{Plum}{*}}}$ 的核函数 $\mathbb{e}^{-\mathbb{i}\mathcolor{gray}{k_{\symup{z}}} \mathcolor{gray}{z}}$ 共轭于 $\mathcolor{gray}{\mathcal F_{z}^{-1}}$ 的核函数 $\mathbb{e}^{\mathbb{i}\mathcolor{gray}{k_{\symup{z}}} \mathcolor{gray}{z}}$ 的空域 $\mathcolor{gray}{z} \in \mathcolor{gray}{\bar{\mathbb{R}}_{\textcolor{Plum}{1}}}$ 向 1 维\textcolor{Plum}{傅立叶正} $\mathcolor{gray}{\mathcal F_{z}^{\mathcolor{Plum}{*}}}$、\textcolor{Plum}{逆} $\mathcolor{gray}{\mathcal F_{z}^{-\mathcolor{Plum}{*}}}$ \textcolor{Plum}{变换对}:
\begin{subequations} \label{eq:FT*-z_kz}
\begin{align}
	\mathcolor{gray}{\mathcal F_{z}^{\mathcolor{Plum}{*}}} \left[ \cdot \right] &:= \frac{ 1 }{ 2\symup{\pi} } \mathcolor{gray}{\int_{-\infty}^{+\infty}} \cdot~ \mathbb{e}^{\mathbb{i}\mathcolor{gray}{k_{\symup{z}}} \mathcolor{gray}{z}} \hphantom{^-} \mathbb{d}\mathcolor{gray}{z} ~, \label{eq:FT*-kz} \\
	\mathcolor{gray}{\mathcal F_{z}^{-\mathcolor{Plum}{*}}} \left[ \cdot \right] &:= \hphantom{\frac{ 1 }{ 2\symup{\pi} }} \mathcolor{gray}{\int_{-\infty}^{+\infty}} \cdot~ \mathbb{e}^{-\mathbb{i}\mathcolor{gray}{k_{\symup{z}}} \mathcolor{gray}{z}} \mathbb{d}\mathcolor{gray}{k_{\symup{z}}} ~. \label{eq:IFT*-z}
\end{align}
\end{subequations}

将 \bref{eq:DP^(2)-1_32-spectrum-DFG7} 写成左侧\textcolor{Plum}{矢量}、右侧\textcolor{Plum}{半张量}的形式即
\begin{subequations} \label{eq:DP^(2)-1_32-spectrum-G}
\begin{align}
	\xint{\mathcolor{gray}{-}}{30}{\bar{P}}^{\;\! \mathcolor{gray}{\omega} \textcolor{PineGreen}{\hat{1}} }_{\;\! \mathcolor{gray}{z} \textcolor{Maroon}{(2)} } &\xrightarrow[]{\text{\bref{eq:DP^(2)-1_32-spectrum-DFG7}}} \bar{\chi}^{\;\! \textcolor{PineGreen}{\hat{1}} \mathcolor{gray}{\omega} \hat{3} \hat{2} \textcolor{Plum}{*} }_{\;\! \textcolor{Maroon}{(2)} \textcolor{PineGreen}{\hat{3} \hat{2}}} \odot \mathcolor{gray}{\mathcal F_{z}^{-\textcolor{Plum}{*}}} \left[ \xint{\mathcolor{gray}{-}}{18}{\bar{M}}^{\;\! \mathcolor{gray}{\omega} \hat{3} \hat{2} }_{\;\! \mathcolor{gray}{k_{\symup{z}}} \textcolor{Maroon}{(2)} } \mathcolor{gray}{\circ} \xint{\mathcolor{gray}{-}}{295}{E}^{\;\!\textcolor{PineGreen}{\hat{2}} \mathcolor{gray}{\omega} }_{\;\! \hat{2} \mathcolor{gray}{z}} ~\mathcolor{gray}{\widetilde \circledcirc}~ \xint{\mathcolor{gray}{-}}{295}{E}^{\;\!\textcolor{PineGreen}{\hat{3}} \mathcolor{gray}{\omega}}_{\;\! \hat{3} \mathcolor{gray}{z}} \right] \label{eq:DP^(2)-1_32-spectrum-G1} \\
	&\xrightarrow[]{\text{$\sim$ \bref{eq:components-eigenwave'}}} \bar{\chi}^{\;\! \textcolor{PineGreen}{\hat{1}} \mathcolor{gray}{\omega} \hat{3} \hat{2} \textcolor{Plum}{*} }_{\;\! \textcolor{Maroon}{(2)} \textcolor{PineGreen}{\hat{3} \hat{2}}} \odot \mathcolor{gray}{\mathcal F_{z}^{-\textcolor{Plum}{*}}} \left[ \xint{\mathcolor{gray}{-}}{18}{\bar{M}}^{\;\! \mathcolor{gray}{\omega} \hat{3} \hat{2} }_{\;\! \mathcolor{gray}{k_{\symup{z}}} \textcolor{Maroon}{(2)} } \mathcolor{gray}{\circ} \left( \xint{\mathcolor{gray}{-}}{20}{\mathtt{G}}^{\;\! \textcolor{PineGreen}{\hat{2}} \mathcolor{gray}{\omega} }_{\;\! \mathcolor{gray}{z}} \xint{{}^{}_{\mathcolor{gray}{-}}}{10}{\hat{g}}^{\;\! \textcolor{PineGreen}{\hat{2}} \mathcolor{gray}{\omega} }_{\;\! \hat{2}} \right) ~\mathcolor{gray}{\widetilde \circledcirc}~ \left( \xint{\mathcolor{gray}{-}}{20}{\mathtt{G}}^{\;\! \textcolor{PineGreen}{\hat{3}} \mathcolor{gray}{\omega} }_{\;\! \mathcolor{gray}{z}} \xint{{}^{}_{\mathcolor{gray}{-}}}{10}{\hat{g}}^{\;\! \textcolor{PineGreen}{\hat{3}} \mathcolor{gray}{\omega} }_{\;\! \hat{3}} \right) \right] \label{eq:DP^(2)-1_32-spectrum-G2} \\ 
	&\xrightarrow[]{\text{\bref{eq:scalar_nonlinear_drive-spectrum}}} \bar{\chi}^{\;\! \textcolor{Maroon}{(2)} \textcolor{PineGreen}{\hat{1}} \textcolor{Plum}{*} }_{\;\! \mathcolor{gray}{\omega} \hat{3} \hat{2} \textcolor{PineGreen}{\hat{3} \hat{2}} } ~ {\hat{g}}^{\;\! \mathcolor{gray}{\omega} }_{\;\! \hat{2} \textcolor{PineGreen}{\hat{2}}} ~\mathcolor{gray}{\widetilde \circ}~ {\hat{g}}^{\;\! \mathcolor{gray}{\omega} }_{\;\! \hat{3} \textcolor{PineGreen}{\hat{3}}} \odot \mathcolor{gray}{\mathcal F_{z}^{-\textcolor{Plum}{*}}} \left[ \xint{\mathcolor{gray}{-}}{18}{\bar{M}}^{\;\! \mathcolor{gray}{\omega} \hat{3} \hat{2} }_{\;\! \mathcolor{gray}{k_{\symup{z}}} \textcolor{Maroon}{(2)} } \mathcolor{gray}{\circ} \xint{\mathcolor{gray}{-}}{20}{\mathtt{G}}^{\;\! \textcolor{PineGreen}{\hat{2}} \mathcolor{gray}{\omega} }_{\;\! \mathcolor{gray}{z}} ~\mathcolor{gray}{\widetilde \circledcirc}~ \xint{\mathcolor{gray}{-}}{20}{\mathtt{G}}^{\;\! \textcolor{PineGreen}{\hat{3}} \mathcolor{gray}{\omega} }_{\;\! \mathcolor{gray}{z}} \right] \label{eq:DP^(2)-1_32-spectrum-G3} \\
	&\xrightarrow[]{\text{\bref{eq:scalar_chi2_modulation}}} \bar{\chi}^{\;\! \mathcolor{gray}{\omega} \textcolor{PineGreen}{\hat{1}} \hat{3} \hat{2} \textcolor{Plum}{*} }_{\;\! \textcolor{Maroon}{(2)} \textcolor{PineGreen}{\hat{3}} \textcolor{PineGreen}{\hat{2}} } ~ {\hat{g}}^{\;\! \mathcolor{gray}{\omega} }_{\;\! \hat{2} \textcolor{PineGreen}{\hat{2}}} ~\mathcolor{gray}{\widetilde \circ}~ {\hat{g}}^{\;\! \mathcolor{gray}{\omega} }_{\;\! \hat{3} \textcolor{PineGreen}{\hat{3}}} \odot \mathcolor{gray}{\mathcal F_{z}^{-\textcolor{Plum}{*}}} \left[ M^{\;\! \mathcolor{gray}{\omega} }_{\;\! \mathcolor{gray}{k_{\symup{z}}} \textcolor{Maroon}{(2)} } \mathcolor{gray}{\circ} \xint{\mathcolor{gray}{-}}{20}{\mathtt{G}}^{\;\! \textcolor{PineGreen}{\hat{2}} \mathcolor{gray}{\omega} }_{\;\! \mathcolor{gray}{z}} ~\mathcolor{gray}{\widetilde \circledcirc}~ \xint{\mathcolor{gray}{-}}{20}{\mathtt{G}}^{\;\! \textcolor{PineGreen}{\hat{3}} \mathcolor{gray}{\omega} }_{\;\! \mathcolor{gray}{z}} \right] ~, \label{eq:DP^(2)-1_32-spectrum-G4}
\end{align}
\end{subequations}
其中,在“\textbf{标量\textcolor{Plum}{非线性}\textcolor{NavyBlue}{波源}}”条件 \bref{eq:scalar_nonlinear_drive} 下,通过 \bref{eq:DP^(2)-1_32-spectrum-G3} 定义了以\textcolor{NavyBlue}{脉冲}\textcolor{Maroon}{光整流}、\textcolor{NavyBlue}{连续光}\textcolor{Maroon}{差频}为代表的\textcolor{Maroon}{下转换}过程的\textcolor{NavyBlue}{有效非线性系数}(三阶)张量(场)
\begin{subequations} \label{eq:chieff*}
\begin{align}
	\xint{{}^{}_{\mathcolor{gray}{-}}}{23}{\widetilde{\chi}}^{ \textcolor{Maroon}{(2)} \textcolor{PineGreen}{\hat{1}} \mathcolor{gray}{\omega} }_{ \textcolor{NavyBlue}{\text{eff}} \hat{3} \hat{2} \textcolor{PineGreen}{\hat{3} \hat{2}} } &\xrightarrow[\text{$\sim$ \bref{eq:simplify8-scalar-g-modulus}}]{\text{\bref{eq:DP^(2)-1_32-spectrum-G3}}} \xint{{}^{}_{\mathcolor{gray}{-}}}{10}{\hat{g}}^{\;\! \textcolor{PineGreen}{\hat{1}} \textcolor{Plum}{*}}_{\;\! \mathcolor{gray}{\omega}} \odot \bar{\chi}^{\;\! \textcolor{Maroon}{(2)} \textcolor{PineGreen}{\hat{1}} \textcolor{Plum}{*} }_{\;\! \mathcolor{gray}{\omega} \hat{3} \hat{2} \textcolor{PineGreen}{\hat{3} \hat{2}} } ~ {\hat{g}}^{\;\! \mathcolor{gray}{\omega} }_{\;\! \hat{2} \textcolor{PineGreen}{\hat{2}} } ~\mathcolor{gray}{\widetilde \circ}~ {\hat{g}}^{\;\! \mathcolor{gray}{\omega} }_{\;\! \hat{3} \textcolor{PineGreen}{\hat{3}} } ~, \label{eq:chieff*-spectrum} \\
	\xint{{}^{}_{\mathcolor{gray}{-}}}{23}{\bar{\chi}}^{ \textcolor{Maroon}{(2)} \textcolor{PineGreen}{\hat{1}} }_{\textcolor{NavyBlue}{\text{eff}} \hat{3} \textcolor{PineGreen}{\hat{3}} \hat{2} \textcolor{PineGreen}{\hat{2}} } &\xrightarrow[\text{$\sim$ \bref{eq:simplify8-scalar-g-modulus}}]{\text{$\sim$ \bref{eq:DP^(2)-1_32-spectrum-G3}}} \xint{{}^{}_{\mathcolor{gray}{-}}}{10}{\hat{g}}^{\;\! \textcolor{PineGreen}{\hat{1}} \textcolor{Plum}{*}}_{\;\! } \odot \bar{\chi}^{\;\! \textcolor{PineGreen}{\hat{1}} \textcolor{Maroon}{(2)} \textcolor{Plum}{*} }_{\;\! \hat{3} \textcolor{PineGreen}{\hat{3}} \hat{2} \textcolor{PineGreen}{\hat{2}}} ~ {\hat{g}}_{\;\! \hat{3} \textcolor{PineGreen}{\hat{3}}} ~ {\hat{g}}^{\;\! \textcolor{Plum}{*}}_{\;\! \hat{2} \textcolor{PineGreen}{\hat{2}} } ~, \label{eq:chieff*-discrete}
\end{align}
\end{subequations}
注意,其中的 $\chi$ 头上的波浪符号是\textbf{黑色的} `$\sim$' 而不是\textbf{\textcolor{gray}{灰色的}} `$\mathcolor{gray}{\sim}$',意味着 $\widetilde{\chi}$ 既是矢量,又参与 \textcolor{gray}{$\omega$ 域}的一维\textcolor{Plum}{卷积积分}。这源于,将 \bref{eq:DP^(2)-1_32-spectrum-G3} 代入 \bref{eq:simplify8-scalar-g-modulus},所得的以\textcolor{NavyBlue}{脉冲}\textcolor{Maroon}{光整流}、\textcolor{NavyBlue}{连续光}\textcolor{Maroon}{差频}为代表的\textcolor{Maroon}{下转换}电场\textcolor{PineGreen}{本征复振幅}方程
\begin{subequations} \label{eq:scalar-g-modulus-P-chieff*}
\begin{align}
	\mathcolor{gray}{\nabla_z} \xint{\begin{smallmatrix} ~ \\ {}^{}_{\mathcolor{gray}{-}} \\ ~ \end{smallmatrix}}{09}{\mathtt{g}}^{\;\!\mathcolor{gray}{\omega} \textcolor{PineGreen}{\hat{1}}}_{\;\! \mathcolor{gray}{z}} &\xrightarrow[\text{$\sim$ \bref{eq:simplify8-scalar-g-modulus}}]{\text{\bref{eq:DP^(2)-1_32-spectrum-G3}}} \mathbb{i} k_{\textcolor{Maroon}{\mathsf{o}} \mathcolor{gray}{\omega}}^{\;\! 2} \frac{ \xint{{}^{}_{\mathcolor{gray}{-}}}{23}{\widetilde{\chi}}^{ \textcolor{Maroon}{(2)} \textcolor{PineGreen}{\hat{1}} \mathcolor{gray}{\omega} \mathsf{\textcolor{Plum}{T}} }_{ \textcolor{NavyBlue}{\text{eff}} \hat{3} \hat{2} \textcolor{PineGreen}{\hat{3} \hat{2}} } \cdot \mathcolor{gray}{\mathcal F_{z}^{-\textcolor{Plum}{*}}} \left[ \xint{\mathcolor{gray}{-}}{18}{\bar{M}}^{\;\! \mathcolor{gray}{\omega} \hat{3} \hat{2} }_{\;\! \mathcolor{gray}{k_{\symup{z}}} \textcolor{Maroon}{(2)} } \mathcolor{gray}{\circ} \xint{\mathcolor{gray}{-}}{15}{\mathtt{G}}^{\;\! \textcolor{PineGreen}{\hat{2}} \mathcolor{gray}{\omega} }_{\;\! \mathcolor{gray}{z}} ~\mathcolor{gray}{\widetilde \circledcirc}~ \xint{\mathcolor{gray}{-}}{15}{\mathtt{G}}^{\;\! \textcolor{PineGreen}{\hat{3}} \mathcolor{gray}{\omega} }_{\;\! \mathcolor{gray}{z}} \right] }{ 2 \lvert \xint{{}^{}_{\mathcolor{gray}{-}}}{10}{\hat{g}}^{\;\! \textcolor{PineGreen}{\hat{1}}}_{\;\! \mathcolor{gray}{\omega}} \rvert^2 \xint{\begin{smallmatrix} ~ \\ {}^{}_{\mathcolor{gray}{-}} \\ ~ \end{smallmatrix}}{15}{k}_{\;\! \symup{z}}^{\;\! \mathcolor{gray}{\omega} \textcolor{PineGreen}{\hat{1}}} \mathbb{e}^{\mathbb{i} \xint{\begin{smallmatrix} ~ \\ {}^{}_{\mathcolor{gray}{-}} \\ ~ \end{smallmatrix}}{15}{k}_{\symup{z}}^{\;\! \mathcolor{gray}{\omega} \textcolor{PineGreen}{\hat{1}}} \mathcolor{gray}{z}}} ~, \label{eq:scalar-g-modulus-P-chieff*-spectrum} \\
	\mathcolor{gray}{\nabla_z} \xint{\begin{smallmatrix} ~ \\ {}^{}_{\mathcolor{gray}{-}} \\ ~ \end{smallmatrix}}{09}{\mathtt{g}}^{\;\! \textcolor{PineGreen}{\hat{1}}}_{\;\! \mathcolor{gray}{z}} &\xrightarrow[\text{$\sim$ \bref{eq:simplify8-scalar-g-modulus}}]{\text{$\sim$ \bref{eq:DP^(2)-1_32-spectrum-G3}}} \mathbb{i} k_{\textcolor{Maroon}{\mathsf{o}} \mathcolor{gray}{1}}^{\;\! 2} \frac{ \xint{{}^{}_{\mathcolor{gray}{-}}}{23}{\bar{\chi}}^{ \textcolor{Maroon}{(2)} \textcolor{PineGreen}{\hat{1}}  \mathsf{\textcolor{Plum}{T}} }_{\textcolor{NavyBlue}{\text{eff}} \hat{3} \textcolor{PineGreen}{\hat{3}} \hat{2} \textcolor{PineGreen}{\hat{2}} } \cdot \mathcolor{gray}{\mathcal F_{z}^{-\textcolor{Plum}{*}}} \left[ \xint{\mathcolor{gray}{-}}{18}{\bar{M}}^{\;\! \mathcolor{gray}{1} \hat{3} \hat{2} }_{\;\! \mathcolor{gray}{k_{\symup{z}}} \textcolor{Maroon}{(2)} } \mathcolor{gray}{\circ} \left( \xint{\mathcolor{gray}{-}}{15}{\mathtt{G}}^{\;\! \textcolor{PineGreen}{\hat{2}} }_{\;\! \mathcolor{gray}{z}} \mathcolor{gray}{\circ} \xint{\mathcolor{gray}{-}}{15}{\mathtt{G}}^{\;\! \textcolor{PineGreen}{\hat{3}} }_{\;\! \mathcolor{gray}{z}} \right) \right]}{ 2 \lvert \xint{{}^{}_{\mathcolor{gray}{-}}}{10}{\hat{g}}^{\;\! \textcolor{PineGreen}{\hat{1}}} \rvert^2 \xint{\begin{smallmatrix} ~ \\ {}^{}_{\mathcolor{gray}{-}} \\ ~ \end{smallmatrix}}{15}{k}_{\;\! \symup{z}}^{\;\!  \textcolor{PineGreen}{\hat{1}}} \mathbb{e}^{\mathbb{i} \xint{\begin{smallmatrix} ~ \\ {}^{}_{\mathcolor{gray}{-}} \\ ~ \end{smallmatrix}}{15}{k}_{\symup{z}}^{\;\!  \textcolor{PineGreen}{\hat{1}}} \mathcolor{gray}{z}}} ~, \label{eq:scalar-g-modulus-P-chieff*-discrete}
\end{align}
\end{subequations}
注,\bref{eq:scalar-g-modulus-P-chieff*} 中含\textcolor{Plum}{逆变}矢量 $\xint{{}^{}_{\mathcolor{gray}{-}}}{23}{\widetilde{\chi}}^{ \textcolor{Maroon}{(2)} \textcolor{PineGreen}{\hat{1}} \mathcolor{gray}{\omega} \mathsf{\textcolor{Plum}{T}} }_{ \textcolor{NavyBlue}{\text{eff}} \hat{3} \hat{2} \textcolor{PineGreen}{\hat{3} \hat{2}} }, \xint{{}^{}_{\mathcolor{gray}{-}}}{23}{\bar{\chi}}^{ \textcolor{Maroon}{(2)} \textcolor{PineGreen}{\hat{1}}  \mathsf{\textcolor{Plum}{T}} }_{\textcolor{NavyBlue}{\text{eff}} \hat{3} \textcolor{PineGreen}{\hat{3}} \hat{2} \textcolor{PineGreen}{\hat{2}} }$ 与\textcolor{Plum}{协变}矢量 $\xint{\mathcolor{gray}{-}}{18}{\bar{M}}^{\;\! \mathcolor{gray}{\omega} \hat{3} \hat{2} }_{\;\! \mathcolor{gray}{k_{\symup{z}}} \textcolor{Maroon}{(2)} }, \xint{\mathcolor{gray}{-}}{18}{\bar{M}}^{\;\! \mathcolor{gray}{1} \hat{3} \hat{2} }_{\;\! \mathcolor{gray}{k_{\symup{z}}} \textcolor{Maroon}{(2)} }$ 的点积。

在“\textbf{标量\textcolor{Plum}{非线性}\textcolor{NavyBlue}{波源}}” \bref{eq:scalar_nonlinear_drive} 和“\textbf{标量场 $\chi^{\;\! \mathcolor{gray}{\omega} }_{\;\! \mathcolor{gray}{z} \textcolor{Maroon}{(2)}}$ \textcolor{NavyBlue}{调制}}” \bref{eq:scalar_chi2_modulation} 这 2 个条件的共同作用下,\bref{eq:DP^(2)-1_32-spectrum-G4,eq:simplify8-scalar-g-modulus} 所共同辅助定义的,以\textcolor{NavyBlue}{脉冲}\textcolor{Maroon}{光整流}、\textcolor{NavyBlue}{连续光}\textcolor{Maroon}{差频}为代表的\textcolor{Maroon}{下转换}过程的\textcolor{NavyBlue}{有效非线性系数}张量 \bref{eq:chieff*} 退化为标量(场)
\begin{subequations} \label{eq:chieff*-scalar}
\begin{align}
	\textcolor{gray}{\xint{{}^{}_{\mathcolor{gray}{-}}}{23}{\widetilde{\textcolor{black}{\chi}}}}^{ \mathcolor{gray}{\omega} \textcolor{PineGreen}{\hat{1}} \textcolor{Maroon}{(2)} }_{ \textcolor{NavyBlue}{\text{eff}} \textcolor{PineGreen}{\hat{3} \hat{2}} } &\xrightarrow[\text{\bref{eq:chieff*-spectrum}}]{\text{\bref{eq:scalar_chi2_modulation}}} \xint{{}^{}_{\mathcolor{gray}{-}}}{10}{\hat{g}}^{\;\! \textcolor{PineGreen}{\hat{1}} \textcolor{Plum}{\dag}}_{\;\! \mathcolor{gray}{\omega}} \cdot \bar{\chi}^{\;\! \textcolor{PineGreen}{\hat{1}} \mathcolor{gray}{\omega} \hat{3} \hat{2} \textcolor{Plum}{*} }_{\;\!  \textcolor{Maroon}{(2)} \textcolor{PineGreen}{\hat{3} \hat{2}} } ~ {\hat{g}}^{\;\! \mathcolor{gray}{\omega} }_{\;\! \hat{2} \textcolor{PineGreen}{\hat{2}} } ~\mathcolor{gray}{\widetilde \circ}~ {\hat{g}}^{\;\! \mathcolor{gray}{\omega} }_{\;\! \hat{3} \textcolor{PineGreen}{\hat{3}} } ~, \label{eq:chieff*-scalar-spectrum} \\
	\xint{{}^{}_{\mathcolor{gray}{-}}}{23}{\chi}^{ \textcolor{PineGreen}{\hat{1}} \textcolor{Maroon}{(2)} }_{\textcolor{NavyBlue}{\text{eff}} \textcolor{PineGreen}{\hat{3}} \textcolor{PineGreen}{\hat{2}} } &\xrightarrow[\text{\bref{eq:chieff*-discrete}}]{\text{\bref{eq:scalar_chi2_modulation}}} \xint{{}^{}_{\mathcolor{gray}{-}}}{10}{\hat{g}}^{\;\! \textcolor{PineGreen}{\hat{1}} \textcolor{Plum}{\dag}}_{\;\! } \cdot \bar{\chi}^{\;\! \textcolor{PineGreen}{\hat{1}} \hat{3} \hat{2} \textcolor{Plum}{*} }_{\;\! \textcolor{Maroon}{(2)} \textcolor{PineGreen}{\hat{3} \hat{2}}} ~ {\hat{g}}_{\;\! \hat{3} \textcolor{PineGreen}{\hat{3}}} ~ {\hat{g}}^{\;\! \textcolor{Plum}{*}}_{\;\! \hat{2} \textcolor{PineGreen}{\hat{2}} } ~, \label{eq:chieff*-scalar-discrete}
\end{align}
\end{subequations}
自此,\textcolor{NavyBlue}{脉冲}\textcolor{Maroon}{光整流}、\textcolor{NavyBlue}{连续光}\textcolor{Maroon}{差频}的电场\textcolor{PineGreen}{本征复振幅}方程 \bref{eq:scalar-g-modulus-P-chieff*} 变为
\begin{subequations} \label{eq:scalar-g-modulus-P-chieff*-scalar}
\begin{align}
	\mathcolor{gray}{\nabla_z} \xint{\begin{smallmatrix} ~ \\ {}^{}_{\mathcolor{gray}{-}} \\ ~ \end{smallmatrix}}{09}{\mathtt{g}}^{\;\!\mathcolor{gray}{\omega} \textcolor{PineGreen}{\hat{1}}}_{\;\! \mathcolor{gray}{z}} &\xrightarrow[\text{\bref{eq:scalar-g-modulus-P-chieff*-spectrum}}]{\text{\bref{eq:scalar_chi2_modulation}}} \mathbb{i} k_{\textcolor{Maroon}{\mathsf{o}} \mathcolor{gray}{\omega}}^{\;\! 2} \frac{ \textcolor{gray}{\xint{{}^{}_{\mathcolor{gray}{-}}}{23}{\widetilde{\textcolor{black}{\chi}}}}^{ \mathcolor{gray}{\omega} \textcolor{PineGreen}{\hat{1}} \textcolor{Maroon}{(2)} }_{ \textcolor{NavyBlue}{\text{eff}} \textcolor{PineGreen}{\hat{3} \hat{2}} } ~ \mathcolor{gray}{\mathcal F_{z}^{-\textcolor{Plum}{*}}} \left[ \xint{\mathcolor{gray}{-}}{18}{M}^{\;\! \mathcolor{gray}{\omega} }_{\;\! \mathcolor{gray}{k_{\symup{z}}} \textcolor{Maroon}{(2)} } \mathcolor{gray}{\circ} \xint{\mathcolor{gray}{-}}{15}{\mathtt{G}}^{\;\! \textcolor{PineGreen}{\hat{2}} \mathcolor{gray}{\omega} }_{\;\! \mathcolor{gray}{z}} ~\mathcolor{gray}{\widetilde \circledcirc}~ \xint{\mathcolor{gray}{-}}{15}{\mathtt{G}}^{\;\! \textcolor{PineGreen}{\hat{3}} \mathcolor{gray}{\omega} }_{\;\! \mathcolor{gray}{z}} \right] }{ 2 \lvert \xint{{}^{}_{\mathcolor{gray}{-}}}{10}{\hat{g}}^{\;\! \textcolor{PineGreen}{\hat{1}}}_{\;\! \mathcolor{gray}{\omega}} \rvert^2 \xint{\begin{smallmatrix} ~ \\ {}^{}_{\mathcolor{gray}{-}} \\ ~ \end{smallmatrix}}{15}{k}_{\;\! \symup{z}}^{\;\! \mathcolor{gray}{\omega} \textcolor{PineGreen}{\hat{1}}} \mathbb{e}^{\mathbb{i} \xint{\begin{smallmatrix} ~ \\ {}^{}_{\mathcolor{gray}{-}} \\ ~ \end{smallmatrix}}{15}{k}_{\symup{z}}^{\;\! \mathcolor{gray}{\omega} \textcolor{PineGreen}{\hat{1}}} \mathcolor{gray}{z}}} ~, \label{eq:scalar-g-modulus-P-chieff*-scalar-spectrum} \\
	\mathcolor{gray}{\nabla_z} \xint{\begin{smallmatrix} ~ \\ {}^{}_{\mathcolor{gray}{-}} \\ ~ \end{smallmatrix}}{09}{\mathtt{g}}^{\;\! \textcolor{PineGreen}{\hat{1}}}_{\;\! \mathcolor{gray}{z}} &\xrightarrow[\text{\bref{eq:scalar-g-modulus-P-chieff*-discrete}}]{\text{\bref{eq:scalar_chi2_modulation}}} \mathbb{i} k_{\textcolor{Maroon}{\mathsf{o}} \mathcolor{gray}{1}}^{\;\! 2} \frac{ \xint{{}^{}_{\mathcolor{gray}{-}}}{23}{\chi}^{ \textcolor{PineGreen}{\hat{1}} \textcolor{Maroon}{(2)} }_{\textcolor{NavyBlue}{\text{eff}} \textcolor{PineGreen}{\hat{3}} \textcolor{PineGreen}{\hat{2}} } ~ \mathcolor{gray}{\mathcal F_{z}^{-\textcolor{Plum}{*}}} \left[ \xint{\mathcolor{gray}{-}}{18}{M}^{\;\! \mathcolor{gray}{1} }_{\;\! \mathcolor{gray}{k_{\symup{z}}} \textcolor{Maroon}{(2)} } \mathcolor{gray}{\circ} \left( \xint{\mathcolor{gray}{-}}{15}{\mathtt{G}}^{\;\! \textcolor{PineGreen}{\hat{2}} }_{\;\! \mathcolor{gray}{z}} \mathcolor{gray}{\circ} \xint{\mathcolor{gray}{-}}{15}{\mathtt{G}}^{\;\! \textcolor{PineGreen}{\hat{3}} }_{\;\! \mathcolor{gray}{z}} \right) \right]}{ 2 \lvert \xint{{}^{}_{\mathcolor{gray}{-}}}{10}{\hat{g}}^{\;\! \textcolor{PineGreen}{\hat{1}}} \rvert^2 \xint{\begin{smallmatrix} ~ \\ {}^{}_{\mathcolor{gray}{-}} \\ ~ \end{smallmatrix}}{15}{k}_{\;\! \symup{z}}^{\;\!  \textcolor{PineGreen}{\hat{1}}} \mathbb{e}^{\mathbb{i} \xint{\begin{smallmatrix} ~ \\ {}^{}_{\mathcolor{gray}{-}} \\ ~ \end{smallmatrix}}{15}{k}_{\symup{z}}^{\;\!  \textcolor{PineGreen}{\hat{1}}} \mathcolor{gray}{z}}} ~, \label{eq:scalar-g-modulus-P-chieff*-scalar-discrete}
\end{align}
\end{subequations}
注意,对 \bref{eq:scalar-g-modulus-P-chieff*-scalar-discrete,eq:scalar-g-modulus-P-chieff*-discrete} 添加了类似 \bref{eq:(AcircB)circC!=Acirc(BcircC)g} 的括号,以保证从右往左计算两个\textcolor{Plum}{互相关}。

%\marginLeft[-2.4em]{ssec:3wavemix}\subsection{三波混频、光整流级联电光效应 - 电场本征复振幅方程}\label{ssec:3wavemix}
%\marginLeft[-2.4em]{ssec:3wavemix}\subsection{脉冲光、连续光三波混频 - 电场本征复振幅方程}\label{ssec:3wavemix}
\marginLeft[-2.4em]{ssec:pulse-3wavemix}\subsection{脉冲光三波混频 - 电场本征复振幅方程}\label{ssec:pulse-3wavemix}

结合 \bref{ssec:SHG_spectrum} 中,以\textcolor{NavyBlue}{脉冲光}\textcolor{Maroon}{倍频}、\textcolor{NavyBlue}{脉冲}\textcolor{Maroon}{光整流}后的级联\textcolor{Maroon}{电光效应}为代表的\textcolor{NavyBlue}{超快}\textcolor{Maroon}{上转换}过程的电场\textcolor{PineGreen}{本征复振幅}方程 \bref{eq:simplify8-scalar-g-modulus-P-spectrum} 及其(由 \bref{eq:simplify8-scalar-g-modulus,eq:DP^(2)-1_32-spectrum-DFG7} 构建的)\textcolor{Maroon}{下转换}版本,可以得到仅电子和光子间\textcolor{NavyBlue}{瞬态}相互作用的\textcolor{Maroon}{三波混频}耦合波方程组
\begin{subequations} \label{eq:3wavemix-scalar-g-EE-spectrum}
\begin{align}
	\mathcolor{gray}{\nabla_z} \xint{\begin{smallmatrix} ~ \\ {}^{}_{\mathcolor{gray}{-}} \\ ~ \end{smallmatrix}}{09}{\mathtt{g}}^{\;\!\mathcolor{gray}{\omega} \textcolor{PineGreen}{\hat{3}}}_{\;\! \mathcolor{gray}{z}} &\xrightarrow[]{\text{\bref{eq:simplify8-scalar-g-modulus-P-spectrum}}} \mathbb{i} k_{\textcolor{Maroon}{\mathsf{o}} \mathcolor{gray}{\omega}}^{\;\! 2} \frac{\xint{{}^{}_{\mathcolor{gray}{-}}}{10}{\hat{g}}^{\;\! \hat{3} \textcolor{PineGreen}{\hat{3}} \textcolor{Plum}{*}}_{\;\! \mathcolor{gray}{\omega}} {\chi}^{\;\! \textcolor{PineGreen}{\hat{3}} \mathcolor{gray}{\omega} \hat{1} \hat{2} }_{\;\! \hat{3} \textcolor{Maroon}{(2)} \textcolor{PineGreen}{\hat{1} \hat{2}}} ~ \mathcolor{gray}{\mathcal F_{z}^{-1}} \left[ \xint{\mathcolor{gray}{-}}{18}{M}^{\;\! \mathcolor{gray}{\omega} \hat{1} \hat{2} }_{\;\! \hat{3} \mathcolor{gray}{k_{\symup{z}}} \textcolor{Maroon}{(2)} } \mathcolor{gray}{*} \xint{\mathcolor{gray}{-}}{25}{E}^{\;\! \textcolor{PineGreen}{\hat{1}} \mathcolor{gray}{\omega} }_{\;\! \hat{1} \mathcolor{gray}{z}} ~\mathcolor{gray}{\widetilde \circledast}~ \xint{\mathcolor{gray}{-}}{25}{E}^{\;\! \textcolor{PineGreen}{\hat{2}} \mathcolor{gray}{\omega} }_{\;\! \hat{2} \mathcolor{gray}{z}} \right]}{ 2 \lvert \xint{{}^{}_{\mathcolor{gray}{-}}}{10}{\hat{g}}^{\;\! \textcolor{PineGreen}{\hat{3}}}_{\;\! \mathcolor{gray}{\omega}} \rvert^2 \xint{\begin{smallmatrix} ~ \\ {}^{}_{\mathcolor{gray}{-}} \\ ~ \end{smallmatrix}}{15}{k}_{\;\! \symup{z}}^{\;\! \mathcolor{gray}{\omega} \textcolor{PineGreen}{\hat{3}}} \mathbb{e}^{\mathbb{i} \xint{\begin{smallmatrix} ~ \\ {}^{}_{\mathcolor{gray}{-}} \\ ~ \end{smallmatrix}}{15}{k}_{\symup{z}}^{\;\! \mathcolor{gray}{\omega} \textcolor{PineGreen}{\hat{3}}} \mathcolor{gray}{z}}} ~, \label{eq:up-scalar-g-EE-312-spectrum} \\
	\mathcolor{gray}{\nabla_z} \xint{\begin{smallmatrix} ~ \\ {}^{}_{\mathcolor{gray}{-}} \\ ~ \end{smallmatrix}}{09}{\mathtt{g}}^{\;\!\mathcolor{gray}{\omega} \textcolor{PineGreen}{\hat{1}}}_{\;\! \mathcolor{gray}{z}} &\xrightarrow[\text{$\sim$ \bref{eq:simplify8-scalar-g-modulus}}]{\text{\bref{eq:DP^(2)-1_32-spectrum-DFG7}}} \mathbb{i} k_{\textcolor{Maroon}{\mathsf{o}} \mathcolor{gray}{\omega}}^{\;\! 2} \frac{\xint{{}^{}_{\mathcolor{gray}{-}}}{10}{\hat{g}}^{\;\! \hat{1} \textcolor{PineGreen}{\hat{1}} \textcolor{Plum}{*}}_{\;\! \mathcolor{gray}{\omega}} {\chi}^{\;\! \textcolor{PineGreen}{\hat{1}} \mathcolor{gray}{\omega} \hat{3} \hat{2} \textcolor{Plum}{*} }_{\;\! \hat{1} \textcolor{Maroon}{(2)} \textcolor{PineGreen}{\hat{3} \hat{2}}} ~\mathcolor{gray}{\mathcal F_{z}^{-\textcolor{Plum}{*}}} \left[ \xint{\mathcolor{gray}{-}}{18}{M}^{\;\! \mathcolor{gray}{\omega} \hat{3} \hat{2} }_{\;\! \hat{1} \mathcolor{gray}{k_{\symup{z}}} \textcolor{Maroon}{(2)} } \mathcolor{gray}{\circ} \xint{\mathcolor{gray}{-}}{255}{E}^{\;\!\textcolor{PineGreen}{\hat{2}} \mathcolor{gray}{\omega} }_{\;\! \hat{2} \mathcolor{gray}{z}} ~\mathcolor{gray}{\widetilde \circledcirc}~ \xint{\mathcolor{gray}{-}}{255}{E}^{\;\! \textcolor{PineGreen}{\hat{3}} \mathcolor{gray}{\omega}}_{\;\! \hat{3} \mathcolor{gray}{z}} \right]}{ 2 \lvert \xint{{}^{}_{\mathcolor{gray}{-}}}{10}{\hat{g}}^{\;\! \textcolor{PineGreen}{\hat{1}}}_{\;\! \mathcolor{gray}{\omega}} \rvert^2 \xint{\begin{smallmatrix} ~ \\ {}^{}_{\mathcolor{gray}{-}} \\ ~ \end{smallmatrix}}{15}{k}_{\;\! \symup{z}}^{\;\! \mathcolor{gray}{\omega} \textcolor{PineGreen}{\hat{1}}} \mathbb{e}^{\mathbb{i} \xint{\begin{smallmatrix} ~ \\ {}^{}_{\mathcolor{gray}{-}} \\ ~ \end{smallmatrix}}{15}{k}_{\symup{z}}^{\;\! \mathcolor{gray}{\omega} \textcolor{PineGreen}{\hat{1}}} \mathcolor{gray}{z}}} ~, \label{eq:down-scalar-g-EE-132-spectrum} \\
	\mathcolor{gray}{\nabla_z} \xint{\begin{smallmatrix} ~ \\ {}^{}_{\mathcolor{gray}{-}} \\ ~ \end{smallmatrix}}{09}{\mathtt{g}}^{\;\!\mathcolor{gray}{\omega} \textcolor{PineGreen}{\hat{2}}}_{\;\! \mathcolor{gray}{z}} &\xrightarrow[]{\text{$\sim$ \bref{eq:down-scalar-g-EE-132-spectrum}}} \mathbb{i} k_{\textcolor{Maroon}{\mathsf{o}} \mathcolor{gray}{\omega}}^{\;\! 2} \frac{\xint{{}^{}_{\mathcolor{gray}{-}}}{10}{\hat{g}}^{\;\! \hat{2} \textcolor{PineGreen}{\hat{2}} \textcolor{Plum}{*}}_{\;\! \mathcolor{gray}{\omega}} {\chi}^{\;\! \textcolor{PineGreen}{\hat{2}} \mathcolor{gray}{\omega} \hat{3} \hat{1} \textcolor{Plum}{*} }_{\;\! \hat{2} \textcolor{Maroon}{(2)} \textcolor{PineGreen}{\hat{3} \hat{1}}} ~\mathcolor{gray}{\mathcal F_{z}^{-\textcolor{Plum}{*}}} \left[ \xint{\mathcolor{gray}{-}}{18}{M}^{\;\! \mathcolor{gray}{\omega} \hat{3} \hat{1} }_{\;\! \hat{2} \mathcolor{gray}{k_{\symup{z}}} \textcolor{Maroon}{(2)} } \mathcolor{gray}{\circ} \xint{\mathcolor{gray}{-}}{255}{E}^{\;\!\textcolor{PineGreen}{\hat{1}} \mathcolor{gray}{\omega} }_{\;\! \hat{1} \mathcolor{gray}{z}} ~\mathcolor{gray}{\widetilde \circledcirc}~ \xint{\mathcolor{gray}{-}}{255}{E}^{\;\!\textcolor{PineGreen}{\hat{3}} \mathcolor{gray}{\omega}}_{\;\! \hat{3} \mathcolor{gray}{z}} \right]}{ 2 \lvert \xint{{}^{}_{\mathcolor{gray}{-}}}{10}{\hat{g}}^{\;\! \textcolor{PineGreen}{\hat{2}}}_{\;\! \mathcolor{gray}{\omega}} \rvert^2 \xint{\begin{smallmatrix} ~ \\ {}^{}_{\mathcolor{gray}{-}} \\ ~ \end{smallmatrix}}{15}{k}_{\;\! \symup{z}}^{\;\! \mathcolor{gray}{\omega} \textcolor{PineGreen}{\hat{2}}} \mathbb{e}^{\mathbb{i} \xint{\begin{smallmatrix} ~ \\ {}^{}_{\mathcolor{gray}{-}} \\ ~ \end{smallmatrix}}{15}{k}_{\symup{z}}^{\;\! \mathcolor{gray}{\omega} \textcolor{PineGreen}{\hat{2}}} \mathcolor{gray}{z}}} ~. \label{eq:down-scalar-g-EE-231-spectrum}
\end{align}
\end{subequations}

在“\textbf{标量\textcolor{Plum}{非线性}\textcolor{NavyBlue}{波源}}(\textcolor{NavyBlue}{脉冲})”条件 \bref{eq:scalar_nonlinear_drive-spectrum} 下,上述描述电子和光子\textcolor{NavyBlue}{宽带}相互作用的\textcolor{Maroon}{三波混频}方程组,从 \bref{eq:3wavemix-scalar-g-EE-spectrum} 变为
\begin{subequations} \label{eq:3wavemix-scalar-g-EE-chieff-spectrum}
\begin{align}
	\mathcolor{gray}{\nabla_z} \xint{\begin{smallmatrix} ~ \\ {}^{}_{\mathcolor{gray}{-}} \\ ~ \end{smallmatrix}}{09}{\mathtt{g}}^{\;\!\mathcolor{gray}{\omega} \textcolor{PineGreen}{\hat{3}}}_{\;\! \mathcolor{gray}{z}} &\xrightarrow[]{\text{\bref{eq:scalar-g-modulus-P-chieff-spectrum}}} \mathbb{i} k_{\textcolor{Maroon}{\mathsf{o}} \mathcolor{gray}{\omega}}^{\;\! 2} \frac{\textcolor{gray}{\xint{{}^{}_{\mathcolor{gray}{-}}}{23}{\widetilde{\textcolor{black}{\chi}}}}^{ \hat{3} \textcolor{PineGreen}{\hat{3}} \textcolor{Maroon}{(2)} \mathcolor{gray}{\omega} }_{ \textcolor{NavyBlue}{\text{eff}} \hat{1} \hat{2} \textcolor{PineGreen}{\hat{1} \hat{2}} } ~ \mathcolor{gray}{\mathcal F_{z}^{-1}} \left[ \xint{\mathcolor{gray}{-}}{18}{M}^{\;\! \mathcolor{gray}{\omega} \hat{1} \hat{2} }_{\;\! \hat{3} \mathcolor{gray}{k_{\symup{z}}} \textcolor{Maroon}{(2)} } \mathcolor{gray}{*} \xint{\mathcolor{gray}{-}}{15}{\mathtt{G}}^{\;\! \textcolor{PineGreen}{\hat{1}} \mathcolor{gray}{\omega} }_{\;\! \mathcolor{gray}{z}} ~\mathcolor{gray}{\widetilde \circledast}~ \xint{\mathcolor{gray}{-}}{15}{\mathtt{G}}^{\;\! \textcolor{PineGreen}{\hat{2}} \mathcolor{gray}{\omega} }_{\;\! \mathcolor{gray}{z}} \right]}{ 2 \lvert \xint{{}^{}_{\mathcolor{gray}{-}}}{10}{\hat{g}}^{\;\! \textcolor{PineGreen}{\hat{3}}}_{\;\! \mathcolor{gray}{\omega}} \rvert^2 \xint{\begin{smallmatrix} ~ \\ {}^{}_{\mathcolor{gray}{-}} \\ ~ \end{smallmatrix}}{15}{k}_{\;\! \symup{z}}^{\;\! \mathcolor{gray}{\omega} \textcolor{PineGreen}{\hat{3}}} \mathbb{e}^{\mathbb{i} \xint{\begin{smallmatrix} ~ \\ {}^{}_{\mathcolor{gray}{-}} \\ ~ \end{smallmatrix}}{15}{k}_{\symup{z}}^{\;\! \mathcolor{gray}{\omega} \textcolor{PineGreen}{\hat{3}}} \mathcolor{gray}{z}}} ~, \label{eq:up-scalar-g-EE-312-chieff-spectrum} \\
	\mathcolor{gray}{\nabla_z} \xint{\begin{smallmatrix} ~ \\ {}^{}_{\mathcolor{gray}{-}} \\ ~ \end{smallmatrix}}{09}{\mathtt{g}}^{\;\!\mathcolor{gray}{\omega} \textcolor{PineGreen}{\hat{1}}}_{\;\! \mathcolor{gray}{z}} &\xrightarrow[]{\text{\bref{eq:scalar-g-modulus-P-chieff*-spectrum}}} \mathbb{i} k_{\textcolor{Maroon}{\mathsf{o}} \mathcolor{gray}{\omega}}^{\;\! 2} \frac{ \textcolor{gray}{\xint{{}^{}_{\mathcolor{gray}{-}}}{23}{\widetilde{\textcolor{black}{\chi}}}}^{ \hat{1} \textcolor{PineGreen}{\hat{1}} \textcolor{Maroon}{(2)} \mathcolor{gray}{\omega} }_{ \textcolor{NavyBlue}{\text{eff}} \hat{3} \hat{2} \textcolor{PineGreen}{\hat{3} \hat{2}} } \mathcolor{gray}{\mathcal F_{z}^{-\textcolor{Plum}{*}}} \left[ \xint{\mathcolor{gray}{-}}{18}{M}^{\;\! \mathcolor{gray}{\omega} \hat{3} \hat{2} }_{\;\! \hat{1} \mathcolor{gray}{k_{\symup{z}}} \textcolor{Maroon}{(2)} } \mathcolor{gray}{\circ} \xint{\mathcolor{gray}{-}}{15}{\mathtt{G}}^{\;\! \textcolor{PineGreen}{\hat{2}} \mathcolor{gray}{\omega} }_{\;\! \mathcolor{gray}{z}} ~\mathcolor{gray}{\widetilde \circledcirc}~ \xint{\mathcolor{gray}{-}}{15}{\mathtt{G}}^{\;\! \textcolor{PineGreen}{\hat{3}} \mathcolor{gray}{\omega} }_{\;\! \mathcolor{gray}{z}} \right] }{ 2 \lvert \xint{{}^{}_{\mathcolor{gray}{-}}}{10}{\hat{g}}^{\;\! \textcolor{PineGreen}{\hat{1}}}_{\;\! \mathcolor{gray}{\omega}} \rvert^2 \xint{\begin{smallmatrix} ~ \\ {}^{}_{\mathcolor{gray}{-}} \\ ~ \end{smallmatrix}}{15}{k}_{\;\! \symup{z}}^{\;\! \mathcolor{gray}{\omega} \textcolor{PineGreen}{\hat{1}}} \mathbb{e}^{\mathbb{i} \xint{\begin{smallmatrix} ~ \\ {}^{}_{\mathcolor{gray}{-}} \\ ~ \end{smallmatrix}}{15}{k}_{\symup{z}}^{\;\! \mathcolor{gray}{\omega} \textcolor{PineGreen}{\hat{1}}} \mathcolor{gray}{z}}} ~, \label{eq:down-scalar-g-EE-132-chieff*-spectrum} \\
	\mathcolor{gray}{\nabla_z} \xint{\begin{smallmatrix} ~ \\ {}^{}_{\mathcolor{gray}{-}} \\ ~ \end{smallmatrix}}{09}{\mathtt{g}}^{\;\!\mathcolor{gray}{\omega} \textcolor{PineGreen}{\hat{2}}}_{\;\! \mathcolor{gray}{z}} &\xrightarrow[\text{\bref{eq:down-scalar-g-EE-231-spectrum}}]{\text{\bref{eq:scalar_nonlinear_drive-spectrum}}} \mathbb{i} k_{\textcolor{Maroon}{\mathsf{o}} \mathcolor{gray}{\omega}}^{\;\! 2} \frac{ \textcolor{gray}{\xint{{}^{}_{\mathcolor{gray}{-}}}{23}{\widetilde{\textcolor{black}{\chi}}}}^{ \hat{2} \textcolor{PineGreen}{\hat{2}} \textcolor{Maroon}{(2)} \mathcolor{gray}{\omega} }_{ \textcolor{NavyBlue}{\text{eff}} \hat{3} \hat{1} \textcolor{PineGreen}{\hat{3} \hat{1}} } \mathcolor{gray}{\mathcal F_{z}^{-\textcolor{Plum}{*}}} \left[ \xint{\mathcolor{gray}{-}}{18}{M}^{\;\! \mathcolor{gray}{\omega} \hat{3} \hat{1} }_{\;\! \hat{2} \mathcolor{gray}{k_{\symup{z}}} \textcolor{Maroon}{(2)} } \mathcolor{gray}{\circ} \xint{\mathcolor{gray}{-}}{15}{\mathtt{G}}^{\;\! \textcolor{PineGreen}{\hat{1}} \mathcolor{gray}{\omega} }_{\;\! \mathcolor{gray}{z}} ~\mathcolor{gray}{\widetilde \circledcirc}~ \xint{\mathcolor{gray}{-}}{15}{\mathtt{G}}^{\;\! \textcolor{PineGreen}{\hat{3}} \mathcolor{gray}{\omega} }_{\;\! \mathcolor{gray}{z}} \right] }{ 2 \lvert \xint{{}^{}_{\mathcolor{gray}{-}}}{10}{\hat{g}}^{\;\! \textcolor{PineGreen}{\hat{2}}}_{\;\! \mathcolor{gray}{\omega}} \rvert^2 \xint{\begin{smallmatrix} ~ \\ {}^{}_{\mathcolor{gray}{-}} \\ ~ \end{smallmatrix}}{15}{k}_{\;\! \symup{z}}^{\;\! \mathcolor{gray}{\omega} \textcolor{PineGreen}{\hat{2}}} \mathbb{e}^{\mathbb{i} \xint{\begin{smallmatrix} ~ \\ {}^{}_{\mathcolor{gray}{-}} \\ ~ \end{smallmatrix}}{15}{k}_{\symup{z}}^{\;\! \mathcolor{gray}{\omega} \textcolor{PineGreen}{\hat{2}}} \mathcolor{gray}{z}}} ~, \label{eq:down-scalar-g-EE-231-chieff*-spectrum}
\end{align}
\end{subequations}
定义了“\textbf{标量\textcolor{Plum}{非线性}\textcolor{NavyBlue}{波源}}”条件下,该\textcolor{NavyBlue}{超快}过程的\textcolor{NavyBlue}{有效非线性系数}(三阶)张量
\begin{subequations} \label{eq:3wavemix-chieff-spectrum}
\begin{align}
	\textcolor{gray}{\xint{{}^{}_{\mathcolor{gray}{-}}}{23}{\widetilde{\textcolor{black}{\chi}}}}^{ \hat{3} \textcolor{PineGreen}{\hat{3}} \textcolor{Maroon}{(2)} \mathcolor{gray}{\omega} }_{ \textcolor{NavyBlue}{\text{eff}} \hat{1} \hat{2} \textcolor{PineGreen}{\hat{1} \hat{2}} } &\xrightarrow[]{\text{\bref{eq:chieff-spectrum}}} \xint{{}^{}_{\mathcolor{gray}{-}}}{10}{\hat{g}}^{\;\! \hat{3} \textcolor{PineGreen}{\hat{3}} \textcolor{Plum}{*}}_{\;\! \mathcolor{gray}{\omega}} {\chi}^{\;\! \hat{3} \textcolor{PineGreen}{\hat{3}} \textcolor{Maroon}{(2)} }_{\;\! \mathcolor{gray}{\omega} \hat{1} \hat{2} \textcolor{PineGreen}{\hat{1} \hat{2}} } ~ {\hat{g}}^{\;\! \mathcolor{gray}{\omega} }_{\;\! \hat{1} \textcolor{PineGreen}{\hat{1}} } ~\mathcolor{gray}{\widetilde *}~ {\hat{g}}^{\;\! \mathcolor{gray}{\omega} }_{\;\! \hat{2} \textcolor{PineGreen}{\hat{2}} } ~, \label{eq:up-312-chieff*-spectrum} \\
	\textcolor{gray}{\xint{{}^{}_{\mathcolor{gray}{-}}}{23}{\widetilde{\textcolor{black}{\chi}}}}^{ \hat{1} \textcolor{PineGreen}{\hat{1}} \textcolor{Maroon}{(2)} \mathcolor{gray}{\omega} }_{ \textcolor{NavyBlue}{\text{eff}} \hat{3} \hat{2} \textcolor{PineGreen}{\hat{3} \hat{2}} } &\xrightarrow[]{\text{\bref{eq:chieff*-spectrum}}} \xint{{}^{}_{\mathcolor{gray}{-}}}{10}{\hat{g}}^{\;\! \hat{1} \textcolor{PineGreen}{\hat{1}} \textcolor{Plum}{*}}_{\;\! \mathcolor{gray}{\omega}} {\chi}^{\;\! \hat{1} \textcolor{PineGreen}{\hat{1}} \textcolor{Maroon}{(2)} \textcolor{Plum}{*} }_{\;\! \mathcolor{gray}{\omega} \hat{3} \hat{2} \textcolor{PineGreen}{\hat{3} \hat{2}} } ~ {\hat{g}}^{\;\! \mathcolor{gray}{\omega} }_{\;\! \hat{2} \textcolor{PineGreen}{\hat{2}} } ~\mathcolor{gray}{\widetilde \circ}~ {\hat{g}}^{\;\! \mathcolor{gray}{\omega} }_{\;\! \hat{3} \textcolor{PineGreen}{\hat{3}} } ~, \label{eq:down-132-chieff*-spectrum} \\
	\textcolor{gray}{\xint{{}^{}_{\mathcolor{gray}{-}}}{23}{\widetilde{\textcolor{black}{\chi}}}}^{ \hat{2} \textcolor{PineGreen}{\hat{2}} \textcolor{Maroon}{(2)} \mathcolor{gray}{\omega} }_{ \textcolor{NavyBlue}{\text{eff}} \hat{3} \hat{1} \textcolor{PineGreen}{\hat{3} \hat{1}} } &\xrightarrow[]{\text{$\sim$ \bref{eq:down-132-chieff*-spectrum}}} \xint{{}^{}_{\mathcolor{gray}{-}}}{10}{\hat{g}}^{\;\! \hat{2} \textcolor{PineGreen}{\hat{2}} \textcolor{Plum}{*}}_{\;\! \mathcolor{gray}{\omega}} {\chi}^{\;\! \hat{2} \textcolor{PineGreen}{\hat{2}} \textcolor{Maroon}{(2)} \textcolor{Plum}{*} }_{\;\! \mathcolor{gray}{\omega} \hat{3} \hat{1} \textcolor{PineGreen}{\hat{3} \hat{1}} } ~ {\hat{g}}^{\;\! \mathcolor{gray}{\omega} }_{\;\! \hat{1} \textcolor{PineGreen}{\hat{1}} } ~\mathcolor{gray}{\widetilde \circ}~ {\hat{g}}^{\;\! \mathcolor{gray}{\omega} }_{\;\! \hat{3} \textcolor{PineGreen}{\hat{3}} } ~. \label{eq:down-231-chieff*-spectrum}
\end{align}
\end{subequations}

进一步,再在“\textbf{标量场 $\chi^{\;\! \mathcolor{gray}{\omega} }_{\;\! \mathcolor{gray}{z} \textcolor{Maroon}{(2)}}$ \textcolor{NavyBlue}{调制}}” \bref{eq:scalar_chi2_modulation} 的额外条件下,描述电子和光子\textcolor{NavyBlue}{宽带}相互作用的\textcolor{Maroon}{三波混频}方程组从 \bref{eq:3wavemix-scalar-g-EE-chieff-spectrum} 变为
\begin{subequations} \label{eq:3wavemix-scalar-g-EE-chieff-scalar-spectrum}
\begin{align}
	\mathcolor{gray}{\nabla_z} \xint{\begin{smallmatrix} ~ \\ {}^{}_{\mathcolor{gray}{-}} \\ ~ \end{smallmatrix}}{09}{\mathtt{g}}^{\;\!\mathcolor{gray}{\omega} \textcolor{PineGreen}{\hat{3}}}_{\;\! \mathcolor{gray}{z}} &\xrightarrow[]{\text{\bref{eq:scalar-g-modulus-P-chieff-scalar-spectrum}}} \mathbb{i} k_{\textcolor{Maroon}{\mathsf{o}} \mathcolor{gray}{\omega}}^{\;\! 2} \frac{\textcolor{gray}{\xint{{}^{}_{\mathcolor{gray}{-}}}{23}{\widetilde{\textcolor{black}{\chi}}}}^{ \mathcolor{gray}{\omega} \textcolor{PineGreen}{\hat{3}} \textcolor{Maroon}{(2)} }_{ \textcolor{NavyBlue}{\text{eff}} \textcolor{PineGreen}{\hat{1} \hat{2}} } ~ \mathcolor{gray}{\mathcal F_{z}^{-1}} \left[ \xint{\mathcolor{gray}{-}}{18}{M}^{\;\! \mathcolor{gray}{\omega} }_{\;\! \mathcolor{gray}{k_{\symup{z}}} \textcolor{Maroon}{(2)} } \mathcolor{gray}{*} \xint{\mathcolor{gray}{-}}{15}{\mathtt{G}}^{\;\! \textcolor{PineGreen}{\hat{1}} \mathcolor{gray}{\omega} }_{\;\! \mathcolor{gray}{z}} ~\mathcolor{gray}{\widetilde \circledast}~ \xint{\mathcolor{gray}{-}}{15}{\mathtt{G}}^{\;\! \textcolor{PineGreen}{\hat{2}} \mathcolor{gray}{\omega} }_{\;\! \mathcolor{gray}{z}} \right]}{ 2 \lvert \xint{{}^{}_{\mathcolor{gray}{-}}}{10}{\hat{g}}^{\;\! \textcolor{PineGreen}{\hat{3}}}_{\;\! \mathcolor{gray}{\omega}} \rvert^2 \xint{\begin{smallmatrix} ~ \\ {}^{}_{\mathcolor{gray}{-}} \\ ~ \end{smallmatrix}}{15}{k}_{\;\! \symup{z}}^{\;\! \mathcolor{gray}{\omega} \textcolor{PineGreen}{\hat{3}}} \mathbb{e}^{\mathbb{i} \xint{\begin{smallmatrix} ~ \\ {}^{}_{\mathcolor{gray}{-}} \\ ~ \end{smallmatrix}}{15}{k}_{\symup{z}}^{\;\! \mathcolor{gray}{\omega} \textcolor{PineGreen}{\hat{3}}} \mathcolor{gray}{z}}} ~, \label{eq:up-scalar-g-EE-132-chieff-scalar-spectrum} \\
	\mathcolor{gray}{\nabla_z} \xint{\begin{smallmatrix} ~ \\ {}^{}_{\mathcolor{gray}{-}} \\ ~ \end{smallmatrix}}{09}{\mathtt{g}}^{\;\!\mathcolor{gray}{\omega} \textcolor{PineGreen}{\hat{1}}}_{\;\! \mathcolor{gray}{z}} &\xrightarrow[]{\text{\bref{eq:scalar-g-modulus-P-chieff*-scalar-spectrum}}} \mathbb{i} k_{\textcolor{Maroon}{\mathsf{o}} \mathcolor{gray}{\omega}}^{\;\! 2} \frac{ \textcolor{gray}{\xint{{}^{}_{\mathcolor{gray}{-}}}{23}{\widetilde{\textcolor{black}{\chi}}}}^{ \mathcolor{gray}{\omega} \textcolor{PineGreen}{\hat{1}} \textcolor{Maroon}{(2)} }_{ \textcolor{NavyBlue}{\text{eff}} \textcolor{PineGreen}{\hat{3} \hat{2}} } ~ \mathcolor{gray}{\mathcal F_{z}^{-\textcolor{Plum}{*}}} \left[ \xint{\mathcolor{gray}{-}}{18}{M}^{\;\! \mathcolor{gray}{\omega} }_{\;\! \mathcolor{gray}{k_{\symup{z}}} \textcolor{Maroon}{(2)} } \mathcolor{gray}{\circ} \xint{\mathcolor{gray}{-}}{15}{\mathtt{G}}^{\;\! \textcolor{PineGreen}{\hat{2}} \mathcolor{gray}{\omega} }_{\;\! \mathcolor{gray}{z}} ~\mathcolor{gray}{\widetilde \circledcirc}~ \xint{\mathcolor{gray}{-}}{15}{\mathtt{G}}^{\;\! \textcolor{PineGreen}{\hat{3}} \mathcolor{gray}{\omega} }_{\;\! \mathcolor{gray}{z}} \right] }{ 2 \lvert \xint{{}^{}_{\mathcolor{gray}{-}}}{10}{\hat{g}}^{\;\! \textcolor{PineGreen}{\hat{1}}}_{\;\! \mathcolor{gray}{\omega}} \rvert^2 \xint{\begin{smallmatrix} ~ \\ {}^{}_{\mathcolor{gray}{-}} \\ ~ \end{smallmatrix}}{15}{k}_{\;\! \symup{z}}^{\;\! \mathcolor{gray}{\omega} \textcolor{PineGreen}{\hat{1}}} \mathbb{e}^{\mathbb{i} \xint{\begin{smallmatrix} ~ \\ {}^{}_{\mathcolor{gray}{-}} \\ ~ \end{smallmatrix}}{15}{k}_{\symup{z}}^{\;\! \mathcolor{gray}{\omega} \textcolor{PineGreen}{\hat{1}}} \mathcolor{gray}{z}}} ~, \label{eq:down-scalar-g-EE-132-chieff*-scalar-spectrum} \\
	\mathcolor{gray}{\nabla_z} \xint{\begin{smallmatrix} ~ \\ {}^{}_{\mathcolor{gray}{-}} \\ ~ \end{smallmatrix}}{09}{\mathtt{g}}^{\;\!\mathcolor{gray}{\omega} \textcolor{PineGreen}{\hat{2}}}_{\;\! \mathcolor{gray}{z}} &\xrightarrow[\text{\bref{eq:down-scalar-g-EE-231-chieff*-spectrum}}]{\text{\bref{eq:scalar_chi2_modulation}}} \mathbb{i} k_{\textcolor{Maroon}{\mathsf{o}} \mathcolor{gray}{\omega}}^{\;\! 2} \frac{ \textcolor{gray}{\xint{{}^{}_{\mathcolor{gray}{-}}}{23}{\widetilde{\textcolor{black}{\chi}}}}^{ \mathcolor{gray}{\omega} \textcolor{PineGreen}{\hat{2}} \textcolor{Maroon}{(2)} }_{ \textcolor{NavyBlue}{\text{eff}} \textcolor{PineGreen}{\hat{3} \hat{1}} } ~ \mathcolor{gray}{\mathcal F_{z}^{-\textcolor{Plum}{*}}} \left[ \xint{\mathcolor{gray}{-}}{18}{M}^{\;\! \mathcolor{gray}{\omega} }_{\;\! \mathcolor{gray}{k_{\symup{z}}} \textcolor{Maroon}{(2)} } \mathcolor{gray}{\circ} \xint{\mathcolor{gray}{-}}{15}{\mathtt{G}}^{\;\! \textcolor{PineGreen}{\hat{1}} \mathcolor{gray}{\omega} }_{\;\! \mathcolor{gray}{z}} ~\mathcolor{gray}{\widetilde \circledcirc}~ \xint{\mathcolor{gray}{-}}{15}{\mathtt{G}}^{\;\! \textcolor{PineGreen}{\hat{3}} \mathcolor{gray}{\omega} }_{\;\! \mathcolor{gray}{z}} \right] }{ 2 \lvert \xint{{}^{}_{\mathcolor{gray}{-}}}{10}{\hat{g}}^{\;\! \textcolor{PineGreen}{\hat{2}}}_{\;\! \mathcolor{gray}{\omega}} \rvert^2 \xint{\begin{smallmatrix} ~ \\ {}^{}_{\mathcolor{gray}{-}} \\ ~ \end{smallmatrix}}{15}{k}_{\;\! \symup{z}}^{\;\! \mathcolor{gray}{\omega} \textcolor{PineGreen}{\hat{2}}} \mathbb{e}^{\mathbb{i} \xint{\begin{smallmatrix} ~ \\ {}^{}_{\mathcolor{gray}{-}} \\ ~ \end{smallmatrix}}{15}{k}_{\symup{z}}^{\;\! \mathcolor{gray}{\omega} \textcolor{PineGreen}{\hat{2}}} \mathcolor{gray}{z}}} ~, \label{eq:down-scalar-g-EE-231-chieff*-scalar-spectrum}
\end{align}
\end{subequations}
其中,在“\textbf{标量\textcolor{Plum}{非线性}\textcolor{NavyBlue}{波源}}” \bref{eq:scalar_nonlinear_drive-spectrum} 和“\textbf{标量场 $\chi^{\;\! \mathcolor{gray}{\omega} }_{\;\! \mathcolor{gray}{z} \textcolor{Maroon}{(2)}}$ \textcolor{NavyBlue}{调制}}” \bref{eq:scalar_chi2_modulation} 这 2 个条件下,电子和光子\textcolor{NavyBlue}{宽带}相互作用的\textcolor{Maroon}{三波混频}\textcolor{NavyBlue}{有效非线性系数}张量降为标量
\begin{subequations} \label{eq:3wavemix-chieff-scalar-spectrum}
\begin{align}
	\textcolor{gray}{\xint{{}^{}_{\mathcolor{gray}{-}}}{23}{\widetilde{\textcolor{black}{\chi}}}}^{ \mathcolor{gray}{\omega} \textcolor{PineGreen}{\hat{3}} \textcolor{Maroon}{(2)} }_{ \textcolor{NavyBlue}{\text{eff}} \textcolor{PineGreen}{\hat{1} \hat{2}} } &\xrightarrow[]{\text{\bref{eq:chieff-scalar-spectrum}}} \xint{{}^{}_{\mathcolor{gray}{-}}}{10}{\hat{g}}^{\;\! \hat{3} \textcolor{PineGreen}{\hat{3}} \textcolor{Plum}{*}}_{\;\! \mathcolor{gray}{\omega}} {\chi}^{\;\! \textcolor{PineGreen}{\hat{3}} \mathcolor{gray}{\omega} \hat{1} \hat{2} }_{\;\! \hat{3} \textcolor{Maroon}{(2)} \textcolor{PineGreen}{\hat{1} \hat{2}}} ~ {\hat{g}}^{\;\! \mathcolor{gray}{\omega} }_{\;\! \hat{1} \textcolor{PineGreen}{\hat{1}} } ~\mathcolor{gray}{\widetilde *}~ {\hat{g}}^{\;\! \mathcolor{gray}{\omega} }_{\;\! \hat{2} \textcolor{PineGreen}{\hat{2}} } ~, \label{eq:up-312-chieff-scalar-spectrum} \\
	\textcolor{gray}{\xint{{}^{}_{\mathcolor{gray}{-}}}{23}{\widetilde{\textcolor{black}{\chi}}}}^{ \mathcolor{gray}{\omega} \textcolor{PineGreen}{\hat{1}} \textcolor{Maroon}{(2)} }_{ \textcolor{NavyBlue}{\text{eff}} \textcolor{PineGreen}{\hat{3} \hat{2}} } &\xrightarrow[]{\text{\bref{eq:chieff*-scalar-spectrum}}} \xint{{}^{}_{\mathcolor{gray}{-}}}{10}{\hat{g}}^{\;\! \hat{1} \textcolor{PineGreen}{\hat{1}} \textcolor{Plum}{*}}_{\;\! \mathcolor{gray}{\omega}} {\chi}^{\;\! \textcolor{PineGreen}{\hat{1}} \mathcolor{gray}{\omega} \hat{3} \hat{2} \textcolor{Plum}{*} }_{\;\! \hat{1} \textcolor{Maroon}{(2)} \textcolor{PineGreen}{\hat{3} \hat{2}} } ~ {\hat{g}}^{\;\! \mathcolor{gray}{\omega} }_{\;\! \hat{2} \textcolor{PineGreen}{\hat{2}} } ~\mathcolor{gray}{\widetilde \circ}~ {\hat{g}}^{\;\! \mathcolor{gray}{\omega} }_{\;\! \hat{3} \textcolor{PineGreen}{\hat{3}} } ~, \label{eq:down-132-chieff*-scalar-spectrum} \\
	\textcolor{gray}{\xint{{}^{}_{\mathcolor{gray}{-}}}{23}{\widetilde{\textcolor{black}{\chi}}}}^{ \mathcolor{gray}{\omega} \textcolor{PineGreen}{\hat{2}} \textcolor{Maroon}{(2)} }_{ \textcolor{NavyBlue}{\text{eff}} \textcolor{PineGreen}{\hat{3} \hat{1}} } &\xrightarrow[]{\text{$\sim$ \bref{eq:down-132-chieff*-scalar-spectrum}}} \xint{{}^{}_{\mathcolor{gray}{-}}}{10}{\hat{g}}^{\;\! \hat{2} \textcolor{PineGreen}{\hat{2}} \textcolor{Plum}{*}}_{\;\! \mathcolor{gray}{\omega}} {\chi}^{\;\! \textcolor{PineGreen}{\hat{2}} \mathcolor{gray}{\omega} \hat{3} \hat{1} \textcolor{Plum}{*} }_{\;\! \hat{2} \textcolor{Maroon}{(2)} \textcolor{PineGreen}{\hat{3} \hat{1}} } ~ {\hat{g}}^{\;\! \mathcolor{gray}{\omega} }_{\;\! \hat{1} \textcolor{PineGreen}{\hat{1}} } ~\mathcolor{gray}{\widetilde \circ}~ {\hat{g}}^{\;\! \mathcolor{gray}{\omega} }_{\;\! \hat{3} \textcolor{PineGreen}{\hat{3}} } ~. \label{eq:down-231-chieff*-scalar-spectrum}
\end{align}
\end{subequations}

\marginLeft[-2.4em]{ssec:CW-3wavemix}\subsection{连续光三波混频 - 电场本征复振幅方程}\label{ssec:CW-3wavemix}

结合 \bref{ssec:SFG_discrete} 中,以\textcolor{Plum}{离散}个\textcolor{gray}{波长}的\textcolor{NavyBlue}{(准)连续光}\textcolor{Maroon}{和频}为代表的\textcolor{NavyBlue}{窄带}\textcolor{Maroon}{上转换}过程的电场\textcolor{PineGreen}{本征复振幅}方程 \bref{eq:simplify8-scalar-g-modulus-P-discrete} 及其\textcolor{Maroon}{下转换}版本,可以得到仅电子和光子间\textcolor{NavyBlue}{稳态}相互作用的\textcolor{Maroon}{三波混频}耦合波方程组
\begin{subequations} \label{eq:3wavemix-scalar-g-EE-discrete}
\begin{align}
	\mathcolor{gray}{\nabla_z} \xint{\begin{smallmatrix} ~ \\ {}^{}_{\mathcolor{gray}{-}} \\ ~ \end{smallmatrix}}{09}{\mathtt{g}}^{\;\! \textcolor{PineGreen}{\hat{3}}}_{\;\! \mathcolor{gray}{z}} &\xrightarrow[]{\text{\bref{eq:simplify8-scalar-g-modulus-P-discrete}}} \mathbb{i} k_{\textcolor{Maroon}{\mathsf{o}} \mathcolor{gray}{3}}^{\;\! 2} \frac{\xint{{}^{}_{\mathcolor{gray}{-}}}{10}{\hat{g}}^{\;\! \hat{3} \textcolor{PineGreen}{\hat{3}} \textcolor{Plum}{*}}_{\;\! } {\chi}^{\;\! \textcolor{PineGreen}{\hat{3}}  \hat{1} \hat{2} }_{\;\! \hat{3} \textcolor{Maroon}{(2)} \textcolor{PineGreen}{\hat{1} \hat{2}}} ~ \mathcolor{gray}{\mathcal F_{z}^{-1}} \left[ \xint{\mathcolor{gray}{-}}{18}{M}^{\;\! \mathcolor{gray}{3} \hat{1} \hat{2} }_{\;\! \hat{3} \mathcolor{gray}{k_{\symup{z}}} \textcolor{Maroon}{(2)} } \mathcolor{gray}{*} \xint{\mathcolor{gray}{-}}{25}{E}^{\;\! \textcolor{PineGreen}{\hat{1}}  }_{\;\! \hat{1} \mathcolor{gray}{z}} \mathcolor{gray}{*} \xint{\mathcolor{gray}{-}}{25}{E}^{\;\! \textcolor{PineGreen}{\hat{2}}  }_{\;\! \hat{2} \mathcolor{gray}{z}} \right]}{ 2 \lvert \xint{{}^{}_{\mathcolor{gray}{-}}}{10}{\hat{g}}^{\;\! \textcolor{PineGreen}{\hat{3}}} \rvert^2 \xint{\begin{smallmatrix} ~ \\ {}^{}_{\mathcolor{gray}{-}} \\ ~ \end{smallmatrix}}{15}{k}_{\;\! \symup{z}}^{\;\!  \textcolor{PineGreen}{\hat{3}}} \mathbb{e}^{\mathbb{i} \xint{\begin{smallmatrix} ~ \\ {}^{}_{\mathcolor{gray}{-}} \\ ~ \end{smallmatrix}}{15}{k}_{\symup{z}}^{\;\!  \textcolor{PineGreen}{\hat{3}}} \mathcolor{gray}{z}}} ~, \label{eq:up-scalar-g-EE-312-discrete} \\
	\mathcolor{gray}{\nabla_z} \xint{\begin{smallmatrix} ~ \\ {}^{}_{\mathcolor{gray}{-}} \\ ~ \end{smallmatrix}}{09}{\mathtt{g}}^{\;\! \textcolor{PineGreen}{\hat{1}}}_{\;\! \mathcolor{gray}{z}} &\xrightarrow[\text{$\sim$ \bref{eq:simplify8-scalar-g-modulus}}]{\text{$\sim$ \bref{eq:DP^(2)-1_32-spectrum-DFG7}}} \mathbb{i} k_{\textcolor{Maroon}{\mathsf{o}} \mathcolor{gray}{1}}^{\;\! 2} \frac{\xint{{}^{}_{\mathcolor{gray}{-}}}{10}{\hat{g}}^{\;\! \hat{1} \textcolor{PineGreen}{\hat{1}} \textcolor{Plum}{*}} {\chi}^{\;\! \textcolor{PineGreen}{\hat{1}}  \hat{3} \hat{2} \textcolor{Plum}{*} }_{\;\! \hat{1} \textcolor{Maroon}{(2)} \textcolor{PineGreen}{\hat{3} \hat{2}}} ~\mathcolor{gray}{\mathcal F_{z}^{-\textcolor{Plum}{*}}} \left[ \xint{\mathcolor{gray}{-}}{18}{M}^{\;\! \mathcolor{gray}{1} \hat{3} \hat{2} }_{\;\! \hat{1} \mathcolor{gray}{k_{\symup{z}}} \textcolor{Maroon}{(2)} } \mathcolor{gray}{\circ} \xint{\mathcolor{gray}{-}}{255}{E}^{\;\!\textcolor{PineGreen}{\hat{2}}  }_{\;\! \hat{2} \mathcolor{gray}{z}} \mathcolor{gray}{\circ} \xint{\mathcolor{gray}{-}}{255}{E}^{\;\! \textcolor{PineGreen}{\hat{3}} }_{\;\! \hat{3} \mathcolor{gray}{z}} \right]}{ 2 \lvert \xint{{}^{}_{\mathcolor{gray}{-}}}{10}{\hat{g}}^{\;\! \textcolor{PineGreen}{\hat{1}}} \rvert^2 \xint{\begin{smallmatrix} ~ \\ {}^{}_{\mathcolor{gray}{-}} \\ ~ \end{smallmatrix}}{15}{k}_{\;\! \symup{z}}^{\;\!  \textcolor{PineGreen}{\hat{1}}} \mathbb{e}^{\mathbb{i} \xint{\begin{smallmatrix} ~ \\ {}^{}_{\mathcolor{gray}{-}} \\ ~ \end{smallmatrix}}{15}{k}_{\symup{z}}^{\;\!  \textcolor{PineGreen}{\hat{1}}} \mathcolor{gray}{z}}} ~, \label{eq:down-scalar-g-EE-132-discrete} \\
	\mathcolor{gray}{\nabla_z} \xint{\begin{smallmatrix} ~ \\ {}^{}_{\mathcolor{gray}{-}} \\ ~ \end{smallmatrix}}{09}{\mathtt{g}}^{\;\! \textcolor{PineGreen}{\hat{2}}}_{\;\! \mathcolor{gray}{z}} &\xrightarrow[]{\text{$\sim$ \bref{eq:down-scalar-g-EE-132-discrete}}} \mathbb{i} k_{\textcolor{Maroon}{\mathsf{o}} \mathcolor{gray}{2}}^{\;\! 2} \frac{\xint{{}^{}_{\mathcolor{gray}{-}}}{10}{\hat{g}}^{\;\! \hat{2} \textcolor{PineGreen}{\hat{2}} \textcolor{Plum}{*}} {\chi}^{\;\! \textcolor{PineGreen}{\hat{2}}  \hat{3} \hat{1} \textcolor{Plum}{*} }_{\;\! \hat{2} \textcolor{Maroon}{(2)} \textcolor{PineGreen}{\hat{3} \hat{1}}} ~\mathcolor{gray}{\mathcal F_{z}^{-\textcolor{Plum}{*}}} \left[ \xint{\mathcolor{gray}{-}}{18}{M}^{\;\! \mathcolor{gray}{2} \hat{3} \hat{1} }_{\;\! \hat{2} \mathcolor{gray}{k_{\symup{z}}} \textcolor{Maroon}{(2)} } \mathcolor{gray}{\circ} \xint{\mathcolor{gray}{-}}{255}{E}^{\;\!\textcolor{PineGreen}{\hat{1}}  }_{\;\! \hat{1} \mathcolor{gray}{z}} \mathcolor{gray}{\circ} \xint{\mathcolor{gray}{-}}{255}{E}^{\;\!\textcolor{PineGreen}{\hat{3}} }_{\;\! \hat{3} \mathcolor{gray}{z}} \right]}{ 2 \lvert \xint{{}^{}_{\mathcolor{gray}{-}}}{10}{\hat{g}}^{\;\! \textcolor{PineGreen}{\hat{2}}} \rvert^2 \xint{\begin{smallmatrix} ~ \\ {}^{}_{\mathcolor{gray}{-}} \\ ~ \end{smallmatrix}}{15}{k}_{\;\! \symup{z}}^{\;\!  \textcolor{PineGreen}{\hat{2}}} \mathbb{e}^{\mathbb{i} \xint{\begin{smallmatrix} ~ \\ {}^{}_{\mathcolor{gray}{-}} \\ ~ \end{smallmatrix}}{15}{k}_{\symup{z}}^{\;\!  \textcolor{PineGreen}{\hat{2}}} \mathcolor{gray}{z}}} ~, \label{eq:down-scalar-g-EE-231-discrete}
\end{align}
\end{subequations}
注意,在 $\mathcolor{gray}{\bar{k}_{\symup{\rho}}}$ 域,默认先计算\textcolor{NavyBlue}{光场}间的\textcolor{Plum}{互相关},再计算与\textcolor{NavyBlue}{倒格波系数}的\textcolor{Plum}{互相关},即从右到左计算 \bref{eq:down-scalar-g-EE-231-discrete,eq:down-scalar-g-EE-231-discrete} 中的两个\textcolor{Plum}{互相关},以省略一对小括号。

在“\textbf{标量\textcolor{Plum}{非线性}\textcolor{NavyBlue}{波源}}(\textcolor{NavyBlue}{连续})”条件 \bref{eq:scalar_nonlinear_drive-discrete} 下,上述描述电子和光子\textcolor{NavyBlue}{窄带}相互作用的\textcolor{Maroon}{三波混频}方程组,从 \bref{eq:3wavemix-scalar-g-EE-discrete} 变为
\begin{subequations} \label{eq:3wavemix-scalar-g-EE-chieff-discrete}
\begin{align}
	\mathcolor{gray}{\nabla_z} \xint{\begin{smallmatrix} ~ \\ {}^{}_{\mathcolor{gray}{-}} \\ ~ \end{smallmatrix}}{09}{\mathtt{g}}^{\;\! \textcolor{PineGreen}{\hat{3}}}_{\;\! \mathcolor{gray}{z}} &\xrightarrow[]{\text{\bref{eq:scalar-g-modulus-P-chieff-discrete}}} \mathbb{i} k_{\textcolor{Maroon}{\mathsf{o}} \mathcolor{gray}{3}}^{\;\! 2} \frac{ \xint{{}^{}_{\mathcolor{gray}{-}}}{23}{\chi}^{ \hat{3} \textcolor{PineGreen}{\hat{3}} \textcolor{Maroon}{(2)} }_{ \textcolor{NavyBlue}{\text{eff}} \hat{1} \textcolor{PineGreen}{\hat{1}} \hat{2} \textcolor{PineGreen}{\hat{2}} } ~ \mathcolor{gray}{\mathcal F_{z}^{-1}} \left[ \xint{\mathcolor{gray}{-}}{18}{M}^{\;\! \mathcolor{gray}{3} \hat{1} \hat{2} }_{\;\! \hat{3} \mathcolor{gray}{k_{\symup{z}}} \textcolor{Maroon}{(2)} } \mathcolor{gray}{*} \xint{\mathcolor{gray}{-}}{15}{\mathtt{G}}^{\;\! \textcolor{PineGreen}{\hat{1}} }_{\;\! \mathcolor{gray}{z}} \mathcolor{gray}{*} \xint{\mathcolor{gray}{-}}{15}{\mathtt{G}}^{\;\! \textcolor{PineGreen}{\hat{2}} }_{\;\! \mathcolor{gray}{z}} \right]}{ 2 \lvert \xint{{}^{}_{\mathcolor{gray}{-}}}{10}{\hat{g}}^{\;\! \textcolor{PineGreen}{\hat{3}}} \rvert^2 \xint{\begin{smallmatrix} ~ \\ {}^{}_{\mathcolor{gray}{-}} \\ ~ \end{smallmatrix}}{15}{k}_{\;\! \symup{z}}^{\;\!  \textcolor{PineGreen}{\hat{3}}} \mathbb{e}^{\mathbb{i} \xint{\begin{smallmatrix} ~ \\ {}^{}_{\mathcolor{gray}{-}} \\ ~ \end{smallmatrix}}{15}{k}_{\symup{z}}^{\;\!  \textcolor{PineGreen}{\hat{3}}} \mathcolor{gray}{z}}} ~, \label{eq:up-scalar-g-EE-312-chieff-discrete} \\
	\mathcolor{gray}{\nabla_z} \xint{\begin{smallmatrix} ~ \\ {}^{}_{\mathcolor{gray}{-}} \\ ~ \end{smallmatrix}}{09}{\mathtt{g}}^{\;\! \textcolor{PineGreen}{\hat{1}}}_{\;\! \mathcolor{gray}{z}} &\xrightarrow[]{\text{\bref{eq:scalar-g-modulus-P-chieff*-discrete}}} \mathbb{i} k_{\textcolor{Maroon}{\mathsf{o}} \mathcolor{gray}{1}}^{\;\! 2} \frac{ \xint{{}^{}_{\mathcolor{gray}{-}}}{23}{\chi}^{ \hat{1} \textcolor{PineGreen}{\hat{1}} \textcolor{Maroon}{(2)} }_{ \textcolor{NavyBlue}{\text{eff}} \hat{3} \textcolor{PineGreen}{\hat{3}} \hat{2} \textcolor{PineGreen}{\hat{2}} } ~\mathcolor{gray}{\mathcal F_{z}^{-\textcolor{Plum}{*}}} \left[ \xint{\mathcolor{gray}{-}}{18}{M}^{\;\! \mathcolor{gray}{1} \hat{3} \hat{2} }_{\;\! \hat{1} \mathcolor{gray}{k_{\symup{z}}} \textcolor{Maroon}{(2)} } \mathcolor{gray}{\circ} \xint{\mathcolor{gray}{-}}{15}{\mathtt{G}}^{\;\! \textcolor{PineGreen}{\hat{2}} }_{\;\! \mathcolor{gray}{z}} \mathcolor{gray}{\circ} \xint{\mathcolor{gray}{-}}{15}{\mathtt{G}}^{\;\! \textcolor{PineGreen}{\hat{3}} }_{\;\! \mathcolor{gray}{z}} \right]}{ 2 \lvert \xint{{}^{}_{\mathcolor{gray}{-}}}{10}{\hat{g}}^{\;\! \textcolor{PineGreen}{\hat{1}}} \rvert^2 \xint{\begin{smallmatrix} ~ \\ {}^{}_{\mathcolor{gray}{-}} \\ ~ \end{smallmatrix}}{15}{k}_{\;\! \symup{z}}^{\;\!  \textcolor{PineGreen}{\hat{1}}} \mathbb{e}^{\mathbb{i} \xint{\begin{smallmatrix} ~ \\ {}^{}_{\mathcolor{gray}{-}} \\ ~ \end{smallmatrix}}{15}{k}_{\symup{z}}^{\;\!  \textcolor{PineGreen}{\hat{1}}} \mathcolor{gray}{z}}} ~, \label{eq:down-scalar-g-EE-132-chieff*-discrete} \\
	\mathcolor{gray}{\nabla_z} \xint{\begin{smallmatrix} ~ \\ {}^{}_{\mathcolor{gray}{-}} \\ ~ \end{smallmatrix}}{09}{\mathtt{g}}^{\;\! \textcolor{PineGreen}{\hat{2}}}_{\;\! \mathcolor{gray}{z}} &\xrightarrow[\text{\bref{eq:down-scalar-g-EE-231-discrete}}]{\text{\bref{eq:scalar_nonlinear_drive-discrete}}} \mathbb{i} k_{\textcolor{Maroon}{\mathsf{o}} \mathcolor{gray}{2}}^{\;\! 2} \frac{ \xint{{}^{}_{\mathcolor{gray}{-}}}{23}{\chi}^{ \hat{2} \textcolor{PineGreen}{\hat{2}} \textcolor{Maroon}{(2)} }_{ \textcolor{NavyBlue}{\text{eff}} \hat{3} \textcolor{PineGreen}{\hat{3}} \hat{1} \textcolor{PineGreen}{\hat{1}} } ~\mathcolor{gray}{\mathcal F_{z}^{-\textcolor{Plum}{*}}} \left[ \xint{\mathcolor{gray}{-}}{18}{M}^{\;\! \mathcolor{gray}{2} \hat{3} \hat{1} }_{\;\! \hat{2} \mathcolor{gray}{k_{\symup{z}}} \textcolor{Maroon}{(2)} } \mathcolor{gray}{\circ} \xint{\mathcolor{gray}{-}}{15}{\mathtt{G}}^{\;\! \textcolor{PineGreen}{\hat{1}} }_{\;\! \mathcolor{gray}{z}} \mathcolor{gray}{\circ} \xint{\mathcolor{gray}{-}}{15}{\mathtt{G}}^{\;\! \textcolor{PineGreen}{\hat{3}} }_{\;\! \mathcolor{gray}{z}} \right]}{ 2 \lvert \xint{{}^{}_{\mathcolor{gray}{-}}}{10}{\hat{g}}^{\;\! \textcolor{PineGreen}{\hat{2}}} \rvert^2 \xint{\begin{smallmatrix} ~ \\ {}^{}_{\mathcolor{gray}{-}} \\ ~ \end{smallmatrix}}{15}{k}_{\;\! \symup{z}}^{\;\!  \textcolor{PineGreen}{\hat{2}}} \mathbb{e}^{\mathbb{i} \xint{\begin{smallmatrix} ~ \\ {}^{}_{\mathcolor{gray}{-}} \\ ~ \end{smallmatrix}}{15}{k}_{\symup{z}}^{\;\!  \textcolor{PineGreen}{\hat{2}}} \mathcolor{gray}{z}}} ~, \label{eq:down-scalar-g-EE-231-chieff*-discrete}
\end{align}
\end{subequations}
定义了“\textbf{标量\textcolor{Plum}{非线性}\textcolor{NavyBlue}{波源}}”条件下,该\textcolor{NavyBlue}{稳态}过程的\textcolor{NavyBlue}{有效非线性系数}(三阶)张量场
\begin{subequations} \label{eq:3wavemix-chieff-discrete}
\begin{align}
	\xint{{}^{}_{\mathcolor{gray}{-}}}{23}{\chi}^{ \hat{3} \textcolor{PineGreen}{\hat{3}} \textcolor{Maroon}{(2)} }_{ \textcolor{NavyBlue}{\text{eff}} \hat{1} \textcolor{PineGreen}{\hat{1}} \hat{2} \textcolor{PineGreen}{\hat{2}} } &\xrightarrow[]{\text{\bref{eq:chieff-discrete}}} \xint{{}^{}_{\mathcolor{gray}{-}}}{10}{\hat{g}}^{\;\! \hat{3} \textcolor{PineGreen}{\hat{3}} \textcolor{Plum}{*}} {\chi}^{\;\! \hat{3} \textcolor{PineGreen}{\hat{3}} }_{\;\! \textcolor{Maroon}{(2)} \hat{1} \textcolor{PineGreen}{\hat{1}} \hat{2} \textcolor{PineGreen}{\hat{2}} } ~ {\hat{g}}_{\;\! \hat{1} \textcolor{PineGreen}{\hat{1}} } ~ {\hat{g}}_{\;\! \hat{2} \textcolor{PineGreen}{\hat{2}} } ~, \label{eq:up-312-chieff*-discrete} \\
	\xint{{}^{}_{\mathcolor{gray}{-}}}{23}{\chi}^{ \hat{1} \textcolor{PineGreen}{\hat{1}} \textcolor{Maroon}{(2)} }_{ \textcolor{NavyBlue}{\text{eff}} \hat{3} \hat{2} \textcolor{PineGreen}{\hat{3} \hat{2}} } &\xrightarrow[]{\text{\bref{eq:chieff*-discrete}}} \xint{{}^{}_{\mathcolor{gray}{-}}}{10}{\hat{g}}^{\;\! \hat{1} \textcolor{PineGreen}{\hat{1}} \textcolor{Plum}{*}} {\chi}^{\;\! \hat{1} \textcolor{PineGreen}{\hat{1}} \textcolor{Plum}{*} }_{\;\! \textcolor{Maroon}{(2)} \hat{3} \textcolor{PineGreen}{\hat{3}} \hat{2} \textcolor{PineGreen}{\hat{2}} } ~ {\hat{g}}_{\;\! \hat{3} \textcolor{PineGreen}{\hat{3}}} ~ {\hat{g}}^{\;\! \textcolor{Plum}{*}}_{\;\! \hat{2} \textcolor{PineGreen}{\hat{2}} } ~, \label{eq:down-132-chieff*-discrete} \\
	\xint{{}^{}_{\mathcolor{gray}{-}}}{23}{\chi}^{ \hat{2} \textcolor{PineGreen}{\hat{2}} \textcolor{Maroon}{(2)} }_{ \textcolor{NavyBlue}{\text{eff}} \hat{3} \textcolor{PineGreen}{\hat{3}} \hat{1} \textcolor{PineGreen}{\hat{1}} } &\xrightarrow[]{\text{$\sim$ \bref{eq:down-132-chieff*-discrete}}} \xint{{}^{}_{\mathcolor{gray}{-}}}{10}{\hat{g}}^{\;\! \hat{2} \textcolor{PineGreen}{\hat{2}} \textcolor{Plum}{*}} {\chi}^{\;\! \hat{2} \textcolor{PineGreen}{\hat{2}} \textcolor{Plum}{*} }_{\;\! \textcolor{Maroon}{(2)} \hat{3} \hat{1} \textcolor{PineGreen}{\hat{3} \hat{1}} } ~ {\hat{g}}_{\;\! \hat{3} \textcolor{PineGreen}{\hat{3}}} ~ {\hat{g}}^{\;\! \textcolor{Plum}{*}}_{\;\! \hat{1} \textcolor{PineGreen}{\hat{1}} } ~. \label{eq:down-231-chieff*-discrete}
\end{align}
\end{subequations}

进一步,再在“\textbf{标量场 $\chi^{\;\! \mathcolor{gray}{\omega} }_{\;\! \mathcolor{gray}{z} \textcolor{Maroon}{(2)}}$ \textcolor{NavyBlue}{调制}}” \bref{eq:scalar_chi2_modulation} 的额外条件下,描述电子和光子\textcolor{NavyBlue}{窄带}相互作用的\textcolor{Maroon}{三波混频}方程组从 \bref{eq:3wavemix-scalar-g-EE-chieff-discrete} 变为
\begin{subequations} \label{eq:3wavemix-scalar-g-EE-chieff-scalar-discrete}
\begin{align}
	\mathcolor{gray}{\nabla_z} \xint{\begin{smallmatrix} ~ \\ {}^{}_{\mathcolor{gray}{-}} \\ ~ \end{smallmatrix}}{09}{\mathtt{g}}^{\;\! \textcolor{PineGreen}{\hat{3}}}_{\;\! \mathcolor{gray}{z}} &\xrightarrow[]{\text{\bref{eq:scalar-g-modulus-P-chieff-scalar-discrete}}} \mathbb{i} k_{\textcolor{Maroon}{\mathsf{o}} \mathcolor{gray}{3}}^{\;\! 2} \frac{ \xint{{}^{}_{\mathcolor{gray}{-}}}{23}{\chi}^{\textcolor{PineGreen}{\hat{3}} \textcolor{Maroon}{(2)} }_{\textcolor{NavyBlue}{\text{eff}} \textcolor{PineGreen}{\hat{1}} \textcolor{PineGreen}{\hat{2}} } ~ \mathcolor{gray}{\mathcal F_{z}^{-1}} \left[ \xint{\mathcolor{gray}{-}}{18}{M}^{\;\! \mathcolor{gray}{3}}_{\;\! \mathcolor{gray}{k_{\symup{z}}} \textcolor{Maroon}{(2)} } \mathcolor{gray}{*} \xint{\mathcolor{gray}{-}}{15}{\mathtt{G}}^{\;\! \textcolor{PineGreen}{\hat{1}}}_{\;\! \mathcolor{gray}{z}} \mathcolor{gray}{*} \xint{\mathcolor{gray}{-}}{15}{\mathtt{G}}^{\;\! \textcolor{PineGreen}{\hat{2}}}_{\;\! \mathcolor{gray}{z}} \right]}{ 2 \lvert \xint{{}^{}_{\mathcolor{gray}{-}}}{10}{\hat{g}}^{\;\! \textcolor{PineGreen}{\hat{3}}} \rvert^2 \xint{\begin{smallmatrix} ~ \\ {}^{}_{\mathcolor{gray}{-}} \\ ~ \end{smallmatrix}}{15}{k}_{\;\! \symup{z}}^{\;\!  \textcolor{PineGreen}{\hat{3}}} \mathbb{e}^{\mathbb{i} \xint{\begin{smallmatrix} ~ \\ {}^{}_{\mathcolor{gray}{-}} \\ ~ \end{smallmatrix}}{15}{k}_{\symup{z}}^{\;\!  \textcolor{PineGreen}{\hat{3}}} \mathcolor{gray}{z}}} ~, \label{eq:up-scalar-g-EE-312-chieff-scalar-discrete} \\
	\mathcolor{gray}{\nabla_z} \xint{\begin{smallmatrix} ~ \\ {}^{}_{\mathcolor{gray}{-}} \\ ~ \end{smallmatrix}}{09}{\mathtt{g}}^{\;\! \textcolor{PineGreen}{\hat{1}}}_{\;\! \mathcolor{gray}{z}} &\xrightarrow[]{\text{\bref{eq:scalar-g-modulus-P-chieff*-scalar-discrete}}} \mathbb{i} k_{\textcolor{Maroon}{\mathsf{o}} \mathcolor{gray}{1}}^{\;\! 2} \frac{ \xint{{}^{}_{\mathcolor{gray}{-}}}{23}{\chi}^{ \textcolor{PineGreen}{\hat{1}} \textcolor{Maroon}{(2)} }_{\textcolor{NavyBlue}{\text{eff}} \textcolor{PineGreen}{\hat{3}} \textcolor{PineGreen}{\hat{2}} } ~ \mathcolor{gray}{\mathcal F_{z}^{-\textcolor{Plum}{*}}} \left[ \xint{\mathcolor{gray}{-}}{18}{M}^{\;\! \mathcolor{gray}{1} }_{\;\! \mathcolor{gray}{k_{\symup{z}}} \textcolor{Maroon}{(2)} } \mathcolor{gray}{\circ} \xint{\mathcolor{gray}{-}}{15}{\mathtt{G}}^{\;\! \textcolor{PineGreen}{\hat{2}} }_{\;\! \mathcolor{gray}{z}} \mathcolor{gray}{\circ} \xint{\mathcolor{gray}{-}}{15}{\mathtt{G}}^{\;\! \textcolor{PineGreen}{\hat{3}} }_{\;\! \mathcolor{gray}{z}} \right]}{ 2 \lvert \xint{{}^{}_{\mathcolor{gray}{-}}}{10}{\hat{g}}^{\;\! \textcolor{PineGreen}{\hat{1}}} \rvert^2 \xint{\begin{smallmatrix} ~ \\ {}^{}_{\mathcolor{gray}{-}} \\ ~ \end{smallmatrix}}{15}{k}_{\;\! \symup{z}}^{\;\!  \textcolor{PineGreen}{\hat{1}}} \mathbb{e}^{\mathbb{i} \xint{\begin{smallmatrix} ~ \\ {}^{}_{\mathcolor{gray}{-}} \\ ~ \end{smallmatrix}}{15}{k}_{\symup{z}}^{\;\!  \textcolor{PineGreen}{\hat{1}}} \mathcolor{gray}{z}}} ~, \label{eq:down-scalar-g-EE-132-chieff-scalar*-discrete} \\
	\mathcolor{gray}{\nabla_z} \xint{\begin{smallmatrix} ~ \\ {}^{}_{\mathcolor{gray}{-}} \\ ~ \end{smallmatrix}}{09}{\mathtt{g}}^{\;\! \textcolor{PineGreen}{\hat{2}}}_{\;\! \mathcolor{gray}{z}} &\xrightarrow[\text{\bref{eq:down-scalar-g-EE-231-chieff*-discrete}}]{\text{\bref{eq:scalar_chi2_modulation}}} \mathbb{i} k_{\textcolor{Maroon}{\mathsf{o}} \mathcolor{gray}{2}}^{\;\! 2} \frac{ \xint{{}^{}_{\mathcolor{gray}{-}}}{23}{\chi}^{ \textcolor{PineGreen}{\hat{2}} \textcolor{Maroon}{(2)} }_{\textcolor{NavyBlue}{\text{eff}} \textcolor{PineGreen}{\hat{3}} \textcolor{PineGreen}{\hat{1}} } ~ \mathcolor{gray}{\mathcal F_{z}^{-\textcolor{Plum}{*}}} \left[ \xint{\mathcolor{gray}{-}}{18}{M}^{\;\! \mathcolor{gray}{2} }_{\;\! \mathcolor{gray}{k_{\symup{z}}} \textcolor{Maroon}{(2)} } \mathcolor{gray}{\circ} \xint{\mathcolor{gray}{-}}{15}{\mathtt{G}}^{\;\! \textcolor{PineGreen}{\hat{1}} }_{\;\! \mathcolor{gray}{z}} \mathcolor{gray}{\circ} \xint{\mathcolor{gray}{-}}{15}{\mathtt{G}}^{\;\! \textcolor{PineGreen}{\hat{3}} }_{\;\! \mathcolor{gray}{z}} \right]}{ 2 \lvert \xint{{}^{}_{\mathcolor{gray}{-}}}{10}{\hat{g}}^{\;\! \textcolor{PineGreen}{\hat{2}}} \rvert^2 \xint{\begin{smallmatrix} ~ \\ {}^{}_{\mathcolor{gray}{-}} \\ ~ \end{smallmatrix}}{15}{k}_{\;\! \symup{z}}^{\;\!  \textcolor{PineGreen}{\hat{2}}} \mathbb{e}^{\mathbb{i} \xint{\begin{smallmatrix} ~ \\ {}^{}_{\mathcolor{gray}{-}} \\ ~ \end{smallmatrix}}{15}{k}_{\symup{z}}^{\;\!  \textcolor{PineGreen}{\hat{2}}} \mathcolor{gray}{z}}} ~, \label{eq:down-scalar-g-EE-231-chieff-scalar*-discrete}
\end{align}
\end{subequations}
其中,在“\textbf{标量\textcolor{Plum}{非线性}\textcolor{NavyBlue}{波源}}” \bref{eq:scalar_nonlinear_drive-discrete} 和“\textbf{标量场 $\chi^{\;\! \mathcolor{gray}{\omega} }_{\;\! \mathcolor{gray}{z} \textcolor{Maroon}{(2)}}$ \textcolor{NavyBlue}{调制}}” \bref{eq:scalar_chi2_modulation} 这 2 个条件下,电子和光子\textcolor{NavyBlue}{窄带}相互作用的\textcolor{Maroon}{三波混频}\textcolor{NavyBlue}{有效非线性系数}张量降为标量(场)
\begin{subequations} \label{eq:3wavemix-chieff-scalar-discrete}
\begin{align}
	\xint{{}^{}_{\mathcolor{gray}{-}}}{23}{\chi}^{\textcolor{PineGreen}{\hat{3}} \textcolor{Maroon}{(2)} }_{\textcolor{NavyBlue}{\text{eff}} \textcolor{PineGreen}{\hat{1}} \textcolor{PineGreen}{\hat{2}} } &\xrightarrow[]{\text{\bref{eq:chieff-scalar-discrete}}} \xint{{}^{}_{\mathcolor{gray}{-}}}{10}{\hat{g}}^{\;\! \hat{3} \textcolor{PineGreen}{\hat{3}} \textcolor{Plum}{*}}_{\;\! } {\chi}^{\;\! \textcolor{PineGreen}{\hat{3}} \hat{1} \hat{2} }_{\;\! \hat{3} \textcolor{Maroon}{(2)} \textcolor{PineGreen}{\hat{1} \hat{2}}} ~ {\hat{g}}_{\;\! \hat{1} \textcolor{PineGreen}{\hat{1}} } ~ {\hat{g}}_{\;\! \hat{2} \textcolor{PineGreen}{\hat{2}} } ~, \label{eq:up-312-chieff-scalar-discrete} \\
	\xint{{}^{}_{\mathcolor{gray}{-}}}{23}{\chi}^{ \textcolor{PineGreen}{\hat{1}} \textcolor{Maroon}{(2)} }_{\textcolor{NavyBlue}{\text{eff}} \textcolor{PineGreen}{\hat{3}} \textcolor{PineGreen}{\hat{2}} } &\xrightarrow[]{\text{\bref{eq:chieff*-scalar-discrete}}} \xint{{}^{}_{\mathcolor{gray}{-}}}{10}{\hat{g}}^{\;\! \hat{1} \textcolor{PineGreen}{\hat{1}} \textcolor{Plum}{*}}_{\;\! }  {\chi}^{\;\! \textcolor{PineGreen}{\hat{1}} \hat{3} \hat{2} \textcolor{Plum}{*} }_{\;\! \hat{1} \textcolor{Maroon}{(2)} \textcolor{PineGreen}{\hat{3} \hat{2}}} ~ {\hat{g}}_{\;\! \hat{3} \textcolor{PineGreen}{\hat{3}}} ~ {\hat{g}}^{\;\! \textcolor{Plum}{*}}_{\;\! \hat{2} \textcolor{PineGreen}{\hat{2}} } ~, \label{eq:down-132-chieff*-scalar-discrete} \\
	\xint{{}^{}_{\mathcolor{gray}{-}}}{23}{\chi}^{ \textcolor{PineGreen}{\hat{2}} \textcolor{Maroon}{(2)} }_{\textcolor{NavyBlue}{\text{eff}} \textcolor{PineGreen}{\hat{3}} \textcolor{PineGreen}{\hat{1}} } &\xrightarrow[]{\text{$\sim$ \bref{eq:down-132-chieff*-scalar-discrete}}} \xint{{}^{}_{\mathcolor{gray}{-}}}{10}{\hat{g}}^{\;\! \hat{2} \textcolor{PineGreen}{\hat{2}} \textcolor{Plum}{*}}_{\;\! } {\chi}^{\;\! \textcolor{PineGreen}{\hat{2}} \hat{3} \hat{1} \textcolor{Plum}{*} }_{\;\! \hat{2} \textcolor{Maroon}{(2)} \textcolor{PineGreen}{\hat{3} \hat{1}}} ~ {\hat{g}}_{\;\! \hat{3} \textcolor{PineGreen}{\hat{3}}} ~ {\hat{g}}^{\;\! \textcolor{Plum}{*}}_{\;\! \hat{1} \textcolor{PineGreen}{\hat{1}} } ~. \label{eq:down-231-chieff*-scalar-discrete}
\end{align}
\end{subequations}

不采用 \bref{eq:scalar_nonlinear_drive-discrete,eq:scalar_chi2_modulation} 这 2 个近似,但采用\textbf{\textcolor{NavyBlue}{泵浦未耗尽}近似条件}\bref{eq:Nondepleted-Pump-Approximation},通过 \bref{eq:up-scalar-g-EE-312-discrete-since2},\textcolor{NavyBlue}{连续光}\textcolor{Maroon}{三波混频}耦合波方程组 \bref{eq:3wavemix-scalar-g-EE-discrete} 有\textcolor{Plum}{非线性}横向\textcolor{Plum}{卷积}/\textcolor{Plum}{互相关} 4 维积分解
\begin{subequations} \label{eq:3wavemix-scalar-g-EE-discrete-Since}
\begin{align}
	&\mathllap{\xint{\begin{smallmatrix} ~ \\ {}^{}_{\mathcolor{gray}{-}} \\ ~ \end{smallmatrix}}{09}{\mathtt{g}}^{\;\! \textcolor{PineGreen}{\hat{3}}}_{\;\! \mathcolor{gray}{z}} =} \Upsilon^{\;\! \hat{3} \textcolor{PineGreen}{\hat{3}} \hat{1} \hat{2} }_{\;\! \textcolor{Maroon}{(2)} \textcolor{PineGreen}{\hat{1} \hat{2}} } \mathcolor{gray}{\iiint} \! \xint{\mathcolor{gray}{-}}{18}{M}^{\;\! \mathcolor{gray}{3} \hat{1} \hat{2} }_{\;\! \hat{3} \textcolor{Maroon}{(2)} } \left( \mathcolor{gray}{\bar{q}} \right) \mathcolor{gray}{\iint} \! \xint{{}^{}_{\mathcolor{gray}{-}}}{10}{g}^{\;\! \textcolor{PineGreen}{\hat{1}}}_{\;\! \hat{1} \mathcolor{gray}{0}} \! \left( \mathcolor{gray}{\bar{k}_{1\symup{\rho}}} \right) \xint{{}^{}_{\mathcolor{gray}{-}}}{10}{g}^{\;\! \textcolor{PineGreen}{\hat{2}}}_{\;\! \hat{2} \mathcolor{gray}{0}} \! \left( \mathcolor{gray}{\bar{k}_{2\symup{\rho}}} \right) \text{since} \left( \frac{ \Delta \xint{\begin{smallmatrix} ~ \\ {}^{}_{\mathcolor{gray}{-}} \\ ~ \end{smallmatrix}}{15}{k}_{\symup{z}}^{\;\! \textcolor{PineGreen}{\hat{1} \hat{2} \hat{3}} } \mathcolor{gray}{z} }{ 2 } \right) \mathcolor{gray}{z} ~ \mathbb{d} \mathcolor{gray}{\bar{k}_{1\symup{\rho}}} \mathbb{d} \mathcolor{gray}{\bar{q}} ~, \label{eq:up-scalar-g-EE-312-discrete-Since} \\
	&\mathllap{\xint{\begin{smallmatrix} ~ \\ {}^{}_{\mathcolor{gray}{-}} \\ ~ \end{smallmatrix}}{09}{\mathtt{g}}^{\;\! \textcolor{PineGreen}{\hat{1}}}_{\;\! \mathcolor{gray}{z}} =} \Upsilon^{\;\! \hat{1} \textcolor{PineGreen}{\hat{1}} \hat{3} \hat{2} }_{\;\! \textcolor{Maroon}{(2)} \textcolor{PineGreen}{\hat{3} \hat{2}} }  \mathcolor{gray}{\iiint} \! \xint{\mathcolor{gray}{-}}{18}{M}^{\;\! \mathcolor{gray}{1} \hat{3} \hat{2} \textcolor{Plum}{*} }_{\;\! \hat{1} \textcolor{Maroon}{(2)} } \left( \mathcolor{gray}{\bar{q}} \right) \mathcolor{gray}{\iint} \! \xint{{}^{}_{\mathcolor{gray}{-}}}{10}{g}^{\;\! \textcolor{PineGreen}{\hat{3}} }_{\;\! \hat{3} \mathcolor{gray}{0}} \! \left( \mathcolor{gray}{\bar{k}_{3\symup{\rho}}} \right) \xint{{}^{}_{\mathcolor{gray}{-}}}{10}{g}^{\;\! \textcolor{PineGreen}{\hat{2}} \textcolor{Plum}{*} }_{\;\! \hat{2} \mathcolor{gray}{0}} \! \left( \mathcolor{gray}{\bar{k}_{2\symup{\rho}}} \right) \text{since} \left( \frac{ \Delta \xint{\begin{smallmatrix} ~ \\ {}^{}_{\mathcolor{gray}{-}} \\ ~ \end{smallmatrix}}{15}{k}_{\symup{z}}^{\;\! \textcolor{PineGreen}{\hat{3} \hat{2} \hat{1}} } \mathcolor{gray}{z} }{ 2 } \right) \mathcolor{gray}{z} ~ \mathbb{d} \mathcolor{gray}{\bar{k}_{3\symup{\rho}}} \mathbb{d} \mathcolor{gray}{\bar{q}} ~, \label{eq:down-scalar-g-EE-132-discrete-Since} \\
	&\mathllap{\xint{\begin{smallmatrix} ~ \\ {}^{}_{\mathcolor{gray}{-}} \\ ~ \end{smallmatrix}}{09}{\mathtt{g}}^{\;\! \textcolor{PineGreen}{\hat{2}}}_{\;\! \mathcolor{gray}{z}} =} \Upsilon^{\;\! \hat{2} \textcolor{PineGreen}{\hat{2}} \hat{3} \hat{1} }_{\;\! \textcolor{Maroon}{(2)} \textcolor{PineGreen}{\hat{3} \hat{1}} } \mathcolor{gray}{\iiint} \! \xint{\mathcolor{gray}{-}}{18}{M}^{\;\! \mathcolor{gray}{2} \hat{3} \hat{1} \textcolor{Plum}{*} }_{\;\! \hat{2} \textcolor{Maroon}{(2)} } \left( \mathcolor{gray}{\bar{q}} \right) \mathcolor{gray}{\iint} \! \xint{{}^{}_{\mathcolor{gray}{-}}}{10}{g}^{\;\! \textcolor{PineGreen}{\hat{3}} }_{\;\! \hat{3} \mathcolor{gray}{0}} \! \left( \mathcolor{gray}{\bar{k}_{3\symup{\rho}}} \right) \xint{{}^{}_{\mathcolor{gray}{-}}}{10}{g}^{\;\! \textcolor{PineGreen}{\hat{1}} \textcolor{Plum}{*} }_{\;\! \hat{1} \mathcolor{gray}{0}} \! \left( \mathcolor{gray}{\bar{k}_{1\symup{\rho}}} \right) \text{since} \left( \frac{ \Delta \xint{\begin{smallmatrix} ~ \\ {}^{}_{\mathcolor{gray}{-}} \\ ~ \end{smallmatrix}}{15}{k}_{\symup{z}}^{\;\! \textcolor{PineGreen}{\hat{3} \hat{1} \hat{2}} } \mathcolor{gray}{z} }{ 2 } \right) \mathcolor{gray}{z} ~ \mathbb{d} \mathcolor{gray}{\bar{k}_{3\symup{\rho}}} \mathbb{d} \mathcolor{gray}{\bar{q}} ~, \label{eq:down-scalar-g-EE-231-discrete-Since}
\end{align}
\end{subequations}
其中,在 $\mathcolor{gray}{\bar{k}_{\symup{\rho}}}$ 域,定义了 \textcolor{NavyBlue}{耦合(强度)系数} 和 \textcolor{gray}{6 维}的(\textcolor{gray}{纵向})\textcolor{PineGreen}{波矢失配量}
\begin{subequations} \label{eq:3wavemix-dk}
\begin{align}
	&\Upsilon^{\;\! \hat{3} \textcolor{PineGreen}{\hat{3}} \hat{1} \hat{2} }_{\;\! \textcolor{Maroon}{(2)} \textcolor{PineGreen}{\hat{1} \hat{2}}} = \mathbb{i} k_{\textcolor{Maroon}{\mathsf{o}} \mathcolor{gray}{3}}^{\;\! 2} \frac{\xint{{}^{}_{\mathcolor{gray}{-}}}{10}{\hat{g}}^{\;\! \hat{3} \textcolor{PineGreen}{\hat{3}} \textcolor{Plum}{*}}_{\;\! } {\chi}^{\;\! \hat{3} \textcolor{PineGreen}{\hat{3}} \hat{1} \hat{2} }_{\;\! \textcolor{Maroon}{(2)} \textcolor{PineGreen}{\hat{1} \hat{2}}} }{ 2 \lvert \xint{{}^{}_{\mathcolor{gray}{-}}}{10}{\hat{g}}^{\;\! \textcolor{PineGreen}{\hat{3}}} \rvert^2 \xint{\begin{smallmatrix} ~ \\ {}^{}_{\mathcolor{gray}{-}} \\ ~ \end{smallmatrix}}{15}{k}_{\;\! \symup{z}}^{\;\!  \textcolor{PineGreen}{\hat{3}}} } ~, &&\left\{ \begin{aligned} 
		\Delta \xint{\begin{smallmatrix} ~ \\ {}^{}_{\mathcolor{gray}{-}} \\ ~ \end{smallmatrix}}{15}{k}_{\symup{z}}^{\;\! \textcolor{PineGreen}{\hat{1} \hat{2} \hat{3}} } &:= \xint{\begin{smallmatrix} ~ \\ {}^{}_{\mathcolor{gray}{-}} \\ ~ \end{smallmatrix}}{15}{k}_{\symup{z}}^{\;\! \textcolor{PineGreen}{\hat{1} \hat{2}} } - \xint{\begin{smallmatrix} ~ \\ {}^{}_{\mathcolor{gray}{-}} \\ ~ \end{smallmatrix}}{15}{k}_{\symup{z}}^{\;\! \textcolor{PineGreen}{\hat{3}} }
		\\ \xint{\begin{smallmatrix} ~ \\ {}^{}_{\mathcolor{gray}{-}} \\ ~ \end{smallmatrix}}{15}{k}_{\symup{z}}^{\;\! \textcolor{PineGreen}{\hat{1} \hat{2}} } &:= \xint{\begin{smallmatrix} ~ \\ {}^{}_{\mathcolor{gray}{-}} \\ ~ \end{smallmatrix}}{15}{k}_{\symup{z}}^{\;\! \textcolor{PineGreen}{\hat{1}} } \left( \mathcolor{gray}{\bar{k}_{1\symup{\rho}}} \right) + \xint{\begin{smallmatrix} ~ \\ {}^{}_{\mathcolor{gray}{-}} \\ ~ \end{smallmatrix}}{15}{k}_{\symup{z}}^{\;\! \textcolor{PineGreen}{\hat{2}} } \left( \mathcolor{gray}{\bar{k}_{2\symup{\rho}}} \right) + \mathcolor{gray}{q_{\symup{z}}} 
		\\ \mathcolor{gray}{\bar{k}_{2\symup{\rho}}} &:= \mathcolor{gray}{\bar{k}_{\symup{\rho}}} - \mathcolor{gray}{\bar{k}_{1\symup{\rho}}} - \mathcolor{gray}{\bar{q}_{\symup{\rho}}}
	\end{aligned} \right. ~, \label{eq:up-312-dk} \\
	&\Upsilon^{\;\! \hat{1} \textcolor{PineGreen}{\hat{1}} \hat{3} \hat{2} }_{\;\! \textcolor{Maroon}{(2)} \textcolor{PineGreen}{\hat{3} \hat{2}}} = \mathbb{i} k_{\textcolor{Maroon}{\mathsf{o}} \mathcolor{gray}{1}}^{\;\! 2} \frac{\xint{{}^{}_{\mathcolor{gray}{-}}}{10}{\hat{g}}^{\;\! \hat{1} \textcolor{PineGreen}{\hat{1}} \textcolor{Plum}{*}}_{\;\! } {\chi}^{\;\! \hat{1} \textcolor{PineGreen}{\hat{1}} \hat{3} \hat{2} \textcolor{Plum}{*} }_{\;\! \textcolor{Maroon}{(2)} \textcolor{PineGreen}{\hat{3} \hat{2}}} }{ 2 \lvert \xint{{}^{}_{\mathcolor{gray}{-}}}{10}{\hat{g}}^{\;\! \textcolor{PineGreen}{\hat{1}}} \rvert^2 \xint{\begin{smallmatrix} ~ \\ {}^{}_{\mathcolor{gray}{-}} \\ ~ \end{smallmatrix}}{15}{k}_{\;\! \symup{z}}^{\;\!  \textcolor{PineGreen}{\hat{1}}} } ~, &&\left\{ \begin{aligned} 
		\Delta \xint{\begin{smallmatrix} ~ \\ {}^{}_{\mathcolor{gray}{-}} \\ ~ \end{smallmatrix}}{15}{k}_{\symup{z}}^{\;\! \textcolor{PineGreen}{\hat{3} \hat{2} \hat{1}} } &:= \xint{\begin{smallmatrix} ~ \\ {}^{}_{\mathcolor{gray}{-}} \\ ~ \end{smallmatrix}}{15}{k}_{\symup{z}}^{\;\! \textcolor{PineGreen}{\hat{3} \hat{2}} } - \xint{\begin{smallmatrix} ~ \\ {}^{}_{\mathcolor{gray}{-}} \\ ~ \end{smallmatrix}}{15}{k}_{\symup{z}}^{\;\! \textcolor{PineGreen}{\hat{1}} }
		\\ \xint{\begin{smallmatrix} ~ \\ {}^{}_{\mathcolor{gray}{-}} \\ ~ \end{smallmatrix}}{15}{k}_{\symup{z}}^{\;\! \textcolor{PineGreen}{\hat{3} \hat{2}} } &:= \xint{\begin{smallmatrix} ~ \\ {}^{}_{\mathcolor{gray}{-}} \\ ~ \end{smallmatrix}}{15}{k}_{\symup{z}}^{\;\! \textcolor{PineGreen}{\hat{3}} } \left( \mathcolor{gray}{\bar{k}_{3\symup{\rho}}} \right) - \xint{\begin{smallmatrix} ~ \\ {}^{}_{\mathcolor{gray}{-}} \\ ~ \end{smallmatrix}}{15}{k}_{\symup{z}}^{\;\! \textcolor{PineGreen}{\hat{2}} } \left( \mathcolor{gray}{\bar{k}_{2\symup{\rho}}} \right) - \mathcolor{gray}{q_{\symup{z}}} 
		\\ \mathcolor{gray}{\bar{k}_{2\symup{\rho}}} &:= \mathcolor{gray}{\bar{k}_{3\symup{\rho}}} - \mathcolor{gray}{\bar{k}_{\symup{\rho}}} - \mathcolor{gray}{\bar{q}_{\symup{\rho}}}
	\end{aligned} \right. ~, \label{eq:down-132-dk} \\
	&\Upsilon^{\;\! \hat{2} \textcolor{PineGreen}{\hat{2}} \hat{3} \hat{1} }_{\;\! \textcolor{Maroon}{(2)} \textcolor{PineGreen}{\hat{3} \hat{1}}} = \mathbb{i} k_{\textcolor{Maroon}{\mathsf{o}} \mathcolor{gray}{2}}^{\;\! 2} \frac{\xint{{}^{}_{\mathcolor{gray}{-}}}{10}{\hat{g}}^{\;\! \hat{2} \textcolor{PineGreen}{\hat{2}} \textcolor{Plum}{*}}_{\;\! } {\chi}^{\;\! \hat{2} \textcolor{PineGreen}{\hat{2}} \hat{3} \hat{1} \textcolor{Plum}{*} }_{\;\! \textcolor{Maroon}{(2)} \textcolor{PineGreen}{\hat{3} \hat{1}}} }{ 2 \lvert \xint{{}^{}_{\mathcolor{gray}{-}}}{10}{\hat{g}}^{\;\! \textcolor{PineGreen}{\hat{2}}} \rvert^2 \xint{\begin{smallmatrix} ~ \\ {}^{}_{\mathcolor{gray}{-}} \\ ~ \end{smallmatrix}}{15}{k}_{\;\! \symup{z}}^{\;\!  \textcolor{PineGreen}{\hat{2}}} } ~, &&\left\{ \begin{aligned} 
		\Delta \xint{\begin{smallmatrix} ~ \\ {}^{}_{\mathcolor{gray}{-}} \\ ~ \end{smallmatrix}}{15}{k}_{\symup{z}}^{\;\! \textcolor{PineGreen}{\hat{3} \hat{1} \hat{2}} } &:= \xint{\begin{smallmatrix} ~ \\ {}^{}_{\mathcolor{gray}{-}} \\ ~ \end{smallmatrix}}{15}{k}_{\symup{z}}^{\;\! \textcolor{PineGreen}{\hat{3} \hat{1}} } - \xint{\begin{smallmatrix} ~ \\ {}^{}_{\mathcolor{gray}{-}} \\ ~ \end{smallmatrix}}{15}{k}_{\symup{z}}^{\;\! \textcolor{PineGreen}{\hat{2}} }
		\\ \xint{\begin{smallmatrix} ~ \\ {}^{}_{\mathcolor{gray}{-}} \\ ~ \end{smallmatrix}}{15}{k}_{\symup{z}}^{\;\! \textcolor{PineGreen}{\hat{3} \hat{1}} } &:= \xint{\begin{smallmatrix} ~ \\ {}^{}_{\mathcolor{gray}{-}} \\ ~ \end{smallmatrix}}{15}{k}_{\symup{z}}^{\;\! \textcolor{PineGreen}{\hat{3}} } \left( \mathcolor{gray}{\bar{k}_{3\symup{\rho}}} \right) - \xint{\begin{smallmatrix} ~ \\ {}^{}_{\mathcolor{gray}{-}} \\ ~ \end{smallmatrix}}{15}{k}_{\symup{z}}^{\;\! \textcolor{PineGreen}{\hat{1}} } \left( \mathcolor{gray}{\bar{k}_{1\symup{\rho}}} \right) - \mathcolor{gray}{q_{\symup{z}}} 
		\\ \mathcolor{gray}{\bar{k}_{1\symup{\rho}}} &:= \mathcolor{gray}{\bar{k}_{3\symup{\rho}}} - \mathcolor{gray}{\bar{k}_{\symup{\rho}}} - \mathcolor{gray}{\bar{q}_{\symup{\rho}}}
	\end{aligned} \right. ~, \label{eq:down-231-dk}
\end{align}
\end{subequations}
注意,严格意义上讲,在使用了\textbf{\textcolor{NavyBlue}{泵浦未耗尽}近似条件}\bref{eq:Nondepleted-Pump-Approximation} 如 $\xint{\begin{smallmatrix} ~ \\ {}^{}_{\mathcolor{gray}{-}} \\ ~ \end{smallmatrix}}{09}{\mathtt{g}}^{\;\! \textcolor{PineGreen}{\hat{1}}}_{\;\! \mathcolor{gray}{z}} \equiv \xint{\begin{smallmatrix} ~ \\ {}^{}_{\mathcolor{gray}{-}} \\ ~ \end{smallmatrix}}{09}{\mathtt{g}}^{\;\! \textcolor{PineGreen}{\hat{1}}}_{\;\! \mathcolor{gray}{0}}, \xint{\begin{smallmatrix} ~ \\ {}^{}_{\mathcolor{gray}{-}} \\ ~ \end{smallmatrix}}{09}{\mathtt{g}}^{\;\! \textcolor{PineGreen}{\hat{2}}}_{\;\! \mathcolor{gray}{z}} \equiv \xint{\begin{smallmatrix} ~ \\ {}^{}_{\mathcolor{gray}{-}} \\ ~ \end{smallmatrix}}{09}{\mathtt{g}}^{\;\! \textcolor{PineGreen}{\hat{2}}}_{\;\! \mathcolor{gray}{0}}$ 后,则 \bref{eq:3wavemix-scalar-g-EE-discrete-Since} 中的 \bref{eq:down-scalar-g-EE-231-discrete-Since,eq:down-scalar-g-EE-132-discrete-Since} 不再成立,耦合波方程组\textcolor{NavyBlue}{解耦},退化为\textcolor{NavyBlue}{无耦合的}\textcolor{NavyBlue}{连续光}\textcolor{Maroon}{和频}标量\textcolor{Maroon}{时空谱}上转换方程 \bref{eq:up-scalar-g-EE-312-discrete-since2}。故而,\bref{eq:3wavemix-scalar-g-EE-discrete-Since} 整体不成立,只旨在 3 个 \bref{eq:up-scalar-g-EE-312-discrete-Since,eq:down-scalar-g-EE-231-discrete-Since,eq:down-scalar-g-EE-132-discrete-Since} 中的某一个成立,以至于 \bref{eq:3wavemix-scalar-g-EE-discrete-Since} 整体,原则上不能称为``耦合波方程组的解"。将上述 \bref{eq:3wavemix-scalar-g-EE-discrete-Since} 中的 3 个解放在一起的原因只是完整列出,以方便参考和对比。

\marginLeft[-2.4em]{sec:LNO_that_sim_NLO}\section{\textcolor{Maroon}{Self coupling} 像线性的非线性、像非线性的线性光学 \textcolor{Maroon}{process}}\label{sec:LNO_that_sim_NLO}

在所有类型的二阶\textcolor{Plum}{非线性}光学过程中,除了上述连续波\textcolor{Maroon}{三波混频}和\textcolor{NavyBlue}{脉冲}\textcolor{Maroon}{光整流}之外,涉及\textcolor{Plum}{自耦合}\Footnote{即自身的变化率 $\mathcolor{gray}{\nabla^t} f^{\;\!\mathcolor{gray}{t}}$ 与自身的值 $f^{\;\!\mathcolor{gray}{t}}$ 相关。}的动力学过程,还有典型的\textcolor{gray}{射频}/\textcolor{gray}{微波}波段的\textcolor{Maroon}{一次电光效应}以及\textcolor{PineGreen}{折射率微调制}介质中的线性\textcolor{Maroon}{(势)散射}过程。

其中,虽然\textcolor{Maroon}{一次电光效应}属于二阶\textcolor{Plum}{非线性}过程光学过程,其终将收敛到一阶的\textcolor{PineGreen}{线性光学}:由于参与\textcolor{gray}{混频}的一些电场\textcolor{gray}{频率}\textcolor{Plum}{远低于}其他电场\textcolor{gray}{频率},使得电光效应中\textcolor{NavyBlue}{低频场}对\textcolor{NavyBlue}{高频场}的调制,起于\textcolor{Plum}{非线性}、止于\textcolor{Plum}{线性}。

相反,有一些看似是\textcolor{PineGreen}{线性光学}的过程,却由\textcolor{Plum}{非线性}波动方程统治:比如,\textcolor{NavyBlue}{泵浦}在空 $\mathcolor{gray}{\bar{r}} \in \mathcolor{gray}{\bar{\mathbb{R}}_{\textcolor{Plum}{3}}}$ 域上,因受到材料内\textcolor{NavyBlue}{微扰起伏}的\textcolor{PineGreen}{折射率}\textcolor{NavyBlue}{调制},而被线性\textcolor{Maroon}{(势)散射}。该过程就始于\textcolor{Plum}{线性}、但终于\textcolor{Plum}{非线性}。

\marginLeft[-2.4em]{ssec:EO=LO_that_sim_NO}\subsection{一次电光效应:像线性的非线性光学过程}\label{ssec:EO=LO_that_sim_NO}

为完整描述\textcolor{Maroon}{一次电光效应},需要回退至 \bref{eq:nonlinear(2)-wave_wkrho-simplify6} 的\textcolor{Maroon}{非矩阵指数}对应物
\begin{align} \label{eq:nonlinear(2)-wave_wkrho-spectrum}
	\hspace*{-2.4em} \left[ \xint{\mathcolor{gray}{-}}{32}{\bar{\bar{L}}}^{\;\! \mathcolor{gray}{\omega} \textcolor{PineGreen}{\hat{\jmath}}} + \begin{pmatrix}
	2 \xint{\begin{smallmatrix} ~ \\ {}^{}_{\mathcolor{gray}{-}} \\ ~ \end{smallmatrix}}{15}{k}_{\;\! \symup{z}}^{\;\! \mathcolor{gray}{\omega} \textcolor{PineGreen}{\hat{\jmath}}} - \mathbb{i} \mathcolor{gray}{\nabla_z} & 0 & -\textcolor{gray}{k_{\symup{x}}} \\
	0 & 2 \xint{\begin{smallmatrix} ~ \\ {}^{}_{\mathcolor{gray}{-}} \\ ~ \end{smallmatrix}}{15}{k}_{\;\! \symup{z}}^{\;\! \mathcolor{gray}{\omega} \textcolor{PineGreen}{\hat{\jmath}}} - \mathbb{i} \mathcolor{gray}{\nabla_z} & -\textcolor{gray}{k_{\symup{y}}} \\
	-\textcolor{gray}{k_{\symup{x}}} & -\textcolor{gray}{k_{\symup{x}}} & 0 \end{pmatrix} \frac{ \mathbb{i} \mathcolor{gray}{\nabla_z} }{k_{\textcolor{Maroon}{\mathsf{o}} \mathcolor{gray}{\omega}}^{\;\! 2}} \right] \xint{{}^{}_{\mathcolor{gray}{-}}}{10}{\bar{g}}^{\;\!\mathcolor{gray}{\omega} \textcolor{PineGreen}{\hat{\jmath}}}_{\;\! \mathcolor{gray}{z}}
	&\xleftarrow[\text{\bref{eq:zeta-restriction-L,eq:matrix_exp_eq-left_V-1st2nd-decompose}}]{\text{\bref{eq:plane_wave_basis-V1,eq:nonlinear(2)-wave_wkrho-simplify6-V2'-SVA}}} \frac{ - \xint{\mathcolor{gray}{-}}{25}{\bar{P}}^{\;\! \mathcolor{gray}{\omega} \textcolor{PineGreen}{\hat{\jmath}}}_{\;\! \mathcolor{gray}{z} \textcolor{Maroon}{(2)}} }{ \mathbb{e}^{\mathbb{i} \xint{\begin{smallmatrix} ~ \\ {}^{}_{\mathcolor{gray}{-}} \\ ~ \end{smallmatrix}}{15}{k}_{\symup{z}}^{\;\! \mathcolor{gray}{\omega} \textcolor{PineGreen}{\hat{\jmath}}} \mathcolor{gray}{z}} } ~.
\end{align}

当 $\xint{\mathcolor{gray}{-}}{25}{\bar{P}}^{\;\! \mathcolor{gray}{\omega} \textcolor{PineGreen}{\hat{\jmath}}}_{\;\! \mathcolor{gray}{z} \textcolor{Maroon}{(2)}}$ 中参与\textcolor{gray}{混频}的\textcolor{NavyBlue}{驱动场} $\xint{\mathcolor{gray}{-}}{25}{\bar{E}}^{\;\! \mathcolor{gray}{\omega} \textcolor{PineGreen}{\hat{\imath}}}_{\;\! \mathcolor{gray}{z}}, \xint{\mathcolor{gray}{-}}{25}{\bar{E}}^{\;\! \mathcolor{gray}{\omega} \textcolor{PineGreen}{\hat{l}}}_{\;\! \mathcolor{gray}{z}}$ 们,\textcolor{gray}{频率}\textcolor{Plum}{不匹配}程度(差异)$>$ \textcolor{Plum}{$3$ 个数量级}\Footnote{注意,即使在\textcolor{Maroon}{光整流}(包括其后续级联\textcolor{Maroon}{电光})效应中,\textcolor{NavyBlue}{光脉冲}与\textcolor{Maroon}{\text{THz}} \textcolor{NavyBlue}{脉冲}的\textcolor{gray}{中心频率}间的差异,一般来说(上限)也只有 \textcolor{Plum}{3 个数量级}。}时,\bref{eq:nonlinear(2)-wave_wkrho-spectrum} 右侧的\textcolor{Plum}{非线性}\textcolor{NavyBlue}{驱动源} $\xint{\mathcolor{gray}{-}}{25}{\bar{P}}^{\;\! \mathcolor{gray}{\omega} \textcolor{PineGreen}{\hat{\jmath}}}_{\;\! \mathcolor{gray}{z} \textcolor{Maroon}{(2)}}$(对于二阶\textcolor{Plum}{非线性}而言)来自于一个\textcolor{NavyBlue}{低频场}和\textcolor{NavyBlue}{高频场}的\textcolor{Plum}{交叠积分},二者的\textcolor{gray}{频率}\textcolor{Plum}{相加}或\textcolor{Plum}{相减}的结果,十分接近\textcolor{NavyBlue}{混频源}中,原有\textcolor{NavyBlue}{高频场}的\textcolor{gray}{频率},以至于“新产生的” $\xint{\mathcolor{gray}{-}}{25}{\bar{E}}^{\;\! \mathcolor{gray}{\omega} \textcolor{PineGreen}{\hat{\jmath}}}_{\;\! \mathcolor{gray}{z}}$ 无法从\textcolor{gray}{频率}上,与\textcolor{NavyBlue}{驱动场} $\xint{\mathcolor{gray}{-}}{25}{\bar{E}}^{\;\! \mathcolor{gray}{\omega} \textcolor{PineGreen}{\hat{\imath}}}_{\;\! \mathcolor{gray}{z}}, \xint{\mathcolor{gray}{-}}{25}{\bar{E}}^{\;\! \mathcolor{gray}{\omega} \textcolor{PineGreen}{\hat{l}}}_{\;\! \mathcolor{gray}{z}}$ 中的那个\textcolor{NavyBlue}{高频场}区分开。

这意味着相互作用的结果,或者说\textcolor{NavyBlue}{驱动源}的推动方向,不再倾向于产生\textcolor{gray}{新频率}的光(也即,\textcolor{NavyBlue}{高频场}),也就是(\textcolor{gray}{含空地})增加/改变\textcolor{gray}{新频率}的光(\textcolor{NavyBlue}{高频场})的\textcolor{PineGreen}{本征复振幅},即 $\xint{{}^{}_{\mathcolor{gray}{-}}}{10}{\bar{g}}^{\;\!\mathcolor{gray}{\omega} \textcolor{PineGreen}{\jmath}}_{\;\! \mathcolor{gray}{z}}$ 中的 $\xint{\begin{smallmatrix} ~ \\ {}^{}_{\mathcolor{gray}{-}} \\ ~ \end{smallmatrix}}{09}{\mathtt{g}}^{\;\!\mathcolor{gray}{\omega} \textcolor{PineGreen}{\jmath}}_{\;\! \mathcolor{gray}{z}}$(见 \bref{eq:vec-amp_polar}),而是倾向于(\textcolor{gray}{不含空地})改变\textcolor{gray}{旧频率}的光(\textcolor{NavyBlue}{高频场})的\textcolor{PineGreen}{本征偏振态},即 $\xint{{}^{}_{\mathcolor{gray}{-}}}{10}{\bar{g}}^{\;\!\mathcolor{gray}{\omega} \textcolor{PineGreen}{\jmath}}_{\;\! \mathcolor{gray}{z}}$ 中的 $\xint{{}^{}_{\mathcolor{gray}{-}}}{10}{\bar{g}}^{\;\!\mathcolor{gray}{\omega} \textcolor{PineGreen}{\jmath}}$。 

这立马排除了从 \bref{chap:NFCO} 中 \bref{eq:simplify8-scalar-g-conjugate} 到此处的所有标量\textcolor{Maroon}{时空谱}耦合波方程(组),来描述\textcolor{Maroon}{一次电光效应}的可能,因为这些方程的左侧都是\textcolor{gray}{新频率}的\textcolor{NavyBlue}{高频场}的\textcolor{PineGreen}{本征复振幅}\textcolor{gray}{变化率} $\mathcolor{gray}{\nabla_z} \xint{\begin{smallmatrix} ~ \\ {}^{}_{\mathcolor{gray}{-}} \\ ~ \end{smallmatrix}}{09}{\mathtt{g}}^{\;\!\mathcolor{gray}{\omega} \textcolor{PineGreen}{\hat{\jmath}}}_{\;\! \mathcolor{gray}{z}}$。究其原因,这些描述无\textcolor{NavyBlue}{低频场}参与的、\textcolor{NavyBlue}{高频场}与\textcolor{NavyBlue}{高频场}间的\textcolor{Plum}{非线性}过程,均遵循了“\textbf{\textcolor{Plum}{非线性}\textcolor{NavyBlue}{光学}\textcolor{PineGreen}{本征系统},视为\textcolor{Plum}{线性}\textcolor{NavyBlue}{光学}\textcolor{PineGreen}{本征系统}的 \textcolor{NavyBlue}{0 阶微扰}}”的假设 \bref{eq:nonlinear(2)-wave_wkrho-simplify6-L,eq:L+V-decompose,eq:L+V-decompose2,eq:L+V-decompose3,eq:nonlinear(2)-wave_wkrho-simplify6-V},以约束右侧的\textcolor{Plum}{非线性}\textcolor{NavyBlue}{驱动源} $\xint{\mathcolor{gray}{-}}{25}{\bar{P}}^{\;\! \mathcolor{gray}{\omega} \textcolor{PineGreen}{\hat{\jmath}}}_{\;\! \mathcolor{gray}{z} \textcolor{Maroon}{(2)}}$ 只贡献至左侧的\textcolor{Plum}{非线性}算子 $\xint{\mathcolor{gray}{-}}{21}{\bar{\bar{V}}}^{\;\! \mathcolor{gray}{\omega} \textcolor{PineGreen}{\hat{\jmath}}}_{\;\! \mathcolor{gray}{z}}$ 部分。

然而,对于\textcolor{Maroon}{一次电光效应}而言,上述假设(又一次\Footnote{上一次失效是在 \bref{ssec:Exp-waveq} 末。})失效。此时,\bref{eq:nonlinear(2)-wave_wkrho-simplify6} 中,方程右侧的 $\xint{\mathcolor{gray}{-}}{25}{\bar{P}}^{\;\! \mathcolor{gray}{\omega} \textcolor{PineGreen}{\hat{\jmath}}}_{\;\! \mathcolor{gray}{z} \textcolor{Maroon}{(2)}}$ 由于没有了“\textbf{\textcolor{PineGreen}{本征系统} \textcolor{NavyBlue}{0 阶微扰}}”束缚,它不再单独贡献进方程左侧的\textcolor{Plum}{非线性}算子 $\xint{\mathcolor{gray}{-}}{21}{\bar{\bar{V}}}^{\;\! \mathcolor{gray}{\omega} \textcolor{PineGreen}{\hat{\jmath}}}_{\;\! \mathcolor{gray}{z}}$ 部分,还贡献进、甚至只贡献进(由于 $\textcolor{gray}{\omega}$ 失配太大),方程左侧的\textcolor{Plum}{线性}算子 $\xint{\mathcolor{gray}{-}}{30}{\bar{\bar{L}}}^{\;\! \mathcolor{gray}{\omega} \textcolor{PineGreen}{\hat{\jmath}}}$ 部分。

此时,\textcolor{NavyBlue}{高频场}的\textcolor{PineGreen}{本征复振幅}\textcolor{gray}{变化率} $\mathcolor{gray}{\nabla_z} \xint{\begin{smallmatrix} ~ \\ {}^{}_{\mathcolor{gray}{-}} \\ ~ \end{smallmatrix}}{09}{\mathtt{g}}^{\;\!\mathcolor{gray}{\omega} \textcolor{PineGreen}{\hat{\jmath}}}_{\;\! \mathcolor{gray}{z}} \equiv 0$,这等价于 \textcolor{Plum}{非线性}算子部分
\begin{align} \label{eq:nonlinear(2)-wave_wkrho-EO-spectrum-V}
 \xint{\mathcolor{gray}{-}}{21}{\bar{\bar{V}}}^{\;\! \mathcolor{gray}{\omega} \textcolor{PineGreen}{\hat{\jmath}}}_{\;\! \mathcolor{gray}{z}} \xint{{}^{}_{\mathcolor{gray}{-}}}{10}{\bar{g}}^{\;\!\mathcolor{gray}{\omega} \textcolor{PineGreen}{\hat{\jmath}}}_{\;\! \mathcolor{gray}{z}} \equiv \bar{0}
\end{align} 
不再被驱动和起作用,可以忽略。只剩下\textcolor{Plum}{线性}算子部分 $\xint{\mathcolor{gray}{-}}{30}{\bar{\bar{L}}}^{\;\! \mathcolor{gray}{\omega} \textcolor{PineGreen}{\hat{\jmath}}} \xint{{}^{}_{\mathcolor{gray}{-}}}{10}{\bar{g}}^{\;\!\mathcolor{gray}{\omega} \textcolor{PineGreen}{\hat{\jmath}}}_{\;\! \mathcolor{gray}{z}}$,单独受\textcolor{Plum}{非线性}\textcolor{NavyBlue}{驱动源} $\xint{\mathcolor{gray}{-}}{25}{\bar{P}}^{\;\! \mathcolor{gray}{\omega} \textcolor{PineGreen}{\hat{\jmath}}}_{\;\! \mathcolor{gray}{z} \textcolor{Maroon}{(2)}}$ 调制。对应地,\bref{eq:nonlinear(2)-wave_wkrho-spectrum} 变为
\begin{subequations} \label{eq:nonlinear(2)-wave_wkrho-EO-spectrum}
\begin{align}
	\xint{\mathcolor{gray}{-}}{32}{\bar{\bar{L}}}^{\;\! \mathcolor{gray}{\omega} \textcolor{PineGreen}{\hat{1}}} \xint{{}^{}_{\mathcolor{gray}{-}}}{10}{\bar{g}}^{\;\!\mathcolor{gray}{\omega} \textcolor{PineGreen}{\hat{1}}}_{\;\! \mathcolor{gray}{z}}
	&\xleftarrow[\text{\bref{eq:nonlinear(2)-wave_wkrho-EO-spectrum-V}}]{\text{\bref{eq:nonlinear(2)-wave_wkrho-spectrum}}} - \frac{ \xint{\mathcolor{gray}{-}}{25}{\bar{P}}^{\;\! \mathcolor{gray}{\omega} \textcolor{PineGreen}{\hat{1}}}_{\;\! \mathcolor{gray}{z} \textcolor{Maroon}{(2)}} }{ \mathbb{e}^{\mathbb{i} \xint{\begin{smallmatrix} ~ \\ {}^{}_{\mathcolor{gray}{-}} \\ ~ \end{smallmatrix}}{15}{k}_{\symup{z}}^{\;\! \mathcolor{gray}{\omega} \textcolor{PineGreen}{\hat{1}}} \mathcolor{gray}{z}} }  \label{eq:nonlinear(2)-wave_wkrho-EO-spectrum1} \\
	\xint{\mathcolor{gray}{-}}{32}{\bar{\bar{L}}}^{\;\! \mathcolor{gray}{\omega} \textcolor{PineGreen}{\hat{1}}} \xint{\mathcolor{gray}{-}}{255}{\bar{E}}^{\;\! \mathcolor{gray}{\omega} \textcolor{PineGreen}{\hat{1}}}_{\;\! \mathcolor{gray}{z}} &\xleftarrow[\text{\bref{eq:vec-eigenwave}}]{\text{\bref{eq:DP^(2)-3_12-spectrum-G1}}} - \bar{\chi}^{\;\! \mathcolor{gray}{\omega} \hat{2} \hat{1} }_{\;\! \textcolor{Maroon}{(2)} } \odot \mathcolor{gray}{\mathcal F_{z}^{-1}} \left[ \xint{\mathcolor{gray}{-}}{18}{\bar{M}}^{\;\! \mathcolor{gray}{\omega} \hat{2} \hat{1} }_{\;\! \mathcolor{gray}{k_{\symup{z}}} \textcolor{Maroon}{(2)} } \mathcolor{gray}{*} \xint{\mathcolor{gray}{-}}{255}{E}_{\;\! \hat{2} \mathcolor{gray}{z}} \mathcolor{gray}{*} \xint{\mathcolor{gray}{-}}{255}{E}^{\;\! \mathcolor{gray}{\omega} \textcolor{PineGreen}{\hat{1}}}_{\;\! \hat{1} \mathcolor{gray}{z}} \right] ~, \label{eq:nonlinear(2)-wave_wkrho-EO-spectrum2}
\end{align}
\end{subequations}

明显可以看出,上述 \bref{eq:nonlinear(2)-wave_wkrho-EO-spectrum2} 涉及 $\xint{\mathcolor{gray}{-}}{255}{\bar{E}}^{\;\! \mathcolor{gray}{\omega} \textcolor{PineGreen}{\hat{1}}}_{\;\! \mathcolor{gray}{z}}$ 的\textcolor{Plum}{自耦合},并且在一般情况下,其解 $\xint{\mathcolor{gray}{-}}{255}{\bar{E}}^{\;\! \mathcolor{gray}{\omega} \textcolor{PineGreen}{\hat{1}}}_{\;\! \mathcolor{gray}{z}}$ 无法用 \bref{eq:vec-eigenwave,eq:vec-eigenwave'} 表示,并且\textcolor{PineGreen}{本征模} \bref{eq:vec-polar_phase} 连同\textcolor{PineGreen}{本征值} $\xint{\begin{smallmatrix} ~ \\ {}^{}_{\mathcolor{gray}{-}} \\ ~ \end{smallmatrix}}{15}{k}_{\symup{z}}^{\;\! \mathcolor{gray}{\omega} \textcolor{PineGreen}{\hat{1}}}$ \textcolor{PineGreen}{本征向量} $\xint{{}^{}_{\mathcolor{gray}{-}}}{10}{\bar{g}}^{\;\!\mathcolor{gray}{\omega} \textcolor{PineGreen}{\hat{1}}}$,均不再存在、\bref{eq:vec-amp_polar} 也不再成立。究其原因,抽象地说,在于\textcolor{NavyBlue}{混频场}\textcolor{gray}{频率失配}过大,\textcolor{Plum}{非线性}相互作用方式完全改变,\textcolor{Plum}{解的性质}就发生了根本性变化。具体地说,此时,在 $\bar{\bar{\bar{\chi}}}^{\;\! \mathcolor{gray}{\omega} }_{\;\! \mathcolor{gray}{z} \textcolor{Maroon}{(2)}}$ 和\textcolor{gray}{参与混频}的\textcolor{NavyBlue}{准静态}电极电场 ${E}_{\;\! \hat{2} \mathcolor{gray}{z}}$ 的\textcolor{Maroon}{正空间}\textcolor{NavyBlue}{受调制区域}的\textcolor{Plum}{并集}内,对\textcolor{Maroon}{倒空间}中的每一个\textcolor{PineGreen}{平面波}的\textcolor{PineGreen}{波阵面}/\textcolor{PineGreen}{等相位面},进行了\textcolor{Maroon}{正空间}上的\textcolor{Plum}{非均匀}\textcolor{NavyBlue}{调制}。一旦\textcolor{Maroon}{正}/\textcolor{Maroon}{倒空间}混合,以\textcolor{Maroon}{傅立叶光学}为基础的纯/全局\textcolor{Maroon}{倒空间}中运行的\textcolor{Maroon}{谱方法},就失效了。

另外,一阶\textcolor{Plum}{线性}系数与二阶\textcolor{Plum}{非线性}系数有关联。一般来说,当二阶\textcolor{Plum}{非线性}系数\textcolor{NavyBlue}{被调制}时,一阶\textcolor{Plum}{线性}系数(\textcolor{PineGreen}{介电张量})很难完全不\textcolor{NavyBlue}{被调制}\Footnote{因为\textbf{从物理上},它们都是由材料的\textcolor{NavyBlue}{电子能带结构}、\textcolor{NavyBlue}{晶格振动模式}(声子)等微观性质共同决定的。并且\textbf{从数学上},都源于材料在外加电场 $\bar{E}$ 作用下产生的宏观极化强度 $\bar{P}$ 的\textcolor{Plum}{级数项}。以至于,当通过修改/\textcolor{NavyBlue}{调制}材料的微观结构,以修改/\textcolor{NavyBlue}{调制}二阶\textcolor{Plum}{非线性}系数时,与之同宗同源的一阶\textcolor{Plum}{线性}系数也将被修改/\textcolor{NavyBlue}{调制}。},因此 \bref{eq:nonlinear(2)-wave_wkrho-EO-spectrum2} 及其所用到的假设 \bref{eq:nonlinear(2)-wave_wkrho-simplify4-2} $\to$ \bref{eq:nonlinear(2)-wave_wkrho-simplify5},原则上(因理想化)均不成立。而一旦一阶\textcolor{Plum}{线性}系数(\textcolor{PineGreen}{介电张量})被或强或弱地\textcolor{NavyBlue}{调制},则在\textcolor{PineGreen}{线性光学}层面,要么引入了类\textcolor{Maroon}{光子晶体}结构(\textcolor{Plum}{深}\textcolor{NavyBlue}{调制}),要么引入了\textcolor{NavyBlue}{散射势}(\textcolor{Plum}{浅}\textcolor{NavyBlue}{调制},见下文),此时 \bref{eq:nonlinear(2)-wave_wkrho-EO-spectrum2} 中的\textcolor{Plum}{线性}算子 $\xint{\mathcolor{gray}{-}}{30}{\bar{\bar{L}}}^{\;\! \mathcolor{gray}{\omega} \textcolor{PineGreen}{\hat{\jmath}}}$ 将含\textcolor{Plum}{卷积}\Footnote{且也将含 $\mathcolor{gray}{z}$,以至于它将与\textcolor{Plum}{非线性}算子 $\xint{\mathcolor{gray}{-}}{21}{\bar{\bar{V}}}^{\;\! \mathcolor{gray}{\omega} \textcolor{PineGreen}{\hat{\jmath}}}_{\;\! \mathcolor{gray}{z}}$ 同等地位,并无法再称之为\textcolor{Plum}{线性}算子。}(\bref{eq:simplify7-L2-zeta} $\to$ \bref{eq:nonlinear(2)-wave_wkrho-simplify3}),\textcolor{Plum}{自耦合}程度还会加深。

为了 \bref{eq:nonlinear(2)-wave_wkrho-EO-spectrum2} 的\textcolor{Plum}{解析}发展,先取消对 $\bar{\bar{\bar{\chi}}}^{\;\! \mathcolor{gray}{\omega} }_{\;\! \mathcolor{gray}{z} \textcolor{Maroon}{(2)}}$ 的\textcolor{NavyBlue}{调制} $\bar{\bar{\bar{M}}}^{\;\! \mathcolor{gray}{\omega} }_{\;\! \mathcolor{gray}{z} \textcolor{Maroon}{(2)}}$,使得材料对外加电磁场的\textcolor{Plum}{线性}和\textcolor{Plum}{非线性}响应层面,都是\textcolor{Plum}{均匀}的。此时,尽管该条件下的 \bref{eq:nonlinear(2)-wave_wkrho-EO-spectrum2} 不再矛盾,但它仍然是\textcolor{Plum}{几乎无解}的。为此,继续规定\textcolor{NavyBlue}{交直流}电场 ${E}_{\;\! \hat{2} \mathcolor{gray}{z}} \equiv {E}_{\;\! \hat{2}}$ 强度在\textcolor{Maroon}{正空间}中\textcolor{Plum}{均匀}分布。此时,方程右侧的所有\textcolor{Plum}{卷积算符} $\mathcolor{gray}{*}$ 全都消失,\bref{eq:nonlinear(2)-wave_wkrho-EO-spectrum2} 退化为
\begin{subequations} \label{eq:eq:nonlinear(2)-wave_wkrho-EO-bulk-spectrum}
\begin{align}
	\xint{\mathcolor{gray}{-}}{32}{\bar{\bar{L}}}^{\;\! \mathcolor{gray}{\omega} \textcolor{PineGreen}{\hat{1}}} \xint{\mathcolor{gray}{-}}{255}{\bar{E}}^{\;\! \mathcolor{gray}{\omega} \textcolor{PineGreen}{\hat{1}}}_{\;\! \mathcolor{gray}{z}} &= - \bar{\chi}^{\;\! \mathcolor{gray}{\omega} \hat{2} \hat{1} }_{\;\! \textcolor{Maroon}{(2)} } {E}_{\;\! \hat{2} } \xint{\mathcolor{gray}{-}}{255}{E}^{\;\! \mathcolor{gray}{\omega} \textcolor{PineGreen}{\hat{1}}}_{\;\! \hat{1} \mathcolor{gray}{z}} \label{eq:nonlinear(2)-wave_wkrho-EO-bulk-spectrum1} \\
	\left( \xint{\mathcolor{gray}{-}}{32}{\bar{\bar{L}}}^{\;\! \mathcolor{gray}{\omega} \textcolor{PineGreen}{\hat{1}}} + \bar{\bar{\Delta}}^{\;\! \mathcolor{gray}{\omega} }_{\;\! \textcolor{Maroon}{(2)} } \right) \xint{\mathcolor{gray}{-}}{255}{\bar{E}}^{\;\! \mathcolor{gray}{\omega} \textcolor{PineGreen}{\hat{1}}}_{\;\! \mathcolor{gray}{z}} &\xleftarrow[]{ \bar{\Delta}^{\;\! \mathcolor{gray}{\omega} \hat{1} }_{\;\! \textcolor{Maroon}{(2)} } := \bar{\chi}^{\;\! \mathcolor{gray}{\omega} \hat{2} \hat{1} }_{\;\! \textcolor{Maroon}{(2)} } {E}_{\;\! \hat{2} } } \bar{0} \label{eq:nonlinear(2)-wave_wkrho-EO-bulk-spectrum2} \\
	\left( \xint{\mathcolor{gray}{-}}{32}{\bar{\bar{L}}}^{\;\! \mathcolor{gray}{\omega} \textcolor{PineGreen}{\hat{1}}} + \bar{\bar{\Delta}}^{\;\! \mathcolor{gray}{\omega} }_{\;\! \textcolor{Maroon}{(2)} } \right) \xint{{}^{}_{\mathcolor{gray}{-}}}{10}{\bar{g}}^{\;\!\mathcolor{gray}{\omega} \textcolor{PineGreen}{\hat{1}}} &\xleftarrow[]{\text{\bref{eq:vec-eigenwave'}}} \bar{0} ~, \label{eq:nonlinear(2)-wave_wkrho-EO-bulk-spectrum3}
\end{align}
\end{subequations}
其中,定义了\textbf{\textcolor{Maroon}{一次电光}\textcolor{PineGreen}{介电张量}} $\bar{\bar{\Delta}}^{\;\! \mathcolor{gray}{\omega} }_{\;\! \textcolor{Maroon}{(2)}}$,即 \textcolor{Plum}{线性}算子 $\xint{\mathcolor{gray}{-}}{30}{\bar{\bar{L}}}^{\;\! \mathcolor{gray}{\omega} \textcolor{PineGreen}{\hat{\jmath}}} = \left( \xint{\begin{smallmatrix} ~ \\ {}^{}_{\mathcolor{gray}{-}} \\ ~ \end{smallmatrix}}{15}{\bar{k}}_{\;\! \mathcolor{gray}{\omega}}^{\;\! \textcolor{PineGreen}{\jmath}} \xint{\begin{smallmatrix} ~ \\ {}^{}_{\mathcolor{gray}{-}} \\ ~ \end{smallmatrix}}{15}{\bar{k}}_{\;\! \mathcolor{gray}{\omega}}^{\;\! \textcolor{PineGreen}{\jmath} {\mathsf{\textcolor{Plum}{T}}}} - \xint{\begin{smallmatrix} ~ \\ {}^{}_{\mathcolor{gray}{-}} \\ ~ \end{smallmatrix}}{15}{\bar{k}}_{\;\! \mathcolor{gray}{\omega}}^{\;\! \textcolor{PineGreen}{\jmath} {\mathsf{\textcolor{Plum}{T}}}} \xint{\begin{smallmatrix} ~ \\ {}^{}_{\mathcolor{gray}{-}} \\ ~ \end{smallmatrix}}{15}{\bar{k}}_{\;\! \mathcolor{gray}{\omega}}^{\;\! \textcolor{PineGreen}{\jmath}} \right) \big/ k_{\textcolor{Maroon}{\mathsf{o}} \mathcolor{gray}{\omega}}^{\;\! 2} + \bar{\bar{\varepsilon}}^{\;\! \mathcolor{gray}{\omega}}_{\textcolor{Maroon}{(1)}}$ 的\textcolor{Plum}{变化量} $\Delta \xint{\mathcolor{gray}{-}}{30}{\bar{\bar{L}}}^{\;\! \mathcolor{gray}{\omega} \textcolor{PineGreen}{\hat{\jmath}}}$,也即 $\xint{\mathcolor{gray}{-}}{30}{\bar{\bar{L}}}^{\;\! \mathcolor{gray}{\omega} \textcolor{PineGreen}{\hat{\jmath}}}$ 中\textcolor{PineGreen}{介电张量} $\bar{\bar{\varepsilon}}^{\;\! \mathcolor{gray}{\omega}}_{\textcolor{Maroon}{(1)}}$ 的\textcolor{Plum}{变化量} $\Delta \bar{\bar{\varepsilon}}^{\;\! \mathcolor{gray}{\omega}}_{\textcolor{Maroon}{(1)}}$,均为 ${\Delta}^{\;\! \mathcolor{gray}{\omega} \hat{2} \hat{1} }_{\;\! \textcolor{Maroon}{(2)} } := {\chi}^{\;\! \mathcolor{gray}{\omega} \hat{3} \hat{2} \hat{1} }_{\;\! \textcolor{Maroon}{(2)} } {E}_{\;\! \hat{3} } = \Delta \xint{\mathcolor{gray}{-}}{30}{\bar{\bar{L}}}^{\;\! \mathcolor{gray}{\omega} \textcolor{PineGreen}{\hat{\jmath}}} = \Delta \bar{\bar{\varepsilon}}^{\;\! \mathcolor{gray}{\omega}}_{\textcolor{Maroon}{(1)}}$。

从 \bref{eq:nonlinear(2)-wave_wkrho-EO-bulk-spectrum3} 可以看出,由于参与混频的电极电场\textcolor{gray}{频率}大多处于\textcolor{NavyBlue}{直流}/\textcolor{NavyBlue}{交流}/\textcolor{NavyBlue}{射频}/\textcolor{NavyBlue}{微波}波段\Footnote{即使\textcolor{gray}{调制}/\textcolor{gray}{载波}\textcolor{gray}{频率}很高,也远低于\textcolor{gray}{光场频率},甚至远低于\textcolor{gray}{太赫兹频率},因此它们的\textcolor{NavyBlue}{电场性质}非常不同。},(远)低于\textcolor{NavyBlue}{太赫兹}、\textcolor{NavyBlue}{近红外}、\textcolor{NavyBlue}{可见光}波段,导致\textcolor{NavyBlue}{直流}/\textcolor{NavyBlue}{射频}/\textcolor{NavyBlue}{微波}波段与\textcolor{NavyBlue}{光波段}的混频结果,大多只是在以\textcolor{NavyBlue}{直流}/\textcolor{NavyBlue}{射频}/\textcolor{NavyBlue}{微波}\textcolor{gray}{频率}改变\textcolor{NavyBlue}{光波段}的偏振态,而并未产生\textcolor{gray}{新频率}的\textcolor{NavyBlue}{光}(除了从\textcolor{Maroon}{电光频率梳}的意义上来讲\Footnote{偏振态在时域上的动态调制,对信号会展开为时域上多周期的相位调制,最终在频域上等价于产生了新频率。所以两者并不冲突。})。

\marginLeft[-2.4em]{ssec:holo_L=NO_that_sim_LO}\subsection{折射率微扰势散射:像非线性的线性光学过程}\label{ssec:holo_L=NO_that_sim_LO}

回到 \bref{ssec:Exp-waveq} 中,\textbf{\textcolor{Plum}{线性}材料系数\textcolor{NavyBlue}{弱调制}条件}下的 \bref{eq:nonlinear(2)-wave_wkrho-simplify4}。当时,我们继续往下推导时,忽略了这个方程右侧的\textcolor{NavyBlue}{势散射项} $\xint{\mathcolor{gray}{-}}{25}{\tilde{P}}^{\;\! \mathcolor{gray}{\omega} }_{\;\! \mathcolor{gray}{z} \textcolor{Maroon}{(1)}}$ \bref{eq:nonlinear(2)-wave_wkrho-simplify4-2},以集中精力处理方程右侧的二阶\textcolor{Plum}{非线性}\textcolor{NavyBlue}{波源}项 $\xint{\mathcolor{gray}{-}}{25}{\bar{P}}^{\;\! \mathcolor{gray}{\omega} }_{\;\! \mathcolor{gray}{z} \textcolor{Maroon}{(2)}}$。现在,反过来地,在这 2 个\textcolor{NavyBlue}{驱动源}中,选择性地忽略掉\textcolor{Plum}{非线性}\textcolor{NavyBlue}{散射源} $\xint{\mathcolor{gray}{-}}{25}{\bar{P}}^{\;\! \mathcolor{gray}{\omega} }_{\;\! \mathcolor{gray}{z} \textcolor{Maroon}{(2)}}$,所产生的散射式\textcolor{PineGreen}{谐波}生成;只考察\textcolor{Plum}{线性}\textcolor{NavyBlue}{势散射源} $\xint{\mathcolor{gray}{-}}{25}{\tilde{P}}^{\;\! \mathcolor{gray}{\omega} }_{\;\! \mathcolor{gray}{z} \textcolor{Maroon}{(1)}}$,对\textcolor{PineGreen}{基波}散射的影响。为描述\textcolor{PineGreen}{介电张量} $\bar{\bar{\varepsilon}}^{\;\! \mathcolor{gray}{\omega}}_{\textcolor{Maroon}{(1)}}$ 的 3D \textcolor{NavyBlue}{微扰} $\tilde{\tilde{\varepsilon}}^{\;\! \mathcolor{gray}{\omega}}_{\mathcolor{gray}{z} \textcolor{Maroon}{(1)}}$ 所引起的\textcolor{Maroon}{势散射过程},应回退至无 $\xint{\mathcolor{gray}{-}}{25}{\bar{P}}^{\;\! \mathcolor{gray}{\omega} }_{\;\! \mathcolor{gray}{z} \textcolor{Maroon}{(2)}}$ 的 \bref{eq:nonlinear(2)-wave_wkrho-simplify4} 的\textcolor{Maroon}{非矩阵指数}对应物
\begin{align} \label{eq:linear_potential_scattering}
	\xint{\mathcolor{gray}{-}}{32}{\bar{\bar{L}}}^{\;\! \mathcolor{gray}{\omega} } \xint{\mathcolor{gray}{-}}{295}{\bar{E}}^{\;\! \mathcolor{gray}{\omega} }_{\;\! \mathcolor{gray}{z}} + \xint{\mathcolor{gray}{-}}{21}{\bar{\bar{V}}}^{\;\! \mathcolor{gray}{\omega} \textcolor{PineGreen}{\hat{1}}}_{\;\! \mathcolor{gray}{z}} \xint{{}^{}_{\mathcolor{gray}{-}}}{10}{\bar{g}}^{\;\!\mathcolor{gray}{\omega} \textcolor{PineGreen}{\hat{1}}}_{\;\! \mathcolor{gray}{z}} \cdot \mathbb{e}^{\mathbb{i} \xint{\begin{smallmatrix} ~ \\ {}^{}_{\mathcolor{gray}{-}} \\ ~ \end{smallmatrix}}{15}{k}_{\symup{z}}^{\;\! \mathcolor{gray}{\omega} \textcolor{PineGreen}{\hat{1}}} \mathcolor{gray}{z}} &\xleftarrow[\text{\bref{eq:simplify7-L2-zeta}}]{\text{\bref{eq:nonlinear(2)-wave_wkrho-simplify4}}} - \xint{\begin{smallmatrix} ~ \\ {}^{}_{\mathcolor{gray}{-}} \\ ~ \end{smallmatrix}}{16}{\tilde{\tilde{\varepsilon}}}^{\;\! \mathcolor{gray}{\omega}}_{\mathcolor{gray}{z} \textcolor{Maroon}{(1)}} \mathcolor{gray}{*} \xint{\mathcolor{gray}{-}}{295}{\bar{E}}^{\;\! \mathcolor{gray}{\omega} \textcolor{PineGreen}{\hat{2}}}_{\;\! \mathcolor{gray}{z}} ~,
\end{align}
同样,这是一个\textcolor{Plum}{自耦合}的 3D 波动方程\Footnote{怎么理解该方程的角标:\textcolor{NavyBlue}{背景场} $\xint{\mathcolor{gray}{-}}{255}{\bar{E}}^{\;\! \mathcolor{gray}{\omega} \textcolor{PineGreen}{\hat{2}}}_{\;\! \textcolor{Plum}{(0)} \mathcolor{gray}{z}}$、(各阶)\textcolor{NavyBlue}{源场} $\xint{\mathcolor{gray}{-}}{255}{\bar{E}}^{\;\! \mathcolor{gray}{\omega} \textcolor{PineGreen}{\hat{2}}}_{\;\! \mathcolor{gray}{z}}$ 的\textcolor{PineGreen}{本征偏振态}(用 $\textcolor{PineGreen}{\hat{2}}$ 表示),与(各阶)\textcolor{Plum}{新增}\textcolor{NavyBlue}{散射场} $\xint{\mathcolor{gray}{-}}{255}{\tilde{E}}^{\;\! \mathcolor{gray}{\omega} \textcolor{PineGreen}{\hat{1}}}_{\;\! \mathcolor{gray}{z}} := \Delta \xint{\mathcolor{gray}{-}}{255}{\bar{E}}^{\;\! \mathcolor{gray}{\omega} \textcolor{PineGreen}{\hat{1}}}_{\;\! \mathcolor{gray}{z}}$ 的\textcolor{PineGreen}{本征偏振态}(用 $\textcolor{PineGreen}{\hat{1}}$ 表示)不一定相同,但都近似满足\textcolor{PineGreen}{特征方程} $\xint{\mathcolor{gray}{-}}{30}{\bar{\bar{L}}}^{\;\! \mathcolor{gray}{\omega} } \xint{\mathcolor{gray}{-}}{255}{\bar{E}}^{\;\! \mathcolor{gray}{\omega} }_{\;\! \mathcolor{gray}{z}} \equiv \bar{0}$。},类似线性\textcolor{Maroon}{光子晶体}内,磁场主方程的\textcolor{PineGreen}{本征模问题}\cite{sakodaOpticalPropertiesPhotonic2005,joannopoulosPhotonicCrystalsMolding2008},以及相关的 \textcolor{Maroon}{严格耦合波分析}(\textcolor{Maroon}{Rigorous Coupled Wave Analysis, RCWA})/\textcolor{Maroon}{傅里叶模态法}(\textcolor{Maroon}{Fourier Modal Method, FMM})方法\cite{liNoteSmatrixPropagation2003,wangLargeScaleMetasurfaceSimulation2025}、Takagi-Taupin 耦合模理论。但 \bref{eq:linear_potential_scattering} 中的介电常数起伏/对比度,通常没有\textcolor{Maroon}{光子晶体}那么高/大,因为采用了\textbf{\textcolor{Plum}{线性}材料系数\textcolor{NavyBlue}{弱调制}条件}\bref{eq:weak_modulated_varepsilon}。因此,从某种程度上,该\textcolor{Maroon}{势散射问题},相对于\textcolor{Maroon}{光子晶体}情形,是\textcolor{Plum}{弱(自)耦合}的。

由于上述\textcolor{Plum}{弱耦合}的\textcolor{Plum}{非线性}过程,接近\textcolor{Plum}{无耦合}的纯\textcolor{Plum}{线性}过程(比如若将 \bref{eq:linear_potential_scattering} 中的 $\xint{\begin{smallmatrix} ~ \\ {}^{}_{\mathcolor{gray}{-}} \\ ~ \end{smallmatrix}}{16}{\tilde{\tilde{\varepsilon}}}^{\;\! \mathcolor{gray}{\omega}}_{\mathcolor{gray}{z} \textcolor{Maroon}{(1)}}$ 设为 $\bar{\bar{0}}$,则 $\xint{\mathcolor{gray}{-}}{21}{\bar{\bar{V}}}^{\;\! \mathcolor{gray}{\omega} \textcolor{PineGreen}{\hat{1}}}_{\;\! \mathcolor{gray}{z}} \xint{{}^{}_{\mathcolor{gray}{-}}}{10}{\bar{g}}^{\;\!\mathcolor{gray}{\omega} \textcolor{PineGreen}{\hat{1}}}_{\;\! \mathcolor{gray}{z}}$ 也跟着消失),这启发我们将 \bref{eq:linear_potential_scattering} 的解 $\xint{\mathcolor{gray}{-}}{255}{\bar{E}}^{\;\! \mathcolor{gray}{\omega} }_{\;\! \mathcolor{gray}{z}}$ 视为 \bref{eq:simplify7-L-zeta} 的解 $\xint{\mathcolor{gray}{-}}{255}{\bar{E}}^{\;\! \mathcolor{gray}{\omega} \textcolor{PineGreen}{\hat{2}}}_{\;\! \textcolor{Plum}{(0)} \mathcolor{gray}{z}}$ 的\textcolor{NavyBlue}{微扰}。

\textcolor{Plum}{1} 阶\textcolor{NavyBlue}{微扰}(的结果)$\xint{\mathcolor{gray}{-}}{255}{\bar{E}}^{\;\! \mathcolor{gray}{\omega} }_{\;\! \textcolor{Plum}{(1)} \mathcolor{gray}{z}} = {}^{}_{\textcolor{PineGreen}{\hat{2}}} \xint{\mathcolor{gray}{-}}{255}{\bar{E}}^{\;\! \mathcolor{gray}{\omega} \textcolor{PineGreen}{\hat{2}}}_{\;\! \textcolor{Plum}{(0)} \mathcolor{gray}{z}} + {}^{}_{\textcolor{PineGreen}{\hat{1}}} \xint{\mathcolor{gray}{-}}{255}{\tilde{E}}^{\;\! \mathcolor{gray}{\omega} \textcolor{PineGreen}{\hat{1}}}_{\;\! \textcolor{Plum}{(0)} \mathcolor{gray}{z}}$ 对应 \textcolor{Plum}{1} 阶 \textcolor{Maroon}{Born 近似}\cite{PrinciplesOptics7th,gerkeAperiodicVolumeOptics2010},其中,(相对于\textcolor{NavyBlue}{背景场} $\xint{\mathcolor{gray}{-}}{255}{\bar{E}}^{\;\! \mathcolor{gray}{\omega} \textcolor{PineGreen}{\hat{2}}}_{\;\! \textcolor{Plum}{(0)} \mathcolor{gray}{z}}$)\textcolor{Plum}{新增}的 \textcolor{Plum}{1} 阶\textcolor{NavyBlue}{散射场} $\xint{\mathcolor{gray}{-}}{255}{\tilde{E}}^{\;\! \mathcolor{gray}{\omega} \textcolor{PineGreen}{\hat{1}}}_{\;\! \textcolor{Plum}{(0)} \mathcolor{gray}{z}} := \Delta \xint{\mathcolor{gray}{-}}{255}{\bar{E}}^{\;\! \mathcolor{gray}{\omega} \textcolor{PineGreen}{\hat{1}}}_{\;\! \textcolor{Plum}{(0)} \mathcolor{gray}{z}}$,类似 \bref{eq:weak_modulated_linear_sus} 处\Footnote{注意甄别,$\tilde{\cdot}$ 本身只代表对变量 $\cdot$ 的\textcolor{NavyBlue}{微扰} $\delta \cdot$,与变量 $\cdot$ 是否是\textcolor{Plum}{标}/\textcolor{Plum}{矢}/\textcolor{Plum}{张量}无关,正如 \bref{eq:weak_modulated_linear_sus} 处的一样,需要通过上下文\textcolor{Plum}{数学语境},来判断被修饰的变量 $\cdot$ 的\textcolor{Plum}{标}/\textcolor{Plum}{矢}/\textcolor{Plum}{张量}情况。只是这里恰好,$\tilde{\cdot},\tilde{\tilde \cdot}$ 同时表示了对\textcolor{Plum}{矢量}、\textcolor{Plum}{张量}的\textcolor{NavyBlue}{微扰}(严格来说,前者是(被动的)“\textcolor{Plum}{微小增量} $\Delta$”,后者才是(主动的)“\textcolor{NavyBlue}{微扰}”)。}。\textcolor{Plum}{2} 阶\textcolor{NavyBlue}{微扰}(的结果)$\xint{\mathcolor{gray}{-}}{255}{\bar{E}}^{\;\! \mathcolor{gray}{\omega} }_{\;\! \textcolor{Plum}{(2)} \mathcolor{gray}{z}} = {}^{}_{\textcolor{PineGreen}{\hat{2}}} \xint{\mathcolor{gray}{-}}{255}{\bar{E}}^{\;\! \mathcolor{gray}{\omega} \textcolor{PineGreen}{\hat{2}}}_{\;\! \textcolor{Plum}{(0)} \mathcolor{gray}{z}} + {}^{}_{\textcolor{PineGreen}{\hat{1}}} \xint{\mathcolor{gray}{-}}{255}{\tilde{E}}^{\;\! \mathcolor{gray}{\omega} \textcolor{PineGreen}{\hat{1}}}_{\;\! \textcolor{Plum}{(1)} \mathcolor{gray}{z}}$ 对应 \textcolor{Plum}{2} 阶 \textcolor{Maroon}{Born 近似}\Footnote{注意,\textcolor{Plum}{2} 阶\textcolor{NavyBlue}{散射场} $\xint{\mathcolor{gray}{-}}{255}{\tilde{E}}^{\;\! \mathcolor{gray}{\omega} \textcolor{PineGreen}{\hat{1}}}_{\;\! \textcolor{Plum}{(2)} \mathcolor{gray}{z}} := \Delta \xint{\mathcolor{gray}{-}}{255}{\bar{E}}^{\;\! \mathcolor{gray}{\omega} \textcolor{PineGreen}{\hat{1}}}_{\;\! \textcolor{Plum}{(0)} \mathcolor{gray}{z}}$ 仍然是相对于 \textcolor{NavyBlue}{背景场} $\xint{\mathcolor{gray}{-}}{255}{\bar{E}}^{\;\! \mathcolor{gray}{\omega} \textcolor{PineGreen}{\hat{2}}}_{\;\! \textcolor{Plum}{(0)} \mathcolor{gray}{z}}$,而不是相对于 \textcolor{Plum}{1} 阶\textcolor{NavyBlue}{总场} $\xint{\mathcolor{gray}{-}}{255}{\bar{E}}^{\;\! \mathcolor{gray}{\omega} }_{\;\! \textcolor{Plum}{(1)} \mathcolor{gray}{z}}$ \textcolor{Plum}{新增}的。}。(\textcolor{Plum}{1}$\sim$)\textcolor{Plum}{n} 阶\textcolor{NavyBlue}{微扰}对应(至)\textcolor{Plum}{n} 阶 \textcolor{Maroon}{Born 近似}。这使得 \bref{eq:linear_potential_scattering} 的解 $\xint{\mathcolor{gray}{-}}{255}{\bar{E}}^{\;\! \mathcolor{gray}{\omega} }_{\;\! \mathcolor{gray}{z}}$,可以被表示为:
\begin{subequations} \label{eq:L_scatter_sol-Born_approx}
\begin{align}
	\xint{\mathcolor{gray}{-}}{295}{\bar{E}}^{\;\! \mathcolor{gray}{\omega} }_{\;\! \textcolor{Plum}{(n)} \mathcolor{gray}{z}} &\xrightarrow[]{\text{\textbf{第 \textcolor{Plum}{$n$} 阶 \textcolor{Maroon}{Born 近似}}}} \xint{\mathcolor{gray}{-}}{295}{\bar{E}}^{\;\! \mathcolor{gray}{\omega} }_{\;\! \textcolor{Plum}{(0)} \mathcolor{gray}{z}} + \Delta \xint{\mathcolor{gray}{-}}{295}{\bar{E}}^{\;\! \mathcolor{gray}{\omega} }_{\;\! \textcolor{Plum}{(n-1)} \mathcolor{gray}{z}} \label{eq:L_scatter_sol-Born_approx1} \\
 	&\xrightarrow[]{\text{\textbf{第 \textcolor{Plum}{$n-1$} 阶 \textcolor{Maroon}{Born 近似}}}} \xint{\mathcolor{gray}{-}}{295}{\bar{E}}^{\;\! \mathcolor{gray}{\omega} }_{\;\! \textcolor{Plum}{(0)} \mathcolor{gray}{z}} + \Delta \left( \xint{\mathcolor{gray}{-}}{295}{\bar{E}}^{\;\! \mathcolor{gray}{\omega} }_{\;\! \textcolor{Plum}{(0)} \mathcolor{gray}{z}} + \Delta \xint{\mathcolor{gray}{-}}{295}{\bar{E}}^{\;\! \mathcolor{gray}{\omega} }_{\;\! \textcolor{Plum}{(n-2)} \mathcolor{gray}{z}} \right) \label{eq:L_scatter_sol-Born_approx2} \\
 	&\xrightarrow[]{...} ... \xrightarrow[]{\text{\textbf{第 \textcolor{Plum}{$1$} 阶 \textcolor{Maroon}{Born 近似}}}} \sum_{\textcolor{Plum}{i} = \textcolor{Plum}{0}}^{\textcolor{Plum}{n}} \Delta^{\textcolor{Plum}{i}} \xint{\mathcolor{gray}{-}}{295}{\bar{E}}^{\;\! \mathcolor{gray}{\omega} }_{\;\! \textcolor{Plum}{(0)} \mathcolor{gray}{z}} \label{eq:L_scatter_sol-Born_approx3} ~,
\end{align}
\end{subequations}
其中,前 $\textcolor{Plum}{i}$ 阶 \textcolor{Maroon}{Born 近似} 所产生的前 $\textcolor{Plum}{i}$ 次\textcolor{NavyBlue}{散射总场} $\xint{\mathcolor{gray}{-}}{255}{\tilde{E}}^{\;\! \mathcolor{gray}{\omega} \textcolor{PineGreen}{\hat{1}}}_{\;\! \textcolor{Plum}{(i-1)} \mathcolor{gray}{z}}$ 满足如下两个关系:
\begin{subequations}
\begin{align}
	\text{\textbf{\textcolor{Plum}{非线性}部分:}} \hspace{-0.16em} \xint{\mathcolor{gray}{-}}{21}{\bar{\bar{V}}}^{\;\! \mathcolor{gray}{\omega} \textcolor{PineGreen}{\hat{1}}}_{\;\! \mathcolor{gray}{z}} \xint{{}^{}_{\mathcolor{gray}{-}}}{10}{\tilde{g}}^{\;\!\mathcolor{gray}{\omega} \textcolor{PineGreen}{\hat{1}}}_{\;\! \;\! \textcolor{Plum}{(i-1)} \mathcolor{gray}{z}} \cdot \mathbb{e}^{\mathbb{i} \xint{\begin{smallmatrix} ~ \\ {}^{}_{\mathcolor{gray}{-}} \\ ~ \end{smallmatrix}}{15}{k}_{\symup{z}}^{\;\! \mathcolor{gray}{\omega} \textcolor{PineGreen}{\hat{1}}} \mathcolor{gray}{z}} &\xleftarrow[\text{to satisfy \bref{eq:linear_potential_scattering}}]{\text{\textbf{第 \textcolor{Plum}{$i$} 阶 \textcolor{Maroon}{Born 近似}}}} - \xint{\begin{smallmatrix} ~ \\ {}^{}_{\mathcolor{gray}{-}} \\ ~ \end{smallmatrix}}{16}{\tilde{\tilde{\varepsilon}}}^{\;\! \mathcolor{gray}{\omega}}_{\;\! \mathcolor{gray}{z}} \mathcolor{gray}{*} \xint{\mathcolor{gray}{-}}{295}{\bar{E}}^{\;\! \mathcolor{gray}{\omega} \textcolor{PineGreen}{\hat{2}}}_{\;\! \textcolor{Plum}{(i-1)} \mathcolor{gray}{z}} ~, \label{eq:L_scatter_eq-Born_approx} \\
	\text{\textbf{\textcolor{Plum}{线性}部分:}} \hspace{3.6em} \xint{\mathcolor{gray}{-}}{32}{\bar{\bar{L}}}^{\;\! \mathcolor{gray}{\omega} \textcolor{PineGreen}{\hat{1}}} \xint{\mathcolor{gray}{-}}{295}{\tilde{E}}^{\;\! \mathcolor{gray}{\omega} \textcolor{PineGreen}{\hat{1}}}_{\;\! \textcolor{Plum}{(i-1)} \mathcolor{gray}{z}} &\xleftarrow[\text{to satisfy \bref{eq:linear_potential_scattering}}]{} \bar{0} ~, \label{eq:L_scatter_eq-linear_background}
\end{align}
\end{subequations}
其中,前 $\textcolor{Plum}{i}$ 次\textcolor{NavyBlue}{散射场}之和 $\xint{\mathcolor{gray}{-}}{255}{\tilde{E}}^{\;\! \mathcolor{gray}{\omega} \textcolor{PineGreen}{\hat{1}}}_{\;\! \textcolor{Plum}{(i-1)} \mathcolor{gray}{z}}$ \textcolor{PineGreen}{复振幅}的\textcolor{Plum}{增长率}(即\textcolor{Plum}{非线性}部分)由第 \textcolor{Plum}{$i$} 阶 \textcolor{Maroon}{Born 近似} \bref{eq:L_scatter_eq-Born_approx} 控制,\textcolor{PineGreen}{本征值}和\textcolor{PineGreen}{偏振态}(即\textcolor{Plum}{线性}部分)由\textcolor{PineGreen}{特征方程} \bref{eq:L_scatter_eq-linear_background} 控制。后者再加上\textcolor{NavyBlue}{无微扰}时的解(即\textcolor{NavyBlue}{背景场})$\xint{\mathcolor{gray}{-}}{255}{\bar{E}}^{\;\! \mathcolor{gray}{\omega} \textcolor{PineGreen}{\hat{2}}}_{\;\! \textcolor{Plum}{(0)} \mathcolor{gray}{z}}$ 也满足相同的\textcolor{PineGreen}{特征方程} \bref{eq:linear_potential_scattering},可推出 $\xint{\mathcolor{gray}{-}}{255}{\bar{E}}^{\;\! \mathcolor{gray}{\omega} }_{\;\! \textcolor{Plum}{(n)} \mathcolor{gray}{z}}$ 即 \bref{eq:L_scatter_sol-Born_approx1} 也满足\textcolor{PineGreen}{特征方程} $\xint{\mathcolor{gray}{-}}{30}{\bar{\bar{L}}}^{\;\! \mathcolor{gray}{\omega} } \xint{\mathcolor{gray}{-}}{255}{\bar{E}}^{\;\! \mathcolor{gray}{\omega} }_{\;\! \mathcolor{gray}{z}} \equiv \bar{0}$。若再有 \bref{eq:linear_potential_scattering} 的解 $\xint{\mathcolor{gray}{-}}{255}{\bar{E}}^{\;\! \mathcolor{gray}{\omega} }_{\;\! \mathcolor{gray}{z}} \approx \xint{\mathcolor{gray}{-}}{255}{\bar{E}}^{\;\! \mathcolor{gray}{\omega} }_{\;\! \textcolor{Plum}{(n)} \mathcolor{gray}{z}}$,则将其代入 \bref{eq:linear_potential_scattering} 会得到 $\xint{\mathcolor{gray}{-}}{21}{\bar{\bar{V}}}^{\;\! \mathcolor{gray}{\omega} \textcolor{PineGreen}{\hat{1}}}_{\;\! \mathcolor{gray}{z}} \xint{{}^{}_{\mathcolor{gray}{-}}}{10}{\bar{g}}^{\;\!\mathcolor{gray}{\omega} \textcolor{PineGreen}{\hat{1}}}_{\;\! \mathcolor{gray}{z}} \approx - \xint{\begin{smallmatrix} ~ \\ {}^{}_{\mathcolor{gray}{-}} \\ ~ \end{smallmatrix}}{16}{\tilde{\tilde{\varepsilon}}}^{\;\! \mathcolor{gray}{\omega}}_{\mathcolor{gray}{z} \textcolor{Maroon}{(1)}} \mathcolor{gray}{*} \xint{\mathcolor{gray}{-}}{255}{\bar{E}}^{\;\! \mathcolor{gray}{\omega} \textcolor{PineGreen}{\hat{2}}}_{\;\! \mathcolor{gray}{z}}$,而这正是 \bref{eq:L_scatter_eq-Born_approx}(的形式)--- 意味着各阶 \textcolor{Maroon}{Born 近似} 后的结果,即\textcolor{NavyBlue}{背景场} $\xint{\mathcolor{gray}{-}}{255}{\bar{E}}^{\;\! \mathcolor{gray}{\omega} \textcolor{PineGreen}{\hat{2}}}_{\;\! \textcolor{Plum}{(0)} \mathcolor{gray}{z}}$ 与其各阶\textcolor{NavyBlue}{微扰},即各阶\textcolor{NavyBlue}{散射场}之和 $\xint{\mathcolor{gray}{-}}{255}{\tilde{E}}^{\;\! \mathcolor{gray}{\omega} \textcolor{PineGreen}{\hat{1}}}_{\;\! \textcolor{Plum}{(i)} \mathcolor{gray}{z}}$ 之和,近似收敛至 \textcolor{Plum}{真值} $\xint{\mathcolor{gray}{-}}{255}{\bar{E}}^{\;\! \mathcolor{gray}{\omega} }_{\;\! \mathcolor{gray}{z}}$。

为了更物理地描述该\textcolor{Plum}{自耦合}的\textcolor{Maroon}{势散射}微-积分方程的\textcolor{Plum}{解}的得来,现尝试更形象地解释应用至该\textcolor{Maroon}{势散射问题}的\textbf{一阶或多阶 \textcolor{Maroon}{Born 近似}}\cite{bornPrinciplesOptics60th2019}条件:即认为,波动方程 \bref{eq:linear_potential_scattering} 右侧\textcolor{Plum}{非线性}\textcolor{NavyBlue}{波源}项中的 $\xint{\mathcolor{gray}{-}}{255}{\bar{E}}^{\;\! \mathcolor{gray}{\omega} \textcolor{PineGreen}{\hat{2}}}_{\;\! \mathcolor{gray}{z}}$ 不同于方程左侧的 $\xint{\mathcolor{gray}{-}}{255}{\bar{E}}^{\;\! \mathcolor{gray}{\omega} \textcolor{PineGreen}{\hat{1}}}_{\;\! \mathcolor{gray}{z}}$,二者有 {\one} \textbf{因果/先后顺序关系}:先有 \bref{eq:L_scatter_eq-Born_approx} 右侧的第 $\textcolor{Plum}{i}$ 次\Footnote{求根问底而言,“阶”比“次”更适当:因为上一阶并不是\textcolor{NavyBlue}{入射}的\textcolor{NavyBlue}{泵浦},下一阶也并不是\textcolor{NavyBlue}{实际产生}的\textcolor{NavyBlue}{出射场}。}\textcolor{NavyBlue}{入射}/\textcolor{NavyBlue}{驱动}/\textcolor{NavyBlue}{泵浦}/\textcolor{NavyBlue}{源场} $\xint{\mathcolor{gray}{-}}{255}{\bar{E}}^{\;\! \mathcolor{gray}{\omega} \textcolor{PineGreen}{\hat{2}}}_{\;\! \textcolor{Plum}{(i-1)} \mathcolor{gray}{z}}$,后有 \bref{eq:L_scatter_eq-Born_approx} 左侧的第 $\textcolor{Plum}{i}$ 次\textcolor{NavyBlue}{出射}/\textcolor{NavyBlue}{产生}/\textcolor{Plum}{新增}/\textcolor{NavyBlue}{散射场} $ = \xint{\mathcolor{gray}{-}}{255}{\tilde{E}}^{\;\! \mathcolor{gray}{\omega} \textcolor{PineGreen}{\hat{1}}}_{\;\! \textcolor{Plum}{(i-1)} \mathcolor{gray}{z}}$;左侧慢半拍,对应是结果,右侧是原因;同时,二者也有 {\two} \textbf{强弱关系},即一般右侧的\textcolor{NavyBlue}{驱动源}比左侧的\textcolor{Plum}{新增场}更强,而作为结果的左侧的\textcolor{Plum}{新增场},每次都与第一次的\textcolor{NavyBlue}{无散射背景场}一起,求和后 $\xint{\mathcolor{gray}{-}}{255}{\bar{E}}^{\;\! \mathcolor{gray}{\omega} \textcolor{PineGreen}{\hat{2}}}_{\;\! \textcolor{Plum}{(0)} \mathcolor{gray}{z}} + \xint{\mathcolor{gray}{-}}{255}{\tilde{E}}^{\;\! \mathcolor{gray}{\omega} \textcolor{PineGreen}{\hat{1}}}_{\;\! \textcolor{Plum}{(i-1)} \mathcolor{gray}{z}}$,作为下一阶(第 $\textcolor{Plum}{i+1}$ 阶)波动方程右侧的\textcolor{NavyBlue}{新驱动源} $\xint{\mathcolor{gray}{-}}{255}{\bar{E}}^{\;\! \mathcolor{gray}{\omega} \textcolor{PineGreen}{\hat{2}}}_{\;\! \textcolor{Plum}{(i)} \mathcolor{gray}{z}}$,继续产生\textbf{下一阶(第 $\textcolor{Plum}{i+1}$ 阶)\textcolor{Maroon}{Born 近似}}更高精度的左侧\textcolor{NavyBlue}{散射场} $\xint{\mathcolor{gray}{-}}{255}{\tilde{E}}^{\;\! \mathcolor{gray}{\omega} \textcolor{PineGreen}{\hat{1}}}_{\;\! \textcolor{Plum}{(i)} \mathcolor{gray}{z}}$。上一段已经证明,满足\textbf{各阶 \textcolor{Maroon}{Born 近似}} \bref{eq:L_scatter_eq-Born_approx} 的初始\textcolor{NavyBlue}{入射电场} $\xint{\mathcolor{gray}{-}}{255}{\bar{E}}^{\;\! \mathcolor{gray}{\omega} \textcolor{PineGreen}{\hat{2}}}_{\;\! \textcolor{Plum}{(0)} \mathcolor{gray}{z}}$ 与各阶\textcolor{NavyBlue}{散射场} $\Delta^{\textcolor{Plum}{i}} \xint{\mathcolor{gray}{-}}{295}{\bar{E}}^{\;\! \mathcolor{gray}{\omega} }_{\;\! \textcolor{Plum}{(0)} \mathcolor{gray}{z}}$ $\left( \textcolor{Plum}{1} \leq \textcolor{Plum}{i} \leq \textcolor{Plum}{n} \right)$ 之和,所线性叠加出的\textcolor{NavyBlue}{总电场} \bref{eq:L_scatter_sol-Born_approx} 近似满足 \bref{eq:linear_potential_scattering}。实际上,由于高一阶的\textcolor{NavyBlue}{散射场},总比低一阶的\textcolor{NavyBlue}{散射场}弱很多,一阶 \textcolor{Maroon}{Born 近似} 的\textcolor{Plum}{精度}已经足够(以至于更高阶的修正大多没有必要:不论\textcolor{NavyBlue}{调制}\textcolor{Plum}{深}还是\textcolor{Plum}{浅}\Footnote{\textcolor{NavyBlue}{调制}\textcolor{Plum}{深},则\textcolor{Maroon}{散射势}太强,以至于 \textcolor{Maroon}{Born 近似} 失效;\textcolor{NavyBlue}{调制}\textcolor{Plum}{浅},则下一阶\textcolor{Plum}{散射幅值}相比上一阶小很多,而可\textcolor{Plum}{忽略不计}。关于\textcolor{Plum}{无穷阶} \textcolor{Maroon}{Born 近似} 的收敛性问题:在\textcolor{Maroon}{没有交叉偏振模耦合}(比如,\textcolor{PineGreen}{o 型本征偏振}的\textcolor{NavyBlue}{源}不会转换为 \textcolor{PineGreen}{e 型本征偏振}的\textcolor{NavyBlue}{散射场})的情况下,\bref{eq:L_scatter_eq-Born_approx} 所产生的 \bref{eq:L_scatter_sol-Born_approx3} 中的\textcolor{Plum}{通项}将构成一个\textcolor{Plum}{等比级数},它是收敛的:$1 + \left( - \mathbb{i} 0.01 \right)^2 + \left( - \mathbb{i} 0.01 \right)^4 ...$。})。

使用\textbf{\textcolor{NavyBlue}{缓变振幅}近似}条件(见 \bref{eq:nonlinear(2)-wave_wkrho-simplify6-SVA}),再加上 \bref{eq:k_rho<<k_z} 处的 2 个条件,\bref{eq:linear_potential_scattering,eq:L_scatter_eq-Born_approx} 中的 $\xint{\mathcolor{gray}{-}}{21}{\bar{\bar{V}}}^{\;\! \mathcolor{gray}{\omega} \textcolor{PineGreen}{\hat{1}}}_{\;\! \mathcolor{gray}{z}}$ 先从 \bref{eq:nonlinear(2)-wave_wkrho-spectrum} 的 $\textcolor{Plum}{3 \times 3}$ 简化为 $\textcolor{Plum}{2 \times 2}$(见 \bref{eq:plane_wave_basis-V1-nokxky-zeta}),再升级回 $\textcolor{Plum}{3 \times 3}$(见 \bref{eq:simplify7-scalar-g-conjugate} $\to$ \bref{eq:simplify8-scalar-g-conjugate}),使得 \bref{eq:L_scatter_eq-Born_approx} 公式转变为了 \bref{eq:simplify8-scalar-g-modulus} 的类似物:
\begin{align} \label{eq:Born_approx-scalar-g-modulus}
	\mathcolor{gray}{\nabla_z} \xint{\begin{smallmatrix} ~ \\ {}^{}_{\mathcolor{gray}{-}} \\ ~ \end{smallmatrix}}{09}{\tilde{\mathtt{g}}}^{\;\!\mathcolor{gray}{\omega} \textcolor{PineGreen}{\hat{1}}}_{\;\! \textcolor{Plum}{(i-1)} \mathcolor{gray}{z}} = - \mathbb{i} k_{\textcolor{Maroon}{\mathsf{o}} \mathcolor{gray}{\omega}}^{\;\! 2} \frac{\xint{{}^{}_{\mathcolor{gray}{-}}}{10}{\hat{g}}^{\;\! \textcolor{PineGreen}{\hat{1}} \textcolor{Plum}{\dag}}_{\;\! \mathcolor{gray}{\omega}} \cdot \left( \xint{\begin{smallmatrix} ~ \\ {}^{}_{\mathcolor{gray}{-}} \\ ~ \end{smallmatrix}}{16}{\tilde{\tilde{\varepsilon}}}^{\;\! \mathcolor{gray}{\omega}}_{\;\! \mathcolor{gray}{z}} \mathcolor{gray}{*} \xint{\mathcolor{gray}{-}}{295}{\bar{E}}^{\;\! \mathcolor{gray}{\omega} \textcolor{PineGreen}{\hat{2}}}_{\;\! \textcolor{Plum}{(i-1)} \mathcolor{gray}{z}} \right) }{ 2 \lvert \xint{{}^{}_{\mathcolor{gray}{-}}}{10}{\hat{g}}^{\;\! \textcolor{PineGreen}{\hat{1}}}_{\;\! \mathcolor{gray}{\omega}} \rvert^2 \xint{\begin{smallmatrix} ~ \\ {}^{}_{\mathcolor{gray}{-}} \\ ~ \end{smallmatrix}}{15}{k}_{\;\! \symup{z}}^{\;\! \mathcolor{gray}{\omega} \textcolor{PineGreen}{\hat{1}}} \mathbb{e}^{\mathbb{i} \xint{\begin{smallmatrix} ~ \\ {}^{}_{\mathcolor{gray}{-}} \\ ~ \end{smallmatrix}}{15}{k}_{\symup{z}}^{\;\! \mathcolor{gray}{\omega} \textcolor{PineGreen}{\hat{1}}} \mathcolor{gray}{z}}} ~,
\end{align}
其中,{\one} $\mathcolor{gray}{\nabla_z}$ 来自于 $\xint{\mathcolor{gray}{-}}{21}{\bar{\bar{V}}}^{\;\! \mathcolor{gray}{\omega} \textcolor{PineGreen}{\hat{1}}}_{\;\! \mathcolor{gray}{z}}$,{\two} \textcolor{NavyBlue}{散射总场}(的\textcolor{PineGreen}{本征})\textcolor{PineGreen}{复振幅} $\xint{\begin{smallmatrix} ~ \\ {}^{}_{\mathcolor{gray}{-}} \\ ~ \end{smallmatrix}}{09}{\tilde{\mathtt{g}}}^{\;\!\mathcolor{gray}{\omega} \textcolor{PineGreen}{\hat{1}}}_{\;\! \textcolor{Plum}{(i-1)} \mathcolor{gray}{z}}$ 为\textcolor{PineGreen}{本征复振幅}\textcolor{NavyBlue}{背景} $\xint{\begin{smallmatrix} ~ \\ {}^{}_{\mathcolor{gray}{-}} \\ ~ \end{smallmatrix}}{09}{\mathtt{g}}^{\;\!\mathcolor{gray}{\omega} \textcolor{PineGreen}{\hat{1}}}_{\;\! \textcolor{Plum}{(0)} \mathcolor{gray}{z}}$ 的\textcolor{Plum}{增量}(且为标量);{\three} $\xint{\begin{smallmatrix} ~ \\ {}^{}_{\mathcolor{gray}{-}} \\ ~ \end{smallmatrix}}{16}{\tilde{\tilde{\varepsilon}}}^{\;\! \mathcolor{gray}{\omega}}_{\;\! \mathcolor{gray}{z}}$ 为 $\xint{\begin{smallmatrix} ~ \\ {}^{}_{\mathcolor{gray}{-}} \\ ~ \end{smallmatrix}}{16}{\bar{\bar{\varepsilon}}}^{\;\! \mathcolor{gray}{\omega}}_{\;\! \mathcolor{gray}{z}}$ 的\textcolor{NavyBlue}{微扰增量}(且为矩阵)。

进一步将其分量化(类似 \bref{eq:up-scalar-g-EE-312-spectrum})、分离出\textcolor{Plum}{调制函数}(类似 \bref{eq:components-chi2-modulate})、定义\textcolor{NavyBlue}{调制场}的\textcolor{NavyBlue}{倒格波系数}(类似 \bref{eq:components-C})后,得
\begin{subequations}
\begin{align}
	\hspace{-1.2em} \mathcolor{gray}{\nabla_z} \xint{\begin{smallmatrix} ~ \\ {}^{}_{\mathcolor{gray}{-}} \\ ~ \end{smallmatrix}}{09}{\tilde{\mathtt{g}}}^{\;\!\mathcolor{gray}{\omega} \textcolor{PineGreen}{\hat{1}}}_{\;\! \textcolor{Plum}{(i-1)} \mathcolor{gray}{z}} = & - \mathbb{i} k_{\textcolor{Maroon}{\mathsf{o}} \mathcolor{gray}{\omega}}^{\;\! 2} \frac{ \xint{{}^{}_{\mathcolor{gray}{-}}}{10}{\hat{g}}^{\;\! \hat{1} \textcolor{PineGreen}{\hat{1}} \textcolor{Plum}{*}}_{\;\! \mathcolor{gray}{\omega}} {\tilde{\varepsilon}}^{\;\! \textcolor{PineGreen}{\hat{1}} \mathcolor{gray}{\omega} \hat{2} }_{\;\! \hat{1} \textcolor{Maroon}{(1)} \textcolor{PineGreen}{\hat{2}}} ~ \mathcolor{gray}{\mathcal F_{z}^{-1}} \left[ \xint{\mathcolor{gray}{-}}{18}{M}^{\;\! \mathcolor{gray}{\omega} \hat{2} }_{\;\! \hat{1} \mathcolor{gray}{k_{\symup{z}}} \textcolor{Maroon}{(1)} } \mathcolor{gray}{*} \xint{\mathcolor{gray}{-}}{25}{E}^{\;\! \textcolor{PineGreen}{\hat{2}} \mathcolor{gray}{\omega} }_{\;\! \hat{2} \textcolor{Plum}{(i-1)} \mathcolor{gray}{z} } \right] }{ 2 \lvert \xint{{}^{}_{\mathcolor{gray}{-}}}{10}{\hat{g}}^{\;\! \textcolor{PineGreen}{\hat{1}}}_{\;\! \mathcolor{gray}{\omega}} \rvert^2 \xint{\begin{smallmatrix} ~ \\ {}^{}_{\mathcolor{gray}{-}} \\ ~ \end{smallmatrix}}{15}{k}_{\;\! \symup{z}}^{\;\! \mathcolor{gray}{\omega} \textcolor{PineGreen}{\hat{1}}} \mathbb{e}^{\mathbb{i} \xint{\begin{smallmatrix} ~ \\ {}^{}_{\mathcolor{gray}{-}} \\ ~ \end{smallmatrix}}{15}{k}_{\symup{z}}^{\;\! \mathcolor{gray}{\omega} \textcolor{PineGreen}{\hat{1}}} \mathcolor{gray}{z}}} \label{eq:Born_approx-scalar-g-E-12-spectrum} \\
	\hspace{-1.2em} \xrightarrow[\text{“\textcolor{NavyBlue}{含衍射}\textcolor{PineGreen}{本征复振幅}”\bref{eq:components-eigenwave'}}]{\text{“\textbf{标量\textcolor{Plum}{非线性}\textcolor{NavyBlue}{波源}}”\bref{eq:scalar_nonlinear_drive-spectrum}}} & - \mathbb{i} k_{\textcolor{Maroon}{\mathsf{o}} \mathcolor{gray}{\omega}}^{\;\! 2} \frac{ \xint{{}^{}_{\mathcolor{gray}{-}}}{10}{\hat{g}}^{\;\! \hat{1} \textcolor{PineGreen}{\hat{1}} \textcolor{Plum}{*}}_{\;\! \mathcolor{gray}{\omega}} {\tilde{\varepsilon}}^{\;\! \textcolor{PineGreen}{\hat{1}} \mathcolor{gray}{\omega} \hat{2} }_{\;\! \hat{1} \textcolor{Maroon}{(1)} \textcolor{PineGreen}{\hat{2}}} {\hat{g}}^{\;\! \textcolor{PineGreen}{\hat{2}} \mathcolor{gray}{\omega} }_{\;\! \hat{2} \textcolor{Plum}{(i-1)} } \mathcolor{gray}{\mathcal F_{z}^{-1}} \left[ \xint{\mathcolor{gray}{-}}{18}{M}^{\;\! \mathcolor{gray}{\omega} \hat{2} }_{\;\! \hat{1} \mathcolor{gray}{k_{\symup{z}}} \textcolor{Maroon}{(1)} } \mathcolor{gray}{*} \xint{\mathcolor{gray}{-}}{15}{\mathtt{G}}^{\;\! \textcolor{PineGreen}{\hat{2}} \mathcolor{gray}{\omega} }_{\;\! \textcolor{Plum}{(i-1)} \mathcolor{gray}{z} } \right] }{ 2 \lvert \xint{{}^{}_{\mathcolor{gray}{-}}}{10}{\hat{g}}^{\;\! \textcolor{PineGreen}{\hat{1}}}_{\;\! \mathcolor{gray}{\omega}} \rvert^2 \xint{\begin{smallmatrix} ~ \\ {}^{}_{\mathcolor{gray}{-}} \\ ~ \end{smallmatrix}}{15}{k}_{\;\! \symup{z}}^{\;\! \mathcolor{gray}{\omega} \textcolor{PineGreen}{\hat{1}}} \mathbb{e}^{\mathbb{i} \xint{\begin{smallmatrix} ~ \\ {}^{}_{\mathcolor{gray}{-}} \\ ~ \end{smallmatrix}}{15}{k}_{\symup{z}}^{\;\! \mathcolor{gray}{\omega} \textcolor{PineGreen}{\hat{1}}} \mathcolor{gray}{z}}} \label{eq:Born_approx-scalar-g-G-12-spectrum} \\
	\hspace{-1.2em} \xrightarrow[ \xint{\begin{smallmatrix} ~ \\ {}^{}_{\mathcolor{gray}{-}} \\ ~ \end{smallmatrix}}{16}{\tilde{\varepsilon}}^{ \hat{1} \textcolor{PineGreen}{\hat{1}} \textcolor{Maroon}{(1)} \mathcolor{gray}{\omega} }_{ \textcolor{NavyBlue}{\text{eff}} \hat{2} \textcolor{PineGreen}{\hat{2}} \textcolor{Plum}{(i-1)} } := \xint{{}^{}_{\mathcolor{gray}{-}}}{10}{\hat{g}}^{\;\! \hat{1} \textcolor{PineGreen}{\hat{1}} \textcolor{Plum}{*}}_{\;\! \mathcolor{gray}{\omega}} {\tilde{\varepsilon}}^{\;\! \hat{1} \textcolor{PineGreen}{\hat{1}} \textcolor{Maroon}{(1)} }_{\;\! \mathcolor{gray}{\omega} \hat{2} \textcolor{PineGreen}{\hat{2}}} {\hat{g}}^{\;\! \mathcolor{gray}{\omega} }_{\;\! \hat{2} \textcolor{PineGreen}{\hat{2}} \textcolor{Plum}{(i-1)} }]{\text{定义 \textbf{\textcolor{NavyBlue}{有效线性系数微扰}}}} & - \mathbb{i} k_{\textcolor{Maroon}{\mathsf{o}} \mathcolor{gray}{\omega}}^{\;\! 2} \frac{ \xint{\begin{smallmatrix} ~ \\ {}^{}_{\mathcolor{gray}{-}} \\ ~ \end{smallmatrix}}{16}{\tilde{\varepsilon}}^{ \hat{1} \textcolor{PineGreen}{\hat{1}} \textcolor{Maroon}{(1)} \mathcolor{gray}{\omega} }_{ \textcolor{NavyBlue}{\text{eff}} \hat{2} \textcolor{PineGreen}{\hat{2}} \textcolor{Plum}{(i-1)} } \mathcolor{gray}{\mathcal F_{z}^{-1}} \left[ \xint{\mathcolor{gray}{-}}{18}{M}^{\;\! \mathcolor{gray}{\omega} \hat{2} }_{\;\! \hat{1} \mathcolor{gray}{k_{\symup{z}}} \textcolor{Maroon}{(1)} } \mathcolor{gray}{*} \xint{\mathcolor{gray}{-}}{15}{\mathtt{G}}^{\;\! \textcolor{PineGreen}{\hat{2}} \mathcolor{gray}{\omega} }_{\;\! \textcolor{Plum}{(i-1)} \mathcolor{gray}{z} } \right] }{ 2 \lvert \xint{{}^{}_{\mathcolor{gray}{-}}}{10}{\hat{g}}^{\;\! \textcolor{PineGreen}{\hat{1}}}_{\;\! \mathcolor{gray}{\omega}} \rvert^2 \xint{\begin{smallmatrix} ~ \\ {}^{}_{\mathcolor{gray}{-}} \\ ~ \end{smallmatrix}}{15}{k}_{\;\! \symup{z}}^{\;\! \mathcolor{gray}{\omega} \textcolor{PineGreen}{\hat{1}}} \mathbb{e}^{\mathbb{i} \xint{\begin{smallmatrix} ~ \\ {}^{}_{\mathcolor{gray}{-}} \\ ~ \end{smallmatrix}}{15}{k}_{\symup{z}}^{\;\! \mathcolor{gray}{\omega} \textcolor{PineGreen}{\hat{1}}} \mathcolor{gray}{z}}} \label{eq:Born_approx-scalar-g-G-12-varepsiloneff-spectrum} \\
	\hspace{-1.2em} \xrightarrow[ \xint{\begin{smallmatrix} ~ \\ {}^{}_{\mathcolor{gray}{-}} \\ ~ \end{smallmatrix}}{16}{\tilde{\varepsilon}}^{ \mathcolor{gray}{\omega} \textcolor{PineGreen}{\hat{1}} \textcolor{Maroon}{(1)} }_{ \textcolor{NavyBlue}{\text{eff}} \textcolor{PineGreen}{\hat{2}} \textcolor{Plum}{(i-1)} } := \xint{{}^{}_{\mathcolor{gray}{-}}}{10}{\hat{g}}^{\;\! \hat{1} \textcolor{PineGreen}{\hat{1}} \textcolor{Plum}{*}}_{\;\! \mathcolor{gray}{\omega}} {\tilde{\varepsilon}}^{\;\! \textcolor{PineGreen}{\hat{1}} \mathcolor{gray}{\omega} \hat{2} }_{\;\! \hat{1} \textcolor{Maroon}{(1)} \textcolor{PineGreen}{\hat{2}}} {\hat{g}}^{\;\! \mathcolor{gray}{\omega} }_{\;\! \hat{2} \textcolor{PineGreen}{\hat{2}} \textcolor{Plum}{(i-1)} } ]{\text{“\textbf{标量场 $\varepsilon^{\;\! \mathcolor{gray}{\omega} }_{\;\! \mathcolor{gray}{z} \textcolor{Maroon}{(1)}}$ \textcolor{NavyBlue}{调制}}”$\xint{\mathcolor{gray}{-}}{18}{\bar{\bar{M}}}^{\;\! \mathcolor{gray}{\omega} }_{\;\! \mathcolor{gray}{z} \textcolor{Maroon}{(1)} } = \xint{\mathcolor{gray}{-}}{18}{M}^{\;\! \mathcolor{gray}{\omega} }_{\;\! \mathcolor{gray}{z} \textcolor{Maroon}{(1)} }$}} & - \mathbb{i} k_{\textcolor{Maroon}{\mathsf{o}} \mathcolor{gray}{\omega}}^{\;\! 2} \frac{ \xint{\begin{smallmatrix} ~ \\ {}^{}_{\mathcolor{gray}{-}} \\ ~ \end{smallmatrix}}{16}{\tilde{\varepsilon}}^{ \mathcolor{gray}{\omega} \textcolor{PineGreen}{\hat{1}} \textcolor{Maroon}{(1)} }_{ \textcolor{NavyBlue}{\text{eff}} \textcolor{PineGreen}{\hat{2}} \textcolor{Plum}{(i-1)} } \mathcolor{gray}{\mathcal F_{z}^{-1}} \left[ \xint{\mathcolor{gray}{-}}{18}{M}^{\;\! \mathcolor{gray}{\omega} }_{\;\! \mathcolor{gray}{k_{\symup{z}}} \textcolor{Maroon}{(1)} } \mathcolor{gray}{*} \xint{\mathcolor{gray}{-}}{15}{\mathtt{G}}^{\;\! \textcolor{PineGreen}{\hat{2}} \mathcolor{gray}{\omega} }_{\;\! \textcolor{Plum}{(i-1)} \mathcolor{gray}{z} } \right] }{ 2 \lvert \xint{{}^{}_{\mathcolor{gray}{-}}}{10}{\hat{g}}^{\;\! \textcolor{PineGreen}{\hat{1}}}_{\;\! \mathcolor{gray}{\omega}} \rvert^2 \xint{\begin{smallmatrix} ~ \\ {}^{}_{\mathcolor{gray}{-}} \\ ~ \end{smallmatrix}}{15}{k}_{\;\! \symup{z}}^{\;\! \mathcolor{gray}{\omega} \textcolor{PineGreen}{\hat{1}}} \mathbb{e}^{\mathbb{i} \xint{\begin{smallmatrix} ~ \\ {}^{}_{\mathcolor{gray}{-}} \\ ~ \end{smallmatrix}}{15}{k}_{\symup{z}}^{\;\! \mathcolor{gray}{\omega} \textcolor{PineGreen}{\hat{1}}} \mathcolor{gray}{z}}} \label{eq:Born_approx-scalar-g-G-12-varepsiloneff-scalar-spectrum} ~.
\end{align}
\end{subequations}

退回 \bref{eq:Born_approx-scalar-g-E-12-spectrum},换一个分支:采取 \bref{ssec:undepleted-pump-approximation} 中的另一个 \textbf{\textcolor{NavyBlue}{泵浦未耗尽}近似条件},可以得到前/第 $\textcolor{Plum}{i}$ 阶 \textcolor{Maroon}{Born 近似} \bref{eq:L_scatter_eq-linear_background} 下的 \textcolor{NavyBlue}{散射(总)场} $\xint{\mathcolor{gray}{-}}{255}{\tilde{E}}^{\;\! \mathcolor{gray}{\omega} \textcolor{PineGreen}{\hat{1}}}_{\;\! \textcolor{Plum}{(i-1)} \mathcolor{gray}{z}}$ 的\textcolor{PineGreen}{本征复振幅}的\textcolor{Plum}{变化率} $\mathcolor{gray}{\nabla_z} \xint{\begin{smallmatrix} ~ \\ {}^{}_{\mathcolor{gray}{-}} \\ ~ \end{smallmatrix}}{09}{\tilde{\mathtt{g}}}^{\;\!\mathcolor{gray}{\omega} \textcolor{PineGreen}{\hat{1}}}_{\;\! \textcolor{Plum}{(i-1)} \mathcolor{gray}{z}}$,及其\textcolor{Plum}{原函数}即\textcolor{Plum}{非线性卷积}解 $\xint{\begin{smallmatrix} ~ \\ {}^{}_{\mathcolor{gray}{-}} \\ ~ \end{smallmatrix}}{09}{\tilde{\mathtt{g}}}^{\;\!\mathcolor{gray}{\omega} \textcolor{PineGreen}{\hat{1}}}_{\;\! \textcolor{Plum}{(i-1)} \mathcolor{gray}{z}}$:
\begin{subequations} \label{eq:Born_approx-scalar-g-E-12convolution}
\begin{align}
	\hspace{-3.4em} \mathcolor{gray}{\nabla_z} \xint{\begin{smallmatrix} ~ \\ {}^{}_{\mathcolor{gray}{-}} \\ ~ \end{smallmatrix}}{09}{\tilde{\mathtt{g}}}^{\;\!\mathcolor{gray}{\omega} \textcolor{PineGreen}{\hat{1}}}_{\;\! \textcolor{Plum}{(i-1)} \mathcolor{gray}{z}} \xrightarrow[]{\text{\bref{eq:Born_approx-scalar-g-E-12-spectrum}}} - \mathbb{i} k_{\textcolor{Maroon}{\mathsf{o}} \mathcolor{gray}{\omega}}^{\;\! 2} \frac{ \xint{{}^{}_{\mathcolor{gray}{-}}}{10}{\hat{g}}^{\;\! \hat{1} \textcolor{PineGreen}{\hat{1}} \textcolor{Plum}{*}}_{\;\! \mathcolor{gray}{\omega}} {\tilde{\varepsilon}}^{\;\! \textcolor{PineGreen}{\hat{1}} \mathcolor{gray}{\omega} \hat{2} }_{\;\! \hat{1} \textcolor{Maroon}{(1)} \textcolor{PineGreen}{\hat{2}}} }{ 2 \lvert \xint{{}^{}_{\mathcolor{gray}{-}}}{10}{\hat{g}}^{\;\! \textcolor{PineGreen}{\hat{1}}}_{\;\! \mathcolor{gray}{\omega}} \rvert^2 \xint{\begin{smallmatrix} ~ \\ {}^{}_{\mathcolor{gray}{-}} \\ ~ \end{smallmatrix}}{15}{k}_{\;\! \symup{z}}^{\;\! \mathcolor{gray}{\omega} \textcolor{PineGreen}{\hat{1}}} } ~ \mathcolor{gray}{\mathcal F_{z}^{-1}} & \left[ \xint{\mathcolor{gray}{-}}{18}{M}^{\;\! \mathcolor{gray}{\omega} \hat{2} }_{\;\! \hat{1} \mathcolor{gray}{k_{\symup{z}}} \textcolor{Maroon}{(1)} } \mathcolor{gray}{*} \xint{\mathcolor{gray}{-}}{25}{E}^{\;\! \textcolor{PineGreen}{\hat{2}} \mathcolor{gray}{\omega} }_{\;\! \hat{2} \textcolor{Plum}{(i-1)} \mathcolor{gray}{z} } \right] \big/ \mathbb{e}^{\mathbb{i} \xint{\begin{smallmatrix} ~ \\ {}^{}_{\mathcolor{gray}{-}} \\ ~ \end{smallmatrix}}{15}{k}_{\symup{z}}^{\;\! \mathcolor{gray}{\omega} \textcolor{PineGreen}{\hat{1}}} \mathcolor{gray}{z}}  \label{eq:Born_approx-scalar-g-E-12-convolution1} \\
	\hspace{-3.4em} \xrightarrow[\text{“\textcolor{PineGreen}{本征复振幅} $\times$ \textcolor{PineGreen}{向量}”\bref{eq:components-eigenwave}}]{\Upsilon^{\;\! \hat{1} \textcolor{PineGreen}{\hat{1}} \hat{2} \mathcolor{gray}{\omega} }_{\;\! \textcolor{Maroon}{(1)} \textcolor{PineGreen}{\hat{2}} } ~ := ~ - \mathbb{i} k_{\textcolor{Maroon}{\mathsf{o}} \mathcolor{gray}{\omega}}^{\;\! 2} ~ ...} \Upsilon^{\;\! \hat{1} \textcolor{PineGreen}{\hat{1}} \hat{2} \mathcolor{gray}{\omega} }_{\;\! \textcolor{Maroon}{(1)} \textcolor{PineGreen}{\hat{2}} }~ \mathcolor{gray}{\mathcal F_{z}^{-1}} \left[ \xint{\mathcolor{gray}{-}}{18}{M}^{\;\! \mathcolor{gray}{\omega} \hat{2} }_{\;\! \hat{1} \mathcolor{gray}{k_{\symup{z}}} \textcolor{Maroon}{(1)} } \mathcolor{gray}{*} \right.&\left. \! \big( \xint{{}^{}_{\mathcolor{gray}{-}}}{10}{g}^{\;\! \textcolor{PineGreen}{\hat{2}} \mathcolor{gray}{\omega} }_{\;\! \hat{2} \textcolor{Plum}{(i-1)} \mathcolor{gray}{z} } \mathbb{e}^{\mathbb{i} \xint{\begin{smallmatrix} ~ \\ {}^{}_{\mathcolor{gray}{-}} \\ ~ \end{smallmatrix}}{15}{k}_{\symup{z}}^{\;\! \textcolor{PineGreen}{\hat{2}} \mathcolor{gray}{\omega} } \mathcolor{gray}{z} } \big) \right] \big/ \mathbb{e}^{\mathbb{i} \xint{\begin{smallmatrix} ~ \\ {}^{}_{\mathcolor{gray}{-}} \\ ~ \end{smallmatrix}}{15}{k}_{\symup{z}}^{\;\! \mathcolor{gray}{\omega} \textcolor{PineGreen}{\hat{1}}} \mathcolor{gray}{z}} \label{eq:Born_approx-scalar-g-E-12-convolution2} \\
	\hspace{-3.4em} \xrightarrow[]{\text{“\textbf{\textcolor{NavyBlue}{泵浦未耗尽}近似}”\bref{eq:Nondepleted-Pump-Approximation-spectrum}}} \Upsilon^{\;\! \hat{1} \textcolor{PineGreen}{\hat{1}} \hat{2} \mathcolor{gray}{\omega} }_{\;\! \textcolor{Maroon}{(1)} \textcolor{PineGreen}{\hat{2}} }~ \mathcolor{gray}{\mathcal F_{z}^{-1}} \left[ \xint{\mathcolor{gray}{-}}{18}{M}^{\;\! \mathcolor{gray}{\omega} \hat{2} }_{\;\! \hat{1} \mathcolor{gray}{k_{\symup{z}}} \textcolor{Maroon}{(1)} } \mathcolor{gray}{*} \right.&\left. \! \big( \xint{{}^{}_{\mathcolor{gray}{-}}}{10}{g}^{\;\! \textcolor{PineGreen}{\hat{2}} \mathcolor{gray}{\omega} }_{\;\! \hat{2} \textcolor{Plum}{(i-1)} \mathcolor{gray}{0} } \mathbb{e}^{\mathbb{i} \xint{\begin{smallmatrix} ~ \\ {}^{}_{\mathcolor{gray}{-}} \\ ~ \end{smallmatrix}}{15}{k}_{\symup{z}}^{\;\! \textcolor{PineGreen}{\hat{2}} \mathcolor{gray}{\omega} } \mathcolor{gray}{z} } \big) \right] \big/ \mathbb{e}^{\mathbb{i} \xint{\begin{smallmatrix} ~ \\ {}^{}_{\mathcolor{gray}{-}} \\ ~ \end{smallmatrix}}{15}{k}_{\symup{z}}^{\;\! \mathcolor{gray}{\omega} \textcolor{PineGreen}{\hat{1}}} \mathcolor{gray}{z}} \label{eq:Born_approx-scalar-g-E-12-convolution3} \\
	= \Upsilon^{\;\! \hat{1} \textcolor{PineGreen}{\hat{1}} \hat{2} \mathcolor{gray}{\omega} }_{\;\! \textcolor{Maroon}{(1)} \textcolor{PineGreen}{\hat{2}} } \mathcolor{gray}{\iiint} \xint{\mathcolor{gray}{-}}{18}{M}^{\;\! \mathcolor{gray}{\omega} \hat{2} }_{\;\! \hat{1} \textcolor{Maroon}{(1)} } \left( \mathcolor{gray}{\bar{q}} \right) \mathcolor{gray}{\iint} & \xint{{}^{}_{\mathcolor{gray}{-}}}{10}{g}^{\;\! \textcolor{PineGreen}{\hat{2}} \mathcolor{gray}{\omega} }_{\;\! \hat{2} \textcolor{Plum}{(i-1)} \mathcolor{gray}{0} } \left( \mathcolor{gray}{\bar{k}_{2\symup{\rho}}} \right) \mathbb{e}^{\mathbb{i} \Delta \xint{\begin{smallmatrix} ~ \\ {}^{}_{\mathcolor{gray}{-}} \\ ~ \end{smallmatrix}}{15}{k}_{\symup{z}}^{\;\! \textcolor{PineGreen}{\hat{2} \hat{1}} } \mathcolor{gray}{z} } ~\mathbb{d} \mathcolor{gray}{\bar{q}} \label{eq:Born_approx-scalar-g-E-12-convolution4} \\
	\hspace{-3.4em} \xint{\begin{smallmatrix} ~ \\ {}^{}_{\mathcolor{gray}{-}} \\ ~ \end{smallmatrix}}{09}{\tilde{\mathtt{g}}}^{\;\!\mathcolor{gray}{\omega} \textcolor{PineGreen}{\hat{1}}}_{\;\! \textcolor{Plum}{(i-1)} \mathcolor{gray}{z}} \xrightarrow[\text{\bref{eq:up-scalar-g-EE-312-discrete-since3}}]{\text{$\mathcolor{gray}{\int}_{\mathcolor{gray}{0}}^{\mathcolor{gray}{z}} \cdot \mathbb{d} \mathcolor{gray}{z}$}} \mathcolor{gray}{z} \Upsilon^{\;\! \hat{1} \textcolor{PineGreen}{\hat{1}} \hat{2} \mathcolor{gray}{\omega} }_{\;\! \textcolor{Maroon}{(1)} \textcolor{PineGreen}{\hat{2}} } \mathcolor{gray}{\iiint} \xint{\mathcolor{gray}{-}}{18}{M}^{\;\! \mathcolor{gray}{\omega} \hat{2} }_{\;\! \hat{1} \textcolor{Maroon}{(1)} } \left( \mathcolor{gray}{\bar{q}} \right) \mathcolor{gray}{\iint} & \xint{{}^{}_{\mathcolor{gray}{-}}}{10}{g}^{\;\! \textcolor{PineGreen}{\hat{2}} \mathcolor{gray}{\omega} }_{\;\! \hat{2} \textcolor{Plum}{(i-1)} \mathcolor{gray}{0} } \left( \mathcolor{gray}{\bar{k}_{2\symup{\rho}}} \right) \text{since} \frac{ \Delta \xint{\begin{smallmatrix} ~ \\ {}^{}_{\mathcolor{gray}{-}} \\ ~ \end{smallmatrix}}{15}{k}_{\symup{z}}^{\;\! \textcolor{PineGreen}{\hat{2} \hat{1}} } \mathcolor{gray}{z} }{ 2 } ~\mathbb{d} \mathcolor{gray}{\bar{q}} ~, \label{eq:Born_approx-scalar-g-E-12-since5} \\
	\hspace{-3.4em} \text{where}\ \ \ \ \text{\textcolor{gray}{4D}} \ \ \ \ \ \Delta \xint{\begin{smallmatrix} ~ \\ {}^{}_{\mathcolor{gray}{-}} \\ ~ \end{smallmatrix}}{15}{k}_{\symup{z}}^{\;\! \textcolor{PineGreen}{\hat{2} \hat{1}} } :=\; & \xint{\begin{smallmatrix} ~ \\ {}^{}_{\mathcolor{gray}{-}} \\ ~ \end{smallmatrix}}{15}{k}_{\symup{z}}^{\;\! \textcolor{PineGreen}{\hat{2}} } \left( \mathcolor{gray}{\bar{k}_{2\symup{\rho}}} \right) + \mathcolor{gray}{q_{\symup{z}}} - \xint{\begin{smallmatrix} ~ \\ {}^{}_{\mathcolor{gray}{-}} \\ ~ \end{smallmatrix}}{15}{k}_{\symup{z}}^{\;\! \textcolor{PineGreen}{\hat{1}} } \left( \mathcolor{gray}{\bar{k}_{\symup{\rho}}} \right) ~, \label{eq:Born_approx-scalar-g-E-12-convolution6} \\
	\hspace{-3.4em} \text{in which}\ \ \ \ \text{\textcolor{gray}{4D}} \ \ \ \ \ \ \ \ \;\! \mathcolor{gray}{\bar{k}_{2\symup{\rho}}} :=\; & \mathcolor{gray}{\bar{k}_{\symup{\rho}}} - \mathcolor{gray}{\bar{q}_{\symup{\rho}}} \label{eq:Born_approx-scalar-g-E-12-convolution7} ~.
\end{align}
\end{subequations}

从\textcolor{PineGreen}{波矢失配量} \bref{eq:Born_approx-scalar-g-E-12-convolution6} 可以看出,对于\textcolor{PineGreen}{折射率}\textcolor{NavyBlue}{调制}诱导的,第 $\textcolor{Plum}{i}$ 阶\textcolor{Plum}{线性}\textcolor{NavyBlue}{散射}过程而言,如果所考虑的\textcolor{NavyBlue}{散射场}与\textcolor{NavyBlue}{泵浦源}的\textcolor{PineGreen}{本征偏振}相同($\textcolor{PineGreen}{\hat{2}} = \textcolor{PineGreen}{\hat{1}}$),则\textcolor{gray}{通光方向}($\mathcolor{gray}{z}$ 向)无\textcolor{PineGreen}{折射率}\textcolor{NavyBlue}{调制}($\mathcolor{gray}{q_{\symup{z}}} = \mathcolor{gray}{0}$)的\textcolor{Plum}{线性}\textcolor{NavyBlue}{散射}过程,反而在该\textcolor{gray}{通光方向}($\mathcolor{gray}{z}$ 向)是\textcolor{PineGreen}{波矢匹配}的,与一般的\textcolor{Plum}{非线性}过程恰恰相反:无二阶\textcolor{Plum}{非线性}系数\textcolor{NavyBlue}{倒格矢}辅助的(即 $\mathcolor{gray}{q_{\symup{z}}} = \mathcolor{gray}{0}$ 的)\textcolor{Plum}{非线性}过程一般是\textcolor{PineGreen}{失配}的。因此,尽管\textcolor{PineGreen}{折射率}\textcolor{NavyBlue}{调制}诱导的\textcolor{Plum}{线性}\textcolor{NavyBlue}{散射}过程本质上是\textcolor{Plum}{非线性}过程,它与\textcolor{Plum}{非线性}过程在\textcolor{PineGreen}{匹配}/\textcolor{PineGreen}{失配}条件上,却完全相反、对立,或相互交换。

\bref{eq:Born_approx-scalar-g-E-12-since5} 给出了统治该\textcolor{Plum}{线性}\textcolor{NavyBlue}{散射}过程的 \bref{eq:Born_approx-scalar-g-E-12-convolution1} 的\textcolor{Plum}{非线性卷积}解,其本质为\textcolor{PineGreen}{波矢完美匹配}附近的\textcolor{Plum}{非线性卷积}过程,因此该\textcolor{PineGreen}{线性光学}过程\Footnote{称其“\textcolor{Plum}{线性}”,是因为它:不涉及\textcolor{gray}{频率转换}。},在数学上并没有那么\textcolor{Plum}{线性},在性质上与 \bref{ssec:holo_L=NO_that_sim_LO} 的\textcolor{Maroon}{一次电光效应}过程,截然相反\Footnote{\textcolor{NavyBlue}{非直流}电场参与的\textcolor{Maroon}{一次电光效应},因涉及\textcolor{gray}{频率转换},而属于\textcolor{Maroon}{非线性}效应。但这却是通过高速切换\textcolor{PineGreen}{本征偏振态}做到的,而这本质上受\textcolor{PineGreen}{线性光学}控制。}。

一般来说,\textbf{\textcolor{Plum}{线性}材料系数\textcolor{NavyBlue}{弱调制}条件} \bref{eq:weak_modulated_varepsilon} 成立时,\textbf{\textcolor{NavyBlue}{泵浦未耗尽}近似条件} \bref{eq:Nondepleted-Pump-Approximation-spectrum} 也成立(即前者是后者的\textcolor{Plum}{充分不必要}条件),因为\textcolor{NavyBlue}{弱调制}导致\textcolor{NavyBlue}{弱散射}。这使得解 \bref{eq:Born_approx-scalar-g-E-12-since5} 的适用条件只有一个:\textbf{\textcolor{PineGreen}{折射率}\textcolor{NavyBlue}{弱调制}}?并非如此。实际还需满足另一个限制:\textbf{\textcolor{gray}{通光方向}上的\textcolor{Plum}{调制区域}不太厚}(以降低\textcolor{NavyBlue}{多重散射}\textcolor{Plum}{概率}),这第二个条件。

在这两个条件(且不考虑反射的情况)下,\bref{eq:Born_approx-scalar-g-E-12-since5} 的\textcolor{Plum}{非线性卷积}解以及更进一步的 \bref{chap:NLAST} 的\textcolor{Maroon}{非线性角谱}/\textcolor{Maroon}{傅立叶光学}解,收敛到\textcolor{Maroon}{严格耦合波分析}(RCWA)/\textcolor{Maroon}{傅里叶模态法}(FMM)的结果;并且能快速、准确地处理\textcolor{Plum}{二维}和\textcolor{Plum}{三维}\textcolor{PineGreen}{折射率}\textcolor{NavyBlue}{弱调制}的情况,为\textcolor{Plum}{三维}\textcolor{Plum}{各向异性}\textcolor{NavyBlue}{弹性散射}出新偏振分量的矢量\textcolor{Plum}{线性}\textcolor{Maroon}{全息},以及\textcolor{Plum}{线性}、\textcolor{Plum}{非线性}系数\textcolor{NavyBlue}{同时调制}\Footnote{\textcolor{Maroon}{光折变效应}、\textcolor{Maroon}{多层薄膜}、\textcolor{Maroon}{飞秒激光直写}\cite{chenQuasiphasematchingdivisionMultiplexingHolography2021b,gerkeAperiodicVolumeOptics2010}等机理或技术,均可导致材料的\textcolor{PineGreen}{折射率}起伏变化,并且连带对\textcolor{Plum}{二阶}及\textcolor{Plum}{高阶}\textcolor{Plum}{非线性}系数进行\textcolor{NavyBlue}{调制}。反观\textcolor{Maroon}{高压电场极化},一般只影响材料的\textcolor{Plum}{二阶}及以上的\textcolor{Plum}{非线性}系数,保持了\textcolor{Plum}{线性}\textcolor{PineGreen}{折射率}基底的大面积高度\textcolor{Plum}{均匀性}(除\textcolor{NavyBlue}{畴壁}外)。}所可能呈现的\textcolor{NavyBlue}{耦合}新现象,提供前所未有的高效算法。

\marginLeft[-2.4em]{sec:summary-chapter5}\section{\textcolor{Maroon}{Summary} 小结 \textcolor{Maroon}{of chapter 5}}\label{sec:summary-chapter5}

从 \bref{ssec:SHG_spectrum} 到 \bref{ssec:CW-3wavemix},推导了 {\one} 描述典型的\textcolor{NavyBlue}{脉冲}、\textcolor{NavyBlue}{准连续}二阶\textcolor{Plum}{非线性}(包括\textcolor{Maroon}{倍}/\textcolor{Maroon}{和}/\textcolor{Maroon}{差频}、\textcolor{Maroon}{光整流}及其级联的与之耦合的\textcolor{Maroon}{电光效应}、\textcolor{Maroon}{上转换}和同时发生的\textcolor{Maroon}{下转换},以及\textcolor{Maroon}{三波混频})过程中,电场矢量所满足的 3 维耦合波动方程(组);{\two} 它们在\textbf{标量\textcolor{Plum}{非线性}\textcolor{NavyBlue}{波源}} \bref{eq:scalar_nonlinear_drive}、\textbf{标量\textcolor{NavyBlue}{调制场}} \bref{eq:scalar_chi2_modulation}、\textbf{\textbf{\textcolor{NavyBlue}{泵浦未耗尽}}} \bref{eq:Nondepleted-Pump-Approximation} 这三个\textbf{近似条件}的排列组合下化简后的单个方程、方程组;{\three} 最终得到了它们在 3 维\textcolor{gray}{谱域} $\left( \mathcolor{gray}{\omega}, \mathcolor{gray}{\bar{k}_{\symup{\rho}}} \right)$ 的严格\textcolor{Plum}{数学表达式},及这些方程所对应的\textcolor{Plum}{非线性卷积解}。

还有一些二阶\textcolor{Plum}{非线性}过程由于参与\textcolor{Maroon}{混频}的光场的\textcolor{NavyBlue}{能量}/\textcolor{gray}{频率}\textcolor{PineGreen}{失配量}过大,\textcolor{Plum}{解}性质大变,其在形式上不再是\textcolor{Plum}{非线性卷积},而是趋于一阶\textcolor{PineGreen}{线性过程}的\textcolor{Plum}{解},如 \bref{ssec:EO=LO_that_sim_NO} 中的一次\textcolor{Maroon}{电光效应};反而,一些\textcolor{Plum}{线性}过程的\textcolor{Plum}{解},却拥有\textcolor{Plum}{非线性卷积}形式,如 \bref{ssec:EO=LO_that_sim_NO} 中的 \textcolor{PineGreen}{折射率}\textcolor{NavyBlue}{微扰}\textcolor{Maroon}{势散射}。

除了上述二阶\textcolor{Plum}{非线性}过程外,三阶及以上的\textcolor{Plum}{非线性}效应所对应的耦合波方程(组),也均可以按照上述模板,“格式化”为\textcolor{Plum}{非线性}\textcolor{Plum}{傅立叶}光学框架下的\textcolor{Maroon}{时空谱}耦合波方程组。

以三阶\textcolor{Plum}{非线性}为例,通过照搬 \bref{ssec:pulse-3wavemix,ssec:CW-3wavemix} 的\textcolor{Plum}{推导},则可获得\textcolor{NavyBlue}{连续}/\textcolor{NavyBlue}{脉冲光}\textcolor{Maroon}{四波混频}的\textcolor{Maroon}{时空谱}耦合波方程组;而通过复制 \bref{ssec:EO=LO_that_sim_NO,ssec:holo_L=NO_that_sim_LO} 的形式,则可以得到\textcolor{Maroon}{二次电光效应}、\textcolor{gray}{时}/\textcolor{gray}{空域}\textcolor{Maroon}{自}/\textcolor{Maroon}{散聚焦}、\textcolor{Maroon}{自}/\textcolor{Maroon}{交叉相位调制}的\textcolor{Maroon}{时空谱}耦合波方程。然而,我们不再重复上述过程,以得到\textcolor{Plum}{三维}三阶及以上\textcolor{Plum}{非线性}波动\textcolor{Plum}{方程},在\textcolor{gray}{时间频率}和\textcolor{gray}{空间频率域}的\textcolor{Maroon}{时空谱}耦合波方程。

这是因为:一方面,由于三阶及以上\textcolor{Plum}{非线性}光学过程至少需要四个电场参与\textcolor{Maroon}{混频},导致可能会\textcolor{Plum}{计算}四个(频率)及以上矢量光的\textcolor{PineGreen}{本征值}问题,以至于即使基于 \bref{chap:LFCO} 节彻底\textcolor{Plum}{解析}的\textcolor{Maroon}{傅立叶}\textcolor{Plum}{线性}光学模型,上述\textcolor{PineGreen}{偏振态}的\textcolor{Plum}{计算量}也会随着\textcolor{Maroon}{混频}场数量\textcolor{Plum}{线性}升高,尽管这并不是问题。

最主要的 2 个限制是:其一,从后续 \bref{chap:NLAST} 对\textcolor{Plum}{非线性卷积}\textcolor{Plum}{解}的\textcolor{Plum}{线性卷积}化的经验上来看,任何带\textcolor{NavyBlue}{耦合}的\textcolor{Maroon}{时空谱}耦合波方程组,最终都会\textcolor{Plum}{收敛}到一类(\textcolor{Plum}{非线性})\textcolor{Plum}{卷积}型\textcolor{Plum}{高维}\textcolor{Plum}{微-积分方程组},它们的\textcolor{Plum}{解析}\textcolor{Plum}{解}对于目前的人类而言遥不可及(至今,人类甚至只能得到少数特殊形式的\textcolor{Plum}{一维}\textcolor{Plum}{积分方程}的\textcolor{Plum}{封闭}\textcolor{Plum}{解}\cite{ponomarevAsymptoticSolutionConvolution2021})。

除此之外,第二个困难是,即使材料在\textcolor{Plum}{线性}和\textcolor{Plum}{非线性}系数上都是\textcolor{Plum}{均匀}的,材料本身也可通过正逆\textcolor{Maroon}{压电}、\textcolor{Maroon}{弹光}、\textcolor{Maroon}{电光}、\textcolor{Maroon}{声光}、\textcolor{Maroon}{布里渊散射}、\textcolor{Maroon}{拉曼散射}等效应,直接或间接地以不同程度参与\textcolor{NavyBlue}{光-物质}\textcolor{NavyBlue}{耦合};以至于晶体内实际进行的,不仅只有\textcolor{NavyBlue}{光-光相互作用}的\textcolor{NavyBlue}{参数过程};而一旦涉及\textcolor{NavyBlue}{光-物质}的\textcolor{Plum}{非线性}\textcolor{NavyBlue}{耦合},宏观层面的\textcolor{Plum}{三维}空间中的\textcolor{Maroon}{时空}\textcolor{Plum}{解析}\textcolor{Plum}{解}是肉眼可见地屈指可数:几乎是虚位以待。

因此,\textcolor{NavyBlue}{光-光}\textcolor{NavyBlue}{耦合}、\textcolor{NavyBlue}{光-物质}\textcolor{NavyBlue}{耦合}中,任何一种\textcolor{NavyBlue}{耦合}的存在,都将导致\textcolor{Plum}{三维}\textcolor{Plum}{非线性}波动方程组,不仅从\textcolor{gray}{空域}上无法得到\textcolor{Plum}{解析}\textcolor{Plum}{解},在对其进行\textcolor{Plum}{二维}或以上的\textcolor{Plum}{傅立叶变换}以转化为\textcolor{Plum}{偏微-积分方程组}后,仍然只能终结于\textcolor{Plum}{非线性卷积方程组}\textcolor{Plum}{解},而无法继续简化为\textcolor{Plum}{线性卷积方程组},进而继续简化为\textcolor{Plum}{线性}\textcolor{Plum}{傅立叶变换}\textcolor{Plum}{解}。

此外,即便没有\textcolor{NavyBlue}{耦合},若无后续 \bref{chap:NLAST} 或其它的加速策略,单个\textcolor{Plum}{非线性卷积}的\textcolor{Plum}{计算量}仍然很大,以至于作为\textcolor{Plum}{非线性卷积}的孩子的人类,将被\textcolor{NavyBlue}{智子}锁定般地,对于彻底理解造物主最强大的工具:\textbf{\textcolor{Plum}{全局}\textcolor{Plum}{非线性}}之\textcolor{Plum}{非线性卷积}本身,彻底无缘。

鲁迅曾说,绝望就是把美好的东西撕破给人看。之后(\bref{chap:NLAST})我们会深深地体会到这一点:不带\textcolor{NavyBlue}{耦合}的\textcolor{Plum}{单个}\textcolor{Plum}{高维}\textcolor{Plum}{非线性卷积}的\textcolor{Plum}{傅立叶变换}\textcolor{Plum}{解}是可以漂亮地存在的;但问题就是,上帝只给你看这幅画的冰山一角,其他广袤的区域(如带\textcolor{NavyBlue}{耦合}的多个\textcolor{Plum}{高维}\textcolor{Plum}{非线性卷积}\textcolor{Plum}{方程组}的\textcolor{Plum}{傅立叶变换}\textcolor{Plum}{解})故作保留、鸦雀无声般地闭口不谈;这种戛然而止的余音袅袅、这种万赖此都寂,但余钟磬音的渐行渐远,让我感到彻头彻尾的造物弄人。

知不可乎骤得,托遗响于悲风。

\cite{dregerSecondharmonicGenerationNonlinear1990,zubairyAnalyticApproachSecondharmonic1985}

too many diffs, cannot push to gitee, but github is ok

\cite{katoSecondharmonicGeneration20481986,katoTemperaturetuned90Phasematching1994,brunerTemperaturedependentSellmeierEquation2003,jundtTemperaturedependentSellmeierEquation1997,katoSellmeierThermoopticDispersion2002}