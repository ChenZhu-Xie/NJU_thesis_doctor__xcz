\marginLeft[-2.4em]{chap:NFCO}\chapter{任意 \texorpdfstring{$\bar{\bar{\bar{\chi}}}$}{$\bar{\bar{\bar{\text{χ}}}}$} 材料里的矢量非线性傅立叶晶体光学}\label{chap:NFCO}

\bref{ssec:E-waveq-nonlinear} 末得到了\textcolor{PineGreen}{纯电(非磁)各向异性}介质中的\textcolor{Plum}{非线性}矢量电场波动方程 \bref{eq:simplify7-LE0-SVA-V_1singular-nokxky-zeta-g},其右侧的二阶\textcolor{Plum}{局域}\textcolor{Plum}{非线性}\textcolor{NavyBlue}{电偶-$(\text{电偶}\otimes\text{电偶})$}极\textcolor{NavyBlue}{波源}(的\textcolor{Plum}{横向}部分) $\Xint{\mathcolor{gray}{-}}{25}{\bar{P}}^{\;\! \mathcolor{gray}{\omega} \textcolor{PineGreen}{\imath}}_{\;\! \symup{\rho} \mathcolor{gray}{z} \textcolor{Maroon}{(2)}}$(\bref{eq:vec-DP^(2)-p_pp,eq:vec-DP^(2)-p_pp}),可以按需更改为三阶、四阶\textcolor{Plum}{非线性} $\Xint{\mathcolor{gray}{-}}{25}{\bar{P}}^{\;\! \mathcolor{gray}{\omega} \textcolor{PineGreen}{\imath}}_{\;\! \symup{\rho} \mathcolor{gray}{z} \textcolor{Maroon}{(3)}}, \Xint{\mathcolor{gray}{-}}{25}{\bar{P}}^{\;\! \mathcolor{gray}{\omega} \textcolor{PineGreen}{\imath}}_{\;\! \symup{\rho} \mathcolor{gray}{z} \textcolor{Maroon}{(4)}}$,或者一阶\textcolor{NavyBlue}{势散射}源 $\Xint{\mathcolor{gray}{-}}{25}{\bar{P}}^{\;\! \mathcolor{gray}{\omega} \textcolor{PineGreen}{\imath}}_{\;\! \symup{\rho} \mathcolor{gray}{z} \textcolor{Maroon}{(1)}}$(\bref{eq:nonlinear(2)-wave_wkrho-simplify4-2}),甚至他们的\textcolor{Plum}{线性组合}/\textcolor{Plum}{叠加} $\Xint{\mathcolor{gray}{-}}{25}{\bar{P}}^{\;\! \mathcolor{gray}{\omega} \textcolor{PineGreen}{\imath}}_{\;\! \symup{\rho} \mathcolor{gray}{z}}$。

将每一个参与相互作用的\textcolor{gray}{时间频率组分} $\mathcolor{gray}{\omega}$ 所对应的 \bref{eq:simplify7-LE0-SVA-V_1singular-nokxky-zeta-g} 组合起来,即得到\textcolor{PineGreen}{实验室}\textcolor{Plum}{坐标系} \textcolor{PineGreen}{$\mathcal{Z}$ 系}下的矢量(电场)\textcolor{Maroon}{时空谱}耦合波方程组 --- 它既可以由相干\textcolor{NavyBlue}{脉冲光连续谱} $\{ \mathcolor{gray}{\omega} \}$ 对应的\textcolor{Plum}{无穷个}矢量\textcolor{Maroon}{时空谱}耦合波方程组成,也可以由\textcolor{Plum}{有限个}\textcolor{NavyBlue}{连续光} $\{ \mathcolor{gray}{\omega}_{\textcolor{Maroon}{i}} \}$ 对应的有限个矢量\textcolor{Maroon}{时空谱}耦合波方程组成;也可以只是\textcolor{gray}{单一频率} $\mathcolor{gray}{\omega}$ 的光,其自己参与的如\textcolor{Maroon}{三}/\textcolor{Maroon}{四波混频}或\textcolor{PineGreen}{折射率微调制}引起的\textcolor{Maroon}{(势)散射过程}等,所对应的\textcolor{Plum}{单个}矢量\textcolor{Maroon}{时空谱}的传播/波动方程。

%\vspace*{-7.5em}

\marginLeft[-2.4em]{sec:up_convert}\section{\textcolor{Maroon}{Up conversion} 上转换 - 电场本征复振幅 \textcolor{Maroon}{equation}}\label{sec:up_convert}

利用 \bref{eq:vec-amp_polar} 即 $\Xint{{}^{}_{\mathcolor{gray}{-}}}{10}{\bar{g}}^{\;\!\mathcolor{gray}{\omega} \textcolor{PineGreen}{\jmath}}_{\;\! \mathcolor{gray}{z}} := \Xint{\begin{smallmatrix} ~ \\ {}^{}_{\mathcolor{gray}{-}} \\ ~ \end{smallmatrix}}{09}{\mathtt{g}}^{\;\!\mathcolor{gray}{\omega} \textcolor{PineGreen}{\jmath}}_{\;\! \mathcolor{gray}{z}} \Xint{{}^{}_{\mathcolor{gray}{-}}}{10}{\bar{g}}^{\;\!\mathcolor{gray}{\omega} \textcolor{PineGreen}{\jmath}}$ 在\textcolor{PineGreen}{纯电(非磁)各向异性}介质中的\textcolor{Plum}{横向}形式,将\textcolor{Plum}{复}矢量 $\Xint{{}^{}_{\mathcolor{gray}{-}}}{10}{\bar{g}}^{\;\!\mathcolor{gray}{\omega} \textcolor{PineGreen}{\pm}}_{\;\! \textcolor{Maroon}{\Yup}}$ 三分量\textcolor{Plum}{复归一化}\Footnote{注意,不像 \bref{eq:transition_matrix-transverse_input},\textcolor{PineGreen}{本征偏振态} $\Xint{{}^{}_{\mathcolor{gray}{-}}}{10}{\bar{g}}^{\;\!\mathcolor{gray}{\omega} \textcolor{PineGreen}{\pm}}_{\;\! \textcolor{Maroon}{\Yup}}, \Xint{{}^{}_{\mathcolor{gray}{-}}}{10}{\bar{g}}^{\;\!\mathcolor{gray}{\omega} \textcolor{PineGreen}{\pm}}_{\;\! \textcolor{Maroon}{\symup{\rho}}}$ 在\textcolor{Plum}{线性}\textcolor{PineGreen}{晶体光学}中无需\textcolor{Plum}{(三维)复归一化},但在\textcolor{Plum}{非线性}\textcolor{PineGreen}{晶体光学}中最好\textcolor{Plum}{三维复归一化},这既有助于\textcolor{Plum}{标准化}后续引入的\textcolor{NavyBlue}{有效非线性系数},又赋予和明确了\textcolor{PineGreen}{本征复振幅} $\Xint{\begin{smallmatrix} ~ \\ {}^{}_{\mathcolor{gray}{-}} \\ ~ \end{smallmatrix}}{09}{\mathtt{g}}^{\;\!\mathcolor{gray}{\omega} \textcolor{PineGreen}{\pm}}_{\;\! \mathcolor{gray}{z}}$、\textcolor{PineGreen}{本征偏振态} $\Xint{{}^{}_{\mathcolor{gray}{-}}}{10}{\bar{g}}^{\;\!\mathcolor{gray}{\omega} \textcolor{PineGreen}{\pm}}_{\;\! \textcolor{Maroon}{\Yup}}$ 各自的物理意义。但实际上,\textcolor{Plum}{非线性}\textcolor{PineGreen}{晶体光学}也允许\textcolor{Plum}{不归一化} $\Xint{{}^{}_{\mathcolor{gray}{-}}}{10}{\bar{g}}^{\;\!\mathcolor{gray}{\omega} \textcolor{PineGreen}{\pm}}_{\;\! \textcolor{Maroon}{\Yup}}$ 或 $\Xint{{}^{}_{\mathcolor{gray}{-}}}{10}{\bar{g}}^{\;\!\mathcolor{gray}{\omega} \textcolor{PineGreen}{\pm}}_{\;\! \textcolor{Maroon}{\symup{\rho}}}$,见下文。}后的\textcolor{Plum}{二维}\textcolor{PineGreen}{本征偏振态} $\Xint{{}^{}_{\mathcolor{gray}{-}}}{10}{\bar{g}}^{\;\!\mathcolor{gray}{\omega} \textcolor{PineGreen}{\pm}}_{\;\! \textcolor{Maroon}{\symup{\rho}} \mathcolor{gray}{z}} := \Xint{\begin{smallmatrix} ~ \\ {}^{}_{\mathcolor{gray}{-}} \\ ~ \end{smallmatrix}}{09}{\mathtt{g}}^{\;\!\mathcolor{gray}{\omega} \textcolor{PineGreen}{\pm}}_{\;\! \mathcolor{gray}{z}} \Xint{{}^{}_{\mathcolor{gray}{-}}}{10}{\hat{g}}^{\;\!\mathcolor{gray}{\omega} \textcolor{PineGreen}{\pm}}_{\;\! \textcolor{Maroon}{\symup{\rho}}}$代入矢量\textcolor{Plum}{非线性}波动方程 \bref{eq:simplify7-LE0-SVA-V_1singular-nokxky-zeta-g} 的\textcolor{NavyBlue}{左侧场}中,并两侧\textcolor{Plum}{点乘}不含 $\mathcolor{gray}{z}$ 的\textcolor{Plum}{二维横向}\textcolor{PineGreen}{本征偏振态} $\Xint{{}^{}_{\mathcolor{gray}{-}}}{10}{\hat{g}}^{\;\!\mathcolor{gray}{\omega} \textcolor{PineGreen}{\pm}}_{\;\! \textcolor{Maroon}{\symup{\rho}}}$ 的\textcolor{Plum}{共轭转置} $\Xint{{}^{}_{\mathcolor{gray}{-}}}{10}{\hat{g}}^{\;\!\mathcolor{gray}{\omega} \textcolor{PineGreen}{\pm} \textcolor{Plum}{\dag}}_{\;\! \textcolor{Maroon}{\symup{\rho}}}$,可得
\begin{subequations} \label{eq:simplify7-scalar}
	\begin{align}
		\Xint{{}^{}_{\mathcolor{gray}{-}}}{10}{\hat{g}}^{\;\!\mathcolor{gray}{\omega} \textcolor{PineGreen}{\pm}}_{\;\! \textcolor{Maroon}{\symup{\rho}}} \mathcolor{gray}{\nabla_z} \Xint{\begin{smallmatrix} ~ \\ {}^{}_{\mathcolor{gray}{-}} \\ ~ \end{smallmatrix}}{09}{\mathtt{g}}^{\;\!\mathcolor{gray}{\omega} \textcolor{PineGreen}{\pm}}_{\;\! \mathcolor{gray}{z}} &= \mathbb{i} k_{\textcolor{Maroon}{\mathsf{o}} \mathcolor{gray}{\omega}}^{\;\! 2} \frac{\Xint{\mathcolor{gray}{-}}{25}{\bar{P}}^{\;\! \mathcolor{gray}{\omega} \textcolor{PineGreen}{\pm}}_{\;\! \textcolor{Maroon}{\symup{\rho}} \mathcolor{gray}{z}}}{2 \Xint{\begin{smallmatrix} ~ \\ {}^{}_{\mathcolor{gray}{-}} \\ ~ \end{smallmatrix}}{15}{k}_{\;\! \symup{z}}^{\;\! \mathcolor{gray}{\omega} \textcolor{PineGreen}{\pm}} \mathbb{e}^{\mathbb{i} \Xint{\begin{smallmatrix} ~ \\ {}^{}_{\mathcolor{gray}{-}} \\ ~ \end{smallmatrix}}{15}{k}_{\symup{z}}^{\;\! \mathcolor{gray}{\omega} \textcolor{PineGreen}{\pm}} \mathcolor{gray}{z}}} \label{eq:simplify7-scalar-g} \\
		\mathcolor{gray}{\nabla_z} \Xint{\begin{smallmatrix} ~ \\ {}^{}_{\mathcolor{gray}{-}} \\ ~ \end{smallmatrix}}{09}{\mathtt{g}}^{\;\!\mathcolor{gray}{\omega} \textcolor{PineGreen}{\pm}}_{\;\! \mathcolor{gray}{z}} &= \mathbb{i} k_{\textcolor{Maroon}{\mathsf{o}} \mathcolor{gray}{\omega}}^{\;\! 2} \frac{\Xint{{}^{}_{\mathcolor{gray}{-}}}{10}{\hat{g}}^{\;\!\mathcolor{gray}{\omega} \textcolor{PineGreen}{\pm} \textcolor{Plum}{\dag}}_{\;\! \textcolor{Maroon}{\symup{\rho}}} \cdot \Xint{\mathcolor{gray}{-}}{25}{\bar{P}}^{\;\! \mathcolor{gray}{\omega} \textcolor{PineGreen}{\pm}}_{\;\! \textcolor{Maroon}{\symup{\rho}} \mathcolor{gray}{z}}}{\Xint{{}^{}_{\mathcolor{gray}{-}}}{10}{\hat{g}}^{\;\!\mathcolor{gray}{\omega} \textcolor{PineGreen}{\pm} \textcolor{Plum}{\dag}}_{\;\! \textcolor{Maroon}{\symup{\rho}}} \cdot \Xint{{}^{}_{\mathcolor{gray}{-}}}{10}{\hat{g}}^{\;\!\mathcolor{gray}{\omega} \textcolor{PineGreen}{\pm}}_{\;\! \textcolor{Maroon}{\symup{\rho}}} 2 \Xint{\begin{smallmatrix} ~ \\ {}^{}_{\mathcolor{gray}{-}} \\ ~ \end{smallmatrix}}{15}{k}_{\;\! \symup{z}}^{\;\! \mathcolor{gray}{\omega} \textcolor{PineGreen}{\pm}} \mathbb{e}^{\mathbb{i} \Xint{\begin{smallmatrix} ~ \\ {}^{}_{\mathcolor{gray}{-}} \\ ~ \end{smallmatrix}}{15}{k}_{\symup{z}}^{\;\! \mathcolor{gray}{\omega} \textcolor{PineGreen}{\pm}} \mathcolor{gray}{z}}} ~,  \label{eq:simplify7-scalar-g-conjugate}
	\end{align}
\end{subequations}
其中 \bref{eq:simplify7-scalar-g-conjugate} 即为标量\textcolor{Maroon}{时空谱},即\textcolor{PineGreen}{本征复振幅} $\Xint{\begin{smallmatrix} ~ \\ {}^{}_{\mathcolor{gray}{-}} \\ ~ \end{smallmatrix}}{09}{\mathtt{g}}^{\;\!\mathcolor{gray}{\omega} \textcolor{PineGreen}{\pm}}_{\;\! \mathcolor{gray}{z}}$ 满足的矢量\textcolor{Plum}{非线性}波动方程。分母中的 $\Xint{{}^{}_{\mathcolor{gray}{-}}}{10}{\hat{g}}^{\;\!\mathcolor{gray}{\omega} \textcolor{PineGreen}{\pm} \textcolor{Plum}{\dag}}_{\;\! \textcolor{Maroon}{\symup{\rho}}} \cdot \Xint{{}^{}_{\mathcolor{gray}{-}}}{10}{\hat{g}}^{\;\!\mathcolor{gray}{\omega} \textcolor{PineGreen}{\pm}}_{\;\! \textcolor{Maroon}{\symup{\rho}}}$,其实在暗示 $\Xint{{}^{}_{\mathcolor{gray}{-}}}{10}{\hat{g}}^{\;\!\mathcolor{gray}{\omega} \textcolor{PineGreen}{\pm}}_{\;\! \textcolor{Maroon}{\symup{\rho}}}$ 既可以是\textcolor{Plum}{二维复归一化},也可以是\textcolor{Plum}{三维复归一化}后的。这对应 $\Xint{{}^{}_{\mathcolor{gray}{-}}}{10}{\hat{g}}^{\;\!\mathcolor{gray}{\omega} \textcolor{PineGreen}{\pm} \textcolor{Plum}{\dag}}_{\;\! \textcolor{Maroon}{\symup{\rho}}} \cdot \Xint{{}^{}_{\mathcolor{gray}{-}}}{10}{\hat{g}}^{\;\!\mathcolor{gray}{\omega} \textcolor{PineGreen}{\pm}}_{\;\! \textcolor{Maroon}{\symup{\rho}}}$ 的值,既可以是也可以不是 $1$。并且甚至可以对 \bref{eq:simplify7-scalar-g} 左侧随便\textcolor{Plum}{点乘}一个\textcolor{Plum}{二维}复向量(场) $\Xint{{}^{}_{\mathcolor{gray}{-}}}{04}{\bar{c}}^{\;\!\mathcolor{gray}{\omega} \textcolor{PineGreen}{\pm}}_{\;\! \textcolor{Maroon}{\symup{\rho}}}$(不一定非得是\textcolor{Plum}{横向}\textcolor{PineGreen}{本征偏振态}的\textcolor{Plum}{共轭转置} $\Xint{{}^{}_{\mathcolor{gray}{-}}}{10}{\hat{g}}^{\;\!\mathcolor{gray}{\omega} \textcolor{PineGreen}{\pm} \textcolor{Plum}{\dag}}_{\;\! \textcolor{Maroon}{\symup{\rho}}}$)都行,只需保证 \bref{eq:simplify7-scalar-g-conjugate} 分母中的 $\Xint{{}^{}_{\mathcolor{gray}{-}}}{04}{\bar{c}}^{\;\!\mathcolor{gray}{\omega} \textcolor{PineGreen}{\pm}}_{\;\! \textcolor{Maroon}{\symup{\rho}}} \cdot \Xint{{}^{}_{\mathcolor{gray}{-}}}{10}{\hat{g}}^{\;\!\mathcolor{gray}{\omega} \textcolor{PineGreen}{\pm}}_{\;\! \textcolor{Maroon}{\symup{\rho}}} \neq 0$。

\bref{eq:simplify7-scalar} 中所有\textcolor{Plum}{横向} $\textcolor{Maroon}{\symup{\rho}}$ 场,somehow\Footnote{出于物理学家的直觉和对形式美的追求。类似数学家的“注意到”,但没有他们的“注意到”那么严谨。}可进一步写做笛卡尔\textcolor{Plum}{三分量} $\textcolor{Maroon}{\Yup}$ 形式
\begin{subequations} \label{eq:simplify8-scalar}
	\begin{align}
		\Xint{{}^{}_{\mathcolor{gray}{-}}}{10}{\hat{g}}^{\;\!\mathcolor{gray}{\omega} \textcolor{PineGreen}{\pm}}_{\;\! \textcolor{Maroon}{\Yup}} \mathcolor{gray}{\nabla_z} \Xint{\begin{smallmatrix} ~ \\ {}^{}_{\mathcolor{gray}{-}} \\ ~ \end{smallmatrix}}{09}{\mathtt{g}}^{\;\!\mathcolor{gray}{\omega} \textcolor{PineGreen}{\pm}}_{\;\! \mathcolor{gray}{z}} &= \mathbb{i} k_{\textcolor{Maroon}{\mathsf{o}} \mathcolor{gray}{\omega}}^{\;\! 2} \frac{\Xint{\mathcolor{gray}{-}}{25}{\bar{P}}^{\;\! \mathcolor{gray}{\omega} \textcolor{PineGreen}{\pm}}_{\;\! \textcolor{Maroon}{\Yup} \mathcolor{gray}{z}}}{2 \Xint{\begin{smallmatrix} ~ \\ {}^{}_{\mathcolor{gray}{-}} \\ ~ \end{smallmatrix}}{15}{k}_{\;\! \symup{z}}^{\;\! \mathcolor{gray}{\omega} \textcolor{PineGreen}{\pm}} \mathbb{e}^{\mathbb{i} \Xint{\begin{smallmatrix} ~ \\ {}^{}_{\mathcolor{gray}{-}} \\ ~ \end{smallmatrix}}{15}{k}_{\symup{z}}^{\;\! \mathcolor{gray}{\omega} \textcolor{PineGreen}{\pm}} \mathcolor{gray}{z}}} \label{eq:simplify8-scalar-g} \\
		\mathcolor{gray}{\nabla_z} \Xint{\begin{smallmatrix} ~ \\ {}^{}_{\mathcolor{gray}{-}} \\ ~ \end{smallmatrix}}{09}{\mathtt{g}}^{\;\!\mathcolor{gray}{\omega} \textcolor{PineGreen}{\pm}}_{\;\! \mathcolor{gray}{z}} &= \mathbb{i} k_{\textcolor{Maroon}{\mathsf{o}} \mathcolor{gray}{\omega}}^{\;\! 2} \frac{\Xint{{}^{}_{\mathcolor{gray}{-}}}{10}{\hat{g}}^{\;\!\mathcolor{gray}{\omega} \textcolor{PineGreen}{\pm} \textcolor{Plum}{\dag}}_{\;\! \textcolor{Maroon}{\Yup}} \cdot \Xint{\mathcolor{gray}{-}}{25}{\bar{P}}^{\;\! \mathcolor{gray}{\omega} \textcolor{PineGreen}{\pm}}_{\;\! \textcolor{Maroon}{\Yup} \mathcolor{gray}{z}}}{\Xint{{}^{}_{\mathcolor{gray}{-}}}{10}{\hat{g}}^{\;\!\mathcolor{gray}{\omega} \textcolor{PineGreen}{\pm} \textcolor{Plum}{\dag}}_{\;\! \textcolor{Maroon}{\Yup}} \cdot \Xint{{}^{}_{\mathcolor{gray}{-}}}{10}{\hat{g}}^{\;\!\mathcolor{gray}{\omega} \textcolor{PineGreen}{\pm}}_{\;\! \textcolor{Maroon}{\Yup}} 2 \Xint{\begin{smallmatrix} ~ \\ {}^{}_{\mathcolor{gray}{-}} \\ ~ \end{smallmatrix}}{15}{k}_{\;\! \symup{z}}^{\;\! \mathcolor{gray}{\omega} \textcolor{PineGreen}{\pm}} \mathbb{e}^{\mathbb{i} \Xint{\begin{smallmatrix} ~ \\ {}^{}_{\mathcolor{gray}{-}} \\ ~ \end{smallmatrix}}{15}{k}_{\symup{z}}^{\;\! \mathcolor{gray}{\omega} \textcolor{PineGreen}{\pm}} \mathcolor{gray}{z}}} ~,  \label{eq:simplify8-scalar-g-conjugate}
	\end{align}
\end{subequations}
注意,\bref{eq:simplify8-scalar-g-conjugate} 仍属于矢量\textcolor{Plum}{非线性}波动方程,尽管方程左侧是\textcolor{PineGreen}{本征复振幅}标量场 $\Xint{\begin{smallmatrix} ~ \\ {}^{}_{\mathcolor{gray}{-}} \\ ~ \end{smallmatrix}}{09}{\mathtt{g}}^{\;\!\mathcolor{gray}{\omega} \textcolor{PineGreen}{\pm}}_{\;\! \mathcolor{gray}{z}}$ 的 $\mathcolor{gray}{z}$ 向偏导:因为 {\one} 右侧的 $\Xint{\mathcolor{gray}{-}}{25}{\bar{P}}^{\;\! \mathcolor{gray}{\omega} \textcolor{PineGreen}{\pm}}_{\;\! \textcolor{Maroon}{\Yup} \mathcolor{gray}{z}}$ 是矢量的;{\two} \textcolor{PineGreen}{本征复振幅}标量场 $\Xint{\begin{smallmatrix} ~ \\ {}^{}_{\mathcolor{gray}{-}} \\ ~ \end{smallmatrix}}{09}{\mathtt{g}}^{\;\!\mathcolor{gray}{\omega} \textcolor{PineGreen}{\pm}}_{\;\! \mathcolor{gray}{z}}$ 一旦已知,再乘以\textcolor{PineGreen}{本征偏振态} $\Xint{{}^{}_{\mathcolor{gray}{-}}}{10}{\hat{g}}^{\;\!\mathcolor{gray}{\omega} \textcolor{PineGreen}{\pm}}_{\;\! \textcolor{Maroon}{\Yup}}$ 后,可直接转换为矢量\textcolor{Maroon}{时空谱}三分量 $\Xint{{}^{}_{\mathcolor{gray}{-}}}{10}{\bar{g}}^{\;\!\mathcolor{gray}{\omega} \textcolor{PineGreen}{\pm}}_{\;\! \textcolor{Maroon}{\Yup} \mathcolor{gray}{z}} := \Xint{\begin{smallmatrix} ~ \\ {}^{}_{\mathcolor{gray}{-}} \\ ~ \end{smallmatrix}}{09}{\mathtt{g}}^{\;\!\mathcolor{gray}{\omega} \textcolor{PineGreen}{\pm}}_{\;\! \mathcolor{gray}{z}} \Xint{{}^{}_{\mathcolor{gray}{-}}}{10}{\hat{g}}^{\;\!\mathcolor{gray}{\omega} \textcolor{PineGreen}{\pm}}_{\;\! \textcolor{Maroon}{\Yup}}$,或通过 \bref{eq:vec-eigenmode_amp-matrix} 一步到位至电矢量场\textcolor{Maroon}{傅立叶谱} $\Xint{\mathcolor{gray}{-}}{25}{\bar{E}}^{\;\!\mathcolor{gray}{\omega}}_{\;\! \mathcolor{gray}{z}}$。

\marginLeft[-2.4em]{ssec:SHG_spectrum}\subsection{脉冲光倍频 - 电场本征复振幅方程}\label{ssec:SHG_spectrum}

对于以\textcolor{NavyBlue}{脉冲光}\textcolor{Maroon}{倍频}\cite{boydNonlinearOptics2019}为主\Footnote{若 $\mathcolor{gray}{\omega}_{\textcolor{Maroon}{\text{P}}}, \mathcolor{gray}{\omega}$ 分别在单个\textcolor{NavyBlue}{泵浦光}脉冲\textcolor{gray}{中心频率} $\mathcolor{gray}{\Omega}_{\textcolor{Maroon}{\text{P}}} = \mathcolor{gray}{\Omega} \big/ 2$ 及其产生的 $2\mathcolor{gray}{\omega}_{\textcolor{Maroon}{\text{P}}}$ \textcolor{Maroon}{倍频}光脉冲\textcolor{gray}{中心频率} $\mathcolor{gray}{\Omega} = 2\mathcolor{gray}{\Omega}_{\textcolor{Maroon}{\text{P}}}$ 附近,且 $\mathcolor{gray}{\omega} = 2\mathcolor{gray}{\omega}_{\textcolor{Maroon}{\text{P}}} > 0$,则该式代表单\textcolor{NavyBlue}{脉冲光}\textcolor{Maroon}{倍频}过程。}、以\textcolor{NavyBlue}{脉冲}\textcolor{Maroon}{光整流}后续级联\textcolor{Maroon}{电光效应}\cite{jangMulticycleTerahertzPulse2020}为辅\Footnote{若 $\mathcolor{gray}{\omega}, \mathcolor{gray}{\omega}_{\textcolor{Maroon}{\text{THz}}}$ 分别在单个\textcolor{NavyBlue}{泵浦}光脉冲\textcolor{gray}{中心频率} $\mathcolor{gray}{\Omega} \gg \mathcolor{gray}{\Omega}_{\textcolor{Maroon}{\text{THz}}}$ 及其产生的 \textcolor{Maroon}{THz} 脉冲的\textcolor{gray}{中心频率} $\mathcolor{gray}{\Omega}_{\textcolor{Maroon}{\text{THz}}} \ll \mathcolor{gray}{\Omega}$ 附近,且 $\mathcolor{gray}{\omega} \gg \mathcolor{gray}{\omega}_{\textcolor{Maroon}{\text{THz}}} > 0$,则该式代表\textcolor{NavyBlue}{脉冲}\textcolor{Maroon}{光整流}后续级联\textcolor{Maroon}{电光效应}\cite{jangMulticycleTerahertzPulse2020}过程。对应\textcolor{NavyBlue}{脉冲电}(\textcolor{Maroon}{THz})与\textcolor{NavyBlue}{脉冲光}的\textcolor{Maroon}{和频}。}的 $\mathcolor{gray}{\omega'} + \left( \mathcolor{gray}{\omega}-\mathcolor{gray}{\omega'} \right) \to \mathcolor{gray}{\omega} > 0$\Footnote{在映射到\textcolor{NavyBlue}{物理过程}时,默认\textcolor{gray}{各频率}(对应 cos 余弦)为正;但在\textcolor{Plum}{数学积分}(对应 e 指数)中可为负。}二阶\textcolor{Plum}{非线性}\textcolor{gray}{频率}\textcolor{Maroon}{上转换}过程,波动方程 \bref{eq:simplify7-scalar-g} 与 \bref{eq:simplify8-scalar-g} \textcolor{Plum}{非线性}\textcolor{NavyBlue}{波源}项 $\Xint{\mathcolor{gray}{-}}{25}{\bar{P}}^{\;\! \mathcolor{gray}{\omega} \textcolor{PineGreen}{\pm}}_{\;\! \textcolor{Maroon}{\Yup} \mathcolor{gray}{z}} = \mathcal F \left[ \bar{P}^{\;\! \mathcolor{gray}{\omega} \textcolor{PineGreen}{\pm}}_{\;\! \textcolor{Maroon}{\Yup} \mathcolor{gray}{z}} \right]$ 进一步限定为 \bref{eq:vec-DP^(2)-p_pp} 的“\textcolor{PineGreen}{双模}” $\textcolor{PineGreen}{\pm}$ 版
\begin{subequations} \label{eq:DP^(2)-pm_pmpm}
\begin{align}
	\Xint{\mathcolor{gray}{-}}{30}{\bar{P}}^{\;\! \mathcolor{gray}{\omega} \textcolor{PineGreen}{\pm}}_{\;\! \mathcolor{gray}{z} \textcolor{Maroon}{(2)}} &= \Xint{{}^{}_{\mathcolor{gray}{-}}}{23}{\bar{\bar{\bar{\chi}}}}^{\;\! \mathcolor{gray}{\omega} \textcolor{PineGreen}{\pm \pm \pm}}_{\mathcolor{gray}{z} \textcolor{Maroon}{(2)}} ~{}^{\mathcolor{gray}{*}}_{\mathcolor{gray}{*}} \left( \Xint{\mathcolor{gray}{-}}{295}{\bar{E}}^{\;\!\mathcolor{gray}{\omega}}_{\;\! \mathcolor{gray}{z} \textcolor{PineGreen}{\pm} } ~\mathcolor{gray}{\widetilde \circledast}~ \Xint{\mathcolor{gray}{-}}{295}{\bar{E}}^{\;\!\mathcolor{gray}{\omega}}_{\;\! \mathcolor{gray}{z} \textcolor{PineGreen}{\pm} } \right) \label{eq:vec-DP^(2)-pm_pmpm} \\
	\Xint{\mathcolor{gray}{-}}{30}{P}^{\;\! \mathcolor{gray}{\omega} \textcolor{PineGreen}{\pm}}_{\;\! \hat{3}\mathcolor{gray}{z} \textcolor{Maroon}{(2)}} &= \Xint{{}^{}_{\mathcolor{gray}{-}}}{23}{\chi}^{\;\! \mathcolor{gray}{\omega} \hat{1} \hat{2} \textcolor{PineGreen}{\pm}}_{\;\! \hat{3} \mathcolor{gray}{z} \textcolor{Maroon}{(2)} \textcolor{PineGreen}{\pm \pm}} \mathcolor{gray}{*} \left( \Xint{\mathcolor{gray}{-}}{295}{E}^{\;\!\mathcolor{gray}{\omega} \textcolor{PineGreen}{\pm}}_{\;\! \hat{1} \mathcolor{gray}{z}} ~\mathcolor{gray}{\widetilde \circledast}~ \Xint{\mathcolor{gray}{-}}{295}{E}^{\;\!\mathcolor{gray}{\omega} \textcolor{PineGreen}{\pm}}_{\;\! \hat{2} \mathcolor{gray}{z}} \right) ~. \label{eq:components-DP^(2)-pm_pmpm}
\end{align}
\end{subequations}
上式 \bref{eq:DP^(2)-pm_pmpm} 也可替换为用抽象\textcolor{Plum}{符号} $\textcolor{PineGreen}{\hat{3}},\textcolor{PineGreen}{\hat{2}},\textcolor{PineGreen}{\hat{1}} = \textcolor{PineGreen}{\pm},\textcolor{PineGreen}{\pm},\textcolor{PineGreen}{\pm}$ 来代表具体\textcolor{PineGreen}{本征模}的形式
\begin{subequations} \label{eq:DP^(2)-3_12-spectrum}
\begin{align}
	\Xint{\mathcolor{gray}{-}}{30}{\bar{P}}^{\;\! \mathcolor{gray}{\omega} \textcolor{Maroon}{(2)} }_{\;\! \mathcolor{gray}{z} \textcolor{PineGreen}{\hat{3}}} &= \Xint{{}^{}_{\mathcolor{gray}{-}}}{23}{\bar{\bar{\bar{\chi}}}}^{\;\! \mathcolor{gray}{\omega} \textcolor{Maroon}{(2)}}_{\mathcolor{gray}{z} \textcolor{PineGreen}{\hat{3} \hat{1} \hat{2}} } ~{}^{\mathcolor{gray}{*}}_{\mathcolor{gray}{*}} \left( \Xint{\mathcolor{gray}{-}}{295}{\bar{E}}^{\;\! \mathcolor{gray}{\omega} \textcolor{PineGreen}{\hat{1}} }_{\;\! \mathcolor{gray}{z} } ~\mathcolor{gray}{\widetilde \circledast}~ \Xint{\mathcolor{gray}{-}}{295}{\bar{E}}^{\;\! \mathcolor{gray}{\omega} \textcolor{PineGreen}{\hat{2}} }_{\;\! \mathcolor{gray}{z} } \right) \label{eq:vec-DP^(2)-3_12-spectrum} \\
	\Xint{\mathcolor{gray}{-}}{30}{P}^{\;\! \textcolor{PineGreen}{\hat{3}} \mathcolor{gray}{\omega} }_{\;\! \hat{3}\mathcolor{gray}{z} \textcolor{Maroon}{(2)} } &= \Xint{{}^{}_{\mathcolor{gray}{-}}}{23}{\chi}^{\;\! \textcolor{PineGreen}{\hat{3}} \mathcolor{gray}{\omega} \hat{1} \hat{2} }_{\;\! \hat{3} \mathcolor{gray}{z} \textcolor{PineGreen}{\hat{1} \hat{2}} \textcolor{Maroon}{(2)}} \mathcolor{gray}{*} \left( \Xint{\mathcolor{gray}{-}}{295}{E}^{\;\! \textcolor{PineGreen}{\hat{1}} \mathcolor{gray}{\omega} }_{\;\! \hat{1} \mathcolor{gray}{z}} ~\mathcolor{gray}{\widetilde \circledast}~ \Xint{\mathcolor{gray}{-}}{295}{E}^{\;\! \textcolor{PineGreen}{\hat{2}} \mathcolor{gray}{\omega} }_{\;\! \hat{2} \mathcolor{gray}{z}} \right) ~. \label{eq:components-DP^(2)-3_12-spectrum}
\end{align}
\end{subequations}
同时,矢量\textcolor{Plum}{非线性}波动方程 \bref{eq:simplify8-scalar-g-conjugate} 也简写作
\begin{align} \label{eq:simplify8-scalar-g-modulus}
	\mathcolor{gray}{\nabla_z} \Xint{\begin{smallmatrix} ~ \\ {}^{}_{\mathcolor{gray}{-}} \\ ~ \end{smallmatrix}}{09}{\mathtt{g}}^{\;\!\mathcolor{gray}{\omega} \textcolor{PineGreen}{\hat{3}}}_{\;\! \mathcolor{gray}{z}} &= \mathbb{i} k_{\textcolor{Maroon}{\mathsf{o}} \mathcolor{gray}{\omega}}^{\;\! 2} \frac{\Xint{{}^{}_{\mathcolor{gray}{-}}}{10}{\hat{g}}^{\;\! \textcolor{PineGreen}{\hat{3}} \textcolor{Plum}{\dag}}_{\;\! \mathcolor{gray}{\omega}} \cdot \Xint{\mathcolor{gray}{-}}{25}{\bar{P}}^{\;\! \mathcolor{gray}{\omega} \textcolor{PineGreen}{\hat{3}} }_{\;\! \mathcolor{gray}{z}  \textcolor{Maroon}{(2)}}}{ 2 \lvert \Xint{{}^{}_{\mathcolor{gray}{-}}}{10}{\hat{g}}^{\;\! \textcolor{PineGreen}{\hat{3}}}_{\;\! \mathcolor{gray}{\omega}} \rvert^2 \Xint{\begin{smallmatrix} ~ \\ {}^{}_{\mathcolor{gray}{-}} \\ ~ \end{smallmatrix}}{15}{k}_{\;\! \symup{z}}^{\;\! \mathcolor{gray}{\omega} \textcolor{PineGreen}{\hat{3}}} \mathbb{e}^{\mathbb{i} \Xint{\begin{smallmatrix} ~ \\ {}^{}_{\mathcolor{gray}{-}} \\ ~ \end{smallmatrix}}{15}{k}_{\symup{z}}^{\;\! \mathcolor{gray}{\omega} \textcolor{PineGreen}{\hat{3}}} \mathcolor{gray}{z}}} ~, 
\end{align}

二阶\textcolor{Plum}{非线性}系数 $\Xint{{}^{}_{\mathcolor{gray}{-}}}{23}{\chi}^{\;\! \mathcolor{gray}{\omega} \hat{1} \hat{2} }_{\;\! \hat{3} \mathcolor{gray}{z} \textcolor{Maroon}{(2)} }$ 总可分解为\textcolor{Plum}{均匀背景} ${\chi}^{\;\! \mathcolor{gray}{\omega} \hat{1} \hat{2} }_{\;\! \hat{3} \textcolor{Maroon}{(2)} }$ 与\textcolor{Plum}{调制函数} $\Xint{\mathcolor{gray}{-}}{18}{M}^{\;\! \mathcolor{gray}{\omega} \hat{1} \hat{2} }_{\;\! \hat{3} \mathcolor{gray}{z} \textcolor{Maroon}{(2)} }$ 之积
%\Footnote{当不存在“\textcolor{PineGreen}{模式}”\textcolor{Plum}{角标},以\textcolor{Plum}{推断}\textcolor{NavyBlue}{场量}的\textcolor{Plum}{自变量}时,不能省略 $\mathcolor{gray}{\omega}$ \textcolor{Plum}{角标}。}
\begin{subequations} \label{eq:chi2-modulate}
\begin{align}
	\Xint{{}^{}_{\mathcolor{gray}{-}}}{23}{\bar{\bar{\bar{\chi}}}}^{\;\! \mathcolor{gray}{\omega} }_{\;\! \mathcolor{gray}{z} \textcolor{Maroon}{(2)} } &= \bar{\bar{\bar{\chi}}}^{\;\! \mathcolor{gray}{\omega} }_{\;\! \textcolor{Maroon}{(2)} } \odot \Xint{\mathcolor{gray}{-}}{18}{\bar{\bar{\bar{M}}}}^{\;\! \mathcolor{gray}{\omega} }_{\;\! \mathcolor{gray}{z} \textcolor{Maroon}{(2)} } \label{eq:vec-eq:chi2-modulate} \\
	\Xint{{}^{}_{\mathcolor{gray}{-}}}{23}{\chi}^{\;\! \mathcolor{gray}{\omega} \hat{1} \hat{2} }_{\;\! \hat{3} \mathcolor{gray}{z} \textcolor{Maroon}{(2)} } &= {\chi}^{\;\! \mathcolor{gray}{\omega} \hat{1} \hat{2} }_{\;\! \hat{3} \textcolor{Maroon}{(2)} } \Xint{\mathcolor{gray}{-}}{18}{M}^{\;\! \mathcolor{gray}{\omega} \hat{1} \hat{2} }_{\;\! \hat{3} \mathcolor{gray}{z} \textcolor{Maroon}{(2)} } ~, \label{eq:components-eq:chi2-modulate}
\end{align}
\end{subequations}
这里隐式地定义了\textcolor{Plum}{哈达马积}/\textcolor{Plum}{对应元素积} $\odot$ 或 $"{.\cdot}"$,类似于 matlab 的 $"{.*}"$ 语法(矩阵对应元素相乘)。注意,三阶\textcolor{Plum}{调制张量}\textcolor{NavyBlue}{场} $\Xint{\mathcolor{gray}{-}}{18}{\bar{\bar{\bar{M}}}}^{\;\! \mathcolor{gray}{\omega} }_{\;\! \mathcolor{gray}{z} \textcolor{Maroon}{(2)} }$ 像 ${\chi}^{\;\! \mathcolor{gray}{\omega} }_{\;\! \textcolor{Maroon}{(2)} }$(\textcolor{NavyBlue}{非场},没有\textcolor{gray}{“$-$”标志})一样,仍是\textcolor{gray}{波长} $\mathcolor{gray}{\lambda}$ 的函数,即仍是 $\mathcolor{gray}{\omega}$ \textcolor{NavyBlue}{色散}(\textcolor{Plum}{各向异性})的。但分离出的\textcolor{Plum}{定常}\textcolor{Plum}{均匀背景}张量 ${\chi}^{\;\! \mathcolor{gray}{\omega} }_{\;\! \textcolor{Maroon}{(2)} }$ 因子,不再是 $\mathcolor{gray}{\bar{r}}$ 的函数,并且可以从许多\textcolor{NavyBlue}{实验主导}的文献中获得\cite{nyePhysicalPropertiesCrystals2012,zuOpticalSecondHarmonic2024,zuAnalyticalNumericalModeling2022,gananyQuasiphaseMatchingLiNbO32006,segondsLinearNonlinearOptical2004,dolevLinearNonlinearOptical2009,kaschkeCalculationNonlinearOptical1989,itoGeneralizedStudyAngular1975}。注意,二阶\textcolor{Plum}{非线性}系数 $\Xint{{}^{}_{\mathcolor{gray}{-}}}{23}{\bar{\bar{\bar{\chi}}}}^{\;\! \mathcolor{gray}{\omega} }_{\;\! \mathcolor{gray}{z} \textcolor{Maroon}{(2)} }$ 不是\textcolor{PineGreen}{本征模} $\textcolor{PineGreen}{\hat{3}},\textcolor{PineGreen}{\hat{2}},\textcolor{PineGreen}{\hat{1}} = \textcolor{PineGreen}{\pm},\textcolor{PineGreen}{\pm},\textcolor{PineGreen}{\pm}$ 的函数。

将 $\mathcolor{gray}{\bar{r}}$ 域上\textcolor{Plum}{被调制}的二阶\textcolor{Plum}{非线性}系数 \bref{eq:components-eq:chi2-modulate} 代入\textcolor{Plum}{非线性}\textcolor{NavyBlue}{波源} \bref{eq:components-DP^(2)-3_12-spectrum} 得
\begin{subequations} \label{eq:DP^(2)-3_12-spectrum-C}
\begin{align}
	\Xint{\mathcolor{gray}{-}}{30}{P}^{\;\! \textcolor{PineGreen}{\hat{3}} \mathcolor{gray}{\omega} }_{\;\! \hat{3}\mathcolor{gray}{z} \textcolor{Maroon}{(2)} } &= \Xint{{}^{}_{\mathcolor{gray}{-}}}{23}{\chi}^{\;\! \textcolor{PineGreen}{\hat{3}} \mathcolor{gray}{\omega} \hat{1} \hat{2} }_{\;\! \hat{3} \mathcolor{gray}{z} \textcolor{PineGreen}{\hat{1} \hat{2}} \textcolor{Maroon}{(2)}} \mathcolor{gray}{*} \left( \Xint{\mathcolor{gray}{-}}{295}{E}^{\;\!\textcolor{PineGreen}{\hat{1}} \mathcolor{gray}{\omega}}_{\;\! \hat{1} \mathcolor{gray}{z}} ~\mathcolor{gray}{\widetilde \circledast}~ \Xint{\mathcolor{gray}{-}}{295}{E}^{\;\!\textcolor{PineGreen}{\hat{2}} \mathcolor{gray}{\omega}}_{\;\! \hat{2} \mathcolor{gray}{z}} \right) \label{eq:DP^(2)-3_12-spectrum-C1} \\
	&= {\chi}^{\;\! \textcolor{PineGreen}{\hat{3}} \mathcolor{gray}{\omega} \hat{1} \hat{2} }_{\;\! \hat{3} \textcolor{Maroon}{(2)} \textcolor{PineGreen}{\hat{1} \hat{2}}} \Xint{\mathcolor{gray}{-}}{18}{M}^{\;\! \mathcolor{gray}{\omega} \hat{1} \hat{2} }_{\;\! \hat{3} \mathcolor{gray}{z} \textcolor{Maroon}{(2)} } \mathcolor{gray}{*} \left( \Xint{\mathcolor{gray}{-}}{295}{E}^{\;\!\textcolor{PineGreen}{\hat{1}} \mathcolor{gray}{\omega}}_{\;\! \hat{1} \mathcolor{gray}{z}} ~\mathcolor{gray}{\widetilde \circledast}~ \Xint{\mathcolor{gray}{-}}{295}{E}^{\;\!\textcolor{PineGreen}{\hat{2}} \mathcolor{gray}{\omega}}_{\;\! \hat{2} \mathcolor{gray}{z}} \right) \label{eq:DP^(2)-3_12-spectrum-C2} \\
	&= {\chi}^{\;\! \textcolor{PineGreen}{\hat{3}} \mathcolor{gray}{\omega} \hat{1} \hat{2} }_{\;\! \hat{3} \textcolor{Maroon}{(2)} \textcolor{PineGreen}{\hat{1} \hat{2}}} \mathcolor{gray}{\mathcal F} \left[ M^{\;\! \mathcolor{gray}{\omega} \hat{1} \hat{2} }_{\;\! \hat{3} \mathcolor{gray}{z} \textcolor{Maroon}{(2)} } \right] \mathcolor{gray}{*} \left( \Xint{\mathcolor{gray}{-}}{295}{E}^{\;\!\textcolor{PineGreen}{\hat{1}} \mathcolor{gray}{\omega}}_{\;\! \hat{1} \mathcolor{gray}{z}} ~\mathcolor{gray}{\widetilde \circledast}~ \Xint{\mathcolor{gray}{-}}{295}{E}^{\;\!\textcolor{PineGreen}{\hat{2}} \mathcolor{gray}{\omega}}_{\;\! \hat{2} \mathcolor{gray}{z}} \right) \label{eq:DP^(2)-3_12-spectrum-C3} \\
	&= {\chi}^{\;\! \textcolor{PineGreen}{\hat{3}} \mathcolor{gray}{\omega} \hat{1} \hat{2} }_{\;\! \hat{3} \textcolor{Maroon}{(2)} \textcolor{PineGreen}{\hat{1} \hat{2}}} \mathcolor{gray}{\mathcal F_{z}^{-1}} \left[ \mathcolor{gray}{\mathcal F_{\bar{k}}} \left[ M^{\;\! \mathcolor{gray}{\omega} \hat{1} \hat{2} }_{\;\! \hat{3} \mathcolor{gray}{z} \textcolor{Maroon}{(2)} } \right] \right] \mathcolor{gray}{*} \left( \Xint{\mathcolor{gray}{-}}{295}{E}^{\;\!\textcolor{PineGreen}{\hat{1}} \mathcolor{gray}{\omega}}_{\;\! \hat{1} \mathcolor{gray}{z}} ~\mathcolor{gray}{\widetilde \circledast}~ \Xint{\mathcolor{gray}{-}}{295}{E}^{\;\!\textcolor{PineGreen}{\hat{2}} \mathcolor{gray}{\omega}}_{\;\! \hat{2} \mathcolor{gray}{z}} \right) \label{eq:DP^(2)-3_12-spectrum-C4} \\
	&= {\chi}^{\;\! \textcolor{PineGreen}{\hat{3}} \mathcolor{gray}{\omega} \hat{1} \hat{2} }_{\;\! \hat{3} \textcolor{Maroon}{(2)} \textcolor{PineGreen}{\hat{1} \hat{2}}} \mathcolor{gray}{\mathcal F_{z}^{-1}} \left[ \mathcolor{gray}{\mathcal F_{\bar{k}}} \left[ M^{\;\! \mathcolor{gray}{\omega} \hat{1} \hat{2} }_{\;\! \hat{3} \mathcolor{gray}{z} \textcolor{Maroon}{(2)} } \right] \mathcolor{gray}{*} \left( \Xint{\mathcolor{gray}{-}}{295}{E}^{\;\!\textcolor{PineGreen}{\hat{1}} \mathcolor{gray}{\omega}}_{\;\! \hat{1} \mathcolor{gray}{z}} ~\mathcolor{gray}{\widetilde \circledast}~ \Xint{\mathcolor{gray}{-}}{295}{E}^{\;\!\textcolor{PineGreen}{\hat{2}} \mathcolor{gray}{\omega}}_{\;\! \hat{2} \mathcolor{gray}{z}} \right) \right] \label{eq:DP^(2)-3_12-spectrum-C5} \\
	&=: {\chi}^{\;\! \textcolor{PineGreen}{\hat{3}} \mathcolor{gray}{\omega} \hat{1} \hat{2} }_{\;\! \hat{3} \textcolor{Maroon}{(2)} \textcolor{PineGreen}{\hat{1} \hat{2}}} \mathcolor{gray}{\mathcal F_{z}^{-1}} \left[ \Xint{\mathcolor{gray}{-}}{18}{M}^{\;\! \mathcolor{gray}{\omega} \hat{1} \hat{2} }_{\;\! \hat{3} \mathcolor{gray}{k_{\symup{z}}} \textcolor{Maroon}{(2)} } \mathcolor{gray}{*} \left( \Xint{\mathcolor{gray}{-}}{295}{E}^{\;\!\textcolor{PineGreen}{\hat{1}} \mathcolor{gray}{\omega}}_{\;\! \hat{1} \mathcolor{gray}{z}} ~\mathcolor{gray}{\widetilde \circledast}~ \Xint{\mathcolor{gray}{-}}{295}{E}^{\;\!\textcolor{PineGreen}{\hat{2}} \mathcolor{gray}{\omega}}_{\;\! \hat{2} \mathcolor{gray}{z}} \right) \right] ~, \label{eq:DP^(2)-3_12-spectrum-C6}
\end{align}
\end{subequations}
其中,$\Xint{{}^{}_{\mathcolor{gray}{-}}}{23}{\chi}^{\;\! \textcolor{PineGreen}{\hat{3}} \mathcolor{gray}{\omega} \hat{1} \hat{2} }_{\;\! \hat{3} \mathcolor{gray}{z} \textcolor{PineGreen}{\hat{1} \hat{2}} \textcolor{Maroon}{(2)}}$ 的值,与\textcolor{PineGreen}{本征模} $\textcolor{PineGreen}{\hat{3}},\textcolor{PineGreen}{\hat{2}},\textcolor{PineGreen}{\hat{1}} = \textcolor{PineGreen}{\pm},\textcolor{PineGreen}{\pm},\textcolor{PineGreen}{\pm}$ 无关。$\textcolor{PineGreen}{\hat{3}},\textcolor{PineGreen}{\hat{2}},\textcolor{PineGreen}{\hat{1}}$ 在其中,只是帮助 $\Xint{{}^{}_{\mathcolor{gray}{-}}}{23}{\chi}^{\;\! \textcolor{PineGreen}{\hat{3}} \mathcolor{gray}{\omega} \hat{1} \hat{2} }_{\;\! \hat{3} \mathcolor{gray}{z} \textcolor{PineGreen}{\hat{1} \hat{2}} \textcolor{Maroon}{(2)}}$ 与 $\Xint{\mathcolor{gray}{-}}{25}{E}^{\;\!\textcolor{PineGreen}{\hat{1}} \mathcolor{gray}{\omega}}_{\;\! \hat{1} \mathcolor{gray}{z}}, \Xint{\mathcolor{gray}{-}}{25}{E}^{\;\!\textcolor{PineGreen}{\hat{2}} \mathcolor{gray}{\omega}}_{\;\! \hat{2} \mathcolor{gray}{z}}$ 一起,起到\textcolor{Plum}{爱因斯坦求和}的作用。此外,定义了三维 $\mathcolor{gray}{\bar{k}}$ \textcolor{gray}{空间}的\textcolor{Maroon}{倒格波系数}(关于 $\mathcolor{gray}{\bar{k}} \asymp \left( \mathcolor{gray}{\bar{k}_{\symup{\rho}}}, \mathcolor{gray}{k_{\symup{z}}} \right)$ 的三阶\textcolor{Plum}{张量}\textcolor{NavyBlue}{场})
\begin{subequations} \label{eq:C}
\begin{align}
	\Xint{\mathcolor{gray}{-}}{18}{M}^{\;\! \mathcolor{gray}{\omega} \hat{1} \hat{2} }_{\;\! \hat{3} \mathcolor{gray}{k_{\symup{z}}} \textcolor{Maroon}{(2)} } &:= \mathcolor{gray}{\mathcal F_{\bar{k}}} \left[ M^{\;\! \mathcolor{gray}{\omega} \hat{1} \hat{2} }_{\;\! \hat{3} \mathcolor{gray}{z} \textcolor{Maroon}{(2)} } \right] \label{eq:vec-C} \\
	\Xint{\mathcolor{gray}{-}}{18}{\bar{\bar{\bar{M}}}}^{\;\! \mathcolor{gray}{\omega} }_{\;\! \mathcolor{gray}{k_{\symup{z}}} \textcolor{Maroon}{(2)} } &:= \mathcolor{gray}{\mathcal F_{\bar{k}}} \left[ \bar{\bar{\bar{M}}}^{\;\! \mathcolor{gray}{\omega} }_{\;\! \mathcolor{gray}{z} \textcolor{Maroon}{(2)} } \right] ~, \label{eq:components-C}
\end{align}
\end{subequations}
其中,三维空域 $\mathcolor{gray}{\bar{r}} \in \mathcolor{gray}{\bar{\mathbb{R}}_{\textcolor{Plum}{3}}}$ 中的\textcolor{Plum}{傅立叶正变换} $\mathcolor{gray}{\mathcal F_{\bar{k}}}$ 来自 \bref{eq:FT-k}。

利用\textcolor{PineGreen}{本征波}的第 4 种定义 \bref{eq:vec-eigenwave'} 的\textcolor{PineGreen}{本征偏振态} $\Xint{{}^{}_{\mathcolor{gray}{-}}}{10}{\bar{g}}^{\;\! \mathcolor{gray}{\omega} \textcolor{PineGreen}{\pm}}$ \textcolor{Plum}{复归一化}版 $\Xint{\mathcolor{gray}{-}}{25}{\bar{E}}^{\;\! \mathcolor{gray}{\omega} \textcolor{PineGreen}{\pm}}_{\;\! \mathcolor{gray}{z}} := \Xint{\mathcolor{gray}{-}}{16}{\mathtt{G}}^{\;\! \mathcolor{gray}{\omega} \textcolor{PineGreen}{\pm}}_{\;\! \mathcolor{gray}{z}} \Xint{{}^{}_{\mathcolor{gray}{-}}}{10}{\hat{g}}^{\;\! \mathcolor{gray}{\omega} \textcolor{PineGreen}{\pm} }$,将 \bref{eq:DP^(2)-3_12-spectrum-C6} 分离出\textcolor{PineGreen}{含衍射本征复振幅} $\Xint{\mathcolor{gray}{-}}{16}{\mathtt{G}}^{\;\!\mathcolor{gray}{\omega} \textcolor{PineGreen}{\pm}}_{\;\! \mathcolor{gray}{z}}$(\bref{eq:amp_phase})和\textcolor{Plum}{复归一化}\textcolor{PineGreen}{本征偏振态} $\Xint{{}^{}_{\mathcolor{gray}{-}}}{10}{\hat{g}}^{\;\!\mathcolor{gray}{\omega}}_{\;\! \textcolor{PineGreen}{\pm}}$,并将 \bref{eq:DP^(2)-3_12-spectrum-C6}(的系数张量 $\Xint{\mathcolor{gray}{-}}{18}{M}^{\;\! \mathcolor{gray}{\omega} \hat{1} \hat{2} }_{\;\! \hat{3} \mathcolor{gray}{k_{\symup{z}}} \textcolor{Maroon}{(2)} }$)升级为“\textcolor{Plum}{半张量式}”$\Xint{\mathcolor{gray}{-}}{18}{\bar{M}}^{\;\! \mathcolor{gray}{\omega} \hat{1} \hat{2} }_{\;\! \mathcolor{gray}{k_{\symup{z}}} \textcolor{Maroon}{(2)} }$,且预备将\textcolor{Plum}{替换后的}矢量\textcolor{Plum}{非线性}\textcolor{NavyBlue}{波源}项 $\Xint{\mathcolor{gray}{-}}{30}{\bar{P}}^{\;\! \mathcolor{gray}{\omega} \textcolor{PineGreen}{\hat{3}} }_{\;\! \mathcolor{gray}{z} \textcolor{Maroon}{(2)} }$ 整体,代入 \bref{eq:simplify8-scalar-g-modulus}\Footnote{注意,{\one} $\Xint{{}^{}_{\mathcolor{gray}{-}}}{10}{\hat{g}}^{\;\! \mathcolor{gray}{\omega} \textcolor{PineGreen}{\hat{1}}}_{\;\! \hat{1}}$ 是矢量 $\Xint{{}^{}_{\mathcolor{gray}{-}}}{10}{\hat{g}}^{\;\! \mathcolor{gray}{\omega} }_{\;\! \textcolor{PineGreen}{\hat{1}}}$ 在 $\hat{1}$ 方向的分量,即标量;{\two} 对 $\mathcolor{gray}{\widetilde \circledast}$ 的运算\textcolor{Plum}{优先级},整体来说,既不高于,也不低于,也不等于对 $\mathcolor{gray}{*}$ 的\textcolor{Plum}{优先级}:需要拆分后,才能谈\textcolor{Plum}{优先级},见下文 \bref{ssec:scalar} 中 \bref{eq:scalar_nonlinear_drive2} 的下一段。}
\begin{subequations} \label{eq:DP^(2)-3_12-spectrum-G}
\begin{align}
	\Xint{\mathcolor{gray}{-}}{30}{\bar{P}}^{\;\! \mathcolor{gray}{\omega} \textcolor{PineGreen}{\hat{3}} }_{\;\! \mathcolor{gray}{z} \textcolor{Maroon}{(2)} } &= \bar{\chi}^{\;\! \mathcolor{gray}{\omega} \textcolor{PineGreen}{\hat{3}} \hat{1} \hat{2} }_{\;\! \textcolor{Maroon}{(2)} \textcolor{PineGreen}{\hat{1} \hat{2}}} \odot \mathcolor{gray}{\mathcal F_{z}^{-1}} \left[ \Xint{\mathcolor{gray}{-}}{18}{\bar{M}}^{\;\! \mathcolor{gray}{\omega} \hat{1} \hat{2} }_{\;\! \mathcolor{gray}{k_{\symup{z}}} \textcolor{Maroon}{(2)} } \mathcolor{gray}{*} \Xint{\mathcolor{gray}{-}}{295}{E}^{\;\! \mathcolor{gray}{\omega} \textcolor{PineGreen}{\hat{1}}}_{\;\! \hat{1} \mathcolor{gray}{z}} ~\mathcolor{gray}{\widetilde \circledast}~ \Xint{\mathcolor{gray}{-}}{295}{E}^{\;\! \mathcolor{gray}{\omega} \textcolor{PineGreen}{\hat{2}}}_{\;\! \hat{2} \mathcolor{gray}{z}} \right] \label{eq:DP^(2)-3_12-spectrum-G1} \\
	&= \bar{\chi}^{\;\! \mathcolor{gray}{\omega} \textcolor{PineGreen}{\hat{3}} \hat{1} \hat{2} }_{\;\! \textcolor{Maroon}{(2)} \textcolor{PineGreen}{\hat{1} \hat{2}}} \odot \mathcolor{gray}{\mathcal F_{z}^{-1}} \left[ \Xint{\mathcolor{gray}{-}}{18}{\bar{M}}^{\;\! \mathcolor{gray}{\omega} \hat{1} \hat{2} }_{\;\! \mathcolor{gray}{k_{\symup{z}}} \textcolor{Maroon}{(2)} } \mathcolor{gray}{*} \left( \Xint{\mathcolor{gray}{-}}{20}{\mathtt{G}}^{\;\! \mathcolor{gray}{\omega} \textcolor{PineGreen}{\hat{1}}}_{\;\! \mathcolor{gray}{z}} \Xint{{}^{}_{\mathcolor{gray}{-}}}{10}{\hat{g}}^{\;\! \mathcolor{gray}{\omega} \textcolor{PineGreen}{\hat{1}}}_{\;\! \hat{1}} \right) \mathcolor{gray}{\widetilde \circledast} \left( \Xint{\mathcolor{gray}{-}}{20}{\mathtt{G}}^{\;\! \mathcolor{gray}{\omega} \textcolor{PineGreen}{\hat{2}}}_{\;\! \mathcolor{gray}{z}} \Xint{{}^{}_{\mathcolor{gray}{-}}}{10}{\hat{g}}^{\;\! \mathcolor{gray}{\omega} \textcolor{PineGreen}{\hat{2}}}_{\;\! \hat{2}} \right) \right] ~, \label{eq:DP^(2)-3_12-spectrum-G2}
\end{align}
\end{subequations}
这个版本的\textcolor{Plum}{非线性}\textcolor{NavyBlue}{波源},保留了 \bref{eq:vec-DP^(2)-pm_pmpm} 左侧的 $\Xint{\mathcolor{gray}{-}}{25}{\bar{P}}^{\;\! \mathcolor{gray}{\omega} \textcolor{PineGreen}{\pm}}_{\;\! \textcolor{Maroon}{\Yup} \mathcolor{gray}{z}}$ 的矢量形式,以便直接代入 \bref{eq:simplify8-scalar-g-modulus};又丢弃了 \bref{eq:vec-DP^(2)-pm_pmpm} 右侧的 ${}^{\mathcolor{gray}{*}}_{\mathcolor{gray}{*}}$,并采用了 \bref{eq:components-DP^(2)-pm_pmpm} 右侧的 $\mathcolor{gray}{*}$;--- 这样做的代价便是 \bref{eq:DP^(2)-3_12-spectrum-G} 中的二阶\textcolor{Plum}{非线性}系数张量,既从完整的三阶张量 $\Xint{{}^{}_{\mathcolor{gray}{-}}}{23}{\bar{\bar{\bar{\chi}}}}^{\;\! \mathcolor{gray}{\omega} }_{\;\! \mathcolor{gray}{z} \textcolor{Maroon}{(2)} }$ 降阶(\bref{eq:vec-eq:chi2-modulate}),又从零阶张量元 $\Xint{{}^{}_{\mathcolor{gray}{-}}}{23}{\chi}^{\;\! \mathcolor{gray}{\omega} \hat{1} \hat{2} }_{\;\! \hat{3} \mathcolor{gray}{z} \textcolor{Maroon}{(2)} }$ 升阶(\bref{eq:components-eq:chi2-modulate}),至同阶于 \bref{eq:DP^(2)-3_12-spectrum-G} 左侧 $\Xint{\mathcolor{gray}{-}}{25}{\bar{P}}^{\;\! \mathcolor{gray}{\omega} \textcolor{PineGreen}{\pm}}_{\;\! \textcolor{Maroon}{\Yup} \mathcolor{gray}{z}}$ 的一阶张量 $\Xint{{}^{}_{\mathcolor{gray}{-}}}{23}{\bar{\chi}}^{\;\! \mathcolor{gray}{\omega} \hat{1} \hat{2} }_{\;\! \mathcolor{gray}{z} \textcolor{Maroon}{(2)} }$ 的矢量形式。--- 这便体现了 \bref{hook:0bar,hook:1bar,hook:2bar,hook:3bar} 中对“划上线 line up”制度的前置顶层设计,以表达不同阶张量的作用和优势。

将 \bref{eq:DP^(2)-3_12-spectrum-G1} 代入 \bref{eq:simplify8-scalar-g-modulus},即得以\textcolor{NavyBlue}{脉冲光}\textcolor{Maroon}{倍频}\cite{boydNonlinearOptics2019}、\textcolor{NavyBlue}{脉冲}\textcolor{Maroon}{光整流}后的级联\textcolor{Maroon}{电光效应}\cite{jangMulticycleTerahertzPulse2020}等过程为代表的电场\textcolor{PineGreen}{本征复振幅}方程
\begin{align} \label{eq:simplify8-scalar-g-modulus-P-spectrum}
	\mathcolor{gray}{\nabla_z} \Xint{\begin{smallmatrix} ~ \\ {}^{}_{\mathcolor{gray}{-}} \\ ~ \end{smallmatrix}}{09}{\mathtt{g}}^{\;\!\mathcolor{gray}{\omega} \textcolor{PineGreen}{\hat{3}}}_{\;\! \mathcolor{gray}{z}} &= \mathbb{i} k_{\textcolor{Maroon}{\mathsf{o}} \mathcolor{gray}{\omega}}^{\;\! 2} \frac{\Xint{{}^{}_{\mathcolor{gray}{-}}}{10}{\hat{g}}^{\;\! \textcolor{PineGreen}{\hat{3}} \textcolor{Plum}{\dag}}_{\;\! \mathcolor{gray}{\omega}} \cdot \bar{\chi}^{\;\! \mathcolor{gray}{\omega} \textcolor{PineGreen}{\hat{3}} \hat{1} \hat{2} }_{\;\! \textcolor{Maroon}{(2)} \textcolor{PineGreen}{\hat{1} \hat{2}}} \odot \mathcolor{gray}{\mathcal F_{z}^{-1}} \left[ \Xint{\mathcolor{gray}{-}}{18}{\bar{M}}^{\;\! \mathcolor{gray}{\omega} \hat{1} \hat{2} }_{\;\! \mathcolor{gray}{k_{\symup{z}}} \textcolor{Maroon}{(2)} } \mathcolor{gray}{*} \Xint{\mathcolor{gray}{-}}{25}{E}^{\;\! \mathcolor{gray}{\omega} \textcolor{PineGreen}{\hat{1}}}_{\;\! \hat{1} \mathcolor{gray}{z}} ~\mathcolor{gray}{\widetilde \circledast}~ \Xint{\mathcolor{gray}{-}}{25}{E}^{\;\! \mathcolor{gray}{\omega} \textcolor{PineGreen}{\hat{2}}}_{\;\! \hat{2} \mathcolor{gray}{z}} \right]}{ 2 \lvert \Xint{{}^{}_{\mathcolor{gray}{-}}}{10}{\hat{g}}^{\;\! \textcolor{PineGreen}{\hat{3}}}_{\;\! \mathcolor{gray}{\omega}} \rvert^2 \Xint{\begin{smallmatrix} ~ \\ {}^{}_{\mathcolor{gray}{-}} \\ ~ \end{smallmatrix}}{15}{k}_{\;\! \symup{z}}^{\;\! \mathcolor{gray}{\omega} \textcolor{PineGreen}{\hat{3}}} \mathbb{e}^{\mathbb{i} \Xint{\begin{smallmatrix} ~ \\ {}^{}_{\mathcolor{gray}{-}} \\ ~ \end{smallmatrix}}{15}{k}_{\symup{z}}^{\;\! \mathcolor{gray}{\omega} \textcolor{PineGreen}{\hat{3}}} \mathcolor{gray}{z}}} ~, 
\end{align}
注意,\textcolor{Plum}{哈达马积} $\odot$ 的运算\textcolor{Plum}{优先级}恒高于\textcolor{Plum}{点积} $\cdot$(以省略一对小括号)。

\marginLeft[-2.4em]{ssec:SFG_discrete}\subsection{连续光和频 - 电场本征复振幅方程}\label{ssec:SFG_discrete}

对于\textcolor{NavyBlue}{非脉冲}/\textcolor{NavyBlue}{非连续谱},而是两个\textcolor{Plum}{独立}、\textcolor{Plum}{离散}、\textcolor{NavyBlue}{单色}\textcolor{gray}{波长}的\textcolor{Maroon}{和频}或\textcolor{Maroon}{上转换}\Footnote{尽管\textcolor{NavyBlue}{双泵浦}的\textcolor{NavyBlue}{光强}可能不大,这里仍不说“\textcolor{Maroon}{上转换}”:因为在本文的语境中,“\textcolor{Maroon}{上转换}”过程一般是“\textcolor{NavyBlue}{一强一弱}”\textcolor{NavyBlue}{双泵浦}生成\textcolor{NavyBlue}{弱} $\mathcolor{gray}{\omega}_{\textcolor{gray}{3}}$,以至于参与\textcolor{gray}{混频}的三波中,有两束弱光(一入一出)不满足\textcolor{Maroon}{泵浦未耗尽近似}条件,因此只要有“\textcolor{Maroon}{上转换}”则必有“\textcolor{Maroon}{下转换}”过程发生(\textcolor{NavyBlue}{能量}从 $\mathcolor{gray}{\omega}_{\textcolor{gray}{3}}$ 回流到其中一个\textcolor{NavyBlue}{弱泵浦}中),于是不可避免地涉及\textcolor{NavyBlue}{三波混频}\textcolor{Maroon}{时空谱}耦合波方程组中的至少 2 个方程,然而这里只给出了 1 个“\textcolor{Maroon}{上转换}”过程的方程,因此这里只能代表/指\textcolor{Maroon}{和频}过程。}出\textcolor{gray}{第三个波长}的 $\mathcolor{gray}{\omega}_{\textcolor{gray}{1}} + \mathcolor{gray}{\omega}_{\textcolor{gray}{2}} \to \mathcolor{gray}{\omega}_{\textcolor{gray}{3}}$ 过程,即纯\textcolor{NavyBlue}{(准)连续光}\textcolor{gray}{混频}的特例,\bref{eq:DP^(2)-pm_pmpm} 变为
\begin{subequations} \label{eq:DP^(2)-3_12}
	\begin{align}
		\Xint{\mathcolor{gray}{-}}{30}{\bar{P}}^{\;\! \textcolor{Maroon}{(2)} }_{\;\! \mathcolor{gray}{z} \textcolor{PineGreen}{\hat{3}}} &= \Xint{{}^{}_{\mathcolor{gray}{-}}}{23}{\bar{\bar{\bar{\chi}}}}^{\;\!  \textcolor{Maroon}{(2)}}_{\mathcolor{gray}{z} \textcolor{PineGreen}{\hat{3} \hat{1} \hat{2}} } ~{}^{\mathcolor{gray}{*}}_{\mathcolor{gray}{*}} \left( \Xint{\mathcolor{gray}{-}}{295}{\bar{E}}^{\;\! \textcolor{PineGreen}{\hat{1}} }_{\;\! \mathcolor{gray}{z} } \mathcolor{gray}{*} \Xint{\mathcolor{gray}{-}}{295}{\bar{E}}^{\;\! \textcolor{PineGreen}{\hat{2}} }_{\;\! \mathcolor{gray}{z} } \right) \label{eq:vec-DP^(2)-3_12} \\
		\Xint{\mathcolor{gray}{-}}{30}{P}^{\;\! \textcolor{PineGreen}{\hat{3}} \textcolor{Maroon}{(2)} }_{\;\! \hat{3}\mathcolor{gray}{z}} &= \Xint{{}^{}_{\mathcolor{gray}{-}}}{23}{\chi}^{\;\! \textcolor{PineGreen}{\hat{3}} \textcolor{Maroon}{(2)} \hat{1} \hat{2} }_{\;\! \hat{3} \mathcolor{gray}{z} \textcolor{PineGreen}{\hat{1} \hat{2}}} \mathcolor{gray}{*} \left( \Xint{\mathcolor{gray}{-}}{295}{E}^{\;\!\textcolor{PineGreen}{\hat{1}}}_{\;\! \hat{1} \mathcolor{gray}{z}} \mathcolor{gray}{*} \Xint{\mathcolor{gray}{-}}{295}{E}^{\;\!\textcolor{PineGreen}{\hat{2}}}_{\;\! \hat{2} \mathcolor{gray}{z}} \right) ~. \label{eq:components-DP^(2)-3_12}
	\end{align}
\end{subequations}
其中,每个场量都\textcolor{Plum}{未显含}\textcolor{gray}{角频率},但可以\textcolor{Plum}{推断}出来它们运行在 $\mathcolor{gray}{\omega}$ 域:因为\textcolor{PineGreen}{模式} $\textcolor{PineGreen}{\hat{3}},\textcolor{PineGreen}{\hat{2}},\textcolor{PineGreen}{\hat{1}} = \textcolor{PineGreen}{\pm},\textcolor{PineGreen}{\pm},\textcolor{PineGreen}{\pm}$ 只存在于 $\mathcolor{gray}{\omega}~ (, \mathcolor{gray}{\bar{k}_{\symup{\rho}}})$ 域,在时间 $\mathcolor{gray}{t}~ (, \mathcolor{gray}{\bar{k}_{\symup{\rho}}})$ 域内没有“\textcolor{PineGreen}{模式}”这一说法。这样表示,是在最大程度\textcolor{Plum}{省略符号}的同时\textcolor{Plum}{保留全信息}。

将 \bref{eq:components-eq:chi2-modulate} 的\textcolor{NavyBlue}{(准)连续光}/\textcolor{NavyBlue}{离散谱}版本,代入\textcolor{Plum}{非线性}\textcolor{NavyBlue}{波源} \bref{eq:components-DP^(2)-3_12} 得
\begin{subequations} \label{eq:DP^(2)-3_12-C}
\begin{align}
	\Xint{\mathcolor{gray}{-}}{30}{P}^{\;\! \textcolor{PineGreen}{\hat{3}} \textcolor{Maroon}{(2)} }_{\;\! \hat{3}\mathcolor{gray}{z}} &= \Xint{{}^{}_{\mathcolor{gray}{-}}}{23}{\chi}^{\;\! \textcolor{PineGreen}{\hat{3}} \textcolor{Maroon}{(2)} \hat{1} \hat{2} }_{\;\! \hat{3} \mathcolor{gray}{z} \textcolor{PineGreen}{\hat{1} \hat{2}}} \mathcolor{gray}{*} \left( \Xint{\mathcolor{gray}{-}}{295}{E}^{\;\! \textcolor{PineGreen}{\hat{1}}}_{\;\! \hat{1} \mathcolor{gray}{z}} \mathcolor{gray}{*} \Xint{\mathcolor{gray}{-}}{295}{E}^{\;\! \textcolor{PineGreen}{\hat{2}}}_{\;\! \hat{2} \mathcolor{gray}{z}} \right) \label{eq:DP^(2)-3_12-C1} \\
	&= {\chi}^{\;\! \textcolor{PineGreen}{\hat{3}} \hat{1} \hat{2} }_{\;\! \hat{3} \textcolor{PineGreen}{\hat{1} \hat{2}} \textcolor{Maroon}{(2)}} \Xint{\mathcolor{gray}{-}}{18}{M}^{\;\! \mathcolor{gray}{3} \hat{1} \hat{2} }_{\;\! \hat{3} \mathcolor{gray}{z} \textcolor{Maroon}{(2)} } \mathcolor{gray}{*} \left( \Xint{\mathcolor{gray}{-}}{295}{E}^{\;\! \textcolor{PineGreen}{\hat{1}}}_{\;\! \hat{1} \mathcolor{gray}{z}} \mathcolor{gray}{*} \Xint{\mathcolor{gray}{-}}{295}{E}^{\;\! \textcolor{PineGreen}{\hat{2}}}_{\;\! \hat{2} \mathcolor{gray}{z}} \right) \label{eq:DP^(2)-3_12-C2} \\
	&= {\chi}^{\;\! \textcolor{PineGreen}{\hat{3}} \hat{1} \hat{2} }_{\;\! \hat{3} \textcolor{PineGreen}{\hat{1} \hat{2}} \textcolor{Maroon}{(2)}} \mathcolor{gray}{\mathcal F} \left[ M^{\;\! \mathcolor{gray}{3} \hat{1} \hat{2} }_{\;\! \hat{3} \mathcolor{gray}{z} \textcolor{Maroon}{(2)} } \right] \mathcolor{gray}{*} \left( \Xint{\mathcolor{gray}{-}}{295}{E}^{\;\! \textcolor{PineGreen}{\hat{1}}}_{\;\! \hat{1} \mathcolor{gray}{z}} \mathcolor{gray}{*} \Xint{\mathcolor{gray}{-}}{295}{E}^{\;\! \textcolor{PineGreen}{\hat{2}}}_{\;\! \hat{2} \mathcolor{gray}{z}} \right) \label{eq:DP^(2)-3_12-C3} \\
	&= {\chi}^{\;\! \textcolor{PineGreen}{\hat{3}} \hat{1} \hat{2} }_{\;\! \hat{3} \textcolor{PineGreen}{\hat{1} \hat{2}} \textcolor{Maroon}{(2)}} \mathcolor{gray}{\mathcal F_{z}^{-1}} \left[ \mathcolor{gray}{\mathcal F_{\bar{k}}} \left[ M^{\;\! \mathcolor{gray}{3} \hat{1} \hat{2} }_{\;\! \hat{3} \mathcolor{gray}{z} \textcolor{Maroon}{(2)} } \right] \right] \mathcolor{gray}{*} \left( \Xint{\mathcolor{gray}{-}}{295}{E}^{\;\! \textcolor{PineGreen}{\hat{1}}}_{\;\! \hat{1} \mathcolor{gray}{z}} \mathcolor{gray}{*} \Xint{\mathcolor{gray}{-}}{295}{E}^{\;\! \textcolor{PineGreen}{\hat{2}}}_{\;\! \hat{2} \mathcolor{gray}{z}} \right) \label{eq:DP^(2)-3_12-C4} \\
	&= {\chi}^{\;\! \textcolor{PineGreen}{\hat{3}} \hat{1} \hat{2} }_{\;\! \hat{3} \textcolor{PineGreen}{\hat{1} \hat{2}} \textcolor{Maroon}{(2)}} \mathcolor{gray}{\mathcal F_{z}^{-1}} \left[ \mathcolor{gray}{\mathcal F_{\bar{k}}} \left[ M^{\;\! \mathcolor{gray}{3} \hat{1} \hat{2} }_{\;\! \hat{3} \mathcolor{gray}{z} \textcolor{Maroon}{(2)} } \right] \mathcolor{gray}{*} \left( \Xint{\mathcolor{gray}{-}}{295}{E}^{\;\! \textcolor{PineGreen}{\hat{1}}}_{\;\! \hat{1} \mathcolor{gray}{z}} \mathcolor{gray}{*} \Xint{\mathcolor{gray}{-}}{295}{E}^{\;\! \textcolor{PineGreen}{\hat{2}}}_{\;\! \hat{2} \mathcolor{gray}{z}} \right) \right] \label{eq:DP^(2)-3_12-C5} \\
	&=: {\chi}^{\;\! \textcolor{PineGreen}{\hat{3}} \hat{1} \hat{2} }_{\;\! \hat{3} \textcolor{PineGreen}{\hat{1} \hat{2}} \textcolor{Maroon}{(2)}} \mathcolor{gray}{\mathcal F_{z}^{-1}} \left[ \Xint{\mathcolor{gray}{-}}{18}{M}^{\;\! \mathcolor{gray}{3} \hat{1} \hat{2} }_{\;\! \hat{3} \mathcolor{gray}{k_{\symup{z}}} \textcolor{Maroon}{(2)} } \mathcolor{gray}{*} \left( \Xint{\mathcolor{gray}{-}}{295}{E}^{\;\! \textcolor{PineGreen}{\hat{1}}}_{\;\! \hat{1} \mathcolor{gray}{z}} \mathcolor{gray}{*} \Xint{\mathcolor{gray}{-}}{295}{E}^{\;\! \textcolor{PineGreen}{\hat{2}}}_{\;\! \hat{2} \mathcolor{gray}{z}} \right) \right] ~, \label{eq:DP^(2)-3_12-C6}
\end{align}
\end{subequations}
其中,$\Xint{\mathcolor{gray}{-}}{18}{M}^{\;\! \mathcolor{gray}{3} \hat{1} \hat{2} }_{\;\! \hat{3} \mathcolor{gray}{z} \textcolor{Maroon}{(2)} }$ 中的 \textcolor{gray}{灰色数字 3} 表示 $\mathcolor{gray}{\omega}_{\textcolor{gray}{3}}$。

将 \bref{eq:DP^(2)-3_12-C1} 中,由\textcolor{Plum}{分量形式}的 $\Xint{\mathcolor{gray}{-}}{18}{M}^{\;\! \mathcolor{gray}{3} \hat{1} \hat{2} }_{\;\! \hat{3} \mathcolor{gray}{k_{\symup{z}}} \textcolor{Maroon}{(2)} }$ 表示的\textcolor{Plum}{标量形式}的\textcolor{Plum}{非线性}\textcolor{NavyBlue}{波源}项 $\Xint{\mathcolor{gray}{-}}{30}{P}^{\;\! \textcolor{PineGreen}{\hat{3}} \textcolor{Maroon}{(2)} }_{\;\! \hat{3}\mathcolor{gray}{z}}$,升级为由\textcolor{Plum}{半张量形式}的 $\Xint{\mathcolor{gray}{-}}{18}{\bar{M}}^{\;\! \mathcolor{gray}{3} \hat{1} \hat{2} }_{\;\! \mathcolor{gray}{k_{\symup{z}}} \textcolor{Maroon}{(2)} }$ 表示的\textcolor{Plum}{矢量形式} $\Xint{\mathcolor{gray}{-}}{30}{\bar{P}}^{\;\! \textcolor{PineGreen}{\hat{3}} }_{\;\! \mathcolor{gray}{z} \textcolor{Maroon}{(2)} }$,即
\begin{subequations} \label{eq:DP^(2)-3_12-discrete-G}
\begin{align}
	\Xint{\mathcolor{gray}{-}}{30}{\bar{P}}^{\;\! \textcolor{PineGreen}{\hat{3}} }_{\;\! \mathcolor{gray}{z}  \textcolor{Maroon}{(2)}} &= \bar{\chi}^{\;\! \textcolor{PineGreen}{\hat{3}} \hat{1} \hat{2} }_{\;\! \textcolor{Maroon}{(2)} \textcolor{PineGreen}{\hat{1} \hat{2}}} \odot \mathcolor{gray}{\mathcal F_{z}^{-1}} \left[ \Xint{\mathcolor{gray}{-}}{18}{\bar{M}}^{\;\! \mathcolor{gray}{3} \hat{1} \hat{2} }_{\;\! \mathcolor{gray}{k_{\symup{z}}} \textcolor{Maroon}{(2)} } \mathcolor{gray}{*} \Xint{\mathcolor{gray}{-}}{295}{E}^{\;\! \textcolor{PineGreen}{\hat{1}}}_{\;\! \hat{1} \mathcolor{gray}{z}} \mathcolor{gray}{*} \Xint{\mathcolor{gray}{-}}{295}{E}^{\;\! \textcolor{PineGreen}{\hat{2}}}_{\;\! \hat{2} \mathcolor{gray}{z}} \right] \label{eq:DP^(2)-3_12-discrete-G1} \\
	&= \bar{\chi}^{\;\! \textcolor{PineGreen}{\hat{3}} \hat{1} \hat{2} }_{\;\! \textcolor{Maroon}{(2)} \textcolor{PineGreen}{\hat{1} \hat{2}}} \odot \mathcolor{gray}{\mathcal F_{z}^{-1}} \left[ \Xint{\mathcolor{gray}{-}}{18}{\bar{M}}^{\;\! \mathcolor{gray}{3} \hat{1} \hat{2} }_{\;\! \mathcolor{gray}{k_{\symup{z}}} \textcolor{Maroon}{(2)} } \mathcolor{gray}{*} \left( \Xint{\mathcolor{gray}{-}}{20}{\mathtt{G}}^{\;\! \textcolor{PineGreen}{\hat{1}}}_{\;\! \mathcolor{gray}{z}} \Xint{{}^{}_{\mathcolor{gray}{-}}}{10}{\hat{g}}^{\;\! \textcolor{PineGreen}{\hat{1}}}_{\;\! \hat{1}} \right) \mathcolor{gray}{*} \left( \Xint{\mathcolor{gray}{-}}{20}{\mathtt{G}}^{\;\! \textcolor{PineGreen}{\hat{2}}}_{\;\! \mathcolor{gray}{z}} \Xint{{}^{}_{\mathcolor{gray}{-}}}{10}{\hat{g}}^{\;\! \textcolor{PineGreen}{\hat{2}}}_{\;\! \hat{2}} \right) \right] ~, \label{eq:DP^(2)-3_12-discrete-G2}
\end{align}
\end{subequations}
对应地,\bref{eq:simplify8-scalar-g-modulus-P-spectrum} 降为\textcolor{Plum}{离散}个\textcolor{gray}{波长}的\textcolor{NavyBlue}{(准)连续光}\textcolor{Maroon}{和频}或\textcolor{Maroon}{上转换}的\textcolor{NavyBlue}{非超快}版本
\begin{align} \label{eq:simplify8-scalar-g-modulus-P-discrete}
	\mathcolor{gray}{\nabla_z} \Xint{\begin{smallmatrix} ~ \\ {}^{}_{\mathcolor{gray}{-}} \\ ~ \end{smallmatrix}}{09}{\mathtt{g}}^{\;\! \textcolor{PineGreen}{\hat{3}}}_{\;\! \mathcolor{gray}{z}} &= \mathbb{i} k_{\textcolor{Maroon}{\mathsf{o}} \mathcolor{gray}{3}}^{\;\! 2} \frac{\Xint{{}^{}_{\mathcolor{gray}{-}}}{10}{\hat{g}}^{\;\! \textcolor{PineGreen}{\hat{3}} \textcolor{Plum}{\dag}}_{\;\! } \cdot \bar{\chi}^{\;\! \textcolor{PineGreen}{\hat{3}} \hat{1} \hat{2} }_{\;\! \textcolor{Maroon}{(2)} \textcolor{PineGreen}{\hat{1} \hat{2}}} \odot \mathcolor{gray}{\mathcal F_{z}^{-1}} \left[ \Xint{\mathcolor{gray}{-}}{18}{\bar{M}}^{\;\! \mathcolor{gray}{3} \hat{1} \hat{2} }_{\;\! \mathcolor{gray}{k_{\symup{z}}} \textcolor{Maroon}{(2)} } \mathcolor{gray}{*} \Xint{\mathcolor{gray}{-}}{25}{E}^{\;\! \textcolor{PineGreen}{\hat{1}}}_{\;\! \hat{1} \mathcolor{gray}{z}} \mathcolor{gray}{*} \Xint{\mathcolor{gray}{-}}{25}{E}^{\;\! \textcolor{PineGreen}{\hat{2}}}_{\;\! \hat{2} \mathcolor{gray}{z}} \right]}{ 2 \lvert \Xint{{}^{}_{\mathcolor{gray}{-}}}{10}{\hat{g}}^{\;\! \textcolor{PineGreen}{\hat{3}}} \rvert^2 \Xint{\begin{smallmatrix} ~ \\ {}^{}_{\mathcolor{gray}{-}} \\ ~ \end{smallmatrix}}{15}{k}_{\;\! \symup{z}}^{\;\!  \textcolor{PineGreen}{\hat{3}}} \mathbb{e}^{\mathbb{i} \Xint{\begin{smallmatrix} ~ \\ {}^{}_{\mathcolor{gray}{-}} \\ ~ \end{smallmatrix}}{15}{k}_{\symup{z}}^{\;\!  \textcolor{PineGreen}{\hat{3}}} \mathcolor{gray}{z}}} ~, 
\end{align}
同样注意,\textcolor{Plum}{哈达马积} $\odot$ 的运算\textcolor{Plum}{优先级}恒高于\textcolor{Plum}{点积} $\cdot$(以省略一对小括号)。

\vspace*{-2.4em}

\marginLeft[-2.4em]{ssec:scalar}\subsection{标量非线性波源、标量调制场条件}\label{ssec:scalar}

如果\textcolor{Plum}{非线性}\textcolor{NavyBlue}{驱动源}中的\textcolor{NavyBlue}{双泵浦} $\Xint{\mathcolor{gray}{-}}{25}{E}^{\;\! \mathcolor{gray}{\omega} \textcolor{PineGreen}{\hat{1}}}_{\;\! \hat{1} \mathcolor{gray}{z}}, \Xint{\mathcolor{gray}{-}}{25}{E}^{\;\! \mathcolor{gray}{\omega} \textcolor{PineGreen}{\hat{2}}}_{\;\! \hat{2} \mathcolor{gray}{z}}$ 的\textcolor{PineGreen}{本征偏振态} $\Xint{{}^{}_{\mathcolor{gray}{-}}}{10}{\hat{g}}^{\;\! \mathcolor{gray}{\omega} \textcolor{PineGreen}{\hat{1}}}_{\;\! \hat{1}}, \Xint{{}^{}_{\mathcolor{gray}{-}}}{10}{\hat{g}}^{\;\! \mathcolor{gray}{\omega} \textcolor{PineGreen}{\hat{2}}}_{\;\! \hat{2}}$ 固定为\textcolor{Maroon}{倒空间}中的\textcolor{Plum}{定常}矢量 ${\hat{g}}^{\;\! \mathcolor{gray}{\omega} \textcolor{PineGreen}{\hat{1}}}_{\;\! \hat{1}}, {\hat{g}}^{\;\! \mathcolor{gray}{\omega} \textcolor{PineGreen}{\hat{2}}}_{\;\! \hat{2}}$,不是\textcolor{gray}{横向空间频率} $\mathcolor{gray}{\bar{k}_{\symup{\rho}}}$ 的函数,不随\textcolor{PineGreen}{波矢}方向变化而改变,则在该
\begin{subequations} \label{eq:scalar_nonlinear_drive}
\begin{align}
	&\text{\textbf{标量\textcolor{Plum}{非线性}\textcolor{NavyBlue}{波源}}条件(\textcolor{NavyBlue}{脉冲}):} \hspace{0.2em} \Xint{{}^{}_{\mathcolor{gray}{-}}}{10}{\hat{g}}^{\;\! \mathcolor{gray}{\omega} \textcolor{PineGreen}{\hat{1}}}_{\;\! \hat{1}}, \Xint{{}^{}_{\mathcolor{gray}{-}}}{10}{\hat{g}}^{\;\! \mathcolor{gray}{\omega} \textcolor{PineGreen}{\hat{2}}}_{\;\! \hat{2}} \hspace{-4.2em}&&\equiv~ {\hat{g}}^{\;\! \mathcolor{gray}{\omega} \textcolor{PineGreen}{\hat{1}}}_{\;\! \hat{1}}, {\hat{g}}^{\;\! \mathcolor{gray}{\omega} \textcolor{PineGreen}{\hat{2}}}_{\;\! \hat{2}} ~, \label{eq:scalar_nonlinear_drive-spectrum} \\
	&\text{\textbf{标量\textcolor{Plum}{非线性}\textcolor{NavyBlue}{波源}}条件(\textcolor{NavyBlue}{连续}):} \hspace{0.7em} \Xint{{}^{}_{\mathcolor{gray}{-}}}{10}{\hat{g}}^{\;\! \textcolor{PineGreen}{\hat{1}}}_{\;\! \hat{1}}, \Xint{{}^{}_{\mathcolor{gray}{-}}}{10}{\hat{g}}^{\;\! \textcolor{PineGreen}{\hat{2}}}_{\;\! \hat{2}} \hspace{-4.2em}&&\equiv~ {\hat{g}}^{\;\! \textcolor{PineGreen}{\hat{1}}}_{\;\! \hat{1}}, {\hat{g}}^{\;\! \textcolor{PineGreen}{\hat{2}}}_{\;\! \hat{2}} ~, \label{eq:scalar_nonlinear_drive-discrete}
\end{align}
\end{subequations}
下,\bref{eq:DP^(2)-3_12-spectrum-G2,eq:DP^(2)-3_12-discrete-G2} 可进一步\textcolor{Plum}{退化}为\Footnote{注,其中 $\odot$ 运算/相互作用,只作用于数据结构/对象 $\bar{\chi}^{\;\! \mathcolor{gray}{\omega} \textcolor{PineGreen}{\hat{3}} \hat{1} \hat{2} }_{\;\! \textcolor{Maroon}{(2)} \textcolor{PineGreen}{\hat{1} \hat{2}}}$ 与 $\Xint{\mathcolor{gray}{-}}{18}{\bar{M}}^{\;\! \mathcolor{gray}{\omega} \hat{1} \hat{2} }_{\;\! \mathcolor{gray}{k_{\symup{z}}} \textcolor{Maroon}{(2)} }$ 两者,与标量(场)${\hat{g}}^{\;\! \mathcolor{gray}{\omega} \textcolor{PineGreen}{\hat{1}}}_{\;\! \hat{1}}, {\hat{g}}^{\;\! \mathcolor{gray}{\omega} \textcolor{PineGreen}{\hat{2}}}_{\;\! \hat{2}}$ 无关。}
\begin{subequations} \label{eq:DP^(2)-3_12-chieff-G}
\begin{align}
	\Xint{\mathcolor{gray}{-}}{30}{\bar{P}}^{\;\! \mathcolor{gray}{\omega} \textcolor{PineGreen}{\hat{3}} }_{\;\! \mathcolor{gray}{z} \textcolor{Maroon}{(2)} } &= \bar{\chi}^{\;\! \mathcolor{gray}{\omega} \textcolor{PineGreen}{\hat{3}} \hat{1} \hat{2} }_{\;\! \textcolor{Maroon}{(2)} \textcolor{PineGreen}{\hat{1} \hat{2}}} ~ {\hat{g}}^{\;\! \mathcolor{gray}{\omega} \textcolor{PineGreen}{\hat{1}}}_{\;\! \hat{1}} ~\mathcolor{gray}{\widetilde *}~ {\hat{g}}^{\;\! \mathcolor{gray}{\omega} \textcolor{PineGreen}{\hat{2}}}_{\;\! \hat{2}} \odot \mathcolor{gray}{\mathcal F_{z}^{-1}} \left[ \Xint{\mathcolor{gray}{-}}{18}{\bar{M}}^{\;\! \mathcolor{gray}{\omega} \hat{1} \hat{2} }_{\;\! \mathcolor{gray}{k_{\symup{z}}} \textcolor{Maroon}{(2)} } \mathcolor{gray}{*} \Xint{\mathcolor{gray}{-}}{20}{\mathtt{G}}^{\;\! \mathcolor{gray}{\omega} \textcolor{PineGreen}{\hat{1}}}_{\;\! \mathcolor{gray}{z}} ~\mathcolor{gray}{\widetilde \circledast}~ \Xint{\mathcolor{gray}{-}}{20}{\mathtt{G}}^{\;\! \mathcolor{gray}{\omega} \textcolor{PineGreen}{\hat{2}}}_{\;\! \mathcolor{gray}{z}} \right] ~, \label{eq:DP^(2)-3_12-spectrum-chieff-G} \\
	\Xint{\mathcolor{gray}{-}}{30}{\bar{P}}^{\;\! \textcolor{PineGreen}{\hat{3}} }_{\;\! \mathcolor{gray}{z} \textcolor{Maroon}{(2)} } &= \bar{\chi}^{\;\! \textcolor{PineGreen}{\hat{3}} \hat{1} \hat{2} }_{\;\! \textcolor{Maroon}{(2)} \textcolor{PineGreen}{\hat{1} \hat{2}}} ~ {\hat{g}}^{\;\! \textcolor{PineGreen}{\hat{1}}}_{\;\! \hat{1}}  {\hat{g}}^{\;\! \textcolor{PineGreen}{\hat{2}}}_{\;\! \hat{2}} \odot \mathcolor{gray}{\mathcal F_{z}^{-1}} \left[ \Xint{\mathcolor{gray}{-}}{18}{\bar{M}}^{\;\! \mathcolor{gray}{3} \hat{1} \hat{2} }_{\;\! \mathcolor{gray}{k_{\symup{z}}} \textcolor{Maroon}{(2)} } \mathcolor{gray}{*} \Xint{\mathcolor{gray}{-}}{20}{\mathtt{G}}^{\;\! \textcolor{PineGreen}{\hat{1}}}_{\;\! \mathcolor{gray}{z}} \mathcolor{gray}{*} \Xint{\mathcolor{gray}{-}}{20}{\mathtt{G}}^{\;\! \textcolor{PineGreen}{\hat{2}}}_{\;\! \mathcolor{gray}{z}} \right] ~, \label{eq:DP^(2)-3_12-discrete-chieff-G}
\end{align}
\end{subequations}
但这一般是不成立的:因为 \cref{chap:LFCO} 中的\textcolor{Plum}{线性}(\textcolor{Maroon}{傅立叶})\textcolor{PineGreen}{晶体光学}已经解析出结论:在非各向同性材料里,电磁波的\textcolor{PineGreen}{本征偏振态}是 $\mathcolor{gray}{\bar{k}_{\symup{\rho}}}$ 的函数;尽管如此,为了不止步于 \bref{eq:DP^(2)-3_12-spectrum-G},以及为了得到后续的标量\textcolor{Plum}{非线性}\textcolor{Maroon}{时空谱}耦合波方程,我们在
%\bref{chap:LFCO}
\begin{align} \label{eq:scalar_nonlinear_drive2}
	\text{\textbf{参与构成\textcolor{Plum}{非线性}\textcolor{NavyBlue}{波源}的所有行波,均为\textcolor{PineGreen}{偏振态}\textcolor{Plum}{固定}的标量场}}
\end{align}
即“\textbf{标量\textcolor{Plum}{非线性}\textcolor{NavyBlue}{波源}}”的假设 \bref{eq:scalar_nonlinear_drive} 下,从 \bref{eq:DP^(2)-3_12-chieff-G} 开始继续向后推导。

注意,不论正上标带 “$\mathcolor{gray}{\sim}$” 的符号有多少个(这里有两个:“~$\mathcolor{gray}{\widetilde *},~ \mathcolor{gray}{\widetilde \circledast}$~”),只对这些符号所作用的最左(这里即 $\Xint{{}^{}_{\mathcolor{gray}{-}}}{10}{\hat{g}}^{\;\! \mathcolor{gray}{\omega} \textcolor{PineGreen}{\hat{1}}}_{\;\! \hat{1}}$)到最右(这里即 $\Xint{\mathcolor{gray}{-}}{16}{\mathtt{G}}^{\;\! \mathcolor{gray}{\omega} \textcolor{PineGreen}{\hat{2}}}_{\;\! \mathcolor{gray}{z}}$)之间的部分作为\textcolor{Plum}{被积函数}/\textcolor{Plum}{表达式},在\textcolor{gray}{时间频率}维度做一次(而非多次)一维\textcolor{Plum}{卷积积分}。此外,需\textcolor{Plum}{按以下顺序执行积分}:“$\mathcolor{gray}{\widetilde \circledast}$” 的 $\mathcolor{gray}{\bar{k}_{\symup{\rho}}}$ 域 $\to$ $\mathcolor{gray}{\bar{k}_{\symup{\rho}}}$ 域的 “$\mathcolor{gray}{*}$” $\to$ $\mathcolor{gray}{k_{\symup{z}}}$ 域的 $\mathcolor{gray}{\mathcal F^{-1}_z} \left[ \cdot \right]$ $\to$ “$\mathcolor{gray}{\widetilde \circledast}$” 的 $\mathcolor{gray}{\omega}$ 域(即 $\mathcolor{gray}{\omega}$ 域的 "~$\mathcolor{gray}{\widetilde *}$~")。这也是相应程序中 \textbf{for 循环从内到外层的计算顺序}。

在 \bref{eq:scalar_nonlinear_drive,eq:scalar_nonlinear_drive2} 的前提条件下,\bref{eq:simplify8-scalar-g-modulus-P-spectrum,eq:simplify8-scalar-g-modulus-P-discrete} 进一步退化为
\begin{subequations} \label{eq:scalar-g-modulus-P-chieff}
\begin{align}
	\mathcolor{gray}{\nabla_z} \Xint{\begin{smallmatrix} ~ \\ {}^{}_{\mathcolor{gray}{-}} \\ ~ \end{smallmatrix}}{09}{\mathtt{g}}^{\;\!\mathcolor{gray}{\omega} \textcolor{PineGreen}{\hat{3}}}_{\;\! \mathcolor{gray}{z}} &= \mathbb{i} k_{\textcolor{Maroon}{\mathsf{o}} \mathcolor{gray}{\omega}}^{\;\! 2} \frac{\textcolor{gray}{\widetilde{\textcolor{black}{\chi}}}^{\textcolor{Maroon}{(2)} \hat{3} \textcolor{PineGreen}{\hat{3}}}_{\mathcolor{gray}{\omega} \textcolor{NavyBlue}{\text{eff}} \hat{1} \hat{2}} ~ \mathcolor{gray}{\mathcal F_{z}^{-1}} \left[ \Xint{\mathcolor{gray}{-}}{18}{M}^{\;\! \mathcolor{gray}{\omega} \hat{1} \hat{2} }_{\;\! \hat{3} \mathcolor{gray}{k_{\symup{z}}} \textcolor{Maroon}{(2)} } \mathcolor{gray}{*} \Xint{\mathcolor{gray}{-}}{20}{\mathtt{G}}^{\;\! \mathcolor{gray}{\omega} \textcolor{PineGreen}{\hat{1}}}_{\;\! \mathcolor{gray}{z}} ~\mathcolor{gray}{\widetilde \circledast}~ \Xint{\mathcolor{gray}{-}}{20}{\mathtt{G}}^{\;\! \mathcolor{gray}{\omega} \textcolor{PineGreen}{\hat{2}}}_{\;\! \mathcolor{gray}{z}} \right]}{ 2 \lvert \Xint{{}^{}_{\mathcolor{gray}{-}}}{10}{\hat{g}}^{\;\! \textcolor{PineGreen}{\hat{3}}}_{\;\! \mathcolor{gray}{\omega}} \rvert^2 \Xint{\begin{smallmatrix} ~ \\ {}^{}_{\mathcolor{gray}{-}} \\ ~ \end{smallmatrix}}{15}{k}_{\;\! \symup{z}}^{\;\! \mathcolor{gray}{\omega} \textcolor{PineGreen}{\hat{3}}} \mathbb{e}^{\mathbb{i} \Xint{\begin{smallmatrix} ~ \\ {}^{}_{\mathcolor{gray}{-}} \\ ~ \end{smallmatrix}}{15}{k}_{\symup{z}}^{\;\! \mathcolor{gray}{\omega} \textcolor{PineGreen}{\hat{3}}} \mathcolor{gray}{z}}} ~, \label{eq:scalar-g-modulus-P-chieff-spectrum} \\
	\mathcolor{gray}{\nabla_z} \Xint{\begin{smallmatrix} ~ \\ {}^{}_{\mathcolor{gray}{-}} \\ ~ \end{smallmatrix}}{09}{\mathtt{g}}^{\;\! \textcolor{PineGreen}{\hat{3}}}_{\;\! \mathcolor{gray}{z}} &= \mathbb{i} k_{\textcolor{Maroon}{\mathsf{o}} \mathcolor{gray}{3}}^{\;\! 2} \frac{{\chi}^{\hat{3} \textcolor{PineGreen}{\hat{3}} \textcolor{Maroon}{(2)} }_{\textcolor{NavyBlue}{\text{eff}} \hat{1} \hat{2}} ~ \mathcolor{gray}{\mathcal F_{z}^{-1}} \left[ \Xint{\mathcolor{gray}{-}}{18}{M}^{\;\! \mathcolor{gray}{3} \hat{1} \hat{2} }_{\;\! \hat{3} \mathcolor{gray}{k_{\symup{z}}} \textcolor{Maroon}{(2)} } \mathcolor{gray}{*} \Xint{\mathcolor{gray}{-}}{20}{\mathtt{G}}^{\;\! \textcolor{PineGreen}{\hat{1}}}_{\;\! \mathcolor{gray}{z}} \mathcolor{gray}{*} \Xint{\mathcolor{gray}{-}}{20}{\mathtt{G}}^{\;\! \textcolor{PineGreen}{\hat{2}}}_{\;\! \mathcolor{gray}{z}} \right]}{ 2 \lvert \Xint{{}^{}_{\mathcolor{gray}{-}}}{10}{\hat{g}}^{\;\! \textcolor{PineGreen}{\hat{3}}} \rvert^2 \Xint{\begin{smallmatrix} ~ \\ {}^{}_{\mathcolor{gray}{-}} \\ ~ \end{smallmatrix}}{15}{k}_{\;\! \symup{z}}^{\;\!  \textcolor{PineGreen}{\hat{3}}} \mathbb{e}^{\mathbb{i} \Xint{\begin{smallmatrix} ~ \\ {}^{}_{\mathcolor{gray}{-}} \\ ~ \end{smallmatrix}}{15}{k}_{\symup{z}}^{\;\!  \textcolor{PineGreen}{\hat{3}}} \mathcolor{gray}{z}}} ~, \label{eq:scalar-g-modulus-P-chieff-discrete}
\end{align}
\end{subequations}
其中,定义了\textcolor{NavyBlue}{有效非线性系数}
\begin{subequations} \label{eq:chieff}
\begin{align}
	\textcolor{gray}{\widetilde{\textcolor{black}{\chi}}}^{\textcolor{Maroon}{(2)} \hat{3} \textcolor{PineGreen}{\hat{3}}}_{\mathcolor{gray}{\omega} \textcolor{NavyBlue}{\text{eff}} \hat{1} \hat{2}} &= \Xint{{}^{}_{\mathcolor{gray}{-}}}{10}{\hat{g}}^{\;\! \hat{3} \textcolor{PineGreen}{\hat{3}} \textcolor{Plum}{*}}_{\;\! \mathcolor{gray}{\omega}} {\chi}^{\;\! \hat{3} \textcolor{PineGreen}{\hat{3}} }_{\;\! \textcolor{Maroon}{(2)} \hat{1} \hat{2} \textcolor{PineGreen}{\hat{1} \hat{2}} \mathcolor{gray}{\omega}} ~ {\hat{g}}^{\;\! \mathcolor{gray}{\omega} \textcolor{PineGreen}{\hat{1}}}_{\;\! \hat{1}} ~\mathcolor{gray}{\widetilde *}~ {\hat{g}}^{\;\! \mathcolor{gray}{\omega} \textcolor{PineGreen}{\hat{2}}}_{\;\! \hat{2}} ~, \label{eq:chieff-spectrum} \\
	{\chi}^{\hat{3} \textcolor{PineGreen}{\hat{3}} \textcolor{Maroon}{(2)} }_{\textcolor{NavyBlue}{\text{eff}} \hat{1} \hat{2}} &= \Xint{{}^{}_{\mathcolor{gray}{-}}}{10}{\hat{g}}^{\;\! \hat{3} \textcolor{PineGreen}{\hat{3}} \textcolor{Plum}{*}}_{\;\! } {\chi}^{\;\! \hat{3} \textcolor{PineGreen}{\hat{3}} }_{\;\! \textcolor{Maroon}{(2)} \hat{1} \textcolor{PineGreen}{\hat{1}} \hat{2} \textcolor{PineGreen}{\hat{2}}} ~ {\hat{g}}^{\;\! \textcolor{PineGreen}{\hat{1}}}_{\;\! \hat{1}}  {\hat{g}}^{\;\! \textcolor{PineGreen}{\hat{2}}}_{\;\! \hat{2}} ~, \label{eq:chieff-discrete}
\end{align}
\end{subequations}
其中,$\chi$ 头上的一个波浪符号 `$\mathcolor{gray}{\sim}$',表示 $\textcolor{gray}{\widetilde{\textcolor{black}{\chi}}}^{\textcolor{Maroon}{(2)} \hat{3} \textcolor{PineGreen}{\hat{3}}}_{\mathcolor{gray}{\omega} \textcolor{NavyBlue}{\text{eff}} \hat{1} \hat{2}}~, {\chi}^{\hat{3} \textcolor{PineGreen}{\hat{3}} \textcolor{Maroon}{(2)} }_{\textcolor{NavyBlue}{\text{eff}} \hat{1} \hat{2}}$ 整体,作为\textcolor{Plum}{被积函数},处在\textcolor{gray}{时间频率域}的一维\textcolor{Plum}{卷积积分}内。注意,\bref{eq:scalar-g-modulus-P-chieff} 的分子,即 $\textcolor{gray}{\widetilde{\textcolor{black}{\chi}}}^{\textcolor{Maroon}{(2)} \hat{3} \textcolor{PineGreen}{\hat{3}}}_{\mathcolor{gray}{\omega} \textcolor{NavyBlue}{\text{eff}} \hat{1} \hat{2}}~, {\chi}^{\hat{3} \textcolor{PineGreen}{\hat{3}} \textcolor{Maroon}{(2)} }_{\textcolor{NavyBlue}{\text{eff}} \hat{1} \hat{2}}$ 中的 $\Xint{{}^{}_{\mathcolor{gray}{-}}}{10}{\hat{g}}^{\;\! \hat{3} \textcolor{PineGreen}{\hat{3}} \textcolor{Plum}{*}}_{\;\! \mathcolor{gray}{\omega}}, \Xint{{}^{}_{\mathcolor{gray}{-}}}{10}{\hat{g}}^{\;\! \hat{3} \textcolor{PineGreen}{\hat{3}} \textcolor{Plum}{*}}_{\;\! }$ 是(协变)矢量\Footnote{这里对协变 $\bra{\text{bra}}$、逆变 $\ket{\text{ket}}$ 指标 $\hat{3} \textcolor{PineGreen}{\hat{3}}$ 上下位置的定义,相反于 \byperref{another-bra-example}{它处}。可以看出,在这方面,是灵活的。}的分量,即标量;而分母中的 $\Xint{{}^{}_{\mathcolor{gray}{-}}}{10}{\hat{g}}^{\;\! \textcolor{PineGreen}{\hat{3}}}_{\;\! \mathcolor{gray}{\omega}}, \Xint{{}^{}_{\mathcolor{gray}{-}}}{10}{\hat{g}}^{\;\! \textcolor{PineGreen}{\hat{3}}}$ 是矢量。

若三阶张量\textcolor{NavyBlue}{调制场} $\Xint{\mathcolor{gray}{-}}{18}{M}^{\;\! \mathcolor{gray}{\omega} \hat{1} \hat{2} }_{\;\! \hat{3} \mathcolor{gray}{z} \textcolor{Maroon}{(2)} }, \Xint{\mathcolor{gray}{-}}{18}{\bar{M}}^{\;\! \mathcolor{gray}{\omega} \hat{1} \hat{2} }_{\;\! \mathcolor{gray}{z} \textcolor{Maroon}{(2)} }$ 退化为标量\textcolor{NavyBlue}{场} $\Xint{\mathcolor{gray}{-}}{18}{M}^{\;\! \mathcolor{gray}{\omega} }_{\;\! \mathcolor{gray}{z} \textcolor{Maroon}{(2)} }$,则 \bref{eq:scalar-g-modulus-P-chieff} 退化为
\begin{subequations} \label{eq:scalar-g-modulus-P-chieff-scalar}
\begin{align}
	\mathcolor{gray}{\nabla_z} \Xint{\begin{smallmatrix} ~ \\ {}^{}_{\mathcolor{gray}{-}} \\ ~ \end{smallmatrix}}{09}{\mathtt{g}}^{\;\!\mathcolor{gray}{\omega} \textcolor{PineGreen}{\hat{3}}}_{\;\! \mathcolor{gray}{z}} &= \mathbb{i} k_{\textcolor{Maroon}{\mathsf{o}} \mathcolor{gray}{\omega}}^{\;\! 2} \frac{\textcolor{gray}{\widetilde{\textcolor{black}{\chi}}}^{\textcolor{Maroon}{(2)} \textcolor{PineGreen}{\hat{3}}}_{\mathcolor{gray}{\omega} \textcolor{NavyBlue}{\text{eff}}} ~ \mathcolor{gray}{\mathcal F_{z}^{-1}} \left[ \Xint{\mathcolor{gray}{-}}{18}{M}^{\;\! \mathcolor{gray}{\omega} }_{\;\! \mathcolor{gray}{k_{\symup{z}}} \textcolor{Maroon}{(2)} } \mathcolor{gray}{*} \Xint{\mathcolor{gray}{-}}{20}{\mathtt{G}}^{\;\! \mathcolor{gray}{\omega} \textcolor{PineGreen}{\hat{1}}}_{\;\! \mathcolor{gray}{z}} ~\mathcolor{gray}{\widetilde \circledast}~ \Xint{\mathcolor{gray}{-}}{20}{\mathtt{G}}^{\;\! \mathcolor{gray}{\omega} \textcolor{PineGreen}{\hat{2}}}_{\;\! \mathcolor{gray}{z}} \right]}{ 2 \lvert \Xint{{}^{}_{\mathcolor{gray}{-}}}{10}{\hat{g}}^{\;\! \textcolor{PineGreen}{\hat{3}}}_{\;\! \mathcolor{gray}{\omega}} \rvert^2 \Xint{\begin{smallmatrix} ~ \\ {}^{}_{\mathcolor{gray}{-}} \\ ~ \end{smallmatrix}}{15}{k}_{\;\! \symup{z}}^{\;\! \mathcolor{gray}{\omega} \textcolor{PineGreen}{\hat{3}}} \mathbb{e}^{\mathbb{i} \Xint{\begin{smallmatrix} ~ \\ {}^{}_{\mathcolor{gray}{-}} \\ ~ \end{smallmatrix}}{15}{k}_{\symup{z}}^{\;\! \mathcolor{gray}{\omega} \textcolor{PineGreen}{\hat{3}}} \mathcolor{gray}{z}}} ~, \label{eq:scalar-g-modulus-P-chieff-scalar-spectrum} \\
	\mathcolor{gray}{\nabla_z} \Xint{\begin{smallmatrix} ~ \\ {}^{}_{\mathcolor{gray}{-}} \\ ~ \end{smallmatrix}}{09}{\mathtt{g}}^{\;\! \textcolor{PineGreen}{\hat{3}}}_{\;\! \mathcolor{gray}{z}} &= \mathbb{i} k_{\textcolor{Maroon}{\mathsf{o}} \mathcolor{gray}{3}}^{\;\! 2} \frac{{\chi}^{\textcolor{Maroon}{(2)} \textcolor{PineGreen}{\hat{3}}}_{\textcolor{NavyBlue}{\text{eff}}} ~ \mathcolor{gray}{\mathcal F_{z}^{-1}} \left[ \Xint{\mathcolor{gray}{-}}{18}{M}^{\;\! \mathcolor{gray}{3}}_{\;\! \mathcolor{gray}{k_{\symup{z}}} \textcolor{Maroon}{(2)} } \mathcolor{gray}{*} \Xint{\mathcolor{gray}{-}}{20}{\mathtt{G}}^{\;\! \textcolor{PineGreen}{\hat{1}}}_{\;\! \mathcolor{gray}{z}} \mathcolor{gray}{*} \Xint{\mathcolor{gray}{-}}{20}{\mathtt{G}}^{\;\! \textcolor{PineGreen}{\hat{2}}}_{\;\! \mathcolor{gray}{z}} \right]}{ 2 \lvert \Xint{{}^{}_{\mathcolor{gray}{-}}}{10}{\hat{g}}^{\;\! \textcolor{PineGreen}{\hat{3}}} \rvert^2 \Xint{\begin{smallmatrix} ~ \\ {}^{}_{\mathcolor{gray}{-}} \\ ~ \end{smallmatrix}}{15}{k}_{\;\! \symup{z}}^{\;\!  \textcolor{PineGreen}{\hat{3}}} \mathbb{e}^{\mathbb{i} \Xint{\begin{smallmatrix} ~ \\ {}^{}_{\mathcolor{gray}{-}} \\ ~ \end{smallmatrix}}{15}{k}_{\symup{z}}^{\;\!  \textcolor{PineGreen}{\hat{3}}} \mathcolor{gray}{z}}} ~, \label{eq:scalar-g-modulus-P-chieff-scalar-discrete}
\end{align}
\end{subequations}
对应地,\textcolor{NavyBlue}{有效非线性系数} \bref{eq:chieff} 退化为
\begin{subequations} \label{eq:chieff-scalar}
\begin{align}
	\textcolor{gray}{\widetilde{\textcolor{black}{\chi}}}^{\textcolor{Maroon}{(2)}\textcolor{PineGreen}{\hat{3}}}_{\mathcolor{gray}{\omega} \textcolor{NavyBlue}{\text{eff}}} &= \Xint{{}^{}_{\mathcolor{gray}{-}}}{10}{\hat{g}}^{\;\! \hat{3} \textcolor{PineGreen}{\hat{3}} \textcolor{Plum}{*}}_{\;\! \mathcolor{gray}{\omega}} {\chi}^{\;\! \textcolor{PineGreen}{\hat{3}} \mathcolor{gray}{\omega} \hat{1} \hat{2} }_{\;\! \hat{3} \textcolor{Maroon}{(2)} \textcolor{PineGreen}{\hat{1} \hat{2}}} ~ {\hat{g}}^{\;\! \mathcolor{gray}{\omega} \textcolor{PineGreen}{\hat{1}}}_{\;\! \hat{1}} ~\mathcolor{gray}{\widetilde *}~ {\hat{g}}^{\;\! \mathcolor{gray}{\omega} \textcolor{PineGreen}{\hat{2}}}_{\;\! \hat{2}} ~, \label{eq:chieff-scalar-spectrum} \\
	{\chi}^{\textcolor{Maroon}{(2)}\textcolor{PineGreen}{\hat{3}}}_{\textcolor{NavyBlue}{\text{eff}}} &= \Xint{{}^{}_{\mathcolor{gray}{-}}}{10}{\hat{g}}^{\;\! \hat{3} \textcolor{PineGreen}{\hat{3}} \textcolor{Plum}{*}}_{\;\! } {\chi}^{\;\! \textcolor{PineGreen}{\hat{3}} \hat{1} \hat{2} }_{\;\! \hat{3} \textcolor{Maroon}{(2)} \textcolor{PineGreen}{\hat{1} \hat{2}}} ~ {\hat{g}}^{\;\! \textcolor{PineGreen}{\hat{1}}}_{\;\! \hat{1}}  {\hat{g}}^{\;\! \textcolor{PineGreen}{\hat{2}}}_{\;\! \hat{2}} ~, \label{eq:chieff-scalar-discrete}
\end{align}
\end{subequations}
注意,\bref{eq:chieff} 中的 ${\chi}^{\;\! \hat{3} \textcolor{PineGreen}{\hat{3}} }_{\;\! \textcolor{Maroon}{(2)} \hat{1} \hat{2} \textcolor{PineGreen}{\hat{1} \hat{2}} \mathcolor{gray}{\omega}}, {\chi}^{\;\! \hat{3} \textcolor{PineGreen}{\hat{3}} }_{\;\! \textcolor{Maroon}{(2)} \hat{1} \textcolor{PineGreen}{\hat{1}} \hat{2} \textcolor{PineGreen}{\hat{2}}}$,以及 \bref{eq:chieff-scalar} 中的 ${\chi}^{\;\! \textcolor{PineGreen}{\hat{3}} \mathcolor{gray}{\omega} \hat{1} \hat{2} }_{\;\! \hat{3} \textcolor{Maroon}{(2)} \textcolor{PineGreen}{\hat{1} \hat{2}}}, {\chi}^{\;\! \textcolor{PineGreen}{\hat{3}} \hat{1} \hat{2} }_{\;\! \hat{3} \textcolor{Maroon}{(2)} \textcolor{PineGreen}{\hat{1} \hat{2}}}$,如 \bref{eq:chi2-modulate} 下方的说明文字所提,它们的角标 $\textcolor{PineGreen}{\hat{1}}, \textcolor{PineGreen}{\hat{2}}$ 不从属于任何主体(如 $g$ 或 $\chi$),只服务于爱因斯坦求和。

\marginLeft[-2.4em]{sec:down_convert}\section{\textcolor{Maroon}{Down conversion} 下转换 - 电场本征复振幅 \textcolor{Maroon}{equation}}\label{sec:down_convert}

\marginLeft[-2.4em]{ssec:OR_spectrum}\subsection{脉冲光整流 - 电场本征复振幅方程}\label{ssec:OR_spectrum}



\marginLeft[-2.4em]{ssec:DFG_discrete}\subsection{连续光差频 - 电场本征复振幅方程}\label{ssec:DFG_discrete}

\cite{dregerSecondharmonicGenerationNonlinear1990,zubairyAnalyticApproachSecondharmonic1985}

too many diffs, cannot push to gitee, but github is ok

\cite{katoSecondharmonicGeneration20481986,katoTemperaturetuned90Phasematching1994,brunerTemperaturedependentSellmeierEquation2003,jundtTemperaturedependentSellmeierEquation1997,katoSellmeierThermoopticDispersion2002}

