\marginLeft[-2.4em]{chap:NLAST}\chapter{任意 \texorpdfstring{$\chi ( \bar{r} )$}{$\text{χ} ( \bar{r} )$} 材料中的(标量)非线性角谱理论}\label{chap:NLAST}

\bref{ssec:E-waveq-nonlinear} 末得到了\textcolor{PineGreen}{纯电(非磁)各向异性}介质中的\textcolor{Plum}{非线性}矢量电场波动方程 \bref{eq:simplify7-LE0-SVA-V_1singular-nokxky-zeta-g},其右侧的二阶\textcolor{Plum}{局域}\textcolor{Plum}{非线性}\textcolor{NavyBlue}{电偶-$(\text{电偶}\otimes\text{电偶})$}极波源(的\textcolor{Plum}{横向}部分) $\Xint{\mathcolor{gray}{-}}{25}{\bar{P}}^{\;\! \mathcolor{gray}{\omega} \textcolor{PineGreen}{\imath}}_{\;\! \symup{\rho} \mathcolor{gray}{z} \textcolor{Maroon}{(2)}}$(\bref{eq:vec-DP^(2)-p_pp,eq:vec-DP^(2)-plane_wave_basis-p_pp}),可以按需更改为三阶、四阶非线性 $\Xint{\mathcolor{gray}{-}}{25}{\bar{P}}^{\;\! \mathcolor{gray}{\omega} \textcolor{PineGreen}{\imath}}_{\;\! \symup{\rho} \mathcolor{gray}{z} \textcolor{Maroon}{(3)}}, \Xint{\mathcolor{gray}{-}}{25}{\bar{P}}^{\;\! \mathcolor{gray}{\omega} \textcolor{PineGreen}{\imath}}_{\;\! \symup{\rho} \mathcolor{gray}{z} \textcolor{Maroon}{(4)}}$,或者一阶势散射源 $\Xint{\mathcolor{gray}{-}}{25}{\bar{P}}^{\;\! \mathcolor{gray}{\omega} \textcolor{PineGreen}{\imath}}_{\;\! \symup{\rho} \mathcolor{gray}{z} \textcolor{Maroon}{(1)}}$(\bref{eq:nonlinear(2)-wave_wkrho-simplify4-2}),甚至他们的\textcolor{Plum}{线性组合}/\textcolor{Plum}{叠加} $\Xint{\mathcolor{gray}{-}}{25}{\bar{P}}^{\;\! \mathcolor{gray}{\omega} \textcolor{PineGreen}{\imath}}_{\;\! \symup{\rho} \mathcolor{gray}{z}}$。

将每一个参与相互作用的时间频率组分 $\mathcolor{gray}{\omega}$ 所对应的 \bref{eq:simplify7-LE0-SVA-V_1singular-nokxky-zeta-g} 组合起来,即得到\textcolor{PineGreen}{实验室}\textcolor{Plum}{坐标系} \textcolor{PineGreen}{$\mathcal{Z}$ 系}下的矢量(电场)\textcolor{Maroon}{时空谱}耦合波方程组 --- 它既可以由相干脉冲光连续谱 $\{ \mathcolor{gray}{\omega} \}$ 对应的\textcolor{Plum}{无穷个}矢量时空谱耦合波方程组成,也可以由\textcolor{Plum}{有限个}连续光 $\{ \mathcolor{gray}{\omega}_{\textcolor{Maroon}{i}} \}$ 对应的有限个矢量\textcolor{Maroon}{时空谱}耦合波方程组成;也可以只是\textcolor{gray}{单一频率} $\mathcolor{gray}{\omega}$ 的光,其自己参与的如\textcolor{Maroon}{三}/\textcolor{Maroon}{四波混频}或\textcolor{PineGreen}{折射率微调制}引起的\textcolor{Maroon}{(势)散射过程}等,所对应的\textcolor{Plum}{单个}矢量\textcolor{Maroon}{时空谱}的传播/波动方程。

\marginLeft[-2.4em]{sec:test}\section{连续光和频、脉冲光倍频的时空谱耦合波方程}\label{sec:test}

利用 \bref{eq:vec-amp_polar} 即 $\Xint{{}^{}_{\mathcolor{gray}{-}}}{10}{\bar{g}}^{\;\!\mathcolor{gray}{\omega} \textcolor{PineGreen}{\jmath}}_{\;\! \mathcolor{gray}{z}} := \Xint{\begin{smallmatrix} ~ \\ {}^{}_{\mathcolor{gray}{-}} \\ ~ \end{smallmatrix}}{09}{\mathtt{g}}^{\;\!\mathcolor{gray}{\omega} \textcolor{PineGreen}{\jmath}}_{\;\! \mathcolor{gray}{z}} \Xint{{}^{}_{\mathcolor{gray}{-}}}{10}{\bar{g}}^{\;\!\mathcolor{gray}{\omega} \textcolor{PineGreen}{\jmath}}$ 在\textcolor{PineGreen}{纯电(非磁)各向异性}介质中的\textcolor{Plum}{横向}形式,将\textcolor{Plum}{二维}\textcolor{PineGreen}{本征偏振态}\textcolor{Plum}{复}矢量 $\Xint{{}^{}_{\mathcolor{gray}{-}}}{10}{\bar{g}}^{\;\!\mathcolor{gray}{\omega} \textcolor{PineGreen}{\pm}}_{\;\! \textcolor{Maroon}{\Yup}}$ 三分量\textcolor{Plum}{复归一化}\Footnote{注意,不像 \bref{eq:transition_matrix-transverse_input},\textcolor{PineGreen}{本征偏振态} $\Xint{{}^{}_{\mathcolor{gray}{-}}}{10}{\bar{g}}^{\;\!\mathcolor{gray}{\omega} \textcolor{PineGreen}{\pm}}_{\;\! \textcolor{Maroon}{\Yup}}, \Xint{{}^{}_{\mathcolor{gray}{-}}}{10}{\bar{g}}^{\;\!\mathcolor{gray}{\omega} \textcolor{PineGreen}{\pm}}_{\;\! \textcolor{Maroon}{\symup{\rho}}}$ 在\textcolor{Plum}{线性}\textcolor{PineGreen}{晶体光学}中无需\textcolor{Plum}{(三维)复归一化},但在\textcolor{Plum}{非线性}\textcolor{PineGreen}{晶体光学}中最好\textcolor{Plum}{三维复归一化},这既有助于\textcolor{Plum}{标准化}后续引入的\textcolor{PineGreen}{有效非线性系数},又赋予和明确了\textcolor{PineGreen}{复振幅} $\Xint{\begin{smallmatrix} ~ \\ {}^{}_{\mathcolor{gray}{-}} \\ ~ \end{smallmatrix}}{09}{\mathtt{g}}^{\;\!\mathcolor{gray}{\omega} \textcolor{PineGreen}{\pm}}_{\;\! \mathcolor{gray}{z}}$、\textcolor{PineGreen}{本征偏振态} $\Xint{{}^{}_{\mathcolor{gray}{-}}}{10}{\bar{g}}^{\;\!\mathcolor{gray}{\omega} \textcolor{PineGreen}{\pm}}_{\;\! \textcolor{Maroon}{\Yup}}$ 各自的物理意义。但实际上,\textcolor{Plum}{非线性}\textcolor{PineGreen}{晶体光学}也允许\textcolor{Plum}{不归一化} $\Xint{{}^{}_{\mathcolor{gray}{-}}}{10}{\bar{g}}^{\;\!\mathcolor{gray}{\omega} \textcolor{PineGreen}{\pm}}_{\;\! \textcolor{Maroon}{\Yup}}$ 或 $\Xint{{}^{}_{\mathcolor{gray}{-}}}{10}{\bar{g}}^{\;\!\mathcolor{gray}{\omega} \textcolor{PineGreen}{\pm}}_{\;\! \textcolor{Maroon}{\symup{\rho}}}$,见下文。}后的 $\Xint{{}^{}_{\mathcolor{gray}{-}}}{10}{\bar{g}}^{\;\!\mathcolor{gray}{\omega} \textcolor{PineGreen}{\pm}}_{\;\! \textcolor{Maroon}{\symup{\rho}} \mathcolor{gray}{z}} := \Xint{\begin{smallmatrix} ~ \\ {}^{}_{\mathcolor{gray}{-}} \\ ~ \end{smallmatrix}}{09}{\mathtt{g}}^{\;\!\mathcolor{gray}{\omega} \textcolor{PineGreen}{\pm}}_{\;\! \mathcolor{gray}{z}} \Xint{{}^{}_{\mathcolor{gray}{-}}}{10}{\hat{g}}^{\;\!\mathcolor{gray}{\omega} \textcolor{PineGreen}{\pm}}_{\;\! \textcolor{Maroon}{\symup{\rho}}}$代入矢量\textcolor{Plum}{非线性}波动方程 \bref{eq:simplify7-LE0-SVA-V_1singular-nokxky-zeta-g} 的\textcolor{NavyBlue}{左侧场}中,并两侧\textcolor{Plum}{点乘}不含 $\mathcolor{gray}{z}$ 的\textcolor{Plum}{二维横向}\textcolor{PineGreen}{本征偏振态} $\Xint{{}^{}_{\mathcolor{gray}{-}}}{10}{\hat{g}}^{\;\!\mathcolor{gray}{\omega} \textcolor{PineGreen}{\pm}}_{\;\! \textcolor{Maroon}{\symup{\rho}}}$ 的\textcolor{Plum}{共轭转置} $\Xint{{}^{}_{\mathcolor{gray}{-}}}{10}{\hat{g}}^{\;\!\mathcolor{gray}{\omega} \textcolor{PineGreen}{\pm} \textcolor{Plum}{\dag}}_{\;\! \textcolor{Maroon}{\symup{\rho}}}$,可得
\begin{subequations} \label{eq:simplify7-scalar}
\begin{align}
	\Xint{{}^{}_{\mathcolor{gray}{-}}}{10}{\hat{g}}^{\;\!\mathcolor{gray}{\omega} \textcolor{PineGreen}{\pm}}_{\;\! \textcolor{Maroon}{\symup{\rho}}} \mathcolor{gray}{\nabla_z} \Xint{\begin{smallmatrix} ~ \\ {}^{}_{\mathcolor{gray}{-}} \\ ~ \end{smallmatrix}}{09}{\mathtt{g}}^{\;\!\mathcolor{gray}{\omega} \textcolor{PineGreen}{\pm}}_{\;\! \mathcolor{gray}{z}} &= \mathbb{i} k_{\textcolor{Maroon}{\mathsf{o}} \mathcolor{gray}{\omega}}^{\;\! 2} \frac{\Xint{\mathcolor{gray}{-}}{25}{\bar{P}}^{\;\! \mathcolor{gray}{\omega} \textcolor{PineGreen}{\pm}}_{\;\! \textcolor{Maroon}{\symup{\rho}} \mathcolor{gray}{z}}}{2 \Xint{\begin{smallmatrix} ~ \\ {}^{}_{\mathcolor{gray}{-}} \\ ~ \end{smallmatrix}}{15}{k}_{\;\! \symup{z}}^{\;\! \mathcolor{gray}{\omega} \textcolor{PineGreen}{\pm}} \mathbb{e}^{\mathbb{i} \Xint{\begin{smallmatrix} ~ \\ {}^{}_{\mathcolor{gray}{-}} \\ ~ \end{smallmatrix}}{15}{k}_{\symup{z}}^{\;\! \mathcolor{gray}{\omega} \textcolor{PineGreen}{\pm}} \mathcolor{gray}{z}}} \label{eq:simplify7-scalar-g} \\
	\mathcolor{gray}{\nabla_z} \Xint{\begin{smallmatrix} ~ \\ {}^{}_{\mathcolor{gray}{-}} \\ ~ \end{smallmatrix}}{09}{\mathtt{g}}^{\;\!\mathcolor{gray}{\omega} \textcolor{PineGreen}{\pm}}_{\;\! \mathcolor{gray}{z}} &= \mathbb{i} k_{\textcolor{Maroon}{\mathsf{o}} \mathcolor{gray}{\omega}}^{\;\! 2} \frac{\Xint{{}^{}_{\mathcolor{gray}{-}}}{10}{\hat{g}}^{\;\!\mathcolor{gray}{\omega} \textcolor{PineGreen}{\pm} \textcolor{Plum}{\dag}}_{\;\! \textcolor{Maroon}{\symup{\rho}}} \cdot \Xint{\mathcolor{gray}{-}}{25}{\bar{P}}^{\;\! \mathcolor{gray}{\omega} \textcolor{PineGreen}{\pm}}_{\;\! \textcolor{Maroon}{\symup{\rho}} \mathcolor{gray}{z}}}{\Xint{{}^{}_{\mathcolor{gray}{-}}}{10}{\hat{g}}^{\;\!\mathcolor{gray}{\omega} \textcolor{PineGreen}{\pm} \textcolor{Plum}{\dag}}_{\;\! \textcolor{Maroon}{\symup{\rho}}} \cdot \Xint{{}^{}_{\mathcolor{gray}{-}}}{10}{\hat{g}}^{\;\!\mathcolor{gray}{\omega} \textcolor{PineGreen}{\pm}}_{\;\! \textcolor{Maroon}{\symup{\rho}}} 2 \Xint{\begin{smallmatrix} ~ \\ {}^{}_{\mathcolor{gray}{-}} \\ ~ \end{smallmatrix}}{15}{k}_{\;\! \symup{z}}^{\;\! \mathcolor{gray}{\omega} \textcolor{PineGreen}{\pm}} \mathbb{e}^{\mathbb{i} \Xint{\begin{smallmatrix} ~ \\ {}^{}_{\mathcolor{gray}{-}} \\ ~ \end{smallmatrix}}{15}{k}_{\symup{z}}^{\;\! \mathcolor{gray}{\omega} \textcolor{PineGreen}{\pm}} \mathcolor{gray}{z}}} ~,  \label{eq:simplify7-scalar-g-conjugate}
\end{align}
\end{subequations}
其中 \bref{eq:simplify7-scalar-g-conjugate} 即为标量\textcolor{Maroon}{时空谱},即\textcolor{PineGreen}{复振幅} $\Xint{\begin{smallmatrix} ~ \\ {}^{}_{\mathcolor{gray}{-}} \\ ~ \end{smallmatrix}}{09}{\mathtt{g}}^{\;\!\mathcolor{gray}{\omega} \textcolor{PineGreen}{\pm}}_{\;\! \mathcolor{gray}{z}}$ 满足的矢量\textcolor{Plum}{非线性}波动方程。分母中的 $\Xint{{}^{}_{\mathcolor{gray}{-}}}{10}{\hat{g}}^{\;\!\mathcolor{gray}{\omega} \textcolor{PineGreen}{\pm} \textcolor{Plum}{\dag}}_{\;\! \textcolor{Maroon}{\symup{\rho}}} \cdot \Xint{{}^{}_{\mathcolor{gray}{-}}}{10}{\hat{g}}^{\;\!\mathcolor{gray}{\omega} \textcolor{PineGreen}{\pm}}_{\;\! \textcolor{Maroon}{\symup{\rho}}}$,其实在暗示 $\Xint{{}^{}_{\mathcolor{gray}{-}}}{10}{\hat{g}}^{\;\!\mathcolor{gray}{\omega} \textcolor{PineGreen}{\pm}}_{\;\! \textcolor{Maroon}{\symup{\rho}}}$ 既可以是\textcolor{Plum}{二维复归一化},也可以是\textcolor{Plum}{三维复归一化}后的。这对应 $\Xint{{}^{}_{\mathcolor{gray}{-}}}{10}{\hat{g}}^{\;\!\mathcolor{gray}{\omega} \textcolor{PineGreen}{\pm} \textcolor{Plum}{\dag}}_{\;\! \textcolor{Maroon}{\symup{\rho}}} \cdot \Xint{{}^{}_{\mathcolor{gray}{-}}}{10}{\hat{g}}^{\;\!\mathcolor{gray}{\omega} \textcolor{PineGreen}{\pm}}_{\;\! \textcolor{Maroon}{\symup{\rho}}}$ 的值,既可以是也可以不是 $1$。并且甚至可以对 \bref{eq:simplify7-scalar-g} 左侧随便\textcolor{Plum}{点乘}一个\textcolor{Plum}{二维}复向量(场) $\Xint{{}^{}_{\mathcolor{gray}{-}}}{04}{\bar{c}}^{\;\!\mathcolor{gray}{\omega} \textcolor{PineGreen}{\pm}}_{\;\! \textcolor{Maroon}{\symup{\rho}}}$(不一定非得是\textcolor{Plum}{横向}\textcolor{PineGreen}{本征偏振态}的\textcolor{Plum}{共轭转置} $\Xint{{}^{}_{\mathcolor{gray}{-}}}{10}{\hat{g}}^{\;\!\mathcolor{gray}{\omega} \textcolor{PineGreen}{\pm} \textcolor{Plum}{\dag}}_{\;\! \textcolor{Maroon}{\symup{\rho}}}$)都行,只需保证 \bref{eq:simplify7-scalar-g-conjugate} 分母中的 $\Xint{{}^{}_{\mathcolor{gray}{-}}}{04}{\bar{c}}^{\;\!\mathcolor{gray}{\omega} \textcolor{PineGreen}{\pm}}_{\;\! \textcolor{Maroon}{\symup{\rho}}} \cdot \Xint{{}^{}_{\mathcolor{gray}{-}}}{10}{\hat{g}}^{\;\!\mathcolor{gray}{\omega} \textcolor{PineGreen}{\pm}}_{\;\! \textcolor{Maroon}{\symup{\rho}}} \neq 0$。

\bref{eq:simplify7-scalar} 中所有\textcolor{Plum}{横向} $\textcolor{Maroon}{\symup{\rho}}$ 场,somehow\Footnote{出于物理学家的直觉和对形式美的追求。类似数学家的“注意到”,但没有他们的“注意到”那么严谨。}可进一步写做笛卡尔\textcolor{Plum}{三分量} $\textcolor{Maroon}{\Yup}$ 形式
\begin{subequations} \label{eq:simplify8-scalar}
\begin{align}
	\Xint{{}^{}_{\mathcolor{gray}{-}}}{10}{\hat{g}}^{\;\!\mathcolor{gray}{\omega} \textcolor{PineGreen}{\pm}}_{\;\! \textcolor{Maroon}{\Yup}} \mathcolor{gray}{\nabla_z} \Xint{\begin{smallmatrix} ~ \\ {}^{}_{\mathcolor{gray}{-}} \\ ~ \end{smallmatrix}}{09}{\mathtt{g}}^{\;\!\mathcolor{gray}{\omega} \textcolor{PineGreen}{\pm}}_{\;\! \mathcolor{gray}{z}} &= \mathbb{i} k_{\textcolor{Maroon}{\mathsf{o}} \mathcolor{gray}{\omega}}^{\;\! 2} \frac{\Xint{\mathcolor{gray}{-}}{25}{\bar{P}}^{\;\! \mathcolor{gray}{\omega} \textcolor{PineGreen}{\pm}}_{\;\! \textcolor{Maroon}{\Yup} \mathcolor{gray}{z}}}{2 \Xint{\begin{smallmatrix} ~ \\ {}^{}_{\mathcolor{gray}{-}} \\ ~ \end{smallmatrix}}{15}{k}_{\;\! \symup{z}}^{\;\! \mathcolor{gray}{\omega} \textcolor{PineGreen}{\pm}} \mathbb{e}^{\mathbb{i} \Xint{\begin{smallmatrix} ~ \\ {}^{}_{\mathcolor{gray}{-}} \\ ~ \end{smallmatrix}}{15}{k}_{\symup{z}}^{\;\! \mathcolor{gray}{\omega} \textcolor{PineGreen}{\pm}} \mathcolor{gray}{z}}} \label{eq:simplify8-scalar-g} \\
	\mathcolor{gray}{\nabla_z} \Xint{\begin{smallmatrix} ~ \\ {}^{}_{\mathcolor{gray}{-}} \\ ~ \end{smallmatrix}}{09}{\mathtt{g}}^{\;\!\mathcolor{gray}{\omega} \textcolor{PineGreen}{\pm}}_{\;\! \mathcolor{gray}{z}} &= \mathbb{i} k_{\textcolor{Maroon}{\mathsf{o}} \mathcolor{gray}{\omega}}^{\;\! 2} \frac{\Xint{{}^{}_{\mathcolor{gray}{-}}}{10}{\hat{g}}^{\;\!\mathcolor{gray}{\omega} \textcolor{PineGreen}{\pm} \textcolor{Plum}{\dag}}_{\;\! \textcolor{Maroon}{\Yup}} \cdot \Xint{\mathcolor{gray}{-}}{25}{\bar{P}}^{\;\! \mathcolor{gray}{\omega} \textcolor{PineGreen}{\pm}}_{\;\! \textcolor{Maroon}{\Yup} \mathcolor{gray}{z}}}{\Xint{{}^{}_{\mathcolor{gray}{-}}}{10}{\hat{g}}^{\;\!\mathcolor{gray}{\omega} \textcolor{PineGreen}{\pm} \textcolor{Plum}{\dag}}_{\;\! \textcolor{Maroon}{\Yup}} \cdot \Xint{{}^{}_{\mathcolor{gray}{-}}}{10}{\hat{g}}^{\;\!\mathcolor{gray}{\omega} \textcolor{PineGreen}{\pm}}_{\;\! \textcolor{Maroon}{\Yup}} 2 \Xint{\begin{smallmatrix} ~ \\ {}^{}_{\mathcolor{gray}{-}} \\ ~ \end{smallmatrix}}{15}{k}_{\;\! \symup{z}}^{\;\! \mathcolor{gray}{\omega} \textcolor{PineGreen}{\pm}} \mathbb{e}^{\mathbb{i} \Xint{\begin{smallmatrix} ~ \\ {}^{}_{\mathcolor{gray}{-}} \\ ~ \end{smallmatrix}}{15}{k}_{\symup{z}}^{\;\! \mathcolor{gray}{\omega} \textcolor{PineGreen}{\pm}} \mathcolor{gray}{z}}} ~,  \label{eq:simplify8-scalar-g-conjugate}
\end{align}
\end{subequations}
注意,\bref{eq:simplify8-scalar-g-conjugate} 仍属于矢量\textcolor{Plum}{非线性}波动方程,尽管方程左侧是\textcolor{PineGreen}{复振幅}标量场 $\Xint{\begin{smallmatrix} ~ \\ {}^{}_{\mathcolor{gray}{-}} \\ ~ \end{smallmatrix}}{09}{\mathtt{g}}^{\;\!\mathcolor{gray}{\omega} \textcolor{PineGreen}{\pm}}_{\;\! \mathcolor{gray}{z}}$ 的 $\mathcolor{gray}{z}$ 向偏导:因为一方面,右侧的 $\Xint{\mathcolor{gray}{-}}{25}{\bar{P}}^{\;\! \mathcolor{gray}{\omega} \textcolor{PineGreen}{\pm}}_{\;\! \textcolor{Maroon}{\Yup} \mathcolor{gray}{z}}$ 是矢量的;另一方面,\textcolor{PineGreen}{复振幅}标量场 $\Xint{\begin{smallmatrix} ~ \\ {}^{}_{\mathcolor{gray}{-}} \\ ~ \end{smallmatrix}}{09}{\mathtt{g}}^{\;\!\mathcolor{gray}{\omega} \textcolor{PineGreen}{\pm}}_{\;\! \mathcolor{gray}{z}}$ 一旦已知,再乘以\textcolor{PineGreen}{本征偏振态} $\Xint{{}^{}_{\mathcolor{gray}{-}}}{10}{\hat{g}}^{\;\!\mathcolor{gray}{\omega} \textcolor{PineGreen}{\pm}}_{\;\! \textcolor{Maroon}{\Yup}}$ 后,可通过  直接转换为矢量\textcolor{Maroon}{时空谱}三分量 $\Xint{{}^{}_{\mathcolor{gray}{-}}}{10}{\bar{g}}^{\;\!\mathcolor{gray}{\omega} \textcolor{PineGreen}{\pm}}_{\;\! \textcolor{Maroon}{\Yup} \mathcolor{gray}{z}} := \Xint{\begin{smallmatrix} ~ \\ {}^{}_{\mathcolor{gray}{-}} \\ ~ \end{smallmatrix}}{09}{\mathtt{g}}^{\;\!\mathcolor{gray}{\omega} \textcolor{PineGreen}{\pm}}_{\;\! \mathcolor{gray}{z}} \Xint{{}^{}_{\mathcolor{gray}{-}}}{10}{\hat{g}}^{\;\!\mathcolor{gray}{\omega} \textcolor{PineGreen}{\pm}}_{\;\! \textcolor{Maroon}{\Yup}}$,或者通过 \bref{eq:vec-eigenmode_amp-matrix} 一步到位得到电矢量场\textcolor{Maroon}{傅立叶谱} $\Xint{\mathcolor{gray}{-}}{25}{\bar{E}}^{\;\!\mathcolor{gray}{\omega}}_{\;\! \mathcolor{gray}{z}}$。

对于以\textcolor{NavyBlue}{脉冲光}\textcolor{Maroon}{倍频}为主\Footnote{若 $\mathcolor{gray}{\omega}_{\textcolor{Maroon}{\text{P}}}, \mathcolor{gray}{\omega}$ 分别在单个\textcolor{NavyBlue}{泵浦光}脉冲\textcolor{gray}{中心频率} $\mathcolor{gray}{\Omega}_{\textcolor{Maroon}{\text{P}}} = \mathcolor{gray}{\Omega} \big/ 2$ 及其产生的 $2\mathcolor{gray}{\omega}_{\textcolor{Maroon}{\text{P}}}$ \textcolor{Maroon}{倍频}光脉冲\textcolor{gray}{中心频率} $\mathcolor{gray}{\Omega} = 2\mathcolor{gray}{\Omega}_{\textcolor{Maroon}{\text{P}}}$ 附近,且 $\mathcolor{gray}{\omega} = 2\mathcolor{gray}{\omega}_{\textcolor{Maroon}{\text{P}}} > 0$,则该式代表单\textcolor{NavyBlue}{脉冲光}\textcolor{Maroon}{倍频}过程。}、以\textcolor{NavyBlue}{脉冲}\textcolor{Maroon}{光整流}后续级联\textcolor{Maroon}{电光效应}\cite{jangMulticycleTerahertzPulse2020}为辅\Footnote{若 $\mathcolor{gray}{\omega}, \mathcolor{gray}{\omega}_{\textcolor{Maroon}{\text{THz}}}$ 分别在单个\textcolor{NavyBlue}{泵浦}光脉冲\textcolor{gray}{中心频率} $\mathcolor{gray}{\Omega} \gg \mathcolor{gray}{\Omega}_{\textcolor{Maroon}{\text{THz}}}$ 及其产生的 \textcolor{Maroon}{THz} 脉冲的\textcolor{gray}{中心频率} $\mathcolor{gray}{\Omega}_{\textcolor{Maroon}{\text{THz}}} \ll \mathcolor{gray}{\Omega}$ 附近,且 $\mathcolor{gray}{\omega} \gg \mathcolor{gray}{\omega}_{\textcolor{Maroon}{\text{THz}}} > 0$,则该式代表\textcolor{NavyBlue}{脉冲}\textcolor{Maroon}{光整流}后续级联\textcolor{Maroon}{电光效应}过程。}的 $\mathcolor{gray}{\omega'} + \left( \mathcolor{gray}{\omega}-\mathcolor{gray}{\omega'} \right) \to \mathcolor{gray}{\omega} > 0$\Footnote{在映射到物理过程时,默认\textcolor{gray}{各频率}为正;但在\textcolor{Plum}{数学积分}中可为负。}二阶\textcolor{Plum}{非线性}\textcolor{gray}{频率}\textcolor{Maroon}{上转换}过程,波动方程 \bref{eq:simplify7-scalar-g} 与 \bref{eq:simplify8-scalar-g} 右侧\textcolor{Plum}{非线性}波源项 $\Xint{\mathcolor{gray}{-}}{25}{\bar{P}}^{\;\! \mathcolor{gray}{\omega} \textcolor{PineGreen}{\pm}}_{\;\! \textcolor{Maroon}{\Yup} \mathcolor{gray}{z}} = \mathcal F \left[ \bar{P}^{\;\! \mathcolor{gray}{\omega} \textcolor{PineGreen}{\pm}}_{\;\! \textcolor{Maroon}{\Yup} \mathcolor{gray}{z}} \right]$ 进一步限定为 \bref{eq:vec-DP^(2)-plane_wave_basis-p_pp} 的“\textcolor{PineGreen}{双模}” $\textcolor{PineGreen}{\pm}$ 版本
\begin{subequations} \label{eq:DP^(2)-plane_wave_basis-pm_pmpm}
\begin{align}
	\Xint{\mathcolor{gray}{-}}{30}{\bar{P}}^{\;\! \mathcolor{gray}{\omega} \textcolor{PineGreen}{\pm}}_{\;\! \mathcolor{gray}{z} \textcolor{Maroon}{(2)}} &= \Xint{{}^{}_{\mathcolor{gray}{-}}}{23}{\bar{\bar{\bar{\chi}}}}^{\;\! \mathcolor{gray}{\omega} \textcolor{PineGreen}{\pm \pm \pm}}_{\mathcolor{gray}{z} \textcolor{Maroon}{(2)}} ~{}^{\mathcolor{gray}{*}}_{\mathcolor{gray}{*}} \left( \Xint{\mathcolor{gray}{-}}{295}{\bar{E}}^{\;\!\mathcolor{gray}{\omega}}_{\;\! \mathcolor{gray}{z} \textcolor{PineGreen}{\pm} } ~\mathcolor{gray}{\widetilde \circledast}~ \Xint{\mathcolor{gray}{-}}{295}{\bar{E}}^{\;\!\mathcolor{gray}{\omega}}_{\;\! \mathcolor{gray}{z} \textcolor{PineGreen}{\pm} } \right) ~, \label{eq:vec-DP^(2)-plane_wave_basis-pm_pmpm} \\
	\Xint{\mathcolor{gray}{-}}{30}{P}^{\;\! \mathcolor{gray}{\omega} \textcolor{PineGreen}{\pm}}_{\;\! \hat{3}\mathcolor{gray}{z} \textcolor{Maroon}{(2)}} &= \Xint{{}^{}_{\mathcolor{gray}{-}}}{23}{\chi}^{\;\! \mathcolor{gray}{\omega} \hat{1} \hat{2} \textcolor{PineGreen}{\pm}}_{\;\! \hat{3} \mathcolor{gray}{z} \textcolor{Maroon}{(2)} \textcolor{PineGreen}{\pm \pm}} \mathcolor{gray}{*} \left( \Xint{\mathcolor{gray}{-}}{295}{E}^{\;\!\mathcolor{gray}{\omega} \textcolor{PineGreen}{\pm}}_{\;\! \hat{1} \mathcolor{gray}{z}} ~\mathcolor{gray}{\widetilde \circledast}~ \Xint{\mathcolor{gray}{-}}{295}{E}^{\;\!\mathcolor{gray}{\omega} \textcolor{PineGreen}{\pm}}_{\;\! \hat{2} \mathcolor{gray}{z}} \right) ~. \label{eq:components-DP^(2)-plane_wave_basis-pm_pmpm}
\end{align}
\end{subequations}
上式 \bref{eq:DP^(2)-plane_wave_basis-pm_pmpm} 也可简写作\textcolor{Plum}{更现代}的形式(最大程度\textcolor{Plum}{省略符号}的同时\textcolor{Plum}{保留全信息})
\begin{subequations} \label{eq:DP^(2)-plane_wave_basis-3_12-spectrum}
\begin{align}
	\Xint{\mathcolor{gray}{-}}{30}{\bar{P}}^{\;\! \mathcolor{gray}{\omega} \textcolor{Maroon}{(2)} }_{\;\! \mathcolor{gray}{z} \textcolor{PineGreen}{\hat{3}}} &= \Xint{{}^{}_{\mathcolor{gray}{-}}}{23}{\bar{\bar{\bar{\chi}}}}^{\;\! \mathcolor{gray}{\omega} \textcolor{Maroon}{(2)}}_{\mathcolor{gray}{z} \textcolor{PineGreen}{\hat{3} \hat{1} \hat{2}} } ~{}^{\mathcolor{gray}{*}}_{\mathcolor{gray}{*}} \left( \Xint{\mathcolor{gray}{-}}{295}{\bar{E}}^{\;\! \mathcolor{gray}{\omega} \textcolor{PineGreen}{\hat{1}} }_{\;\! \mathcolor{gray}{z} } ~\mathcolor{gray}{\widetilde \circledast}~ \Xint{\mathcolor{gray}{-}}{295}{\bar{E}}^{\;\! \mathcolor{gray}{\omega} \textcolor{PineGreen}{\hat{2}} }_{\;\! \mathcolor{gray}{z} } \right) ~, \label{eq:vec-DP^(2)-plane_wave_basis-3_12-spectrum} \\
	\Xint{\mathcolor{gray}{-}}{30}{P}^{\;\! \textcolor{PineGreen}{\hat{3}} \mathcolor{gray}{\omega} }_{\;\! \hat{3}\mathcolor{gray}{z} \textcolor{Maroon}{(2)} } &= \Xint{{}^{}_{\mathcolor{gray}{-}}}{23}{\chi}^{\;\! \textcolor{PineGreen}{\hat{3}} \mathcolor{gray}{\omega} \hat{1} \hat{2} }_{\;\! \hat{3} \mathcolor{gray}{z} \textcolor{PineGreen}{\hat{1} \hat{2}} \textcolor{Maroon}{(2)}} \mathcolor{gray}{*} \left( \Xint{\mathcolor{gray}{-}}{295}{E}^{\;\! \textcolor{PineGreen}{\hat{1}} \mathcolor{gray}{\omega} }_{\;\! \hat{1} \mathcolor{gray}{z}} ~\mathcolor{gray}{\widetilde \circledast}~ \Xint{\mathcolor{gray}{-}}{295}{E}^{\;\! \textcolor{PineGreen}{\hat{2}} \mathcolor{gray}{\omega} }_{\;\! \hat{2} \mathcolor{gray}{z}} \right) ~. \label{eq:components-DP^(2)-plane_wave_basis-3_12-spectrum}
\end{align}
\end{subequations}
同时,矢量\textcolor{Plum}{非线性}波动方程 \bref{eq:simplify8-scalar-g-conjugate} 也简写作
\begin{align}
	\mathcolor{gray}{\nabla_z} \Xint{\begin{smallmatrix} ~ \\ {}^{}_{\mathcolor{gray}{-}} \\ ~ \end{smallmatrix}}{09}{\mathtt{g}}^{\;\!\mathcolor{gray}{\omega} \textcolor{PineGreen}{\hat{3}}}_{\;\! \mathcolor{gray}{z}} &= \mathbb{i} k_{\textcolor{Maroon}{\mathsf{o}} \mathcolor{gray}{\omega}}^{\;\! 2} \frac{\Xint{{}^{}_{\mathcolor{gray}{-}}}{10}{\hat{g}}^{\;\! \textcolor{PineGreen}{\hat{3}} \textcolor{Plum}{\dag}}_{\;\! \mathcolor{gray}{\omega}} \cdot \Xint{\mathcolor{gray}{-}}{30}{\bar{P}}^{\;\! \mathcolor{gray}{\omega} \textcolor{PineGreen}{\hat{3}} }_{\;\! \mathcolor{gray}{z}  \textcolor{Maroon}{(2)}}}{ 2 \lvert \Xint{{}^{}_{\mathcolor{gray}{-}}}{10}{\hat{g}}^{\;\! \textcolor{PineGreen}{\hat{3}}}_{\;\! \mathcolor{gray}{\omega}} \rvert^2 \Xint{\begin{smallmatrix} ~ \\ {}^{}_{\mathcolor{gray}{-}} \\ ~ \end{smallmatrix}}{15}{k}_{\;\! \symup{z}}^{\;\! \mathcolor{gray}{\omega} \textcolor{PineGreen}{\hat{3}}} \mathbb{e}^{\mathbb{i} \Xint{\begin{smallmatrix} ~ \\ {}^{}_{\mathcolor{gray}{-}} \\ ~ \end{smallmatrix}}{15}{k}_{\symup{z}}^{\;\! \mathcolor{gray}{\omega} \textcolor{PineGreen}{\hat{3}}} \mathcolor{gray}{z}}} ~,  \label{eq:simplify8-scalar-g-modulus}
\end{align}

设二阶\textcolor{Plum}{非线性}系数 $\Xint{{}^{}_{\mathcolor{gray}{-}}}{23}{\chi}^{\;\! \mathcolor{gray}{\omega} \hat{1} \hat{2} }_{\;\! \hat{3} \mathcolor{gray}{z} \textcolor{Maroon}{(2)} }$ 可分解为\textcolor{Plum}{均匀背景} ${\chi}^{\;\! \mathcolor{gray}{\omega} \hat{1} \hat{2} }_{\;\! \hat{3} \textcolor{Maroon}{(2)} }$ 与\textcolor{Plum}{调制函数} $\Xint{\mathcolor{gray}{-}}{18}{M}^{\;\! \mathcolor{gray}{\omega} \hat{1} \hat{2} }_{\;\! \hat{3} \mathcolor{gray}{z} \textcolor{Maroon}{(2)} }$ 之积
%\Footnote{当不存在“\textcolor{PineGreen}{模式}”\textcolor{Plum}{角标},以\textcolor{Plum}{推断}\textcolor{NavyBlue}{场量}的\textcolor{Plum}{自变量}时,不能省略 $\mathcolor{gray}{\omega}$ \textcolor{Plum}{角标}。}
\begin{subequations} \label{eq:chi2-modulate}
\begin{align}
	\Xint{{}^{}_{\mathcolor{gray}{-}}}{23}{\bar{\bar{\bar{\chi}}}}^{\;\! \mathcolor{gray}{\omega} }_{\;\! \mathcolor{gray}{z} \textcolor{Maroon}{(2)} } &= {\chi}^{\;\! \mathcolor{gray}{\omega} }_{\;\! \textcolor{Maroon}{(2)} } \odot \Xint{\mathcolor{gray}{-}}{18}{\bar{\bar{\bar{M}}}}^{\;\! \mathcolor{gray}{\omega} }_{\;\! \mathcolor{gray}{z} \textcolor{Maroon}{(2)} } ~, \label{eq:vec-eq:chi2-modulate} \\
	\Xint{{}^{}_{\mathcolor{gray}{-}}}{23}{\chi}^{\;\! \mathcolor{gray}{\omega} \hat{1} \hat{2} }_{\;\! \hat{3} \mathcolor{gray}{z} \textcolor{Maroon}{(2)} } &= {\chi}^{\;\! \mathcolor{gray}{\omega} \hat{1} \hat{2} }_{\;\! \hat{3} \textcolor{Maroon}{(2)} } \Xint{\mathcolor{gray}{-}}{18}{M}^{\;\! \mathcolor{gray}{\omega} \hat{1} \hat{2} }_{\;\! \hat{3} \mathcolor{gray}{z} \textcolor{Maroon}{(2)} } ~, \label{eq:components-eq:chi2-modulate}
\end{align}
\end{subequations}
这里隐式地定义了\textcolor{Plum}{哈达马积}/\textcolor{Plum}{对应元素积} $\odot$ 或 $"{.\cdot}"$,类似于 matlab 的 $"{.*}"$ 语法(矩阵对应元素相乘)。注意,三阶\textcolor{Plum}{调制张量}\textcolor{NavyBlue}{场} $\Xint{\mathcolor{gray}{-}}{18}{\bar{\bar{\bar{M}}}}^{\;\! \mathcolor{gray}{\omega} }_{\;\! \mathcolor{gray}{z} \textcolor{Maroon}{(2)} }$ 像 ${\chi}^{\;\! \mathcolor{gray}{\omega} }_{\;\! \textcolor{Maroon}{(2)} }$(\textcolor{NavyBlue}{非场},没有\textcolor{gray}{“$-$”标志})一样,仍是\textcolor{gray}{波长} $\mathcolor{gray}{\lambda}$ 的函数,即仍是 $\mathcolor{gray}{\omega}$ \textcolor{NavyBlue}{色散}的。但分离出的定常\textcolor{Plum}{均匀背景}张量 ${\chi}^{\;\! \mathcolor{gray}{\omega} }_{\;\! \textcolor{Maroon}{(2)} }$ 因子,不再是 $\mathcolor{gray}{\bar{r}}$ 的函数,并且可以从许多\textcolor{NavyBlue}{实验主导}的文献中获得\cite{nyePhysicalPropertiesCrystals2012,zuOpticalSecondHarmonic2024,zuAnalyticalNumericalModeling2022,gananyQuasiphaseMatchingLiNbO32006,segondsLinearNonlinearOptical2004,dolevLinearNonlinearOptical2009,kaschkeCalculationNonlinearOptical1989,itoGeneralizedStudyAngular1975}。

将 $\mathcolor{gray}{\bar{r}}$ 域上\textcolor{Plum}{被调制}的二阶\textcolor{Plum}{非线性}系数 \bref{eq:components-eq:chi2-modulate} 代入\textcolor{Plum}{非线性}波源 \bref{eq:components-DP^(2)-plane_wave_basis-3_12-spectrum} 得
\begin{subequations} \label{eq:DP^(2)-plane_wave_basis-3_12-spectrum-C}
\begin{align}
	\Xint{\mathcolor{gray}{-}}{30}{P}^{\;\! \textcolor{PineGreen}{\hat{3}} \mathcolor{gray}{\omega} }_{\;\! \hat{3}\mathcolor{gray}{z} \textcolor{Maroon}{(2)} } &= \Xint{{}^{}_{\mathcolor{gray}{-}}}{23}{\chi}^{\;\! \textcolor{PineGreen}{\hat{3}} \mathcolor{gray}{\omega} \hat{1} \hat{2} }_{\;\! \hat{3} \mathcolor{gray}{z} \textcolor{PineGreen}{\hat{1} \hat{2}} \textcolor{Maroon}{(2)}} \mathcolor{gray}{*} \left( \Xint{\mathcolor{gray}{-}}{295}{E}^{\;\!\textcolor{PineGreen}{\hat{1}} \mathcolor{gray}{\omega}}_{\;\! \hat{1} \mathcolor{gray}{z}} ~\mathcolor{gray}{\widetilde \circledast}~ \Xint{\mathcolor{gray}{-}}{295}{E}^{\;\!\textcolor{PineGreen}{\hat{1}} \mathcolor{gray}{\omega}}_{\;\! \hat{2} \mathcolor{gray}{z}} \right) ~, \label{eq:DP^(2)-3_12-spectrum-C1} \\
	&= {\chi}^{\;\! \textcolor{PineGreen}{\hat{3}} \mathcolor{gray}{\omega} \hat{1} \hat{2} }_{\;\! \hat{3} \mathcolor{gray}{z} \textcolor{PineGreen}{\hat{1} \hat{2}} \textcolor{Maroon}{(2)}} \Xint{\mathcolor{gray}{-}}{18}{M}^{\;\! \mathcolor{gray}{\omega} \hat{1} \hat{2} }_{\;\! \hat{3} \mathcolor{gray}{z} \textcolor{Maroon}{(2)} } \mathcolor{gray}{*} \left( \Xint{\mathcolor{gray}{-}}{295}{E}^{\;\!\textcolor{PineGreen}{\hat{1}} \mathcolor{gray}{\omega}}_{\;\! \hat{1} \mathcolor{gray}{z}} ~\mathcolor{gray}{\widetilde \circledast}~ \Xint{\mathcolor{gray}{-}}{295}{E}^{\;\!\textcolor{PineGreen}{\hat{1}} \mathcolor{gray}{\omega}}_{\;\! \hat{2} \mathcolor{gray}{z}} \right) ~, \label{eq:DP^(2)-3_12-spectrum-C2} \\
	&= {\chi}^{\;\! \textcolor{PineGreen}{\hat{3}} \mathcolor{gray}{\omega} \hat{1} \hat{2} }_{\;\! \hat{3} \mathcolor{gray}{z} \textcolor{PineGreen}{\hat{1} \hat{2}} \textcolor{Maroon}{(2)}} \mathcolor{gray}{\mathcal F} \left[ M^{\;\! \mathcolor{gray}{\omega} \hat{1} \hat{2} }_{\;\! \hat{3} \mathcolor{gray}{z} \textcolor{Maroon}{(2)} } \right] \mathcolor{gray}{*} \left( \Xint{\mathcolor{gray}{-}}{295}{E}^{\;\!\textcolor{PineGreen}{\hat{1}} \mathcolor{gray}{\omega}}_{\;\! \hat{1} \mathcolor{gray}{z}} ~\mathcolor{gray}{\widetilde \circledast}~ \Xint{\mathcolor{gray}{-}}{295}{E}^{\;\!\textcolor{PineGreen}{\hat{1}} \mathcolor{gray}{\omega}}_{\;\! \hat{2} \mathcolor{gray}{z}} \right) ~, \label{eq:DP^(2)-3_12-spectrum-C3} \\
	&= {\chi}^{\;\! \textcolor{PineGreen}{\hat{3}} \mathcolor{gray}{\omega} \hat{1} \hat{2} }_{\;\! \hat{3} \mathcolor{gray}{z} \textcolor{PineGreen}{\hat{1} \hat{2}} \textcolor{Maroon}{(2)}} \mathcolor{gray}{\mathcal F_{z}^{-1}} \left[ \mathcolor{gray}{\mathcal F_{\bar{k}}} \left[ M^{\;\! \mathcolor{gray}{\omega} \hat{1} \hat{2} }_{\;\! \hat{3} \mathcolor{gray}{z} \textcolor{Maroon}{(2)} } \right] \right] \mathcolor{gray}{*} \left( \Xint{\mathcolor{gray}{-}}{295}{E}^{\;\!\textcolor{PineGreen}{\hat{1}} \mathcolor{gray}{\omega}}_{\;\! \hat{1} \mathcolor{gray}{z}} ~\mathcolor{gray}{\widetilde \circledast}~ \Xint{\mathcolor{gray}{-}}{295}{E}^{\;\!\textcolor{PineGreen}{\hat{1}} \mathcolor{gray}{\omega}}_{\;\! \hat{2} \mathcolor{gray}{z}} \right) ~, \label{eq:DP^(2)-3_12-spectrum-C4} \\
	&= {\chi}^{\;\! \textcolor{PineGreen}{\hat{3}} \mathcolor{gray}{\omega} \hat{1} \hat{2} }_{\;\! \hat{3} \mathcolor{gray}{z} \textcolor{PineGreen}{\hat{1} \hat{2}} \textcolor{Maroon}{(2)}} \mathcolor{gray}{\mathcal F_{z}^{-1}} \left[ \mathcolor{gray}{\mathcal F_{\bar{k}}} \left[ M^{\;\! \mathcolor{gray}{\omega} \hat{1} \hat{2} }_{\;\! \hat{3} \mathcolor{gray}{z} \textcolor{Maroon}{(2)} } \right] \mathcolor{gray}{*} \left( \Xint{\mathcolor{gray}{-}}{295}{E}^{\;\!\textcolor{PineGreen}{\hat{1}} \mathcolor{gray}{\omega}}_{\;\! \hat{1} \mathcolor{gray}{z}} ~\mathcolor{gray}{\widetilde \circledast}~ \Xint{\mathcolor{gray}{-}}{295}{E}^{\;\!\textcolor{PineGreen}{\hat{1}} \mathcolor{gray}{\omega}}_{\;\! \hat{2} \mathcolor{gray}{z}} \right) \right] ~, \label{eq:DP^(2)-3_12-spectrum-C5} \\
	&=: {\chi}^{\;\! \textcolor{PineGreen}{\hat{3}} \mathcolor{gray}{\omega} \hat{1} \hat{2} }_{\;\! \hat{3} \mathcolor{gray}{z} \textcolor{PineGreen}{\hat{1} \hat{2}} \textcolor{Maroon}{(2)}} \mathcolor{gray}{\mathcal F_{z}^{-1}} \left[ \Xint{\mathcolor{gray}{-}}{18}{M}^{\;\! \mathcolor{gray}{\omega} \hat{1} \hat{2} }_{\;\! \hat{3} \mathcolor{gray}{k_{\symup{z}}} \textcolor{Maroon}{(2)} } \mathcolor{gray}{*} \left( \Xint{\mathcolor{gray}{-}}{295}{E}^{\;\!\textcolor{PineGreen}{\hat{1}} \mathcolor{gray}{\omega}}_{\;\! \hat{1} \mathcolor{gray}{z}} ~\mathcolor{gray}{\widetilde \circledast}~ \Xint{\mathcolor{gray}{-}}{295}{E}^{\;\!\textcolor{PineGreen}{\hat{1}} \mathcolor{gray}{\omega}}_{\;\! \hat{2} \mathcolor{gray}{z}} \right) \right] ~, \label{eq:DP^(2)-3_12-spectrum-C6}
\end{align}
\end{subequations}
其中,定义了三维 $\mathcolor{gray}{\bar{k}}$ \textcolor{gray}{空间}的\textcolor{Maroon}{倒格波系数}(关于 $\mathcolor{gray}{\bar{k}} \asymp \left( \mathcolor{gray}{\bar{k}_{\symup{\rho}}}, \mathcolor{gray}{k_{\symup{z}}} \right)$ 的三阶\textcolor{Plum}{张量}\textcolor{NavyBlue}{场})
\begin{subequations} \label{eq:C}
\begin{align}
	\Xint{\mathcolor{gray}{-}}{18}{M}^{\;\! \mathcolor{gray}{\omega} \hat{1} \hat{2} }_{\;\! \hat{3} \mathcolor{gray}{k_{\symup{z}}} \textcolor{Maroon}{(2)} } &= \mathcolor{gray}{\mathcal F_{\bar{k}}} \left[ M^{\;\! \mathcolor{gray}{\omega} \hat{1} \hat{2} }_{\;\! \hat{3} \mathcolor{gray}{z} \textcolor{Maroon}{(2)} } \right] ~, \label{eq:vec-C} \\
	\Xint{\mathcolor{gray}{-}}{18}{\bar{\bar{\bar{M}}}}^{\;\! \mathcolor{gray}{\omega} }_{\;\! \mathcolor{gray}{k_{\symup{z}}} \textcolor{Maroon}{(2)} } &= \mathcolor{gray}{\mathcal F_{\bar{k}}} \left[ \bar{\bar{\bar{M}}}^{\;\! \mathcolor{gray}{\omega} }_{\;\! \mathcolor{gray}{z} \textcolor{Maroon}{(2)} } \right] ~, \label{eq:components-C}
\end{align}
\end{subequations}
其中,三维空域 $\mathcolor{gray}{\bar{r}} \in \mathcolor{gray}{\bar{\mathbb{R}}_{\textcolor{Plum}{3}}}$ 中的\textcolor{Plum}{傅立叶正变换} $\mathcolor{gray}{\mathcal F_{\bar{k}}}$ 来自 \bref{eq:FT-k}。

利用第 4 种\textcolor{PineGreen}{本征波}定义 \bref{eq:vec-eigenwave'},将 \bref{eq:DP^(2)-3_12-spectrum-C6} 分离出\textcolor{PineGreen}{含衍射复振幅} $\Xint{\mathcolor{gray}{-}}{16}{\mathtt{G}}^{\;\!\mathcolor{gray}{\omega} \textcolor{PineGreen}{\pm}}_{\;\! \mathcolor{gray}{z}}$(\bref{eq:amp_phase})和\textcolor{PineGreen}{本征偏振态} $\Xint{{}^{}_{\mathcolor{gray}{-}}}{10}{\bar{g}}^{\;\!\mathcolor{gray}{\omega}}_{\;\! \textcolor{PineGreen}{\jmath}}$

对于\textcolor{NavyBlue}{非脉冲}/\textcolor{NavyBlue}{连续谱},而是两个独立\textcolor{gray}{单色波长}的\textcolor{Maroon}{和频}\Footnote{尽管\textcolor{NavyBlue}{双泵浦}的\textcolor{NavyBlue}{光强}可能不大,这里仍不说“\textcolor{Maroon}{上转换}”:因为在本文的语境中,“\textcolor{Maroon}{上转换}”过程一般是“\textcolor{NavyBlue}{一强一弱}”双光\textcolor{NavyBlue}{泵浦}生成\textcolor{NavyBlue}{弱} $\mathcolor{gray}{\omega}_{\textcolor{Maroon}{3}}$,以至于参与\textcolor{gray}{混频}的三波中,有两束光不满足\textcolor{Maroon}{泵浦未耗尽近似}条件,因此只要有“\textcolor{Maroon}{上转换}”则必有“\textcolor{Maroon}{下转换}”过程发生(\textcolor{NavyBlue}{能量}从 $\mathcolor{gray}{\omega}_{\textcolor{Maroon}{3}}$ 回流到其中一个\textcolor{NavyBlue}{弱泵浦}中),于是不可避免地涉及\textcolor{NavyBlue}{三波混频}\textcolor{Maroon}{时空谱}耦合波方程组中的至少 2 个方程,然而这里只给出了 1 个“\textcolor{Maroon}{上转换}”过程的方程,因此这里只能代表/指\textcolor{Maroon}{和频}过程。}或\textcolor{Maroon}{上转换}出\textcolor{gray}{第三个波长}的 $\mathcolor{gray}{\omega}_{\textcolor{Maroon}{1}} + \mathcolor{gray}{\omega}_{\textcolor{Maroon}{2}} \to \mathcolor{gray}{\omega}_{\textcolor{Maroon}{3}}$ 过程,以该\textcolor{Plum}{离散}的特殊情况为例,\bref{eq:DP^(2)-plane_wave_basis-pm_pmpm} 变为
\begin{subequations} \label{eq:DP^(2)-plane_wave_basis-3_12}
\begin{align}
	\Xint{\mathcolor{gray}{-}}{30}{\bar{P}}^{\;\! \textcolor{Maroon}{(2)} }_{\;\! \mathcolor{gray}{z} \textcolor{PineGreen}{\hat{3}}} &= \Xint{{}^{}_{\mathcolor{gray}{-}}}{23}{\bar{\bar{\bar{\chi}}}}^{\;\!  \textcolor{Maroon}{(2)}}_{\mathcolor{gray}{z} \textcolor{PineGreen}{\hat{3} \hat{1} \hat{2}} } ~{}^{\mathcolor{gray}{*}}_{\mathcolor{gray}{*}} \left( \Xint{\mathcolor{gray}{-}}{295}{\bar{E}}^{\;\! \textcolor{PineGreen}{\hat{1}} }_{\;\! \mathcolor{gray}{z} } \mathcolor{gray}{*} \Xint{\mathcolor{gray}{-}}{295}{\bar{E}}^{\;\! \textcolor{PineGreen}{\hat{2}} }_{\;\! \mathcolor{gray}{z} } \right) ~, \label{eq:vec-DP^(2)-plane_wave_basis-3_12} \\
	\Xint{\mathcolor{gray}{-}}{30}{P}^{\;\! \textcolor{PineGreen}{\hat{3}} \textcolor{Maroon}{(2)} }_{\;\! \hat{3}\mathcolor{gray}{z}} &= \Xint{{}^{}_{\mathcolor{gray}{-}}}{23}{\chi}^{\;\! \textcolor{PineGreen}{\hat{3}} \textcolor{Maroon}{(2)} \hat{1} \hat{2} }_{\;\! \hat{3} \mathcolor{gray}{z} \textcolor{PineGreen}{\hat{1} \hat{2}}} \mathcolor{gray}{*} \left( \Xint{\mathcolor{gray}{-}}{295}{E}^{\;\!\textcolor{PineGreen}{\hat{1}}}_{\;\! \hat{1} \mathcolor{gray}{z}} \mathcolor{gray}{*} \Xint{\mathcolor{gray}{-}}{295}{E}^{\;\!\textcolor{PineGreen}{\hat{1}}}_{\;\! \hat{2} \mathcolor{gray}{z}} \right) ~. \label{eq:components-DP^(2)-plane_wave_basis-3_12}
\end{align}
\end{subequations}
其中,每个场量都\textcolor{Plum}{未显含}\textcolor{gray}{角频率},但可以\textcolor{Plum}{推断}出来它们运行在 $\mathcolor{gray}{\omega}$ 域:因为\textcolor{PineGreen}{模式} $\textcolor{PineGreen}{\hat{3}},\textcolor{PineGreen}{\hat{2}},\textcolor{PineGreen}{\hat{1}} = \textcolor{PineGreen}{\pm},\textcolor{PineGreen}{\pm},\textcolor{PineGreen}{\pm}$ 只存在于 $\mathcolor{gray}{\omega}~ (, \mathcolor{gray}{\bar{k}_{\symup{\rho}}})$ 域,在时间 $\mathcolor{gray}{t}~ (, \mathcolor{gray}{\bar{k}_{\symup{\rho}}})$ 域内没有“\textcolor{PineGreen}{模式}”这一说法。
\begin{subequations} \label{eq:DP^(2)-plane_wave_basis-3_12-C}
\begin{align}
	\Xint{\mathcolor{gray}{-}}{30}{P}^{\;\! \textcolor{PineGreen}{\hat{3}} \textcolor{Maroon}{(2)} }_{\;\! \hat{3}\mathcolor{gray}{z}} &= \Xint{{}^{}_{\mathcolor{gray}{-}}}{23}{\chi}^{\;\! \textcolor{PineGreen}{\hat{3}} \textcolor{Maroon}{(2)} \hat{1} \hat{2} }_{\;\! \hat{3} \mathcolor{gray}{z} \textcolor{PineGreen}{\hat{1} \hat{2}}} \mathcolor{gray}{*} \left( \Xint{\mathcolor{gray}{-}}{295}{E}^{\;\!\textcolor{PineGreen}{\hat{1}}}_{\;\! \hat{1} \mathcolor{gray}{z}} \mathcolor{gray}{*} \Xint{\mathcolor{gray}{-}}{295}{E}^{\;\!\textcolor{PineGreen}{\hat{1}}}_{\;\! \hat{2} \mathcolor{gray}{z}} \right) ~, \label{eq:DP^(2)-3_12-C1} \\
	&= {\chi}^{\;\! \textcolor{PineGreen}{\hat{3}} \textcolor{Maroon}{(2)} \hat{1} \hat{2} }_{\;\! \hat{3} \textcolor{PineGreen}{\hat{1} \hat{2}}} \Xint{\mathcolor{gray}{-}}{18}{M}^{\;\! \mathcolor{gray}{\omega} \hat{1} \hat{2} }_{\;\! \hat{3} \mathcolor{gray}{z} \textcolor{Maroon}{(2)} } \mathcolor{gray}{*} \left( \Xint{\mathcolor{gray}{-}}{295}{E}^{\;\!\textcolor{PineGreen}{\hat{1}}}_{\;\! \hat{1} \mathcolor{gray}{z}} \mathcolor{gray}{*} \Xint{\mathcolor{gray}{-}}{295}{E}^{\;\!\textcolor{PineGreen}{\hat{1}}}_{\;\! \hat{2} \mathcolor{gray}{z}} \right) ~, \label{eq:DP^(2)-3_12-C2} \\
	&= {\chi}^{\;\! \textcolor{PineGreen}{\hat{3}} \textcolor{Maroon}{(2)} \hat{1} \hat{2} }_{\;\! \hat{3} \textcolor{PineGreen}{\hat{1} \hat{2}}} \mathcolor{gray}{\mathcal F} \left[ M^{\;\! \mathcolor{gray}{\omega} \hat{1} \hat{2} }_{\;\! \hat{3} \mathcolor{gray}{z} \textcolor{Maroon}{(2)} } \right] \mathcolor{gray}{*} \left( \Xint{\mathcolor{gray}{-}}{295}{E}^{\;\!\textcolor{PineGreen}{\hat{1}}}_{\;\! \hat{1} \mathcolor{gray}{z}} \mathcolor{gray}{*} \Xint{\mathcolor{gray}{-}}{295}{E}^{\;\!\textcolor{PineGreen}{\hat{1}}}_{\;\! \hat{2} \mathcolor{gray}{z}} \right) ~, \label{eq:DP^(2)-3_12-C3} \\
	&= {\chi}^{\;\! \textcolor{PineGreen}{\hat{3}} \textcolor{Maroon}{(2)} \hat{1} \hat{2} }_{\;\! \hat{3} \textcolor{PineGreen}{\hat{1} \hat{2}}} \mathcolor{gray}{\mathcal F_{z}^{-1}} \left[ \mathcolor{gray}{\mathcal F_{\bar{k}}} \left[ M^{\;\! \mathcolor{gray}{\omega} \hat{1} \hat{2} }_{\;\! \hat{3} \mathcolor{gray}{z} \textcolor{Maroon}{(2)} } \right] \right] \mathcolor{gray}{*} \left( \Xint{\mathcolor{gray}{-}}{295}{E}^{\;\!\textcolor{PineGreen}{\hat{1}}}_{\;\! \hat{1} \mathcolor{gray}{z}} \mathcolor{gray}{*} \Xint{\mathcolor{gray}{-}}{295}{E}^{\;\!\textcolor{PineGreen}{\hat{1}}}_{\;\! \hat{2} \mathcolor{gray}{z}} \right) ~, \label{eq:DP^(2)-3_12-C4} \\
	&= {\chi}^{\;\! \textcolor{PineGreen}{\hat{3}} \textcolor{Maroon}{(2)} \hat{1} \hat{2} }_{\;\! \hat{3} \textcolor{PineGreen}{\hat{1} \hat{2}}} \mathcolor{gray}{\mathcal F_{z}^{-1}} \left[ \mathcolor{gray}{\mathcal F_{\bar{k}}} \left[ M^{\;\! \mathcolor{gray}{\omega} \hat{1} \hat{2} }_{\;\! \hat{3} \mathcolor{gray}{z} \textcolor{Maroon}{(2)} } \right] \mathcolor{gray}{*} \left( \Xint{\mathcolor{gray}{-}}{295}{E}^{\;\!\textcolor{PineGreen}{\hat{1}}}_{\;\! \hat{1} \mathcolor{gray}{z}} \mathcolor{gray}{*} \Xint{\mathcolor{gray}{-}}{295}{E}^{\;\!\textcolor{PineGreen}{\hat{1}}}_{\;\! \hat{2} \mathcolor{gray}{z}} \right) \right] ~, \label{eq:DP^(2)-3_12-C5} \\
	&=: {\chi}^{\;\! \textcolor{PineGreen}{\hat{3}} \textcolor{Maroon}{(2)} \hat{1} \hat{2} }_{\;\! \hat{3} \textcolor{PineGreen}{\hat{1} \hat{2}}} \mathcolor{gray}{\mathcal F_{z}^{-1}} \left[ \Xint{\mathcolor{gray}{-}}{18}{M}^{\;\! \mathcolor{gray}{\omega} \hat{1} \hat{2} }_{\;\! \hat{3} \mathcolor{gray}{k_{\symup{z}}} \textcolor{Maroon}{(2)} } \mathcolor{gray}{*} \left( \Xint{\mathcolor{gray}{-}}{295}{E}^{\;\!\textcolor{PineGreen}{\hat{1}}}_{\;\! \hat{1} \mathcolor{gray}{z}} \mathcolor{gray}{*} \Xint{\mathcolor{gray}{-}}{295}{E}^{\;\!\textcolor{PineGreen}{\hat{1}}}_{\;\! \hat{2} \mathcolor{gray}{z}} \right) \right] ~, \label{eq:DP^(2)-3_12-C6}
\end{align}
\end{subequations}

\cite{dregerSecondharmonicGenerationNonlinear1990,zubairyAnalyticApproachSecondharmonic1985}

too many diffs, cannot push to gitee, but github is ok

\cite{katoSecondharmonicGeneration20481986,katoTemperaturetuned90Phasematching1994,brunerTemperaturedependentSellmeierEquation2003,jundtTemperaturedependentSellmeierEquation1997,katoSellmeierThermoopticDispersion2002}