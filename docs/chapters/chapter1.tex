
% based on 北理工博士毕业论文 的 chapter1 https://tex.nju.edu.cn/project/user/10d0ed00-6f46-42fa-8a3a-18270214cfa3/a0ed1339-764d-4cc7-9ac4-d5d05b33da53

\label{第一章开始}
\chapter{绪论}

\section{本论文研究的目的和意义}

屈原:天问。\cite{berryOpticalSingularitiesBirefringent2003,ossikovskiConstitutiveRelationsOptically2021}。

奖励 = 找出该错误的期望耗时 * 期望时薪。

如果收到独立验证的正面反馈,将过程寄给我,也可获得奖励。

不知道这是不是又一个漂亮的黑洞/温柔的良夜,迷途无返。未选择的路。

\section{国内外研究现状及发展趋势}
%\label{sec:***} 可标注label

\subsection{形状记忆聚氨酯的形状记忆机理}
%\label{sec:features}

形状记忆聚合物(SMP)是继形状记忆合金后在80年代发展起来的一种新型形状记忆材料

分别给磁感应场$\bar{B}^{\;\!\mathcolor{gray}{t}}_{\;\!\mathcolor{gray}{z}}$(而非磁场$\bar{H}^{\;\!\mathcolor{gray}{t}}_{\;\!\mathcolor{gray}{z}}$),自由电流源$\bar{J}^{\;\!\mathcolor{gray}{t}}_{\;\!\textcolor{Maroon}{\text{f}}\mathcolor{gray}{z}}$ 以及电位移场$\bar{D}^{\;\!\mathcolor{gray}{t}}_{\;\!\mathcolor{gray}{z}}$,定义了如下 3 个本构关系(\textcolor{Maroon}{\text{Constitutive Relation}} = \textcolor{Maroon}{\text{CR}})。

其一,磁场 $\bar{H}^{\;\!\mathcolor{gray}{t}}_{\;\!\mathcolor{gray}{z}}$ 的本构关系,考虑$\bar{M}^{\;\!\mathcolor{gray}{t}}_{\;\!\mathcolor{gray}{z}}$\Footnote{磁化强度。其在微观上来源于:分子电流(电子轨道运动)产生的磁矩和电子自旋磁矩的矢量和\cite{nelsonLagrangianTreatmentMagnetic1994},细究还将有电子与原子核、电子的自旋轨道耦合、电子电子间相互作用(多电子产生原子磁矩)、原子与原子间(晶体场)的相互作用\cite{chen-zhuChenZhuxieUndergraduate_courses2024}。磁的起源看上去可以是纯电的\cite{lakhtakiaGenesisPostConstraint2004}。}仅由磁偶极矩\Footnote{即只考虑最低阶磁多极矩 = 不考虑磁四极矩及以上\cite{nelsonLagrangianTreatmentMagnetic1994},因为对于受到电磁场的影响后,反过来产生电磁场的电子而言,其受到的电场力是洛伦兹力的c$\big/v$倍\cite{boydNonlinearOptics2019}。此外,(磁/电)多极矩与(磁/电)非线性是互相独立的——即任何阶的多极矩,都有自己的线性项和非线性项\cite{chen-zhuChenZhuxieUndergraduate_courses2024},这些项与其它阶多极矩的任何项都无关。}贡献时,定义为
\begin{subequations} \label{eq:cr-b}
\begin{align}
	\textcolor{Maroon}{\text{CR for magnetism}}\text{:}&\hspace{0.5em} \bar{B}^{\;\!\mathcolor{gray}{t}}_{\;\!\mathcolor{gray}{z}} \hspace{-0.7em} &&={\symup{\varepsilon}}_0 \left( \bar{H}^{\;\!\mathcolor{gray}{t}}_{\;\!\mathcolor{gray}{z}} + \bar{M}^{\;\!\mathcolor{gray}{t}}_{\;\!\mathcolor{gray}{z}} \right) = {\symup{\varepsilon}}_0 \left\{ \bar{\bar{\delta}}^{\;\!\mathcolor{gray}{t}}~\widetilde *~\bar{H}^{\;\!\mathcolor{gray}{t}}_{\;\!\mathcolor{gray}{z}} + \bar{M}^{\;\!\mathcolor{gray}{t}}_{\;\!\mathcolor{gray}{z}} \right\} \label{cr-b1} \\ 
	& &&\xrightarrow[]{\bar{M}^{\;\!\mathcolor{gray}{t}}_{\;\!\mathcolor{gray}{z}} = \bar{M}^{\;\!\textcolor{Maroon}{\text{(1)}} \mathcolor{gray}{t}}_{\;\!\mathcolor{gray}{z}} + \bar{M}^{\;\!\textcolor{Maroon}{\text{NL}}, \mathcolor{gray}{t}}_{\;\!\mathcolor{gray}{z}}} {\symup{\varepsilon}}_0 \left\{ \left[ \bar{\bar{\delta}}^{\;\!\mathcolor{gray}{t}}~\widetilde *~\bar{H}^{\;\!\mathcolor{gray}{t}}_{\;\!\mathcolor{gray}{z}} + \bar{M}^{\;\!\textcolor{Maroon}{\text{(1)}} \mathcolor{gray}{t}}_{\;\!\mathcolor{gray}{z}} \right] + \bar{M}^{\;\!\textcolor{Maroon}{\text{NL}}, \mathcolor{gray}{t}}_{\;\!\mathcolor{gray}{z}} \right\} \label{cr-b2} \\ 
	& &&\xrightarrow[\displaystyle{ \bar{\bar{\mu}}^{\;\!\textcolor{Maroon}{\text{(1)}} \mathcolor{gray}{t}}_{\;\!\textcolor{Maroon}{\text{r}}\mathcolor{gray}{z}} := \bar{\bar{\delta}}^{\;\!\mathcolor{gray}{t}} + \bar{\bar{\chi}}^{\;\!\textcolor{Maroon}{\text{(1)}}\mathcolor{gray}{t}}_{\;\!\textcolor{Maroon}{\text{m}} \mathcolor{gray}{z}}}]{\displaystyle{\bar{M}^{\;\!\textcolor{Maroon}{\text{(1)}} \mathcolor{gray}{t}}_{\;\!\mathcolor{gray}{z}} := \bar{\bar{\chi}}^{\;\!\textcolor{Maroon}{\text{(1)}}\mathcolor{gray}{t}}_{\;\!\textcolor{Maroon}{\text{m}} \mathcolor{gray}{z}} ~\widetilde *~\bar{H}^{\;\!\mathcolor{gray}{t}}_{\;\!\mathcolor{gray}{z}}}} {\symup{\varepsilon}}_0 \left\{ \bar{\bar{\mu}}^{\;\!\textcolor{Maroon}{\text{(1)}} \mathcolor{gray}{t}}_{\;\!\textcolor{Maroon}{\text{r}}\mathcolor{gray}{z}}~\widetilde *~\bar{H}^{\;\!\mathcolor{gray}{t}}_{\;\!\mathcolor{gray}{z}} + \bar{M}^{\;\!\textcolor{Maroon}{\text{NL}}, \mathcolor{gray}{t}}_{\;\!\mathcolor{gray}{z}} \right\} \label{cr-b3} \\ 
	& &&= \bar{\bar{\mu}}^{\;\!\textcolor{Maroon}{\text{(1)}} \mathcolor{gray}{t}}_{\;\!\mathcolor{gray}{z}}~\widetilde *~\bar{H}^{\;\!\mathcolor{gray}{t}}_{\;\!\mathcolor{gray}{z}} + {\symup{\varepsilon}}_0 \bar{M}^{\;\!\textcolor{Maroon}{\text{NL}}, \mathcolor{gray}{t}}_{\;\!\mathcolor{gray}{z}} =: \bar{B}^{\;\!\textcolor{Maroon}{\text{(1)}} \mathcolor{gray}{t}}_{\;\!\mathcolor{gray}{z}} + \bar{B}^{\;\!\textcolor{Maroon}{\text{NL}}, \mathcolor{gray}{t}}_{\;\!\mathcolor{gray}{z}}~, \label{cr-b4}
\end{align}
\end{subequations}
其中,磁通量密度场 $\bar{B}^{\;\!\mathcolor{gray}{t}}_{\;\!\mathcolor{gray}{z}}$(直接/显示地)关于磁场 $\bar{H}^{\;\!\mathcolor{gray}{t}}_{\;\!\mathcolor{gray}{z}}$\Footnote{磁非线性,如郎之万顺磁性理论\cite{chen-zhuChenZhuxieUndergraduate_courses2024}中 $M \propto$ 郎之万函数 $\mathcal{L} \left( \alpha \right) = \coth \left( \alpha \right) - 1 / \alpha$(其中 $\alpha \propto H_{\textcolor{Maroon}{\text{ex}}}$)或其量子化修正之布里渊函数,铁磁体\cite{chen-zhuChenZhuxieUndergraduate_courses2024}或超导体\cite{wenBriefIntroductionFlux2021}中的磁滞现象等(每个时刻$t$,这些场量都是准静态$\Omega \to 0$的)。}、电场 $\bar{E}^{\;\!\mathcolor{gray}{t}}_{\;\!\mathcolor{gray}{z}}$\Footnote{双各向异性,其中的电$\to$磁耦合(如果 $\bar{B}^{\;\!\mathcolor{gray}{t}}_{\;\!\mathcolor{gray}{z}}$ 中的该部分只是 $\bar{E}^{\;\!\mathcolor{gray}{t}}_{\;\!\mathcolor{gray}{z}}$ 的线性函数,则也可归结到线性项中)。}、应力 $\bar{T}^{\;\!\mathcolor{gray}{t}}_{\;\!\mathcolor{gray}{z}}$\Footnote{正逆压磁/磁致伸缩/磁弹效应(这里未作区分)。}等其他场量(即含空 $\mathcolor{gray}{\bar{r}}$ 的物理量)\Footnote{$\bar{B}^{\;\!\mathcolor{gray}{t}}_{\;\!\mathcolor{gray}{z}},\bar{M}^{\;\!\mathcolor{gray}{t}}_{\;\!\mathcolor{gray}{z}}$ 以及各阶 $\bar{\bar{\mu}}^{\;\!\textcolor{Maroon}{\text{(1)}} \mathcolor{gray}{t}}_{\;\!\mathcolor{gray}{z}},\bar{\bar{\bar{\mu}}}^{\;\!\textcolor{Maroon}{\text{(2)}} \mathcolor{gray}{t}}_{\;\!\mathcolor{gray}{z}},\cdots$ 已经是关于温度$T$、波长$\lambda$(或 角频率$\omega$、时间$t$)等(非)场量的函数。}的非线性函数项,悉数包含在 $\bar{B}^{\;\!\textcolor{Maroon}{\text{NL}}, \mathcolor{gray}{t}}_{\;\!\mathcolor{gray}{z}} = {\symup{\varepsilon}}_0 \bar{M}^{\;\!\textcolor{Maroon}{\text{NL}}, \mathcolor{gray}{t}}_{\;\!\mathcolor{gray}{z}}$ 内;剩余的线性项,放在 $\bar{B}^{\;\!\textcolor{Maroon}{\text{(1)}} \mathcolor{gray}{t}}_{\;\!\mathcolor{gray}{z}} = \bar{\bar{\mu}}^{\;\!\textcolor{Maroon}{\text{(1)}} \mathcolor{gray}{t}}_{\;\!\mathcolor{gray}{z}}~\widetilde *~\bar{H}^{\;\!\mathcolor{gray}{t}}_{\;\!\mathcolor{gray}{z}}$ 中。

其二,自由电流源$\bar{J}^{\;\!\mathcolor{gray}{t}}_{\;\!\textcolor{Maroon}{\text{f}}\mathcolor{gray}{z}}$的本构关系,包含欧姆定律的线性部分(漂移项) $\bar{J}^{\;\!\textcolor{Maroon}{\text{(1)}} \mathcolor{gray}{t}}_{\;\!\textcolor{Maroon}{\text{f}}\mathcolor{gray}{z}} = \bar{\bar{\sigma}}^{\;\!\textcolor{Maroon}{\text{(1)}}\mathcolor{gray}{t}}_{\;\!\mathcolor{gray}{z}}~\widetilde *~\bar{E}^{\;\!\mathcolor{gray}{t}}_{\;\!\mathcolor{gray}{z}}$\Footnote{可以由 Drude 模型描述,定量解释一阶电导率$\bar{\bar{\sigma}}^{\;\!\textcolor{Maroon}{\text{(1)}}\mathcolor{gray}{t}}_{\;\!\mathcolor{gray}{z}}$的起源。},及$\bar{J}^{\;\!\mathcolor{gray}{t}}_{\;\!\textcolor{Maroon}{\text{f}}\mathcolor{gray}{z}}$分别关于电场 $\bar{E}^{\;\!\mathcolor{gray}{t}}_{\;\!\mathcolor{gray}{z}}$\Footnote{欧姆定律中的电非线性部分,比如二/三极管的伏安特性曲线\cite{chen-zhuChenZhuxieUndergraduate_courses2024}(尽管输入/输出or自/因变量,即$\bar{E}^{\;\!\mathcolor{gray}{t}}_{\;\!\mathcolor{gray}{z}}$和$\bar{J}^{\;\!\mathcolor{gray}{t}}_{\;\!\textcolor{Maroon}{\text{f}}\mathcolor{gray}{z}}$,一般均在直流或低频$\Omega$,非交流且不在光波段 opt)。}、磁感应场$\bar{B}^{\;\!\mathcolor{gray}{t}}_{\;\!\mathcolor{gray}{z}}$\Footnote{磁感应场$\bar{B}^{\;\!\mathcolor{gray}{t}}_{\;\!\mathcolor{gray}{z}}$所带来的(电场力以外的)洛伦兹力$\left( \bar{J}^{\;\!\mathcolor{gray}{t}}_{\;\!\textcolor{Maroon}{\text{f}}\mathcolor{gray}{z}} + \dot{\bar{P}}^{\;\!\mathcolor{gray}{t}}_{\;\!\mathcolor{gray}{z}} + \mathcolor{gray}{\bar{\nabla} \times} \bar{M}^{\;\!\mathcolor{gray}{t}}_{\;\!\mathcolor{gray}{z}} \right) \times \bar{B}^{\;\!\mathcolor{gray}{t}}_{\;\!\mathcolor{gray}{z}}$\cite{mackayElectromagneticAnisotropyBianisotropy2019,chen-zhuChenZhuxieUndergraduate_courses2024},会影响导/价带电子的运动(速度)$\bar{v}^{\;\!\mathcolor{gray}{t}}_{\;\!\textcolor{Maroon}{\text{f}}\mathcolor{gray}{z}}$,进而全局地影响自由电流$\bar{J}^{\;\!\mathcolor{gray}{t}}_{\;\!\textcolor{Maroon}{\text{f}}\mathcolor{gray}{z}} = {\rho}^{\;\!\mathcolor{gray}{t}}_{\;\!\textcolor{Maroon}{\text{f}}\mathcolor{gray}{z}} \bar{v}^{\;\!\mathcolor{gray}{t}}_{\;\!\textcolor{Maroon}{\text{f}}\mathcolor{gray}{z}}$和(束缚)电(偶)极化强度$\bar{P}^{\;\!\mathcolor{gray}{t}}_{\;\!\mathcolor{gray}{z}}$,包括它们的线性和非线性项\cite{boydNonlinearOptics2019}。对于强场/超快非线性光学,相对论效应使得电磁场是个统一的整体,动生(而不仅是外加)的$\bar{B}^{\;\!\mathcolor{gray}{t}}_{\;\!\mathcolor{gray}{z}}$还将带来额外的影响。磁场$\bar{H}^{\;\!\mathcolor{gray}{t}}_{\;\!\mathcolor{gray}{z}}$对分子/磁化电流体密度$\mathcolor{gray}{\bar{\nabla} \times} \bar{M}^{\;\!\mathcolor{gray}{t}}_{\;\!\mathcolor{gray}{z}}$产生的影响已包含在$\bar{M}^{\;\!\mathcolor{gray}{t}}_{\;\!\mathcolor{gray}{z}}$中了。}、导带电子浓度(数密度)梯度场$\mathcolor{gray}{\bar{\nabla}} {\rho}^{\;\!\mathcolor{gray}{t}}_{\;\!\textcolor{Maroon}{\text{f}}\mathcolor{gray}{z}}$\Footnote{在光折变效应中,作为$\bar{J}^{\;\!\mathcolor{gray}{t}}_{\;\!\textcolor{Maroon}{\text{f}}\mathcolor{gray}{z}}$中的扩散项\cite{boydNonlinearOptics2019}。${\rho}^{\;\!\mathcolor{gray}{t}}_{\;\!\textcolor{Maroon}{\text{f}}\mathcolor{gray}{z}}, \bar{J}^{\;\!\mathcolor{gray}{t}}_{\;\!\textcolor{Maroon}{\text{f}}\mathcolor{gray}{z}}$之间还应满足\bref{eq:Div-e-f}以及$\bar{J}^{\;\!\mathcolor{gray}{t}}_{\;\!\textcolor{Maroon}{\text{f}}\mathcolor{gray}{z}} = {\rho}^{\;\!\mathcolor{gray}{t}}_{\;\!\textcolor{Maroon}{\text{f}}\mathcolor{gray}{z}} \bar{v}^{\;\!\mathcolor{gray}{t}}_{\;\!\textcolor{Maroon}{\text{f}}\mathcolor{gray}{z}}$\cite{chen-zhuChenZhuxieUndergraduate_courses2024}。}和光伏电流场$\propto \lvert \bar{E}^{\;\!\mathcolor{gray}{t}}_{\;\!\mathcolor{gray}{z}} \rvert^2 \hat{c}$\Footnote{与光电导效应并列,属于内光电效应;也可能在光折变效应的$\bar{J}^{\;\!\mathcolor{gray}{t}}_{\;\!\textcolor{Maroon}{\text{f}}\mathcolor{gray}{z}}$中扮演一份角色,特别是沿着一些各向异性晶体的光轴$\hat{c}$产生电势差和内建电场\cite{boydNonlinearOptics2019}(尽管一般也只影响直流或低频$\Omega$的$\bar{J}^{\;\!\mathcolor{gray}{t}}_{\;\!\textcolor{Maroon}{\text{f}}\mathcolor{gray}{z}}$;但$\bar{J}^{\;\!\mathcolor{gray}{t}}_{\;\!\textcolor{Maroon}{\text{f}}\mathcolor{gray}{z}}$会通过影响光波段的介电常数,进而影响光波段的光强$\lvert \bar{E}^{\;\!\mathcolor{gray}{t}}_{\;\!\mathcolor{gray}{z}} \rvert^2$及$\bar{J}^{\;\!\mathcolor{gray}{t}}_{\;\!\textcolor{Maroon}{\text{f}}\mathcolor{gray}{z}}$自己的重新分布);该二阶的带耦合的非线性,看上去很像非线性极化率$\bar{P}^{\;\!\textcolor{Maroon}{\text{(2)}} \mathcolor{gray}{t}}_{\;\!\mathcolor{gray}{z}}$中的光整流项,但其频率比 THz 低,且只服务于自由电流。—— 以至该项可作为差频合并至$\bar{J}^{\;\!\mathcolor{gray}{t}}_{\;\!\textcolor{Maroon}{\text{f}}\mathcolor{gray}{z}}$关于$\bar{E}^{\;\!\mathcolor{gray}{t}}_{\;\!\mathcolor{gray}{z}}$的二阶非线性$\bar{J}^{\;\!\textcolor{Maroon}{\text{(2)}} \mathcolor{gray}{t}}_{\;\!\textcolor{Maroon}{\text{f}}\mathcolor{gray}{z}}$中去?}等其他场量的非线性项 $\bar{J}^{\;\!\textcolor{Maroon}{\text{NL}}, \mathcolor{gray}{t}}_{\;\!\textcolor{Maroon}{\text{f}}\mathcolor{gray}{z}}$:
\begin{align} \label{eq:cr-j}
	\textcolor{Maroon}{\text{Ohm's law}}\text{:}\hspace{0.5em} \bar{J}^{\;\!\mathcolor{gray}{t}}_{\;\!\textcolor{Maroon}{\text{f}}\mathcolor{gray}{z}} = \bar{\bar{\sigma}}^{\;\!\textcolor{Maroon}{\text{(1)}}\mathcolor{gray}{t}}_{\;\!\mathcolor{gray}{z}}~\widetilde *~\bar{E}^{\;\!\mathcolor{gray}{t}}_{\;\!\mathcolor{gray}{z}} + \bar{J}^{\;\!\textcolor{Maroon}{\text{NL}}, \mathcolor{gray}{t}}_{\;\!\textcolor{Maroon}{\text{f}}\mathcolor{gray}{z}} =: \bar{J}^{\;\!\textcolor{Maroon}{\text{(1)}} \mathcolor{gray}{t}}_{\;\!\textcolor{Maroon}{\text{f}}\mathcolor{gray}{z}} + \bar{J}^{\;\!\textcolor{Maroon}{\text{NL}}, \mathcolor{gray}{t}}_{\;\!\textcolor{Maroon}{\text{f}}\mathcolor{gray}{z}}~,
\end{align}

其三,电位移场$\bar{D}^{\;\!\mathcolor{gray}{t}}_{\;\!\mathcolor{gray}{z}}$的本构关系,当$\bar{P}^{\;\!\mathcolor{gray}{t}}_{\;\!\mathcolor{gray}{z}}$只由电偶极矩\Footnote{不考虑电四极矩及以上。但电四极化强度场$\bar{\bar{Q}}^{\;\!\mathcolor{gray}{t}}_{\;\!\mathcolor{gray}{z}}$(的等效电偶极化强度场$\bar{P}^{\;\!\mathcolor{gray}{t}}_{\;\!\textcolor{Maroon}{\text{Q}}\mathcolor{gray}{z}} = - \mathcolor{gray}{\bar{\nabla} \cdot} \bar{\bar{Q}}^{\;\!\mathcolor{gray}{t}}_{\;\!\mathcolor{gray}{z}}$)\cite{chen-zhuChenZhuxieUndergraduate_courses2024}在有些效应中不可忽视且起关键作用:如其对线性晶体光学中的光学活性的贡献\cite{nelsonDerivingTransmissionReflection1995},以及非线性光学中基于$\bar{\bar{Q}}^{\;\!\mathcolor{gray}{t}}_{\;\!\mathcolor{gray}{z}}$的二阶和频\cite{bethuneOpticalQuadrupoleSumfrequency1976}。电四极子对光与物质相互作用的贡献,还会打破$\bar{D}^{\;\!\mathcolor{gray}{t}}_{\;\!\mathcolor{gray}{z}}$法向连续和$\bar{H}^{\;\!\mathcolor{gray}{t}}_{\;\!\mathcolor{gray}{z}}$切向连续边界条件,并与洛伦兹力的定义、(由唯二的无源齐次\cite{lakhtakiaGenesisPostConstraint2004}微分方程\bref{eq:Curl-EK,eq:Div-Bk}导出的)电磁场标/矢势\cite{chen-zhuChenZhuxieUndergraduate_courses2024}等一起,使$\bar{E}^{\;\!\mathcolor{gray}{t}}_{\;\!\mathcolor{gray}{z}},\bar{B}^{\;\!\mathcolor{gray}{t}}_{\;\!\mathcolor{gray}{z}}$而不是$\bar{E}^{\;\!\mathcolor{gray}{t}}_{\;\!\mathcolor{gray}{z}},\bar{H}^{\;\!\mathcolor{gray}{t}}_{\;\!\mathcolor{gray}{z}}$成为基本场\cite{nelsonDerivingTransmissionReflection1995},对应地,坡印亭矢量也需要修正为$\bar{E}^{\;\!\mathcolor{gray}{t}}_{\;\!\mathcolor{gray}{z}} \times \bar{B}^{\;\!\mathcolor{gray}{t}}_{\;\!\mathcolor{gray}{z}} \big/ {\symup{\varepsilon}}_0$\cite{nelsonGeneralizingPoyntingVector1996,loudonPropagationElectromagneticEnergy1997,richterPoyntingsTheoremEnergy2008}而不是$\bar{E}^{\;\!\mathcolor{gray}{t}}_{\;\!\mathcolor{gray}{z}} \times \bar{H}^{\;\!\mathcolor{gray}{t}}_{\;\!\mathcolor{gray}{z}}$。}构成时,定义为
\begin{subequations} \label{eq:cr-d}
\begin{align}
	\textcolor{Maroon}{\text{CR for electricity}}\text{:}&\hspace{0.5em} \bar{D}^{\;\!\mathcolor{gray}{t}}_{\;\!\mathcolor{gray}{z}} \hspace{-2.0em} &&= {\symup{\varepsilon}}_0 \bar{E}^{\;\!\mathcolor{gray}{t}}_{\;\!\mathcolor{gray}{z}} + \bar{P}^{\;\!\mathcolor{gray}{t}}_{\;\!\mathcolor{gray}{z}} = {\symup{\varepsilon}}_0 \bar{\bar{\delta}}^{\;\!\mathcolor{gray}{t}}~\widetilde *~\bar{E}^{\;\!\mathcolor{gray}{t}}_{\;\!\mathcolor{gray}{z}} + \bar{P}^{\;\!\mathcolor{gray}{t}}_{\;\!\mathcolor{gray}{z}} \label{cr-d1} \\ 
	& &&\xrightarrow[]{\bar{P}^{\;\!\mathcolor{gray}{t}}_{\;\!\mathcolor{gray}{z}} = \bar{P}^{\;\!\textcolor{Maroon}{\text{(1)}} \mathcolor{gray}{t}}_{\;\!\mathcolor{gray}{z}} + \bar{P}^{\;\!\textcolor{Maroon}{\text{NL}}, \mathcolor{gray}{t}}_{\;\!\mathcolor{gray}{z}} + } \left[ {\symup{\varepsilon}}_0 \bar{\bar{\delta}}^{\;\!\mathcolor{gray}{t}}~\widetilde *~\bar{E}^{\;\!\mathcolor{gray}{t}}_{\;\!\mathcolor{gray}{z}} + \bar{P}^{\;\!\textcolor{Maroon}{\text{(1)}} \mathcolor{gray}{t}}_{\;\!\mathcolor{gray}{z}} \right] + \bar{P}^{\;\!\textcolor{Maroon}{\text{NL}}, \mathcolor{gray}{t}}_{\;\!\mathcolor{gray}{z}} \label{cr-d2} \\ 
	& &&\xrightarrow[\displaystyle{ \bar{\bar{\varepsilon}}^{\;\!\textcolor{Maroon}{\text{(1)}} \mathcolor{gray}{t}}_{\;\!\textcolor{Maroon}{\text{r}}\mathcolor{gray}{z}} := \bar{\bar{\delta}}^{\;\!\mathcolor{gray}{t}} + \bar{\bar{\chi}}^{\;\!\textcolor{Maroon}{\text{(1)}}\mathcolor{gray}{t}}_{\;\!\textcolor{Maroon}{\text{f}} \mathcolor{gray}{z}}}]{\displaystyle{\bar{P}^{\;\!\textcolor{Maroon}{\text{(1)}} \mathcolor{gray}{t}}_{\;\!\mathcolor{gray}{z}} := \bar{\bar{\chi}}^{\;\!\textcolor{Maroon}{\text{(1)}}\mathcolor{gray}{t}}_{\;\!\textcolor{Maroon}{\text{f}} \mathcolor{gray}{z}} ~\widetilde *~\bar{E}^{\;\!\mathcolor{gray}{t}}_{\;\!\mathcolor{gray}{z}}}} {\symup{\varepsilon}}_0 \bar{\bar{\varepsilon}}^{\;\!\textcolor{Maroon}{\text{(1)}} \mathcolor{gray}{t}}_{\;\!\textcolor{Maroon}{\text{r}}\mathcolor{gray}{z}}~\widetilde *~\bar{E}^{\;\!\mathcolor{gray}{t}}_{\;\!\mathcolor{gray}{z}} + \bar{P}^{\;\!\textcolor{Maroon}{\text{NL}}, \mathcolor{gray}{t}}_{\;\!\mathcolor{gray}{z}} \label{cr-d3} \\ 
	& &&= \bar{\bar{\varepsilon}}^{\;\!\textcolor{Maroon}{\text{(1)}} \mathcolor{gray}{t}}_{\;\!\mathcolor{gray}{z}}~\widetilde *~\bar{E}^{\;\!\mathcolor{gray}{t}}_{\;\!\mathcolor{gray}{z}} + \bar{P}^{\;\!\textcolor{Maroon}{\text{NL}}, \mathcolor{gray}{t}}_{\;\!\mathcolor{gray}{z}} =: \bar{D}^{\;\!\textcolor{Maroon}{\text{(1)}} \mathcolor{gray}{t}}_{\;\!\mathcolor{gray}{z}} + \bar{D}^{\;\!\textcolor{Maroon}{\text{NL}}, \mathcolor{gray}{t}}_{\;\!\mathcolor{gray}{z}}~, \label{cr-d4}
\end{align}
\end{subequations}
关于其组成成分,电位移场 $\bar{D}^{\;\!\mathcolor{gray}{t}}_{\;\!\mathcolor{gray}{z}}$(直接/显示地)关于电场 $\bar{E}^{\;\!\mathcolor{gray}{t}}_{\;\!\mathcolor{gray}{z}}$\Footnote{电非线性,包括高频段的(非)共振非线性、低频低温\cite{lakhtakiaGenesisPostConstraint2004}段的铁电体/畴的电滞现象等。}、磁场 $\bar{H}^{\;\!\mathcolor{gray}{t}}_{\;\!\mathcolor{gray}{z}}$\Footnote{双各向异性中的磁$\to$电耦合(如果 $\bar{D}^{\;\!\mathcolor{gray}{t}}_{\;\!\mathcolor{gray}{z}}$ 中的该部分只是 $\bar{H}^{\;\!\mathcolor{gray}{t}}_{\;\!\mathcolor{gray}{z}}$ 的线性函数,则也可归结到线性项中)。}、应力 $\bar{T}^{\;\!\mathcolor{gray}{t}}_{\;\!\mathcolor{gray}{z}}$\Footnote{正逆压磁/磁致伸缩/磁弹效应(这里未作区分)。}等其他场量的非线性函数项,均由 $\bar{D}^{\;\!\textcolor{Maroon}{\text{NL}}, \mathcolor{gray}{t}}_{\;\!\mathcolor{gray}{z}} = {\symup{\varepsilon}}_0 \bar{M}^{\;\!\textcolor{Maroon}{\text{NL}}, \mathcolor{gray}{t}}_{\;\!\mathcolor{gray}{z}}$ 贡献;剩余的线性项,由 $\bar{B}^{\;\!\textcolor{Maroon}{\text{(1)}} \mathcolor{gray}{t}}_{\;\!\mathcolor{gray}{z}} = \bar{\bar{\mu}}^{\;\!\textcolor{Maroon}{\text{(1)}} \mathcolor{gray}{t}}_{\;\!\mathcolor{gray}{z}}~\widetilde *~\bar{H}^{\;\!\mathcolor{gray}{t}}_{\;\!\mathcolor{gray}{z}}$ 表示。


