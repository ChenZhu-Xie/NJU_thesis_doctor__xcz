
% based on 北理工博士毕业论文 的 chapter1 https://tex.nju.edu.cn/project/user/10d0ed00-6f46-42fa-8a3a-18270214cfa3/a0ed1339-764d-4cc7-9ac4-d5d05b33da53

\label{第一章开始}
\chapter{绪论}

\section{本论文研究的目的和意义}

近年来,随着人们生活水平的不断提高,人们越来越注重周围环境对身体健康的影响。作为服装是人们时时刻刻最贴近的环境,尤其是内衣,对人体健康有很大的影响。由于合时刻刻最贴近的环境,尤其是内衣,对人体健康有很大的影响。由于合成纤维的衣着舒适性、手感性,天然纤维的发展又成为人们关注的一大热点。

……\cite{Takahashi1996Structure,Xia2002Analysis,Jiang1989,Mao2000Motion,Feng1998}

\section{国内外研究现状及发展趋势}
%\label{sec:***} 可标注label

\subsection{形状记忆聚氨酯的形状记忆机理}
%\label{sec:features}

形状记忆聚合物(SMP)是继形状记忆合金后在80年代发展起来的一种新型形状记忆材料\cite{Jiang2005Size}。形状记忆高分子材料在常温范围内具有塑料的性质,即刚性、形状稳定恢复性;同时在一定温度下(所谓记忆温度下)具有橡胶的特性,主要表现为材料的可变形性和形变恢复性。即“记忆初始态-固定变形-恢复起始态”的循环。

固定相只有物理交联结构的聚氨酯称为热塑性SMPU,而有化学交联结构称为热固性SMPU。热塑性和热固性形状记忆聚氨酯的形状记忆原理示意图如图\ref{fig:diagram}所示

\begin{figure}
	\centering
	\includegraphics[width=0.75\textwidth]{figures/figure1}
	% \caption[这里的文字将会显示在 listoffigure 中]{这里的文字将会显示在正文中}
	\caption{热塑性形状记忆聚氨酯的形状记忆机理示意图}\label{fig:diagram}
\end{figure}


\subsection{形状记忆聚氨酯的研究进展}
%\label{sec:requirements}
首例SMPU是日本Mitsubishi公司开发成功的……。

\subsection{水系聚氨酯及聚氨酯整理剂}

水系聚氨酯的形态对其流动性,成膜性及加工织物的性能有重要影响,一般分为三种类型\cite{Jiang2005Size} ,如表 \ref{tab:category}所示。

\begin{table}
	\centering
	\caption{水系聚氨酯分类} \label{tab:category}
	\begin{tabular*}{0.9\textwidth}{@{\extracolsep{\fill}}cccc}
		\toprule
		类别			&水溶型		&胶体分散型		&乳液型 \\
		\midrule
		状态			&溶解$\sim$胶束	&分散		&白浊 \\
		外观			&水溶型		&胶体分散型		&乳液型 \\
		粒径$/\mu m$	&$<0.001$		&$0.001-0.1$		&$>0.1$ \\
		重均分子量	&$1000\sim 10000$	&数千$\sim 20万$ &$>5000$ \\
		\bottomrule
	\end{tabular*}
\end{table}

\subsubsection{四级节标题}

根据需要,也可设四级节标题

由于它们对纤维织物的浸透性和亲和性不同,因此在纺织品染整加工中的用途也有差别,其中以水溶型和乳液型产品较为常用。另外,水系聚氨酯又有反应性和非反应性之分。虽然它们的共同特点是分子结构中不含异氰酸酯基,但前者是用封闭剂将异氰酸酯基暂时封闭,在纺织品整理时复出。相互交联反应形成三维网状结构而固着在织物表面。
……

% 一些对应的映射关系:
% \boldsymbol → \symbf
% \symbf{\overset{\rightharpoonup\!\!\!\! \rightharpoonup}{\chi}}^{(1)t}_{mz} → \overset{\rightharpoonup\!\!\!\! \rightharpoonup}{\symbf{\chi}_m}^{(1)t}_{z}
% \symbf{\overset{\rightharpoonup\!\!\!\! \rightharpoonup\!\!\!\! \rightharpoonup}{\chi}}^{(2)t}_{mz} → \overset{\rightharpoonup\!\!\!\! \rightharpoonup\!\!\!\! \rightharpoonup}{\symbf{\chi}_m}^{(2)t}_{z}

% \symbf{\overset{\rightharpoonup\!\!\!\! \rightharpoonup}{\mu}}^{(1)t}_{rz} → \overset{\rightharpoonup\!\!\!\! \rightharpoonup}{\symbf{\mu}_r}^{(1)t}_{z}
% \symbf{\overset{\rightharpoonup\!\!\!\! \rightharpoonup}{\mu}}^{(1)\omega}_{rz} → \overset{\rightharpoonup\!\!\!\! \rightharpoonup}{\symbf{\mu}_r}^{(1)\omega}_{z}
% \symbf{\overset{\rightharpoonup\!\!\!\! \rightharpoonup}{ → \overset{\rightharpoonup\!\!\!\! \rightharpoonup}{\symbf{

% \symbf{\overset{\rightharpoonup\!\!\!\! \rightharpoonup\!\!\!\! \rightharpoonup\!\!\!\! \rightharpoonup}{\chi}}^{(3)t}_{mz} → \overset{\rightharpoonup\!\!\!\! \rightharpoonup\!\!\!\! \rightharpoonup\!\!\!\! \rightharpoonup}{\symbf{\chi}_m}^{(3)t}_{z}
% \underset{\widetilde *}{\overset{\widetilde *}{\scriptsize \widetilde *}} → \begin{smallmatrix} \widetilde * \\ \widetilde * \\ \widetilde * \end{smallmatrix} 

% \mathcal F^{-1}_\omega \left[ \symbf{\widetilde H} \right] → F^{-1}_\omega \left[ \symbf{\widetilde H} \right] → \symbf H^{\omega}_z
% NL → \mathrm{NL}

% \left\{ → \left\{\ % 大括号 贴得太紧了
