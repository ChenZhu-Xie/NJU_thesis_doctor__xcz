%!TEX root = ..\njuthesis-sample.tex

\label{第一章开始}
\chapter{绪论}

\marginLeft[-2.4em]{sec:ttttt}\section{本论文研究的目的和意义}\label{sec:ttttt}

自从作为\textcolor{Plum}{超定系统}的 \textcolor{Maroon}{Maxwell-Lorentz-Heaviside} 方程组诞生以来,对其\textcolor{Plum}{微分}和\textcolor{Plum}{积分}形式的\textcolor{Plum}{解析}和\textcolor{Plum}{数值}探索一直是\textcolor{NavyBlue}{理论物理学者}追求的\textbf{圣杯}\cite{olyslagerElectromagneticsExoticMedia2002}。作为\textcolor{Plum}{最简单}的\textcolor{Maroon}{本构关系}与 \textcolor{Maroon}{Maxwell-Lorentz-Heaviside} 方程组的结合的产物之一,即使是\textcolor{Plum}{线性}\textcolor{PineGreen}{晶体光学}这片土地,也足够广袤和深刻,其\textcolor{Plum}{数学形式}上至\textcolor{Plum}{外代数}、\textcolor{Plum}{微分形式}以及\textcolor{NavyBlue}{广义相对论}\cite{yakovNonbirefringenceConditionsSpacetime2005,hehlSpacetimeMetricLocal2006,frankelGeometryPhysicsIntroduction2011,dahlNondissipativeElectromagneticMedium2013,ivezicManifestlyCovariantAharonovbohm2015,hehlAxionDilatonMetric2016,favaroRecentAdvancesClassical2012},\textcolor{NavyBlue}{微观起源}下至\textcolor{NavyBlue}{量子力学}\cite{nelsonMechanismsDispersionCrystalline1989,nelsonDerivingTransmissionReflection1995,nelsonLagrangianTreatmentMagnetic1994,nelsonOpticallyActiveSurface2000}和\textcolor{NavyBlue}{量子电动力学}\cite{eimerlQuantumElectrodynamicsOptical1988}。这吸引了一代又一代的扑火飞蛾,呕心沥血般燃尽它们的最后一滴生命。

作为比 \textcolor{Maroon}{Maxwell-Lorentz-Heaviside} 方程组更古老的存在们,光学 $\supsetneq$ \textcolor{Plum}{线性}\textcolor{NavyBlue}{光学} $\supsetneq$ \textcolor{Plum}{线性}\textcolor{PineGreen}{晶体光学},它们中的每一个,在对其\textcolor{Plum}{数学}即\textcolor{NavyBlue}{物理本质}上的理解和诠释的现代化转型之路,没有一个不是异常曲折的。

%引无数英雄前仆后继竞相折腰。

在 32 个\textcolor{Plum}{宏观晶体对称性}中, 21 个显示出\textcolor{Plum}{非中心对称性},使得这些\textcolor{NavyBlue}{点群}内能够产生二次谐波。然而,在这 21 种\textcolor{NavyBlue}{晶类}中,有 18 种是\textcolor{NavyBlue}{双折射}或\textcolor{Plum}{各向异性}的,超过一半(11 种)表现出自然\textcolor{NavyBlue}{旋光}性,即表现出了固有的\textcolor{NavyBlue}{光学活性}或\textcolor{NavyBlue}{手性}\cite{ossikovskiConstitutiveRelationsOptically2021}。因此,由于\textcolor{Plum}{低对称性}材料中\textcolor{PineGreen}{线性光学}过程的复杂性,包括二次、高次谐波和其他\textcolor{gray}{混频}现象在内的\textcolor{Plum}{非线性}\textcolor{PineGreen}{晶体光学}经常遇到巨大的\textcolor{Plum}{计算挑战}\cite{robertsSimplifiedCharacterizationUniaxial1992,dmitrievEffectiveNonlinearityCoefficients1993,diesperovEffectiveNonlinearCoefficient1997},即\textcolor{NavyBlue}{泵浦}/\textcolor{PineGreen}{基波}和由其驱动的\textcolor{gray}{新生成的频率} $\mathcolor{gray}{\omega}$ 在随后的\textcolor{PineGreen}{线性叠加}过程中必须经历独立的(\textcolor{Plum}{各向异性})衍射(\bref{eq:plane_wave_basis_eq-left_L}),同时可能在晶体内承受\textcolor{NavyBlue}{吸收}或\textcolor{NavyBlue}{增益}\cite{shengNonlinearopticalCalculationsUsing1980,grundmannOpticallyAnisotropicMedia2017}。

除了\textcolor{Plum}{均匀分布}的\textcolor{Plum}{复}张量 $\bar{\bar{\varepsilon}}^{\;\! \mathcolor{gray}{\omega}}_{\textcolor{Maroon}{(1)}}$ 本身的\textcolor{Plum}{各向异性}(\bref{eq:nonlinear(2)-wave_wkrho-simplify4})之外,在最近的工程进展中,所制造出的微纳结构 --- 即受调制的\textcolor{Plum}{非均匀}一阶、二阶材料系数张量 $\bar{\bar{\chi}}^{\;\! \mathcolor{gray}{\omega}}_{ \textcolor{Maroon}{(1)}} \left( \bar{r} \right),\bar{\bar{\chi}}^{\;\! \mathcolor{gray}{2\omega}}_{ \textcolor{Maroon}{(1)}} \left( \bar{r} \right),\bar{\bar{\bar{\chi}}}^{\;\! \mathcolor{gray}{2\omega}}_{ \textcolor{Maroon}{(2)}} \left( \bar{r} \right)$ 等本身\cite{xuFemtosecondLaserWriting2022,weiExperimentalDemonstrationThreedimensional2018,xuThreedimensionalNonlinearPhotonic2018,keren-zurNewDimensionNonlinear2018},同样需要立即和精确考虑由材料\textcolor{gray}{(亚)波长}尺度\textcolor{Plum}{不均匀性}(\bref{eq:nonlinear(2)-wave_wkrho-simplify4-2})所引起的(潜在\textcolor{Plum}{各向异性})散射和衍射效应\cite{chenQuasiphasematchingdivisionMultiplexingHolography2021b,chenLaserNanoprinting3D2023,sunFerroelectricTopologiesBaTiO32025,xuLargeFieldofviewNonlinear2024},包括\textcolor{PineGreen}{基波}和新产生的\textcolor{gray}{频率组分}\cite{chenQuasiphasematchingdivisionMultiplexingHolography2021b,chenLaserNanoprinting3D2023}。因此,根据\textcolor{NavyBlue}{第一性原理},\textcolor{Plum}{非线性}\textcolor{PineGreen}{晶体光学}是建立在\textcolor{Plum}{线性}\textcolor{PineGreen}{晶体光学}的基础上的。以至于,在深入研究\textcolor{Plum}{非线性}\textcolor{PineGreen}{晶体光学}之前,彻底掌握\textcolor{Plum}{线性}\textcolor{PineGreen}{晶体光学}是\textcolor{Maroon}{先决条件}。

自上而下\Footnote{从宏观规律或数学形式出发,而不是从微观机制或实验现象出发。以避免陷入“现象学”和“归纳法”等,针对具体现象而特别构造的、适用范围有限/单一的、“为了解释而解释”的、工程导向和表面粉饰的、经不起反复推敲的、用完即焚的“一次性”黑盒子理论。}地重新审视这两个领域,

从\textcolor{PineGreen}{双各向异性}材料中矢量\textcolor{Plum}{非线性}\textcolor{PineGreen}{晶体光学}的精确解 \bref{dd} 一直到以双二次特征方程($\supsetneq$ Fresnel 波法方程)而闻名的 LCO,评估了每种方法的等效性、优点和局限性。

可是我们发现 {\one} \textcolor{Maroon}{矩阵指数} $\mathbb{e}^{\mathbb{i} \Xint{\begin{smallmatrix} ~ \\ {}^{}_{\mathcolor{gray}{-}} \\ ~ \end{smallmatrix}}{15}{\bar{\bar{k}}}_{\textcolor{Maroon}{\symup{z}}}^{\;\! \mathcolor{gray}{\omega}} \mathcolor{gray}{z}}$, $\mathbb{e}^{-\mathbb{i} \bar{\bar{H}} \mathcolor{gray}{t} \big/ \hbar}$,作为\textcolor{PineGreen}{平面波}的最一般形式\cite{berremanOpticsStratifiedAnisotropic1972,stallingaBerreman4x4Matrix1999,molerNineteenDubiousWays2003,zarifiPlaneWaveReflection2014}(\bref{eq:matrix_exp}),以及彻底解析\textcolor{Plum}{大规模退化}\textcolor{PineGreen}{特征向量场}(\bref{eq:e^ikzz-non_diagonalization2})的一种有前景的方法,代表了大多数\textcolor{Plum}{线性系统},包括 \bref{eq:nonlinear(2)-wave_wkrho-simplify6-L4} 所统治的\textcolor{Plum}{线性}\textcolor{PineGreen}{晶体光学}的\textcolor{Plum}{通解}。然而,一个通用且高效的数值解,特别是在\textcolor{PineGreen}{异常点}即\textcolor{PineGreen}{光学奇点}\cite{berryOpticalSingularitiesBirefringent2003,berryOpticalSingularitiesBianisotropic2005} 阵列附近,仍然是难以捉摸的\cite{molerNineteenDubiousWays2003}。{\two} 随着时间或空间的延长,\textcolor{NavyBlue}{增益}甚至\textcolor{NavyBlue}{吸收}都会导致\textcolor{Maroon}{矩阵指数}\textcolor{Plum}{发散},使\textcolor{Maroon}{矩阵指数}不适合处理大规模的\textcolor{Plum}{非厄米}系统\cite{stallingaBerreman4x4Matrix1999}。{\three} 如果\textcolor{Maroon}{矩阵指数}的\textcolor{NavyBlue}{指数}/\textcolor{NavyBlue}{相位部分}被缩减为仅包括\textcolor{PineGreen}{特征值},并且\textcolor{PineGreen}{特征向量}被移出\textcolor{NavyBlue}{相位部分}以与\textcolor{Plum}{指数项}相乘(即 $\Xint{\begin{smallmatrix} ~ \\ {}^{}_{\mathcolor{gray}{-}} \\ ~ \end{smallmatrix}}{15}{\bar{v}}^{\;\!\mathcolor{gray}{\omega}}_{\textcolor{PineGreen}{\jmath}} \cdot \mathbb{e}^{\mathbb{i} \Xint{\begin{smallmatrix} ~ \\ {}^{}_{\mathcolor{gray}{-}} \\ ~ \end{smallmatrix}}{22}{\lambda}^{\;\!\mathcolor{gray}{\omega}}_{\textcolor{PineGreen}{\jmath}} \mathcolor{gray}{z}}$ 或 $\overline{\Xint{\begin{smallmatrix} ~ \\ {}^{}_{\mathcolor{gray}{-}} \\ ~ \end{smallmatrix}}{15}{\bar{v}}^{\;\!\mathcolor{gray}{\omega}}_{\textcolor{PineGreen}{\jmath}}}^{{\mathsf{\textcolor{Plum}{T}}}} \cdot \overline{\overline{\mathbb{e}^{\mathbb{i} \Xint{\begin{smallmatrix} ~ \\ {}^{}_{\mathcolor{gray}{-}} \\ ~ \end{smallmatrix}}{22}{\lambda}^{\;\!\mathcolor{gray}{\omega}}_{\textcolor{PineGreen}{\jmath}} \mathcolor{gray}{z}}}} \cdot \overline{\Xint{\begin{smallmatrix} ~ \\ {}^{}_{\mathcolor{gray}{-}} \\ ~ \end{smallmatrix}}{15}{\bar{v}}^{\;\!\mathcolor{gray}{\omega}}_{\textcolor{PineGreen}{\jmath}}}^{\textcolor{Plum}{-\mathsf{T}}}$)--- 如果不通过 \textcolor{PineGreen}{Jordan-Chevalley 分解}或其等效方法进行分解\cite{borzdovWavesLinearQuadratic1996,sturmElectromagneticWavesCrystals2024} --- 则\textcolor{PineGreen}{例外点(们)}不能再(被)展开,届时在该方向(们)的\textcolor{NavyBlue}{场传播和演化}的潜在\textcolor{NavyBlue}{动力学过程},将变得\textcolor{Plum}{完全无法计算}。

{\four} 作为\textcolor{PineGreen}{平面波}最广为人知的基本形式\cite{borzdovWavesLinearQuadratic1996}和波动方程的\textcolor{Plum}{最简单试探}\textcolor{PineGreen}{本征解}(\bref{eq:plane_wave_basis}),$\mathbb{e}^{\mathbb{i} \Xint{\begin{smallmatrix} ~ \\ {}^{}_{\mathcolor{gray}{-}} \\ ~ \end{smallmatrix}}{15}{\bar{k}}^{\;\! \mathcolor{gray}{\omega}}_{\textcolor{PineGreen}{\jmath}} \cdot \textcolor{gray}{\bar{r}}}$ 为重建任何分布在\textcolor{gray}{实空间} $\in \textcolor{gray}{\mathbb{R}}_{\textcolor{Plum}{3}}$ 中的\textcolor{Plum}{复}标量场 $\in \mathbb{C} (\textcolor{gray}{\mathbb{R}}_{\textcolor{Plum}{3}})$ 提供了一个易于访问的\textcolor{Plum}{正交完备}基础,并为\textcolor{gray}{倒空间} $\in \textcolor{gray}{\mathbb{R}}_{\textcolor{Plum}{2}}$ 中的\textcolor{PineGreen}{本征分析}提供了一个重要工具,于长达 \textcolor{Plum}{2 个世纪}的时间跨幅里,渗透和引领了对\textcolor{NavyBlue}{物理学}各个细分领域的硕果累累的高效探索\cite{borzdovWavesLinearQuadratic1996,olyslagerElectromagneticsExoticMedia2002,berryChapterConicalDiffraction2007}。


屈原:天问。\cite{berryOpticalSingularitiesBirefringent2003,ossikovskiConstitutiveRelationsOptically2021}。

奖励 = 找出该错误的期望耗时 * 期望时薪。

如果收到独立验证的正面反馈,将过程寄给我,也可获得奖励。

不知道这是不是又一个漂亮的黑洞/温柔的良夜,迷途无返。未选择的路。

\marginLeft[-2.4em]{sec:tttt}\section{国内外研究现状及发展趋势}\label{sec:tttt}

\marginLeft[-2.4em]{ssec:tttt}\subsection{形状记忆聚氨酯的形状记忆机理}\label{ssec:tttt}

形状记忆聚合物(SMP)是继形状记忆合金后在80年代发展起来的一种新型形状记忆材料

分别给磁感应场$\bar{B}^{\;\!\mathcolor{gray}{t}}_{\;\!\mathcolor{gray}{z}}$(而非磁场$\bar{H}^{\;\!\mathcolor{gray}{t}}_{\;\!\mathcolor{gray}{z}}$),自由电流源$\bar{J}^{\;\!\mathcolor{gray}{t}}_{\;\!\textcolor{Maroon}{\text{f}}\mathcolor{gray}{z}}$ 以及电位移场$\bar{D}^{\;\!\mathcolor{gray}{t}}_{\;\!\mathcolor{gray}{z}}$,定义了如下 3 个本构关系(\textcolor{Maroon}{\text{Constitutive Relation}} = \textcolor{Maroon}{\text{CR}})。

其一,磁场 $\bar{H}^{\;\!\mathcolor{gray}{t}}_{\;\!\mathcolor{gray}{z}}$ 的本构关系,考虑$\bar{M}^{\;\!\mathcolor{gray}{t}}_{\;\!\mathcolor{gray}{z}}$\Footnote{磁化强度。其在微观上来源于:分子电流(电子轨道运动)产生的磁矩和电子自旋磁矩的矢量和\cite{nelsonLagrangianTreatmentMagnetic1994},细究还将有电子与原子核、电子的自旋轨道耦合、电子电子间相互作用(多电子产生原子磁矩)、原子与原子间(晶体场)的相互作用\cite{chen-zhuChenZhuxieUndergraduate_courses2024}。磁的起源看上去可以是纯电的\cite{lakhtakiaGenesisPostConstraint2004}。}仅由磁偶极矩\Footnote{即只考虑最低阶磁多极矩 = 不考虑磁四极矩及以上\cite{nelsonLagrangianTreatmentMagnetic1994},因为对于受到电磁场的影响后,反过来产生电磁场的电子而言,其受到的电场力是洛伦兹力的c$\big/v$倍\cite{boydNonlinearOptics2019}。此外,(磁/电)多极矩与(磁/电)非线性是互相独立的——即任何阶的多极矩,都有自己的线性项和非线性项\cite{chen-zhuChenZhuxieUndergraduate_courses2024},这些项与其它阶多极矩的任何项都无关。}贡献时,定义为
\begin{subequations} \label{eq:cr-b}
\begin{align}
	\textcolor{Maroon}{\text{CR for magnetism}}\text{:}&\hspace{0.5em} \bar{B}^{\;\!\mathcolor{gray}{t}}_{\;\!\mathcolor{gray}{z}} \hspace{-0.7em} &&={\symup{\varepsilon}}_0 \left( \bar{H}^{\;\!\mathcolor{gray}{t}}_{\;\!\mathcolor{gray}{z}} + \bar{M}^{\;\!\mathcolor{gray}{t}}_{\;\!\mathcolor{gray}{z}} \right) = {\symup{\varepsilon}}_0 \left\{ \bar{\bar{\delta}}^{\;\!\mathcolor{gray}{t}}~\widetilde *~\bar{H}^{\;\!\mathcolor{gray}{t}}_{\;\!\mathcolor{gray}{z}} + \bar{M}^{\;\!\mathcolor{gray}{t}}_{\;\!\mathcolor{gray}{z}} \right\} \label{cr-b1} \\ 
	& &&\xrightarrow[]{\bar{M}^{\;\!\mathcolor{gray}{t}}_{\;\!\mathcolor{gray}{z}} = \bar{M}^{\;\!\textcolor{Maroon}{\text{(1)}} \mathcolor{gray}{t}}_{\;\!\mathcolor{gray}{z}} + \bar{M}^{\;\!\textcolor{Maroon}{\text{NL}}, \mathcolor{gray}{t}}_{\;\!\mathcolor{gray}{z}}} {\symup{\varepsilon}}_0 \left\{ \left[ \bar{\bar{\delta}}^{\;\!\mathcolor{gray}{t}}~\widetilde *~\bar{H}^{\;\!\mathcolor{gray}{t}}_{\;\!\mathcolor{gray}{z}} + \bar{M}^{\;\!\textcolor{Maroon}{\text{(1)}} \mathcolor{gray}{t}}_{\;\!\mathcolor{gray}{z}} \right] + \bar{M}^{\;\!\textcolor{Maroon}{\text{NL}}, \mathcolor{gray}{t}}_{\;\!\mathcolor{gray}{z}} \right\} \label{cr-b2} \\ 
	& &&\xrightarrow[\displaystyle{ \bar{\bar{\mu}}^{\;\!\textcolor{Maroon}{\text{(1)}} \mathcolor{gray}{t}}_{\;\!\textcolor{Maroon}{\text{r}}\mathcolor{gray}{z}} := \bar{\bar{\delta}}^{\;\!\mathcolor{gray}{t}} + \bar{\bar{\chi}}^{\;\!\textcolor{Maroon}{\text{(1)}}\mathcolor{gray}{t}}_{\;\!\textcolor{Maroon}{\text{m}} \mathcolor{gray}{z}}}]{\displaystyle{\bar{M}^{\;\!\textcolor{Maroon}{\text{(1)}} \mathcolor{gray}{t}}_{\;\!\mathcolor{gray}{z}} := \bar{\bar{\chi}}^{\;\!\textcolor{Maroon}{\text{(1)}}\mathcolor{gray}{t}}_{\;\!\textcolor{Maroon}{\text{m}} \mathcolor{gray}{z}} ~\widetilde *~\bar{H}^{\;\!\mathcolor{gray}{t}}_{\;\!\mathcolor{gray}{z}}}} {\symup{\varepsilon}}_0 \left\{ \bar{\bar{\mu}}^{\;\!\textcolor{Maroon}{\text{(1)}} \mathcolor{gray}{t}}_{\;\!\textcolor{Maroon}{\text{r}}\mathcolor{gray}{z}}~\widetilde *~\bar{H}^{\;\!\mathcolor{gray}{t}}_{\;\!\mathcolor{gray}{z}} + \bar{M}^{\;\!\textcolor{Maroon}{\text{NL}}, \mathcolor{gray}{t}}_{\;\!\mathcolor{gray}{z}} \right\} \label{cr-b3} \\ 
	& &&= \bar{\bar{\mu}}^{\;\!\textcolor{Maroon}{\text{(1)}} \mathcolor{gray}{t}}_{\;\!\mathcolor{gray}{z}}~\widetilde *~\bar{H}^{\;\!\mathcolor{gray}{t}}_{\;\!\mathcolor{gray}{z}} + {\symup{\varepsilon}}_0 \bar{M}^{\;\!\textcolor{Maroon}{\text{NL}}, \mathcolor{gray}{t}}_{\;\!\mathcolor{gray}{z}} =: \bar{B}^{\;\!\textcolor{Maroon}{\text{(1)}} \mathcolor{gray}{t}}_{\;\!\mathcolor{gray}{z}} + \bar{B}^{\;\!\textcolor{Maroon}{\text{NL}}, \mathcolor{gray}{t}}_{\;\!\mathcolor{gray}{z}}~, \label{cr-b4}
\end{align}
\end{subequations}
其中,磁通量密度场 $\bar{B}^{\;\!\mathcolor{gray}{t}}_{\;\!\mathcolor{gray}{z}}$(直接/显示地)关于磁场 $\bar{H}^{\;\!\mathcolor{gray}{t}}_{\;\!\mathcolor{gray}{z}}$\Footnote{磁非线性,如郎之万顺磁性理论\cite{chen-zhuChenZhuxieUndergraduate_courses2024}中 $M \propto$ 郎之万函数 $\mathcal{L} \left( \alpha \right) = \coth \left( \alpha \right) - 1 / \alpha$(其中 $\alpha \propto H_{\textcolor{Maroon}{\text{ex}}}$)或其量子化修正之布里渊函数,铁磁体\cite{chen-zhuChenZhuxieUndergraduate_courses2024}或超导体\cite{wenBriefIntroductionFlux2021}中的磁滞现象等(每个时刻$t$,这些场量都是准静态$\Omega \to 0$的)。}、电场 $\bar{E}^{\;\!\mathcolor{gray}{t}}_{\;\!\mathcolor{gray}{z}}$\Footnote{双各向异性,其中的电$\to$磁耦合(如果 $\bar{B}^{\;\!\mathcolor{gray}{t}}_{\;\!\mathcolor{gray}{z}}$ 中的该部分只是 $\bar{E}^{\;\!\mathcolor{gray}{t}}_{\;\!\mathcolor{gray}{z}}$ 的线性函数,则也可归结到线性项中)。}、应力 $\bar{T}^{\;\!\mathcolor{gray}{t}}_{\;\!\mathcolor{gray}{z}}$\Footnote{正逆压磁/磁致伸缩/磁弹效应(这里未作区分)。}等其他场量(即含空 $\mathcolor{gray}{\bar{r}}$ 的物理量)\Footnote{$\bar{B}^{\;\!\mathcolor{gray}{t}}_{\;\!\mathcolor{gray}{z}},\bar{M}^{\;\!\mathcolor{gray}{t}}_{\;\!\mathcolor{gray}{z}}$ 以及各阶 $\bar{\bar{\mu}}^{\;\!\textcolor{Maroon}{\text{(1)}} \mathcolor{gray}{t}}_{\;\!\mathcolor{gray}{z}},\bar{\bar{\bar{\mu}}}^{\;\!\textcolor{Maroon}{\text{(2)}} \mathcolor{gray}{t}}_{\;\!\mathcolor{gray}{z}},\cdots$ 已经是关于温度$T$、波长$\lambda$(或 角频率$\omega$、时间$t$)等(非)场量的函数。}的非线性函数项,悉数包含在 $\bar{B}^{\;\!\textcolor{Maroon}{\text{NL}}, \mathcolor{gray}{t}}_{\;\!\mathcolor{gray}{z}} = {\symup{\varepsilon}}_0 \bar{M}^{\;\!\textcolor{Maroon}{\text{NL}}, \mathcolor{gray}{t}}_{\;\!\mathcolor{gray}{z}}$ 内;剩余的线性项,放在 $\bar{B}^{\;\!\textcolor{Maroon}{\text{(1)}} \mathcolor{gray}{t}}_{\;\!\mathcolor{gray}{z}} = \bar{\bar{\mu}}^{\;\!\textcolor{Maroon}{\text{(1)}} \mathcolor{gray}{t}}_{\;\!\mathcolor{gray}{z}}~\widetilde *~\bar{H}^{\;\!\mathcolor{gray}{t}}_{\;\!\mathcolor{gray}{z}}$ 中。

其二,自由电流源$\bar{J}^{\;\!\mathcolor{gray}{t}}_{\;\!\textcolor{Maroon}{\text{f}}\mathcolor{gray}{z}}$的本构关系,包含欧姆定律的线性部分(漂移项) $\bar{J}^{\;\!\textcolor{Maroon}{\text{(1)}} \mathcolor{gray}{t}}_{\;\!\textcolor{Maroon}{\text{f}}\mathcolor{gray}{z}} = \bar{\bar{\sigma}}^{\;\!\textcolor{Maroon}{\text{(1)}}\mathcolor{gray}{t}}_{\;\!\mathcolor{gray}{z}}~\widetilde *~\bar{E}^{\;\!\mathcolor{gray}{t}}_{\;\!\mathcolor{gray}{z}}$\Footnote{可以由 Drude 模型描述,定量解释一阶电导率$\bar{\bar{\sigma}}^{\;\!\textcolor{Maroon}{\text{(1)}}\mathcolor{gray}{t}}_{\;\!\mathcolor{gray}{z}}$的起源。},及$\bar{J}^{\;\!\mathcolor{gray}{t}}_{\;\!\textcolor{Maroon}{\text{f}}\mathcolor{gray}{z}}$分别关于电场 $\bar{E}^{\;\!\mathcolor{gray}{t}}_{\;\!\mathcolor{gray}{z}}$\Footnote{欧姆定律中的电非线性部分,比如二/三极管的伏安特性曲线\cite{chen-zhuChenZhuxieUndergraduate_courses2024}(尽管输入/输出or自/因变量,即$\bar{E}^{\;\!\mathcolor{gray}{t}}_{\;\!\mathcolor{gray}{z}}$和$\bar{J}^{\;\!\mathcolor{gray}{t}}_{\;\!\textcolor{Maroon}{\text{f}}\mathcolor{gray}{z}}$,一般均在直流或低频$\Omega$,非交流且不在光波段 opt)。}、磁感应场$\bar{B}^{\;\!\mathcolor{gray}{t}}_{\;\!\mathcolor{gray}{z}}$\Footnote{磁感应场$\bar{B}^{\;\!\mathcolor{gray}{t}}_{\;\!\mathcolor{gray}{z}}$所带来的(电场力以外的)洛伦兹力$\left( \bar{J}^{\;\!\mathcolor{gray}{t}}_{\;\!\textcolor{Maroon}{\text{f}}\mathcolor{gray}{z}} + \dot{\bar{P}}^{\;\!\mathcolor{gray}{t}}_{\;\!\mathcolor{gray}{z}} + \mathcolor{gray}{\bar{\nabla} \times} \bar{M}^{\;\!\mathcolor{gray}{t}}_{\;\!\mathcolor{gray}{z}} \right) \times \bar{B}^{\;\!\mathcolor{gray}{t}}_{\;\!\mathcolor{gray}{z}}$\cite{mackayElectromagneticAnisotropyBianisotropy2019,chen-zhuChenZhuxieUndergraduate_courses2024},会影响导/价带电子的运动(速度)$\bar{v}^{\;\!\mathcolor{gray}{t}}_{\;\!\textcolor{Maroon}{\text{f}}\mathcolor{gray}{z}}$,进而全局地影响自由电流$\bar{J}^{\;\!\mathcolor{gray}{t}}_{\;\!\textcolor{Maroon}{\text{f}}\mathcolor{gray}{z}} = {\rho}^{\;\!\mathcolor{gray}{t}}_{\;\!\textcolor{Maroon}{\text{f}}\mathcolor{gray}{z}} \bar{v}^{\;\!\mathcolor{gray}{t}}_{\;\!\textcolor{Maroon}{\text{f}}\mathcolor{gray}{z}}$和(束缚)电(偶)极化强度$\bar{P}^{\;\!\mathcolor{gray}{t}}_{\;\!\mathcolor{gray}{z}}$,包括它们的线性和非线性项\cite{boydNonlinearOptics2019}。对于强场/超快非线性光学,相对论效应使得电磁场是个统一的整体,动生(而不仅是外加)的$\bar{B}^{\;\!\mathcolor{gray}{t}}_{\;\!\mathcolor{gray}{z}}$还将带来额外的影响。磁场$\bar{H}^{\;\!\mathcolor{gray}{t}}_{\;\!\mathcolor{gray}{z}}$对分子/磁化电流体密度$\mathcolor{gray}{\bar{\nabla} \times} \bar{M}^{\;\!\mathcolor{gray}{t}}_{\;\!\mathcolor{gray}{z}}$产生的影响已包含在$\bar{M}^{\;\!\mathcolor{gray}{t}}_{\;\!\mathcolor{gray}{z}}$中了。}、导带电子浓度(数密度)梯度场$\mathcolor{gray}{\bar{\nabla}} {\rho}^{\;\!\mathcolor{gray}{t}}_{\;\!\textcolor{Maroon}{\text{f}}\mathcolor{gray}{z}}$\Footnote{在光折变效应中,作为$\bar{J}^{\;\!\mathcolor{gray}{t}}_{\;\!\textcolor{Maroon}{\text{f}}\mathcolor{gray}{z}}$中的扩散项\cite{boydNonlinearOptics2019}。${\rho}^{\;\!\mathcolor{gray}{t}}_{\;\!\textcolor{Maroon}{\text{f}}\mathcolor{gray}{z}}, \bar{J}^{\;\!\mathcolor{gray}{t}}_{\;\!\textcolor{Maroon}{\text{f}}\mathcolor{gray}{z}}$之间还应满足\bref{eq:Div-e-f}以及$\bar{J}^{\;\!\mathcolor{gray}{t}}_{\;\!\textcolor{Maroon}{\text{f}}\mathcolor{gray}{z}} = {\rho}^{\;\!\mathcolor{gray}{t}}_{\;\!\textcolor{Maroon}{\text{f}}\mathcolor{gray}{z}} \bar{v}^{\;\!\mathcolor{gray}{t}}_{\;\!\textcolor{Maroon}{\text{f}}\mathcolor{gray}{z}}$\cite{chen-zhuChenZhuxieUndergraduate_courses2024}。}和光伏电流场$\propto \lvert \bar{E}^{\;\!\mathcolor{gray}{t}}_{\;\!\mathcolor{gray}{z}} \rvert^2 \hat{c}$\Footnote{与光电导效应并列,属于内光电效应;也可能在光折变效应的$\bar{J}^{\;\!\mathcolor{gray}{t}}_{\;\!\textcolor{Maroon}{\text{f}}\mathcolor{gray}{z}}$中扮演一份角色,特别是沿着一些各向异性晶体的光轴$\hat{c}$产生电势差和内建电场\cite{boydNonlinearOptics2019}(尽管一般也只影响直流或低频$\Omega$的$\bar{J}^{\;\!\mathcolor{gray}{t}}_{\;\!\textcolor{Maroon}{\text{f}}\mathcolor{gray}{z}}$;但$\bar{J}^{\;\!\mathcolor{gray}{t}}_{\;\!\textcolor{Maroon}{\text{f}}\mathcolor{gray}{z}}$会通过影响光波段的介电常数,进而影响光波段的光强$\lvert \bar{E}^{\;\!\mathcolor{gray}{t}}_{\;\!\mathcolor{gray}{z}} \rvert^2$及$\bar{J}^{\;\!\mathcolor{gray}{t}}_{\;\!\textcolor{Maroon}{\text{f}}\mathcolor{gray}{z}}$自己的重新分布);该二阶的带耦合的非线性,看上去很像非线性极化率$\bar{P}^{\;\!\textcolor{Maroon}{\text{(2)}} \mathcolor{gray}{t}}_{\;\!\mathcolor{gray}{z}}$中的光整流项,但其频率比 THz 低,且只服务于自由电流。—— 以至该项可作为差频合并至$\bar{J}^{\;\!\mathcolor{gray}{t}}_{\;\!\textcolor{Maroon}{\text{f}}\mathcolor{gray}{z}}$关于$\bar{E}^{\;\!\mathcolor{gray}{t}}_{\;\!\mathcolor{gray}{z}}$的二阶非线性$\bar{J}^{\;\!\textcolor{Maroon}{\text{(2)}} \mathcolor{gray}{t}}_{\;\!\textcolor{Maroon}{\text{f}}\mathcolor{gray}{z}}$中去?}等其他场量的非线性项 $\bar{J}^{\;\!\textcolor{Maroon}{\text{NL}}, \mathcolor{gray}{t}}_{\;\!\textcolor{Maroon}{\text{f}}\mathcolor{gray}{z}}$:
\begin{align} \label{eq:cr-j}
	\textcolor{Maroon}{\text{Ohm's law}}\text{:}\hspace{0.5em} \bar{J}^{\;\!\mathcolor{gray}{t}}_{\;\!\textcolor{Maroon}{\text{f}}\mathcolor{gray}{z}} = \bar{\bar{\sigma}}^{\;\!\textcolor{Maroon}{\text{(1)}}\mathcolor{gray}{t}}_{\;\!\mathcolor{gray}{z}}~\widetilde *~\bar{E}^{\;\!\mathcolor{gray}{t}}_{\;\!\mathcolor{gray}{z}} + \bar{J}^{\;\!\textcolor{Maroon}{\text{NL}}, \mathcolor{gray}{t}}_{\;\!\textcolor{Maroon}{\text{f}}\mathcolor{gray}{z}} =: \bar{J}^{\;\!\textcolor{Maroon}{\text{(1)}} \mathcolor{gray}{t}}_{\;\!\textcolor{Maroon}{\text{f}}\mathcolor{gray}{z}} + \bar{J}^{\;\!\textcolor{Maroon}{\text{NL}}, \mathcolor{gray}{t}}_{\;\!\textcolor{Maroon}{\text{f}}\mathcolor{gray}{z}}~,
\end{align}

其三,电位移场$\bar{D}^{\;\!\mathcolor{gray}{t}}_{\;\!\mathcolor{gray}{z}}$的本构关系,当$\bar{P}^{\;\!\mathcolor{gray}{t}}_{\;\!\mathcolor{gray}{z}}$只由电偶极矩\Footnote{不考虑电四极矩及以上。但电四极化强度场$\bar{\bar{Q}}^{\;\!\mathcolor{gray}{t}}_{\;\!\mathcolor{gray}{z}}$(的等效电偶极化强度场$\bar{P}^{\;\!\mathcolor{gray}{t}}_{\;\!\textcolor{Maroon}{\text{Q}}\mathcolor{gray}{z}} = - \mathcolor{gray}{\bar{\nabla} \cdot} \bar{\bar{Q}}^{\;\!\mathcolor{gray}{t}}_{\;\!\mathcolor{gray}{z}}$)\cite{chen-zhuChenZhuxieUndergraduate_courses2024}在有些效应中不可忽视且起关键作用:如其对线性晶体光学中的光学活性的贡献\cite{nelsonDerivingTransmissionReflection1995},以及非线性光学中基于$\bar{\bar{Q}}^{\;\!\mathcolor{gray}{t}}_{\;\!\mathcolor{gray}{z}}$的二阶和频\cite{bethuneOpticalQuadrupoleSumfrequency1976}。电四极子对光与物质相互作用的贡献,还会打破$\bar{D}^{\;\!\mathcolor{gray}{t}}_{\;\!\mathcolor{gray}{z}}$法向连续和$\bar{H}^{\;\!\mathcolor{gray}{t}}_{\;\!\mathcolor{gray}{z}}$切向连续边界条件,并与洛伦兹力的定义、(由唯二的无源齐次\cite{lakhtakiaGenesisPostConstraint2004}微分方程\bref{eq:Curl-EK,eq:Div-Bk}导出的)电磁场标/矢势\cite{chen-zhuChenZhuxieUndergraduate_courses2024}等一起,使$\bar{E}^{\;\!\mathcolor{gray}{t}}_{\;\!\mathcolor{gray}{z}},\bar{B}^{\;\!\mathcolor{gray}{t}}_{\;\!\mathcolor{gray}{z}}$而不是$\bar{E}^{\;\!\mathcolor{gray}{t}}_{\;\!\mathcolor{gray}{z}},\bar{H}^{\;\!\mathcolor{gray}{t}}_{\;\!\mathcolor{gray}{z}}$成为基本场\cite{nelsonDerivingTransmissionReflection1995},对应地,坡印亭矢量也需要修正为$\bar{E}^{\;\!\mathcolor{gray}{t}}_{\;\!\mathcolor{gray}{z}} \times \bar{B}^{\;\!\mathcolor{gray}{t}}_{\;\!\mathcolor{gray}{z}} \big/ {\symup{\varepsilon}}_0$\cite{nelsonGeneralizingPoyntingVector1996,loudonPropagationElectromagneticEnergy1997,richterPoyntingsTheoremEnergy2008}而不是$\bar{E}^{\;\!\mathcolor{gray}{t}}_{\;\!\mathcolor{gray}{z}} \times \bar{H}^{\;\!\mathcolor{gray}{t}}_{\;\!\mathcolor{gray}{z}}$。}构成时,定义为
\begin{subequations} \label{eq:cr-d}
\begin{align}
	\textcolor{Maroon}{\text{CR for electricity}}\text{:}&\hspace{0.5em} \bar{D}^{\;\!\mathcolor{gray}{t}}_{\;\!\mathcolor{gray}{z}} \hspace{-2.0em} &&= {\symup{\varepsilon}}_0 \bar{E}^{\;\!\mathcolor{gray}{t}}_{\;\!\mathcolor{gray}{z}} + \bar{P}^{\;\!\mathcolor{gray}{t}}_{\;\!\mathcolor{gray}{z}} = {\symup{\varepsilon}}_0 \bar{\bar{\delta}}^{\;\!\mathcolor{gray}{t}}~\widetilde *~\bar{E}^{\;\!\mathcolor{gray}{t}}_{\;\!\mathcolor{gray}{z}} + \bar{P}^{\;\!\mathcolor{gray}{t}}_{\;\!\mathcolor{gray}{z}} \label{cr-d1} \\ 
	& &&\xrightarrow[]{\bar{P}^{\;\!\mathcolor{gray}{t}}_{\;\!\mathcolor{gray}{z}} = \bar{P}^{\;\!\textcolor{Maroon}{\text{(1)}} \mathcolor{gray}{t}}_{\;\!\mathcolor{gray}{z}} + \bar{P}^{\;\!\textcolor{Maroon}{\text{NL}}, \mathcolor{gray}{t}}_{\;\!\mathcolor{gray}{z}} + } \left[ {\symup{\varepsilon}}_0 \bar{\bar{\delta}}^{\;\!\mathcolor{gray}{t}}~\widetilde *~\bar{E}^{\;\!\mathcolor{gray}{t}}_{\;\!\mathcolor{gray}{z}} + \bar{P}^{\;\!\textcolor{Maroon}{\text{(1)}} \mathcolor{gray}{t}}_{\;\!\mathcolor{gray}{z}} \right] + \bar{P}^{\;\!\textcolor{Maroon}{\text{NL}}, \mathcolor{gray}{t}}_{\;\!\mathcolor{gray}{z}} \label{cr-d2} \\ 
	& &&\xrightarrow[\displaystyle{ \bar{\bar{\varepsilon}}^{\;\!\textcolor{Maroon}{\text{(1)}} \mathcolor{gray}{t}}_{\;\!\textcolor{Maroon}{\text{r}}\mathcolor{gray}{z}} := \bar{\bar{\delta}}^{\;\!\mathcolor{gray}{t}} + \bar{\bar{\chi}}^{\;\!\textcolor{Maroon}{\text{(1)}}\mathcolor{gray}{t}}_{\;\!\textcolor{Maroon}{\text{f}} \mathcolor{gray}{z}}}]{\displaystyle{\bar{P}^{\;\!\textcolor{Maroon}{\text{(1)}} \mathcolor{gray}{t}}_{\;\!\mathcolor{gray}{z}} := \bar{\bar{\chi}}^{\;\!\textcolor{Maroon}{\text{(1)}}\mathcolor{gray}{t}}_{\;\!\textcolor{Maroon}{\text{f}} \mathcolor{gray}{z}} ~\widetilde *~\bar{E}^{\;\!\mathcolor{gray}{t}}_{\;\!\mathcolor{gray}{z}}}} {\symup{\varepsilon}}_0 \bar{\bar{\varepsilon}}^{\;\!\textcolor{Maroon}{\text{(1)}} \mathcolor{gray}{t}}_{\;\!\textcolor{Maroon}{\text{r}}\mathcolor{gray}{z}}~\widetilde *~\bar{E}^{\;\!\mathcolor{gray}{t}}_{\;\!\mathcolor{gray}{z}} + \bar{P}^{\;\!\textcolor{Maroon}{\text{NL}}, \mathcolor{gray}{t}}_{\;\!\mathcolor{gray}{z}} \label{cr-d3} \\ 
	& &&= \bar{\bar{\varepsilon}}^{\;\!\textcolor{Maroon}{\text{(1)}} \mathcolor{gray}{t}}_{\;\!\mathcolor{gray}{z}}~\widetilde *~\bar{E}^{\;\!\mathcolor{gray}{t}}_{\;\!\mathcolor{gray}{z}} + \bar{P}^{\;\!\textcolor{Maroon}{\text{NL}}, \mathcolor{gray}{t}}_{\;\!\mathcolor{gray}{z}} =: \bar{D}^{\;\!\textcolor{Maroon}{\text{(1)}} \mathcolor{gray}{t}}_{\;\!\mathcolor{gray}{z}} + \bar{D}^{\;\!\textcolor{Maroon}{\text{NL}}, \mathcolor{gray}{t}}_{\;\!\mathcolor{gray}{z}}~, \label{cr-d4}
\end{align}
\end{subequations}
关于其组成成分,电位移场 $\bar{D}^{\;\!\mathcolor{gray}{t}}_{\;\!\mathcolor{gray}{z}}$(直接/显示地)关于电场 $\bar{E}^{\;\!\mathcolor{gray}{t}}_{\;\!\mathcolor{gray}{z}}$\Footnote{电非线性,包括高频段的(非)共振非线性、低频低温\cite{lakhtakiaGenesisPostConstraint2004}段的铁电体/畴的电滞现象等。}、磁场 $\bar{H}^{\;\!\mathcolor{gray}{t}}_{\;\!\mathcolor{gray}{z}}$\Footnote{双各向异性中的磁$\to$电耦合(如果 $\bar{D}^{\;\!\mathcolor{gray}{t}}_{\;\!\mathcolor{gray}{z}}$ 中的该部分只是 $\bar{H}^{\;\!\mathcolor{gray}{t}}_{\;\!\mathcolor{gray}{z}}$ 的线性函数,则也可归结到线性项中)。}、应力 $\bar{T}^{\;\!\mathcolor{gray}{t}}_{\;\!\mathcolor{gray}{z}}$\Footnote{正逆压磁/磁致伸缩/磁弹效应(这里未作区分)。}等其他场量的非线性函数项,均由 $\bar{D}^{\;\!\textcolor{Maroon}{\text{NL}}, \mathcolor{gray}{t}}_{\;\!\mathcolor{gray}{z}} = {\symup{\varepsilon}}_0 \bar{M}^{\;\!\textcolor{Maroon}{\text{NL}}, \mathcolor{gray}{t}}_{\;\!\mathcolor{gray}{z}}$ 贡献;剩余的线性项,由 $\bar{B}^{\;\!\textcolor{Maroon}{\text{(1)}} \mathcolor{gray}{t}}_{\;\!\mathcolor{gray}{z}} = \bar{\bar{\mu}}^{\;\!\textcolor{Maroon}{\text{(1)}} \mathcolor{gray}{t}}_{\;\!\mathcolor{gray}{z}}~\widetilde *~\bar{H}^{\;\!\mathcolor{gray}{t}}_{\;\!\mathcolor{gray}{z}}$ 表示。


