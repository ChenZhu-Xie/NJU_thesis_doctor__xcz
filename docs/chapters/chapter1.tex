
% based on 北理工博士毕业论文 的 chapter1 https://tex.nju.edu.cn/project/user/10d0ed00-6f46-42fa-8a3a-18270214cfa3/a0ed1339-764d-4cc7-9ac4-d5d05b33da53

\label{第一章开始}
\chapter{绪论}

\section{本论文研究的目的和意义}

屈原:天问。\cite{berryOpticalSingularitiesBirefringent2003,ossikovskiConstitutiveRelationsOptically2021}。

\section{国内外研究现状及发展趋势}
%\label{sec:***} 可标注label

\subsection{形状记忆聚氨酯的形状记忆机理}
%\label{sec:features}

形状记忆聚合物(SMP)是继形状记忆合金后在80年代发展起来的一\Footnote{test}种新型形状记忆材料
\bref{fig:1-1}

\begin{figure}[htbp]
	\centering
	\includegraphics[width=0.7\textwidth]{D:/C2D/Desktop/article_fig/berry_mcleod_fig/fig_v3/fig_1.pdf}
	%	\caption{The core procedure for calculating the evolution of the optical vector field between any two sections within an arbitrary $\bar{\bar{\varepsilon}}$ material: sequentially left-multiply three eigensystem matrix fields. \textbf{b} An example of the workflow calculating the output field distribution (from the leftmost column) for the x, y, z components of the input 1064 nm pump, which incidents normally on the 15-mm-long KTP crystal with a 2$^{\circ}$ deviation from its optical axis. The vector pump (from the rightmost column) is composed of vertically polarized $\text{LG}^{p=2}_{l=50}$ and horizontally polarized $\text{HG}_{6,6}$.}\label{fig:r-1}
	\backcaption{1pt}{ddddddddddddddddddddddddddddddd}{fig:1-1}
\end{figure}


