\marginLeft[-2.4em]{chap:LFCO}\chapter{任意 \texorpdfstring{$\bar{\bar{\varepsilon}}$}{$\bar{\bar{\text{ε}}}$} 电介质内的矢量线性傅立叶晶体光学}\label{chap:LFCO}

\vspace*{-6.5em}

\marginLeft[-2.4em]{sec:eigen-analysis}\section{\textcolor{Maroon}{Eigen-analyze} 电磁双各向异性材料中的光场 \textcolor{Maroon}{Fourier spectrum}}\label{sec:eigen-analysis}

\bref{ssec:Exp-solution-linear} 提到,\textcolor{Plum}{线性}\textcolor{PineGreen}{晶体光学}的美,隐藏在\textcolor{Plum}{复各向异性}的“材料系数-\textcolor{PineGreen}{波矢}方向”矩阵 $\Xint{\begin{smallmatrix} ~ \\ {}^{}_{\mathcolor{gray}{-}} \\ ~ \end{smallmatrix}}{15}{\bar{\bar{k}}}_{\textcolor{Maroon}{\symup{z}}}^{\;\! \mathcolor{gray}{\omega}} \in \bar{\bar{\mathbb{C}}}_{\textcolor{Plum}{\left[3 \times 3\right]}}$ 的潜在\textcolor{Plum}{非对角化}(\textcolor{Plum}{缺陷})特质 \bref{eq:kz-non_diagonalization} 中。其中,分解 \bref{eq:e^ikzz-non_diagonalization2} 并不总是成立的。因此在特定的 $\{\mathcolor{gray}{\bar{k}_{\symup{\rho}}}, \mathcolor{gray}{\omega}, \bar{\bar{\varepsilon}}^{\;\! \mathcolor{gray}{\omega}}_{\textcolor{Maroon}{(1)}}, \bar{\bar{\zeta}}^{\;\! \mathcolor{gray}{\omega} \mathcolor{gray}{\check{1}}}_{\textcolor{Maroon}{(1)}}, \bar{\bar{\zeta}}^{\;\! \mathcolor{gray}{\omega} \mathcolor{gray}{\check{1} \check{2}}}_{\textcolor{Maroon}{(1)}}\}$ 组合\cite{grundmannSingularOpticalAxes2016}下,$\Xint{\begin{smallmatrix} ~ \\ {}^{}_{\mathcolor{gray}{-}} \\ ~ \end{smallmatrix}}{15}{\bar{\bar{k}}}_{\textcolor{Maroon}{\symup{z}}}^{\;\! \mathcolor{gray}{\omega}}$ 退化为缺陷矩阵,此时要么通过 $\Xint{\begin{smallmatrix} ~ \\ {}^{}_{\mathcolor{gray}{-}} \\ ~ \end{smallmatrix}}{15}{\bar{\bar{k}}}_{\textcolor{Maroon}{\symup{z}}}^{\;\! \mathcolor{gray}{\omega}}$ 的 \textcolor{PineGreen}{Jordan 标准型} \bref{eq:kz-Jordan_normal_form} 对其进行 \textcolor{PineGreen}{Jordan-Chevalley 分解}\cite{wiersigDistanceExceptionalPoints2022,wiersigRevisitingHierarchicalConstruction2022,wiersigMovingExceptionalSurface2023,wiersigReviewExceptionalPointbased2020,kirillovGeometricalOpticsStability2025,keckUnfoldingDiabolicPoint2003,kirillovUnfoldingEigenvalueSurfaces2005}(\bref{eq:kz-Jordan_decompose})以辅助求解\textcolor{Maroon}{矩阵指数} $\mathbb{e}^{\mathbb{i} \Xint{\begin{smallmatrix} ~ \\ {}^{}_{\mathcolor{gray}{-}} \\ ~ \end{smallmatrix}}{15}{\bar{\bar{k}}}_{\textcolor{Maroon}{\symup{z}}}^{\;\! \mathcolor{gray}{\omega}} \mathcolor{gray}{z}}$(\bref{eq:e^ikzz-Jordan_decompose}),并破除\textcolor{Maroon}{转移矩阵} \bref{eq:e^ikzz-non_diagonalization} 中父\textcolor{PineGreen}{特征向量} $\Xint{\begin{smallmatrix} ~ \\ {}^{}_{\mathcolor{gray}{-}} \\ ~ \end{smallmatrix}}{15}{\bar{v}}^{\;\!\mathcolor{gray}{\omega}}_{\textcolor{PineGreen}{\jmath}}$ 的简并,至广义子\textcolor{PineGreen}{特征向量} $\{\Xint{\begin{smallmatrix} ~ \\ {}^{}_{\mathcolor{gray}{-}} \\ ~ \end{smallmatrix}}{15}{\bar{v}}^{\;\!\mathcolor{gray}{\omega}}_{\textcolor{PineGreen}{\imath}}\}_j$,以真正实施探索并最终分辨清晰\textcolor{PineGreen}{光学奇点}的\textcolor{Plum}{内部结构};要么直接数值求解\textcolor{Maroon}{矩阵指数} $\mathbb{e}^{\mathbb{i} \Xint{\begin{smallmatrix} ~ \\ {}^{}_{\mathcolor{gray}{-}} \\ ~ \end{smallmatrix}}{15}{\bar{\bar{k}}}_{\textcolor{Maroon}{\symup{z}}}^{\;\! \mathcolor{gray}{\omega}} \mathcolor{gray}{z}}$\cite{zarifiPlaneWaveReflection2014,pessoaAvoidingMatrixExponentials2024},以事实上避免在这些\textcolor{PineGreen}{异常点}处计算发散。对 \bref{eq:kz-non_diagonalization} 中父\textcolor{PineGreen}{本征向量} $\Xint{\begin{smallmatrix} ~ \\ {}^{}_{\mathcolor{gray}{-}} \\ ~ \end{smallmatrix}}{15}{\bar{v}}^{\;\!\mathcolor{gray}{\omega}}_{\textcolor{PineGreen}{\jmath}}$ 的二/三(/四)重简并(代数重数 $f\{\Xint{\begin{smallmatrix} ~ \\ {}^{}_{\mathcolor{gray}{-}} \\ ~ \end{smallmatrix}}{22}{\bar{\lambda}}^{\;\!\mathcolor{gray}{\omega}}_{\textcolor{PineGreen}{\imath}}\}_j$ $\geq$ 几何重数 $f\{\Xint{\begin{smallmatrix} ~ \\ {}^{}_{\mathcolor{gray}{-}} \\ ~ \end{smallmatrix}}{15}{\bar{v}}^{\;\!\mathcolor{gray}{\omega}}_{\textcolor{PineGreen}{\imath}}\}_j$)的 \textcolor{PineGreen}{Jordan 链} 展开\cite{xieAnalytic3DVector},引发了对\textcolor{NavyBlue}{奇点光学}\cite{berryOpticalSingularitiesBirefringent2003,berryOpticalSingularitiesBianisotropic2005,kirillovUnfoldingEigenvalueSurfaces2005}、各种\textcolor{Plum}{非厄米}\cite{yangNonabelianPhysicsLight2024}\textcolor{Plum}{非正规}\cite{wiersigDistanceExceptionalPoints2022}系统中的高阶\textcolor{PineGreen}{例外点}\cite{mackayExceptionalGuidedWaves2021,wiersigMovingExceptionalSurface2023}及表现出\textcolor{Plum}{线性}、二次和三次坐标(等参数)依赖的\textcolor{PineGreen}{振幅}行为的\textcolor{Plum}{非均匀}\textcolor{PineGreen}{本征子波}\cite{lakhtakiaElectromagneticSurfaceWaves2020,gerardinConditionsVoigtWave2001,borzdovWavesLinearQuadratic1996,sturmElectromagneticWavesCrystals2024,sturmPropagationElectromagneticWaves}、超灵敏传感器\cite{wiersigReviewExceptionalPointbased2020,wiersigMovingExceptionalSurface2023}的深入研究。

然而,\bref{ssec:Exp-solution-linear} 中\textcolor{Maroon}{矩阵指数}形式的试探解 \bref{eq:matrix_exp},除了能帮助\textcolor{Plum}{逐个}(而无法\textcolor{Plum}{批量})理解和获得/揭示\textcolor{PineGreen}{光学奇点}们的\textcolor{Plum}{内部结构}(并且只能恰好在这些\textcolor{PineGreen}{例外点}处,而不包括其附近\Footnote{有趣的是,\textcolor{PineGreen}{线性叠加的\textcolor{gray}{单色}平面波基} \bref{eq:vec-plane_wave_basis} 也在\textcolor{PineGreen}{例外点}附近失效(取决于计算对象即\textcolor{PineGreen}{本征向量} $\Xint{{}^{}_{\mathcolor{gray}{-}}}{10}{\bar{g}}^{\;\!\mathcolor{gray}{\omega} \textcolor{PineGreen}{\jmath}}$ 的\textcolor{Plum}{因变量} $\mathcolor{gray}{\bar{k}_{\symup{\rho}}}$ 靠近\textcolor{PineGreen}{例外点}的程度,与浮点运算的机器精度上限的博弈),并在\textcolor{PineGreen}{例外点}处彻底失效。--- 这意味着,两种方法都无法处理\textcolor{PineGreen}{例外点}的邻域!以至于在这点上,似乎只剩直接计算\textcolor{Maroon}{矩阵指数}这一条路。}),相比 \bref{ssec:E-waveq-linear} 中\textcolor{PineGreen}{线性叠加的\textcolor{gray}{单色}平面波基}形式的试探解 \bref{eq:vec-plane_wave_basis},在形式和作用/效果上没有其他任何优势,反而存在 \bref{ssec:Exp-solution-linear} 末陈述的一系列劣势。据此,本节(以及本节以后的所有章节)完全把自身束缚在 \bref{ssec:E-waveq-linear} 的\textcolor{PineGreen}{线性叠加的\textcolor{gray}{单色}平面波基}框架内,并因此不(由于“无法再”)考虑奇点的\textcolor{Plum}{内部结构}(但仍考虑其外部属性,如其在 $\mathcolor{gray}{\bar{k}_{\symup{\rho}}}$ 域中的数量、

\clearpage

\noindent 位置,以及随参数变化的成对产生、合并湮灭、运动和分布情况等),如绝大多数“浅尝辄止”的\textcolor{PineGreen}{例外点}\cite{hernandezExceptionalPointsNonHermitian2011,hanExceptionalEntanglementPhenomena2023,baiObservationNonlinearExceptional2024,baiNonlinearExceptionalPoints2023}或\textcolor{PineGreen}{光学奇点}\cite{richterExceptionalPointsAnisotropic2017,grundmannSingularOpticalAxes2016,berryOpticalSingularitiesBirefringent2003,berryOpticalSingularitiesBianisotropic2005,grundmannOpticallyAnisotropicMedia2017}相关的文献一样。

\vspace*{-4.5em}

\marginLeft[-2.4em]{ssec:eigenmodes-compamp}\subsection{线性与非线性晶体光学的 2 个核心:本征模·复振幅}\label{ssec:eigenmodes-compamp}

\bref{ssec:E-waveq-linear} 对\textcolor{PineGreen}{线性叠加的\textcolor{gray}{单色}平面波基} \bref{eq:plane_wave_basis} 中的\textcolor{NavyBlue}{非相位部分},也即(电场)\textcolor{Maroon}{时空谱} $\Xint{{}^{}_{\mathcolor{gray}{-}}}{10}{\bar{g}}^{\;\!\mathcolor{gray}{\omega} \textcolor{PineGreen}{\jmath}}_{\;\! \mathcolor{gray}{z}}$,在 \bref{eq:L+V-polar} 的条件下,进行了下一层次的因式分解 \bref{eq:vec-amp_polar},并得到了分解得最“原子化的”\textcolor{PineGreen}{线性叠加的\textcolor{gray}{单色}平面波基} \bref{eq:amp_polar_phase}。其中,拆分出的不含 $\mathcolor{gray}{z}$ 的\textcolor{PineGreen}{本征偏振态} $\Xint{{}^{}_{\mathcolor{gray}{-}}}{10}{\bar{g}}^{\;\!\mathcolor{gray}{\omega} \textcolor{PineGreen}{\imath}}$,与\textcolor{Plum}{线性}算子 $\Xint{\mathcolor{gray}{-}}{30}{\bar{\bar{L}}}^{\;\! \mathcolor{gray}{\omega} \textcolor{PineGreen}{\imath}}$ 一起,构成了\textcolor{Plum}{线性}\textcolor{PineGreen}{晶体光学}的\textcolor{Maroon}{核心之一}:\textcolor{PineGreen}{特征方程}\Footnote{尽管形式上可以写作\cite{asoubarSimulationBirefringenceEffects2015,kirillovEigenvalueSurfacesDiabolic2005,berryOpticalSingularitiesBirefringent2003,xieAnalytic3DVector},但它不是标准\textcolor{PineGreen}{特征方程}形式。因此 $\Xint{\begin{smallmatrix} ~ \\ {}^{}_{\mathcolor{gray}{-}} \\ ~ \end{smallmatrix}}{15}{k}_{\symup{z}}^{\;\! \mathcolor{gray}{\omega} \textcolor{PineGreen}{\imath}}, \Xint{{}^{}_{\mathcolor{gray}{-}}}{10}{\bar{g}}^{\;\!\mathcolor{gray}{\omega} \textcolor{PineGreen}{\imath}}$ 不是标准\textcolor{PineGreen}{本征值}-\textcolor{PineGreen}{向量}对。} \bref{eq:nonlinear(2)-wave_wkrho-simplify6-L3''}。

满足该方程的\textcolor{PineGreen}{本征系统},即\textcolor{PineGreen}{本征值}-\textcolor{PineGreen}{本征向量}对 $\Xint{\begin{smallmatrix} ~ \\ {}^{}_{\mathcolor{gray}{-}} \\ ~ \end{smallmatrix}}{15}{k}_{\symup{z}}^{\;\! \mathcolor{gray}{\omega} \textcolor{PineGreen}{\imath}}, \Xint{{}^{}_{\mathcolor{gray}{-}}}{10}{\bar{g}}^{\;\!\mathcolor{gray}{\omega} \textcolor{PineGreen}{\imath}}$,构成了电磁\textcolor{PineGreen}{双各向异性}材料内(\textcolor{NavyBlue}{无源}\Footnote{两种源:\textcolor{gray}{频率转换}的\textcolor{NavyBlue}{驱动源}(如 \bref{eq:nonlinear(2)-wave_wkrho-simplify4} 右一)、材料系数\textcolor{Plum}{不均匀}起伏诱导的散射源(如 \bref{eq:nonlinear(2)-wave_wkrho-simplify4-2})。}\textcolor{NavyBlue}{被动}\textcolor{Plum}{均匀}\textcolor{Plum}{各向异性}衍射)光场的\textcolor{PineGreen}{本征模}式
\begin{subequations} \label{eq:polar_phase}
	\begin{align}
		\Xint{\mathcolor{gray}{-}}{20}{\bar{G}}^{\;\!\mathcolor{gray}{\omega} \textcolor{PineGreen}{\jmath}}_{\;\! \mathcolor{gray}{z}} &:= \Xint{{}^{}_{\mathcolor{gray}{-}}}{10}{\bar{g}}^{\;\!\mathcolor{gray}{\omega} \textcolor{PineGreen}{\jmath}} \mathbb{e}^{\mathbb{i} \Xint{\begin{smallmatrix} ~ \\ {}^{}_{\mathcolor{gray}{-}} \\ ~ \end{smallmatrix}}{15}{k}_{\symup{z}}^{\;\! \mathcolor{gray}{\omega} \textcolor{PineGreen}{\jmath}} \mathcolor{gray}{z}} ~, \label{eq:vec-polar_phase} \\
		\Xint{\mathcolor{gray}{-}}{20}{G}^{\;\!\mathcolor{gray}{\omega} \textcolor{PineGreen}{\jmath}}_{\;\! \hat{1} \mathcolor{gray}{z}} &:= \Xint{{}^{}_{\mathcolor{gray}{-}}}{10}{g}^{\;\!\mathcolor{gray}{\omega} \textcolor{PineGreen}{\jmath}}_{\;\! \hat{1}} \mathbb{e}^{\mathbb{i} \Xint{\begin{smallmatrix} ~ \\ {}^{}_{\mathcolor{gray}{-}} \\ ~ \end{smallmatrix}}{15}{k}_{\symup{z}}^{\;\! \mathcolor{gray}{\omega} \textcolor{PineGreen}{\jmath}} \mathcolor{gray}{z}} ~, \label{eq:components-polar_phase}
	\end{align}
\end{subequations}
\textcolor{PineGreen}{特征方程} \bref{eq:nonlinear(2)-wave_wkrho-simplify6-L3''} 一共可解出 6 对(或 4 对)\textcolor{PineGreen}{本征模}\Footnote{在线性\textcolor{PineGreen}{晶体光学}层面,如果不忽略 \textcolor{NavyBlue}{磁偶-电四/磁偶}极、\textcolor{NavyBlue}{电偶-电八/磁四}极(子/矩的)贡献,即不忽略 \bref{eq:p<->n} 中的电磁交叉耦合系数 $\bar{\bar{\zeta}}^{\;\! \mathcolor{gray}{\omega} \mathcolor{gray}{\check{1} \check{2}}}_{\textcolor{Maroon}{(1)}}$ 的绝大部分分量时,$\textcolor{Plum}{\det} \left[ \Xint{\mathcolor{gray}{-}}{28}{\bar{\bar{L}}}^{\;\! \mathcolor{gray}{\omega} \textcolor{PineGreen}{\imath}} \right] = \textcolor{Plum}{\det} \left[ \bar{\bar{\varepsilon}}^{\;\! \mathcolor{gray}{\omega}}_{\textcolor{Maroon}{(1)}} + \bar{\bar{\zeta}}^{\;\! \mathcolor{gray}{\omega} \mathcolor{gray}{\check{1}}}_{\textcolor{Maroon}{(1)}} \mathbb{i} \Xint{\begin{smallmatrix} ~ \\ {}^{}_{\mathcolor{gray}{-}} \\ ~ \end{smallmatrix}}{15}{k}_{\;\! \mathcolor{gray}{\check{1}}}^{\;\! \mathcolor{gray}{\omega} \textcolor{PineGreen}{\imath}} - \bar{\bar{\zeta}}^{\;\! \mathcolor{gray}{\omega} \mathcolor{gray}{\check{1} \check{2}}}_{\textcolor{Maroon}{(1)}} \Xint{\begin{smallmatrix} ~ \\ {}^{}_{\mathcolor{gray}{-}} \\ ~ \end{smallmatrix}}{15}{k}_{\;\! \mathcolor{gray}{\check{2}}}^{\;\! \mathcolor{gray}{\omega} \textcolor{PineGreen}{\imath}} \Xint{\begin{smallmatrix} ~ \\ {}^{}_{\mathcolor{gray}{-}} \\ ~ \end{smallmatrix}}{15}{k}_{\;\! \mathcolor{gray}{\check{1}}}^{\;\! \mathcolor{gray}{\omega} \textcolor{PineGreen}{\imath}} \right]$ 一般是关于 $\Xint{\begin{smallmatrix} ~ \\ {}^{}_{\mathcolor{gray}{-}} \\ ~ \end{smallmatrix}}{15}{k}_{\symup{z}}^{\;\! \mathcolor{gray}{\omega} \textcolor{PineGreen}{\imath}}$ 的 6 次方程\cite{raabMultipoleTheoryElectromagnetism2004},否则一般是关于 $\Xint{\begin{smallmatrix} ~ \\ {}^{}_{\mathcolor{gray}{-}} \\ ~ \end{smallmatrix}}{15}{k}_{\symup{z}}^{\;\! \mathcolor{gray}{\omega} \textcolor{PineGreen}{\imath}}$ 的 \textcolor{PineGreen}{Booker} 4 次方程\cite{chenCoordinatefreeApproachWave1981,chenTheoryElectromagneticWaves1983,buddenPropagationRadioWaves1985,abdulhalimExactMatrixMethod1999,changWavePropagationBianisotropic2014}。}。查看这些的\textcolor{PineGreen}{模式}的 $\textcolor{Plum}{\text{Re}} \left[ \Xint{\begin{smallmatrix} ~ \\ {}^{}_{\mathcolor{gray}{-}} \\ ~ \end{smallmatrix}}{20}{s}^{\;\! \mathcolor{gray}{\omega} \textcolor{PineGreen}{\imath}}_{\mathrm{z}} \right] := \textcolor{Plum}{\text{Re}} \left[ \left( \Xint{{}^{}_{\mathcolor{gray}{-}}}{10}{\bar{g}}^{\;\!\mathcolor{gray}{\omega} \textcolor{PineGreen}{\imath}} \times \Xint{\begin{smallmatrix} ~ \\ {}^{}_{\mathcolor{gray}{-}} \\ ~ \end{smallmatrix}}{11}{\bar{h}}^{\;\!\mathcolor{gray}{\omega} \textcolor{PineGreen}{\imath}} \right)_{\mathrm{z}} \right]$ 的正负,可以区分出来 3 个(或 2 个)\textcolor{Plum}{正向传播}的\textcolor{PineGreen}{模式},以及相等数量的\textcolor{Plum}{反向传播}\textcolor{PineGreen}{模式}\cite{asoubarSimulationBirefringenceEffects2015,zhangFullyVectorialSimulation2016,liReformulationFourierModal1998},其中允许一些\textcolor{PineGreen}{模式}的\textcolor{PineGreen}{群}/\textcolor{PineGreen}{相速度}的 $\symup{z}$ 分量接近或等于 0,即不传播(但不是驻波)。这些\textcolor{PineGreen}{本征模}的\textcolor{PineGreen}{本征向量},一般接近“\textcolor{Plum}{相互垂直}”,且至少包含 2 个横模,有可能还额外包含 1 个纵模\cite{raabMultipoleTheoryElectromagnetism2004}。

至少 2 个、最多 3 个互相\textcolor{Plum}{独立}(但不一定\textcolor{Plum}{复正交})的同向\textcolor{PineGreen}{本征模} $\Xint{\mathcolor{gray}{-}}{16}{\bar{G}}^{\;\!\mathcolor{gray}{\omega} \textcolor{PineGreen}{\jmath}}_{\;\! \mathcolor{gray}{z}}$ 张/组成该传播方向上完备的\textcolor{Plum}{解空间},并以一定的\textcolor{PineGreen}{本征复振幅} $\Xint{\begin{smallmatrix} ~ \\ {}^{}_{\mathcolor{gray}{-}} \\ ~ \end{smallmatrix}}{09}{\mathtt{g}}^{\;\!\mathcolor{gray}{\omega} \textcolor{PineGreen}{\jmath}}_{\;\! \mathcolor{gray}{z}}$ 系数比例\Footnote{各\textcolor{PineGreen}{模式}系数的绝对值和之间的相对比例,是允许含时含空变化的:初始比例由\textcolor{Maroon}{边界条件} \bref{eq:1BC} 确定,后续配置比例变化由\textcolor{Plum}{非线性}\textcolor{gray}{混频}过程驱动、并由具体的\textcolor{PineGreen}{波矢} $\Xint{\begin{smallmatrix} ~ \\ {}^{}_{\mathcolor{gray}{-}} \\ ~ \end{smallmatrix}}{15}{\bar{k}}^{\;\! \mathcolor{gray}{\omega} \textcolor{PineGreen}{\jmath}}$ \textcolor{PineGreen}{相位匹配条件}和场 $\Xint{\mathcolor{gray}{-}}{25}{\bar{E}}^{\;\!\mathcolor{gray}{\omega} \textcolor{PineGreen}{\jmath}}_{\;\! \mathcolor{gray}{z}}$ \textcolor{Plum}{交叠积分}确定。}\textcolor{PineGreen}{线性叠加}后,即得\textcolor{PineGreen}{特征方程} \bref{eq:nonlinear(2)-wave_wkrho-simplify6-L3''} 通解,也即 \bref{eq:amp_polar_phase,eq:plane_wave_basis} 的第三种表示
\begin{subequations} \label{eq:amp_eigenmode}
	\begin{align}
		\Xint{\mathcolor{gray}{-}}{30}{\bar{E}}^{\;\!\mathcolor{gray}{\omega}}_{\;\! \mathcolor{gray}{z}} &:= \leftindex_{\textcolor{PineGreen}{\jmath}} \;\! \Xint{\mathcolor{gray}{-}}{30}{\bar{E}}^{\;\!\mathcolor{gray}{\omega} \textcolor{PineGreen}{\jmath}}_{\;\! \mathcolor{gray}{z}} := \Xint{\begin{smallmatrix} ~ \\ {}^{}_{\mathcolor{gray}{-}} \\ ~ \end{smallmatrix}}{09}{\mathtt{g}}^{\;\!\mathcolor{gray}{\omega}}_{\;\! \mathcolor{gray}{z} \textcolor{PineGreen}{\jmath}} \Xint{\mathcolor{gray}{-}}{20}{\bar{G}}^{\;\!\mathcolor{gray}{\omega} \textcolor{PineGreen}{\jmath}}_{\;\! \mathcolor{gray}{z}} ~, \label{eq:vec-amp_eigenmode} \\
		\Xint{\mathcolor{gray}{-}}{30}{E}^{\;\!\mathcolor{gray}{\omega}}_{\;\! \hat{1} \mathcolor{gray}{z}} &:= \leftindex_{\textcolor{PineGreen}{\jmath}} \;\! \Xint{\mathcolor{gray}{-}}{30}{E}^{\;\!\mathcolor{gray}{\omega} \textcolor{PineGreen}{\jmath}}_{\;\! \hat{1} \mathcolor{gray}{z}} := \Xint{\begin{smallmatrix} ~ \\ {}^{}_{\mathcolor{gray}{-}} \\ ~ \end{smallmatrix}}{09}{\mathtt{g}}^{\;\!\mathcolor{gray}{\omega}}_{\;\! \mathcolor{gray}{z} \textcolor{PineGreen}{\jmath}} \Xint{\mathcolor{gray}{-}}{20}{G}^{\;\!\mathcolor{gray}{\omega} \textcolor{PineGreen}{\jmath}}_{\;\! \hat{1} \mathcolor{gray}{z}} ~, \label{eq:components-amp_eigenmode}
	\end{align}
\end{subequations}
注意,\bref{eq:amp_polar_phase,eq:plane_wave_basis,eq:amp_eigenmode} 中均隐式地定义了\textcolor{PineGreen}{本征波}
\begin{subequations} \label{eq:eigenwave}
	\begin{align}
		\Xint{\mathcolor{gray}{-}}{30}{\bar{E}}^{\;\!\mathcolor{gray}{\omega} \textcolor{PineGreen}{\jmath}}_{\;\! \mathcolor{gray}{z}} := \Xint{\begin{smallmatrix} ~ \\ {}^{}_{\mathcolor{gray}{-}} \\ ~ \end{smallmatrix}}{09}{\mathtt{g}}^{\;\!\mathcolor{gray}{\omega} \textcolor{PineGreen}{\jmath}}_{\;\! \mathcolor{gray}{z}} \Xint{\mathcolor{gray}{-}}{20}{\bar{G}}^{\;\!\mathcolor{gray}{\omega} \textcolor{PineGreen}{\jmath}}_{\;\! \mathcolor{gray}{z}} := \Xint{\begin{smallmatrix} ~ \\ {}^{}_{\mathcolor{gray}{-}} \\ ~ \end{smallmatrix}}{09}{\mathtt{g}}^{\;\!\mathcolor{gray}{\omega} \textcolor{PineGreen}{\jmath}}_{\;\! \mathcolor{gray}{z}} \Xint{{}^{}_{\mathcolor{gray}{-}}}{10}{\bar{g}}^{\;\!\mathcolor{gray}{\omega} \textcolor{PineGreen}{\jmath}} \mathbb{e}^{\mathbb{i} \Xint{\begin{smallmatrix} ~ \\ {}^{}_{\mathcolor{gray}{-}} \\ ~ \end{smallmatrix}}{15}{k}_{\symup{z}}^{\;\! \mathcolor{gray}{\omega} \textcolor{PineGreen}{\jmath}} \mathcolor{gray}{z}} =: \Xint{{}^{}_{\mathcolor{gray}{-}}}{10}{\bar{g}}^{\;\!\mathcolor{gray}{\omega} \textcolor{PineGreen}{\jmath}}_{\;\! \mathcolor{gray}{z}} \mathbb{e}^{\mathbb{i} \Xint{\begin{smallmatrix} ~ \\ {}^{}_{\mathcolor{gray}{-}} \\ ~ \end{smallmatrix}}{15}{k}_{\symup{z}}^{\;\! \mathcolor{gray}{\omega} \textcolor{PineGreen}{\jmath}} \mathcolor{gray}{z}} ~, \label{eq:vec-eigenwave} \\
		\Xint{\mathcolor{gray}{-}}{30}{E}^{\;\!\mathcolor{gray}{\omega} \textcolor{PineGreen}{\jmath}}_{\;\! \hat{1} \mathcolor{gray}{z}} := \Xint{\begin{smallmatrix} ~ \\ {}^{}_{\mathcolor{gray}{-}} \\ ~ \end{smallmatrix}}{09}{\mathtt{g}}^{\;\!\mathcolor{gray}{\omega} \textcolor{PineGreen}{\jmath}}_{\;\! \mathcolor{gray}{z}} \Xint{\mathcolor{gray}{-}}{20}{G}^{\;\!\mathcolor{gray}{\omega} \textcolor{PineGreen}{\jmath}}_{\;\! \hat{1} \mathcolor{gray}{z}} := \Xint{\begin{smallmatrix} ~ \\ {}^{}_{\mathcolor{gray}{-}} \\ ~ \end{smallmatrix}}{09}{\mathtt{g}}^{\;\!\mathcolor{gray}{\omega} \textcolor{PineGreen}{\jmath}}_{\;\! \mathcolor{gray}{z}} \Xint{{}^{}_{\mathcolor{gray}{-}}}{10}{g}^{\;\!\mathcolor{gray}{\omega} \textcolor{PineGreen}{\jmath}}_{\;\! \hat{1}} \mathbb{e}^{\mathbb{i} \Xint{\begin{smallmatrix} ~ \\ {}^{}_{\mathcolor{gray}{-}} \\ ~ \end{smallmatrix}}{15}{k}_{\symup{z}}^{\;\! \mathcolor{gray}{\omega} \textcolor{PineGreen}{\jmath}} \mathcolor{gray}{z}} =: \Xint{{}^{}_{\mathcolor{gray}{-}}}{10}{g}^{\;\!\mathcolor{gray}{\omega} \textcolor{PineGreen}{\jmath}}_{\;\! \hat{1} \mathcolor{gray}{z}} \mathbb{e}^{\mathbb{i} \Xint{\begin{smallmatrix} ~ \\ {}^{}_{\mathcolor{gray}{-}} \\ ~ \end{smallmatrix}}{15}{k}_{\symup{z}}^{\;\! \mathcolor{gray}{\omega} \textcolor{PineGreen}{\jmath}} \mathcolor{gray}{z}} ~, \label{eq:components-eigenwave}
	\end{align}
\end{subequations}
相对于\textcolor{PineGreen}{本征模} \bref{eq:polar_phase},在定义上多乘上了一个含 $\mathcolor{gray}{z}$ 的\textcolor{PineGreen}{本征复振幅}系数 $\Xint{\begin{smallmatrix} ~ \\ {}^{}_{\mathcolor{gray}{-}} \\ ~ \end{smallmatrix}}{09}{\mathtt{g}}^{\;\!\mathcolor{gray}{\omega} \textcolor{PineGreen}{\jmath}}_{\;\! \mathcolor{gray}{z}}$,允许按比例缩放,多提供了一个额外的复标量\textcolor{Plum}{自由度},以匹配初始的\textcolor{Maroon}{边界条件} \bref{eq:1BC},和允许后续的\textcolor{Plum}{非线性}\textcolor{gray}{频率转换}过程 \bref{eq:simplify7-LE0-SVA-V_1singular-nokxky-zeta-g} 导致\textcolor{PineGreen}{本征复振幅}变化。

注意到,从排列组合的角度,\textcolor{PineGreen}{本征波} \bref{eq:eigenwave} 还允许有(至多)第四种定义
\begin{subequations} \label{eq:eigenwave'}
	\begin{align}
		\Xint{\mathcolor{gray}{-}}{30}{\bar{E}}^{\;\!\mathcolor{gray}{\omega} \textcolor{PineGreen}{\jmath}}_{\;\! \mathcolor{gray}{z}} := \Xint{\mathcolor{gray}{-}}{20}{\mathtt{G}}^{\;\!\mathcolor{gray}{\omega} \textcolor{PineGreen}{\jmath}}_{\;\! \mathcolor{gray}{z}} \Xint{{}^{}_{\mathcolor{gray}{-}}}{10}{\bar{g}}^{\;\!\mathcolor{gray}{\omega} \textcolor{PineGreen}{\jmath}} := \Xint{\begin{smallmatrix} ~ \\ {}^{}_{\mathcolor{gray}{-}} \\ ~ \end{smallmatrix}}{09}{\mathtt{g}}^{\;\!\mathcolor{gray}{\omega} \textcolor{PineGreen}{\jmath}}_{\;\! \mathcolor{gray}{z}} \mathbb{e}^{\mathbb{i} \Xint{\begin{smallmatrix} ~ \\ {}^{}_{\mathcolor{gray}{-}} \\ ~ \end{smallmatrix}}{15}{k}_{\symup{z}}^{\;\! \mathcolor{gray}{\omega} \textcolor{PineGreen}{\jmath}} \mathcolor{gray}{z}} \Xint{{}^{}_{\mathcolor{gray}{-}}}{10}{\bar{g}}^{\;\!\mathcolor{gray}{\omega} \textcolor{PineGreen}{\jmath}} ~, \label{eq:vec-eigenwave'} \\
		\Xint{\mathcolor{gray}{-}}{30}{E}^{\;\!\mathcolor{gray}{\omega} \textcolor{PineGreen}{\jmath}}_{\;\! \hat{1} \mathcolor{gray}{z}} := \Xint{\mathcolor{gray}{-}}{20}{\mathtt{G}}^{\;\!\mathcolor{gray}{\omega} \textcolor{PineGreen}{\jmath}}_{\;\! \mathcolor{gray}{z}} \Xint{{}^{}_{\mathcolor{gray}{-}}}{10}{g}^{\;\!\mathcolor{gray}{\omega} \textcolor{PineGreen}{\jmath}}_{\;\! \hat{1}} := \Xint{\begin{smallmatrix} ~ \\ {}^{}_{\mathcolor{gray}{-}} \\ ~ \end{smallmatrix}}{09}{\mathtt{g}}^{\;\!\mathcolor{gray}{\omega} \textcolor{PineGreen}{\jmath}}_{\;\! \mathcolor{gray}{z}} \mathbb{e}^{\mathbb{i} \Xint{\begin{smallmatrix} ~ \\ {}^{}_{\mathcolor{gray}{-}} \\ ~ \end{smallmatrix}}{15}{k}_{\symup{z}}^{\;\! \mathcolor{gray}{\omega} \textcolor{PineGreen}{\jmath}} \mathcolor{gray}{z}} \Xint{{}^{}_{\mathcolor{gray}{-}}}{10}{g}^{\;\!\mathcolor{gray}{\omega} \textcolor{PineGreen}{\jmath}}_{\;\! \hat{1}} ~, \label{eq:components-eigenwave'}
	\end{align}
\end{subequations}
其中,定义了对(矢量)\textcolor{Plum}{非线性}(\textcolor{Maroon}{傅立叶})\textcolor{PineGreen}{晶体光学}很重要的\textcolor{PineGreen}{含衍射本征复振幅}
\begin{align} \label{eq:amp_phase}
	\Xint{\mathcolor{gray}{-}}{20}{\mathtt{G}}^{\;\!\mathcolor{gray}{\omega} \textcolor{PineGreen}{\jmath}}_{\;\! \mathcolor{gray}{z}} &:= \Xint{\begin{smallmatrix} ~ \\ {}^{}_{\mathcolor{gray}{-}} \\ ~ \end{smallmatrix}}{09}{\mathtt{g}}^{\;\!\mathcolor{gray}{\omega} \textcolor{PineGreen}{\jmath}}_{\;\! \mathcolor{gray}{z}} \mathbb{e}^{\mathbb{i} \Xint{\begin{smallmatrix} ~ \\ {}^{}_{\mathcolor{gray}{-}} \\ ~ \end{smallmatrix}}{15}{k}_{\symup{z}}^{\;\! \mathcolor{gray}{\omega} \textcolor{PineGreen}{\jmath}} \mathcolor{gray}{z}} ~, 
\end{align}
相应地,\bref{eq:amp_polar_phase,eq:plane_wave_basis,eq:amp_eigenmode} 之外也存在第四种\textcolor{PineGreen}{线性叠加的\textcolor{gray}{单色}平面波}公式
\begin{subequations} \label{eq:G_polar}
	\abovedisplayskip=-8pt
	\begin{align}
		\Xint{\mathcolor{gray}{-}}{30}{\bar{E}}^{\;\!\mathcolor{gray}{\omega}}_{\;\! \mathcolor{gray}{z}} &:= \leftindex_{\textcolor{PineGreen}{\jmath}} \;\! \Xint{\mathcolor{gray}{-}}{30}{\bar{E}}^{\;\!\mathcolor{gray}{\omega} \textcolor{PineGreen}{\jmath}}_{\;\! \mathcolor{gray}{z}} := \Xint{{}^{}_{\mathcolor{gray}{-}}}{10}{\bar{g}}^{\;\!\mathcolor{gray}{\omega}}_{\;\! \textcolor{PineGreen}{\jmath}} \Xint{\mathcolor{gray}{-}}{20}{\mathtt{G}}^{\;\!\mathcolor{gray}{\omega} \textcolor{PineGreen}{\jmath}}_{\;\! \mathcolor{gray}{z}} := \Xint{{}^{}_{\mathcolor{gray}{-}}}{10}{\bar{g}}^{\;\!\mathcolor{gray}{\omega}}_{\;\! \textcolor{PineGreen}{\jmath}} \Xint{\begin{smallmatrix} ~ \\ {}^{}_{\mathcolor{gray}{-}} \\ ~ \end{smallmatrix}}{09}{\mathtt{g}}^{\;\!\mathcolor{gray}{\omega} \textcolor{PineGreen}{\jmath}}_{\;\! \mathcolor{gray}{z}} \mathbb{e}^{\mathbb{i} \Xint{\begin{smallmatrix} ~ \\ {}^{}_{\mathcolor{gray}{-}} \\ ~ \end{smallmatrix}}{15}{k}_{\symup{z}}^{\;\! \mathcolor{gray}{\omega} \textcolor{PineGreen}{\jmath}} \mathcolor{gray}{z}} ~, \label{eq:vec-G_polar} \\
		\Xint{\mathcolor{gray}{-}}{30}{E}^{\;\!\mathcolor{gray}{\omega}}_{\;\! \hat{1} \mathcolor{gray}{z}} &:= \leftindex_{\textcolor{PineGreen}{\jmath}} \;\! \Xint{\mathcolor{gray}{-}}{30}{E}^{\;\!\mathcolor{gray}{\omega} \textcolor{PineGreen}{\jmath}}_{\;\! \hat{1} \mathcolor{gray}{z}} := \Xint{{}^{}_{\mathcolor{gray}{-}}}{10}{g}^{\;\!\mathcolor{gray}{\omega}}_{\;\! \hat{1} \textcolor{PineGreen}{\jmath}} \Xint{\mathcolor{gray}{-}}{20}{\mathtt{G}}^{\;\!\mathcolor{gray}{\omega} \textcolor{PineGreen}{\jmath}}_{\;\! \mathcolor{gray}{z}} := \Xint{{}^{}_{\mathcolor{gray}{-}}}{10}{g}^{\;\!\mathcolor{gray}{\omega}}_{\;\! \hat{1} \textcolor{PineGreen}{\jmath}} \Xint{\begin{smallmatrix} ~ \\ {}^{}_{\mathcolor{gray}{-}} \\ ~ \end{smallmatrix}}{09}{\mathtt{g}}^{\;\!\mathcolor{gray}{\omega} \textcolor{PineGreen}{\jmath}}_{\;\! \mathcolor{gray}{z}} \mathbb{e}^{\mathbb{i} \Xint{\begin{smallmatrix} ~ \\ {}^{}_{\mathcolor{gray}{-}} \\ ~ \end{smallmatrix}}{15}{k}_{\symup{z}}^{\;\! \mathcolor{gray}{\omega} \textcolor{PineGreen}{\jmath}} \mathcolor{gray}{z}} ~, \label{eq:components-G_polar}
	\end{align}
\end{subequations}
利用矢量形式的\textcolor{Plum}{广义列向量}\Footnote{这里(它处 \bypertarget{another-bra-example} 不一定)隐含地定义了:本征\textcolor{PineGreen}{模式}符号 $\textcolor{PineGreen}{\jmath}$ 作为下标的 $\Xint{{}^{}_{\mathcolor{gray}{-}}}{10}{\bar{g}}^{\;\!\mathcolor{gray}{\omega}}_{\;\! \textcolor{PineGreen}{\jmath}}$ 被“翻译为”广义行向量;而将 $\textcolor{PineGreen}{\jmath}$ 在上标的 $\Xint{\mathcolor{gray}{-}}{16}{\mathtt{G}}^{\;\!\mathcolor{gray}{\omega} \textcolor{PineGreen}{\jmath}}_{\;\! \mathcolor{gray}{z}}$ “翻译为”广义列向量。这一点与协变 $\bra{\text{bra}}$、逆变 $\ket{\text{ket}}$ 指标,及余切、切矢量的指标,的位置类似。}替代标量形式的\textcolor{Plum}{爱因斯坦求和},\bref{eq:vec-G_polar,eq:vec-amp_eigenmode} 还可以进一步写为矩阵的形式
\begin{subequations}
	\begin{align}	\Xint{\mathcolor{gray}{-}}{30}{\bar{E}}^{\;\!\mathcolor{gray}{\omega}}_{\;\! \mathcolor{gray}{z}} \xrightarrow[]{\text{\bref{eq:vec-amp_eigenmode-matrix}}} \overline{\Xint{\mathcolor{gray}{-}}{20}{\bar{G}}^{\;\!\mathcolor{gray}{\omega}}_{\;\! \mathcolor{gray}{z} \textcolor{PineGreen}{\jmath}}}^{\mathsf{\textcolor{Plum}{T}}} \cdot \overline{\Xint{\begin{smallmatrix} ~ \\ {}^{}_{\mathcolor{gray}{-}} \\ ~ \end{smallmatrix}}{09}{\mathtt{g}}^{\;\!\mathcolor{gray}{\omega}}_{\;\! \mathcolor{gray}{z} \textcolor{PineGreen}{\jmath}}} &\xrightarrow[]{\text{\bref{eq:vec-polar_phase}}} \overline{\Xint{{}^{}_{\mathcolor{gray}{-}}}{10}{\bar{g}}^{\;\!\mathcolor{gray}{\omega}}_{\;\! \textcolor{PineGreen}{\jmath}} \mathbb{e}^{\mathbb{i} \Xint{\begin{smallmatrix} ~ \\ {}^{}_{\mathcolor{gray}{-}} \\ ~ \end{smallmatrix}}{15}{k}_{\symup{z} \textcolor{PineGreen}{\jmath}}^{\;\! \mathcolor{gray}{\omega}} \mathcolor{gray}{z}}}^{\mathsf{\textcolor{Plum}{T}}} \cdot \overline{\Xint{\begin{smallmatrix} ~ \\ {}^{}_{\mathcolor{gray}{-}} \\ ~ \end{smallmatrix}}{09}{\mathtt{g}}^{\;\!\mathcolor{gray}{\omega}}_{\;\! \mathcolor{gray}{z} \textcolor{PineGreen}{\jmath}}} \label{eq:vec-eigenmode_amp-matrix} \\
		\xrightarrow[]{\text{\bref{eq:vec-G_polar}}} \overline{\Xint{{}^{}_{\mathcolor{gray}{-}}}{10}{\bar{g}}^{\;\!\mathcolor{gray}{\omega}}_{\;\! \textcolor{PineGreen}{\jmath}}}^{\mathsf{\textcolor{Plum}{T}}} \cdot \overline{\Xint{\mathcolor{gray}{-}}{20}{\mathtt{G}}^{\;\!\mathcolor{gray}{\omega}}_{\;\! \mathcolor{gray}{z} \textcolor{PineGreen}{\jmath}}} &\xrightarrow[]{\text{\bref{eq:amp_phase}}} \overline{\Xint{{}^{}_{\mathcolor{gray}{-}}}{10}{\bar{g}}^{\;\!\mathcolor{gray}{\omega}}_{\;\! \textcolor{PineGreen}{\jmath}}}^{\mathsf{\textcolor{Plum}{T}}} \cdot \overline{\Xint{\begin{smallmatrix} ~ \\ {}^{}_{\mathcolor{gray}{-}} \\ ~ \end{smallmatrix}}{09}{\mathtt{g}}^{\;\!\mathcolor{gray}{\omega}}_{\;\! \mathcolor{gray}{z} \textcolor{PineGreen}{\jmath}} \mathbb{e}^{\mathbb{i} \Xint{\begin{smallmatrix} ~ \\ {}^{}_{\mathcolor{gray}{-}} \\ ~ \end{smallmatrix}}{15}{k}_{\symup{z} \textcolor{PineGreen}{\jmath}}^{\;\! \mathcolor{gray}{\omega}} \mathcolor{gray}{z}}} \label{eq:vec-polar_G-matrix} \\ 
		&= \overline{\Xint{{}^{}_{\mathcolor{gray}{-}}}{10}{\bar{g}}^{\;\!\mathcolor{gray}{\omega}}_{\;\! \textcolor{PineGreen}{\jmath}}}^{\mathsf{\textcolor{Plum}{T}}} \cdot \overline{\overline{\mathbb{e}^{\mathbb{i} \Xint{\begin{smallmatrix} ~ \\ {}^{}_{\mathcolor{gray}{-}} \\ ~ \end{smallmatrix}}{15}{k}_{\symup{z} \textcolor{PineGreen}{\jmath}}^{\;\! \mathcolor{gray}{\omega}} \mathcolor{gray}{z}}}} \cdot \overline{\Xint{\begin{smallmatrix} ~ \\ {}^{}_{\mathcolor{gray}{-}} \\ ~ \end{smallmatrix}}{09}{\mathtt{g}}^{\;\!\mathcolor{gray}{\omega}}_{\;\! \mathcolor{gray}{z} \textcolor{PineGreen}{\jmath}}} ~, \label{eq:pre-on_the_way-transition_matrix}
	\end{align}
\end{subequations}
其中,前两个 $\textcolor{Plum}{3 \times 3}$ 矩阵 $\overline{\Xint{{}^{}_{\mathcolor{gray}{-}}}{10}{\bar{g}}^{\;\!\mathcolor{gray}{\omega}}_{\;\! \textcolor{PineGreen}{\jmath}}}^{\mathsf{\textcolor{Plum}{T}}}$, $\overline{\overline{\mathbb{e}^{\mathbb{i} \Xint{\begin{smallmatrix} ~ \\ {}^{}_{\mathcolor{gray}{-}} \\ ~ \end{smallmatrix}}{15}{k}_{\symup{z} \textcolor{PineGreen}{\jmath}}^{\;\! \mathcolor{gray}{\omega}} \mathcolor{gray}{z}}}}$ 分别由 3 对\textcolor{PineGreen}{特征方程} \bref{eq:nonlinear(2)-wave_wkrho-simplify6-L3''} 的\textcolor{PineGreen}{特征向量} $\Xint{{}^{}_{\mathcolor{gray}{-}}}{10}{\bar{g}}^{\;\!\mathcolor{gray}{\omega}}_{\;\! \textcolor{PineGreen}{\jmath}}$和\textcolor{PineGreen}{特征值}$\Xint{\begin{smallmatrix} ~ \\ {}^{}_{\mathcolor{gray}{-}} \\ ~ \end{smallmatrix}}{15}{k}_{\symup{z} \textcolor{PineGreen}{\jmath}}^{\;\! \mathcolor{gray}{\omega}}$决定,剩下的唯一任务是:确定由对应的 3 个标量场\textcolor{PineGreen}{基系数}/\textcolor{PineGreen}{本征复振幅} $\Xint{\begin{smallmatrix} ~ \\ {}^{}_{\mathcolor{gray}{-}} \\ ~ \end{smallmatrix}}{09}{\mathtt{g}}^{\;\!\mathcolor{gray}{\omega}}_{\;\! \mathcolor{gray}{z} \textcolor{PineGreen}{\jmath}}$ 构造的最后一个 $\textcolor{Plum}{3 \times 1}$ 列矢量场 $\overline{\Xint{\begin{smallmatrix} ~ \\ {}^{}_{\mathcolor{gray}{-}} \\ ~ \end{smallmatrix}}{09}{\mathtt{g}}^{\;\!\mathcolor{gray}{\omega}}_{\;\! \mathcolor{gray}{z} \textcolor{PineGreen}{\jmath}}}$。这是\textcolor{Plum}{线性}和\textcolor{Plum}{非线性}\textcolor{PineGreen}{晶体光学}共同面临的\textcolor{Maroon}{核心之二}。--- 也就是说,如果说计算\textcolor{PineGreen}{本征方程} \bref{eq:nonlinear(2)-wave_wkrho-simplify6-L3''} 的\textcolor{PineGreen}{本征系统} or \textcolor{PineGreen}{本征模}式 \bref{eq:vec-polar_phase} 是\textcolor{Maroon}{首要任务},那么计算满足\textcolor{Maroon}{边界条件} \bref{eq:1BC} 和\textcolor{Plum}{非线性}\textcolor{gray}{混频} \bref{eq:simplify7-LE0-SVA-V_1singular-nokxky-zeta-solution-g} 联合决定的\textcolor{PineGreen}{基系数} or \textcolor{PineGreen}{本征复振幅} $\Xint{\begin{smallmatrix} ~ \\ {}^{}_{\mathcolor{gray}{-}} \\ ~ \end{smallmatrix}}{09}{\mathtt{g}}^{\;\!\mathcolor{gray}{\omega}}_{\;\! \mathcolor{gray}{z} \textcolor{PineGreen}{\jmath}}$ 则是\textcolor{Maroon}{第二紧急任务}。

为此,利用含广义列向量形式的 \bref{eq:vec-amp_eigenmode},即
\begin{align} \label{eq:vec-amp_eigenmode-matrix}
	\Xint{\mathcolor{gray}{-}}{30}{\bar{E}}^{\;\!\mathcolor{gray}{\omega}}_{\;\! \mathcolor{gray}{z}} &:= \overline{\Xint{\begin{smallmatrix} ~ \\ {}^{}_{\mathcolor{gray}{-}} \\ ~ \end{smallmatrix}}{09}{\mathtt{g}}^{\;\!\mathcolor{gray}{\omega}}_{\;\! \mathcolor{gray}{z} \textcolor{PineGreen}{\jmath}}}^{\mathsf{\textcolor{Plum}{T}}} \cdot \overline{\Xint{\mathcolor{gray}{-}}{20}{\bar{G}}^{\;\!\mathcolor{gray}{\omega}}_{\;\! \mathcolor{gray}{z} \textcolor{PineGreen}{\jmath}}} = \overline{\Xint{\mathcolor{gray}{-}}{20}{\bar{G}}^{\;\!\mathcolor{gray}{\omega}}_{\;\! \mathcolor{gray}{z} \textcolor{PineGreen}{\jmath}}}^{\mathsf{\textcolor{Plum}{T}}} \cdot \overline{\Xint{\begin{smallmatrix} ~ \\ {}^{}_{\mathcolor{gray}{-}} \\ ~ \end{smallmatrix}}{09}{\mathtt{g}}^{\;\!\mathcolor{gray}{\omega}}_{\;\! \mathcolor{gray}{z} \textcolor{PineGreen}{\jmath}}} ~,
\end{align}
可以在形式上给出 $\overline{\Xint{\begin{smallmatrix} ~ \\ {}^{}_{\mathcolor{gray}{-}} \\ ~ \end{smallmatrix}}{09}{\mathtt{g}}^{\;\!\mathcolor{gray}{\omega}}_{\;\! \mathcolor{gray}{z} \textcolor{PineGreen}{\jmath}}}$ 的表达式
\begin{align} \label{eq:amp_vec}
	\overline{\Xint{\begin{smallmatrix} ~ \\ {}^{}_{\mathcolor{gray}{-}} \\ ~ \end{smallmatrix}}{09}{\mathtt{g}}^{\;\!\mathcolor{gray}{\omega}}_{\;\! \mathcolor{gray}{z} \textcolor{PineGreen}{\jmath}}} &:= \overline{\Xint{\mathcolor{gray}{-}}{20}{\bar{G}}^{\;\!\mathcolor{gray}{\omega}}_{\;\! \mathcolor{gray}{z} \textcolor{PineGreen}{\jmath}}}^{\textcolor{Plum}{-\mathsf{T}}} \cdot \Xint{\mathcolor{gray}{-}}{30}{\bar{E}}^{\;\!\mathcolor{gray}{\omega}}_{\;\! \mathcolor{gray}{z}} ~,
\end{align}
将其代回 \bref{eq:pre-on_the_way-transition_matrix},则将\textcolor{PineGreen}{线性叠加的\textcolor{gray}{单色}平面波基}重构/解构为了形如 \bref{eq:e^ikzz-non_diagonalization} 的三明治矩阵连乘表达式
\begin{align} \label{eq:pre-on_the_way-transition_matrix'}
	\Xint{\mathcolor{gray}{-}}{30}{\bar{E}}^{\;\!\mathcolor{gray}{\omega}}_{\;\! \mathcolor{gray}{z}} &:= \overline{\Xint{{}^{}_{\mathcolor{gray}{-}}}{10}{\bar{g}}^{\;\!\mathcolor{gray}{\omega}}_{\;\! \textcolor{PineGreen}{\jmath}}}^{\mathsf{\textcolor{Plum}{T}}} \cdot \overline{\overline{\mathbb{e}^{\mathbb{i} \Xint{\begin{smallmatrix} ~ \\ {}^{}_{\mathcolor{gray}{-}} \\ ~ \end{smallmatrix}}{15}{k}_{\symup{z} \textcolor{PineGreen}{\jmath}}^{\;\! \mathcolor{gray}{\omega}} \mathcolor{gray}{z}}}} \cdot \overline{\Xint{\mathcolor{gray}{-}}{20}{\bar{G}}^{\;\!\mathcolor{gray}{\omega}}_{\;\! \mathcolor{gray}{z} \textcolor{PineGreen}{\jmath}}}^{\textcolor{Plum}{-\mathsf{T}}} \cdot \Xint{\mathcolor{gray}{-}}{30}{\bar{E}}^{\;\!\mathcolor{gray}{\omega}}_{\;\! \mathcolor{gray}{z}} ~,
\end{align}
然而,由于循环定义,$\overline{\Xint{{}^{}_{\mathcolor{gray}{-}}}{10}{\bar{g}}^{\;\!\mathcolor{gray}{\omega}}_{\;\! \textcolor{PineGreen}{\jmath}}}^{\mathsf{\textcolor{Plum}{T}}} \cdot \overline{\overline{\mathbb{e}^{\mathbb{i} \Xint{\begin{smallmatrix} ~ \\ {}^{}_{\mathcolor{gray}{-}} \\ ~ \end{smallmatrix}}{15}{k}_{\symup{z} \textcolor{PineGreen}{\jmath}}^{\;\! \mathcolor{gray}{\omega}} \mathcolor{gray}{z}}}} \cdot \overline{\Xint{\mathcolor{gray}{-}}{16}{\bar{G}}^{\;\!\mathcolor{gray}{\omega}}_{\;\! \mathcolor{gray}{z} \textcolor{PineGreen}{\jmath}}}^{\textcolor{Plum}{-\mathsf{T}}} \xrightarrow[]{\text{\bref{eq:vec-polar_phase}}} \overline{\Xint{\mathcolor{gray}{-}}{16}{\bar{G}}^{\;\!\mathcolor{gray}{\omega}}_{\;\! \mathcolor{gray}{z} \textcolor{PineGreen}{\jmath}}}^{\mathsf{\textcolor{Plum}{T}}} \cdot \overline{\Xint{\mathcolor{gray}{-}}{16}{\bar{G}}^{\;\!\mathcolor{gray}{\omega}}_{\;\! \mathcolor{gray}{z} \textcolor{PineGreen}{\jmath}}}^{\textcolor{Plum}{-\mathsf{T}}} = \bar{\bar{\symup{I}}}$,以至于上式是无用的。但是上式的结构是具有启发意义和值得借鉴的,接下来将进一步针对\textcolor{Plum}{线性}\textcolor{PineGreen}{晶体光学},给出类似结构的有用表达式。

\vspace*{-4.5em}

\marginLeft[-2.4em]{ssec:sandwich-eigen-matrices}\subsection{线性晶体光学的核心之首:矢量电场三分量转移矩阵}\label{ssec:sandwich-eigen-matrices}

事实上,\textcolor{PineGreen}{基系数} $\Xint{\begin{smallmatrix} ~ \\ {}^{}_{\mathcolor{gray}{-}} \\ ~ \end{smallmatrix}}{09}{\mathtt{g}}^{\;\!\mathcolor{gray}{\omega}}_{\;\! \mathcolor{gray}{z} \textcolor{PineGreen}{\jmath}}$ 的初始值 $\Xint{\begin{smallmatrix} ~ \\ {}^{}_{\mathcolor{gray}{-}} \\ ~ \end{smallmatrix}}{09}{\mathtt{g}}^{\;\!\mathcolor{gray}{\omega}}_{\;\! \mathcolor{gray}{0} \textcolor{PineGreen}{\jmath}}$ 纯粹由\textcolor{Maroon}{边界条件} \bref{eq:1BC} 决定(而不是 \bref{eq:amp_vec})。这意味着,如果它所对应的\textcolor{Maroon}{傅立叶谱分量} $\Xint{\mathcolor{gray}{-}}{25}{\bar{E}}^{\;\!\mathcolor{gray}{\omega}}_{\;\! \mathcolor{gray}{z}}$ 不参与\textcolor{Plum}{非线性}\textcolor{gray}{频率转换}过程,则\textcolor{PineGreen}{本征复振幅}(列向量)的值
\begin{align} \label{eq:const_amp}
	\overline{\Xint{\mathcolor{gray}{-}}{20}{\bar{G}}^{\;\!\mathcolor{gray}{\omega}}_{\;\! \mathcolor{gray}{z} \textcolor{PineGreen}{\jmath}}}^{\textcolor{Plum}{-\mathsf{T}}} \cdot \Xint{\mathcolor{gray}{-}}{30}{\bar{E}}^{\;\!\mathcolor{gray}{\omega}}_{\;\! \mathcolor{gray}{z}} \xrightarrow[]{\text{\bref{eq:amp_vec}}} \overline{\Xint{\begin{smallmatrix} ~ \\ {}^{}_{\mathcolor{gray}{-}} \\ ~ \end{smallmatrix}}{09}{\mathtt{g}}^{\;\!\mathcolor{gray}{\omega}}_{\;\! \mathcolor{gray}{z} \textcolor{PineGreen}{\jmath}}} \equiv \overline{\Xint{\begin{smallmatrix} ~ \\ {}^{}_{\mathcolor{gray}{-}} \\ ~ \end{smallmatrix}}{09}{\mathtt{g}}^{\;\!\mathcolor{gray}{\omega}}_{\;\! \textcolor{PineGreen}{\jmath}}} = \overline{\Xint{\begin{smallmatrix} ~ \\ {}^{}_{\mathcolor{gray}{-}} \\ ~ \end{smallmatrix}}{09}{\mathtt{g}}^{\;\!\mathcolor{gray}{\omega}}_{\;\! \mathcolor{gray}{z_0} \textcolor{PineGreen}{\jmath}}} \xleftarrow[]{\text{\bref{eq:amp_vec}}} \overline{\Xint{\mathcolor{gray}{-}}{20}{\bar{G}}^{\;\!\mathcolor{gray}{\omega}}_{\;\! \mathcolor{gray}{z_0} \textcolor{PineGreen}{\jmath}}}^{\textcolor{Plum}{-\mathsf{T}}} \cdot \Xint{\mathcolor{gray}{-}}{30}{\bar{E}}^{\;\!\mathcolor{gray}{\omega}}_{\;\! \mathcolor{gray}{z_0}}
\end{align}
在仅\textcolor{Plum}{线性}\textcolor{NavyBlue}{被动}衍射/传播过程中(即使可能遭遇\textcolor{NavyBlue}{吸收}和\textcolor{NavyBlue}{光放大}也)将保持恒定\Footnote{$\mathcolor{gray}{z_0}$ 与 $\mathcolor{gray}{z}$ 一样,可以\textcolor{Maroon}{不跨越材料种类地}取任意值,包括 $\mathcolor{gray}{0}$。},类似于\textcolor{PineGreen}{本征偏振态} $\Xint{{}^{}_{\mathcolor{gray}{-}}}{10}{\bar{g}}^{\;\!\mathcolor{gray}{\omega}}_{\;\! \textcolor{PineGreen}{\jmath}}$ 的 $\mathcolor{gray}{z}$ 无关性。也就是说,在单纯的\textcolor{Plum}{线性}\textcolor{PineGreen}{晶体光学}框架内(若不涉及\textcolor{Plum}{非线性}\textcolor{NavyBlue}{光学}过程,则)\bref{eq:pre-on_the_way-transition_matrix} 可以进一步写作
\begin{subequations} \label{eq:on_the_way-transition_matrix}
	\begin{align}
		\Xint{\mathcolor{gray}{-}}{30}{\bar{E}}^{\;\!\mathcolor{gray}{\omega}}_{\;\! \mathcolor{gray}{z}} := \overline{\Xint{{}^{}_{\mathcolor{gray}{-}}}{10}{\bar{g}}^{\;\!\mathcolor{gray}{\omega}}_{\;\! \textcolor{PineGreen}{\jmath}}}^{\mathsf{\textcolor{Plum}{T}}} \cdot \overline{\overline{\mathbb{e}^{\mathbb{i} \Xint{\begin{smallmatrix} ~ \\ {}^{}_{\mathcolor{gray}{-}} \\ ~ \end{smallmatrix}}{15}{k}_{\symup{z} \textcolor{PineGreen}{\jmath}}^{\;\! \mathcolor{gray}{\omega}} \mathcolor{gray}{z}}}} \cdot \overline{\Xint{\begin{smallmatrix} ~ \\ {}^{}_{\mathcolor{gray}{-}} \\ ~ \end{smallmatrix}}{09}{\mathtt{g}}^{\;\!\mathcolor{gray}{\omega}}_{\;\! \mathcolor{gray}{z} \textcolor{PineGreen}{\jmath}}} &\xleftrightarrow[]{\text{\bref{eq:const_amp}}} \overline{\Xint{{}^{}_{\mathcolor{gray}{-}}}{10}{\bar{g}}^{\;\!\mathcolor{gray}{\omega}}_{\;\! \textcolor{PineGreen}{\jmath}}}^{\mathsf{\textcolor{Plum}{T}}} \cdot \overline{\overline{\mathbb{e}^{\mathbb{i} \Xint{\begin{smallmatrix} ~ \\ {}^{}_{\mathcolor{gray}{-}} \\ ~ \end{smallmatrix}}{15}{k}_{\symup{z} \textcolor{PineGreen}{\jmath}}^{\;\! \mathcolor{gray}{\omega}} \mathcolor{gray}{z}}}} \cdot \overline{\Xint{\begin{smallmatrix} ~ \\ {}^{}_{\mathcolor{gray}{-}} \\ ~ \end{smallmatrix}}{09}{\mathtt{g}}^{\;\!\mathcolor{gray}{\omega}}_{\;\! \mathcolor{gray}{z_0} \textcolor{PineGreen}{\jmath}}}  \label{eq:on_the_way-transition_matrix1} \\
		&\xrightarrow[]{\text{\bref{eq:amp_vec}}}
		\overline{\Xint{{}^{}_{\mathcolor{gray}{-}}}{10}{\bar{g}}^{\;\!\mathcolor{gray}{\omega}}_{\;\! \textcolor{PineGreen}{\jmath}}}^{\mathsf{\textcolor{Plum}{T}}} \cdot \overline{\overline{\mathbb{e}^{\mathbb{i} \Xint{\begin{smallmatrix} ~ \\ {}^{}_{\mathcolor{gray}{-}} \\ ~ \end{smallmatrix}}{15}{k}_{\symup{z} \textcolor{PineGreen}{\jmath}}^{\;\! \mathcolor{gray}{\omega}} \mathcolor{gray}{z}}}} \cdot \overline{\Xint{\mathcolor{gray}{-}}{20}{\bar{G}}^{\;\!\mathcolor{gray}{\omega}}_{\;\! \mathcolor{gray}{z_0} \textcolor{PineGreen}{\jmath}}}^{\textcolor{Plum}{-\mathsf{T}}} \cdot \Xint{\mathcolor{gray}{-}}{30}{\bar{E}}^{\;\!\mathcolor{gray}{\omega}}_{\;\! \mathcolor{gray}{z_0}} \label{eq:on_the_way-transition_matrix2} \\
		&\xrightarrow[]{\text{\bref{eq:vec-polar_phase}}}
		\overline{\Xint{{}^{}_{\mathcolor{gray}{-}}}{10}{\bar{g}}^{\;\!\mathcolor{gray}{\omega}}_{\;\! \textcolor{PineGreen}{\jmath}}}^{\mathsf{\textcolor{Plum}{T}}} \cdot \overline{\overline{\mathbb{e}^{\mathbb{i} \Xint{\begin{smallmatrix} ~ \\ {}^{}_{\mathcolor{gray}{-}} \\ ~ \end{smallmatrix}}{15}{k}_{\symup{z} \textcolor{PineGreen}{\jmath}}^{\;\! \mathcolor{gray}{\omega}} \mathcolor{gray}{z}}}} \cdot \overline{\overline{\mathbb{e}^{-\mathbb{i} \Xint{\begin{smallmatrix} ~ \\ {}^{}_{\mathcolor{gray}{-}} \\ ~ \end{smallmatrix}}{15}{k}_{\symup{z} \textcolor{PineGreen}{\jmath}}^{\;\! \mathcolor{gray}{\omega}} \mathcolor{gray}{z_0}}}} \cdot \overline{\Xint{{}^{}_{\mathcolor{gray}{-}}}{10}{\bar{g}}^{\;\!\mathcolor{gray}{\omega}}_{\;\! \textcolor{PineGreen}{\jmath}}}^{\textcolor{Plum}{-\mathsf{T}}} \cdot \Xint{\mathcolor{gray}{-}}{30}{\bar{E}}^{\;\!\mathcolor{gray}{\omega}}_{\;\! \mathcolor{gray}{z_0}} ~, \label{eq:on_the_way-transition_matrix3}
	\end{align}
\end{subequations}
并因此最终有以下两种对 \bref{eq:on_the_way-transition_matrix3} 的解读方式
\begin{subequations} \label{eq:E=transition_matrix-E}
	\begin{align}
		\Xint{\mathcolor{gray}{-}}{30}{\bar{E}}^{\;\!\mathcolor{gray}{\omega}}_{\;\! \mathcolor{gray}{z}} &:= \Xint{\mathcolor{gray}{-}}{32}{\bar{\bar{T}}}^{\;\!\mathcolor{gray}{\omega}}_{\;\! \mathcolor{gray}{z}} \cdot \Xint{\mathcolor{gray}{-}}{30}{\bar{E}}^{\;\!\mathcolor{gray}{\omega}}_{\;\! \mathcolor{gray}{z_0}} :=
		\overline{\Xint{\mathcolor{gray}{-}}{20}{\bar{G}}^{\;\!\mathcolor{gray}{\omega}}_{\;\! \mathcolor{gray}{z} \textcolor{PineGreen}{\jmath}}}^{\mathsf{\textcolor{Plum}{T}}} \cdot \overline{\Xint{\mathcolor{gray}{-}}{20}{\bar{G}}^{\;\!\mathcolor{gray}{\omega}}_{\;\! \mathcolor{gray}{z_0} \textcolor{PineGreen}{\jmath}}}^{\textcolor{Plum}{-\mathsf{T}}} \cdot \Xint{\mathcolor{gray}{-}}{30}{\bar{E}}^{\;\!\mathcolor{gray}{\omega}}_{\;\! \mathcolor{gray}{z_0}} \label{eq:E=transition_matrix-E1} \\
		&\xrightarrow[]{\text{\bref{eq:vec-polar_phase}}}
		\overline{\Xint{{}^{}_{\mathcolor{gray}{-}}}{10}{\bar{g}}^{\;\!\mathcolor{gray}{\omega}}_{\;\! \textcolor{PineGreen}{\jmath}}}^{\mathsf{\textcolor{Plum}{T}}} \cdot \overline{\overline{\mathbb{e}^{\mathbb{i} \Xint{\begin{smallmatrix} ~ \\ {}^{}_{\mathcolor{gray}{-}} \\ ~ \end{smallmatrix}}{15}{k}_{\symup{z} \textcolor{PineGreen}{\jmath}}^{\;\! \mathcolor{gray}{\omega}} \left( \mathcolor{gray}{z} - \mathcolor{gray}{z_0} \right)}}} \cdot \overline{\Xint{{}^{}_{\mathcolor{gray}{-}}}{10}{\bar{g}}^{\;\!\mathcolor{gray}{\omega}}_{\;\! \textcolor{PineGreen}{\jmath}}}^{\textcolor{Plum}{-\mathsf{T}}} \cdot \Xint{\mathcolor{gray}{-}}{30}{\bar{E}}^{\;\!\mathcolor{gray}{\omega}}_{\;\! \mathcolor{gray}{z_0}} ~, \label{eq:E=transition_matrix-E2}
	\end{align}
\end{subequations}
其中,定义了连接任意两个横截面 $\mathcolor{gray}{z} \longleftrightarrow \mathcolor{gray}{z_0}$ 上的矢量电场三分量 $\Xint{\mathcolor{gray}{-}}{25}{\bar{E}}^{\;\!\mathcolor{gray}{\omega}}_{\;\! \mathcolor{gray}{z}} \longleftrightarrow \Xint{\mathcolor{gray}{-}}{25}{\bar{E}}^{\;\!\mathcolor{gray}{\omega}}_{\;\! \mathcolor{gray}{z_0}}$(之间相互转换)的 $\textcolor{Plum}{3 \times 3}$ \textcolor{Maroon}{转移矩阵}
\begin{align} \label{eq:transition_matrix}
	\Xint{\mathcolor{gray}{-}}{32}{\bar{\bar{T}}}^{\;\!\mathcolor{gray}{\omega}}_{\;\! \mathcolor{gray}{z} \mathcolor{gray}{z_0}} = \overline{\Xint{\mathcolor{gray}{-}}{20}{\bar{G}}^{\;\!\mathcolor{gray}{\omega}}_{\;\! \mathcolor{gray}{z} \textcolor{PineGreen}{\jmath}}}^{\mathsf{\textcolor{Plum}{T}}} \cdot \overline{\Xint{\mathcolor{gray}{-}}{20}{\bar{G}}^{\;\!\mathcolor{gray}{\omega}}_{\;\! \mathcolor{gray}{z_0} \textcolor{PineGreen}{\jmath}}}^{\textcolor{Plum}{-\mathsf{T}}} = \overline{\Xint{{}^{}_{\mathcolor{gray}{-}}}{10}{\bar{g}}^{\;\!\mathcolor{gray}{\omega}}_{\;\! \textcolor{PineGreen}{\jmath}}}^{\mathsf{\textcolor{Plum}{T}}} \cdot \overline{\overline{\mathbb{e}^{\mathbb{i} \Xint{\begin{smallmatrix} ~ \\ {}^{}_{\mathcolor{gray}{-}} \\ ~ \end{smallmatrix}}{15}{k}_{\symup{z} \textcolor{PineGreen}{\jmath}}^{\;\! \mathcolor{gray}{\omega}} \left( \mathcolor{gray}{z} - \mathcolor{gray}{z_0} \right)}}} \cdot \overline{\Xint{{}^{}_{\mathcolor{gray}{-}}}{10}{\bar{g}}^{\;\!\mathcolor{gray}{\omega}}_{\;\! \textcolor{PineGreen}{\jmath}}}^{\textcolor{Plum}{-\mathsf{T}}} ~,
\end{align}
它作用于 $\Xint{\mathcolor{gray}{-}}{25}{\bar{E}}^{\;\!\mathcolor{gray}{\omega}}_{\;\! \mathcolor{gray}{z_0}}$ 的过程,既可以被认为
\begin{align} \label{eq:E=transition_matrix-E-comprehension1}
	\Xint{\mathcolor{gray}{-}}{25}{\bar{E}}^{\;\!\mathcolor{gray}{\omega}}_{\;\! \mathcolor{gray}{z_0}} \xrightarrow[\text{\bref{eq:amp_vec}}]{\overline{\Xint{\mathcolor{gray}{-}}{16}{\bar{G}}^{\;\!\mathcolor{gray}{\omega}}_{\;\! \mathcolor{gray}{z_0} \textcolor{PineGreen}{\jmath}}}^{\textcolor{Plum}{-\mathsf{T}}} \cdot} \overline{\Xint{\begin{smallmatrix} ~ \\ {}^{}_{\mathcolor{gray}{-}} \\ ~ \end{smallmatrix}}{09}{\mathtt{g}}^{\;\!\mathcolor{gray}{\omega}}_{\;\! \mathcolor{gray}{z_0} \textcolor{PineGreen}{\jmath}}}  \xrightarrow[\text{\bref{eq:const_amp}}]{\overline{\Xint{\begin{smallmatrix} ~ \\ {}^{}_{\mathcolor{gray}{-}} \\ ~ \end{smallmatrix}}{09}{\mathtt{g}}^{\;\!\mathcolor{gray}{\omega}}_{\;\! \mathcolor{gray}{z} \textcolor{PineGreen}{\jmath}}} \equiv \overline{\Xint{\begin{smallmatrix} ~ \\ {}^{}_{\mathcolor{gray}{-}} \\ ~ \end{smallmatrix}}{09}{\mathtt{g}}^{\;\!\mathcolor{gray}{\omega}}_{\;\! \mathcolor{gray}{z_0} \textcolor{PineGreen}{\jmath}}}} \overline{\Xint{\begin{smallmatrix} ~ \\ {}^{}_{\mathcolor{gray}{-}} \\ ~ \end{smallmatrix}}{09}{\mathtt{g}}^{\;\!\mathcolor{gray}{\omega}}_{\;\! \mathcolor{gray}{z} \textcolor{PineGreen}{\jmath}}} \xrightarrow[\text{\bref{eq:vec-amp_eigenmode-matrix}}]{\overline{\Xint{\mathcolor{gray}{-}}{16}{\bar{G}}^{\;\!\mathcolor{gray}{\omega}}_{\;\! \mathcolor{gray}{z} \textcolor{PineGreen}{\jmath}}}^{\mathsf{\textcolor{Plum}{T}}} \cdot} \Xint{\mathcolor{gray}{-}}{25}{\bar{E}}^{\;\!\mathcolor{gray}{\omega}}_{\;\! \mathcolor{gray}{z}} ~,
\end{align}
这对应 $\Xint{\mathcolor{gray}{-}}{30}{\bar{\bar{T}}}^{\;\!\mathcolor{gray}{\omega}}_{\;\! \mathcolor{gray}{z} \mathcolor{gray}{z_0}} = \overline{\Xint{\mathcolor{gray}{-}}{16}{\bar{G}}^{\;\!\mathcolor{gray}{\omega}}_{\;\! \mathcolor{gray}{z} \textcolor{PineGreen}{\jmath}}}^{\mathsf{\textcolor{Plum}{T}}} \cdot \overline{\Xint{\mathcolor{gray}{-}}{16}{\bar{G}}^{\;\!\mathcolor{gray}{\omega}}_{\;\! \mathcolor{gray}{z_0} \textcolor{PineGreen}{\jmath}}}^{\textcolor{Plum}{-\mathsf{T}}}$。也可以理解为
\begin{align} \label{eq:E=transition_matrix-E-comprehension2}
	\Xint{\mathcolor{gray}{-}}{25}{\bar{E}}^{\;\!\mathcolor{gray}{\omega}}_{\;\! \mathcolor{gray}{z_0}} \xrightarrow[\text{\bref{eq:vec-polar_G-matrix}}]{\overline{\Xint{{}^{}_{\mathcolor{gray}{-}}}{10}{\bar{g}}^{\;\!\mathcolor{gray}{\omega}}_{\;\! \textcolor{PineGreen}{\jmath}}}^{\textcolor{Plum}{-\mathsf{T}}} \cdot} \overline{\Xint{\mathcolor{gray}{-}}{20}{\mathtt{G}}^{\;\!\mathcolor{gray}{\omega}}_{\;\! \mathcolor{gray}{z_0} \textcolor{PineGreen}{\jmath}}} \xrightarrow[\text{\bref{eq:pre-on_the_way-transition_matrix}}]{\overline{\overline{\mathbb{e}^{\mathbb{i} \Xint{\begin{smallmatrix} ~ \\ {}^{}_{\mathcolor{gray}{-}} \\ ~ \end{smallmatrix}}{15}{k}_{\symup{z} \textcolor{PineGreen}{\jmath}}^{\;\! \mathcolor{gray}{\omega}} \left( \mathcolor{gray}{z} - \mathcolor{gray}{z_0} \right)}}} \cdot} \overline{\Xint{\mathcolor{gray}{-}}{20}{\mathtt{G}}^{\;\!\mathcolor{gray}{\omega}}_{\;\! \mathcolor{gray}{z} \textcolor{PineGreen}{\jmath}}} \xrightarrow[\text{\bref{eq:vec-polar_G-matrix}}]{\overline{\Xint{{}^{}_{\mathcolor{gray}{-}}}{10}{\bar{g}}^{\;\!\mathcolor{gray}{\omega}}_{\;\! \textcolor{PineGreen}{\jmath}}}^{\mathsf{\textcolor{Plum}{T}}} \cdot} \Xint{\mathcolor{gray}{-}}{25}{\bar{E}}^{\;\!\mathcolor{gray}{\omega}}_{\;\! \mathcolor{gray}{z}} ~,
\end{align}
这对应 $\Xint{\mathcolor{gray}{-}}{30}{\bar{\bar{T}}}^{\;\!\mathcolor{gray}{\omega}}_{\;\! \mathcolor{gray}{z} \mathcolor{gray}{z_0}} = \overline{\Xint{{}^{}_{\mathcolor{gray}{-}}}{10}{\bar{g}}^{\;\!\mathcolor{gray}{\omega}}_{\;\! \textcolor{PineGreen}{\jmath}}}^{\mathsf{\textcolor{Plum}{T}}} \cdot \overline{\overline{\mathbb{e}^{\mathbb{i} \Xint{\begin{smallmatrix} ~ \\ {}^{}_{\mathcolor{gray}{-}} \\ ~ \end{smallmatrix}}{15}{k}_{\symup{z} \textcolor{PineGreen}{\jmath}}^{\;\! \mathcolor{gray}{\omega}} \left( \mathcolor{gray}{z} - \mathcolor{gray}{z_0} \right)}}} \cdot \overline{\Xint{{}^{}_{\mathcolor{gray}{-}}}{10}{\bar{g}}^{\;\!\mathcolor{gray}{\omega}}_{\;\! \textcolor{PineGreen}{\jmath}}}^{\textcolor{Plum}{-\mathsf{T}}}$。

有了 \bref{eq:transition_matrix} 定义的 $\textcolor{Plum}{3 \times 3}$ \textcolor{Maroon}{转移矩阵} $\Xint{\mathcolor{gray}{-}}{30}{\bar{\bar{T}}}^{\;\!\mathcolor{gray}{\omega}}_{\;\! \mathcolor{gray}{z} \mathcolor{gray}{z_0}}$ 后,只需简单地知道(无论通过什么手段)晶体内任何特定 $\mathcolor{gray}{z_0}$ 处横截面上的\textcolor{Maroon}{傅立叶谱},即\textcolor{PineGreen}{总矢量场}(三分量) $\Xint{\mathcolor{gray}{-}}{25}{\bar{E}}^{\;\!\mathcolor{gray}{\omega}}_{\;\! \mathcolor{gray}{z_0}}$,就可通过左乘矩阵场 $\Xint{\mathcolor{gray}{-}}{30}{\bar{\bar{T}}}^{\;\!\mathcolor{gray}{\omega}}_{\;\! \mathcolor{gray}{z} \mathcolor{gray}{z_0}}$ 来获得/预言另一个其他 $\mathcolor{gray}{z}$ 处指定截面的\textcolor{Maroon}{傅立叶谱}\textcolor{PineGreen}{总矢量场}(三分量) $\Xint{\mathcolor{gray}{-}}{25}{\bar{E}}^{\;\!\mathcolor{gray}{\omega}}_{\;\! \mathcolor{gray}{z}}$,以至于整个\textcolor{Plum}{线性}\textcolor{Plum}{均匀}\Footnote{方程左侧的“场项”对应的\textcolor{Plum}{线性}材料系数需\textcolor{Plum}{均匀},右侧的“源项”对应的\textcolor{Plum}{非线性}(\textcolor{NavyBlue}{波源}的)材料系数,以及\textcolor{Plum}{线性}散射源项的材料系数可以\textcolor{Plum}{不均匀},见\bref{eq:nonlinear(2)-wave_wkrho-simplify4-2}。}\textcolor{Plum}{各向异性}同种材料内 3D(三维)分布的\textcolor{gray}{单色}光场的电场三分量,都可以通过 \bref{eq:E=transition_matrix-E-comprehension1} 或 \bref{eq:E=transition_matrix-E-comprehension2} 获得。注意,这不涉及\textcolor{Maroon}{边界条件} \bref{eq:1BC},因为根本没用到它。

\textcolor{gray}{正空间}中即 $\left( \mathcolor{gray}{\omega}, \mathcolor{gray}{\bar{\rho}}, \mathcolor{gray}{z} \right) \asymp \left( \mathcolor{gray}{\omega}, \mathcolor{gray}{\bar{r}} \right)$ 域上,\textcolor{gray}{单色}光场的电场矢量 $\bar{E}^{\;\!\mathcolor{gray}{\omega}}_{\;\!\mathcolor{gray}{z}}$ 在\textcolor{Plum}{各向异性}材料中的\textcolor{Plum}{线性}\textcolor{NavyBlue}{传播}、\textcolor{Plum}{线性}\textcolor{NavyBlue}{衍射}、\textcolor{Plum}{线性}\textcolor{NavyBlue}{吸收}/\textcolor{NavyBlue}{增益}、偏振态演化、\textcolor{PineGreen}{能流}$\Xint{\begin{smallmatrix} ~ \\ {}^{}_{\mathcolor{gray}{-}} \\ ~ \end{smallmatrix}}{20}{\bar{s}}^{\;\! \mathcolor{gray}{\omega} \textcolor{PineGreen}{\jmath}}$-\textcolor{PineGreen}{波矢}$\Xint{\begin{smallmatrix} ~ \\ {}^{}_{\mathcolor{gray}{-}} \\ ~ \end{smallmatrix}}{15}{\bar{k}}^{\;\! \mathcolor{gray}{\omega} \textcolor{PineGreen}{\jmath}}$走离、\textcolor{PineGreen}{群速度}$\Xint{\begin{smallmatrix} ~ \\ {}^{}_{\mathcolor{gray}{-}} \\ ~ \end{smallmatrix}}{20}{\bar{v}}^{\;\! \mathcolor{gray}{\omega} \textcolor{PineGreen}{\jmath}}_{\textcolor{Maroon}{\text{g}}}$/\textcolor{PineGreen}{波矢}$\Xint{\begin{smallmatrix} ~ \\ {}^{}_{\mathcolor{gray}{-}} \\ ~ \end{smallmatrix}}{15}{\bar{k}}^{\;\! \mathcolor{gray}{\omega} \textcolor{PineGreen}{\jmath}}$的\textcolor{Plum}{线性}\textcolor{PineGreen}{双折射}、\textcolor{PineGreen}{锥折射}、\textcolor{PineGreen}{双锥折射}、横/纵\textcolor{NavyBlue}{自旋轨道耦合}等动力学过程,完全由\textcolor{gray}{倒空间}中即 $\left( \mathcolor{gray}{\omega}, \mathcolor{gray}{\bar{k}_{\symup{\rho}}}, \mathcolor{gray}{z} \right)$ 域的 $\textcolor{Plum}{3 \times 3}$ \textcolor{Maroon}{转移矩阵} \bref{eq:transition_matrix} 所给出的 $\Xint{\mathcolor{gray}{-}}{30}{\bar{\bar{T}}}^{\;\!\mathcolor{gray}{\omega}}_{\;\! \mathcolor{gray}{z} \mathcolor{gray}{z_0}}$,配合用于\textcolor{gray}{正} $\longleftrightarrow$ \textcolor{gray}{倒空间}转换的 2D(二维)\textcolor{Plum}{横向} $\mathcolor{gray}{\bar{k}_{\symup{\rho}}}, \mathcolor{gray}{\bar{\rho}} \in \mathcolor{gray}{\bar{\mathbb{R}}_{\textcolor{Plum}{2}}}$ 域上的 \textcolor{Plum}{傅立叶正} $\mathcolor{gray}{\mathcal F}$、\textcolor{Maroon}{逆} $\mathcolor{gray}{\mathcal F^{-1}}$ \textcolor{Plum}{变换对} \bref{eq:FT-rho_krho} 一起,来统一描述、预测、控制和确定。

如果继续级联 $\mathcolor{gray}{t}, \mathcolor{gray}{\omega} \in \mathcolor{gray}{\mathbb{R}}$ 域的 1D(一维)\textcolor{Plum}{傅立叶变换} \bref{eq:IFT-trho},则可以获得\textcolor{gray}{正空间}中 $\mathcolor{gray}{\bar{x}} = \left( \mathcolor{gray}{t}, \mathcolor{gray}{\bar{r}} \right) \asymp \left( \mathcolor{gray}{t}, \mathcolor{gray}{\bar{\rho}}, \mathcolor{gray}{z} \right)$ 域上的光电场的 $1+3$ 维时空分布。--- 这是所有光学模型、模拟软件、仿真系统最终\textcolor{Plum}{输出}的计算结果,也是实验室 CCD(电荷耦合元件)相机、生物感光器官(如肉眼)所能直接捕获和探测到的实验数据。

如果不级联 2D/3D \textcolor{Plum}{傅立叶变换},仅仔细观察 \bref{eq:transition_matrix} 本身,可以发现\textcolor{Maroon}{转移矩阵} $\Xint{\mathcolor{gray}{-}}{30}{\bar{\bar{T}}}^{\;\!\mathcolor{gray}{\omega}}_{\;\! \mathcolor{gray}{z} \mathcolor{gray}{z_0}}$ 由\textcolor{gray}{倒空间}的 3 个\textcolor{PineGreen}{本征系统}矩阵场构成:2 个互逆的\textcolor{PineGreen}{本征偏振态矩阵}(场)$\overline{\Xint{{}^{}_{\mathcolor{gray}{-}}}{10}{\bar{g}}^{\;\!\mathcolor{gray}{\omega}}_{\;\! \textcolor{PineGreen}{\jmath}}}^{\mathsf{\textcolor{Plum}{T}}}, \overline{\Xint{{}^{}_{\mathcolor{gray}{-}}}{10}{\bar{g}}^{\;\!\mathcolor{gray}{\omega}}_{\;\! \textcolor{PineGreen}{\jmath}}}^{\textcolor{Plum}{-\mathsf{T}}}$,1 个\textcolor{PineGreen}{传播矩阵}(场)$\overline{\overline{\mathbb{e}^{\mathbb{i} \Xint{\begin{smallmatrix} ~ \\ {}^{}_{\mathcolor{gray}{-}} \\ ~ \end{smallmatrix}}{15}{k}_{\symup{z} \textcolor{PineGreen}{\jmath}}^{\;\! \mathcolor{gray}{\omega}} \left( \mathcolor{gray}{z} - \mathcolor{gray}{z_0} \right)}}}$。将三者进一步分解至(不可再分的)原子水平,最终便发现再次回归\textcolor{PineGreen}{本征值}-\textcolor{PineGreen}{向量}对 $\Xint{\begin{smallmatrix} ~ \\ {}^{}_{\mathcolor{gray}{-}} \\ ~ \end{smallmatrix}}{15}{k}_{\symup{z} \textcolor{PineGreen}{\jmath}}^{\;\! \mathcolor{gray}{\omega}}, \Xint{{}^{}_{\mathcolor{gray}{-}}}{10}{\bar{g}}^{\;\!\mathcolor{gray}{\omega}}_{\;\! \textcolor{PineGreen}{\jmath}}$,首尾呼应了 \bref{ssec:eigenmodes-compamp} 开篇的论点:即\textcolor{Plum}{线性}\textcolor{PineGreen}{晶体光学}始于\textcolor{PineGreen}{特征方程} \bref{eq:nonlinear(2)-wave_wkrho-simplify6-L3''}。

\textcolor{Plum}{线性}\textcolor{Plum}{均匀}材料中的\textcolor{Plum}{线性}\textcolor{PineGreen}{晶体光学},由 \bref{eq:E=transition_matrix-E} 或 \bref{eq:E=transition_matrix-E-comprehension1} 或 \bref{eq:E=transition_matrix-E-comprehension2} 完全描述(包含全部信息)。可见(通过 \bref{eq:transition_matrix} 定义的)\textcolor{Maroon}{转移矩阵} $\Xint{\mathcolor{gray}{-}}{30}{\bar{\bar{T}}}^{\;\!\mathcolor{gray}{\omega}}_{\;\! \mathcolor{gray}{z} \mathcolor{gray}{z_0}}$ 只是其\textcolor{Maroon}{核心之一},它一般只能通过理论分析获得。剩下的一个未知\textcolor{NavyBlue}{物理量} $\Xint{\mathcolor{gray}{-}}{25}{\bar{E}}^{\;\!\mathcolor{gray}{\omega}}_{\;\! \mathcolor{gray}{z_0}}$ 是其\textcolor{Maroon}{核心之二},它既可以由实验数据测定,也可以由\textcolor{Maroon}{边界条件} \bref{eq:1BC} 确定,或者/甚至(被)直接设定\Footnote{对于某些\textcolor{Plum}{正向}的科学探索,如研究问题导向(确实需要)、\textcolor{NavyBlue}{泵浦}空域预调制、光束整形等;或者某些\textcolor{Plum}{逆问题},如光存储、全息图设计、激光加工焦斑形貌优化、像差矫正等。}。当 $\mathcolor{gray}{z_0} = \mathcolor{gray}{0}$ 时,$\Xint{\mathcolor{gray}{-}}{25}{\bar{E}}^{\;\!\mathcolor{gray}{\omega}}_{\;\! \mathcolor{gray}{z_0}} = \Xint{\mathcolor{gray}{-}}{25}{\bar{E}}^{\;\!\mathcolor{gray}{\omega}}_{\;\! \mathcolor{gray}{0}}$ 来源于从原本的 \bref{eq:pre-on_the_way-transition_matrix} 中的 \textcolor{PineGreen}{本征复振幅} $\overline{\Xint{\begin{smallmatrix} ~ \\ {}^{}_{\mathcolor{gray}{-}} \\ ~ \end{smallmatrix}}{09}{\mathtt{g}}^{\;\!\mathcolor{gray}{\omega}}_{\;\! \mathcolor{gray}{z} \textcolor{PineGreen}{\jmath}}}$ 在材料边界处的初始值 $\overline{\Xint{\begin{smallmatrix} ~ \\ {}^{}_{\mathcolor{gray}{-}} \\ ~ \end{smallmatrix}}{09}{\mathtt{g}}^{\;\!\mathcolor{gray}{\omega}}_{\;\! \mathcolor{gray}{0} \textcolor{PineGreen}{\jmath}}}$ 通过 \bref{eq:amp_vec} 分离而来,与\textcolor{Plum}{非线性}\textcolor{PineGreen}{晶体光学}的解 \bref{eq:simplify6-LE0-SVA-V_1nonsingular-solution-g} 中的通解/常量部分:初始电场\textcolor{Maroon}{时空谱} $\Xint{{}^{}_{\mathcolor{gray}{-}}}{10}{\bar{g}}^{\;\!\mathcolor{gray}{\omega} \textcolor{PineGreen}{\imath}}_{\;\! \mathcolor{gray}{0}}$ 异曲同工。

\vspace*{-4.5em}

\marginLeft[-2.4em]{ssec:3times2sandwich-eigen-matrices}\subsection{线性均匀纯电各向异性材料中的光场:正/倒空间视角}\label{ssec:3times2sandwich-eigen-matrices}

在上一章最后 \bref{ssec:E-waveq-nonlinear} 的末尾,提到了\textcolor{NavyBlue}{缓变振幅近似}下的\textcolor{PineGreen}{平面波基}\textcolor{Plum}{非线性}矢量电场波动方程 \bref{eq:nonlinear(2)-wave_wkrho-simplify6-V2'-SVA},在\textcolor{Plum}{线性}\textcolor{Plum}{均匀}\textcolor{PineGreen}{纯电(非磁)各向异性}材料所对应的\textcolor{Plum}{非线性}算子 $\Xint{\mathcolor{gray}{-}}{25}{\bar{\bar{\mathsfit{V}}}}^{\;\! \mathcolor{gray}{\omega} \textcolor{PineGreen}{\imath}}_{\textcolor{Maroon}{\mathbb{1}}}$ 的作用下,存在唯一确定的解析解。然而,对\textcolor{Plum}{非线性}算子 $\Xint{\mathcolor{gray}{-}}{25}{\bar{\bar{\mathsfit{V}}}}^{\;\! \mathcolor{gray}{\omega} \textcolor{PineGreen}{\imath}}_{\textcolor{Maroon}{\mathbb{1}}}$ 的上述近似和简化条件,不一定必须同步应用于\textcolor{Plum}{线性}算子 $\Xint{\mathcolor{gray}{-}}{30}{\bar{\bar{L}}}^{\;\! \mathcolor{gray}{\omega} \textcolor{PineGreen}{\imath}}$。

也就是说,即使在电磁\textcolor{PineGreen}{双各向异性}材料中,也可以对\textcolor{Plum}{非线性}算子 $\Xint{\mathcolor{gray}{-}}{25}{\bar{\bar{\mathsfit{V}}}}^{\;\! \mathcolor{gray}{\omega} \textcolor{PineGreen}{\imath}}_{\textcolor{Maroon}{\mathbb{1}}}$ \bref{eq:plane_wave_basis-V1} 实施上述近似和简化,并且同时保留不做任何近似或简化的\textcolor{Plum}{线性}算子 $\Xint{\mathcolor{gray}{-}}{30}{\bar{\bar{L}}}^{\;\! \mathcolor{gray}{\omega} \textcolor{PineGreen}{\imath}}$ \bref{eq:nonlinear(2)-wave_wkrho-simplify6-L3''},以及被其确定的所有\textcolor{PineGreen}{本征模}式 \bref{eq:vec-polar_phase}。这当然是不严格自洽的,但仍然是合理和适定的:保持底层基础(\textcolor{Plum}{线性}\textcolor{PineGreen}{晶体光学})不变,只略微精简上层建筑(\textcolor{Plum}{非线性}\textcolor{PineGreen}{晶体光学}),以在尽可能少地牺牲模型的普适性的前提下,使\textcolor{Plum}{非线性}矢量电场波动方程有解且解唯一。

然而,为了不引发争议,本文将\textcolor{Plum}{线性}、\textcolor{Plum}{非线性}算子 $\Xint{\mathcolor{gray}{-}}{30}{\bar{\bar{L}}}^{\;\! \mathcolor{gray}{\omega} \textcolor{PineGreen}{\imath}}, \Xint{\mathcolor{gray}{-}}{25}{\bar{\bar{\mathsfit{V}}}}^{\;\! \mathcolor{gray}{\omega} \textcolor{PineGreen}{\imath}}_{\textcolor{Maroon}{\mathbb{1}}}$ 统一设在\textcolor{Plum}{线性}\textcolor{Plum}{均匀}\textcolor{PineGreen}{纯电各向异性}材料背景下,以允许直接使用适定的\textcolor{Plum}{非线性}解 \bref{eq:simplify7-LE0-SVA-V_1singular-nokxky-zeta} 的同时,使\textcolor{Plum}{线性}算子 $\Xint{\mathcolor{gray}{-}}{30}{\bar{\bar{L}}}^{\;\! \mathcolor{gray}{\omega} \textcolor{PineGreen}{\imath}}$ 经历包含其全部简化条件并因此无矛盾的化简流程\Footnote{注意,不像\textcolor{Plum}{非线性}算子 $\Xint{\mathcolor{gray}{-}}{25}{\bar{\bar{\mathsfit{V}}}}^{\;\! \mathcolor{gray}{\omega} \textcolor{PineGreen}{\imath}}_{\textcolor{Maroon}{\mathbb{1}}}$,\textcolor{Plum}{线性}算子 $\Xint{\mathcolor{gray}{-}}{30}{\bar{\bar{L}}}^{\;\! \mathcolor{gray}{\omega} \textcolor{PineGreen}{\imath}}$ 没有用到近似 \bref{eq:k_rho<<k_z}。},包括 $\bar{\bar{\zeta}}^{\;\! \mathcolor{gray}{\omega} \mathcolor{gray}{\check{1}}}_{\textcolor{Maroon}{(1)}} \equiv \bar{\bar{0}}$ 和 $\bar{\bar{\zeta}}^{\;\! \mathcolor{gray}{\omega} \mathcolor{gray}{\check{1} \check{2}}}_{\textcolor{Maroon}{(1)}} \equiv \frac{1}{\mathbb{i} k_{\textcolor{Maroon}{\mathsf{o}}}^{\;\! \mathcolor{gray}{\omega}}} \Xint{\begin{smallmatrix} ~ \\ {}^{}_{\mathcolor{gray}{-}} \\ ~ \end{smallmatrix}}{13}{\varsigma}^{\;\! \mathcolor{gray}{\omega} \hat{1} \mathcolor{gray}{\check{1}}}_{\;\! \dot{2} \mathcolor{gray}{z} \textcolor{Maroon}{(1)}} \epsilon^{\hphantom{\symup{\iota}}\mathcolor{gray}{\check{2}}\dot{2}}_{\symup{\iota}} = \frac{1}{\mathbb{i} k_{\textcolor{Maroon}{\mathsf{o}}}^{\;\! \mathcolor{gray}{\omega}}} \left( \frac{\symup{c}}{\mathbb{i} \mathcolor{gray}{\omega}} \epsilon^{\;\! \hphantom{\dot{2}} \mathcolor{gray}{\check{1}} \hat{1}}_{\;\! \dot{2}} \right) \epsilon^{\hphantom{\symup{\iota}}\mathcolor{gray}{\check{2}}\dot{2}}_{\symup{\iota}} = \frac{1}{\mathbb{i} k_{\textcolor{Maroon}{\mathsf{o}}}^{\;\! \mathcolor{gray}{\omega}}}  \frac{\symup{c}}{\mathbb{i} \mathcolor{gray}{\omega}} \left( \delta^{\;\! \hphantom{\symup{\iota}} \mathcolor{gray}{\check{1}}}_{\;\! \symup{\iota} \hphantom{z}} \delta^{\;\! \mathcolor{gray}{\check{2}} \hat{1}} - \delta^{\;\! \hphantom{\symup{\iota}} \hat{1}}_{\;\! \symup{\iota} \hphantom{z}} \delta^{\;\! \mathcolor{gray}{\check{1}} \mathcolor{gray}{\check{2}}} \right) = - \frac{1}{k_{\textcolor{Maroon}{\mathsf{o}} \mathcolor{gray}{\omega}}^{\;\! 2}} \left( \hat{\symup{e}}^{\mathcolor{gray}{\check{1}}} \otimes \hat{\symup{e}}^{\mathcolor{gray}{\check{2}}} - \bar{\bar{\symup{I}}} \delta^{\;\! \mathcolor{gray}{\check{1}} \mathcolor{gray}{\check{2}}} \right) =: \frac{1}{k_{\textcolor{Maroon}{\mathsf{o}} \mathcolor{gray}{\omega}}^{\;\! 2}} \left( \bar{\bar{\symup{I}}} \delta^{\;\! \mathcolor{gray}{\check{1}} \mathcolor{gray}{\check{2}}} - \hat{\mathcolor{gray}{\check{1}}} \hat{\mathcolor{gray}{\check{2}}} \right)$,即
\begin{align} \label{eq:zeta-restriction-L}
	\bar{\bar{\zeta}}^{\;\! \mathcolor{gray}{\omega} \mathcolor{gray}{\check{1}}}_{\textcolor{Maroon}{(1)}} \equiv \bar{\bar{0}} ~, ~~~~~~ \bar{\bar{\zeta}}^{\;\! \mathcolor{gray}{\omega} \mathcolor{gray}{\check{1} \check{2}}}_{\textcolor{Maroon}{(1)}} \equiv \frac{1}{k_{\textcolor{Maroon}{\mathsf{o}} \mathcolor{gray}{\omega}}^{\;\! 2}} \left( \delta^{\;\! \mathcolor{gray}{\check{1}} \mathcolor{gray}{\check{2}}} - \hat{\mathcolor{gray}{\check{1}}} \hat{\mathcolor{gray}{\check{2}}}^{\mathsf{\textcolor{Plum}{T}}} \right) ~,
\end{align}
最终将\textcolor{Plum}{线性}算子 $\Xint{\mathcolor{gray}{-}}{30}{\bar{\bar{L}}}^{\;\! \mathcolor{gray}{\omega} \textcolor{PineGreen}{\imath}}$ 从 \bref{eq:nonlinear(2)-wave_wkrho-simplify6-L3''} 精减为:\textcolor{Plum}{线性}\textcolor{Plum}{均匀}\textcolor{PineGreen}{纯电各向异性}材料中的
\begin{subequations} \label{eq:simplify7-L-zeta}
	\begin{align}
		\!\!\!\! \Xint{\mathcolor{gray}{-}}{32}{\bar{\bar{L}}}^{\;\! \mathcolor{gray}{\omega} \textcolor{PineGreen}{\imath}} \Xint{{}^{}_{\mathcolor{gray}{-}}}{10}{\bar{g}}^{\;\!\mathcolor{gray}{\omega} \textcolor{PineGreen}{\imath}} &= \left[ \left( \delta^{\;\! \mathcolor{gray}{\check{1}} \mathcolor{gray}{\check{2}}} - \hat{\mathcolor{gray}{\check{1}}} \hat{\mathcolor{gray}{\check{2}}}^{\mathsf{\textcolor{Plum}{T}}} \right) \Xint{\begin{smallmatrix} ~ \\ {}^{}_{\mathcolor{gray}{-}} \\ ~ \end{smallmatrix}}{15}{k}_{\;\! \mathcolor{gray}{\check{2}}}^{\;\! \mathcolor{gray}{\omega} \textcolor{PineGreen}{\imath}} \Xint{\begin{smallmatrix} ~ \\ {}^{}_{\mathcolor{gray}{-}} \\ ~ \end{smallmatrix}}{15}{k}_{\;\! \mathcolor{gray}{\check{1}}}^{\;\! \mathcolor{gray}{\omega} \textcolor{PineGreen}{\imath}} - k_{\textcolor{Maroon}{\mathsf{o}} \mathcolor{gray}{\omega}}^{\;\! 2} \bar{\bar{\varepsilon}}^{\;\! \mathcolor{gray}{\omega}}_{\textcolor{Maroon}{(1)}} \right] \Xint{{}^{}_{\mathcolor{gray}{-}}}{10}{\bar{g}}^{\;\!\mathcolor{gray}{\omega} \textcolor{PineGreen}{\imath}} \label{eq:simplify7-L1-zeta} \\ 
		&= \left[ \Xint{\begin{smallmatrix} ~ \\ {}^{}_{\mathcolor{gray}{-}} \\ ~ \end{smallmatrix}}{15}{\bar{k}}_{\;\! \mathcolor{gray}{\omega}}^{\;\! \textcolor{PineGreen}{\imath} {\mathsf{\textcolor{Plum}{T}}}} \Xint{\begin{smallmatrix} ~ \\ {}^{}_{\mathcolor{gray}{-}} \\ ~ \end{smallmatrix}}{15}{\bar{k}}_{\;\! \mathcolor{gray}{\omega}}^{\;\! \textcolor{PineGreen}{\imath}} - \Xint{\begin{smallmatrix} ~ \\ {}^{}_{\mathcolor{gray}{-}} \\ ~ \end{smallmatrix}}{15}{\bar{k}}_{\;\! \mathcolor{gray}{\omega}}^{\;\! \textcolor{PineGreen}{\imath}} \Xint{\begin{smallmatrix} ~ \\ {}^{}_{\mathcolor{gray}{-}} \\ ~ \end{smallmatrix}}{15}{\bar{k}}_{\;\! \mathcolor{gray}{\omega}}^{\;\! \textcolor{PineGreen}{\imath} {\mathsf{\textcolor{Plum}{T}}}} - k_{\textcolor{Maroon}{\mathsf{o}} \mathcolor{gray}{\omega}}^{\;\! 2} \bar{\bar{\varepsilon}}^{\;\! \mathcolor{gray}{\omega}}_{\textcolor{Maroon}{(1)}} \right] \Xint{{}^{}_{\mathcolor{gray}{-}}}{10}{\bar{g}}^{\;\!\mathcolor{gray}{\omega} \textcolor{PineGreen}{\imath}} = \bar{0} ~, \label{eq:simplify7-L2-zeta}
	\end{align}
\end{subequations}
此时,\textcolor{Plum}{线性}、\textcolor{Plum}{非线性}\textcolor{PineGreen}{晶体光学} \bref{eq:simplify7-L-zeta,eq:simplify7-LE0-SVA-V_1singular-nokxky-zeta} 的适用条件,统一从\textcolor{Plum}{线性}\textcolor{Plum}{均匀}电磁\textcolor{PineGreen}{双各向异性}材料降级为:\textcolor{Plum}{线性}\textcolor{Plum}{均匀}\textcolor{PineGreen}{纯电/非磁各向异性}材料。

由于\textcolor{Plum}{线性}算子 $\Xint{\mathcolor{gray}{-}}{30}{\bar{\bar{L}}}^{\;\! \mathcolor{gray}{\omega} \textcolor{PineGreen}{\imath}}$ 中,关于\textcolor{Plum}{非均匀}复\textcolor{PineGreen}{波矢} $\Xint{\begin{smallmatrix} ~ \\ {}^{}_{\mathcolor{gray}{-}} \\ ~ \end{smallmatrix}}{15}{\bar{k}}^{\;\! \textcolor{PineGreen}{\imath}}_{\mathcolor{gray}{\omega}}$ 的二次项 $\bar{\bar{\zeta}}^{\;\! \mathcolor{gray}{\omega} \mathcolor{gray}{\check{1} \check{2}}}_{\textcolor{Maroon}{(1)}} \Xint{\begin{smallmatrix} ~ \\ {}^{}_{\mathcolor{gray}{-}} \\ ~ \end{smallmatrix}}{15}{k}_{\;\! \mathcolor{gray}{\check{2}}}^{\;\! \mathcolor{gray}{\omega} \textcolor{PineGreen}{\imath}} \Xint{\begin{smallmatrix} ~ \\ {}^{}_{\mathcolor{gray}{-}} \\ ~ \end{smallmatrix}}{15}{k}_{\;\! \mathcolor{gray}{\check{1}}}^{\;\! \mathcolor{gray}{\omega} \textcolor{PineGreen}{\imath}}$ 简化为了 $\Xint{\begin{smallmatrix} ~ \\ {}^{}_{\mathcolor{gray}{-}} \\ ~ \end{smallmatrix}}{15}{\bar{k}}_{\;\! \mathcolor{gray}{\omega}}^{\;\! \textcolor{PineGreen}{\imath} {\mathsf{\textcolor{Plum}{T}}}} \Xint{\begin{smallmatrix} ~ \\ {}^{}_{\mathcolor{gray}{-}} \\ ~ \end{smallmatrix}}{15}{\bar{k}}_{\;\! \mathcolor{gray}{\omega}}^{\;\! \textcolor{PineGreen}{\imath}} - \Xint{\begin{smallmatrix} ~ \\ {}^{}_{\mathcolor{gray}{-}} \\ ~ \end{smallmatrix}}{15}{\bar{k}}_{\;\! \mathcolor{gray}{\omega}}^{\;\! \textcolor{PineGreen}{\imath}} \Xint{\begin{smallmatrix} ~ \\ {}^{}_{\mathcolor{gray}{-}} \\ ~ \end{smallmatrix}}{15}{\bar{k}}_{\;\! \mathcolor{gray}{\omega}}^{\;\! \textcolor{PineGreen}{\imath} {\mathsf{\textcolor{Plum}{T}}}}$,因此\textcolor{Plum}{线性}算子的行列式 $\det \left[ \Xint{\mathcolor{gray}{-}}{30}{\bar{\bar{L}}}^{\;\! \mathcolor{gray}{\omega} \textcolor{PineGreen}{\imath}} \right]$ 便从关于\textcolor{PineGreen}{本征值} $\Xint{\begin{smallmatrix} ~ \\ {}^{}_{\mathcolor{gray}{-}} \\ ~ \end{smallmatrix}}{15}{k}_{\symup{z}}^{\;\! \mathcolor{gray}{\omega} \textcolor{PineGreen}{\imath}}$ 的六次方程,降为四次\textcolor{PineGreen}{特征方程}\cite{hehlAxionDilatonMetric2016}。对应的\textcolor{PineGreen}{本征向量} $\Xint{{}^{}_{\mathcolor{gray}{-}}}{10}{\bar{g}}^{\;\!\mathcolor{gray}{\omega} \textcolor{PineGreen}{\imath}}$ 数量,也从 6 个减少为了 4 个。这意味着,从电磁\textcolor{PineGreen}{双各向异性}降至\textcolor{PineGreen}{纯电非磁各向异性}后,原来可能存在的一对相向传播的纵\textcolor{PineGreen}{本征波},会被排除在\textcolor{Plum}{解空间}之外,以至于\textcolor{PineGreen}{纯电/非磁各向异性}材料内只剩下 2 对\textcolor{Plum}{正}/\textcolor{Plum}{反向传播}的横\textcolor{PineGreen}{本征波}。

这使得同向传播的\textcolor{PineGreen}{本征模} \bref{eq:polar_phase} 的数量,即 $\textcolor{PineGreen}{\imath}, \Xint{\mathcolor{gray}{-}}{16}{\bar{G}}^{\;\!\mathcolor{gray}{\omega} \textcolor{PineGreen}{\imath}}_{\;\! \mathcolor{gray}{z}}$ 的取值个数,也即参与\textcolor{PineGreen}{线性叠加}出\textcolor{PineGreen}{总矢量电场} $\Xint{\mathcolor{gray}{-}}{25}{\bar{E}}^{\;\!\mathcolor{gray}{\omega}}_{\;\! \mathcolor{gray}{z}}$ 的\bref{eq:amp_eigenmode} 中的\textcolor{PineGreen}{模式}数量,亦即 \bref{eq:vec-amp_eigenmode-matrix} 中的广义列、行向量个数,全都从 3 个减为 2 个,且具体取值为 $\textcolor{PineGreen}{\imath}, \Xint{\mathcolor{gray}{-}}{16}{\bar{G}}^{\;\!\mathcolor{gray}{\omega} \textcolor{PineGreen}{\imath}}_{\;\! \mathcolor{gray}{z}} = \textcolor{PineGreen}{\pm}, \Xint{\mathcolor{gray}{-}}{16}{\bar{G}}^{\;\!\mathcolor{gray}{\omega} \textcolor{PineGreen}{\pm}}_{\;\! \mathcolor{gray}{z}}$\Footnote{注,这里的 $\textcolor{PineGreen}{+},\textcolor{PineGreen}{-}$ 不代表\textcolor{Plum}{正}/\textcolor{Plum}{反向传播},而分别代表更大/小的\textcolor{PineGreen}{相速度} $\Xint{\begin{smallmatrix} ~ \\ {}^{}_{\mathcolor{gray}{-}} \\ ~ \end{smallmatrix}}{20}{\bar{v}}^{\;\! \mathcolor{gray}{\omega} \textcolor{PineGreen}{\jmath}}$\cite{berryOpticalSingularitiesBirefringent2003}。两种\textcolor{PineGreen}{模式}都是同向传播的。},以至于 $\overline{\Xint{\mathcolor{gray}{-}}{16}{\bar{G}}^{\;\!\mathcolor{gray}{\omega}}_{\;\! \mathcolor{gray}{z} \textcolor{PineGreen}{\jmath}}}^{\mathsf{\textcolor{Plum}{T}}} = \overline{\Xint{\mathcolor{gray}{-}}{16}{\bar{G}}^{\;\!\mathcolor{gray}{\omega}}_{\;\! \mathcolor{gray}{z} \textcolor{PineGreen}{\pm}}}^{\mathsf{\textcolor{Plum}{T}}}$ 不再是 $\textcolor{Plum}{3 \times 3}$ 方阵,而是 $\textcolor{Plum}{3 \times 2}$ 长方形矩阵。--- 这使得 $\overline{\Xint{\mathcolor{gray}{-}}{16}{\bar{G}}^{\;\!\mathcolor{gray}{\omega}}_{\;\! \mathcolor{gray}{z} \textcolor{PineGreen}{\pm}}}^{\mathsf{\textcolor{Plum}{T}}}$ 的逆 $\overline{\Xint{\mathcolor{gray}{-}}{16}{\bar{G}}^{\;\!\mathcolor{gray}{\omega}}_{\;\! \mathcolor{gray}{z} \textcolor{PineGreen}{\pm}}}^{\textcolor{Plum}{-\mathsf{T}}}$ 不存在,以至于 \bref{eq:amp_vec} 不成立,那么其后续的三明治表达式 \bref{eq:pre-on_the_way-transition_matrix'},以及整个 \bref{ssec:sandwich-eigen-matrices} 的结论都不成立(若不加修缮)。

为此,我们\cite{xieAnalytic3DVector}提出使用 \bref{eq:amp_vec} 的替代物,即
\begin{align} \label{eq:amp_vec-transverse_input}
	\overline{\Xint{\begin{smallmatrix} ~ \\ {}^{}_{\mathcolor{gray}{-}} \\ ~ \end{smallmatrix}}{09}{\mathtt{g}}^{\;\!\mathcolor{gray}{\omega} \textcolor{PineGreen}{\pm}}_{\;\! \mathcolor{gray}{z}}} &:= \overline{\Xint{\mathcolor{gray}{-}}{20}{\bar{G}}^{\;\!\mathcolor{gray}{\omega} \textcolor{PineGreen}{\pm}}_{\;\! \textcolor{Maroon}{\symup{\rho}} \mathcolor{gray}{z}}}^{\textcolor{Plum}{-\mathsf{T}}} \cdot \Xint{\mathcolor{gray}{-}}{30}{\bar{E}}^{\;\!\mathcolor{gray}{\omega}}_{\;\! \textcolor{Maroon}{\symup{\rho}} \mathcolor{gray}{z}} ~,
\end{align}
来得到 2 个\textcolor{PineGreen}{本征复振幅}系数 $\Xint{\begin{smallmatrix} ~ \\ {}^{}_{\mathcolor{gray}{-}} \\ ~ \end{smallmatrix}}{09}{\mathtt{g}}^{\;\!\mathcolor{gray}{\omega} \textcolor{PineGreen}{\pm}}_{\;\! \mathcolor{gray}{z}}$,所构成的 $\textcolor{Plum}{2 \times 1}$ 列向量 $\overline{\Xint{\begin{smallmatrix} ~ \\ {}^{}_{\mathcolor{gray}{-}} \\ ~ \end{smallmatrix}}{09}{\mathtt{g}}^{\;\!\mathcolor{gray}{\omega} \textcolor{PineGreen}{\pm}}_{\;\! \mathcolor{gray}{z}}}$。此时,\textcolor{Plum}{输入}也从 $\textcolor{Plum}{3 \times 1}$ \textcolor{PineGreen}{总电场列向量}三分量 $\Xint{\mathcolor{gray}{-}}{25}{\bar{E}}^{\;\!\mathcolor{gray}{\omega}}_{\;\! \textcolor{Maroon}{\Yup} \mathcolor{gray}{z}} := \Xint{\mathcolor{gray}{-}}{25}{\bar{E}}^{\;\!\mathcolor{gray}{\omega}}_{\;\! \mathcolor{gray}{z}}$\Footnote{追加了一个下标 $\textcolor{Maroon}{\Yup}$ 以强调其相对于 $\textcolor{Maroon}{\symup{\rho}}$ 的“三分量”属性。相比而言,下标含 $\textcolor{Maroon}{\symup{\rho}}$ 的矢量只包含 2 个分量。} 降维成其 $\textcolor{Plum}{2 \times 1}$ \textcolor{Plum}{横向}部分 $\Xint{\mathcolor{gray}{-}}{25}{\bar{E}}^{\;\!\mathcolor{gray}{\omega}}_{\;\! \textcolor{Maroon}{\symup{\rho}} \mathcolor{gray}{z}}$。将 \bref{eq:amp_vec-transverse_input} 代入 \bref{eq:pre-on_the_way-transition_matrix'},将其更新为
\begin{align} \label{eq:pre-on_the_way-transition_matrix'-transverse_input}
	\Xint{\mathcolor{gray}{-}}{30}{\bar{E}}^{\;\!\mathcolor{gray}{\omega}}_{\;\! \textcolor{Maroon}{\Yup} \mathcolor{gray}{z}} &:= \overline{\Xint{\mathcolor{gray}{-}}{20}{\bar{G}}^{\;\!\mathcolor{gray}{\omega} \textcolor{PineGreen}{\pm}}_{\;\! \textcolor{Maroon}{\Yup} \mathcolor{gray}{z}}}^{\mathsf{\textcolor{Plum}{T}}} \cdot \overline{\Xint{\mathcolor{gray}{-}}{20}{\bar{G}}^{\;\!\mathcolor{gray}{\omega} \textcolor{PineGreen}{\pm}}_{\;\! \textcolor{Maroon}{\symup{\rho}} \mathcolor{gray}{z}}}^{\textcolor{Plum}{-\mathsf{T}}} \cdot \Xint{\mathcolor{gray}{-}}{30}{\bar{E}}^{\;\!\mathcolor{gray}{\omega}}_{\;\! \textcolor{Maroon}{\symup{\rho}} \mathcolor{gray}{z}} \xleftrightarrow[]{\text{\bref{eq:vec-polar_phase}}} \overline{\Xint{{}^{}_{\mathcolor{gray}{-}}}{10}{\bar{g}}^{\;\!\mathcolor{gray}{\omega} \textcolor{PineGreen}{\pm}}_{\;\! \textcolor{Maroon}{\Yup}}}^{\mathsf{\textcolor{Plum}{T}}} \cdot \overline{\overline{\mathbb{e}^{\mathbb{i} \Xint{\begin{smallmatrix} ~ \\ {}^{}_{\mathcolor{gray}{-}} \\ ~ \end{smallmatrix}}{15}{k}_{\symup{z}}^{\;\! \mathcolor{gray}{\omega} \textcolor{PineGreen}{\pm}} \mathcolor{gray}{z}}}} \cdot \overline{\Xint{\mathcolor{gray}{-}}{20}{\bar{G}}^{\;\!\mathcolor{gray}{\omega} \textcolor{PineGreen}{\pm}}_{\;\! \textcolor{Maroon}{\symup{\rho}} \mathcolor{gray}{z}}}^{\textcolor{Plum}{-\mathsf{T}}} \cdot \Xint{\mathcolor{gray}{-}}{30}{\bar{E}}^{\;\!\mathcolor{gray}{\omega}}_{\;\! \textcolor{Maroon}{\symup{\rho}} \mathcolor{gray}{z}} ~,
\end{align}
其中,$\overline{\overline{\mathbb{e}^{\mathbb{i} \Xint{\begin{smallmatrix} ~ \\ {}^{}_{\mathcolor{gray}{-}} \\ ~ \end{smallmatrix}}{15}{k}_{\symup{z}}^{\;\! \mathcolor{gray}{\omega} \textcolor{PineGreen}{\pm}} \mathcolor{gray}{z}}}}$ 也从 $\textcolor{Plum}{3 \times 3}$ 方阵降维为了 $\textcolor{Plum}{2 \times 2}$ 方阵,但保持\textcolor{Plum}{输出} $\Xint{\mathcolor{gray}{-}}{25}{\bar{E}}^{\;\!\mathcolor{gray}{\omega}}_{\;\! \textcolor{Maroon}{\Yup} \mathcolor{gray}{z}}$ 的\textcolor{Plum}{维度} $\textcolor{Plum}{3 \times 1}$ 没变。进一步地,更新 \bref{ssec:sandwich-eigen-matrices} 中所有涉及 $\overline{\Xint{\mathcolor{gray}{-}}{16}{\bar{G}}^{\;\!\mathcolor{gray}{\omega} \textcolor{PineGreen}{\pm}}_{\;\! \textcolor{Maroon}{\symup{\rho}} \mathcolor{gray}{z}}}^{\textcolor{Plum}{-\mathsf{T}}}$ 的公式,特别是\textcolor{PineGreen}{总电场矢量}三分量 $\Xint{\mathcolor{gray}{-}}{25}{\bar{E}}^{\;\!\mathcolor{gray}{\omega}}_{\;\! \textcolor{Maroon}{\Yup} \mathcolor{gray}{z}} \longleftrightarrow \Xint{\mathcolor{gray}{-}}{25}{\bar{E}}^{\;\!\mathcolor{gray}{\omega}}_{\;\! \textcolor{Maroon}{\Yup} \mathcolor{gray}{z_0}}$ 之间的 $\textcolor{Plum}{3 \times 3}$ \textcolor{Maroon}{转移矩阵},将其降维为 $\textcolor{Plum}{3 \times 2}$ \textcolor{Maroon}{转移矩阵}
\begin{align} \label{eq:transition_matrix-transverse_input}
	\Xint{\mathcolor{gray}{-}}{32}{\bar{\bar{T}}}^{\;\!\mathcolor{gray}{\omega}}_{\;\! \textcolor{Maroon}{\Yup} \mathcolor{gray}{z} \textcolor{Maroon}{\symup{\rho}} \mathcolor{gray}{z_0}} = \overline{\Xint{\mathcolor{gray}{-}}{20}{\bar{G}}^{\;\!\mathcolor{gray}{\omega} \textcolor{PineGreen}{\pm}}_{\;\! \textcolor{Maroon}{\Yup} \mathcolor{gray}{z}}}^{\mathsf{\textcolor{Plum}{T}}} \cdot \overline{\Xint{\mathcolor{gray}{-}}{20}{\bar{G}}^{\;\!\mathcolor{gray}{\omega} \textcolor{PineGreen}{\pm}}_{\;\! \textcolor{Maroon}{\symup{\rho}} \mathcolor{gray}{z_0}}}^{\textcolor{Plum}{-\mathsf{T}}} = \overline{\Xint{{}^{}_{\mathcolor{gray}{-}}}{10}{\bar{g}}^{\;\!\mathcolor{gray}{\omega} \textcolor{PineGreen}{\pm}}_{\;\! \textcolor{Maroon}{\Yup}}}^{\mathsf{\textcolor{Plum}{T}}} \cdot \overline{\overline{\mathbb{e}^{\mathbb{i} \Xint{\begin{smallmatrix} ~ \\ {}^{}_{\mathcolor{gray}{-}} \\ ~ \end{smallmatrix}}{15}{k}_{\symup{z}}^{\;\! \mathcolor{gray}{\omega} \textcolor{PineGreen}{\pm}} \left( \mathcolor{gray}{z} - \mathcolor{gray}{z_0} \right)}}} \cdot \overline{\Xint{{}^{}_{\mathcolor{gray}{-}}}{10}{\bar{g}}^{\;\!\mathcolor{gray}{\omega} \textcolor{PineGreen}{\pm}}_{\;\! \textcolor{Maroon}{\symup{\rho}}}}^{\textcolor{Plum}{-\mathsf{T}}} ~,
\end{align}
这样一来,在 \bref{eq:const_amp} 的条件下,\bref{eq:pre-on_the_way-transition_matrix'-transverse_input} 转变为 \bref{eq:E=transition_matrix-E} 的 $\textcolor{Plum}{3 \times 1}$ \textcolor{Plum}{输入}场 $\Xint{\mathcolor{gray}{-}}{25}{\bar{E}}^{\;\!\mathcolor{gray}{\omega}}_{\;\! \textcolor{Maroon}{\Yup} \mathcolor{gray}{z}}$ 的 $\textcolor{Plum}{2 \times 1}$ 降维 $\Xint{\mathcolor{gray}{-}}{25}{\bar{E}}^{\;\!\mathcolor{gray}{\omega}}_{\;\! \textcolor{Maroon}{\symup{\rho}} \mathcolor{gray}{z_0}}$ 对应物
\begin{align} \label{eq:E=transition_matrix-Erho}
	\Xint{\mathcolor{gray}{-}}{30}{\bar{E}}^{\;\!\mathcolor{gray}{\omega}}_{\;\! \textcolor{Maroon}{\Yup} \mathcolor{gray}{z}} &:= \Xint{\mathcolor{gray}{-}}{32}{\bar{\bar{T}}}^{\;\!\mathcolor{gray}{\omega}}_{\;\! \textcolor{Maroon}{\Yup} \mathcolor{gray}{z} \textcolor{Maroon}{\symup{\rho}} \mathcolor{gray}{z_0}} \cdot \Xint{\mathcolor{gray}{-}}{30}{\bar{E}}^{\;\!\mathcolor{gray}{\omega}}_{\;\! \textcolor{Maroon}{\symup{\rho}} \mathcolor{gray}{z_0}} ~,
\end{align}
以至于原本的 $\Xint{\mathcolor{gray}{-}}{25}{\bar{E}}^{\;\!\mathcolor{gray}{\omega}}_{\;\! \textcolor{Maroon}{\Yup} \mathcolor{gray}{z}} \longleftrightarrow \Xint{\mathcolor{gray}{-}}{25}{\bar{E}}^{\;\!\mathcolor{gray}{\omega}}_{\;\! \textcolor{Maroon}{\Yup} \mathcolor{gray}{z_0}}$ 通过 $\Xint{\mathcolor{gray}{-}}{30}{\bar{\bar{T}}}^{\;\!\mathcolor{gray}{\omega} \textcolor{Plum}{\left[3 \times 3\right]}}_{\;\! \textcolor{Maroon}{\Yup} \mathcolor{gray}{z} \textcolor{Maroon}{\Yup} \mathcolor{gray}{z_0}}, \Xint{\mathcolor{gray}{-}}{30}{\bar{\bar{T}}}^{\;\!\mathcolor{gray}{\omega} \textcolor{Plum}{\left[3 \times 3\right]}}_{\;\! \textcolor{Maroon}{\Yup} \mathcolor{gray}{z_0} \textcolor{Maroon}{\Yup} \mathcolor{gray}{z}}$ 的双向转换,现变为了 $\Xint{\mathcolor{gray}{-}}{25}{\bar{E}}^{\;\!\mathcolor{gray}{\omega}}_{\;\! \textcolor{Maroon}{\Yup} \mathcolor{gray}{z}} \longleftarrow \Xint{\mathcolor{gray}{-}}{25}{\bar{E}}^{\;\!\mathcolor{gray}{\omega}}_{\;\! \textcolor{Maroon}{\symup{\rho}} \mathcolor{gray}{z_0}}$ 或 $\Xint{\mathcolor{gray}{-}}{25}{\bar{E}}^{\;\!\mathcolor{gray}{\omega}}_{\;\! \textcolor{Maroon}{\Yup} \mathcolor{gray}{z_0}} \longleftarrow \Xint{\mathcolor{gray}{-}}{25}{\bar{E}}^{\;\!\mathcolor{gray}{\omega}}_{\;\! \textcolor{Maroon}{\symup{\rho}} \mathcolor{gray}{z}}$ 通过 $\Xint{\mathcolor{gray}{-}}{30}{\bar{\bar{T}}}^{\;\!\mathcolor{gray}{\omega} \textcolor{Plum}{\left[3 \times 2\right]}}_{\;\! \textcolor{Maroon}{\Yup} \mathcolor{gray}{z} \textcolor{Maroon}{\symup{\rho}} \mathcolor{gray}{z_0}}, \Xint{\mathcolor{gray}{-}}{30}{\bar{\bar{T}}}^{\;\!\mathcolor{gray}{\omega} \textcolor{Plum}{\left[3 \times 2\right]}}_{\;\! \textcolor{Maroon}{\Yup} \mathcolor{gray}{z_0} \textcolor{Maroon}{\symup{\rho}} \mathcolor{gray}{z}}$ 的 2 个相反方向的单向转换。

相应地,\bref{eq:E=transition_matrix-E-comprehension1} 修改为
\begin{align} \label{eq:E=transition_matrix-Erho-comprehension1}
	\Xint{\mathcolor{gray}{-}}{25}{\bar{E}}^{\;\!\mathcolor{gray}{\omega}}_{\;\! \textcolor{Maroon}{\Yup} \mathcolor{gray}{z}} \xleftarrow[\text{\bref{eq:vec-amp_eigenmode-matrix}}]{\overline{\Xint{\mathcolor{gray}{-}}{16}{\bar{G}}^{\;\!\mathcolor{gray}{\omega} \textcolor{PineGreen}{\pm}}_{\;\! \textcolor{Maroon}{\Yup} \mathcolor{gray}{z}}}^{\mathsf{\textcolor{Plum}{T}}} \cdot} \overline{\Xint{\begin{smallmatrix} ~ \\ {}^{}_{\mathcolor{gray}{-}} \\ ~ \end{smallmatrix}}{09}{\mathtt{g}}^{\;\!\mathcolor{gray}{\omega} \textcolor{PineGreen}{\pm}}_{\;\! \mathcolor{gray}{z}}}  \xleftarrow[\text{\bref{eq:const_amp}}]{\overline{\Xint{\begin{smallmatrix} ~ \\ {}^{}_{\mathcolor{gray}{-}} \\ ~ \end{smallmatrix}}{09}{\mathtt{g}}^{\;\!\mathcolor{gray}{\omega} \textcolor{PineGreen}{\pm}}_{\;\! \mathcolor{gray}{z}}} \equiv \overline{\Xint{\begin{smallmatrix} ~ \\ {}^{}_{\mathcolor{gray}{-}} \\ ~ \end{smallmatrix}}{09}{\mathtt{g}}^{\;\!\mathcolor{gray}{\omega} \textcolor{PineGreen}{\pm}}_{\;\! \mathcolor{gray}{z_0}}}} \overline{\Xint{\begin{smallmatrix} ~ \\ {}^{}_{\mathcolor{gray}{-}} \\ ~ \end{smallmatrix}}{09}{\mathtt{g}}^{\;\!\mathcolor{gray}{\omega} \textcolor{PineGreen}{\pm}}_{\;\! \mathcolor{gray}{z_0}}} \xleftarrow[\text{\bref{eq:amp_vec}}]{\overline{\Xint{\mathcolor{gray}{-}}{16}{\bar{G}}^{\;\!\mathcolor{gray}{\omega} \textcolor{PineGreen}{\pm}}_{\;\! \textcolor{Maroon}{\symup{\rho}} \mathcolor{gray}{z_0}}}^{\textcolor{Plum}{-\mathsf{T}}} \cdot} \Xint{\mathcolor{gray}{-}}{25}{\bar{E}}^{\;\!\mathcolor{gray}{\omega}}_{\;\! \textcolor{Maroon}{\symup{\rho}} \mathcolor{gray}{z_0}} ~,
\end{align}
同样,\bref{eq:E=transition_matrix-E-comprehension2} 修改为
\begin{align} \label{eq:E=transition_matrix-Erho-comprehension2}
	\Xint{\mathcolor{gray}{-}}{25}{\bar{E}}^{\;\!\mathcolor{gray}{\omega}}_{\;\! \textcolor{Maroon}{\Yup} \mathcolor{gray}{z}} \xleftarrow[\text{\bref{eq:vec-polar_G-matrix}}]{\overline{\Xint{{}^{}_{\mathcolor{gray}{-}}}{10}{\bar{g}}^{\;\!\mathcolor{gray}{\omega} \textcolor{PineGreen}{\pm}}_{\;\! \textcolor{Maroon}{\Yup}}}^{\mathsf{\textcolor{Plum}{T}}} \cdot} \overline{\Xint{\mathcolor{gray}{-}}{20}{\mathtt{G}}^{\;\!\mathcolor{gray}{\omega} \textcolor{PineGreen}{\pm}}_{\;\! \mathcolor{gray}{z}}}  \xleftarrow[\text{\bref{eq:pre-on_the_way-transition_matrix}}]{\overline{\overline{\mathbb{e}^{\mathbb{i} \Xint{\begin{smallmatrix} ~ \\ {}^{}_{\mathcolor{gray}{-}} \\ ~ \end{smallmatrix}}{15}{k}_{\symup{z}}^{\;\! \mathcolor{gray}{\omega} \textcolor{PineGreen}{\pm}} \left( \mathcolor{gray}{z} - \mathcolor{gray}{z_0} \right)}}} \cdot} \overline{\Xint{\mathcolor{gray}{-}}{20}{\mathtt{G}}^{\;\!\mathcolor{gray}{\omega} \textcolor{PineGreen}{\pm}}_{\;\! \mathcolor{gray}{z_0}}} \xleftarrow[\text{\bref{eq:vec-polar_G-matrix}}]{\overline{\Xint{{}^{}_{\mathcolor{gray}{-}}}{10}{\bar{g}}^{\;\!\mathcolor{gray}{\omega} \textcolor{PineGreen}{\pm}}_{\;\! \textcolor{Maroon}{\symup{\rho}}}}^{\textcolor{Plum}{-\mathsf{T}}} \cdot} \Xint{\mathcolor{gray}{-}}{25}{\bar{E}}^{\;\!\mathcolor{gray}{\omega}}_{\;\! \textcolor{Maroon}{\symup{\rho}} \mathcolor{gray}{z_0}} ~,
\end{align}
在这里,给出上述 \bref{eq:E=transition_matrix-Erho-comprehension2} 中各场量所对应的\textcolor{NavyBlue}{物理图像} \bref{fig:sandwich-eigen-matrices},包括\textcolor{Plum}{输入}/\textcolor{Plum}{输出}/\textcolor{Plum}{过程量} --- 4 个矢量/列向量场 $\Xint{\mathcolor{gray}{-}}{25}{\bar{E}}^{\;\!\mathcolor{gray}{\omega}}_{\;\! \textcolor{Maroon}{\Yup} \mathcolor{gray}{z}}, \overline{\Xint{\mathcolor{gray}{-}}{16}{\mathtt{G}}^{\;\!\mathcolor{gray}{\omega} \textcolor{PineGreen}{\pm}}_{\;\! \mathcolor{gray}{z}}}, \overline{\Xint{\mathcolor{gray}{-}}{16}{\mathtt{G}}^{\;\!\mathcolor{gray}{\omega} \textcolor{PineGreen}{\pm}}_{\;\! \mathcolor{gray}{z_0}}}, \Xint{\mathcolor{gray}{-}}{25}{\bar{E}}^{\;\!\mathcolor{gray}{\omega}}_{\;\! \textcolor{Maroon}{\symup{\rho}} \mathcolor{gray}{z_0}}$、\textcolor{Plum}{中间算符} --- 3 个张量/矩阵场 $\overline{\Xint{{}^{}_{\mathcolor{gray}{-}}}{10}{\bar{g}}^{\;\!\mathcolor{gray}{\omega} \textcolor{PineGreen}{\pm}}_{\;\! \textcolor{Maroon}{\Yup}}}^{\mathsf{\textcolor{Plum}{T}}}, \overline{\overline{\mathbb{e}^{\mathbb{i} \Xint{\begin{smallmatrix} ~ \\ {}^{}_{\mathcolor{gray}{-}} \\ ~ \end{smallmatrix}}{15}{k}_{\symup{z}}^{\;\! \mathcolor{gray}{\omega} \textcolor{PineGreen}{\pm}} \left( \mathcolor{gray}{z} - \mathcolor{gray}{z_0} \right)}}}, \overline{\Xint{{}^{}_{\mathcolor{gray}{-}}}{10}{\bar{g}}^{\;\!\mathcolor{gray}{\omega} \textcolor{PineGreen}{\pm}}_{\;\! \textcolor{Maroon}{\symup{\rho}}}}^{\textcolor{Plum}{-\mathsf{T}}}$,的实例化对象。

具体来说,\textcolor{PineGreen}{矢量总电场}的\textcolor{Maroon}{傅立叶谱} $\Xint{\mathcolor{gray}{-}}{25}{\bar{E}}^{\;\!\mathcolor{gray}{\omega}}_{\;\! \textcolor{Maroon}{\Yup} \mathcolor{gray}{z}}$ 及其\textcolor{Plum}{横向}部分 $\Xint{\mathcolor{gray}{-}}{25}{\bar{E}}^{\;\!\mathcolor{gray}{\omega}}_{\;\! \textcolor{Maroon}{\symup{\rho}} \mathcolor{gray}{z_0}}$ 和\textcolor{PineGreen}{含衍射本征复振幅}列向量 $\overline{\Xint{\mathcolor{gray}{-}}{16}{\mathtt{G}}^{\;\!\mathcolor{gray}{\omega} \textcolor{PineGreen}{\pm}}_{\;\! \mathcolor{gray}{z}}},\overline{\Xint{\mathcolor{gray}{-}}{16}{\mathtt{G}}^{\;\!\mathcolor{gray}{\omega} \textcolor{PineGreen}{\pm}}_{\;\! \mathcolor{gray}{z_0}}}$,作为 4 个被计算的\textcolor{Plum}{输入}/\textcolor{Plum}{输出}对象,被 2 个\textcolor{PineGreen}{本征偏振态矩阵场} $\overline{\Xint{{}^{}_{\mathcolor{gray}{-}}}{10}{\bar{g}}^{\;\!\mathcolor{gray}{\omega} \textcolor{PineGreen}{\pm}}_{\;\! \textcolor{Maroon}{\Yup}}}^{\mathsf{\textcolor{Plum}{T}}}, \overline{\Xint{{}^{}_{\mathcolor{gray}{-}}}{10}{\bar{g}}^{\;\!\mathcolor{gray}{\omega} \textcolor{PineGreen}{\pm}}_{\;\! \textcolor{Maroon}{\symup{\rho}}}}^{\textcolor{Plum}{-\mathsf{T}}}$,1 个\textcolor{PineGreen}{传播矩阵场} $\overline{\overline{\mathbb{e}^{\mathbb{i} \Xint{\begin{smallmatrix} ~ \\ {}^{}_{\mathcolor{gray}{-}} \\ ~ \end{smallmatrix}}{15}{k}_{\symup{z}}^{\;\! \mathcolor{gray}{\omega} \textcolor{PineGreen}{\pm}} \left( \mathcolor{gray}{z} - \mathcolor{gray}{z_0} \right)}}}$ 这 3 个算子所连接。

\clearpage

\begin{figure}[htbp!]
	\centering
	\includegraphics[width=0.9\textwidth]{D:/C2D/Desktop/article_fig/phd_thesis_fig/chapter-03/fig_1.主流程图-half-大论文.pdf}
	\biackcaption[\textbf{The core procedure for computing the optical vector fields between any two sections within an arbitrary $\bar{\bar{\varepsilon}}$ material, i.e., sequentially left-multipling three eigensystem matrix fields in reciprocal space.} 2D Fourier transform(FT) and its inverse(IFT) are used to realize transitions between real and reciprocal spaces. Below the first row, an example is provided for calculating the x, y, z components of the output vector optical field distribution (in the leftmost column). Initial condition: the known x, y components of the input 1064 nm pump (from the rightmost column), incident normally on the 15-mm-long KTP crystal with a 2$^{\circ}$ deviation off its optical axis. The vector pump is composed of vertically polarized $\textcolor{Maroon}{\text{LG}}^{p=2}_{l=50}$ and horizontally polarized $\text{HG}_{6,6}$. ]{-0.7em}{\textbf{计算任意 $\bar{\bar{\varepsilon}}$ 材料内任意两个横截面之间的\textcolor{PineGreen}{矢量光场}的核心过程,即在\textcolor{gray}{倒空间}中顺序左乘三个\textcolor{PineGreen}{本征系统}矩阵场。}配合 2D \textcolor{Plum}{傅立叶正}/\textcolor{Plum}{逆变换}(\textcolor{Plum}{FT}/\textcolor{Plum}{IFT})以实现\textcolor{gray}{正} $\longleftrightarrow$ \textcolor{gray}{倒空间}的过渡。在第一行下方,提供了一个计算\textcolor{Plum}{输出}\textcolor{PineGreen}{矢量光场}分布的 x,y,z 分量(在最左侧的列中)的示例。\textcolor{Maroon}{初始条件}:\textcolor{Plum}{输入} 1064 nm \textcolor{NavyBlue}{泵浦}的已知 x,y 分量(来自最右侧的列)\textcolor{Plum}{垂直}入射到 15 mm 长的 KTP 晶体上,偏离其\textcolor{PineGreen}{光轴} 2$^{\circ}$。矢量\textcolor{NavyBlue}{泵浦}由\textcolor{PineGreen}{垂直偏振}的 $\text{LG}^{p=2}_{l=50}$ 和\textcolor{PineGreen}{水平极化}的 $\text{HG}_{6,6}$ 组成\\}{fig:sandwich-eigen-matrices}
\end{figure}

其中,\textcolor{PineGreen}{快}、\textcolor{PineGreen}{慢}(\textcolor{PineGreen}{相速度} $\Xint{\begin{smallmatrix} ~ \\ {}^{}_{\mathcolor{gray}{-}} \\ ~ \end{smallmatrix}}{17}{\bar{v}}_{\mathcolor{gray}{\omega}}^{\;\! \textcolor{PineGreen}{+}}, \Xint{\begin{smallmatrix} ~ \\ {}^{}_{\mathcolor{gray}{-}} \\ ~ \end{smallmatrix}}{17}{\bar{v}}_{\mathcolor{gray}{\omega}}^{\;\! \textcolor{PineGreen}{-}}$ 所对应的\textcolor{PineGreen}{模式}的)\textcolor{PineGreen}{折射率} $\Xint{\begin{smallmatrix} ~ \\ {}^{}_{\mathcolor{gray}{-}} \\ ~ \end{smallmatrix}}{11}{n}^{\textcolor{PineGreen}{+}}_{\mathcolor{gray}{\omega}}, \Xint{\begin{smallmatrix} ~ \\ {}^{}_{\mathcolor{gray}{-}} \\ ~ \end{smallmatrix}}{11}{n}^{\textcolor{PineGreen}{-}}_{\mathcolor{gray}{\omega}}$ 片,关于这里透明 KTP 晶体内的\textcolor{Plum}{均匀}实\textcolor{PineGreen}{波矢} $\Xint{\begin{smallmatrix} ~ \\ {}^{}_{\mathcolor{gray}{-}} \\ ~ \end{smallmatrix}}{15}{\bar{k}}^{\;\! \textcolor{PineGreen}{\pm}}_{\mathcolor{gray}{\omega}}$ 的函数的定义,见文献\cite{berryOpticalSingularitiesBirefringent2003,xieAnalytic3DVector}。这 $\textcolor{PineGreen}{\pm}$ 两个\textcolor{PineGreen}{本征模}式的\textcolor{PineGreen}{本征值}隐式地构成了 \bref{fig:sandwich-eigen-matrices} 正中间列的 $\textcolor{Plum}{2 \times 2}$  \textcolor{PineGreen}{传播矩阵}(场)$\overline{\overline{\mathbb{e}^{\mathbb{i} \Xint{\begin{smallmatrix} ~ \\ {}^{}_{\mathcolor{gray}{-}} \\ ~ \end{smallmatrix}}{15}{k}_{\symup{z}}^{\;\! \mathcolor{gray}{\omega} \textcolor{PineGreen}{\pm}} \left( \mathcolor{gray}{z} - \mathcolor{gray}{z_0} \right)}}}$。

然而,上 \bref{fig:sandwich-eigen-matrices} 只给出了\textcolor{PineGreen}{透明双轴}晶体中\textcolor{PineGreen}{矢量光束}\textcolor{gray}{正空间}电场三分量,在正空间中的\textcolor{PineGreen}{双折射}和\textcolor{Plum}{各向异性}衍射的例子,是如何通过倒空间的“三明治矩阵连乘” \bref{eq:E=transition_matrix-Erho-comprehension2} 的方法,配合\textcolor{gray}{正} $\longleftrightarrow$ \textcolor{gray}{倒空间}的 2D \textcolor{Plum}{傅立叶正}/\textcolor{Plum}{逆变换}来计算得到。从材料系数的角度,对应到\textcolor{Plum}{数学}上,这只涉及了 $\bar{\bar{\varepsilon}}^{\;\! \mathcolor{gray}{\omega}}_{\textcolor{Maroon}{(1)}}$ 的\textcolor{Plum}{实数部分}、\textcolor{Plum}{厄米部分}和\textcolor{Plum}{对称部分},三者的\textcolor{Plum}{交集}。因此,下一节将通过参数化扫描 $\bar{\bar{\varepsilon}}^{\;\! \mathcolor{gray}{\omega}}_{\textcolor{Maroon}{(1)}}$ 的“\textcolor{Plum}{三条边}”的数值实验,探索并证明:实际上本模型可以计算具有任意\textcolor{Plum}{非厄米}、\textcolor{Plum}{非对称}的\textcolor{Plum}{复} $\bar{\bar{\varepsilon}}^{\;\! \mathcolor{gray}{\omega}}_{\textcolor{Maroon}{(1)}}$ 张量的\textcolor{PineGreen}{纯电各向异性}材料。

\marginLeft[-2.4em]{sec:birefringent-chiral-dichroic}\section{\textcolor{Maroon}{Compute} 双折射-手性-非厄米晶体中的 \textcolor{Maroon}{electromagnetic waves}}\label{sec:birefringent-chiral-dichroic}

\vspace*{-5.5em}

\marginLeft[-2.4em]{ssec:2D-reciprocal-eigensystems}\subsection{任意 \texorpdfstring{$\bar{\bar{\varepsilon}}$}{$\bar{\bar{\text{ε}}}$} 材料内,本征值-向量场在 2D 倒空间中的分布}\label{ssec:2D-reciprocal-eigensystems}

最广义的\textcolor{Plum}{复}介电张量 $\bar{\bar{\varepsilon}}^{\;\! \mathcolor{gray}{\omega}}_{\textcolor{Maroon}{(1)}}$ 对应\textcolor{NavyBlue}{双折射-手性/旋光性/光学活性-耗散/增益/吸收/二向色性/非厄米} birefringent-chiral-dichroic 电介质\cite{samlanChiralDynamicsExceptional2018}。为了将 $\bar{\bar{\varepsilon}}^{\;\! \mathcolor{gray}{\omega}}_{\textcolor{Maroon}{(1)}}$ 矩阵的\textcolor{Plum}{数学性质},与材料的\textcolor{NavyBlue}{光学}/\textcolor{NavyBlue}{物理性质}联系起来,我们提出了 \bref{fig:MMTC-2D_eigensystems} 中的\textcolor{NavyBlue}{材料-矩阵正四面体罗盘},即 the material-matrix tetrahedral compass(\textcolor{NavyBlue}{M-M TC})。

\textcolor{NavyBlue}{正四面体罗盘}由 \textcolor{Plum}{6 条边}组成:\textcolor{Plum}{厄米} Hermitian(\textcolor{Plum}{H})、\textcolor{Plum}{反厄米} anti-Hermitian(\textcolor{Plum}{$!$H})\Footnote{是\textcolor{Plum}{非厄米} non-Hermitian 的\textcolor{Plum}{子集},是更“纯净”的\textcolor{Plum}{非厄米}。}、\textcolor{Plum}{对称} Symmetric(\textcolor{Plum}{S})、\textcolor{Plum}{反对称} anti-Symmetric(\textcolor{Plum}{$!$S})、\textcolor{Plum}{实部} Real part(\textcolor{Plum}{Re}) 和\textcolor{Plum}{虚部} Imaginary part(\textcolor{Plum}{Im}) --- 对应于以 $\bar{\bar{\varepsilon}}^{\;\! \mathcolor{gray}{\omega}}_{\textcolor{Maroon}{(1)}}$ 为代表的复矩阵的六个\textcolor{Plum}{数学属性}。

\textcolor{NavyBlue}{正四面体罗盘}由 \textcolor{NavyBlue}{4 个顶点}组成:\textcolor{NavyBlue}{双折射性} Birefringence(\textcolor{NavyBlue}{Bi})、\textcolor{NavyBlue}{光学活性} Optical Activity(\textcolor{NavyBlue}{OA})、\textcolor{NavyBlue}{线二色性} Linear Dichroism(\textcolor{NavyBlue}{LD}) 和\textcolor{NavyBlue}{圆二色性} Circular Dichroism(\textcolor{NavyBlue}{CD}),描述了物质的四种\textcolor{NavyBlue}{光学性质}\Footnote{使用了\textcolor{NavyBlue}{海军蓝}来描述\textcolor{NavyBlue}{物理术语},见\bref{hook:NavyBlue}。}。

\textcolor{PineGreen}{纯电各向异性}材料中的\textcolor{Plum}{线性}\textcolor{PineGreen}{晶体光学}从 \bref{eq:simplify7-L2-zeta} 的\textcolor{PineGreen}{本征系统}开始。\bref{eq:simplify7-L2-zeta} 中的可能为\textcolor{Plum}{复}张量的 $\bar{\bar{\varepsilon}}^{\;\! \mathcolor{gray}{\omega}}_{\textcolor{Maroon}{(1)}}$ 在 11 种情况中的 7 种情况下通常是不可对角化的,除非 $\bar{\bar{\varepsilon}}^{\;\! \mathcolor{gray}{\omega}}_{\textcolor{Maroon}{(1)}}$ 是一个\textcolor{Plum}{正规矩阵},对应于 4 种可能的组合\Footnote{\textcolor{Plum}{正规矩阵}并非只有这 4 种组合,比如\textcolor{Plum}{正交}、\textcolor{Plum}{酉矩阵},以及其他具有\textcolor{Plum}{正规性}但不满足上述条件的矩阵。},即 \textcolor{NavyBlue}{光学活性 OA} $+$ \textcolor{NavyBlue}{双折射性 Bi}、\textcolor{NavyBlue}{线二色性 LD} $+$ \textcolor{NavyBlue}{圆二色性 CD}、\textcolor{NavyBlue}{双折射性 Bi} $+$ \textcolor{NavyBlue}{线二色性 LD}、\textcolor{NavyBlue}{光学活性 OA} $+$ \textcolor{NavyBlue}{圆二色性 CD},分别对应 4 条边 \textcolor{Plum}{厄米 H}、\textcolor{Plum}{反厄米 $!$H}、\textcolor{Plum}{对称 S}、\textcolor{Plum}{反对称 $!$S}。

在\textcolor{NavyBlue}{正四面体罗盘}的帮助下,现开始探索\textcolor{NavyBlue}{圆二色性 CD}、\textcolor{NavyBlue}{光学活性 OA} 和\textcolor{NavyBlue}{线二色性 LD} 之间的\textcolor{Plum}{成对}/\textcolor{Plum}{两两竞争},以及它如何影响\textcolor{PineGreen}{特征系统}对的分布的\textcolor{NavyBlue}{绝热演化}。

\begin{figure}[htbp!]
	\centering
	\includegraphics[width=1.0\textwidth]{D:/C2D/Desktop/article_fig/phd_thesis_fig/chapter-03/fig_2.正四面体_2D_本征系统.pdf}
	\biackcaption[\textbf{Scan OA, LD, and CD along 3 edges (`!S', `!H', and `Im') of the M-M TC to see the adiabatic evolution of eigensystems in 2D reciprocal $\symbf{\mathcolor{gray}{\bar{k}_{\symup{\rho}}}}$ space.} In the pairwise competition between OA, LD, and CD, one remains fixed, as reflected by the Plum number 1 in the proportional scale. The colors of polarization ellipses, C points(smaller circles) and singularities(solid dots), map to ellipticity $\Xint{\begin{smallmatrix} ~ \\ {}^{}_{\mathcolor{gray}{-}} \\ ~ \end{smallmatrix}}{08}{e}^{\;\!\textcolor{PineGreen}{\pm}}_{\mathcolor{gray}{\omega}} = \tan \left( \Xint{{}^{}_{\mathcolor{gray}{-}}}{23}{\chi}^{\;\!\textcolor{PineGreen}{\pm}}_{\mathcolor{gray}{\omega}} \right) = \Xint{\mathcolor{gray}{-}}{08}{b}^{\;\!\textcolor{PineGreen}{\pm}}_{\mathcolor{gray}{\omega}}/\Xint{\begin{smallmatrix} ~ \\ {}^{}_{\mathcolor{gray}{-}} \\ ~ \end{smallmatrix}}{08}{a}^{\;\!\textcolor{PineGreen}{\pm}}_{\mathcolor{gray}{\omega}}$, where red/blue correspond to right-/left-handed chirality, $-$/$+$ spin quantum number $\sigma$, as well as $+$/$-$ ellipticity, respectively. The eigenvector ellipses and the L shorelines, both depicted with dashed/solid lines, represent the slow/fast $=$ s/f modes, corresponding to the upper/lower eigenvalue $\Xint{\mathcolor{gray}{-}}{10}{N}^{\;\!\pm}_{\omega}$ sheets of $-$/$+$ modes. \textbf{a} By maintaining a fixed CD, increase OA incrementally from 0 to analyze how the CD-OA competition influences the eigensystems. \textbf{b} With CD (3 columns from the left) or LD (rightmost column) held constant, progressively raise LD or CD starting from 0 to observe the effects of CD-LD competition. \textbf{c} While keeping OA (3 rows from the top) or LD (bottom row) constant, gradually increase LD or OA from 0 to examine the impact of the OA-LD competition on the eigensystems.]{-0.7em}{\textbf{沿\textcolor{NavyBlue}{材料-矩阵正四面体罗盘}的 3 条边(\textcolor{Plum}{反对称 `!S'},\textcolor{Plum}{反厄米 `!H'} 和\textcolor{Plum}{虚部 `Im'})扫描 \textcolor{NavyBlue}{光学活性 OA}、\textcolor{NavyBlue}{线二色性 LD} 和\textcolor{NavyBlue}{圆二色性 CD},以查看二维\textcolor{gray}{倒} $\symbf{\mathcolor{gray}{\bar{k}_{\symup{\rho}}}}$ \textcolor{gray}{空间}中\textcolor{PineGreen}{本征系统}的\textcolor{NavyBlue}{绝热演化}。}在这\textcolor{Plum}{三个属性}之间的\textcolor{Plum}{两两竞争}中,恒保持其中一个不变,正如比例标度中的\textcolor{Purple}{紫色数字 1} 所反映的那样。\textcolor{PineGreen}{本征偏振态} $\Xint{{}^{}_{\mathcolor{gray}{-}}}{10}{\bar{g}}^{\;\!\mathcolor{gray}{\omega} \textcolor{PineGreen}{\pm}}_{\;\! \textcolor{Maroon}{\Yup}}$ \textcolor{PineGreen}{椭圆}、\textcolor{PineGreen}{C 点}(较小的圆圈,代表圆偏振)和\textcolor{PineGreen}{光学奇点}/\textcolor{PineGreen}{奇异轴}/\textcolor{PineGreen}{例外点}(实心点)的颜色,全都映射到\textcolor{Plum}{椭圆率} $\Xint{\begin{smallmatrix} ~ \\ {}^{}_{\mathcolor{gray}{-}} \\ ~ \end{smallmatrix}}{08}{e}^{\;\!\textcolor{PineGreen}{\pm}}_{\mathcolor{gray}{\omega}} = \tan \left( \Xint{{}^{}_{\mathcolor{gray}{-}}}{23}{\chi}^{\;\!\textcolor{PineGreen}{\pm}}_{\mathcolor{gray}{\omega}} \right) = \Xint{\mathcolor{gray}{-}}{08}{b}^{\;\!\textcolor{PineGreen}{\pm}}_{\mathcolor{gray}{\omega}}/\Xint{\begin{smallmatrix} ~ \\ {}^{}_{\mathcolor{gray}{-}} \\ ~ \end{smallmatrix}}{08}{a}^{\;\!\textcolor{PineGreen}{\pm}}_{\mathcolor{gray}{\omega}}$,其中红色/蓝色分别对应右手/左手\textcolor{NavyBlue}{手性}、$-$/$+$ \textcolor{NavyBlue}{自旋量子数} $\textcolor{NavyBlue}{\sigma}$ 以及 $+$/$-$ \textcolor{Plum}{椭圆率}。\textcolor{PineGreen}{特征向量}\textcolor{PineGreen}{椭圆}和“\textcolor{PineGreen}{L 海岸}”的\textcolor{PineGreen}{虚线}/\textcolor{PineGreen}{实线},分别表示\textcolor{PineGreen}{慢}/\textcolor{PineGreen}{快} $=\textcolor{PineGreen}{s}$/$\textcolor{PineGreen}{f}$ \textcolor{PineGreen}{模式},对应于 $\textcolor{PineGreen}{-}$/$\textcolor{PineGreen}{+}$ \textcolor{PineGreen}{模式}的上/下\textcolor{Plum}{复}\textcolor{PineGreen}{有效折射率}的\textcolor{Plum}{实部} $\Xint{\mathcolor{gray}{-}}{10}{N}^{\;\!\textcolor{PineGreen}{\pm}}_{\mathcolor{gray}{\omega}}$ 片。\textbf{a} 通过维持固定的\textcolor{NavyBlue}{圆二色性 CD}, 从 0 开始逐步增加\textcolor{NavyBlue}{光学活性 OA}, 以分析 \textcolor{NavyBlue}{CD-OA} 竞争如何影响\textcolor{PineGreen}{特征系统}。\textbf{b} \textcolor{NavyBlue}{圆二色性 CD}(靠左的 3 列)或 \textcolor{NavyBlue}{线二色性 LD}(最右列)保持恒定,从 0 开始逐渐提高 \textcolor{NavyBlue}{LD} 或 \textcolor{NavyBlue}{CD},以观察 \textcolor{NavyBlue}{LD-CD} 竞争的效果。\textbf{c} 保持\textcolor{NavyBlue}{光学活性 OA}(顶部 3 行)或 \textcolor{NavyBlue}{线二色性 LD}(底行)恒定,逐渐从 0 增加 \textcolor{NavyBlue}{LD} 或 \textcolor{NavyBlue}{OA},以检查 \textcolor{NavyBlue}{OA-LD} 竞争对\textcolor{PineGreen}{特征系统}的影响\\}{fig:MMTC-2D_eigensystems}
\end{figure}

让我们从 \bref{fig:MMTC-2D_eigensystems} 中三个完整内圈的最外圈开始,它表示\textcolor{PineGreen}{有效折射率}的\textcolor{Plum}{实部} $\Xint{\mathcolor{gray}{-}}{10}{N}^{\;\!\textcolor{PineGreen}{\pm}}_{\mathcolor{gray}{\omega}}$\cite{xieAnalytic3DVector}\Footnote{对于\textcolor{NavyBlue}{吸收}以\textcolor{NavyBlue}{衰减}/\textcolor{NavyBlue}{增益}以\textcolor{NavyBlue}{放大}特定频段的电磁波的材料,在\textcolor{NavyBlue}{傅立叶光学}的\textcolor{Plum}{非均匀}\textcolor{Plum}{复}\textcolor{PineGreen}{波矢} $\Xint{\begin{smallmatrix} ~ \\ {}^{}_{\mathcolor{gray}{-}} \\ ~ \end{smallmatrix}}{15}{\bar{k}}^{\;\! \textcolor{PineGreen}{\pm}}_{\mathcolor{gray}{\omega}}$\cite{wangComplexRayTracing2008a,changRayTracingAbsorbing2005,sturmElectromagneticWavesCrystals2024}的视角下,原则上\textbf{没有通常意义上的\textcolor{PineGreen}{折射率}可言},因为 $\Xint{\begin{smallmatrix} ~ \\ {}^{}_{\mathcolor{gray}{-}} \\ ~ \end{smallmatrix}}{15}{\bar{k}}^{\;\! \textcolor{PineGreen}{\pm}}_{\mathcolor{gray}{\omega}} \in \mathcolor{gray}{\bar{\mathbb{C}}_{\textcolor{Plum}{3}}}$ 的实部 $\Xint{\begin{smallmatrix} ~ \\ {}^{}_{\mathcolor{gray}{-}} \\ ~ \end{smallmatrix}}{15}{\bar{k}}^{\;\! \mathcolor{gray}{\omega} \textcolor{PineGreen}{\pm}}_{\textcolor{Plum}{\text{R}}} \in \mathcolor{gray}{\bar{\mathbb{R}}_{\textcolor{Plum}{3}}}$ 和虚部 $\Xint{\begin{smallmatrix} ~ \\ {}^{}_{\mathcolor{gray}{-}} \\ ~ \end{smallmatrix}}{15}{\bar{k}}^{\;\! \mathcolor{gray}{\omega} \textcolor{PineGreen}{\pm}}_{\textcolor{Plum}{\text{I}}} \in \mathcolor{gray}{\bar{\mathbb{R}}_{\textcolor{Plum}{3}}}$ 两个实矢量的方向不同。对此,最多只能定义一个\textcolor{PineGreen}{\textbf{有效/表观折射率}}\cite{changRayTracingAbsorbing2005} $\left( \Xint{\mathcolor{gray}{-}}{10}{N}^{\;\!\textcolor{PineGreen}{\pm}}_{\mathcolor{gray}{\omega}}, \Xint{\mathcolor{gray}{-}}{10}{K}^{\;\!\textcolor{PineGreen}{\pm}}_{\mathcolor{gray}{\omega}} \right) := \left( |\Xint{\begin{smallmatrix} ~ \\ {}^{}_{\mathcolor{gray}{-}} \\ ~ \end{smallmatrix}}{15}{\bar{k}}^{\;\! \mathcolor{gray}{\omega} \textcolor{PineGreen}{\pm}}_{\textcolor{Plum}{\text{R}}}|, |\Xint{\begin{smallmatrix} ~ \\ {}^{}_{\mathcolor{gray}{-}} \\ ~ \end{smallmatrix}}{15}{\bar{k}}^{\;\! \mathcolor{gray}{\omega} \textcolor{PineGreen}{\pm}}_{\textcolor{Plum}{\text{I}}}| \right) \big/ k_{\textcolor{Maroon}{\mathsf{o}}}^{\;\! \mathcolor{gray}{\omega}}$。},然后是\textcolor{PineGreen}{有效折射率}的\textcolor{Plum}{虚部} $\Xint{\mathcolor{gray}{-}}{10}{K}^{\;\!\textcolor{PineGreen}{\pm}}_{\mathcolor{gray}{\omega}}$,然后是\textcolor{PineGreen}{本征偏振态} $\Xint{{}^{}_{\mathcolor{gray}{-}}}{10}{\bar{g}}^{\;\!\mathcolor{gray}{\omega} \textcolor{PineGreen}{\pm}}_{\;\! \textcolor{Maroon}{\Yup}}$,从外圈到内圈排列。关于这三个圆圈的完整\textcolor{NavyBlue}{绝热演化}过程,见\href{https://www.youtube.com/watch?v=Inl1AkDAblo}{视频 1.1}或\href{https://www.youtube.com/watch?v=CuFjKcdAIJQ}{视频 1.2}。

\bref{fig:MMTC-2D_eigensystems}\textbf{a} 中\textcolor{NavyBlue}{正四面体罗盘}的 “\textcolor{Plum}{反对称 $!$S}” 边显示了 \textcolor{NavyBlue}{OA} 从\textcolor{Plum}{顶点}“\textcolor{NavyBlue}{圆二色性 `CD'}” 处的 0 增加到 “\textcolor{NavyBlue}{光学活性 OA}” 处达到其\textcolor{Plum}{最大值}时,\textcolor{PineGreen}{本征系统}的\textcolor{NavyBlue}{绝热演化}(由内侧浅蓝色箭头引导)。在 \textcolor{NavyBlue}{OA} $<$ \textcolor{NavyBlue}{CD} 期间,\textcolor{PineGreen}{有效折射率}对片的\textcolor{Plum}{实部}和\textcolor{Plum}{虚部} $\Xint{\mathcolor{gray}{-}}{10}{N}^{\;\!\textcolor{PineGreen}{\pm}}_{\mathcolor{gray}{\omega}}, \Xint{\mathcolor{gray}{-}}{10}{K}^{\;\!\textcolor{PineGreen}{\pm}}_{\mathcolor{gray}{\omega}}$ 将首先与\textcolor{Plum}{恒定半径}的\textcolor{PineGreen}{奇异盘}内围绕\textcolor{PineGreen}{光轴}的初始\textcolor{PineGreen}{粘附状态}分离。当 \textcolor{NavyBlue}{OA} 增加至可以与 \textcolor{NavyBlue}{CD} 抗衡(1:1)时,\textcolor{PineGreen}{固着的奇异盘}从内部完全分解,形成一个中空的\textcolor{PineGreen}{奇异环}。在 \textcolor{NavyBlue}{OA} $>$ \textcolor{NavyBlue}{CD} 之后,\textcolor{PineGreen}{粘附区域} $=$ \textcolor{PineGreen}{奇异点} $=$ \textcolor{PineGreen}{简并特征值对}溶解,两个 $\Xint{\mathcolor{gray}{-}}{10}{N}^{\;\!\textcolor{PineGreen}{\pm}}_{\mathcolor{gray}{\omega}}, \Xint{\mathcolor{gray}{-}}{10}{K}^{\;\!\textcolor{PineGreen}{\pm}}_{\mathcolor{gray}{\omega}}$ 片都演变成两个\textcolor{Plum}{镜像对称}的“\textcolor{Maroon}{墨西哥宽边帽}”,其中 $\Xint{\mathcolor{gray}{-}}{10}{N}^{\;\!\textcolor{PineGreen}{\pm}}_{\mathcolor{gray}{\omega}}$ 片的冠部更平坦,边缘更卷曲;相反,$\Xint{\mathcolor{gray}{-}}{10}{K}^{\;\!\textcolor{PineGreen}{\pm}}_{\mathcolor{gray}{\omega}}$ 片的边缘完全\textcolor{Plum}{水平},它的冠部却更凸\cite{kirillovUnfoldingEigenvalueSurfaces2005}。随着 \textcolor{NavyBlue}{OA} 从 0 开始增加,成对的 $\Xint{\mathcolor{gray}{-}}{10}{N}^{\;\!\textcolor{PineGreen}{\pm}}_{\mathcolor{gray}{\omega}}, \Xint{\mathcolor{gray}{-}}{10}{K}^{\;\!\textcolor{PineGreen}{\pm}}_{\mathcolor{gray}{\omega}}$ 片的分离顺序如下:从\textcolor{PineGreen}{圆盘}的中心到外边缘(即\textcolor{PineGreen}{圆环})$\to$ 沿着\textcolor{PineGreen}{圆环} $\to$ $\Xint{\mathcolor{gray}{-}}{10}{N}^{\;\!\textcolor{PineGreen}{\pm}}_{\mathcolor{gray}{\omega}}, \Xint{\mathcolor{gray}{-}}{10}{K}^{\;\!\textcolor{PineGreen}{\pm}}_{\mathcolor{gray}{\omega}}$ 整体。在这个过程中,$\Xint{\mathcolor{gray}{-}}{10}{K}^{\;\!\textcolor{PineGreen}{\pm}}_{\mathcolor{gray}{\omega}}$ 的两个阔边帽的\textcolor{Plum}{绝对高度}几乎保持不变,几乎完全由 \textcolor{NavyBlue}{CD} 决定,与 \textcolor{NavyBlue}{OA} 的相关性很小。

现在考虑相应\textcolor{PineGreen}{特征向量}分布所对应的\textcolor{NavyBlue}{绝热演化}。当 \textcolor{NavyBlue}{光学活性 OA} $= 0$ 时,光盘内充斥着\textcolor{Plum}{无限数量}的\textcolor{PineGreen}{光学奇点},其中\textcolor{PineGreen}{特征向量}成对退化为随机\textcolor{NavyBlue}{左旋}或\textcolor{NavyBlue}{右旋}\textcolor{PineGreen}{椭圆偏振},其\textcolor{Plum}{椭圆度}的大小 $|\Xint{\begin{smallmatrix} ~ \\ {}^{}_{\mathcolor{gray}{-}} \\ ~ \end{smallmatrix}}{08}{e}^{\;\!\textcolor{PineGreen}{\pm}}_{\mathcolor{gray}{\omega}}|$ 从中心(即\textcolor{PineGreen}{光轴})到外围减小。随着 \textcolor{NavyBlue}{OA} 的增加,\textcolor{PineGreen}{盘}内的\textcolor{PineGreen}{奇点}逐渐以\textcolor{Plum}{十字形图案}消散,最终在 \textcolor{NavyBlue}{OA} 到达 \textcolor{NavyBlue}{圆二色性 CD} 时,完全消失。此时,“\textcolor{PineGreen}{披萨盘}”剩余的四个\textcolor{PineGreen}{退化的}扇形区域也消失了,留下了一圈\textcolor{PineGreen}{环状奇点},这是判断 \textcolor{NavyBlue}{CD} $=$ \textcolor{NavyBlue}{OA} 的“\textcolor{Maroon}{金标准}”。当 \textcolor{NavyBlue}{OA} 开始超过 \textcolor{NavyBlue}{CD} 时,\textcolor{PineGreen}{环状奇点}开始随机和零星地消散。一旦\textcolor{PineGreen}{奇点环}完全消散,对应于 $|\Xint{\begin{smallmatrix} ~ \\ {}^{}_{\mathcolor{gray}{-}} \\ ~ \end{smallmatrix}}{08}{e}^{\;\!\textcolor{PineGreen}{\pm}}_{\mathcolor{gray}{\omega}}|_{\textcolor{Plum}{\min}} + 20\% (\textcolor{Plum}{\langle\textcolor{black}{|\Xint{\begin{smallmatrix} ~ \\ {}^{}_{\mathcolor{gray}{-}} \\ ~ \end{smallmatrix}}{08}{e}^{\;\!\textcolor{PineGreen}{\pm}}_{\mathcolor{gray}{\omega}}|}\rangle} - |\Xint{\begin{smallmatrix} ~ \\ {}^{}_{\mathcolor{gray}{-}} \\ ~ \end{smallmatrix}}{08}{e}^{\;\!\textcolor{PineGreen}{\pm}}_{\mathcolor{gray}{\omega}}|_{\textcolor{Plum}{\min}})$\Footnote{$\textcolor{Plum}{\langle\textcolor{black}{|\Xint{\begin{smallmatrix} ~ \\ {}^{}_{\mathcolor{gray}{-}} \\ ~ \end{smallmatrix}}{08}{e}^{\;\!\textcolor{PineGreen}{\pm}}_{\mathcolor{gray}{\omega}}|}\rangle}$ 表示 $|\Xint{\begin{smallmatrix} ~ \\ {}^{}_{\mathcolor{gray}{-}} \\ ~ \end{smallmatrix}}{08}{e}^{\;\!\textcolor{PineGreen}{\pm}}_{\mathcolor{gray}{\omega}}|$ 的\textcolor{Plum}{平均值}。}的接近\textcolor{PineGreen}{线性极化}的\textcolor{PineGreen}{特征向量}的\textcolor{Plum}{阈值环}开始膨胀,直到它与 \textcolor{NavyBlue}{OA} 一起达到\textcolor{Plum}{最大值}。

沿着 \bref{fig:MMTC-2D_eigensystems}\textbf{c} 中\textcolor{NavyBlue}{正四面体罗盘}的“\textcolor{Plum}{虚部 Im}” \textcolor{Plum}{边},随着 \textcolor{NavyBlue}{线二色性 LD} 从\textcolor{Plum}{顶点}“\textcolor{NavyBlue}{光学活性 `OA'}”处的 0 增加到“\textcolor{NavyBlue}{LD}”处的\textcolor{Plum}{最大值}(由内侧浅绿色箭头表示),成对的 $\Xint{\mathcolor{gray}{-}}{10}{N}^{\;\!\textcolor{PineGreen}{\pm}}_{\mathcolor{gray}{\omega}}$ 片从“\textcolor{Plum}{分离}”状态过渡到“\textcolor{Plum}{线接触}”,在整体更高的 $\Xint{\mathcolor{gray}{-}}{10}{N}^{\;\!\textcolor{PineGreen}{-}}_{\mathcolor{gray}{\omega}}$ 内形成\textcolor{Maroon}{半沟槽凹陷},\textcolor{Plum}{降低}至(并部分\textcolor{Plum}{穿透})总体位置较低的 $\Xint{\mathcolor{gray}{-}}{10}{N}^{\;\!\textcolor{PineGreen}{+}}_{\mathcolor{gray}{\omega}}$ 以下。$\Xint{\mathcolor{gray}{-}}{10}{N}^{\;\!\textcolor{PineGreen}{-}}_{\mathcolor{gray}{\omega}}$ 的\textcolor{Maroon}{沟槽}的一侧逐渐下降到 $\Xint{\mathcolor{gray}{-}}{10}{N}^{\;\!\textcolor{PineGreen}{+}}_{\mathcolor{gray}{\omega}}$ 下方,而另一侧突然上升并再次超过 $\Xint{\mathcolor{gray}{-}}{10}{N}^{\;\!\textcolor{PineGreen}{+}}_{\mathcolor{gray}{\omega}}$。相比之下,$\Xint{\mathcolor{gray}{-}}{10}{N}^{\;\!\textcolor{PineGreen}{+}}_{\mathcolor{gray}{\omega}}$ 的\textcolor{Plum}{峰值}表现恰恰相反:一侧稳步上升到 $\Xint{\mathcolor{gray}{-}}{10}{N}^{\;\!\textcolor{PineGreen}{-}}_{\mathcolor{gray}{\omega}}$ 之上,而另一侧则急剧下降到 $\Xint{\mathcolor{gray}{-}}{10}{N}^{\;\!\textcolor{PineGreen}{-}}_{\mathcolor{gray}{\omega}}$ 之下。除\textcolor{Maroon}{台阶状悬崖}处的不连续性外,$\Xint{\mathcolor{gray}{-}}{10}{N}^{\;\!\textcolor{PineGreen}{+}}_{\mathcolor{gray}{\omega}}, \Xint{\mathcolor{gray}{-}}{10}{N}^{\;\!\textcolor{PineGreen}{-}}_{\mathcolor{gray}{\omega}}$ 片共同形成了一个\textcolor{Plum}{连续互补的相交流形},其\textcolor{Maroon}{半峰/沟槽}相互连接,形成了一条“\textcolor{Maroon}{中空的墨西哥卷}”。

对于永久\textcolor{Plum}{相交}的 $\Xint{\mathcolor{gray}{-}}{10}{K}^{\;\!\textcolor{PineGreen}{\pm}}_{\mathcolor{gray}{\omega}}$ 片,\textcolor{NavyBlue}{线二色性 LD} 引起的共享\textcolor{Maroon}{悬崖}可以直接用两个相反的类\textcolor{Plum}{阶跃函数}来描述,这两个函数共享同一个零点。每种\textcolor{PineGreen}{模式}下\textcolor{Maroon}{悬崖断层}的\textcolor{Plum}{高度},由其两端的\textcolor{Plum}{极限值}决定,这些\textcolor{Plum}{极限值}朝着远离\textcolor{PineGreen}{光轴}的两个相反方向延伸。这个\textcolor{Plum}{高度}也几乎完全由 \textcolor{NavyBlue}{LD} 本身决定,对 \textcolor{NavyBlue}{光学活性 OA} 的依赖性很小,类似于 \bref{fig:MMTC-2D_eigensystems}\textbf{a} 中 \textcolor{NavyBlue}{OA} 与 \textcolor{NavyBlue}{圆二色性 CD} 的情况。在 \textcolor{NavyBlue}{OA} 的额外作用下,\textcolor{Maroon}{悬崖}的部分类似于 arctan、sigmoid 或 tanh 函数;而与 \textcolor{NavyBlue}{CD} 竞争的\textcolor{Plum}{镜像}\textcolor{Maroon}{悬崖},却形成一对 \textcolor{NavyBlue}{C2 对称}的 \textcolor{Maroon}{Z 扣}。

相应地,就\textcolor{PineGreen}{特征向量}而言,由\textcolor{PineGreen}{虚}/\textcolor{PineGreen}{实} \textcolor{PineGreen}{L 海岸线}包围的灰色/绿色背景层,也称为“\textcolor{PineGreen}{L 湖}”,勾勒出 $\textcolor{PineGreen}{-}$/$\textcolor{PineGreen}{+}$ \textcolor{PineGreen}{模式}的\textcolor{PineGreen}{本征极化}接近\textcolor{PineGreen}{线偏振}的区域,作为 \textcolor{PineGreen}{L 线}\cite{berryOpticalSingularitiesBirefringent2003}的替代品。当 \textcolor{NavyBlue}{线二色性 LD} $=$ 0 时,两个 $\textcolor{PineGreen}{\pm}$ \textcolor{PineGreen}{模式}的 \textcolor{PineGreen}{C 点}和 \textcolor{PineGreen}{L 岸线}分别完全重叠,$\textcolor{PineGreen}{\pm}$ \textcolor{PineGreen}{L 湖泊}也重合形成深绿色。随着 \textcolor{NavyBlue}{LD} 的增加,\textcolor{PineGreen}{L 海岸线}和具有相反\textcolor{NavyBlue}{手性}的成对 \textcolor{PineGreen}{C 点}(分别属于不同的$\textcolor{PineGreen}{\pm}$\textcolor{PineGreen}{模式})不再重合。\textcolor{PineGreen}{C 点}远离\textcolor{PineGreen}{光轴},\textcolor{PineGreen}{L 海岸线}类似于\textcolor{NavyBlue}{磁流体力学},在与自身 \textcolor{PineGreen}{C 点}运动方向一致的方向上被\textcolor{PineGreen}{光轴}吸引。当\textcolor{NavyBlue}{光学活性 OA} $=$ \textcolor{NavyBlue}{LD}时,两条 \textcolor{PineGreen}{L 形海岸线}在\textcolor{PineGreen}{光轴}处相交,形成\textcolor{Plum}{垂直}于\textcolor{Maroon}{连接 C 点的线}的第一个“\textcolor{Maroon}{8 字形环}”。在这个阶段,成对的\textcolor{PineGreen}{奇异轴}(最终)作为 \textcolor{PineGreen}{Voigt 波}的\textcolor{PineGreen}{退化}\textcolor{PineGreen}{特征向量},在\textcolor{PineGreen}{光轴}处出现。随着 \textcolor{NavyBlue}{LD} 继续超过 \textcolor{NavyBlue}{OA},另一对 \textcolor{PineGreen}{C 点}和\textcolor{PineGreen}{奇点}出现,第二个 \textcolor{Maroon}{8 形环}沿着远离\textcolor{PineGreen}{光轴}的方向出现,平行于\textcolor{Maroon}{连接 C 点的线}。最后,当 \textcolor{NavyBlue}{LD} 达到\textcolor{Plum}{最大值}且 \textcolor{NavyBlue}{OA} $=$ 0 时,\textcolor{Plum}{垂直}于\textcolor{Maroon}{连接 C 点的线}的第一个 \textcolor{Maroon}{8 形圈}消失。同时,$\textcolor{PineGreen}{\pm}$ \textcolor{PineGreen}{L 湖}再次重叠,像\textcolor{Maroon}{楚河汉界}一样分割了剩下的第二个 \textcolor{Maroon}{8 字形环}。

沿着 \bref{fig:MMTC-2D_eigensystems}\textbf{b} 中\textcolor{NavyBlue}{正四面体罗盘}的 “\textcolor{Plum}{反厄米 $!$H}” \textcolor{Plum}{边},当\textcolor{NavyBlue}{线二色性 LD} 从\textcolor{Plum}{顶点}“\textcolor{NavyBlue}{圆二色性 `CD'}” 处的 0 增加到 “\textcolor{NavyBlue}{LD}” 处的\textcolor{Plum}{最大值}(由内侧浅红色箭头表示)时,在过程的前半部分,即 \textcolor{NavyBlue}{LD} $\leq$ \textcolor{NavyBlue}{CD} 时,$\Xint{\mathcolor{gray}{-}}{10}{N}^{\;\!\textcolor{PineGreen}{\pm}}_{\mathcolor{gray}{\omega}}, \Xint{\mathcolor{gray}{-}}{10}{K}^{\;\!\textcolor{PineGreen}{\pm}}_{\mathcolor{gray}{\omega}}$ \textcolor{PineGreen}{特征值片}的行为与在 \bref{fig:MMTC-2D_eigensystems}\textbf{a} 中观察到的相同(\textcolor{NavyBlue}{OA} 沿 \textcolor{Plum}{反对称 `!S' 边}从 0 增加,在非零常数 \textcolor{NavyBlue}{CD} 的始终存在下)。唯一的区别是,被 \textcolor{NavyBlue}{LD} 所切割的 $\Xint{\mathcolor{gray}{-}}{10}{N}^{\;\!\textcolor{PineGreen}{\pm}}_{\mathcolor{gray}{\omega}}, \Xint{\mathcolor{gray}{-}}{10}{K}^{\;\!\textcolor{PineGreen}{\pm}}_{\mathcolor{gray}{\omega}}$ 的左半部分被颠倒了,即在\textcolor{PineGreen}{奇点圈}/\textcolor{PineGreen}{环}内交换了它们各自的值,这象征着 \textcolor{NavyBlue}{LD} $=$ \textcolor{NavyBlue}{CD} 的平衡。在 \textcolor{NavyBlue}{LD} 主导的后半段,两个 $\Xint{\mathcolor{gray}{-}}{10}{K}^{\;\!\textcolor{PineGreen}{\pm}}_{\mathcolor{gray}{\omega}}$ 的两侧、下 $\Xint{\mathcolor{gray}{-}}{10}{N}^{\;\!\textcolor{PineGreen}{+}}_{\mathcolor{gray}{\omega}}$ 片的左侧和上 $\Xint{\mathcolor{gray}{-}}{10}{N}^{\;\!\textcolor{PineGreen}{-}}_{\mathcolor{gray}{\omega}}$ 的右侧被熨烫得\textcolor{Plum}{水平且笔直},而 $\Xint{\mathcolor{gray}{-}}{10}{N}^{\;\!\textcolor{PineGreen}{+}}_{\mathcolor{gray}{\omega}}$ 的右侧和 $\Xint{\mathcolor{gray}{-}}{10}{N}^{\;\!\textcolor{PineGreen}{-}}_{\mathcolor{gray}{\omega}}$ 的左侧被拉直成一个与\textcolor{PineGreen}{锥形折射}角度\textcolor{Plum}{相同的角度}。由 \textcolor{NavyBlue}{LD} 和 \textcolor{NavyBlue}{OA} 共同作用出的 $\Xint{\mathcolor{gray}{-}}{10}{K}^{\;\!\textcolor{PineGreen}{\pm}}_{\mathcolor{gray}{\omega}}$ 中的\textcolor{Maroon}{悬崖}/ \textcolor{Maroon}{Z 扣}的\textcolor{Plum}{绝对高度},在 \textcolor{NavyBlue}{LD} 的变化过程中会发生变化,因此不完全由 \textcolor{NavyBlue}{CD} 决定。这与 \bref{fig:MMTC-2D_eigensystems}\textbf{a} 中 $\Xint{\mathcolor{gray}{-}}{10}{K}^{\;\!\textcolor{PineGreen}{\pm}}_{\mathcolor{gray}{\omega}}$ 的两个\textcolor{Maroon}{头顶}/\textcolor{Maroon}{峰谷}/\textcolor{Maroon}{帽冠}的\textcolor{Plum}{极值}之间的\textcolor{Plum}{距离}(的决定性因素和变化情况)不同。

相应地,该过程中\textcolor{PineGreen}{特征向量}的\textcolor{NavyBlue}{绝热演化}出乎意料地有趣。随着\textcolor{NavyBlue}{线二色性 LD} 从 0 逐渐增加,与恒定\textcolor{NavyBlue}{圆二色性 CD} 背景竞争,\textcolor{Plum}{反厄米 `!H' 边}的前半部分仍然与 \bref{fig:MMTC-2D_eigensystems}\textbf{a} 的 \textcolor{Plum}{反对称 `!S' 边}相似。当 \textcolor{NavyBlue}{LD} $\gtrsim$ \textcolor{NavyBlue}{CD} 时,\textcolor{PineGreen}{奇点环}被 \textcolor{NavyBlue}{LD} 撕裂,剩余的\textcolor{PineGreen}{新月形奇点}沿即将到来的\textcolor{Maroon}{连接 C 点的线}方向排列。之后,当 \textcolor{NavyBlue}{LD} 超越 \textcolor{NavyBlue}{CD} 时,两种 $\textcolor{PineGreen}{\pm}$ \textcolor{PineGreen}{模式}的\textcolor{Plum}{镜像对称} \textcolor{PineGreen}{L 形海岸线}(每一条都具有两条\textcolor{Plum}{中心对称}的\textcolor{Plum}{半 $\symup{\pi}$ 螺旋线}),开始整体膨胀,就像一串\textcolor{Maroon}{装饰着空心水滴的心形吊坠}。成对的 \textcolor{PineGreen}{C 点}向剩余的\textcolor{PineGreen}{奇点}移动,并最终超过它们的原始位置。在整个过程中,仅由 \textcolor{NavyBlue}{CD} 控制的不再完整的\textcolor{PineGreen}{新月形奇点}的两端(位置)固定不变。在 \textcolor{PineGreen}{C 点}到达之前,\textcolor{PineGreen}{环}/\textcolor{PineGreen}{新月}彻底消散,只留下\textcolor{PineGreen}{一对奇点}(而不是\textcolor{PineGreen}{无限奇点}),就像经典\textcolor{PineGreen}{晶体光学}中的普通情况一样。随着 \textcolor{NavyBlue}{LD} 的进一步增加,\textcolor{PineGreen}{奇点}沿与 \textcolor{PineGreen}{C 点}相同的方向移动,但速度较慢,直到它们最终相遇并重合,此时 \textcolor{NavyBlue}{LD} 达到最大值,\textcolor{NavyBlue}{CD} 降至零。在 \textcolor{NavyBlue}{CD} \textcolor{NavyBlue}{衰减}过程中,同一\textcolor{PineGreen}{模式}的\textcolor{PineGreen}{分裂(成 2 个区域的)L 湖}开始从两侧朝着\textcolor{PineGreen}{光轴}靠内移动,最终合并成一条相同颜色(绿色或灰色)的条带。该条带的粗细完全/仅由 \textcolor{NavyBlue}{LD} 控制。在整个过程中,如果 \textcolor{NavyBlue}{LD} 保持不变,\textcolor{PineGreen}{C 点们}的\textcolor{Plum}{绝对位置}也将保持不变(根据\textcolor{Maroon}{萦绕定理}\cite{berryOpticalSingularitiesBirefringent2003}),而\textcolor{PineGreen}{奇点们}将缩回到 \textcolor{PineGreen}{C 点们}的位置。

对于由\textcolor{PineGreen}{特征向量}形成的、最外层的第四个不完整圆圈排列的子图,只是为了推广\textcolor{Maroon}{萦绕定理}而提出它。即,在\textcolor{NavyBlue}{线二色性 LD} 恒定的条件下,\textcolor{Maroon}{萦绕定理}不仅适用于变化的\textcolor{NavyBlue}{光学活性 OA}(\bref{fig:MMTC-2D_eigensystems}\textbf{c}),也适用于变化的 \textcolor{NavyBlue}{圆二色性 CD}(\bref{fig:MMTC-2D_eigensystems}\textbf{b})。更多详细信息请参见\href{https://www.youtube.com/watch?v=xkMWXW29HYc}{视频 2.1}或\href{https://www.youtube.com/watch?v=PPZHhTvdfb0}{视频 2.2}。关于\bref{fig:MMTC-2D_eigensystems} 中最外圈的缺失部分,相应的现象要么太微不足道(\bref{fig:MMTC-2D_eigensystems}\textbf{a}),要么超过了感兴趣\textcolor{gray}{倒空间}视野/区域(\bref{fig:MMTC-2D_eigensystems}\textbf{b})。

\textcolor{NavyBlue}{圆二色性 CD},作为 $\bar{\bar{\varepsilon}}^{\;\! \mathcolor{gray}{\omega}}_{\textcolor{Maroon}{(1)}}$ 张量的\textcolor{Plum}{实部}和\textcolor{Plum}{反对称}部分,以及它对\textcolor{PineGreen}{模式}和光传播的影响,在这次探险中,基于 \textcolor{Maroon}{Born-Landau} \textcolor{Maroon}{本构关系},得到了额外的强调。作为 \textcolor{Maroon}{Drude Boys (Post)} \textcolor{Maroon}{本构关系}的\textcolor{Plum}{不对称}对应物,\textcolor{Maroon}{Born-Landau} \textcolor{Maroon}{本构关系}在很大程度上是适用的,但它们并不严格正确\cite{frantaConstitutiveEquationsDescribing2021,ossikovskiConstitutiveRelationsOptically2021}。

这种局限性表现在两个方面:{\one} \textcolor{NavyBlue}{圆二色性 CD} 在\textcolor{gray}{倒空间}中引入的\textcolor{Plum}{无限数量}的\textcolor{PineGreen}{光学奇点},以及 {\two} 在\textcolor{gray}{正空间}传播过程中观察到的对\textcolor{NavyBlue}{物理实在性}的违反。特别是后者:在 \textcolor{Maroon}{Born-Landau} \textcolor{Maroon}{本构关系}下,\textcolor{NavyBlue}{CD} --- 通常作为\textcolor{Plum}{实}矩阵的\textcolor{Plum}{反对称}部分 --- 通常会产生成对的\textcolor{Plum}{复共轭}\textcolor{PineGreen}{本征值}。这样的线性代数结果,翻译为\textcolor{NavyBlue}{物理语言}即,在沿传播方向的两个\textcolor{PineGreen}{本征模}中,若一个\textcolor{PineGreen}{模式}表现出\textcolor{NavyBlue}{吸收},则另一个表现出\textcolor{NavyBlue}{增益}。在数值模拟中,两个\textcolor{PineGreen}{模式}\textcolor{PineGreen}{线性叠加}出的\textcolor{PineGreen}{总模场}的\textcolor{NavyBlue}{总光强}(在不\textcolor{PineGreen}{检偏}的情况下),在传播过程中几乎总是表现出\textcolor{NavyBlue}{净增益}\Footnote{在\textcolor{Plum}{几乎纯净}的\textcolor{NavyBlue}{圆二色性 CD} 作用背景下。当然,需要增添一定的\textcolor{NavyBlue}{手性 OA} 或 \textcolor{NavyBlue}{线二向色性 LD} 以避免在\textcolor{PineGreen}{零频}等\textcolor{NavyBlue}{光强集中分布}的地方,遇上大规模\textcolor{Plum}{数值奇点},如\bref{fig:MMTC-2D_eigensystems}\textbf{a},\textbf{b}所示。}。这与实验上观察到的结果相矛盾,即 \textcolor{NavyBlue}{CD} 理应类似于 \textcolor{NavyBlue}{线二色性 LD},即在 $\textcolor{PineGreen}{\pm}$ 两种模式\cite{multunasCircularDichroismCrystals2023}下都表现出\textcolor{NavyBlue}{吸收},尽管材料中总是伴随着的各向同性\textcolor{NavyBlue}{损耗},可能会提供一种解释。因此,对\textcolor{PineGreen}{双各向异性}材料中\textcolor{Plum}{对称}\textcolor{Maroon}{本构关系}内 \textcolor{NavyBlue}{CD} 部分的深入而集中的研究,变得迫在眉睫。

实际上,如果考虑 $\textcolor{Plum}{2 \times 2}$ \textcolor{NavyBlue}{哈密顿量} $\bar{\bar{H}}$,\textcolor{NavyBlue}{圆二色性 CD} 的性质在\textcolor{Plum}{数学}上将更容易理解:设\textcolor{Plum}{复非厄米}矩阵 $\bar{\bar{H}} = \bar{\bar{S}} + \bar{\bar{A}}$ 由\textcolor{Plum}{对称}部分 $\bar{\bar{S}} = \left[ \left[ a,b \right],\left[ b,c \right] \right]$ 和\textcolor{Plum}{反对称}部分 $\bar{\bar{A}} = \left[ \left[ 0,d \right],\left[ -d,0 \right] \right]$(其\textcolor{Plum}{实部}即 \textcolor{NavyBlue}{CD})组成,它的两个\textcolor{PineGreen}{特征值}为 $\lambda^{\;\!\textcolor{PineGreen}{\pm}} = (a+c) \big/ 2 \textcolor{PineGreen}{\pm} \sqrt{\Delta} \big/ 2$,其中 $\Delta = \left( a-c \right)^2 + 4 \left( b^2 - d^2 \right)$。从中可以看出,要想保证\textcolor{PineGreen}{电矢量总场}\textcolor{NavyBlue}{吸收},$\lambda^{\;\!\textcolor{PineGreen}{+}}, \lambda^{\;\!\textcolor{PineGreen}{-}}$ 中的那个\textcolor{Plum}{虚部}为正且更大的\textcolor{PineGreen}{本征值},它的\textcolor{Plum}{虚部}的\textcolor{Plum}{绝对值},需要在\textcolor{Plum}{较大面积比例的}(\textcolor{PineGreen}{总光场}\textcolor{Maroon}{傅立叶谱分量} $\Xint{\mathcolor{gray}{-}}{25}{\bar{E}}^{\;\!\mathcolor{gray}{\omega}}_{\;\! \textcolor{Maroon}{\Yup} \mathcolor{gray}{z}}$ 中\textcolor{NavyBlue}{强度}占比较高的)\textcolor{gray}{空间频率} $\mathcolor{gray}{\bar{k}_{\symup{\rho}}}$ 上(比另一个\textcolor{Plum}{虚部}可能为负的\textcolor{PineGreen}{本征值}的\textcolor{Plum}{虚部}的\textcolor{Plum}{绝对值})更大。而实现这一点,只需要简单地令 $\textcolor{Plum}{\text{Tr}} \left[ \bar{\bar{H}} \right]_{\textcolor{Plum}{\text{I}}} = \textcolor{Plum}{\text{Tr}} \left[ \bar{\bar{S}}_{\textcolor{Plum}{\text{I}}} \right] = a_{\textcolor{Plum}{\text{I}}} + c_{\textcolor{Plum}{\text{I}}} > 0$ 即可\cite{berryOpticalSingularitiesBirefringent2003}\Footnote{其中,定义了矩阵的\textcolor{Plum}{迹} $\textcolor{Plum}{\text{Tr}} \left[ \cdot \right]$。注意,\textcolor{PineGreen}{晶体光学}中可能需要对\textcolor{PineGreen}{本征值} $\lambda^{\;\!\textcolor{PineGreen}{+}}, \lambda^{\;\!\textcolor{PineGreen}{-}}$ 取\textcolor{Plum}{倒数},才正比于对应模式 $\textcolor{PineGreen}{+}, \textcolor{PineGreen}{-}$ 的\textcolor{Plum}{复}\textcolor{PineGreen}{折射率}或\textcolor{Plum}{复}\textcolor{PineGreen}{波矢}\cite{mailybaevStrongWeakCoupling2005,berryOpticalSingularitiesBirefringent2003,berryOpticalSingularitiesBianisotropic2005}。以至于结论和实际的应对措施或许相反。}。--- 反过来,当$\textcolor{Plum}{\text{Tr}} \left[ \bar{\bar{H}} \right]_{\textcolor{Plum}{\text{I}}} = \textcolor{Plum}{\text{Tr}} \left[ \bar{\bar{S}}_{\textcolor{Plum}{\text{I}}} \right] = a_{\textcolor{Plum}{\text{I}}} + c_{\textcolor{Plum}{\text{I}}} < 0$ 时,$\lambda^{\;\!\textcolor{PineGreen}{+}}, \lambda^{\;\!\textcolor{PineGreen}{-}}$ 中\textcolor{Plum}{虚部}为负且更小的\textcolor{PineGreen}{本征值}的\textcolor{Plum}{虚部}的\textcolor{Plum}{绝对值}更大。原则上这对于所有的\textcolor{Plum}{自变量} $\mathcolor{gray}{\bar{k}_{\symup{\rho}}}$ 都成立。
%在存在线或/和圆二向色性的情况下,2 个本征值 $\lambda^{\;\!\textcolor{PineGreen}{+}}, \lambda^{\;\!\textcolor{PineGreen}{-}}$ 的虚部同号的情况不多见?

%当\textcolor{Plum}{对称}部分 $\bar{\bar{S}}$ 的\textcolor{Plum}{实部} $\bar{\bar{S}}_{\textcolor{Plum}{\text{R}}}$ 的 2 个主张量元的差异 $a_{\textcolor{Plum}{\text{R}}} - c_{\textcolor{Plum}{\text{R}}}$ 越大,即当\textcolor{Plum}{主轴各向异性}越强、\textcolor{PineGreen}{双折射性}越大时,$\sqrt{\Delta}$ 越靠近\textcolor{Plum}{复平面上正实轴的较远端}。同样的道理,$\bar{\bar{S}}$ 斜对角元的“\textcolor{NavyBlue}{偏振模式间的耦合/混合程度}” $b_{\textcolor{Plum}{\text{R}}}$ 相对于 \textcolor{NavyBlue}{圆二色性 CD} 的强度 $d_{\textcolor{Plum}{\text{R}}}$ 之差 $b_{\textcolor{Plum}{\text{R}}} - d_{\textcolor{Plum}{\text{R}}}$,也有类似的效果(但一般来说 $b_{\textcolor{Plum}{\text{R}}}$ 会因 $\bar{\bar{H}}$ 的\textcolor{Plum}{实对称}部分 $\bar{\bar{S}}_{\textcolor{Plum}{\text{R}}}$ 的对角化而接近 0)。以上便从两个角度分别提供了减弱\textcolor{PineGreen}{晶体光学}中的\textcolor{NavyBlue}{圆二色性 CD} 的不合理性的途径。

然而,即便进行了上述“保证式的”操作,在实际的数值实验中,较为纯净的\textcolor{NavyBlue}{圆二色性 CD} 仍然倾向于引起\textcolor{PineGreen}{矢量光场}\textcolor{NavyBlue}{总光强}随着\textcolor{Plum}{传播距离}\textcolor{Plum}{单调指数上升}、\textcolor{Plum}{先下降后上升},与常见的\textcolor{Plum}{单调指数下降}、\textcolor{Plum}{先上升后下降}背道而驰。这在一定程度上可能归功于\textcolor{PineGreen}{本征偏振态}\textcolor{Plum}{非正交}且\textcolor{NavyBlue}{反向自旋}/\textcolor{NavyBlue}{手性}的缘故。--- 相反地,致使\textcolor{PineGreen}{本征偏振态矩阵} $\overline{\Xint{{}^{}_{\mathcolor{gray}{-}}}{10}{\bar{g}}^{\;\!\mathcolor{gray}{\omega} \textcolor{PineGreen}{\pm}}_{\;\! \textcolor{Maroon}{\symup{\rho}}}}^{\mathsf{\textcolor{Plum}{T}}}, \overline{\Xint{{}^{}_{\mathcolor{gray}{-}}}{10}{\bar{g}}^{\;\!\mathcolor{gray}{\omega} \textcolor{PineGreen}{\pm}}_{\;\! \textcolor{Maroon}{\symup{\rho}}}}^{\textcolor{Plum}{-\mathsf{T}}}$ 同样\textcolor{Plum}{非酉的}\textcolor{NavyBlue}{线二色性 LD},就因倾向引起\textcolor{NavyBlue}{同向自旋}/\textcolor{NavyBlue}{手性}而确保从\textcolor{PineGreen}{本征波} $\Xint{\mathcolor{gray}{-}}{25}{\bar{E}}^{\;\!\mathcolor{gray}{\omega} \textcolor{PineGreen}{\pm}}_{\;\! \mathcolor{gray}{z}}$ 投影至电场矢量 $\Xint{\mathcolor{gray}{-}}{25}{\bar{E}}^{\;\!\mathcolor{gray}{\omega}}_{\;\! \textcolor{Maroon}{\Yup} \mathcolor{gray}{z}}$ 的 x,y,z 三分量 $\Xint{\mathcolor{gray}{-}}{25}{\bar{E}}^{\;\!\mathcolor{gray}{\omega}}_{\;\! \symup{x} \mathcolor{gray}{z}}, \Xint{\mathcolor{gray}{-}}{25}{\bar{E}}^{\;\!\mathcolor{gray}{\omega}}_{\;\! \symup{y} \mathcolor{gray}{z}}, \Xint{\mathcolor{gray}{-}}{25}{\bar{E}}^{\;\!\mathcolor{gray}{\omega}}_{\;\! \symup{z} \mathcolor{gray}{z}}$ 并\textcolor{PineGreen}{干涉}后的模方和 $\lvert \Xint{\mathcolor{gray}{-}}{25}{\bar{E}}^{\;\!\mathcolor{gray}{\omega}}_{\;\! \textcolor{Maroon}{\Yup} \mathcolor{gray}{z}} \rvert^2 = \lvert \Xint{\mathcolor{gray}{-}}{25}{\bar{E}}^{\;\!\mathcolor{gray}{\omega}}_{\;\! \symup{x} \mathcolor{gray}{z}} \rvert^2 + \lvert \Xint{\mathcolor{gray}{-}}{25}{\bar{E}}^{\;\!\mathcolor{gray}{\omega}}_{\;\! \symup{y} \mathcolor{gray}{z}} \rvert^2 + \lvert \Xint{\mathcolor{gray}{-}}{25}{\bar{E}}^{\;\!\mathcolor{gray}{\omega}}_{\;\! \symup{z} \mathcolor{gray}{z}} \rvert^2$ 的总\textcolor{NavyBlue}{吸收}。

\clearpage

\vspace*{-7.5em}

\marginLeft[-2.4em]{ssec:3D-real-propagation}\subsection{任意 \texorpdfstring{$\bar{\bar{\varepsilon}}$}{$\bar{\bar{\text{ε}}}$} 材料内,复矢量光电场在 3D 正空间中的传播}\label{ssec:3D-real-propagation}

通过在 \bref{fig:Peet-3D_propagation} 中详细描述从远离\textcolor{PineGreen}{光轴}的\textcolor{PineGreen}{双折射(DR)}现象,到靠近和沿着\textcolor{PineGreen}{光轴}方向的\textcolor{PineGreen}{锥折射(CR)}的\textcolor{NavyBlue}{绝热演化},我们强调\textcolor{Maroon}{晶体-2f 配置}在三维\textcolor{gray}{正} $\mathcolor{gray}{\bar{r}}$ \textcolor{gray}{空间}的数值实验中的优越性。这张图中的绝大多数子图(除了 \bref{fig:Peet-3D_propagation}\textbf{d2},\textbf{f1-3},\textbf{g1-3}),\textcolor{PineGreen}{泵浦-相机 CCD} 均设置在 \textcolor{Maroon}{R-L(左右圆)起偏器-检偏器的配置}下,以分析晶体内部的\textcolor{Plum}{横向} \textcolor{NavyBlue}{R$\to$L 自旋轨道相互作用(SOI)}。
\bref{fig:Peet-3D_propagation} 的整体可以被视为 \textcolor{Maroon}{Peet} 于 2014 年的实验工作\cite{peetExperimentalStudyInternal2014}在 2 个\textcolor{Plum}{维度}/\textcolor{Plum}{自由度}上的“\textcolor{Plum}{解析延拓}”:\textcolor{Plum}{传播距离} $\mathcolor{gray}{z}$ 和离\textcolor{PineGreen}{光轴}角 $\theta$。
\begin{figure}[htbp!]
	\centering
	\includegraphics[width=1.0\textwidth]{D:/C2D/Desktop/article_fig/phd_thesis_fig/chapter-03/fig_3.Peet_3D_正空间传播.pdf}
	\biackcaption[\textbf{The adiabatic transition from double refraction} (\textbf{a} and the leftmost column of \textbf{e}) \textbf{to conical refraction} (\textbf{c} and the rightmost column of \textbf{e}) \textbf{as the pump deviates/approaches the diabolical point, focusing on the focal-plane} (top row of \textbf{e}) \textbf{and far-field} (bottom row of \textbf{e}) \textbf{evolution, as well as the full-field propagation of double refraction} (\textbf{a}) \textbf{and conical refraction} (\textbf{b}, \textbf{d}, and \textbf{c}), \textbf{all of which are investigated by a standard crystal-2f system.}]{-0.7em}{\textbf{当\textcolor{NavyBlue}{泵浦}偏离/接近\textcolor{PineGreen}{恶魔点}时},\textbf{从\textcolor{PineGreen}{双折射}}(\textbf{a} 和 \textbf{e} 的最左列)\textbf{到\textcolor{PineGreen}{锥折射}}(\textbf{c} 和 \textbf{e} 的最右列)过程中,\textbf{\textcolor{Plum}{焦平面}}(\textbf {e} 的最上行)\textbf{和\textcolor{Plum}{远场}}(\textbf{e} 的最下行)\textbf{的\textcolor{NavyBlue}{绝热演化},以及\textcolor{PineGreen}{双折射}}(\textbf{a}) \textbf{和\textcolor{PineGreen}{锥折射}}(\textbf{b}、\textbf{d} 和 \textbf{c})\textbf{的全场传播。}\textbf{所有这些研究都通过标准\textcolor{Maroon}{晶体-2f 系统}进行}\\}{fig:Peet-3D_propagation}
\end{figure}

\bref{fig:Peet-3D_propagation}\textbf{a},\textbf{c} 描述了在\textcolor{NavyBlue}{右旋圆极化}(从\textcolor{NavyBlue}{波源}看向传播方向顺时针 ⥁)的 $\textcolor{Maroon}{\text{LG}}^{p=0}_{l=2}$ \textcolor{NavyBlue}{泵浦}下,\textcolor{gray}{正空间}中的\textcolor{PineGreen}{矢量光场} $\bar{E}^{\;\!\mathcolor{gray}{\omega}}_{\;\! \textcolor{Maroon}{\Yup} \mathcolor{gray}{z}}$ 的\textcolor{NavyBlue}{左旋圆偏振}(↺)分量在 KTP 晶体内、外的二维剖面切片演化。具体来说,分别对应于离\textcolor{PineGreen}{光轴} $20^\circ$泵浦下的\textcolor{PineGreen}{双折射}(\textbf{a})、离\textcolor{PineGreen}{光轴} $0^\circ$泵浦下的\textcolor{PineGreen}{锥形衍射}(\textbf{c})过程。

\bref{fig:Peet-3D_propagation}\textbf{b} 是 \textbf{c} 的三维版本,其中\textcolor{NavyBlue}{泵浦}被\textcolor{Maroon}{高斯光束}取代,且\textcolor{Plum}{焦平面}从透镜处被移动至位于晶体后表面和透镜之间。\bref{fig:Peet-3D_propagation}\textbf{d1} 给出了当\textcolor{NavyBlue}{泵浦} $\textcolor{Maroon}{\text{LG}}^{p=0}_{l=2}$ 分别被对齐在离\textcolor{PineGreen}{光轴} $0^\circ$ 和 $5^\circ$ 处时,KTP 晶体内部\textcolor{PineGreen}{光电场}的\textcolor{Plum}{横向} R$\to$L \textcolor{NavyBlue}{旋轨耦合}效率演化。\bref{fig:Peet-3D_propagation}\textbf{d2} 绘制了在经典\textcolor{PineGreen}{锥折射}背景下的二维\textcolor{PineGreen}{无检偏器}版本的 \textbf{b}, 其中\textcolor{NavyBlue}{泵浦}损失了其\textcolor{NavyBlue}{自旋角动量} $\sigma_{\textcolor{Maroon}{\text{P}}}$ 的一半,这些光场的\textcolor{NavyBlue}{角动量}被转移到物质性的晶体上,对其持续施加\textcolor{NavyBlue}{额外扭矩}\cite{berryOrbitalSpinAngular2005}。\bref{fig:Peet-3D_propagation}\textbf{e} 是基于 \textcolor{Maroon}{Peet} 实验结果的扩展研究,其中 \textbf{f1}-\textbf{f4} 和 \textbf{g1}-\textbf{g4} 对应于 \textcolor{Maroon}{Peet} 的 Figures 7(a-h)\cite{peetExperimentalStudyInternal2014}, 在其实验的\textcolor{NavyBlue}{泵浦}-\textcolor{PineGreen}{光轴}夹角附近,从 $0^\circ$ \textcolor{Plum}{水平扫描}至 $20^\circ$,而\textcolor{Plum}{传播距离}则从 $\mathcolor{gray}{z}=\mathcolor{gray}{f}$ 的\textcolor{Plum}{焦平面}\textcolor{Plum}{垂直扫描}到 $\mathcolor{gray}{z}=2\mathcolor{gray}{f}$ 的\textcolor{Plum}{傅里叶面};在此期间,\textcolor{PineGreen}{离轴}的\textcolor{PineGreen}{拉曼尖点}出现在 $(\mathcolor{gray}{f},2\mathcolor{gray}{f})$ 之间的某个位置。\textbf{e} 的四条边(端到端)相互连接,在另一对参数$\theta,\mathcolor{gray}{z}$下在三维\textcolor{gray}{正空间} $\mathcolor{gray}{\bar{r}}$ 中形成一条(\textcolor{PineGreen}{起-检偏}后的)$\bar{E}^{\;\!\mathcolor{gray}{\omega}}_{\;\! \textcolor{Maroon}{\Yup} \mathcolor{gray}{z}}$ 的\textcolor{NavyBlue}{绝热演化}带。--- 类似但不同于 \bref{ssec:2D-reciprocal-eigensystems} 在二维\textcolor{gray}{倒空间} $\mathcolor{gray}{\bar{k}_{\symup{\rho}}}$ 中扫描 3 个参数\textcolor{NavyBlue}{光学活性 OA}、\textcolor{NavyBlue}{线二色性 LD}、\textcolor{NavyBlue}{线二色性 CD} 所形成的 $\Xint{{}^{}_{\mathcolor{gray}{-}}}{10}{\bar{g}}^{\;\!\mathcolor{gray}{\omega} \textcolor{PineGreen}{\pm}}_{\;\! \textcolor{Maroon}{\Yup}}$ 的\textcolor{NavyBlue}{绝热演化}带。

在通过 \bref{fig:Peet-3D_propagation} 完成\textcolor{PineGreen}{光轴}(即典型\textcolor{Plum}{双轴}材料的\textcolor{PineGreen}{恶魔点},在该方向,\textcolor{PineGreen}{特征值}的简并不延伸到\textcolor{PineGreen}{特征向量})周围的定量数值实验验证后,现在从 3D \textcolor{gray}{正空间} $\mathcolor{gray}{\bar{r}}$(而不是之前的二 2D \textcolor{gray}{倒空间} $\mathcolor{gray}{\bar{k}_{\symup{\rho}}}$)的视角下重新审视\textcolor{NavyBlue}{正四面体罗盘}。沿着它的\textcolor{Plum}{三条边},即\textcolor{Plum}{厄米 `H'}、\textcolor{Plum}{对称 `S'} 和\textcolor{Plum}{虚部 `Im'},本章广泛地再现了 \textcolor{Maroon}{Pancharatnam}\cite{pancharatnamPropagationLightAbsorbing1955}, \textcolor{Maroon}{Bloembergen}\cite{schellLaserStudiesInternal1978b}, \textcolor{Maroon}{Brenier}\cite{brenierVoigtWaveInvestigation2015,brenierLasingConicalDiffraction2016,brenierChiralityDichroismCompetition2017}等人\cite{ballantineConicalDiffractionDispersion2014,zhangNonparaxialIdealizedPolarizer2018,berryOrbitalSpinAngular2005,tangHarmonicSpinOrbit2020,schellLaserStudiesInternal1978b,pancharatnamPropagationLightAbsorbing1955,brenierVoigtWaveInvestigation2015,brenierLasingConicalDiffraction2016,brenierChiralityDichroismCompetition2017}在 3D \textcolor{gray}{正空间} $\mathcolor{gray}{\bar{r}}$ 中的实验结果,重点关注物质的三种光学性质,即\textcolor{NavyBlue}{双折射性 Bi}、\textcolor{NavyBlue}{光学活性 OA} 和\textcolor{NavyBlue}{线二色性 LD}。

沿 \bref{fig:MMTC-3D_propagation} 中的\textcolor{NavyBlue}{正四面体罗盘}的\textcolor{Plum}{三条边}进行\textcolor{Plum}{逆时针}扫描。首先,沿\textcolor{NavyBlue}{正四面体罗盘}的\textcolor{Plum}{厄米 `H' 边}将\textcolor{NavyBlue}{光学活性 OA} 从 0 增加,得到旋光场从\textcolor{Plum}{顶点}\textcolor{NavyBlue}{双折射性 `Bi'} 到\textcolor{NavyBlue}{光学活性 `OA'} 的\textcolor{NavyBlue}{绝热演化}图(由淡蓝色箭头指向),最终收敛到 \textcolor{Maroon}{Bloembergen} 采用 $\alpha$-HIO$_3$ 晶体进行的实验:\textcolor{NavyBlue}{手性}\textcolor{PineGreen}{内圆锥衍射}\cite{schellLaserStudiesInternal1978b}。

接下来,在保持\textcolor{NavyBlue}{光学活性 OA} 的同时,随着\textcolor{NavyBlue}{线二色性 LD} 从 0 沿\textcolor{Plum}{虚部 `Im' 边}增加(由浅绿色箭头表示),获得了 \textcolor{Maroon}{Brenier} 关于掺杂 Nd$^{3+}$ 的\textcolor{Plum}{非中心对称}\textcolor{NavyBlue}{手性} BZBO 晶体的\textcolor{NavyBlue}{手性}与\textcolor{NavyBlue}{线二向色性}竞争的实验结果,其中 \textcolor{NavyBlue}{OA} $=$ \textcolor{NavyBlue}{LD} 有一个明确的标准:当\textcolor{NavyBlue}{泵浦}的偏振被对齐至与 \textcolor{Maroon}{Voigt} 波的偏振平行,且与\textcolor{PineGreen}{检偏器}/\textcolor{PineGreen}{分析器}的透光方向\textcolor{Plum}{垂直}时,在\textcolor{Plum}{远场}观察到\textcolor{PineGreen}{中心消光}。

随后,取消\textcolor{NavyBlue}{光学活性 OA}, 仅保留\textcolor{NavyBlue}{线二色性 LD}, 得到了激光晶体 KGd(WO$_4$)$_2$ 的角向吸收光谱分布\cite{brenierVoigtWaveInvestigation2015,brenierLasingConicalDiffraction2016}。接着,沿着\textcolor{Plum}{对称 `S' 边}逐渐减少 \textcolor{NavyBlue}{LD}(与浅红色箭头的方向相反),\textcolor{Maroon}{Pancharatnam} 现象逐渐出现。
\begin{figure}[htbp!]
	\centering
	\includegraphics[width=1.0\textwidth]{D:/C2D/Desktop/article_fig/phd_thesis_fig/chapter-03/fig_4.正四面体_3D_正空间传播.pdf}
	\biackcaption[\textbf{Scan Bi, OA and LD along 3 edges (`H', `S', and `Im') of the M-M TC to see the adiabatic evolution of optical fields in the 3D real $\boldsymbol{\bar{r}}$ space.} \textbf{a} Increase OA to see chiral conical refraction evolution at the focal plane, which ultimately results in (\textbf{c1}-\textbf{c5}) and (\textbf{d1}-\textbf{d5}), matching Bloembergen's experiment\cite{schellLaserStudiesInternal1978b} in his FIG. 5A-9A and 5B-9B respectively. \textbf{b} Increase LD to see Pancharatnam phenomenon\cite{pancharatnamPropagationLightAbsorbing1955} and Brenier's anisotropic absorbing spectrum\cite{brenierVoigtWaveInvestigation2015,brenierLasingConicalDiffraction2016} (in his Fig. 6b). \textbf{e2} By Increasing LD while keeping OA constant, the competition between LD and OA is examined, providing extinction in the far field as the experimental criterion\cite{brenierChiralityDichroismCompetition2017} for OA $=$ LD under the setup where the pump's polarization $\parallel$ eigenvectors $\perp$ analyzer. \textbf{e1} The corresponding evolution of $45^\circ$ $\to$ $135^\circ$ linear polarization SOI efficiency when OA $=$ LD.]{-0.7em}{\textbf{沿\textcolor{NavyBlue}{材料-矩阵正四面体罗盘}的 3 条\textcolor{Plum}{边}(\textcolor{Plum}{厄米 `H'},\textcolor{Plum}{虚部 `Im'} 和\textcolor{Plum}{对称 `S'})扫描\textcolor{NavyBlue}{双折射性 Bi}、\textcolor{NavyBlue}{光学活性 OA}和\textcolor{NavyBlue}{线二色性 LD},以查看三维\textcolor{gray}{正} $\mathcolor{gray}{\bar{r}}$ \textcolor{gray}{空间}中\textcolor{PineGreen}{光场分布}的\textcolor{NavyBlue}{绝热演化}。} \textbf{a} 增加 \textcolor{NavyBlue}{OA} 以观察\textcolor{NavyBlue}{手性}\textcolor{PineGreen}{内锥折射}在\textcolor{Plum}{焦平面}的演化,最终导致仿真图(\textbf{c1}-\textbf{c5})和(\textbf{d1}-\textbf{d5}),它们分别与 \textcolor{Maroon}{Bloembergen} 的实验 FIG. 5A-9A 和 5B-9B 一一对应地匹配\cite{schellLaserStudiesInternal1978b}。\textbf{b} 增加 \textcolor{NavyBlue}{LD} 以观察 \textcolor{Maroon}{Pancharatnam} 现象\cite{pancharatnamPropagationLightAbsorbing1955}和 \textcolor{Maroon}{Brenier} 的\textcolor{Plum}{各向异性}吸收光谱\cite{brenierVoigtWaveInvestigation2015,brenierLasingConicalDiffraction2016}(在他的 Fig. 6b 中)。\textbf{e2} 通过在保持 \textcolor{NavyBlue}{OA} 恒定的同时增加 \textcolor{NavyBlue}{LD},研究了 \textcolor{NavyBlue}{LD} 和 \textcolor{NavyBlue}{OA} 之间的竞争。在\textcolor{NavyBlue}{泵浦}的\textcolor{PineGreen}{极化} $\parallel$ \textcolor{PineGreen}{简并}\textcolor{PineGreen}{特征向量} $\perp$ \textcolor{PineGreen}{检偏器}的设置下,将远场消光作为 \textcolor{NavyBlue}{OA} $=$ \textcolor{NavyBlue}{LD} 的实验标准 \cite{brenierChiralityDichroismCompetition2017}。\textbf{e1} 当 \textcolor{NavyBlue}{OA} $=$ \textcolor{NavyBlue}{LD} 时,$45^\circ$ $\to$ $135^\circ$ \textcolor{PineGreen}{线偏振} 的\textcolor{NavyBlue}{旋轨耦合}效率的相应演变\\}{fig:MMTC-3D_propagation}
\end{figure}
最后,当\textcolor{NavyBlue}{线二色性 LD} 减小至 0 时,只剩下\textcolor{NavyBlue}{双折射性 `Bi'},包括\textcolor{Plum}{单轴}性(\bref{fig:hyperbolic}\textbf{b} 和\bref{fig:SOI_energy})、\textcolor{Plum}{双轴}性(\bref{fig:hyperbolic}\textbf{a} 和 \bref{fig:SOI_energy}\textbf{b1})以及它们的\textcolor{Plum}{双曲}对应物(\bref{fig:hyperbolic}\textbf{a},\textbf{b})。

对于 \bref{fig:hyperbolic}\textbf{a},\textbf{b} 和 \bref{fig:high_N.A.}\textbf{d} 中的\textcolor{NavyBlue}{介电型}\textcolor{Plum}{双曲材料},\bref{fig:hyperbolic}\textbf{b} 中的\textcolor{PineGreen}{双折射(DR)}、\bref{fig:hyperbolic}\textbf{a} 中的\textcolor{PineGreen}{圆锥折射(CR)}和 \bref{fig:high_N.A.}\textbf{d} 的\textcolor{PineGreen}{双圆锥折射(DCR)}等典型现象变得不平凡,所有这些都表明,与\textcolor{NavyBlue}{椭球材料}相比,\textcolor{PineGreen}{能流}/\textcolor{PineGreen}{群速度}方向经历了进一步的调制。这导致了诸如\textcolor{PineGreen}{非常 e 光}的\textcolor{Maroon}{负折射}和\textcolor{PineGreen}{普通 o 光}在 \bref{fig:hyperbolic}\textbf{b} 中的\textcolor{Maroon}{正折射}、在 \bref{fig:hyperbolic}\textbf{a} 的两个光轴中间的\textcolor{Maroon}{十字交叉星衍射},以及在 \bref{fig:hyperbolic}\textbf{a} 的\textcolor{Plum}{非圆对称}(在\textcolor{Plum}{焦平面}处)甚至\textcolor{Plum}{非 C2 对称}(远离\textcolor{Plum}{焦平面})的\textcolor{PineGreen}{锥形衍射}图案\cite{ballantineConicalDiffractionDispersion2014}之类的效应。

\begin{figure}[htbp!]
	\centering
	\includegraphics[width=1.0\textwidth]{D:/C2D/Desktop/article_fig/phd_thesis_fig/chapter-03/fig_4.1.双曲.pdf}
	\biackcaption[\textbf{The refractive index surfaces of hyperbolic materials and the corresponding energy flux profiles they modulate.} \textbf{a} Conical refraction and cross beam propagating in hyperbolic biaxial materials\cite{ballantineConicalDiffractionDispersion2014}. \textbf{b} Negative refraction in hyperbolic uniaxial materials, and the iterative convergence of the highly anisotropic refractive index hyperboloid/ellipsoid.]{-0.7em}{\textbf{\textcolor{Plum}{双曲材料}的\textcolor{PineGreen}{折射率}\textcolor{Plum}{表面}及在其调制下的相应\textcolor{NavyBlue}{光强分布}。} \textbf{a} \textcolor{PineGreen}{内锥衍射}和\textcolor{PineGreen}{十字星光束}在\textcolor{Plum}{双曲双轴材料}中的传播\cite{ballantineConicalDiffractionDispersion2014}. \textbf{b} \textcolor{Plum}{双曲单轴材料}中的\textcolor{Maroon}{负折射}和\textcolor{Plum}{高度各向异性}的\textcolor{PineGreen}{折射率}\textcolor{Plum}{双曲面}/\textcolor{Plum}{椭球体}的迭代收敛\\}{fig:hyperbolic}
\end{figure}

在 \bref{fig:SOI_energy} 中使用\textcolor{Plum}{单轴材料},给出了 \bref{fig:SOI_energy}\textbf{c} 中 H$\to$V \textcolor{NavyBlue}{旋轨耦合} 和 \bref{fig:SOI_energy}\textbf{b2} 中 R$\to$L \textcolor{NavyBlue}{旋轨耦合}的典型(\textcolor{Plum}{单调})和非典型(\textcolor{Plum}{振荡})转换效率的 1D 连续演化。相应地,我们提供了离散\textcolor{Plum}{传播距离} $\{\mathcolor{gray}{z_i}\}$ 处的\textcolor{Plum}{横向}\textcolor{PineGreen}{光场分布}的切片序列。可以观察到,除了由\textcolor{Plum}{非零入射角}(\textcolor{PineGreen}{离轴})引起的高频振荡及其包络(\bref{fig:Peet-3D_propagation}\textbf{d1} 中的蓝线)外,\textcolor{NavyBlue}{泵浦}的\textcolor{NavyBlue}{非零轨道角动量(OAM)}(\bref{fig:Peet-3D_propagation}\textbf{d1} 中红线,以及 \bref{fig:SOI_energy}\textbf{b2})、\textcolor{Maroon}{倒易空间}中的\textcolor{PineGreen}{非傍轴性}和\textcolor{gray}{正}/\textcolor{gray}{实空间}中的束腰尺寸(\bref{fig:SOI_energy}\textbf{c})等因素,也会影响 H$\to$V 和 R$\to$L \textcolor{NavyBlue}{旋轨耦合}效率的曲线形状。

有趣的是,与典型的\textcolor{Plum}{稳态值} 25\% 相比,在\textcolor{Plum}{泵浦参数}和\textcolor{Plum}{传播距离}的特定组合下,\textcolor{PineGreen}{线偏}\textcolor{Maroon}{贝塞尔光束}的 H$\to$V 转换效率的\textcolor{Plum}{瞬时值},最高可以高达 50\%\cite{belyiPropagationHighorderCircularly2011,belyiSpintoorbitalAngularMomentum2013,khoninaComparativeInvestigationNonparaxial2015}(\bref{fig:SOI_energy}\textbf{c3})。此外,在 \bref{fig:SOI_energy}\textbf{b1} 中,我们从\textcolor{Maroon}{光与物质相互作用}中的\textcolor{NavyBlue}{总角动量}\textcolor{Maroon}{守恒} $\Delta J = 0$ 的角度出发,给出了\textcolor{Plum}{双轴}($l_{\textcolor{Maroon}{\perp}}=J_{\textcolor{Maroon}{\text{P}}}=l_{\textcolor{Maroon}{\text{P}}}+\sigma_{\textcolor{Maroon}{\text{P}}}$)\cite{berryOrbitalSpinAngular2005}和\textcolor{Plum}{单轴}材料($l_{\textcolor{Maroon}{\perp}}=l_{\textcolor{Maroon}{\text{P}}}+2\sigma_{\textcolor{Maroon}{\text{P}}}$)\cite{ciattoniCircularlyPolarizedBeams2003,ciattoniAngularMomentumDynamics2003,tangHarmonicSpinOrbit2020}中\textcolor{Plum}{横向}\textcolor{NavyBlue}{旋轨耦合}\textcolor{Plum}{输出}\textcolor{NavyBlue}{拓扑荷} $l_{\textcolor{Maroon}{\perp}}$ 表达式的简单推导。这允许人们\textcolor{Plum}{不进行具体细节计算},就直接模糊预测\textcolor{Plum}{输出}所有\textcolor{NavyBlue}{旋轨态}和其\textcolor{NavyBlue}{光强比例分配}。

此外,从\textcolor{Maroon}{散度方程}出发,我们在 \bref{fig:SOI_energy}\textbf{d1} 中还推导出了一个公式,用于基于\textcolor{NavyBlue}{泵浦}的\textcolor{Plum}{横向}\textcolor{NavyBlue}{自旋轨道状态} $\sigma_{\textcolor{Maroon}{\text{P}}},l_{\textcolor{Maroon}{\text{P}}}$ 预测 z 分量电场 $\Xint{\mathcolor{gray}{-}}{25}{E}^{\;\!\mathcolor{gray}{\omega}}_{\;\! \symup{z} \mathcolor{gray}{z}}$ 的\textcolor{NavyBlue}{拓扑电荷} $l_{\textcolor{Maroon}{\text{z}}}$。\textcolor{Plum}{横向}和\textcolor{Plum}{纵向}\textcolor{NavyBlue}{旋轨耦合}的相应说明性示例,分别由 \bref{fig:SOI_energy}\textbf{b2},\textbf{d2} 提供。

\begin{figure}[htbp!]
	\centering
	\includegraphics[width=1.0\textwidth]{D:/C2D/Desktop/article_fig/phd_thesis_fig/chapter-03/fig_4.2.旋轨耦合_energy.pdf}
	\biackcaption[\textbf{Transverse (H-V, R-L) and longitudinal ($\boldsymbol{\uprho}$-z) SOI in uniaxial and bianxial materials.} H$\to$V SOI and its efficiency evolution in (\textbf{a}) non-ideal analyzers\cite{zhangNonparaxialIdealizedPolarizer2018} and (\textbf{c}) uniaxial crystals pumped by bessel beams with different waist (in real space) and cone angle (in reciprocal space). \textbf{b1} The conservation of total angular momentum $\Delta J = 0$ of light and matter leads to the explicit formula of transverse SOI in biaxial\cite{berryOrbitalSpinAngular2005} and uniaxial crystals\cite{ciattoniCircularlyPolarizedBeams2003,ciattoniAngularMomentumDynamics2003,tangHarmonicSpinOrbit2020}; \textbf{d1} the divergence equation yields the explicit expression for the SOI in the z-direction. Illustrative explanations for both (\textbf{b2}) transverse and (\textbf{d2}) longitudinal SOI are also provided.]{-0.7em}{\textbf{\textcolor{Plum}{单轴}和\textcolor{Plum}{双轴材料}中的\textcolor{Plum}{横向}(H-V,R-L)和\textcolor{Plum}{纵向}($\boldsymbol{\uprho}$-z)\textcolor{NavyBlue}{旋轨耦合}。} H$\to$V \textcolor{NavyBlue}{旋轨耦合}和其在非理想/实际检偏器\cite{zhangNonparaxialIdealizedPolarizer2018}(\textbf{a})中,以及(由不同\textcolor{gray}{正空间}束腰、\textcolor{gray}{倒空间}锥角的\textcolor{Maroon}{贝塞尔光束}\textcolor{NavyBlue}{泵浦}的)\textcolor{Plum}{单轴}晶体(\textbf{c})中的转换效率的 1D 演化。\textbf{b1} 光和物质的\textcolor{Maroon}{总角动量守恒} $\Delta J = 0$ 导致\textcolor{Plum}{双轴}\cite{berryOrbitalSpinAngular2005} 和\textcolor{Plum}{单轴}晶体\cite{tangHarmonicSpinOrbit2020} 中\textcolor{Plum}{横向}\textcolor{NavyBlue}{旋轨耦合}的\textcolor{Plum}{显式公式};\textbf{d1} \textcolor{Maroon}{散度方程}给出了 z 方向上\textcolor{NavyBlue}{旋轨耦合}的\textcolor{Plum}{显式表达式}。还提供了(\textbf{b2})\textcolor{Plum}{横向}和(\textbf{d2})\textcolor{Plum}{纵向}\textcolor{NavyBlue}{旋轨耦合}的说明性解释\\}{fig:SOI_energy}
\end{figure}

作为 $\textcolor{Plum}{2 \times 2}$ \textcolor{PineGreen}{本征极化矩阵} $\overline{\Xint{{}^{}_{\mathcolor{gray}{-}}}{10}{\bar{g}}^{\;\!\mathcolor{gray}{\omega} \textcolor{PineGreen}{\pm}}_{\;\! \textcolor{Maroon}{\symup{\rho}}}}^{\mathsf{\textcolor{Plum}{T}}}, \overline{\Xint{{}^{}_{\mathcolor{gray}{-}}}{10}{\bar{g}}^{\;\!\mathcolor{gray}{\omega} \textcolor{PineGreen}{\pm}}_{\;\! \textcolor{Maroon}{\symup{\rho}}}}^{\textcolor{Plum}{-\mathsf{T}}}$ 的\textcolor{Plum}{非对角元素}引起的效应,和\textcolor{Maroon}{守恒定律}的体现,\textcolor{NavyBlue}{旋轨耦合}的\textcolor{Plum}{全局把握}和\textcolor{Plum}{详细计算},不仅在\textcolor{Plum}{线性}\textcolor{PineGreen}{晶体光学}中起着重要作用,而且在\textcolor{Plum}{非线性}\textcolor{PineGreen}{晶体光学}\cite{tangHarmonicSpinOrbit2020}中也起着重要的作用。

当材料表现出\textcolor{NavyBlue}{吸收}或\textcolor{NavyBlue}{增益}(显示\textcolor{NavyBlue}{线}和/或\textcolor{NavyBlue}{圆二色性})时,来自 \bref{eq:varepsilon_rho} 的 $\bar{\bar{\varepsilon}}^{\;\! \mathcolor{gray}{\omega}}_{\textcolor{Maroon}{(1)}}$ 和来自 \bref{eq:simplify7-L2-zeta} 的相应“\textcolor{Plum}{特征矩阵}”即\textcolor{Plum}{线性}算子 $\Xint{\mathcolor{gray}{-}}{30}{\bar{\bar{L}}}^{\;\! \mathcolor{gray}{\omega} \textcolor{PineGreen}{\pm}}$ 都是\textcolor{Plum}{实}/\textcolor{Plum}{复非厄米}矩阵。--- 相应地,两个\textcolor{PineGreen}{特征值} $\Xint{\begin{smallmatrix} ~ \\ {}^{}_{\mathcolor{gray}{-}} \\ ~ \end{smallmatrix}}{15}{k}_{\symup{z} }^{\;\! \textcolor{PineGreen}{\pm} \mathcolor{gray}{\omega}}$ 是\textcolor{Plum}{复数}(具有\textcolor{Plum}{非零虚部}),相应的\textcolor{PineGreen}{特征向量} $\Xint{{}^{}_{\mathcolor{gray}{-}}}{10}{\bar{g}}^{\;\!\mathcolor{gray}{\omega} \textcolor{PineGreen}{\pm}}_{\;\! \textcolor{Maroon}{\Yup}}$ 是\textcolor{Plum}{复非正交的}。对应地,由这些成对的 3 分量\textcolor{PineGreen}{特征向量} $\Xint{{}^{}_{\mathcolor{gray}{-}}}{10}{\bar{g}}^{\;\!\mathcolor{gray}{\omega} \textcolor{PineGreen}{\pm}}_{\;\! \textcolor{Maroon}{\Yup}}$ 形成的 $\textcolor{Plum}{3 \times 2}$ \textcolor{PineGreen}{本征偏振矩阵} $\overline{\Xint{{}^{}_{\mathcolor{gray}{-}}}{10}{\bar{g}}^{\;\!\mathcolor{gray}{\omega} \textcolor{PineGreen}{\pm}}_{\;\! \textcolor{Maroon}{\Yup}}}^{\mathsf{\textcolor{Plum}{T}}}, \overline{\Xint{{}^{}_{\mathcolor{gray}{-}}}{10}{\bar{g}}^{\;\!\mathcolor{gray}{\omega} \textcolor{PineGreen}{\pm}}_{\;\! \textcolor{Maroon}{\Yup}}}^{\textcolor{Plum}{-\mathsf{T}}}$(的\textcolor{Plum}{横向} $\textcolor{Plum}{2 \times 2}$ 部分 $\overline{\Xint{{}^{}_{\mathcolor{gray}{-}}}{10}{\bar{g}}^{\;\!\mathcolor{gray}{\omega} \textcolor{PineGreen}{\pm}}_{\;\! \textcolor{Maroon}{\symup{\rho}}}}^{\mathsf{\textcolor{Plum}{T}}}, \overline{\Xint{{}^{}_{\mathcolor{gray}{-}}}{10}{\bar{g}}^{\;\!\mathcolor{gray}{\omega} \textcolor{PineGreen}{\pm}}_{\;\! \textcolor{Maroon}{\symup{\rho}}}}^{\textcolor{Plum}{-\mathsf{T}}}$)是\textcolor{Plum}{非酉的}。

在工业界和学术界的共同前沿 --- \textcolor{Maroon}{激光加工}、\textcolor{Maroon}{像差校正}和\textcolor{Maroon}{逆聚焦工程}领域中,高度\textcolor{Plum}{各向异性}材料和\textcolor{PineGreen}{紧密聚焦}的光场之间强烈相互作用的极端情况下,我们的模型提供了一个统一的解决方案,如 \bref{fig:high_N.A.} 所示。

\clearpage

\begin{figure}[htbp!]
	\centering
	\includegraphics[width=1.0\textwidth]{D:/C2D/Desktop/article_fig/phd_thesis_fig/chapter-03/fig_5.紧聚焦.pdf}
	\backcaption{-0.7em}{\textbf{高\textcolor{Plum}{数值孔径 N.A.}\textcolor{NavyBlue}{泵浦}下\textcolor{Plum}{各向异性}材料内的\textcolor{Plum}{正向}和\textcolor{Plum}{反向传播}。} \textbf{a} 重建 \textcolor{Maroon}{Zusin} 等人的\textcolor{Maroon}{贝塞尔}焦散线实验数据\cite{zusinBesselBeamTransformation2010}。 \textbf{b} \textcolor{PineGreen}{锥折射}的\textcolor{PineGreen}{拉曼尖峰}(\textcolor{PineGreen}{Raman spike})仅起源于\textcolor{PineGreen}{慢模}。选择 \textcolor{PineGreen}{V 方向线偏}的离\textcolor{PineGreen}{光轴} \textcolor{Maroon}{\text{LG}}$_1$ 来\textcolor{NavyBlue}{泵浦} $\alpha$-HIO$_3$。\textbf{c1} 紧\textcolor{PineGreen}{聚焦}焦场的逆向设计:叠加的 $\textcolor{PineGreen}{\pm}$ \textcolor{PineGreen}{本征模}(左)的\textcolor{Plum}{反向传播},消除了\textcolor{Plum}{焦平面}中目标矢量场(\textcolor{PineGreen}{水平极化}的 \textcolor{Maroon}{\text{LG}}$_{l=-10}$\cite{zusinBesselBeamTransformation2010} $+$ \textcolor{PineGreen}{垂直极化}的 \textcolor{Maroon}{\text{HG}}$_{11}$)的直接各向同性\textcolor{Plum}{反向传播}(右)引入的像差。该技术不需要任何\textcolor{Maroon}{泽尼克多项式}补偿。\textbf{c2} 在高\textcolor{Plum}{数值孔径}的情况下,在先\textcolor{PineGreen}{聚焦}后\textcolor{PineGreen}{发散}的轴向场之外,独立存在一个持续\textcolor{PineGreen}{发散}的“\textcolor{NavyBlue}{切伦科夫锥}”。\textbf{c3} 用于模拟(\textbf{c1}、 \textbf{c2},和 \textbf{c4})的铌酸锂(LN)的\textcolor{PineGreen}{本征系统},其中\textcolor{PineGreen}{垂直极化}的\textcolor{Maroon}{高斯光束}用于模拟(\textbf{c2})。\textbf{c4} \textcolor{NavyBlue}{右旋圆偏}\textcolor{Maroon}{高斯}\textcolor{NavyBlue}{泵浦}沿着 LN 的\textcolor{PineGreen}{光轴}运动,物镜出瞳被设置在材料表面。从左起的前 6 个光焦点:通过切换焦距至材料内的不同深度处,在不进行波前校正的情况下,光强剖面分布。光焦点 7-10:将第 6 个焦点分解为\textcolor{NavyBlue}{左}/\textcolor{NavyBlue}{右旋圆偏振}或\textcolor{PineGreen}{本征模们}:分别揭示了在不进行\textcolor{Maroon}{像差校正}的情况下,\textcolor{Plum}{横向}\textcolor{NavyBlue}{旋轨耦合}和轴向多焦点的本质 --- 失配 \textcolor{PineGreen}{o 波}和 \textcolor{PineGreen}{e 波}之间的\textcolor{PineGreen}{干涉}。\textbf{d1} 单个光束(左和右)或两个相干光束(中间),因其在 $\mathcolor{gray}{\bar{k}_{\symup{\rho}}}$ 的\textcolor{Maroon}{角谱} $\Xint{\mathcolor{gray}{-}}{25}{\bar{E}}^{\;\!\mathcolor{gray}{\omega}}_{\;\! \textcolor{Maroon}{\Yup} \mathcolor{gray}{z}}$ 分布,覆盖了\textcolor{Plum}{(双曲)双轴}晶体的两个\textcolor{PineGreen}{光轴},而经历两次单独的\textcolor{PineGreen}{锥形折射},叠加并\textcolor{PineGreen}{干涉}。--- 在单光束的情况下,光场可能会发生自扭曲行为,特别是当\textcolor{NavyBlue}{泵浦}\textcolor{Maroon}{角谱}在\textcolor{gray}{倒空间}的覆盖面积大于(右)或等于(左)\textcolor{Plum}{双轴夹角}时。\textbf{d2} 来自(\textbf{d1})的三个\textcolor{NavyBlue}{泵浦},在\textcolor{gray}{互易空间}中对应的\textcolor{PineGreen}{折射率}分布}{fig:high_N.A.}
\end{figure}

在这里,我们展示了\textcolor{PineGreen}{拉曼亮点}(\textcolor{PineGreen}{Raman spike})的起源(\bref{fig:high_N.A.}\textbf{b}),无\textcolor{Maroon}{泽尼克多项式}的\textcolor{Plum}{焦平面}处目标\textcolor{PineGreen}{矢量复场}的逆向工程技术(\bref{fig:high_N.A.}\textbf{c1}),内光束外的“\textcolor{NavyBlue}{切伦科夫锥}”(\bref{fig:high_N.A.}\textbf{c2}),高\textcolor{Plum}{数值孔径 N.A.}\textcolor{NavyBlue}{泵浦}的 R-L 和 o-e 分解分析(\bref{fig:high_N.A.}\textbf{c4}),以及\textcolor{PineGreen}{双圆锥折射}及其引起的光场自扭曲效应(\bref{fig:high_N.A.}\textbf{d1})。

\begin{figure}[htbp!]
	\captionsetup{labelformat=adja-page}
	\ContinuedFloat
	\captionsetup[figure]{name=Figure}
	\backcaption{-0.7em}{\textbf{Forward and backward propagation in anisotropic materials with high N.A. pump.} \textbf{a} Bessel caustics reconstructed from Zusin et al.\cite{zusinBesselBeamTransformation2010}. \textbf{b} Raman spike of CR origins solely from the slow mode. The vertically polarized off-optical-axis LG$_1$ pump is chosen to pump $\alpha$-HIO$_3$. \textbf{c1} Inverse design of tightly focused focal fields: The backward propagation(BP) of the superposed $\pm$ eigenmodes (left) eliminates the aberrations introduced by the direct isotropic BP (right) of the target desired vector field in the focal plane: horizontally polarized LG$_{l=-10}$\cite{linFastVectorialCalculation2012} $+$ vertically polarized HG$_{11}$, without the need for any Zernike polynomial compensation. \textbf{c2} In the case of high N.A., outside the axial field that first focuses and then diverges, a ``Cherenkov cone'' that continuously diverges exists independently. \textbf{c3} Lithium niobate(LN)'s eigensystems used to simulate (\textbf{c1}, \textbf{c2}, and \textbf{c4}), where vertically polarized Gaussian is used to simulate (\textbf{c2}). \textbf{c4} Right-handed circularly polarized(RHCP) Gauss pump travels along the optical axis of LN, with the objective aperture at the surface of the material. The first 6 spots from the left: By switching focal lengths, the intensity section distribution within the material at different depths without wavefront correction. Spots 7-10: Decomposing the sixth spot into L/RHCP or eigenmodes: respectively reveals the transverse SOI and the nature of axial multifocality when no aberration correction is applied --- interference between mismatched o- and e-waves. \textbf{d1} A single light beam (left \& right), or two coherent beams (middle), whose angular spectrum distributions in $\bar{k}_{\uprho}$ domain cover both optical axes of the (hyperbolic) biaxial crystal, undergo two separate conical refractions that superimpose and interfere. --- In the case of a single beam, light field self-twisting behavior is likely to occur, especially when the reciprocal space coverage of the pump's angular spectrum is greater than (right) or equal to (left) the angle of bi-axes. \textbf{d2} The distributions of the three pumps from (\textbf{d1}) in reciprocal space relative to the refractive index.}{fig:high_N.A.-ecaption}
	\captionsetup[figure]{name=图}
\end{figure}

复场的\textcolor{PineGreen}{干涉}会产生复杂和扭曲的光斑形状,特别是在\textcolor{PineGreen}{紧密聚焦}的情况下。这一现象是\textcolor{Maroon}{波动光学}(平行于\textcolor{Maroon}{量子力学})的象征,是其最显著的特征 --- 区分于\textcolor{Maroon}{射线光学}(类似于\textcolor{Maroon}{经典力学}),它不仅代表了\textcolor{Plum}{高度振荡}偏微分方程(PDE)的数学前沿,其中内部\textcolor{Plum}{被积函数}和\textcolor{Plum}{积分结果}都可能是高度振荡的\cite{agocsAdaptiveSpectralMethod2024},而且在\textcolor{Maroon}{采样定理}\cite{wangTheoryAlgorithmHomeomorphic2020} 的约束下,在提高\textcolor{Plum}{速度精度乘积}方面也在一直推动着突破\textcolor{Plum}{计算极限},因此是全面挑战所有声称是物理的神经网络性能的最有希望的候选者/场景之一\cite{wrightDeepPhysicalNeural2022}。

材料的\textcolor{Plum}{强各向异性}进一步提高了计算光场分布的\textcolor{Plum}{计算资源需求}。这源于电磁场\textcolor{PineGreen}{特征值曲面}及其相关\textcolor{NavyBlue}{波前(等相面)} 的增强\textcolor{Plum}{畸变},再加上控制偏振方向的\textcolor{PineGreen}{特征矢量}变化更剧烈,最终沿 \textcolor{Plum}{x}、\textcolor{Plum}{y}、\textcolor{Plum}{z 轴}实现\textcolor{PineGreen}{干涉},加速了噩梦的过早到来:即,在 $\mathcolor{gray}{z}=\mathcolor{gray}{z_0}$ 处的典型平面\textcolor{PineGreen}{干涉}图案上,相邻(采样)像素之间的\textcolor{NavyBlue}{相位差}超过 $\symup{\pi}$ \cite{leuteneggerFastFocusField2006,heintzmannScalableAngularSpectrum2023}。

在不进行任何优化的情况下,使用\textcolor{PineGreen}{晶体光学}版本的\textcolor{Maroon}{矢量角谱法(ASM)},我们在\textcolor{Maroon}{采样定理}的边缘操作,在\textcolor{Plum}{混叠误差}发生之前,探索了各种\textcolor{PineGreen}{紧聚焦}的焦场分布。\bref{fig:high_N.A.} 中的每个子图(除了 \bref{fig:high_N.A.}\textbf{c2})都接近\textcolor{Plum}{计算极限}。最初,我们认为 \textbf{c2} 中\textcolor{NavyBlue}{外锥}的起源是\textcolor{NavyBlue}{物理的},因为 {\one} 它甚至在传播之前就开始形成了,{\two} 它出现在图像的中心,而不是(像通常的\textcolor{Plum}{混叠误差}一样,表现地)从边缘“\textcolor{PineGreen}{反射}”。然而,我们后来发现,它仍然是由\textcolor{Plum}{循环卷积}引起的\textcolor{Plum}{混叠误差}在\textcolor{gray}{正空间}的表现。得出这一结论是出于,为了在 LN 材料内移动 \bref{fig:high_N.A.}\textbf{c2} 的\textcolor{Plum}{焦平面},我们将材料入口处的入射场分布设置为\textcolor{PineGreen}{线偏}\textcolor{Maroon}{高斯光束},其 N.A. 接近 1 (见 \bref{fig:high_N.A.}\textbf{c3} 的顶行),向后传播 $\mathcolor{gray}{z_0}=\mathcolor{gray}{-0.2}$ \textcolor{gray}{mm}。然而,当 $\mathcolor{gray}{z_0}$ 设置为零时,\textcolor{NavyBlue}{外环}即“\textcolor{NavyBlue}{切伦科夫锥}”消失了。相比之下,\bref{fig:high_N.A.}\textbf{c4} 使用\textcolor{PineGreen}{物镜}的\textcolor{Plum}{传递函数}而不是\textcolor{Plum}{反向传播}来将\textcolor{Plum}{焦平面}移动到 LN 上表面以下,从而完全消除了\textcolor{Plum}{混叠误差}。

我们发现,通过将 $\underline{\varepsilon}^{\;\! \mathcolor{gray}{\omega} \textcolor{Maroon}{(1)}}_{\symup{zz}}$\Footnote{给变量底部加上一条短横线,代表它是在\textcolor{PineGreen}{主轴}/\textcolor{PineGreen}{晶体学}\textcolor{Plum}{坐标系}\cite{xieAnalytic3DVector}(\textcolor{PineGreen}{$\mathcal{C}$ 系})下的\textcolor{NavyBlue}{物理量},见 \bref{hook:C_frame}。}设置为\textcolor{Plum}{负值},从而将\textcolor{PineGreen}{折射率曲面}从\textcolor{Plum}{椭球面}转换为\textcolor{NavyBlue}{介电型}\textcolor{Plum}{双叶双曲面},可以使两个\textcolor{PineGreen}{光轴}更靠近,如 \bref{fig:high_N.A.}\textbf{d2} 所示。对于 1064 nm 的 KTP 晶体,两个\textcolor{PineGreen}{光轴}之间的角度从\textcolor{Plum}{椭球面}情况下的 34.6° × 2 减小到\textcolor{Plum}{双曲面}情况下的 7.7° × 2,使我们能够在整个\textcolor{Maroon}{晶体-2f 系统}相对较长的\textcolor{Plum}{传播距离}内以\textcolor{PineGreen}{非傍轴}方式计算\textcolor{PineGreen}{矢量复场}在晶体内外的完整演化。\bref{fig:high_N.A.}\textbf{d1} 的第一列和第三列中的光场(打)结,超出了我们目前的\textcolor{Plum}{数学理解}和\textcolor{Plum}{解释技术}。

\vspace*{-4.5em}

\marginLeft[-2.4em]{ssec:LFCO-Superstructure}\subsection{该线性晶体光学模型的上层建筑:非线性晶体光学}\label{ssec:LFCO-Superstructure}

\textcolor{Plum}{非线性}\textcolor{NavyBlue}{光学}作为探索和理解\textcolor{Maroon}{光-物质相互作用}的深刻门户,在\textcolor{Maroon}{本构关系}中引入了所有高阶\textcolor{Plum}{非线性}项 $\bar{P}^{\;\! \mathcolor{gray}{\omega}}_{\;\! \mathcolor{gray}{z} \textcolor{Maroon}{(2)}} + \bar{P}^{\;\! \mathcolor{gray}{\omega}}_{\;\! \mathcolor{gray}{z} \textcolor{Maroon}{(3)}} + \cdots$,超出了一阶\textcolor{Plum}{线性}电极化率 $\bar{\bar{\chi}}^{\;\! \mathcolor{gray}{\omega} \mathcolor{gray}{\overbrace{\check{\symup{\jmath}}}}}_{\mathcolor{gray}{z} \textcolor{Maroon}{(1)}} = \bar{\bar{\varepsilon}}^{\;\! \mathcolor{gray}{\omega} \mathcolor{gray}{\overbrace{\check{\symup{\jmath}}}}}_{\mathcolor{gray}{z} \textcolor{Maroon}{(1)}} - 1$。这些高阶项表示电\textcolor{Plum}{多极}子对外部\textcolor{NavyBlue}{泵浦光场}驱动的\textcolor{Plum}{非线性}响应。最终,它们充当\textcolor{gray}{跨带}\textcolor{gray}{光源},通过\textcolor{Maroon}{参量过程}在晶体内\textcolor{PineGreen}{相干}地产生\textcolor{gray}{新频率}的辐射(也有涉及声子或分子等的\textcolor{Maroon}{非参量}/\textcolor{Maroon}{非弹性}场景\cite{grundmannOpticallyAnisotropicMedia2017,boydNonlinearOptics2019})。新产生的\textcolor{gray}{新频率} $\{ \mathcolor{gray}{\omega}_{\textcolor{Maroon}{i}} \}$ 的相干辐射和参与相互作用的\textcolor{NavyBlue}{泵浦},都受到它们自己的\textcolor{gray}{单色}\textcolor{NavyBlue}{被动}波动方程 \bref{eq:simplify7-L2-zeta} 的约束,使得它们须分解为晶体的\textcolor{PineGreen}{本征模们}(\bref{fig:conical_phase_match}\textbf{d}-\textbf{e},\textbf{g}-\textbf{h})之后,再进行互相独立的衍射/传播。

\textcolor{Maroon}{参数}\textcolor{gray}{频率转换}\textcolor{Maroon}{过程}须首先满足\textcolor{gray}{能量守恒},其次是\textcolor{gray}{动量守恒}(通常称为\textcolor{PineGreen}{波矢}或\textcolor{PineGreen}{相位匹配}),这进一步测试了\textcolor{Plum}{线性}\textcolor{PineGreen}{晶体光学}框架内的\textcolor{PineGreen}{特征值}的\textcolor{Plum}{计算}精度(\bref{fig:conical_phase_match}\textbf{d}-\textbf{e})。如果还额外涉及二阶\textcolor{Plum}{非线性}系数张量 $\Xint{{}^{}_{\mathcolor{gray}{-}}}{23}{\bar{\bar{\bar{\chi}}}}^{\;\! \mathcolor{gray}{\omega} \textcolor{PineGreen}{\imath \jmath l}}_{\mathcolor{gray}{z} \textcolor{Maroon}{(2)}}$ 的\textcolor{Plum}{各向异性},则还必须保证精确\textcolor{Plum}{计算} \textcolor{PineGreen}{$\mathcal{C}$ 系}下\cite{midwinterEffectsPhaseMatching1965,yaoAccurateCalculationOptimum1992,dmitrievEffectiveNonlinearityCoefficients1993,diesperovEffectiveNonlinearCoefficient1997}的\textcolor{Plum}{线性}\textcolor{PineGreen}{晶体光学}\textcolor{PineGreen}{特征向量}(\bref{fig:conical_phase_match}g-h)作为先决条件之二。

两个\textbf{核心原理},即参与\textcolor{Plum}{非线性}\textcolor{NavyBlue}{光学}过程的所有\textcolor{gray}{频率} $\{ \mathcolor{gray}{\omega}_{\textcolor{Maroon}{i}} \}$ 的电磁波的\textbf{被动独立衍射}  \bref{eq:simplify7-L2-zeta} 和\textbf{主动耦合转换}  \bref{eq:simplify7-LE0-SVA-V_1singular-nokxky-zeta-g},将所有\textcolor{Plum}{非线性}\textcolor{NavyBlue}{光学}现象锚定在\textcolor{Plum}{线性}\textcolor{PineGreen}{晶体光学}(至少其 \textcolor{NavyBlue}{0 阶微扰})的框架内。这使得\textcolor{Plum}{各向异性}材料中\textcolor{Plum}{非线性}过程的精确\textcolor{NavyBlue}{理论建模}自然成为对所有已建立的\textcolor{Plum}{线性}\textcolor{PineGreen}{晶体光学}模型的更严格的测试。

为了展示矢量\textcolor{Plum}{非线性}\textcolor{Maroon}{傅立叶}\textcolor{PineGreen}{晶体光学}的基础知识,\bref{fig:conical_phase_match}\textbf{j} 再现了 \textcolor{Maroon}{Belyi} 等人\cite{belyiPropagationHighorderCircularly2011}提出的\textcolor{PineGreen}{全锥形相位匹配}二次谐波发生(SHG),说明了如何通过 \bref{fig:conical_phase_match}\textbf{a}-\textbf{i} 计算\textcolor{Plum}{单轴} BBO 晶体中 $\mathcolor{gray}{\omega} \to \mathcolor{gray}{2\omega}$ 的\textcolor{gray}{频率}转换,其中$\mathcolor{gray}{\omega}$、$\mathcolor{gray}{2\omega}$ 两个\textcolor{gray}{频率}处的一阶、二阶极化率 $\bar{\bar{\chi}}^{\;\! \mathcolor{gray}{\omega}}_{ \textcolor{Maroon}{(1)}},\bar{\bar{\chi}}^{\;\! \mathcolor{gray}{2\omega}}_{ \textcolor{Maroon}{(1)}},\bar{\bar{\bar{\chi}}}^{\;\! \mathcolor{gray}{2\omega}}_{ \textcolor{Maroon}{(2)}}$ 全都是\textcolor{Plum}{均匀}\textcolor{Plum}{各向异性}的。

\begin{figure}[htbp!]
	\centering
	\includegraphics[width=1.0\textwidth]{D:/C2D/Desktop/article_fig/phd_thesis_fig/chapter-03/fig_6.1.conical_phase_match.pdf}
	\biackcaption[\textbf{Reinterprete type-I o$+$o$\to$e full conical phase matching\cite{belyiPropagationHighorderCircularly2011} along the optical axis of BBO crystal for SHG within the framework of nonlinear Fourier crystal optics(NFCO).} On the first `Light' row, by squaring the RHCP Bessel fundamental wave (\textbf{a}) and expanding its field of view with a twofold interpolation in the $\bar{k}_{\uprho}$ domain, the intracrystal nonlinear driven source $\bar{P}^{(2)}_{2\omega}$ (\textbf{b}) is obtained. This traveling field, entirely determined by the pump, is multiplied by the ``eigenvalue efficiency mask'' = longitudinal phase matching coherence level (\textbf{c}) derived from the path (\textbf{d}$-$\textbf{e}$\to$\textbf{f}$\to$\textbf{c}), followed by the ``eigenvector efficiency mask'' = effective nonlinear coefficient distribution\cite{midwinterEffectsPhaseMatching1965,yaoAccurateCalculationOptimum1992,dmitrievEffectiveNonlinearityCoefficients1993,diesperovEffectiveNonlinearCoefficient1997} (\textbf{i}) from the path (\textbf{g}$-$\textbf{h}$\to$\textbf{i}). The resulting output in (\textbf{j}) is a hexagonal conical radial vector light field purely composed of extraordinary light of BBO at $2\omega$.]{-0.7em}{\textbf{在\textcolor{Plum}{非线性}\textcolor{Maroon}{傅立叶}\textcolor{PineGreen}{晶体光学}框架内,重新解释沿 BBO 晶体\textcolor{PineGreen}{光轴}的 I 型 \textcolor{PineGreen}{o}$+$\textcolor{PineGreen}{o}$\to$\textcolor{PineGreen}{e} \textcolor{PineGreen}{全锥形相位匹配}\cite{belyiPropagationHighorderCircularly2011},用于二次谐波生成(SHG)。}在第一行 `Light' 中,通过将\textcolor{NavyBlue}{右旋圆偏振}\textcolor{Maroon}{贝塞尔}\textcolor{PineGreen}{基波}(\textbf{a})\textcolor{Plum}{平方},并在 $\mathcolor{gray}{\bar{k}_{\symup{\rho}}}$ 域中通过\textcolor{Plum}{两倍插值}扩大其视野,获得了晶体内\textcolor{Plum}{非线性}\textcolor{NavyBlue}{驱动源} $\bar{P}^{\;\! \mathcolor{gray}{\omega}}_{\;\! \mathcolor{gray}{z} \textcolor{Maroon}{(2)}}$(\textbf{b})。这个完全由\textcolor{NavyBlue}{泵浦}决定的\textcolor{NavyBlue}{行波场}乘以衍生自路径(\textbf{d}$-$\textbf{e}$\to$\textbf{f}$\to$\textbf{c})的``\textcolor{PineGreen}{特征值}效率掩模'' $=$ \textcolor{Plum}{纵向}\textcolor{PineGreen}{相位匹配的相干水平}(\textbf{c}),再乘以来自路径(\textbf{g}$-$\textbf{h}$\to$\textbf{i})的“\textcolor{PineGreen}{特征向量}效率掩模” $=$ \textcolor{NavyBlue}{有效非线性系数}分布\cite{midwinterEffectsPhaseMatching1965,yaoAccurateCalculationOptimum1992,dmitrievEffectiveNonlinearityCoefficients1993,diesperovEffectiveNonlinearCoefficient1997}。最终得到的\textcolor{Plum}{输出}是(\textbf{j})中的一个 \textcolor{NavyBlue}{C3 对称}的\textcolor{Plum}{锥形径向}\textcolor{PineGreen}{矢量光场},完全由 $\mathcolor{gray}{2\omega}$ 的 BBO 的\textcolor{PineGreen}{非常光}组成\\}{fig:conical_phase_match}
\end{figure}

在矢量\textcolor{Plum}{非线性}\textcolor{Maroon}{傅立叶}\textcolor{PineGreen}{晶体光学}中,\textcolor{Plum}{横向}\textcolor{PineGreen}{波矢}(即\textcolor{gray}{空间频率})\textcolor{gray}{守恒}(\bref{fig:conical_phase_match}\textbf{a}-\textbf{b})被提升至与 $\mathcolor{gray}{\omega}$ \textcolor{gray}{守恒}和\textcolor{Maroon}{边界条件}(如切向\textcolor{Plum}{不连续} \bref{eq:^1Y_0}、相位\textcolor{Plum}{连续性}等)相同的基本水平,使其成为另一个必须优先考虑的先决条件。转换效率仅由\textcolor{Plum}{纵向}\textcolor{PineGreen}{相位失配量} $\Delta k_\mathrm{z} z$(\bref{fig:conical_phase_match}\textbf{f})和\textcolor{NavyBlue}{有效非线性系数}(\bref{fig:conical_phase_match}\textbf{i})决定\cite{midwinterEffectsPhaseMatching1965,yaoAccurateCalculationOptimum1992,dmitrievEffectiveNonlinearityCoefficients1993,diesperovEffectiveNonlinearCoefficient1997},在优先级上次要于\textcolor{gray}{横向动量守恒}的满足。

作为决定\textcolor{gray}{频率转换}效率的角向分布的两个掩模,来自 \bref{fig:conical_phase_match}\textbf{c} 的\textcolor{Plum}{纵向}\textcolor{PineGreen}{相位匹配程度} sinc$(\Delta k_\mathrm{z} z)$ 和来自 \bref{fig:conical_phase_match}\textbf{i} 的\textcolor{NavyBlue}{有效非线性系数} $\chi^{(2)\text{ooe}}_{2\omega;\text{eff}}$ 分别取决于\textcolor{Plum}{线性}\textcolor{Maroon}{傅里叶}\textcolor{PineGreen}{晶体光学}的\textcolor{PineGreen}{特征矢量}和\textcolor{PineGreen}{特征值},重申了\textcolor{Plum}{非线性}\textcolor{PineGreen}{晶体光学}本质上植根于\textcolor{Plum}{线性}\textcolor{PineGreen}{晶体光学},并且\textcolor{Plum}{线性}和\textcolor{Plum}{非线性}\textcolor{NavyBlue}{光学}最终都应纳入\textcolor{NavyBlue}{傅里叶光学}框架。

\begin{figure}[htbp!]
	\centering
	\includegraphics[width=1.0\textwidth]{D:/C2D/Desktop/article_fig/phd_thesis_fig/chapter-03/fig_6.2.VNFCO_stepstones.pdf}
	\biackcaption[\textbf{Towards scalar/vector nonlinear Fourier crystal optics(NFCO).} \textbf{b1} Reconstructed experiment results from Grant et al.\cite{grantFrequencydoubledConicallyrefractedGaussian2014} in their Fig. 4 on chiral SHCR by utilizing the third stage (in \textbf{b3}) of this LCO model: vector NFCO. \textbf{b2} The real-space and reciprocal-space distributions $U,G$ of the 532 nm second harmonic wave(SHW) at the focal plane, generated by pumping a vertically polarized 1064 nm Gauss fundamental wave along KTP's optical axis at 532 nm, along with its decomposition into six phase-matching types.]{-0.7em}{\textbf{通往(标/矢量)\textcolor{Plum}{非线性}\textcolor{Maroon}{傅里叶}\textcolor{PineGreen}{晶体光学}。} \textbf{b1} 通过该\textcolor{Plum}{线性}\textcolor{PineGreen}{晶体光学}模型的\textbf{第三阶段}(在 \textbf{b3} 中)即矢量\textcolor{Plum}{非线性}\textcolor{Maroon}{傅里叶}\textcolor{PineGreen}{晶体光学},重建来自 \textcolor{Maroon}{Grant} 等人关于\textcolor{NavyBlue}{手性}二次谐波\textcolor{PineGreen}{锥衍射}实验结果\cite{grantFrequencydoubledConicallyrefractedGaussian2014}(其 Fig.4)。\textbf{b2} 通过沿 KTP 晶体在 532 nm 处的\textcolor{PineGreen}{光轴},\textcolor{NavyBlue}{泵浦}\textcolor{PineGreen}{垂直极化}的 1064 nm \textcolor{Maroon}{高斯}\textcolor{PineGreen}{基波},在\textcolor{Plum}{焦平面}处产生 532 nm 二次谐波的\textcolor{gray}{正}、\textcolor{gray}{倒空间}分布 \textcolor{Maroon}{$U$}、\textcolor{Maroon}{$G$},并将其分解成 6 种\textcolor{PineGreen}{相位匹配类型}\\}{fig:VNFCO_stepstones}
\end{figure}

在本文的逻辑链/树/网下,(\textcolor{Maroon}{傅里叶})\textcolor{PineGreen}{晶体光学}分类为\textcolor{Plum}{线性}和\textcolor{Plum}{非线性}。其中,\textcolor{Plum}{线性}\textcolor{PineGreen}{晶体光学}一定是矢量的(见 \bref{eq:simplify7-L2-zeta}),而\textcolor{Plum}{非线性}\textcolor{PineGreen}{晶体光学}则进一步分为标量、矢量版本(见)。并且,\textcolor{Plum}{非线性}\textcolor{PineGreen}{晶体光学}的矢量版本还可再进一步划分为。\bref{fig:VNFCO_stepstones} 中展示的是细分至最末梢的,也可以说是\textbf{目前据我所知全球范围内最先进的}\Footnote{解析解的存在之本身,已经是参数最少且信息量最大的白盒子模型了(这里引用 \textcolor{Maroon}{Oliver Heaviside} 的名言:\textbf{Logic can be patient for it is eternal.})。再加上正在数值上将其积极推进到速度精度积最大。}\textcolor{Plum}{非线性}\textcolor{PineGreen}{晶体光学}模型。%(也是\textbf{目前据我所知全球范围内最先进的})\textcolor{Plum}{非线性}\textcolor{PineGreen}{晶体光学}模型

以 \bref{fig:VNFCO_stepstones} 为代表的矢量\textcolor{Plum}{非线性}\textcolor{PineGreen}{晶体光学}实验数据图的复现困难程度比较高(在缺乏本章的\textcolor{Plum}{线性}\textcolor{PineGreen}{晶体光学}前置知识、后续\textcolor{Plum}{非线性}\textcolor{PineGreen}{晶体光学}进阶数学知识的情况下),目前世界范围内不仅\textbf{暂未出现其他}\textcolor{Plum}{非线性}模型,能够完全复现 \bref{fig:VNFCO_stepstones} 的每一张子图,而且\textbf{作者} \textcolor{Maroon}{Kalkandjiev} 等人\textbf{本人},\textbf{也没有展示}如何通过\textcolor{NavyBlue}{理论建模}模拟/仿真出他们自己文章中的相关实验图中的任何一张子图\cite{zolotovskayaSecondharmonicConicalRefraction2011,kroupaSecondharmonicConicalRefraction2010,alekseevaShadowConicalRefraction1999,shihConicalRefractionSecondHarmonic1969,schellLaserStudiesInternal1978,velichkinaDemonstrationPhenomenaConical1980,stroganovConicalRefractionSecond1980,illarionovExperimentalObservationConical1979,féveExperimentalStudyInternal1994,grantFrequencydoubledConicallyrefractedGaussian2014,maSumfrequencyGenerationFemtosecond2018,peetFrequencyDoublingLaser2011},尽管 \textcolor{Maroon}{Bloembergen} 等人作了很大力气的尝试\cite{shihConicalRefractionSecondHarmonic1969,schellLaserStudiesInternal1978}。

原因在于:{\one} 客观$\Big|$物:由于该过程是\textcolor{PineGreen}{相位失配}的,所有\textcolor{PineGreen}{相位匹配类型}全都参与进来的同时,材料\textcolor{Plum}{非线性}系数的所有\textcolor{Plum}{非零张量元}/\textcolor{Plum}{分量}也几乎都贡献至\textcolor{Plum}{非线性}相互作用\cite{kroupaSecondharmonicConicalRefraction2010},此外晶体的\textcolor{Plum}{对称性}较低,不仅\textcolor{NavyBlue}{双轴}而且是\textcolor{NavyBlue}{旋光}的\cite{zolotovskayaSecondharmonicConicalRefraction2011}。{\two} 主观$\Big|$人:\textcircled{1} 绝大部分\textcolor{Plum}{非线性}\textcolor{PineGreen}{晶体光学}研究人员,由于或多或少偏向\textcolor{NavyBlue}{实验主导},从现象到现象,从表到表,至于里子,关心得还不够。\textbf{特别是对\textcolor{Plum}{线性}\textcolor{PineGreen}{晶体光学}的忽略,甚至不知道它是\textcolor{Plum}{非线性}\textcolor{PineGreen}{晶体光学}的基础}:想想看 \textcolor{PineGreen}{o} 光 \textcolor{PineGreen}{e} 光这些词都是从哪来的?\textcircled{2} 未从源头上去认真处理,甚至不打算、没有兴趣去认真对待:\textcolor{Plum}{非线性}\textcolor{PineGreen}{晶体光学}的\textcolor{Plum}{数学起源}。比如\textcolor{NavyBlue}{物理微观}层面少有人知道\textcolor{Maroon}{多极理论}\Footnote{然而,本科\textcolor{NavyBlue}{电动力学}理应曾涉及过\textcolor{NavyBlue}{电四极矩}。},\textcolor{Plum}{数学形式}上更少有人听说过 \textcolor{Maroon}{Volterra 级数}。 --- 然而,原则上只要他们对二阶\textcolor{Plum}{非线性}系数的\textcolor{NavyBlue}{量子力学}表达式挖得够深,每一个人都应该独立发现上述这两个理论。
%\Footnote{只要能解释实验现象,什么模型不重要,抓到耗子就是好猫。至于抓得快不快、对耗子是否有损伤、抓的方法好不好,在这上面没有数学品味或研究兴趣。}
%以至于几乎从不关注与\textcolor{Plum}{非线性}\textcolor{NavyBlue}{光学}看似不相关的现象和领域\Footnote{当然,对于想方设法“借鉴”别的领域,他们还是有一番属于他们自己的“心得”的。}(毕竟{非线性}光学本身在实验现象上的丰富已经足够他们忙活)

要是他们怀揣着与我相同的疑惑,从 \textcolor{Maroon}{Maxwell-Lorentz-Heaviside} 方程组开始,逻辑严密且不跳过任何步骤,他们也会独立得到与我相近/同的结论。但很可惜的是,\textbf{从 \textcolor{Maroon}{Franken}\cite{frankenGenerationOpticalHarmonics1961} 至今,没有多少人尝试着不省略任何推导步骤}。在这个意义上,偏工程/实验的\textcolor{Plum}{非线性}\textcolor{NavyBlue}{光学}从业者的平均\textcolor{Plum}{数学基础},相比同样偏向工程的物理学科:\textcolor{Plum}{计算}\textcolor{Maroon}{结构}/\textcolor{Maroon}{工程力学}\cite{ZhongWanXieZuiYouKongZhiYuJiSuanJieGouLiXueDeMoNiLiLun1993,chandrupatlaIntroductionFiniteElements2011}、\textcolor{Plum}{计算}\textcolor{Maroon}{固体}/\textcolor{Maroon}{流体力学}\cite{kirillovInstabilitiesViscothermodiffusiveSwirling2025,kirillovGeometricalOpticsStability2025}从业者,不知道低多少个数量级。这也是本文的存在意义之首 --- 提醒大小同行们,或多或少看看周围\Footnote{当然,对于想方设法“借鉴”别的领域,他们还是有一番属于他们自己的“心得”的。}:同样的\textcolor{NavyBlue}{物理大类}领域内,隔壁分支是怎么做\textcolor{NavyBlue}{基础研究}的,以及是基于什么样的\textcolor{NavyBlue}{基础研究}在搭上层建筑,和指导\textcolor{NavyBlue}{实验研究}。

当然,\textcolor{Plum}{非线性}\textcolor{NavyBlue}{光学}实验现象之丰富,机理之复杂,远超本文的几个模型的适用范围\Footnote{即使是\textcolor{Plum}{线性}\textcolor{PineGreen}{晶体光学},都已经上至\textcolor{NavyBlue}{广义相对论}\cite{yakovNonbirefringenceConditionsSpacetime2005,hehlSpacetimeMetricLocal2006,frankelGeometryPhysicsIntroduction2011,dahlNondissipativeElectromagneticMedium2013,ivezicManifestlyCovariantAharonovbohm2015,hehlAxionDilatonMetric2016,favaroRecentAdvancesClassical2012},下至\textcolor{NavyBlue}{量子力学}\cite{nelsonMechanismsDispersionCrystalline1989,nelsonDerivingTransmissionReflection1995,nelsonLagrangianTreatmentMagnetic1994,nelsonOpticallyActiveSurface2000}、\textcolor{NavyBlue}{相对论}\cite{hehlRecentDevelopmentsPremetric2006,obukhovGeneralTreatmentQuantum2017}和/或\textcolor{NavyBlue}{量子电动力学}\cite{eimerlQuantumElectrodynamicsOptical1988}。更别说\textcolor{Plum}{非线性}\textcolor{PineGreen}{晶体光学}和包含之的\textcolor{Plum}{非线性}\textcolor{NavyBlue}{光学}、\textcolor{Plum}{非线性}\textcolor{NavyBlue}{电动力学}\cite{laxLinearNonlinearElectrodynamics1971,nelsonLagrangianTreatmentMagnetic1994}了:后者还包含场强更高的\textcolor{NavyBlue}{相对论性}\textcolor{Plum}{非线性}\textcolor{NavyBlue}{量子电动力学}\cite{boillatNonlinearElectrodynamicsLagrangians1970,peresNonlinearElectrodynamicsGeneral1961,anNakedSingularityCensoring2024,adornodefreitasSingularElectromagneticFields2023,adornodefreitasNonlinearityElectroMagnetostatics2015,adornoWhenElectricCharge2015,adornoMagneticResponseApplied2014,adornoQuantumElectromagneticNonlinearity2017,adornoMagneticPoleProduced2020}。},从这个意义上讲,像本文一样尝试着去“公理化”\textcolor{Plum}{非线性}\textcolor{PineGreen}{晶体光学},从整个行业的角度似乎是不可取的:因为\textbf{\textcolor{Plum}{非线性}\textcolor{PineGreen}{晶体光学}是\textcolor{Plum}{理}\textcolor{NavyBlue}{工}融合的典范},所以这一点我也持保留态度。毕竟也确实需要源源不断的新鲜实验现象,为拓展理论的边界、优化理论的内核,甚至推翻理论本身,注入生命活力。总的来说,置身于同样一片旷野中,飞鸟和青蛙的选择不同罢。

此外,\textcolor{Plum}{非线性}\textcolor{NavyBlue}{光学}不等于\textcolor{Plum}{非线性}\textcolor{PineGreen}{晶体光学},前者包含了后者,因此前者对后者的\textcolor{Plum}{补集},可以不那么关注\textcolor{PineGreen}{晶体光学},比如\textcolor{NavyBlue}{飞秒泵浦}各向同性\textcolor{NavyBlue}{固}/\textcolor{NavyBlue}{液}/\textcolor{NavyBlue}{气}/\textcolor{NavyBlue}{等离子体}所对应的\textcolor{Plum}{非线性}\textcolor{NavyBlue}{光学}。但后文会提到,它们可能仍然受标量\textcolor{Plum}{非线性}\textcolor{Maroon}{角谱}理论描述。从这个角度,\textcolor{Plum}{非线性}\textcolor{NavyBlue}{光学}内部一些看似迥异的分支及其现象,实际上背后或许存在一个大的统一框架将它们全都包含在内。这是\textcolor{Plum}{非线性}\textcolor{NavyBlue}{光学}\textcolor{NavyBlue}{理论物理}研究所需要做的:往后退、连连看/找相同/抽象成大类、数学为主导语言,物理规律服从数学结构、从理论预测实验;而不是像实验工作者一样,往前进、反常识/找不同/具象成实例、数学作为描述工具,服务于物理现象的归纳、从实验反推理论。--- 尽管二者都在帮助彼此互相往前进。

\vspace*{-1.5em}

\marginLeft[-2.4em]{sec:summary-chapter2}\section{\textcolor{Maroon}{Summary} 小结 \textcolor{Maroon}{of chapter 3}}\label{sec:summary-chapter3}

在任意 $\bar{\bar{\chi}}^{\;\! \mathcolor{gray}{\omega}}_{\mathcolor{gray}{z} \textcolor{Maroon}{(1)}}$ 的\textcolor{NavyBlue}{双折射}-\textcolor{NavyBlue}{手性}-\textcolor{NavyBlue}{二向色性}\textcolor{Plum}{均匀}\textcolor{Plum}{各向异性}电介质中,\textcolor{PineGreen}{光场电矢量}在 \textcolor{gray}{2D 倒} $\mathcolor{gray}{\bar{k}_{\symup{\rho}}}$ \textcolor{gray}{空间}中的\textcolor{Plum}{复}\textcolor{PineGreen}{本征模}(\textcolor{Plum}{理论部分}:\bref{sec:eigen-analysis};\textcolor{NavyBlue}{数值实验}:\bref{ssec:2D-reciprocal-eigensystems}),及其所导致的\textcolor{PineGreen}{光场电矢量}在 \textcolor{gray}{3D 正} $\mathcolor{gray}{\bar{r}}$ \textcolor{gray}{空间}中的\textcolor{Plum}{各向异性}衍射行为(\textcolor{Plum}{理论部分}:\bref{ssec:3times2sandwich-eigen-matrices};\textcolor{NavyBlue}{数值实验}:\bref{ssec:3D-real-propagation}),构成了\textcolor{Plum}{均匀}纯电\textcolor{Plum}{各向异性}材料中所有\textcolor{Maroon}{光与物质相互作用}的高级研究(包括后续的\textcolor{Plum}{非线性}\textcolor{PineGreen}{晶体光学})的基本基础。在为解决这一问题而创建的所有\textcolor{Plum}{数学}/\textcolor{NavyBlue}{物理模型}中,\textcolor{PineGreen}{Jordan-Chevalley 分解}后\textcolor{Maroon}{矩阵指数}解 $\mathbb{e}^{\mathbb{i} \Xint{\begin{smallmatrix} ~ \\ {}^{}_{\mathcolor{gray}{-}} \\ ~ \end{smallmatrix}}{15}{\bar{\bar{k}}}_{\textcolor{Maroon}{\symup{z}}}^{\;\! \mathcolor{gray}{\omega}} \mathcolor{gray}{z}}$ 被认为是破译具有\textcolor{Plum}{二阶或更高阶简并}性的\textcolor{PineGreen}{光学奇点}\textcolor{Plum}{内部结构}的最有可能的关键。然而,这种方法的\textcolor{Plum}{数值实现是不稳定的},无法管理更\textcolor{Plum}{厚}的\textcolor{Plum}{非厄米}材料。

相反,允许处理\textcolor{Plum}{厚非厄米板}的\textcolor{Plum}{数值稳定}的\textcolor{PineGreen}{经典平面波}模型 $\mathbb{e}^{\mathbb{i} \Xint{\begin{smallmatrix} ~ \\ {}^{}_{\mathcolor{gray}{-}} \\ ~ \end{smallmatrix}}{15}{\bar{k}}^{\;\! \mathcolor{gray}{\omega}}_{\textcolor{PineGreen}{\jmath}} \cdot \textcolor{gray}{\bar{r}}}$,却完全\textcolor{Plum}{无法解析}\textcolor{PineGreen}{光学奇点}。除此之外,它在其他方面没有劣势。鉴于此,我们发展了一套\textcolor{Plum}{非均匀}\textcolor{PineGreen}{平面波基}下的\textcolor{Maroon}{傅立叶}\textcolor{PineGreen}{晶体光学}模型\cite{xieAnalytic3DVector},推导出其在板状电介质内任意两横截面之间的 $\textcolor{Plum}{3 \times 2}$ \textcolor{Maroon}{转移矩阵场}的\textcolor{Plum}{显式形式},如 \bref{eq:transition_matrix-transverse_input,fig:sandwich-eigen-matrices} 所示。

接下来,我们在 \bref{fig:MMTC-2D_eigensystems} 中提出了\textcolor{NavyBlue}{材料-矩阵正四面体罗盘},沿其\textcolor{Plum}{三条边}进行\textcolor{Plum}{参数化扫描},描绘了二维 $\mathcolor{gray}{\bar{k}_{\symup{\rho}}}$ 域内电场\textcolor{PineGreen}{本征模}\textcolor{NavyBlue}{绝热演化}的\textcolor{NavyBlue}{理论全景},在三种主要材料性质,即\textcolor{NavyBlue}{线性二色性(LD)}、\textcolor{NavyBlue}{圆二向色性(CD)}和\textcolor{NavyBlue}{光学活性(OA)}之间展开\textcolor{Plum}{两两竞争}。在这个过程中,我们观察到,\textcolor{NavyBlue}{圆二向色性}也遵循\textcolor{Maroon}{扩展的萦绕定理}\cite{berryOpticalSingularitiesBirefringent2003},并且可以在二维 $\mathcolor{gray}{\bar{k}_{\symup{\rho}}}$ 域中产生\textcolor{Plum}{心形}的 \textcolor{PineGreen}{L 海岸线}和无限\textcolor{PineGreen}{光学奇点}/\textcolor{PineGreen}{例外点}阵列,它们通常排列成类似\textcolor{Plum}{圆盘}、\textcolor{Plum}{环}\cite{kirillovUnfoldingEigenvalueSurfaces2005} 或\textcolor{Plum}{新月形}图案。

熟悉了 \textcolor{gray}{2D 倒} $\mathcolor{gray}{\bar{k}_{\symup{\rho}}}$ \textcolor{gray}{空间}中的\textcolor{PineGreen}{本征系统}(即\textcolor{PineGreen}{本征值-向量对})后,便允许计算 \textcolor{gray}{3D 正} $\mathcolor{gray}{\bar{r}}$ \textcolor{gray}{空间}中的\textcolor{NavyBlue}{场分布和演化}的动力学过程。在 \bref{fig:Peet-3D_propagation} 提出的\textcolor{Maroon}{晶体-2f 系统}的帮助下,扫描\textcolor{Plum}{传播距离} $\mathcolor{gray}{z}$ 和离\textcolor{PineGreen}{光轴}角 $\theta$,获得了沿\textcolor{PineGreen}{光轴}/\textcolor{PineGreen}{恶魔点}的\textcolor{PineGreen}{双折射},到离\textcolor{PineGreen}{光轴}的\textcolor{PineGreen}{锥折射}的整个\textcolor{NavyBlue}{演化}过程,以及二者各自从晶体\textcolor{Plum}{前端面} $\longrightarrow$ \textcolor{Plum}{后端面} $\longrightarrow$ \textcolor{Plum}{远场}(\textcolor{Plum}{后端面}的\textcolor{Plum}{傅立叶面})的\textcolor{NavyBlue}{传播过程},完成了对 \textcolor{Maroon}{Peet} 实验结果的\textcolor{Plum}{解析延拓}。

同样借助\textcolor{Maroon}{晶体-2f 系统},接续在三维 $\mathcolor{gray}{\bar{r}}$ 域中,绘制了第二幅相对 \bref{fig:MMTC-2D_eigensystems} 更\textcolor{NavyBlue}{实验相关}的全景 \bref{fig:MMTC-3D_propagation},冻结了\textcolor{NavyBlue}{材料-矩阵正四面体罗盘}\textcolor{Plum}{三个顶点}之间光场分布的\textcolor{NavyBlue}{绝热演化},对应于材料的\textcolor{NavyBlue}{双折射性(Bi)}、\textcolor{NavyBlue}{线二向色性(LD)}和\textcolor{NavyBlue}{光学活性(OA)}。

此外,在 \bref{fig:high_N.A.} 中,该\textcolor{Plum}{线性}\textcolor{PineGreen}{晶体光学}模型,还为高\textcolor{Plum}{数值孔径 N.A.}下焦场的\textcolor{Plum}{正向传播}和\textcolor{Plum}{逆向设计}(\bref{fig:high_N.A.}\textbf{c1})提供了一个统一的解决方案。在此过程中,观察到了两个令人惊讶的新现象:\textcolor{PineGreen}{双圆锥折射}(\bref{fig:high_N.A.}\textbf{d1}左侧两列)和由此产生的\textcolor{PineGreen}{光场结}(\bref{fig:high_N.A.}\textbf{d1}最右列)。通过\textcolor{PineGreen}{特征分解},我们还发现\textcolor{PineGreen}{锥折射}的\textcolor{PineGreen}{拉曼尖峰}(\textcolor{PineGreen}{Raman spike})仅起源于\textcolor{PineGreen}{慢模}(\bref{fig:high_N.A.}\textbf{b});并揭示了在不进行\textcolor{Maroon}{像差校正}的情况下,光场在铌酸锂(LN)内部\textcolor{PineGreen}{紧聚焦}过程中\textcolor{Plum}{横向}\textcolor{NavyBlue}{旋轨耦合}和轴向多焦点的本质,来源于相位失配的 \textcolor{PineGreen}{o 波}和 \textcolor{PineGreen}{e 波}之间的\textcolor{PineGreen}{干涉}(\bref{fig:high_N.A.}\textbf{c4})。为解决聚焦激光加工各向异性材料内部深处时,像差导致的纵向场分裂和横向光斑畸变问题,我们提出了无需\textcolor{Maroon}{泽尼克多项式}补偿的“\textcolor{PineGreen}{本征模}\textcolor{Plum}{反向传播}”法(\bref{fig:high_N.A.}\textbf{c1})。

最后,在 \bref{ssec:LFCO-Superstructure},提出了扩展该\textcolor{Plum}{线性}\textcolor{PineGreen}{晶体光学}模型的路线 \bref{fig:step1-LFCO},通过 \bref{fig:conical_phase_match} 中的\textcolor{PineGreen}{全圆锥相位匹配}和 \bref{fig:VNFCO_stepstones} 中的\textcolor{NavyBlue}{手性}二次谐波\textcolor{PineGreen}{圆锥折射},初步展示了其在矢量\textcolor{Plum}{非线性}\textcolor{PineGreen}{晶体光学}中的应用。

\begin{figure}[htbp!]
	\centering
	\includegraphics[width=1.0\textwidth]{D:/C2D/Desktop/article_fig/phd_thesis_fig/chapter-03/step1-LFCO.pdf}
	\biackcaption[\textbf{Vector linear Fourier crystal optics(LFCO): laying a solid foundation for (scalar/vector) nonlinear Fourier crystal optics(NFCO).}]{-0.7em}{\textbf{矢量\textcolor{Plum}{线性}\textcolor{Maroon}{傅里叶}\textcolor{PineGreen}{晶体光学}:为(标/矢量)\textcolor{Plum}{非线性}\textcolor{Maroon}{傅里叶}\textcolor{PineGreen}{晶体光学}打下坚实基础。}\\}{fig:step1-LFCO}
\end{figure}





