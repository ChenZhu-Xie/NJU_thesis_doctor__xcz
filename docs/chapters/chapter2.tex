% based on 北理工博士毕业论文 的 chapter2 https://tex.nju.edu.cn/project/user/10d0ed00-6f46-42fa-8a3a-18270214cfa3/a0ed1339-764d-4cc7-9ac4-d5d05b33da53

\chapter{晶体中的(非)线性光学过程}

%\chapter{\protect\hyperlink{chap:\thechapter}{晶体中的线性、非线性光学过程的解析解}}
%\addtocontents{toc}{\protect\linkdest{chap:\thechapter}}
%\label{晶体中的线性、非线性光学过程的解析解}
%
%%\section{晶体中的电场混频方程组}
%\section{\protect\hyperlink{chap:\thesection}{晶体中的电场混频方程组}}
%\addtocontents{toc}{\protect\linkdest{chap:\thesection}}
%\label{晶体中的电场混频方程组}

相对观察者静止\Footnote{否则 $\bar{E}^{\;\!\textcolor{gray}{t}}_{\;\!\textcolor{gray}{z}},\bar{D}^{\;\!\textcolor{gray}{t}}_{\;\!\textcolor{gray}{z}}$ 与 $\bar{B}^{\;\!\textcolor{gray}{t}}_{\;\!\textcolor{gray}{z}}$ 将进一步相互耦合\cite{berryOpticalSingularitiesBianisotropic2005,chen-zhuChenZhuxieUndergraduate_courses2024},以致材料的本构关系将呈现固有双各向异性\cite{mackayElectromagneticAnisotropyBianisotropy2019,mackayModernAnalyticalElectromagnetic2020,lakhtakiaCovariancesInvariancesMaxwell1995};$\bar{J}^{\;\!\textcolor{gray}{t}}_{\;\!\textcolor{Maroon}{\symup{e}}\textcolor{gray}{z}}$ 也需要扩展到四维\cite{lakhtakiaCovariancesInvariancesMaxwell1995,chen-zhuChenZhuxieUndergraduate_courses2024}。此外,束缚电荷$- \bar{\nabla} \cdot \bar{P}^{\;\!\textcolor{gray}{t}}_{\;\!\textcolor{gray}{z}}$\cite{mackayElectromagneticAnisotropyBianisotropy2019,chen-zhuChenZhuxieUndergraduate_courses2024}、自由电子${\rho}^{\;\!\textcolor{gray}{t}}_{\;\!\textcolor{Maroon}{\symup{e}}\textcolor{gray}{z}}$的(有效)运动质量(或相对论速度)也会变大,并因此同时对线性和非线性极化$\bar{P}^{\;\!\textcolor{gray}{t}}_{\;\!\textcolor{gray}{z}}$、自由电流$\bar{J}^{\;\!\textcolor{gray}{t}}_{\;\!\textcolor{Maroon}{\symup{e}}\textcolor{gray}{z}} = {\rho}^{\;\!\textcolor{gray}{t}}_{\;\!\textcolor{Maroon}{\symup{e}}\textcolor{gray}{z}} \bar{v}^{\;\!\textcolor{gray}{t}}_{\;\!\textcolor{Maroon}{\symup{e}}\textcolor{gray}{z}}$产生额外影响。——上述场景可能发生在超快和强场非线性光学中\cite{boydNonlinearOptics2019}。} 的三维空间$\textcolor{gray}{\bar{r}}$ 坐标系下,在可能存在非零的自由电荷源和自由电流源${\rho}_{\;\!\textcolor{Maroon}{\symup{e}}} \left( \textcolor{gray}{\bar{r}}, \textcolor{gray}{t} \right), \bar{J}_{\;\!\textcolor{Maroon}{\symup{e}}} \left( \textcolor{gray}{\bar{r}}, \textcolor{gray}{t} \right)$\Footnote{对于符号约定,比如下标 $\textcolor{Maroon}{\symup{e}}$ 的\textcolor{Maroon}{褐红色}及其含义 `\textcolor{Maroon}{electricity}',其定义见\bref{Maroon};此外,在 \bref{1bar} 中还约定:总使用 1 条上短横线 $\bar{~}$(而不是粗体)来表示矢量(如 $\bar{J}_{\;\!\textcolor{Maroon}{\symup{e}}}$),以区别于 \bref{0bar} 中定义的无上短横线的标量(如 ${\rho}_{\;\!\textcolor{Maroon}{\symup{e}}}$);粗体在本文中另有其含义,不用于表示矢量,见。} 的一般电磁介质内部,4 个空域时变\Footnote{指复矢量场 $\bar{E}^{\;\!\textcolor{gray}{t}}_{\;\!\textcolor{gray}{z}}, \bar{H}^{\;\!\textcolor{gray}{t}}_{\;\!\textcolor{gray}{z}}, \bar{D}^{\;\!\textcolor{gray}{t}}_{\;\!\textcolor{gray}{z}}, \bar{B}^{\;\!\textcolor{gray}{t}}_{\;\!\textcolor{gray}{z}}$ 均是四维时空 $\textcolor{gray}{\bar{r}}, \textcolor{gray}{t}$ 的函数,且因此均是复色 $\left\{ \omega \in \mathbbm{R} \right\}$ 的复场 $\in \mathbbm{C}^3 \left( \mathbbm{R}^3 \right)$,属于在视觉上占主导的因变量,并用黑色(见 \bref{black})的 \textit{斜体 oblique}(见 \bref{oblique})表示;相对地,自变量用视觉和含义上均更次要的\textcolor{gray}{灰色}来表示,见 \bref{gray}。}复色场 
$\bar{E}^{\;\!\textcolor{gray}{t}}_{\;\!\textcolor{gray}{z}}, \bar{H}^{\;\!\textcolor{gray}{t}}_{\;\!\textcolor{gray}{z}}, \bar{D}^{\;\!\textcolor{gray}{t}}_{\;\!\textcolor{gray}{z}}, \bar{B}^{\;\!\textcolor{gray}{t}}_{\;\!\textcolor{gray}{z}}$\Footnote{由于傅立叶光学一般运行在平行平面间,约定下述表示相互等价:$\bar{E} \left( \textcolor{gray}{\bar{r}}, \textcolor{gray}{t} \right) = \bar{E}^{\;\!\textcolor{gray}{t}}_{\;\!\textcolor{gray}{\bar{r}}} = \bar{E}^{\;\!\textcolor{gray}{t}}_{\;\!\textcolor{gray}{z}} \left( \textcolor{gray}{\bar{\rho}} \right)$,并因此经常省略面内自变量 $\textcolor{gray}{\bar{\rho}}$,以只写作朝 $\textcolor{Maroon}{+\symup{z}}$ 轴传播距离 $\textcolor{gray}{z}$ 的函数 $\bar{E}^{\;\!\textcolor{gray}{t}}_{\;\!\textcolor{gray}{z}} := \bar{E}^{\;\!\textcolor{gray}{t}}_{\;\!\textcolor{gray}{z}} \left( \textcolor{gray}{\bar{\rho}} \right)$。—— 同样的规则也适用于其他场量(如 ${\rho}_{\;\!\textcolor{Maroon}{\symup{e}}} \left( \textcolor{gray}{\bar{r}}, \textcolor{gray}{t} \right) \to {\rho}^{\;\!\textcolor{gray}{t}}_{\;\!\textcolor{Maroon}{\symup{e}}\textcolor{gray}{z}}$),且适用于空间频率域,见。},满足微分形式\Footnote{尽管是微分形式,仍然处于(相对的)宏观层面:典型的光波长 $1$um 是原子特征尺寸 $1\text{\r{A}} = 0.1$nm 的 $10^4$ 倍,因此\bref{eq:maxwell-eh,eq:maxwell-db,eq:continuity-pj}涉及的所有物理量均是空间平均后的结果\cite{mackayElectromagneticAnisotropyBianisotropy2019}。}的麦氏方程组的 2 个旋度假设:
\begin{subequations} \label{eq:maxwell-eh}
\begin{align}
	\textcolor{Maroon}{\text{Faraday's law of electromagnetic induction}}\text{:}&\hspace{0.5em} \bar{\nabla} \times \bar{E}^{\;\!\textcolor{gray}{t}}_{\;\!\textcolor{gray}{z}} \hspace{-1.2em} &&= - \bar{J}^{\;\!\textcolor{gray}{t}}_{\;\!\textcolor{Maroon}{\symup{m}}\textcolor{gray}{z}} - \frac{\partial \bar{B}^{\;\!\textcolor{gray}{t}}_{\;\!\textcolor{gray}{z}}}{\partial t}~, \label{eq:maxwell-e} \\ \textcolor{Maroon}{\text{Jefimenko's}} \to \textcolor{Maroon}{\text{Maxwell-Amp\`{e}re circuital law}}\text{:}&\hspace{0.5em} \bar{\nabla} \times \bar{H}^{\;\!\textcolor{gray}{t}}_{\;\!\textcolor{gray}{z}} \hspace{-1.2em} &&= \bar{J}^{\;\!\textcolor{gray}{t}}_{\;\!\textcolor{Maroon}{\symup{e}}\textcolor{gray}{z}} + \frac{\partial \bar{D}^{\;\!\textcolor{gray}{t}}_{\;\!\textcolor{gray}{z}}}{\partial t}~, \label{eq:maxwell-h}
\end{align}
\end{subequations}
以及 2 个散度假设:
\begin{subequations} \label{eq:maxwell-db}
\begin{align}
	\textcolor{Maroon}{\text{Coulomb's}} \to \textcolor{Maroon}{\text{Gauss's law for electric field}}\text{:}&\hspace{0.5em} \bar{\nabla} \cdot \bar{D}^{\;\!\textcolor{gray}{t}}_{\;\!\textcolor{gray}{z}} \hspace{-3.2em} &&= {\rho}^{\;\!\textcolor{gray}{t}}_{\;\!\textcolor{Maroon}{\symup{e}}\textcolor{gray}{z}}~, \label{eq:maxwell-d} \\ \textcolor{Maroon}{\text{Biot-Savart}} \to \textcolor{Maroon}{\text{Gauss's law for magnetic field}}\text{:}&\hspace{0.5em} \bar{\nabla} \cdot \bar{B}^{\;\!\textcolor{gray}{t}}_{\;\!\textcolor{gray}{z}} \hspace{-3.2em} &&= {\rho}^{\;\!\textcolor{gray}{t}}_{\;\!\textcolor{Maroon}{\symup{m}}\textcolor{gray}{z}}~. \label{eq:maxwell-b}
\end{align}
\end{subequations}
其中,为数学形式上的对称(以方便引入狭义相对论效应和检验其洛伦兹协变性),和物理上不排除可能存在的磁单极子,除自由电(荷/流)源${\rho}^{\;\!\textcolor{gray}{t}}_{\;\!\textcolor{Maroon}{\symup{e}}\textcolor{gray}{z}}, \bar{J}^{\;\!\textcolor{gray}{t}}_{\;\!\textcolor{Maroon}{\symup{e}}\textcolor{gray}{z}}$外,还添加了自由磁源${\rho}^{\;\!\textcolor{gray}{t}}_{\;\!\textcolor{Maroon}{\symup{m}}\textcolor{gray}{z}}, \bar{J}^{\;\!\textcolor{gray}{t}}_{\;\!\textcolor{Maroon}{\symup{m}}\textcolor{gray}{z}}$\cite{lakhtakiaCovariancesInvariancesMaxwell1995},其满足2个连续性假设\cite{mackayElectromagneticAnisotropyBianisotropy2019,lakhtakiaCovariancesInvariancesMaxwell1995,chen-zhuChenZhuxieUndergraduate_courses2024}:
\begin{subequations} \label{eq:continuity-pj}
\begin{align}
	\textcolor{Maroon}{\text{Continuity for electric free source}}\text{:}&\hspace{0.5em} \bar{\nabla} \cdot \bar{J}^{\;\!\textcolor{gray}{t}}_{\;\!\textcolor{Maroon}{\symup{e}}\textcolor{gray}{z}} + \frac{\partial {\rho}^{\;\!\textcolor{gray}{t}}_{\;\!\textcolor{Maroon}{\symup{e}}\textcolor{gray}{z}}}{\partial t} \hspace{-5.2em} &&= 0~, \label{eq:continuity-e} \\ \textcolor{Maroon}{\text{Continuity for magnetic free source}}\text{:}&\hspace{0.5em} \bar{\nabla} \cdot \bar{J}^{\;\!\textcolor{gray}{t}}_{\;\!\textcolor{Maroon}{\symup{m}}\textcolor{gray}{z}} + \frac{\partial {\rho}^{\;\!\textcolor{gray}{t}}_{\;\!\textcolor{Maroon}{\symup{m}}\textcolor{gray}{z}}}{\partial t} \hspace{-5.2em} &&= 0~. \label{eq:continuity-m}
\end{align}
\end{subequations}
注,对旋度 \bref{eq:maxwell-eh} 两边取散度($\bar{\nabla} \cdot$),连续性 \bref{eq:continuity-pj} 可导出散度 \bref{eq:maxwell-db}。因此,2 条散度方程均不是必需的,可将其视为冗余。

接着,分别给磁感应场$\bar{B}^{\;\!\textcolor{gray}{t}}_{\;\!\textcolor{gray}{z}}$,自由电流源$\bar{J}^{\;\!\textcolor{gray}{t}}_{\;\!\textcolor{Maroon}{\symup{e}}\textcolor{gray}{z}}$ 以及电位移场$\bar{D}^{\;\!\textcolor{gray}{t}}_{\;\!\textcolor{gray}{z}}$,定义了如下 3 个本构关系(\textcolor{Maroon}{\text{Constitutive Relation}} = \textcolor{Maroon}{\text{CR}})。

其一,构成磁感应场 $\bar{B}^{\;\!\textcolor{gray}{t}}_{\;\!\textcolor{gray}{z}}$ 的本构关系,定义为
\begin{subequations} \label{eq:cr-b}
\begin{align}
	\textcolor{Maroon}{\text{CR for magnetism}}\text{:}&\hspace{0.5em} \bar{B}^{\;\!\textcolor{gray}{t}}_{\;\!\textcolor{gray}{z}} \hspace{-0.7em} &&={\symup{\mu}}_0 \left( \bar{H}^{\;\!\textcolor{gray}{t}}_{\;\!\textcolor{gray}{z}} + \bar{M}^{\;\!\textcolor{gray}{t}}_{\;\!\textcolor{gray}{z}} \right) = {\symup{\mu}}_0 \left\{ \bar{\bar{\delta}}^{\;\!\textcolor{gray}{t}}~\widetilde *~\bar{H}^{\;\!\textcolor{gray}{t}}_{\;\!\textcolor{gray}{z}} + \bar{M}^{\;\!\textcolor{gray}{t}}_{\;\!\textcolor{gray}{z}} \right\} \label{cr-b1} \\ & &&\xrightarrow[]{\bar{M}^{\;\!\textcolor{gray}{t}}_{\;\!\textcolor{gray}{z}} = \bar{M}^{\;\!\textcolor{Maroon}{\text{(1)}} \textcolor{gray}{t}}_{\;\!\textcolor{gray}{z}} + \bar{M}^{\;\!\textcolor{Maroon}{\text{NL}}, \textcolor{gray}{t}}_{\;\!\textcolor{gray}{z}}} {\symup{\mu}}_0 \left\{ \left[ \bar{\bar{\delta}}^{\;\!\textcolor{gray}{t}}~\widetilde *~\bar{H}^{\;\!\textcolor{gray}{t}}_{\;\!\textcolor{gray}{z}} + \bar{M}^{\;\!\textcolor{Maroon}{\text{(1)}} \textcolor{gray}{t}}_{\;\!\textcolor{gray}{z}} \right] + \bar{M}^{\;\!\textcolor{Maroon}{\text{NL}}, \textcolor{gray}{t}}_{\;\!\textcolor{gray}{z}} \right\} \label{cr-b2} \\ & &&\xrightarrow[\displaystyle{ \bar{\bar{\mu}}^{\;\!\textcolor{Maroon}{\text{(1)}} \textcolor{gray}{t}}_{\;\!\textcolor{Maroon}{\text{r}}\textcolor{gray}{z}} := \bar{\bar{\delta}}^{\;\!\textcolor{gray}{t}} + \bar{\bar{\chi}}^{\;\!\textcolor{Maroon}{\text{(1)}}\textcolor{gray}{t}}_{\;\!\textcolor{Maroon}{\text{m}} \textcolor{gray}{z}}}]{\displaystyle{\bar{M}^{\;\!\textcolor{Maroon}{\text{(1)}} \textcolor{gray}{t}}_{\;\!\textcolor{gray}{z}} := \bar{\bar{\chi}}^{\;\!\textcolor{Maroon}{\text{(1)}}\textcolor{gray}{t}}_{\;\!\textcolor{Maroon}{\text{m}} \textcolor{gray}{z}} ~\widetilde *~\bar{H}^{\;\!\textcolor{gray}{t}}_{\;\!\textcolor{gray}{z}}}} {\symup{\mu}}_0 \left\{ \bar{\bar{\mu}}^{\;\!\textcolor{Maroon}{\text{(1)}} \textcolor{gray}{t}}_{\;\!\textcolor{Maroon}{\text{r}}\textcolor{gray}{z}}~\widetilde *~\bar{H}^{\;\!\textcolor{gray}{t}}_{\;\!\textcolor{gray}{z}} + \bar{M}^{\;\!\textcolor{Maroon}{\text{NL}}, \textcolor{gray}{t}}_{\;\!\textcolor{gray}{z}} \right\} \label{cr-b3} \\ & &&= \bar{\bar{\mu}}^{\;\!\textcolor{Maroon}{\text{(1)}} \textcolor{gray}{t}}_{\;\!\textcolor{gray}{z}}~\widetilde *~\bar{H}^{\;\!\textcolor{gray}{t}}_{\;\!\textcolor{gray}{z}} + {\symup{\mu}}_0 \bar{M}^{\;\!\textcolor{Maroon}{\text{NL}}, \textcolor{gray}{t}}_{\;\!\textcolor{gray}{z}} =: \bar{B}^{\;\!\textcolor{Maroon}{\text{(1)}} \textcolor{gray}{t}}_{\;\!\textcolor{gray}{z}} + \bar{B}^{\;\!\textcolor{Maroon}{\text{NL}}, \textcolor{gray}{t}}_{\;\!\textcolor{gray}{z}}~, \label{cr-b4}
\end{align}
\end{subequations}
其中,磁通量密度场 $\bar{B}^{\;\!\textcolor{gray}{t}}_{\;\!\textcolor{gray}{z}}$(直接/显示地)关于磁场 $\bar{H}^{\;\!\textcolor{gray}{t}}_{\;\!\textcolor{gray}{z}}$\Footnote{磁非线性,如在郎之万顺磁性理论\cite{chen-zhuChenZhuxieUndergraduate_courses2024}中 $M \propto$ 郎之万函数 $\mathcal{L} \left( \alpha \right) = \coth \left( \alpha \right) - 1 / \alpha$(其中 $\alpha \propto H_{\textcolor{Maroon}{\text{ex}}}$)或其量子化修正之布里渊函数,铁磁体\cite{chen-zhuChenZhuxieUndergraduate_courses2024}或超导体\cite{wenBriefIntroductionFlux2021}中的磁滞现象等(每个时刻$t$,这些场量都是准静态$\Omega \to 0$的)。}、电场 $\bar{E}^{\;\!\textcolor{gray}{t}}_{\;\!\textcolor{gray}{z}}$\Footnote{双各向异性,其中的电$\to$磁耦合(如果 $\bar{B}^{\;\!\textcolor{gray}{t}}_{\;\!\textcolor{gray}{z}}$ 中的该部分只是 $\bar{E}^{\;\!\textcolor{gray}{t}}_{\;\!\textcolor{gray}{z}}$ 的线性函数,则也可归结到线性项中)。}、应力 $\bar{T}^{\;\!\textcolor{gray}{t}}_{\;\!\textcolor{gray}{z}}$\Footnote{正逆压磁/磁致伸缩/磁弹效应(这里未作区分)。}等其他场量(即含空 $\textcolor{gray}{\bar{r}}$ 的物理量)\Footnote{$\bar{B}^{\;\!\textcolor{gray}{t}}_{\;\!\textcolor{gray}{z}},\bar{M}^{\;\!\textcolor{gray}{t}}_{\;\!\textcolor{gray}{z}}$ 以及各阶 $\bar{\bar{\mu}}^{\;\!\textcolor{Maroon}{\text{(1)}} \textcolor{gray}{t}}_{\;\!\textcolor{gray}{z}},\bar{\bar{\bar{\mu}}}^{\;\!\textcolor{Maroon}{\text{(2)}} \textcolor{gray}{t}}_{\;\!\textcolor{gray}{z}},\cdots$ 已经是关于温度$T$、波长$\lambda$(或 角频率$\omega$、时间$t$)等(非)场量的函数。}的非线性函数项,悉数包含在 $\bar{B}^{\;\!\textcolor{Maroon}{\text{NL}}, \textcolor{gray}{t}}_{\;\!\textcolor{gray}{z}} = {\symup{\mu}}_0 \bar{M}^{\;\!\textcolor{Maroon}{\text{NL}}, \textcolor{gray}{t}}_{\;\!\textcolor{gray}{z}}$ 内;剩余的线性项,放在 $\bar{B}^{\;\!\textcolor{Maroon}{\text{(1)}} \textcolor{gray}{t}}_{\;\!\textcolor{gray}{z}} = \bar{\bar{\mu}}^{\;\!\textcolor{Maroon}{\text{(1)}} \textcolor{gray}{t}}_{\;\!\textcolor{gray}{z}}~\widetilde *~\bar{H}^{\;\!\textcolor{gray}{t}}_{\;\!\textcolor{gray}{z}}$ 中。

其二,自由电流源$\bar{J}^{\;\!\textcolor{gray}{t}}_{\;\!\textcolor{Maroon}{\symup{e}}\textcolor{gray}{z}}$的组成成分,由欧姆定律的线性部分(漂移项) $\bar{J}^{\;\!\textcolor{Maroon}{\text{(1)}} \textcolor{gray}{t}}_{\;\!\textcolor{Maroon}{\symup{e}}\textcolor{gray}{z}} = \bar{\bar{\sigma}}^{\;\!\textcolor{Maroon}{\text{(1)}}\textcolor{gray}{t}}_{\;\!\textcolor{gray}{z}}~\widetilde *~\bar{E}^{\;\!\textcolor{gray}{t}}_{\;\!\textcolor{gray}{z}}$\Footnote{可以由 Drude 模型描述,定量解释一阶电导率$\bar{\bar{\sigma}}^{\;\!\textcolor{Maroon}{\text{(1)}}\textcolor{gray}{t}}_{\;\!\textcolor{gray}{z}}$的起源。},及$\bar{J}^{\;\!\textcolor{gray}{t}}_{\;\!\textcolor{Maroon}{\symup{e}}\textcolor{gray}{z}}$分别关于电场 $\bar{E}^{\;\!\textcolor{gray}{t}}_{\;\!\textcolor{gray}{z}}$\Footnote{欧姆定律中的电非线性部分,比如二/三极管的伏安特性曲线\cite{chen-zhuChenZhuxieUndergraduate_courses2024}(尽管输入/输出or自/因变量,即$\bar{E}^{\;\!\textcolor{gray}{t}}_{\;\!\textcolor{gray}{z}}$和$\bar{J}^{\;\!\textcolor{gray}{t}}_{\;\!\textcolor{Maroon}{\symup{e}}\textcolor{gray}{z}}$,一般均在直流或低频$\Omega$,非交流且不在光波段 opt)。}、磁感应场$\bar{B}^{\;\!\textcolor{gray}{t}}_{\;\!\textcolor{gray}{z}}$\Footnote{磁场所带来的(库伦力以外的)洛伦兹力$\left( \bar{J}^{\;\!\textcolor{gray}{t}}_{\;\!\textcolor{Maroon}{\symup{e}}\textcolor{gray}{z}} + \dot{\bar{P}}^{\;\!\textcolor{gray}{t}}_{\;\!\textcolor{gray}{z}} + \bar{\nabla} \times \bar{M}^{\;\!\textcolor{gray}{t}}_{\;\!\textcolor{gray}{z}} \right) \times \bar{B}^{\;\!\textcolor{gray}{t}}_{\;\!\textcolor{gray}{z}}$\cite{mackayElectromagneticAnisotropyBianisotropy2019,chen-zhuChenZhuxieUndergraduate_courses2024},会影响导/价带电子的运动(速度)$\bar{v}^{\;\!\textcolor{gray}{t}}_{\;\!\textcolor{Maroon}{\symup{e}}\textcolor{gray}{z}}$,进而全局地影响自由电流$\bar{J}^{\;\!\textcolor{gray}{t}}_{\;\!\textcolor{Maroon}{\symup{e}}\textcolor{gray}{z}} = {\rho}^{\;\!\textcolor{gray}{t}}_{\;\!\textcolor{Maroon}{\symup{e}}\textcolor{gray}{z}} \bar{v}^{\;\!\textcolor{gray}{t}}_{\;\!\textcolor{Maroon}{\symup{e}}\textcolor{gray}{z}}$和(束缚)电(偶)极化强度$\bar{P}^{\;\!\textcolor{gray}{t}}_{\;\!\textcolor{gray}{z}}$,包括它们的线性和非线性项\cite{boydNonlinearOptics2019}。对于强场/超快非线性光学,相对论效应使得电磁场是个统一的整体,动生(而不仅是外加)的$\bar{B}^{\;\!\textcolor{gray}{t}}_{\;\!\textcolor{gray}{z}}$还将带来额外的影响。}、导带电子浓度(数密度)梯度场$\bar{\nabla} {\rho}^{\;\!\textcolor{gray}{t}}_{\;\!\textcolor{Maroon}{\symup{e}}\textcolor{gray}{z}}$\Footnote{在光折变效应中,作为$\bar{J}^{\;\!\textcolor{gray}{t}}_{\;\!\textcolor{Maroon}{\symup{e}}\textcolor{gray}{z}}$中的扩散项\cite{boydNonlinearOptics2019}。${\rho}^{\;\!\textcolor{gray}{t}}_{\;\!\textcolor{Maroon}{\symup{e}}\textcolor{gray}{z}}, \bar{J}^{\;\!\textcolor{gray}{t}}_{\;\!\textcolor{Maroon}{\symup{e}}\textcolor{gray}{z}}$之间还应满足\bref{eq:continuity-e}以及$\bar{J}^{\;\!\textcolor{gray}{t}}_{\;\!\textcolor{Maroon}{\symup{e}}\textcolor{gray}{z}} = {\rho}^{\;\!\textcolor{gray}{t}}_{\;\!\textcolor{Maroon}{\symup{e}}\textcolor{gray}{z}} \bar{v}^{\;\!\textcolor{gray}{t}}_{\;\!\textcolor{Maroon}{\symup{e}}\textcolor{gray}{z}}$\cite{chen-zhuChenZhuxieUndergraduate_courses2024}。}和光伏电流场$\propto \lvert \bar{E}^{\;\!\textcolor{gray}{t}}_{\;\!\textcolor{gray}{z}} \rvert^2 \hat{c}$\Footnote{与光电导效应并列,属于内光电效应;也可能在光折变效应的$\bar{J}^{\;\!\textcolor{gray}{t}}_{\;\!\textcolor{Maroon}{\symup{e}}\textcolor{gray}{z}}$中扮演一份角色,特别是沿着一些各向异性晶体的光轴$\hat{c}$产生电势差和内建电场\cite{boydNonlinearOptics2019}(尽管一般也只影响直流或低频$\Omega$的$\bar{J}^{\;\!\textcolor{gray}{t}}_{\;\!\textcolor{Maroon}{\symup{e}}\textcolor{gray}{z}}$;但$\bar{J}^{\;\!\textcolor{gray}{t}}_{\;\!\textcolor{Maroon}{\symup{e}}\textcolor{gray}{z}}$会通过影响光波段的介电常数,进而影响光波段的光强$\lvert \bar{E}^{\;\!\textcolor{gray}{t}}_{\;\!\textcolor{gray}{z}} \rvert^2$及$\bar{J}^{\;\!\textcolor{gray}{t}}_{\;\!\textcolor{Maroon}{\symup{e}}\textcolor{gray}{z}}$自己的重新分布);该二阶的带耦合的非线性,看上去很像非线性极化率$\bar{P}^{\;\!\textcolor{Maroon}{\text{(2)}} \textcolor{gray}{t}}_{\;\!\textcolor{gray}{z}}$中的光整流项,但其频率比 THz 低,且只服务于自由电流。—— 以至该项可作为差频合并至$\bar{J}^{\;\!\textcolor{gray}{t}}_{\;\!\textcolor{Maroon}{\symup{e}}\textcolor{gray}{z}}$关于$\bar{E}^{\;\!\textcolor{gray}{t}}_{\;\!\textcolor{gray}{z}}$的二阶非线性$\bar{J}^{\;\!\textcolor{Maroon}{\text{(2)}} \textcolor{gray}{t}}_{\;\!\textcolor{Maroon}{\symup{e}}\textcolor{gray}{z}}$中去?}等其他场量的非线性项 $\bar{J}^{\;\!\textcolor{Maroon}{\text{NL}}, \textcolor{gray}{t}}_{\;\!\textcolor{Maroon}{\symup{e}}\textcolor{gray}{z}}$ 构成:
\begin{equation} \label{eq:cr-j}
	\textcolor{Maroon}{\text{Ohm's law}}\text{:}\hspace{0.5em} \bar{J}^{\;\!\textcolor{gray}{t}}_{\;\!\textcolor{Maroon}{\symup{e}}\textcolor{gray}{z}} = \bar{\bar{\sigma}}^{\;\!\textcolor{Maroon}{\text{(1)}}\textcolor{gray}{t}}_{\;\!\textcolor{gray}{z}}~\widetilde *~\bar{E}^{\;\!\textcolor{gray}{t}}_{\;\!\textcolor{gray}{z}} + \bar{J}^{\;\!\textcolor{Maroon}{\text{NL}}, \textcolor{gray}{t}}_{\;\!\textcolor{Maroon}{\symup{e}}\textcolor{gray}{z}} =: \bar{J}^{\;\!\textcolor{Maroon}{\text{(1)}} \textcolor{gray}{t}}_{\;\!\textcolor{Maroon}{\symup{e}}\textcolor{gray}{z}} + \bar{J}^{\;\!\textcolor{Maroon}{\text{NL}}, \textcolor{gray}{t}}_{\;\!\textcolor{Maroon}{\symup{e}}\textcolor{gray}{z}}~,
\end{equation}
其三,贡献进
\begin{subequations} \label{eq:cr-d}
\begin{align}
	\textcolor{Maroon}{\text{CR for electricity}}\text{:}&\hspace{0.5em} \bar{D}^{\;\!\textcolor{gray}{t}}_{\;\!\textcolor{gray}{z}} \hspace{-2.0em} &&= {\symup{\varepsilon}}_0 \bar{E}^{\;\!\textcolor{gray}{t}}_{\;\!\textcolor{gray}{z}} + \bar{P}^{\;\!\textcolor{gray}{t}}_{\;\!\textcolor{gray}{z}} = {\symup{\varepsilon}}_0 \bar{\bar{\delta}}^{\;\!\textcolor{gray}{t}}~\widetilde *~\bar{E}^{\;\!\textcolor{gray}{t}}_{\;\!\textcolor{gray}{z}} + \bar{P}^{\;\!\textcolor{gray}{t}}_{\;\!\textcolor{gray}{z}} \label{cr-d1} \\ & &&\xrightarrow[]{\bar{P}^{\;\!\textcolor{gray}{t}}_{\;\!\textcolor{gray}{z}} = \bar{P}^{\;\!\textcolor{Maroon}{\text{(1)}} \textcolor{gray}{t}}_{\;\!\textcolor{gray}{z}} + \bar{P}^{\;\!\textcolor{Maroon}{\text{NL}}, \textcolor{gray}{t}}_{\;\!\textcolor{gray}{z}} + } \left[ {\symup{\varepsilon}}_0 \bar{\bar{\delta}}^{\;\!\textcolor{gray}{t}}~\widetilde *~\bar{E}^{\;\!\textcolor{gray}{t}}_{\;\!\textcolor{gray}{z}} + \bar{P}^{\;\!\textcolor{Maroon}{\text{(1)}} \textcolor{gray}{t}}_{\;\!\textcolor{gray}{z}} \right] + \bar{P}^{\;\!\textcolor{Maroon}{\text{NL}}, \textcolor{gray}{t}}_{\;\!\textcolor{gray}{z}} \label{cr-d2} \\ & &&\xrightarrow[\displaystyle{ \bar{\bar{\varepsilon}}^{\;\!\textcolor{Maroon}{\text{(1)}} \textcolor{gray}{t}}_{\;\!\textcolor{Maroon}{\text{r}}\textcolor{gray}{z}} := \bar{\bar{\delta}}^{\;\!\textcolor{gray}{t}} + \bar{\bar{\chi}}^{\;\!\textcolor{Maroon}{\text{(1)}}\textcolor{gray}{t}}_{\;\!\textcolor{Maroon}{\text{e}} \textcolor{gray}{z}}}]{\displaystyle{\bar{P}^{\;\!\textcolor{Maroon}{\text{(1)}} \textcolor{gray}{t}}_{\;\!\textcolor{gray}{z}} := \bar{\bar{\chi}}^{\;\!\textcolor{Maroon}{\text{(1)}}\textcolor{gray}{t}}_{\;\!\textcolor{Maroon}{\text{e}} \textcolor{gray}{z}} ~\widetilde *~\bar{E}^{\;\!\textcolor{gray}{t}}_{\;\!\textcolor{gray}{z}}}} {\symup{\varepsilon}}_0 \bar{\bar{\varepsilon}}^{\;\!\textcolor{Maroon}{\text{(1)}} \textcolor{gray}{t}}_{\;\!\textcolor{Maroon}{\text{r}}\textcolor{gray}{z}}~\widetilde *~\bar{E}^{\;\!\textcolor{gray}{t}}_{\;\!\textcolor{gray}{z}} + \bar{P}^{\;\!\textcolor{Maroon}{\text{NL}}, \textcolor{gray}{t}}_{\;\!\textcolor{gray}{z}} \label{cr-d3} \\ & &&= \bar{\bar{\varepsilon}}^{\;\!\textcolor{Maroon}{\text{(1)}} \textcolor{gray}{t}}_{\;\!\textcolor{gray}{z}}~\widetilde *~\bar{E}^{\;\!\textcolor{gray}{t}}_{\;\!\textcolor{gray}{z}} + \bar{P}^{\;\!\textcolor{Maroon}{\text{NL}}, \textcolor{gray}{t}}_{\;\!\textcolor{gray}{z}} =: \bar{D}^{\;\!\textcolor{Maroon}{\text{(1)}} \textcolor{gray}{t}}_{\;\!\textcolor{gray}{z}} + \bar{D}^{\;\!\textcolor{Maroon}{\text{NL}}, \textcolor{gray}{t}}_{\;\!\textcolor{gray}{z}}~, \label{cr-d4}
\end{align}
\end{subequations}



