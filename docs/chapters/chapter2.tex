% based on 北理工博士毕业论文 的 chapter2 https://tex.nju.edu.cn/project/user/10d0ed00-6f46-42fa-8a3a-18270214cfa3/a0ed1339-764d-4cc7-9ac4-d5d05b33da53

%\chapter{晶体中的线性、非线性光学过程}
\chapter{\protect\hyperlink{chap:\thechapter}{晶体中的线性、非线性光学过程的解析解}}
\addtocontents{toc}{\protect\linkdest{chap:\thechapter}}
\label{晶体中的线性、非线性光学过程的解析解}

%\section{晶体中的电场混频方程组}
\section{\protect\hyperlink{chap:\thesection}{晶体中的电场混频方程组}}
\addtocontents{toc}{\protect\linkdest{chap:\thesection}}
\label{晶体中的电场混频方程组}

晶体\myHyperFootnote{这里的“晶体”,如 Berry\cite{berryOpticalSingularitiesBianisotropic2005} 所言,
指的是波长尺度均匀的任何材料,因此包括液晶、塑料、玻璃、微观复合材料、磁化等离子体等内部。
尽管可以但一般不包括带有自由电荷的体区域,如导体表面薄层;如果只使用 Eq.(\ref{eq:2-16}) 中的波动方程而不使用散度方程,则可以处理导体、半导体等内部甚至表面。},
作为非线性光学过程的主要发生场所,大多是纯电各向异性\myHyperFootnote{即材料的两个二阶张量中,磁是各向同性的,电是各向异性的,这是典型的天然光学材料\cite{mcleodVectorFourierOptics2014}。}的。
也就是说,参与非线性光学过程的电场的各 $\{{\omega, \symbf{k}}\}$ 组分,在晶体中各自的独立线性衍射,
不再由电各向同性(如空气)中的矢量亥姆霍兹方程的解 —— 矢量角谱所控制,以致更复杂一些\myHyperFootnote{形式上类似,但额外受到晶体的二阶张量限制,导致其复振幅必须正比于晶体所允许的两个复本征模(偏振态),相应的复波矢也只能取晶体所允许的两个本征值。}。
而独立于这些线性光学过程之外,波动方程右侧的非零非线性波源项,
同时也在无时无刻不在驱动着非线性频率转换过程的发生。

形象且完整地描述非线性混频的参数过程:即驱动源 $\{{\omega_i, \symbf{k}_i}\}$ 们,一边利用能量守恒、动量守恒,
在晶体内源源不断地产生 $\{{\omega_j, \symbf{k}_j}\}$ 新波长下新空间频率的光,一边又在晶体内独立\myHyperFootnote{这种“独立”,既独立于驱动源 $\{{\omega_i, \symbf{k}_i}\}$ 内其他谱分量,
又独立于其所转换出的 $\{{\omega_j, \symbf{k}_j}\}$ 的每个谱分量。} 衍射;
同时,任何新组分 $\{{\omega_j, \symbf{k}_j}\}$ 一旦产生,就开始独立于其他任何组分\myHyperFootnote{同样,既独立于驱动源 $\{{\omega_i, \symbf{k}_i}\}$ 的所有谱分量,
也独立于新 $\{{\omega_j, \symbf{k}_j}\}$ 内,除自己外的其他时空谱分量。} 地,线性衍射。
这些新产生的 $\{{\omega_j, \symbf{k}_j}\}$ 分量,一边在线性衍射,一边又在与之前、以及后续产生的、
波长相同的时空谱分量,在空域干涉\myHyperFootnote{即空间频率域二维傅立叶逆变换(线性组合/叠加)。},
再在波长域线性叠加\myHyperFootnote{即时间频率域一维傅立叶正变换、“时域干涉”。},并最终产生混频所得的结果。

因此,晶体中任意地点任意时刻的任意一个 $\{{\omega, \symbf{k}}\}$ 组分,同时参与:线性衍射、频率转换,这两个过程。
换句话说,线性、非线性光学过程,对于任意一个 $\{{\omega, \symbf{k}}\}$ 对象而言,本质上均不可解耦\myHyperFootnote{这是三维数值解法如分步傅立叶算法,需以线性、非线性算符小步长交替迭代,才能将线性衍射、非线性频率转换两个过程都纳入考虑,并逼近准确的原因;频域龙格库塔法无需线性算符,但仍需迭代。}。
这是接下来将要推导的时空频率域的波动方程,所自然体现出的结论。

%\subsection{磁线性均匀各向异性介质中的电场}
\subsection{\protect\hyperlink{chap:\thesubsection}{磁线性均匀各向异性介质中的电场}}
\addtocontents{toc}{\protect\linkdest{chap:\thesubsection}}
\label{磁线性均匀各向异性介质中的电场}

相对静止\myHyperFootnote{否则 $\widetilde{\symbf D}, \widetilde{\symbf B}$ 间将相互耦合 
\cite{berryOpticalSingularitiesBianisotropic2005},以致材料的本构关系通常会呈现出双各向异性\cite{mackayElectromagneticAnisotropyBianisotropy2019,mackayModernAnalyticalElectromagnetic2020};$\widetilde{\symbf J}_{\symup{f}}$ 也需要扩展到四维\cite{XieQuanMianHuiYiWas}。} 的坐标系下,在无电荷源(${\widetilde \rho}_{\symup{f}} \rightarrow 0$)、可能有电流源($\widetilde{\symbf J}_{\symup{f}}$)的一般电磁介质内部,4 个空域时变\myHyperFootnote{指复矢量场 $ \widetilde{\symbf E}, \widetilde{\symbf H}, \widetilde{\symbf D}, \widetilde{\symbf B} $ 
均是四维时空 $(\symbf{r},t)$ 的函数,并因此通常不是单色的(头上加一波浪号 $"\sim"$)。}复色场 
$ \widetilde{\symbf E}, \widetilde{\symbf H},  \widetilde{\symbf D}, \widetilde{\symbf B} $,满足微分形式的麦氏方程组:
\begin{equation} \label{eq:2-1}
	\left\{\ \begin{aligned}\nabla \cdot \widetilde{\symbf D} &= {\widetilde \rho}_{\symup{f}}  \\\nabla \cdot \widetilde{\symbf B} &= 0 \\\nabla \times \widetilde{\symbf E} &= - \frac{\partial \widetilde{\symbf B}}{\partial t} \\ \nabla \times \widetilde{\symbf H} &= \widetilde{\symbf J}_{\symup{f}} + \frac{\partial \widetilde{\symbf D}}{\partial t} \end{aligned}\right. \xrightarrow[{\symup{inside\ material}}]{\left\{\ \begin{aligned}{\widetilde \rho}_{\symup{f}} &\rightarrow 0  \\ \widetilde{\symbf J}_{\symup{f}} &= {\widetilde \rho}_{\symup{f}} \cdot {\symbf v}_{\symup{f}} \end{aligned}\right.} \left\{\ \begin{aligned}\nabla \cdot \widetilde{\symbf D} &= 0 \\\nabla \cdot \widetilde{\symbf B} &= 0 \\\nabla \times \widetilde{\symbf E} &= - \frac{\partial \widetilde{\symbf B}}{\partial t} \\ \nabla \times \widetilde{\symbf H} &= \widetilde{\symbf J}_{\symup{f}} + \frac{\partial \widetilde{\symbf D}}{\partial t} \end{aligned}\right. ~,
\end{equation}

接着需先将磁场 $\widetilde{\symbf H} = H^t_{{\symup{\mu}} z} \symbfup{e}_{{\symup{\mu}}} = \mathcal F_t \left[ H^\omega_{{\symup{\mu}} z} \right] \symbfup{e}_{{\symup{\mu}}} 
= \mathcal F_t \left[ \symbf{H}^\omega_{z} \right] $ 从时域,单色化到波长域;
再先后从波长域\myHyperFootnote{$ \widetilde{\symbf E}, \widetilde{\symbf H}$ 在 $ \widetilde{\symbf P}, \widetilde{\symbf D}, \widetilde{\symbf M}, \widetilde{\symbf B} $ 的非线性项中,于时域的卷积,是因果律的体现;于波长域/时间频率域的卷积,是能量守恒的体现;二者均用 $"\ \widetilde *\ "$ 表示,以区别于空域/空间频率域的卷积算符 $"*"$。}、时域,定义磁化强度 $\widetilde{\symbf M} = M^t_{{\symup{\mu}} z} \symbfup{e}_{{\symup{\mu}}} = \mathcal F_t \left[ M^\omega_{{\symup{\mu}} z} \right] \symbfup{e}_{{\symup{\mu}}}
= \mathcal F_t \left[ \symbf{M}^\omega_{z} \right] $ 关于 $\widetilde{\symbf H}$ 的非线性函数(其中\myHyperFootnote{时域傅立叶变换的核,需要与空域的共轭,这样正频率正波矢才对应前向行波 $\mathbb{e}^{\mathbb{i} \left( {\symbf k}^\omega \cdot \symbf r - \omega t \right)}$\cite{mcleodVectorFourierOptics2014}。} $\mathcal F_t \left[ \cdot \right] := \frac{ 1 }{ 2\pi } \int_{-\infty}^{+\infty} \cdot\ \mathbb{e}^{-\mathbb{i}\omega t} \mathbb{d}\omega $、$\mathcal F^{-1}_\omega \left[ \cdot \right] = \int_{-\infty}^{+\infty} \cdot\ \mathbb{e}^{\mathbb{i}\omega t} \mathbb{d}t $):
\begin{equation} \label{eq:2-2}
	\left\{\ \begin{aligned} M^\omega_{{\symup{\nu}} z} &= \mathcal F^{-1}_\omega \left[ M^t_{{\symup{\nu}} z} \right] = M^{(1)\omega}_{{\symup{\nu}} z} + M^{(2)\omega}_{{\symup{\nu}} z} + M^{(3)\omega}_{{\symup{\nu}} z} + \cdots \\ &= \chi^{{\symup{m}}(1)\omega}_{{\symup{\nu}} {\symup{\mu}}_1z} H^\omega_{{\symup{\mu}}_1 z} + \chi^{{\symup{m}}(2)\omega}_{{\symup{\nu}} {\symup{\mu}}_1 {\symup{\mu}}_2z} H^\omega_{{\symup{\mu}}_1 z}\ \widetilde *\ H^\omega_{{\symup{\mu}}_2 z} \\ &+ \chi^{{\symup{m}}(3)\omega}_{{\symup{\nu}} {\symup{\mu}}_1 {\symup{\mu}}_2 {\symup{\mu}}_3z} H^\omega_{{\symup{\mu}}_1 z}\ \widetilde *\ H^\omega_{{\symup{\mu}}_2 z}\ \widetilde *\ H^\omega_{{\symup{\mu}}_3 z} + \cdots \\ &= \chi^{{\symup{m}}(1)\omega}_{{\symup{\nu}} {\symup{\mu}}_1z} \mathcal F^{-1}_\omega \left[ H^t_{{\symup{\mu}}_1 z} \right] + \chi^{{\symup{m}}(2)\omega}_{{\symup{\nu}} {\symup{\mu}}_1 {\symup{\mu}}_2z} \mathcal F^{-1}_\omega \left[ H^t_{{\symup{\mu}}_1 z} H^t_{{\symup{\mu}}_2 z} \right] \\ &+ \chi^{{\symup{m}}(3)\omega}_{{\symup{\nu}} {\symup{\mu}}_1 {\symup{\mu}}_2 {\symup{\mu}}_3z} \mathcal F^{-1}_\omega \left[ H^t_{{\symup{\mu}}_1 z} H^t_{{\symup{\mu}}_2 z} H^t_{{\symup{\mu}}_3 z} \right] + \cdots \\ M^t_{{\symup{\nu}} z} &= \mathcal F_t \left[ M^\omega_{{\symup{\nu}} z} \right] = M^{(1)t}_{{\symup{\nu}} z} + M^{(2)t}_{{\symup{\nu}} z} + M^{(3)t}_{{\symup{\nu}} z} + \cdots \\ &= \chi^{{\symup{m}}(1)t}_{{\symup{\nu}} {\symup{\mu}}_1z}\ \widetilde *\ H^t_{{\symup{\mu}}_1 z} + \chi^{{\symup{m}}(2)t}_{{\symup{\nu}} {\symup{\mu}}_1 {\symup{\mu}}_2z}\ \widetilde *\left( H^t_{{\symup{\mu}}_1 z} H^t_{{\symup{\mu}}_2 z} \right) \\ &+ \chi^{{\symup{m}}(3)t}_{{\symup{\nu}} {\symup{\mu}}_1 {\symup{\mu}}_2 {\symup{\mu}}_3z}\ \widetilde *\left( H^t_{{\symup{\mu}}_1 z} H^t_{{\symup{\mu}}_2 z} H^t_{{\symup{\mu}}_3 z} \right) + \cdots  \end{aligned}\right. ~,
\end{equation}
其中正体 ${\symup{\mu}},{\symup{\mu}}_i,{\symup{\nu}} \in {\symup{x}},{\symup{y}},{\symup{z}}$ 表示三维笛卡尔坐标系下的分量;且对角标 ${\symup{\mu}}_i$ 使用爱因斯坦求和约定:对于重复的符号对,两两遍历三分量 ${\symup{x}},{\symup{y}},{\symup{z}}$ 并求和。
最末多出的一个斜体下标 $z$,表示随空域 $z$ 向坐标的变化而变化,是角谱理论的“标志”。

上述三分量标量形式,也可简写作下面的矢量形式:
\begin{equation} \label{eq:2-3}
	\left\{\ \begin{aligned} \symbf M^\omega_z &= \mathcal F^{-1}_\omega \left[ \widetilde{\symbf M} \right] = \symbf M^{(1)\omega}_z + \symbf M^{(2)\omega}_z + \symbf M^{(3)\omega}_z + \cdots \\ &= \overset{\rightharpoonup\!\!\!\! \rightharpoonup}{\symbf{\chi}_{\symup{m}}}^{(1)t}_{z} \cdot \symbf H^{\omega}_z + \overset{\rightharpoonup\!\!\!\! \rightharpoonup\!\!\!\! \rightharpoonup}{\symbf{\chi}_{\symup{m}}}^{(2)t}_{z} \colon \mathcal F^{-1}_\omega \left[ \widetilde{\symbf H} \widetilde{\symbf H} \right] + \overset{\rightharpoonup\!\!\!\! \rightharpoonup\!\!\!\! \rightharpoonup\!\!\!\! \rightharpoonup}{\symbf{\chi}_{\symup{m}}}^{(3)t}_{z} \vdots \mathcal F^{-1}_\omega \left[ \widetilde{\symbf H} \widetilde{\symbf H} \widetilde{\symbf H} \right] + \cdots \\ \widetilde{\symbf M} &= \mathcal F_t \left[ \symbf M^{\omega}_z \right] = \widetilde{\symbf M}^{(1)} + \widetilde{\symbf M}^{(2)} + \widetilde{\symbf M}^{(3)} + \cdots \\&= \overset{\rightharpoonup\!\!\!\! \rightharpoonup}{\symbf{\chi}_{\symup{m}}}^{(1)t}_{z}\ \widetilde *\ \widetilde{\symbf H} + \overset{\rightharpoonup\!\!\!\! \rightharpoonup\!\!\!\! \rightharpoonup}{\symbf{\chi}_{\symup{m}}}^{(2)t}_{z}\ {}^{\widetilde *}_{\widetilde *} \left( \widetilde{\symbf H} \widetilde{\symbf H} \right) + \overset{\rightharpoonup\!\!\!\! \rightharpoonup\!\!\!\! \rightharpoonup\!\!\!\! \rightharpoonup}{\symbf{\chi}_{\symup{m}}}^{(3)t}_{z}\ \begin{smallmatrix} \widetilde * \\ \widetilde * \\ \widetilde * \end{smallmatrix} \left( \widetilde{\symbf H} \widetilde{\symbf H} \widetilde{\symbf H} \right) + \cdots \end{aligned}\right. ~.
\end{equation}

在得到了磁场 $\widetilde{\symbf H}$ 到磁化强度 $\widetilde{\symbf M}$ 的非线性映射关系的时域表达式后,
可利用 $\widetilde{\symbf B}-\widetilde{\symbf H}$ 本构关系\myHyperFootnote{尽管这样写,基本场却是前者 $\widetilde{\symbf B}$;不像 $\widetilde{\symbf D}-\widetilde{\symbf E}$ 关系中,基本场为后者 $\widetilde{\symbf E}$\cite{nelsonDerivingTransmissionReflection1995}。},自然地导出磁感应强度 $\widetilde{\symbf B}$ 在时域上关于 $\widetilde{\symbf H}$ 的非线性函数:
%对时域磁感应强度 $\widetilde{\symbf B}$ 一维傅立叶逆变换,得其在波长域下的表达式 $\symbf B^\omega_z$ :
\begin{equation} \label{eq:2-4}
	\abovedisplayshortskip=0pt
	\abovedisplayskip=0pt
	\left\{\ \begin{aligned} \widetilde{\symbf B} &= {\symup{\mu}}_0 \left( \widetilde{\symbf H} + \widetilde{\symbf M} \right) \\ &= {\symup{\mu}}_0 \left\{\ \overset{\rightharpoonup\!\!\!\! \rightharpoonup}{\symbfup{\delta}} \left( t \right)\ \widetilde *\ \widetilde{\symbf H} + \widetilde{\symbf M} \right\} = {\symup{\mu}}_0 \left\{\ \left[ \overset{\rightharpoonup\!\!\!\! \rightharpoonup}{\symbfup{\delta}} \left( t \right)\ \widetilde *\ \widetilde{\symbf H} + \widetilde{\symbf M}^{(1)} \right] + \widetilde{\symbf M}^{{\symup{NL}}} \right\} \\& \xrightarrow[]{\displaystyle{ \overset{\rightharpoonup\!\!\!\! \rightharpoonup}{\symbf{\mu}_{\symup{r}}}^{(1)t}_{z} := \overset{\rightharpoonup\!\!\!\! \rightharpoonup}{\symbfup{\delta}} \left( t \right) + \overset{\rightharpoonup\!\!\!\! \rightharpoonup}{\symbf{\chi}_{\symup{m}}}^{(1)t}_{z}}} {\symup{\mu}}_0 \left\{\ \overset{\rightharpoonup\!\!\!\! \rightharpoonup}{\symbf{\mu}_{\symup{r}}}^{(1)t}_{z}\ \widetilde *\ \widetilde{\symbf H} + \widetilde{\symbf M}^{{\symup{NL}}} \right\} \\ &= \overset{\rightharpoonup\!\!\!\! \rightharpoonup}{\symbf{\mu}}^{(1)t}_{z}\ \widetilde *\ \widetilde{\symbf H} + {\symup{\mu}}_0 \widetilde{\symbf M}^{{\symup{NL}}} =: \widetilde{\symbf B}^{(1)} + \widetilde{\symbf B}^{{\symup{NL}}} \\ \symbf B^\omega_z &= \mathcal F^{-1}_\omega \left[ \widetilde{\symbf B} \right] \xrightarrow[]{\displaystyle{ \overset{\rightharpoonup\!\!\!\! \rightharpoonup}{\symbf{\mu}_{\symup{r}}}^{(1)\omega}_{z} := \overset{\rightharpoonup\!\!\!\! \rightharpoonup}{\symbfup{I}} + \overset{\rightharpoonup\!\!\!\! \rightharpoonup}{\symbf{\chi}_{\symup{m}}}^{(1)\omega}_{z}}} {\symup{\mu}}_0 \left\{\ \overset{\rightharpoonup\!\!\!\! \rightharpoonup}{\symbf{\mu}_{\symup{r}}}^{(1)\omega}_{z} \cdot \symbf H^{\omega}_z + \symbf M^{{\symup{NL}},\omega}_z \right\} \\ &= \overset{\rightharpoonup\!\!\!\! \rightharpoonup}{\symbf{\mu}}^{(1)\omega}_{z} \cdot \symbf H^{\omega}_z + {\symup{\mu}}_0 \symbf M^{{\symup{NL}},\omega}_z =: \symbf B^{(1)\omega}_z + \symbf B^{{\symup{NL}},\omega}_z \end{aligned}\right. ~,
\end{equation}
其中 $ \overset{\rightharpoonup\!\!\!\! \rightharpoonup}{\symbfup{I}} = \mathcal F^{-1}_\omega \left[ \overset{\rightharpoonup\!\!\!\! \rightharpoonup}{\symbfup{\delta}} \left( t \right) \right] $ 是对角单位二阶张量。

有了上述 $\widetilde{\symbf B}-\widetilde{\symbf H}$ 关系,才能对 Eq.(\ref{eq:2-1}) 中的电场旋度方程左右两侧取旋度:
\begin{subequations} \label{eq:2-5}
\abovedisplayshortskip=0pt
\abovedisplayskip=0pt
\begin{align}
	\nabla \times \left( \nabla \times \widetilde{\symbf E} \right) = &- \displaystyle{\frac{\partial}{\partial t}} \left( \nabla \times \widetilde{\symbf B} \right) \label{eq:2-5a}\\ \xrightarrow[\displaystyle{ \widetilde{\symbf M}^{\symup{NL}}\ \equiv\ \symbf 0 }]{\displaystyle{ \widetilde{\symbf B} = \overset{\rightharpoonup\!\!\!\! \rightharpoonup}{\symbf{\mu}}^{(1)t}_{z}\ \widetilde *\ \widetilde{\symbf H} + {\symup{\mu}}_0 \widetilde{\symbf M}^{\symup{NL}}}} &- \displaystyle{\frac{\partial }{\partial t}} \left[ \nabla \times \left( \overset{\rightharpoonup\!\!\!\! \rightharpoonup}{\symbf{\mu}}^{(1)t}_{z}\ \widetilde *\ \widetilde{\symbf H} \right) \right] \label{eq:2-5b}\\ \xrightarrow[{\symup{independent\ of}}\ \symbf r]{\displaystyle{ \overset{\rightharpoonup\!\!\!\! \rightharpoonup}{\symbf{\mu}}^{(1)t}_{z}} \equiv \overset{\rightharpoonup\!\!\!\! \rightharpoonup}{\symbf{\mu}}^{(1)t}} &- \displaystyle{\frac{\partial }{\partial t}} \left[ \overset{\rightharpoonup\!\!\!\! \rightharpoonup}{\symbf{\mu}}^{(1)t}\ \widetilde *\ \left( \nabla \times \widetilde{\symbf H} \right) \right] \label{eq:2-5c}\\ \xrightarrow[]{\displaystyle{ \nabla \times \widetilde{\symbf H} = \widetilde{\symbf J}_{\symup{f}} + \frac{\partial \widetilde{\symbf D}}{\partial t}}} &- \displaystyle{\frac{\partial }{\partial t}} \left[ \overset{\rightharpoonup\!\!\!\! \rightharpoonup}{\symbf{\mu}}^{(1)t}\ \widetilde *\ \left( \widetilde{\symbf J}_{\symup{f}} + \frac{\partial \widetilde{\symbf D}}{\partial t} \right) \right] \label{eq:2-5d}\\ \xrightarrow[{\symup{derivative\ properties\ of\ convolution}}]{\displaystyle{ \frac{\partial \left[ f\left( t \right) \ \widetilde *\ g\left( t \right) \right]}{\partial t} = f\left( t \right) \ \widetilde *\ \frac{\partial g\left( t \right) }{\partial t }}} &- \overset{\rightharpoonup\!\!\!\! \rightharpoonup}{\symbf{\mu}}^{(1)t}\ \widetilde *\  \displaystyle{\left( \frac{\partial \widetilde{\symbf J}_{\symup{f}}}{\partial t } + \frac{\partial^2 \widetilde{\symbf D}}{\partial t^2} \right)} \label{eq:2-5e}~,
\end{align}
\end{subequations}
即在 $\widetilde{\symbf B} = \overset{\rightharpoonup\!\!\!\! \rightharpoonup}{\symbf{\mu}}^{(1)t}\ \widetilde *\ \widetilde{\symbf H}$ 条件下,得到了磁线性均匀各向异性电磁介质内的空域电场波动方程;但除此之外,还需满足 $\widetilde{\symbf D}$ 的散度方程:
\begin{equation} \label{eq:2-6}
    % \setlength{\abovedisplayskip}{0cm}
    % \setlength{\belowdisplayskip}{0.3cm}
    % \belowdisplayskip=10pt
	\left\{\ \begin{aligned} \nabla \times \left( \nabla \times \widetilde{\symbf E} \right) + \overset{\rightharpoonup\!\!\!\! \rightharpoonup}{\symbf{\mu}}^{(1)t}\ \widetilde *\  \displaystyle{\left( \frac{\partial \widetilde{\symbf J}_{\symup{f}}}{\partial t } + \frac{\partial^2 \widetilde{\symbf D}}{\partial t^2} \right)} &= \symbf 0 \\ \nabla \cdot \widetilde{\symbf D} &= 0 \end{aligned}\right. ~.
\end{equation}

从 Eq.(\ref{eq:2-1}) 四个方程到 Eq.(\ref{eq:2-6}) 两个方程后,沿着 \cref{eq:2-2,eq:2-3,eq:2-4} 的步骤,先从波长域,定义电流密度 $\symbf{J}^\omega_{{\symup{f}}z}$、电极化强度 $\symbf{P}^\omega_{z}$ 分别关于电场 $\symbf{E}^\omega_{z}$ 的线性、非线性展开,再得到时域上,$\widetilde{\symbf J}_{\symup{f}}$、$ \widetilde{\symbf P}, \widetilde{\symbf D}$ 关于 $\widetilde{\symbf E}$ 的线性、非线性关系:
\begin{equation} \label{eq:2-7}
    \belowdisplayskip=10pt
	\left\{\ \begin{aligned} J^\omega_{{\symup{f}}{\symup{\nu}} z} &= \mathcal F^{-1}_\omega \left[ J^t_{{\symup{f}}{\symup{\nu}} z} \right] = \sigma^{\omega}_{{\symup{\nu}} {\symup{\mu}} z} E^\omega_{{\symup{\mu}} z} \\ J^t_{{\symup{f}}{\symup{\nu}} z} &= \mathcal F_t \left[ J^\omega_{{\symup{f}}{\symup{\nu}} z} \right] = \sigma^{t}_{{\symup{\nu}} {\symup{\mu}} z}\ \widetilde *\ E^t_{{\symup{\mu}} z} \end{aligned}\right. ~,
\end{equation}
\begin{equation} \label{eq:2-8}
	\left\{\ \begin{aligned} \symbf J^\omega_{\symup{f} z} &= \mathcal F^{-1}_\omega \left[ \widetilde{\symbf J}_{\symup{f}} \right] = \overset{\rightharpoonup\!\!\!\! \rightharpoonup}{\symbf{\sigma}}^{t}_{z} \cdot \symbf E^{\omega}_z \\ \widetilde{\symbf J}_{\symup{f}} &= \mathcal F_t \left[ \symbf J^{\omega}_{\symup{f} z} \right] = \overset{\rightharpoonup\!\!\!\! \rightharpoonup}{\symbf{\sigma}}^{t}_{z}\ \widetilde *\ \widetilde{\symbf E} \end{aligned}\right. ~;
\end{equation}
\begin{equation} \label{eq:2-9}
	\left\{\ \begin{aligned} P^\omega_{{\symup{\nu}} z} &= \mathcal F^{-1}_\omega \left[ P^t_{{\symup{\nu}} z} \right] = P^{(1)\omega}_{{\symup{\nu}} z} + P^{(2)\omega}_{{\symup{\nu}} z} + P^{(3)\omega}_{{\symup{\nu}} z} + \cdots \\ &= {\symup{\varepsilon_0}} \left\{ \chi^{{\symup{e}}(1)\omega}_{{\symup{\nu}} {\symup{\mu}}_1z} E^\omega_{{\symup{\mu}}_1 z} + \chi^{{\symup{e}}(2)\omega}_{{\symup{\nu}} {\symup{\mu}}_1 {\symup{\mu}}_2z} E^\omega_{{\symup{\mu}}_1 z}\ \widetilde *\ E^\omega_{{\symup{\mu}}_2 z} \right. \\  &+ \left. \chi^{{\symup{e}}(3)\omega}_{{\symup{\nu}} {\symup{\mu}}_1 {\symup{\mu}}_2 {\symup{\mu}}_3z} E^\omega_{{\symup{\mu}}_1 z}\ \widetilde *\ E^\omega_{{\symup{\mu}}_2 z}\ \widetilde *\ E^\omega_{{\symup{\mu}}_3 z} + \cdots \right\} \\ &= {\symup{\varepsilon_0}} \left\{ \chi^{{\symup{e}}(1)\omega}_{{\symup{\nu}} {\symup{\mu}}_1z} \mathcal F^{-1}_\omega \left[ E^t_{{\symup{\mu}}_1 z} \right] + \chi^{{\symup{e}}(2)\omega}_{{\symup{\nu}} {\symup{\mu}}_1 {\symup{\mu}}_2z} \mathcal F^{-1}_\omega \left[ E^t_{{\symup{\mu}}_1 z} E^t_{{\symup{\mu}}_2 z} \right] \right. \\ &+ \left. \chi^{{\symup{e}}(3)\omega}_{{\symup{\nu}} {\symup{\mu}}_1 {\symup{\mu}}_2 {\symup{\mu}}_3z} \mathcal F^{-1}_\omega \left[ E^t_{{\symup{\mu}}_1 z} E^t_{{\symup{\mu}}_2 z} E^t_{{\symup{\mu}}_3 z} \right] + \cdots \right\} \\ P^t_{{\symup{\nu}} z} &= \mathcal F_t \left[ P^\omega_{{\symup{\nu}} z} \right] = P^{(1)t}_{{\symup{\nu}} z} + P^{(2)t}_{{\symup{\nu}} z} + P^{(3)t}_{{\symup{\nu}} z} + \cdots \\ &= {\symup{\varepsilon_0}} \left\{ \chi^{{\symup{e}}(1)t}_{{\symup{\nu}} {\symup{\mu}}_1z}\ \widetilde *\ E^t_{{\symup{\mu}}_1 z} + \chi^{{\symup{e}}(2)t}_{{\symup{\nu}} {\symup{\mu}}_1 {\symup{\mu}}_2z}\ \widetilde *\left( E^t_{{\symup{\mu}}_1 z} E^t_{{\symup{\mu}}_2 z} \right) \right. \\ &+ \left. \chi^{{\symup{e}}(3)t}_{{\symup{\nu}} {\symup{\mu}}_1 {\symup{\mu}}_2 {\symup{\mu}}_3z}\ \widetilde *\left( E^t_{{\symup{\mu}}_1 z} E^t_{{\symup{\mu}}_2 z} E^t_{{\symup{\mu}}_3 z} \right) + \cdots \right\} \end{aligned}\right. ~,
\end{equation}
\begin{numcases}{} \label{eq:2-10}
    % \belowdisplayskip=0pt
	\begin{aligned} \symbf P^\omega_z &= \mathcal F^{-1}_\omega \left[ \widetilde{\symbf P} \right] = \symbf P^{(1)\omega}_z + \symbf P^{(2)\omega}_z + \symbf P^{(3)\omega}_z + \cdots \\ &= {\symup{\varepsilon_0}} \left\{ \overset{\rightharpoonup\!\!\!\! \rightharpoonup}{\symbf{\chi}_{\symup{e}}}^{(1)t}_{z} \cdot \symbf E^{\omega}_z + \overset{\rightharpoonup\!\!\!\! \rightharpoonup\!\!\!\! \rightharpoonup}{\symbf{\chi}_{\symup{e}}}^{(2)t}_{z} \colon \mathcal F^{-1}_\omega \left[ \widetilde{\symbf E} \widetilde{\symbf E} \right] + \overset{\rightharpoonup\!\!\!\! \rightharpoonup\!\!\!\! \rightharpoonup\!\!\!\! \rightharpoonup}{\symbf{\chi}_{\symup{e}}}^{(3)t}_{z} \vdots \mathcal F^{-1}_\omega \left[ \widetilde{\symbf E} \widetilde{\symbf E} \widetilde{\symbf E} \right] + \cdots \right\} \\ \widetilde{\symbf P} &= \mathcal F_t \left[ \symbf P^{\omega}_z \right] = \widetilde{\symbf P}^{(1)} + \widetilde{\symbf P}^{(2)} + \widetilde{\symbf P}^{(3)} + \cdots \\&= {\symup{\varepsilon_0}} \left\{ \overset{\rightharpoonup\!\!\!\! \rightharpoonup}{\symbf{\chi}_{\symup{e}}}^{(1)t}_{z}\ \widetilde *\ \widetilde{\symbf E} + \overset{\rightharpoonup\!\!\!\! \rightharpoonup\!\!\!\! \rightharpoonup}{\symbf{\chi}_{\symup{e}}}^{(2)t}_{z}\ {}^{\widetilde *}_{\widetilde *} \left( \widetilde{\symbf E} \widetilde{\symbf E} \right) + \overset{\rightharpoonup\!\!\!\! \rightharpoonup\!\!\!\! \rightharpoonup\!\!\!\! \rightharpoonup}{\symbf{\chi}_{\symup{e}}}^{(3)t}_{z}\ \begin{smallmatrix} \widetilde * \\ \widetilde * \\ \widetilde * \end{smallmatrix} \left( \widetilde{\symbf E} \widetilde{\symbf E} \widetilde{\symbf E} \right) + \cdots \right\} ~, \end{aligned}
\end{numcases}
\begin{equation} \label{eq:2-11}
 %    \abovedisplayshortskip=0pt
	% \abovedisplayskip=0pt
	\left\{\ \begin{aligned} \widetilde{\symbf D} &= {\symup{\varepsilon_0}} \widetilde{\symbf E} + \widetilde{\symbf P} \\ &= {\symup{\varepsilon_0}} \overset{\rightharpoonup\!\!\!\! \rightharpoonup}{\symbfup{\delta}} \left( t \right)\ \widetilde *\ \widetilde{\symbf E} + \widetilde{\symbf P} = \left[{\symup{\varepsilon_0}} \overset{\rightharpoonup\!\!\!\! \rightharpoonup}{\symbfup{\delta}} \left( t \right)\ \widetilde *\ \widetilde{\symbf E} + \widetilde{\symbf P}^{(1)} \right] + \widetilde{\symbf P}^{{\symup{NL}}} \\& \xrightarrow[]{\displaystyle{ \overset{\rightharpoonup\!\!\!\! \rightharpoonup}{\symbf{\varepsilon}_{\symup{r}}}^{(1)t}_{z} := \overset{\rightharpoonup\!\!\!\! \rightharpoonup}{\symbfup{\delta}} \left( t \right) + \overset{\rightharpoonup\!\!\!\! \rightharpoonup}{\symbf{\chi}_{\symup{e}}}^{(1)t}_{z}}} {\symup{\varepsilon_0}} \overset{\rightharpoonup\!\!\!\! \rightharpoonup}{\symbf{\varepsilon}_{\symup{r}}}^{(1)t}_{z}\ \widetilde *\ \widetilde{\symbf E} + \widetilde{\symbf P}^{{\symup{NL}}} \\ &= \overset{\rightharpoonup\!\!\!\! \rightharpoonup}{\symbf{\varepsilon}}^{(1)t}_{z}\ \widetilde *\ \widetilde{\symbf E} + \widetilde{\symbf P}^{{\symup{NL}}} =: \widetilde{\symbf D}^{(1)} + \widetilde{\symbf D}^{{\symup{NL}}} \\ \symbf D^\omega_z &= \mathcal F^{-1}_\omega \left[ \widetilde{\symbf D} \right] \xrightarrow[]{\displaystyle{ \overset{\rightharpoonup\!\!\!\! \rightharpoonup}{\symbf{\varepsilon}_{\symup{r}}}^{(1)\omega}_{z} := \overset{\rightharpoonup\!\!\!\! \rightharpoonup}{\symbfup{I}} + \overset{\rightharpoonup\!\!\!\! \rightharpoonup}{\symbf{\chi}_{\symup{e}}}^{(1)\omega}_{z}}} {\symup{\varepsilon_0}} \overset{\rightharpoonup\!\!\!\! \rightharpoonup}{\symbf{\varepsilon}_{\symup{r}}}^{(1)\omega}_{z} \cdot \symbf E^{\omega}_z + \symbf P^{{\symup{NL}},\omega}_z \\ &= \overset{\rightharpoonup\!\!\!\! \rightharpoonup}{\symbf{\varepsilon}}^{(1)\omega}_{z} \cdot \symbf E^{\omega}_z + \symbf P^{{\symup{NL}},\omega}_z =: \symbf D^{(1)\omega}_z + \symbf D^{{\symup{NL}},\omega}_z \end{aligned}\right. ~,
\end{equation}
并将其中 $\widetilde{\symbf J}_{\symup{f}}$、$\widetilde{\symbf D}$ 的表达式,代入 Eq.(\ref{eq:2-6}) 中;再次利用卷积的微分性质
\begin{equation} \label{eq:2-12}
	\frac{\partial^2 \widetilde{\symbf D}}{\partial t^2} = \frac{\partial^2 }{\partial t^2} \left( \overset{\rightharpoonup\!\!\!\! \rightharpoonup}{\symbf{\varepsilon}}^{(1)t}_{z}\ \widetilde *\ \widetilde{\symbf E} \right) + \frac{\partial^2 \widetilde{\symbf P}^{{\symup{NL}}}}{\partial t^2} = \overset{\rightharpoonup\!\!\!\! \rightharpoonup}{\symbf{\varepsilon}}^{(1)t}_{z}\ \widetilde *\ \frac{\partial^2 \widetilde{\symbf E}}{\partial t^2} + \frac{\partial^2 \widetilde{\symbf P}^{{\symup{NL}}}}{\partial t^2} ~,
\end{equation}
可得磁线性均匀各向异性介质中的复色电场的非线性波动方程,及其散度约束:
\begin{equation} \label{eq:2-13}
    \abovedisplayshortskip=0pt
	\abovedisplayskip=0pt
    \belowdisplayshortskip=0pt
    \belowdisplayskip=0pt
	\left\{\ \begin{aligned} \nabla \times \left( \nabla \times \widetilde{\symbf E} \right) + \overset{\rightharpoonup\!\!\!\! \rightharpoonup}{\symbf{\mu}}^{(1)t}\ \widetilde * \left( \overset{\rightharpoonup\!\!\!\! \rightharpoonup}{\symbf{\varepsilon}}^{(1)t}_{z}\ \widetilde *\ \displaystyle{\frac{\partial^2 }{\partial t^2}} + \overset{\rightharpoonup\!\!\!\! \rightharpoonup}{\symbf{\sigma}}^{t}_{z}\ \widetilde *\ \displaystyle{\frac{\partial }{\partial t}} \right) \widetilde{\symbf E} &= - \overset{\rightharpoonup\!\!\!\! \rightharpoonup}{\symbf{\mu}}^{(1)t}\ \widetilde *\  \displaystyle{\frac{\partial^2 \widetilde{\symbf P}^{{\symup{NL}}}}{\partial t^2}} \\ \nabla \cdot \left( \overset{\rightharpoonup\!\!\!\! \rightharpoonup}{\symbf{\varepsilon}}^{(1)t}_{z}\ \widetilde *\ \widetilde{\symbf E} \right) &= - \nabla \cdot \widetilde{\symbf P}^{{\symup{NL}}} \end{aligned}\right. ~.
\end{equation}

将时域一维傅立叶积分 $\mathcal F^{-1}_\omega \left[ \cdot \right] $ 作用于 Eq.(\ref{eq:2-13}) 两侧,利用其导数性质
\begin{equation} \label{eq:2-14}
	\mathcal F^{-1}_\omega \left[ \displaystyle{\frac{\partial^2 \widetilde{\symbf E}}{\partial t^2}} \right] = \left( -i\omega \right)^2 \cdot \mathcal F^{-1}_\omega \left[ \widetilde{\symbf E} \right] ~,
\end{equation}
以及定义包含电导率张量的(新)相对介电常数张量
\begin{equation} \label{eq:2-15}
	\overset{\rightharpoonup\!\!\!\! \rightharpoonup}{\symbf{\varepsilon}'_{\symup{r}}}^{(1)\omega}_{z} := \overset{\rightharpoonup\!\!\!\! \rightharpoonup}{\symbf{\varepsilon}_{\symup{r}}}^{(1)\omega}_{z} + \displaystyle{ \frac{\mathbb{i}}{{\symup{\varepsilon_0} \omega}} } \overset{\rightharpoonup\!\!\!\! \rightharpoonup}{\symbf{\sigma}}^{\omega}_{z} ~,
\end{equation}
且通过卷积定理$ \mathcal F^{-1}_\omega \left[ f\left( t \right)\ \widetilde *\ g\left( t \right) \right] = \mathcal F^{-1}_\omega \left[ f\left( t \right) \right] \cdot \mathcal F^{-1}_\omega \left[ g\left( t \right) \right] $、光速 $\symup{c}=1 \big/ \sqrt{{\symup{\mu}}_0{\symup{\varepsilon_0}}}$、真空波数 $k^{\omega}_{0} = \omega \big/ \symup{c}$,得磁线性均匀各向异性介质中单色电场满足的非线性波动方程,及其散度约束:
\begin{equation} \label{eq:2-16}
	\left\{\ \begin{aligned} \nabla \times \left( \nabla \times \symbf E^{\omega}_z \right) - k^{2}_{0\omega}\ \overset{\rightharpoonup\!\!\!\! \rightharpoonup}{\symbf{\mu}_{\symup{r}}}^{(1)\omega} \cdot \overset{\rightharpoonup\!\!\!\! \rightharpoonup}{\symbf{\varepsilon}'_{\symup{r}}}^{(1)\omega}_{z} \cdot \symbf E^{\omega}_z &= \frac{k^{2}_{0\omega}}{{\symup{\varepsilon_0}}}\ \overset{\rightharpoonup\!\!\!\! \rightharpoonup}{\symbf{\mu}_{\symup{r}}}^{(1)\omega} \cdot \symbf P^{{\symup{NL}},\omega}_z \\ \nabla \cdot \left( \overset{\rightharpoonup\!\!\!\! \rightharpoonup}{\symbf{\varepsilon}_{\symup{r}}}^{(1)\omega}_{z} \cdot \symbf E^{\omega}_z \right) &= - \displaystyle{ \frac{1}{{\symup{\varepsilon_0}}} }\ \nabla \cdot \symbf P^{{\symup{NL}},\omega}_z \end{aligned}\right. ~.
\end{equation}

注意,在 Eq.(\ref{eq:2-16}) 中,波动方程与散度方程内的介电张量不同。在 \ref{纯电各向异性介质中线性光学均匀平面波解析解} 节会提到,Eq.(\ref{eq:2-16}) 中的散度方程只适用于不包含分界面的无限或半无限介质,但波动方程却还适用于包含分界面的不连续介质\cite{mcleodVectorFourierOptics2014,chenWavevectorspaceMethodWave1993,nelsonDerivingTransmissionReflection1995};因此,不论从空间的适用范围,还是从介电常数的广义性上讲,Eq.(\ref{eq:2-16}) 中的波动方程,均比散度方程更普适;甚至无需散度方程,就可仅通过直接求解 Eq.(\ref{eq:2-16}) 中的波动方程\cite{mcleodVectorFourierOptics2014}得出本征解,以及无需麦氏方程组 Eq.(\ref{eq:2-1}) 积分形式的边界条件,就可仅通过 Eq.(\ref{eq:2-16}) 中的波动方程,直接导出更正确的边界条件\cite{chenWavevectorspaceMethodWave1993,nelsonDerivingTransmissionReflection1995}(见 \ref{纯电各向异性介质中傅立叶线性光学解析解} 节)。

在彻底变换到波长域后,后续需将二者继续变换到时空频率域,对应介质内,某单一方向上传播的,单色变振幅平面波,所满足的波动方程、散度方程。

%\subsection{电磁线性均匀各向异性介质中的电场时空谱}
\subsection{\protect\hyperlink{chap:\thesubsection}{电磁线性均匀各向异性介质中的电场时空谱}}
\addtocontents{toc}{\protect\linkdest{chap:\thesubsection}}
\label{电磁线性均匀各向异性介质中的电场时空谱}

为此,定义横向空间频率 $\symbf k_{\symup{\rho}} \in \mathbb{R}^2$ 且 $\symbf k^\omega_{\symup{z}} \in \mathbb{C}$ 的复波矢\myHyperFootnote{由于傅立叶变换的缘故,一般在实验室三维直角坐标系下定义,且 $\symbfup{e}_{\symup{z}}$ 为晶体前端面内法向;见 \ref{复波矢的两种表示形式} 小节。}
\begin{equation} \label{eq:2-17}
	\symbf k^\omega := \symbf k_{\symup{\rho}} + \symbf k^\omega_{\symup{z}} := k_{\symup{x}} \symbfup{e}_{\symup{x}} + k_{\symup{y}} \symbfup{e}_{\symup{y}} + k^\omega_{\symup{z}} \symbfup{e}_{\symup{z}} ~,
\end{equation}
将电场 $\symbf E^{\omega}_z \left( \symbf \rho \right)$ 的时空谱 $\symbf G^{\omega}_z \left( \symbf k_{\symup{\rho}} \right)$ 在空间频率域的二维傅立叶积分(倒空间中各 $\symbf k_{\symup{\rho}}$ 方向单色等 $\omega$ 平面波的叠加)
\begin{subequations} \label{eq:2-18}
%	\abovedisplayskip=0pt
%	\belowdisplayskip=0pt
	\begin{align}
		\symbf E^{\omega}_z \left( \symbf \rho \right) = \mathcal F^{-1} \left[ \symbf G^{\omega}_z \right] &= \iint_{-\infty}^{+\infty} \symbf G^{\omega}_z \cdot {\mathbb{e}^{\mathbb{i} \symbf k_{\symup{\rho}} \cdot \symbf \rho}} ~ {\mathbb{d}{k_{\symup{x}}} \mathbb{d}{k_{\symup{y}}}} ~,\label{eq:2-18a}\\ \text{where}\ \ \ \ \ \ \symbf G^{\omega}_z \left( \symbf k_{\symup{\rho}} \right) := \mathcal F \left[ \symbf E^{\omega}_z \right] &:= \frac{1}{(2\pi)^2} \iint_{-\infty}^{+\infty} \symbf E^{\omega}_z \cdot {\mathbb{e}^{-\mathbb{i} \symbf k_{\symup{\rho}} \cdot \symbf \rho}} ~ {\mathbb{d}{x} \mathbb{d}{y}} ~,\label{eq:2-18b}
	\end{align}
\end{subequations}
代入电磁线性均匀\myHyperFootnote{即 Eq.(\ref{eq:2-16}) 中的 $\displaystyle{ \overset{\rightharpoonup\!\!\!\! \rightharpoonup}{\symbf{\mu}_{\symup{r}}}^{(1)}_{z}} \equiv \overset{\rightharpoonup\!\!\!\! \rightharpoonup}{\symbf{\mu}_{\symup{r}}}^{(1)}$、$\displaystyle{ \overset{\rightharpoonup\!\!\!\! \rightharpoonup}{\symbf{\varepsilon}_{\symup{r}}}^{(1)}_{z}} \equiv \overset{\rightharpoonup\!\!\!\! \rightharpoonup}{\symbf{\varepsilon}_{\symup{r}}}^{(1)}$、$\displaystyle{ \overset{\rightharpoonup\!\!\!\! \rightharpoonup}{\symbf{\sigma}_{\symup{r}}}_{z}} \equiv \overset{\rightharpoonup\!\!\!\! \rightharpoonup}{\symbf{\sigma}_{\symup{r}}}$、$\displaystyle{ \overset{\rightharpoonup\!\!\!\! \rightharpoonup}{\symbf{\varepsilon}'_{\symup{r}}}_{z}} \equiv \overset{\rightharpoonup\!\!\!\! \rightharpoonup}{\symbf{\varepsilon}'_{\symup{r}}}$ 均与空间坐标 $\symbf r$ 无关。}各向异性介质中单色电场非线性波动方程,及其散度约束:
\begin{equation} \label{eq:2-19}
	\left\{\ \begin{aligned} \nabla \left( \nabla \cdot \symbf E^{\omega}_z \right) - \nabla^2 \symbf E^{\omega}_z - k^{2}_{0\omega}\ \overset{\rightharpoonup\!\!\!\! \rightharpoonup}{\symbf{\mu}_{\symup{r}}}^{(1)\omega} \cdot \overset{\rightharpoonup\!\!\!\! \rightharpoonup}{\symbf{\varepsilon}'_{\symup{r}}}^{(1)\omega} \cdot \symbf E^{\omega}_z &= \frac{k^{2}_{0\omega}}{{\symup{\varepsilon_0}}}\ \overset{\rightharpoonup\!\!\!\! \rightharpoonup}{\symbf{\mu}_{\symup{r}}}^{(1)\omega} \cdot \symbf P^{{\symup{NL}},\omega}_z \\ \nabla \cdot \left( \overset{\rightharpoonup\!\!\!\! \rightharpoonup}{\symbf{\varepsilon}_{\symup{r}}}^{(1)\omega} \cdot \symbf E^{\omega}_z \right) &= - \displaystyle{ \frac{1}{{\symup{\varepsilon_0}}} }\ \nabla \cdot \symbf P^{{\symup{NL}},\omega}_z \end{aligned}\right. ~,
\end{equation}
并在空域上对 Eq.(\ref{eq:2-19}) 两边执行横向二维傅立叶变换 $\mathcal F \left[ \cdot \right]$,利用算符变换关系\myHyperFootnote{定义非伴随(非复共轭)行向量/算符 $ \bra{\symbf{\mathsfit{N}}} = \symbf{\mathsfit{N}}^{\symup{T}} := \mathbb{i} \symbf k_{\symup{\rho}} + \frac{ \partial }{ \partial z } \symbfup{e}_{\symup{z}} $,则内积/点积/数量积 $\braket{\symbf{\mathsfit{N}}}{\symbf{\mathsfit{N}}} = \symbf{\mathsfit{N}} \cdot \symbf{\mathsfit{N}} := \symbf{\mathsfit{N}}^{\symup{T}} \symbf{\mathsfit{N}} = \frac{ \partial^2 }{ \partial z^2 } - k^2_{\symup{\rho}}$,定义克罗内克积/并矢积/张量积 $\ketbra{\symbf{\mathsfit{N}}}{\symbf{\mathsfit{N}}} = \overset{\rightharpoonup\!\!\!\! \rightharpoonup}{\symbf{\mathsfit{N}} \symbf{\mathsfit{N}}} := \symbf{\mathsfit{N}} \otimes \symbf{\mathsfit{N}}^{\symup{T}}$。}
\begin{subequations} \label{eq:2-20}
\begin{align}
	\mathcal F \left[ \nabla \left( \nabla \cdot \mathcal F^{-1} \left[ \symbf G^{\omega}_z \right] \right) \right] &= \overset{\rightharpoonup\!\!\!\! \rightharpoonup}{\symbf{\mathsfit{N}} \symbf{\mathsfit{N}}} \cdot \symbf G^{\omega}_z ~, \label{eq:2-20a}\\ \mathcal F \left[ \nabla^2 \mathcal F^{-1} \left[ \symbf G^{\omega}_z \right] \right] &= \left( \symbf{\mathsfit{N}} \cdot \symbf{\mathsfit{N}} \right) \symbf G^{\omega}_z ~, \label{eq:2-20b}\\ \mathcal F \left[ \overset{\rightharpoonup\!\!\!\! \rightharpoonup}{\symbf{\mu}_{\symup{r}}}^{(1)\omega} \cdot \overset{\rightharpoonup\!\!\!\! \rightharpoonup}{\symbf{\varepsilon}'_{\symup{r}}}^{(1)\omega} \cdot \mathcal F^{-1} \left[ \symbf G^{\omega}_z \right] \right] &= \overset{\rightharpoonup\!\!\!\! \rightharpoonup}{\symbf{\mu}_{\symup{r}}}^{(1)\omega} \cdot \overset{\rightharpoonup\!\!\!\! \rightharpoonup}{\symbf{\varepsilon}'_{\symup{r}}}^{(1)\omega} \cdot \symbf G^{\omega}_z ~, \label{eq:2-20c}\\ \mathcal F \left[ \nabla \cdot \left( \overset{\rightharpoonup\!\!\!\! \rightharpoonup}{\symbf{\varepsilon}_{\symup{r}}}^{(1)\omega} \cdot \mathcal F^{-1} \left[ \symbf G^{\omega}_z \right] \right) \right] &= \symbf{\mathsfit{N}} \cdot \overset{\rightharpoonup\!\!\!\! \rightharpoonup}{\symbf{\varepsilon}_{\symup{r}}}^{(1)\omega} \cdot \symbf G^{\omega}_z ~, \label{eq:2-20d} \\
	\text{where}\ \ \ \ \ \ \bra{\symbf{\mathsfit{N}}} = \symbf{\mathsfit{N}}^{\symup{T}} &:= \mathbb{i} \symbf k_{\symup{\rho}} + \frac{ \partial }{ \partial z } \symbfup{e}_{\symup{z}} ~, \label{eq:2-20e}
\end{align}
\end{subequations}
得电磁线性均匀各向异性介质中的电场时空谱 $\symbf G^{\omega}_z$ 非线性波动方程及散度约束\myHyperFootnote{对方程右侧的操作与左侧类似:先将 $\symbf P^{{\symup{NL}},\omega}_z \left( \symbf \rho \right) \big/ {\symup{\varepsilon_0}}$ 写成 $\mathcal F^{-1} \left[ \symbf Q^{{\symup{NL}},\omega}_z \left( \symbf k_{\symup{\rho}} \right) \right]$,再 $\mathcal F \left[ \cdot \right]$ 作用于方程右侧整体。}:
\begin{subequations} \label{eq:2-21}
	\abovedisplayshortskip=0pt
	\abovedisplayskip=0pt
	\begin{align}
		\left\{\ \begin{aligned} \left( \overset{\rightharpoonup\!\!\!\! \rightharpoonup}{\symbf{\mathsfit{N}} \symbf{\mathsfit{N}}} \cdot - \symbf{\mathsfit{N}} \cdot \symbf{\mathsfit{N}} - k^{2}_{0\omega}\ \overset{\rightharpoonup\!\!\!\! \rightharpoonup}{\symbf{\mu}_{\symup{r}}}^{(1)\omega} \cdot \overset{\rightharpoonup\!\!\!\! \rightharpoonup}{\symbf{\varepsilon}'_{\symup{r}}}^{(1)\omega} \cdot \right) \symbf G^{\omega}_z &= k^{2}_{0\omega}\ \overset{\rightharpoonup\!\!\!\! \rightharpoonup}{\symbf{\mu}_{\symup{r}}}^{(1)\omega} \cdot \symbf Q^{{\symup{NL}},\omega}_z \\ \symbf{\mathsfit{N}} \cdot \overset{\rightharpoonup\!\!\!\! \rightharpoonup}{\symbf{\varepsilon}_{\symup{r}}}^{(1)\omega} \cdot \symbf G^{\omega}_z &= -\ \symbf{\mathsfit{N}} \cdot \symbf Q^{{\symup{NL}},\omega}_z \end{aligned}\right. \label{eq:2-21a}~, \\ \text{where}\ \ \ \ \ \ \symbf Q^{{\symup{NL}},\omega}_z \left( \symbf k_{\symup{\rho}} \right) := \mathcal F \left[ \symbf P^{{\symup{NL}},\omega}_z \left( \symbf \rho \right) \right] \big/ {\symup{\varepsilon_0}} \label{eq:2-21b}~, 
	\end{align}
\end{subequations}
或更严谨地,将 Eq.(\ref{eq:2-21a}) 记为左右态矢的形式:
\begin{equation} \label{eq:2-22}
    % \belowdisplayskip=0pt
	\left\{\ \begin{aligned} \left( \ketbra{\symbf{\mathsfit{N}}}{\symbf{\mathsfit{N}}} - \braket{\symbf{\mathsfit{N}}}{\symbf{\mathsfit{N}}} - k^{2}_{0\omega}\  \bar{\bar{\symbf{\mu}}}_{\symup{r}}^{(1)\omega} \bar{\bar{\symbf{\varepsilon}}}_{\symup{r}}^{\prime(1)\omega} \right) \ket{\symbf G^{\omega}_z} &= k^{2}_{0\omega}\ \bar{\bar{\symbf{\mu}}}_{\symup{r}}^{(1)\omega} \ket{\symbf Q^{{\symup{NL}},\omega}_z} \\ \bra{\symbf{\mathsfit{N}}} \bar{\bar{\symbf{\varepsilon}}}_{\symup{r}}^{(1)\omega} \ket{\symbf G^{\omega}_z} &= -\braket{\symbf{\mathsfit{N}}}{\symbf Q^{{\symup{NL}},\omega}_z} \end{aligned}\right. ~.
\end{equation}

将电场 $\symbf E^{\omega}_z$ 的时空谱 $\symbf G^{\omega}_z = \symbf g^{\omega}_z \cdot \mathbb{e}^{\mathbb{i} k^\omega_{\symup{z}} z}$ 分离出衍射因子后,代入 Eq.(\ref{eq:2-22}) 得:
\begin{subequations} \label{eq:2-23}
%	\abovedisplayshortskip=0pt
%	\abovedisplayskip=0pt
	\begin{align}
		&\!\!\!\! \left\{\ \begin{aligned} \left( \ketbra{\symbf{\mathsfit{n}}^\omega}{\symbf{\mathsfit{n}}^\omega} - \braket{\symbf{\mathsfit{n}}^\omega}{\symbf{\mathsfit{n}}^\omega} - k^{2}_{0\omega}\ \bar{\bar{\symbf{\mu}}}_{\symup{r}}^{(1)\omega} \bar{\bar{\symbf{\varepsilon}}}_{\symup{r}}^{\prime(1)\omega} \right) \ket{\symbf g^{\omega}_z} \mathbb{e}^{\mathbb{i} k^\omega_{\symup{z}} z} &= k^{2}_{0\omega}\ \bar{\bar{\symbf{\mu}}}_{\symup{r}}^{(1)\omega} \ket{\symbf Q^{{\symup{NL}},\omega}_z} \\ \bra{\symbf{\mathsfit{n}}^\omega} \bar{\bar{\symbf{\varepsilon}}}_{\symup{r}}^{(1)\omega} \ket{\symbf g^{\omega}_z} \mathbb{e}^{\mathbb{i} k^\omega_{\symup{z}} z} &= -\braket{\symbf{\mathsfit{N}}}{\symbf Q^{{\symup{NL}},\omega}_z} \end{aligned}\right. \label{eq:2-23a}~, \\ &\text{where}\ \ \ \ \ \ \bra{\symbf{\mathsfit{n}}^\omega} = \symbf{\mathsfit{n}}^{\symup{T}}_\omega := \mathbb{i}k_{\symup{x}} \symbfup{e}_{\symup{x}} + \mathbb{i}k_{\symup{y}} \symbfup{e}_{\symup{y}} + \left( \mathbb{i}k^\omega_{\symup{z}} \symbf + \frac{ \partial }{ \partial z } \right) \symbfup{e}_{\symup{z}} =: \mathbb{i} \symbf k^\omega + \frac{ \partial }{ \partial z } \symbfup{e}_{\symup{z}} \label{eq:2-23b}~, 
	\end{align}
\end{subequations}
其中 $\bra{\symbf{\mathsfit{n}}^\omega}$ 与 $\bra{\symbf{\mathsfit{N}}}$ 一样,是非伴随/非复共轭算符,且有:
\begin{subequations} \label{eq:2-24}
	%	\abovedisplayshortskip=0pt
	%	\abovedisplayskip=0pt
	\begin{align}
		&\left\{\ \begin{aligned} \ketbra{\symbf{\mathsfit{n}}^\omega}{\symbf{\mathsfit{n}}^\omega} - \braket{\symbf{\mathsfit{n}}^\omega}{\symbf{\mathsfit{n}}^\omega} &= \braket{\symbf k^\omega}{\symbf k^\omega} - \ketbra{\symbf k^\omega}{\symbf k^\omega} + \displaystyle{\frac{\partial}{\partial z}} \bar{\bar{\symbf V}}^\omega \\ \bra{\symbf{\mathsfit{n}}^\omega} \bar{\bar{\symbf{\varepsilon}}}_{\symup{r}}^{(1)\omega} &= \mathbb{i} \bra{\symbf k^\omega} \bar{\bar{\symbf{\varepsilon}}}_{\symup{r}}^{(1)\omega} + \displaystyle{\frac{\partial}{\partial z}} \bra{\symbf{\varepsilon}_{\symup{rz}}^{(1)\omega}} \end{aligned}\right. \label{eq:2-24a}~, \\ &\ \ \ \ \ \ \ \ \text{where}\ \ \ \ \ \ \bra{\symbf{\varepsilon}_{\symup{rz}}^{(1)\omega}} := \bra{\symbfup{e}_{\symup{z}}} \bar{\bar{\symbf{\varepsilon}}}_{\symup{r}}^{(1)\omega} \label{eq:2-24b}~, 
	\end{align}
\end{subequations}
其中 $\bra{\symbf{\varepsilon}_{\symup{rz}}^{(1)\omega}}$ 是个仍有 $\symbfup{e}_{\symup{x}}, \symbfup{e}_{\symup{y}}, \symbfup{e}_{\symup{z}}$ 三分量的定矢量,则 Eq.(\ref{eq:2-23a}) 变为:
\begin{equation} \label{eq:2-25}
	\left\{\ \begin{aligned} \left( \bar{\bar{\symbf L}}^\omega + \displaystyle{\frac{\partial}{\partial z}} \bar{\bar{\symbf V}}^\omega \right) \ket{\symbf g^{\omega}_z} \mathbb{e}^{\mathbb{i} k^\omega_{\symup{z}} z} &= k^{2}_{0\omega}\ \bar{\bar{\symbf{\mu}}}_{\symup{r}}^{(1)\omega} \ket{\symbf Q^{{\symup{NL}},\omega}_z} \\ \left( \mathbb{i} \bra{\symbf k^\omega} \bar{\bar{\symbf{\varepsilon}}}_{\symup{r}}^{(1)\omega} + \displaystyle{\frac{\partial}{\partial z}} \bra{\symbf{\varepsilon}_{\symup{rz}}^{(1)\omega}} \right) \ket{\symbf g^{\omega}_z} \mathbb{e}^{\mathbb{i} k^\omega_{\symup{z}} z} &= -\braket{\symbf{\mathsfit{N}}}{\symbf Q^{{\symup{NL}},\omega}_z} \end{aligned}\right. ~,
\end{equation}
且波动方程中,分别定义了线性算符 $\bar{\bar{\symbf L}}^\omega$、非线性算符\myHyperFootnote{也可称为“势能算符”,类比薛定谔方程中的势能项,将导致波函数不再是自由粒子的平面波形式,是“非线性”的来源,也是这里为什么用“V”表示的原因。—— 当然,也能与上面的“N”互换,以用“N”表示“Nonlinear”而非“Nabla”,用“V”表示“$\nabla$”而非势能项,但表示“nabla”的小写“v”像速度...} $\bar{\bar{\symbf V}}^\omega$:
\begin{subequations} \label{eq:2-26}
	\belowdisplayskip=0pt
	\begin{align}
		\bar{\bar{\symbf L}}^\omega &:= k^2_\omega - \ketbra{\symbf k^\omega}{\symbf k^\omega} - k^{2}_{0\omega}\ \bar{\bar{\symbf{\mu}}}_{\symup{r}}^{(1)\omega} \bar{\bar{\symbf{\varepsilon}}}_{\symup{r}}^{\prime(1)\omega} \label{eq:2-26a}\\ &\xrightarrow[]{\displaystyle{ k^2_\omega =: k^{2}_{0\omega} n^2_\omega}} k^2_\omega \left( \bar{\bar{\symbfup{I}}} - \ketbra{\hat{\symbf k}^\omega}{\hat{\symbf k}^\omega} - \frac{ \bar{\bar{\symbf{\mu}}}_{\symup{r}}^{(1)\omega} \bar{\bar{\symbf{\varepsilon}}}_{\symup{r}}^{\prime(1)\omega} }{ n^2_\omega } \right) ~,\label{eq:2-26b}
	\end{align}
\end{subequations}
\begin{equation} \label{eq:2-27}
	\abovedisplayskip=0pt
	\bar{\bar{\symbf V}}^\omega := \begin{pmatrix} \displaystyle{- \frac{\partial}{\partial z}} - 2 \mathbb{i} \symbf k^\omega_{\symup{z}} & 0 & \mathbb{i} \symbf k_{\symup{x}} \\ 0 & \displaystyle{- \frac{\partial}{\partial z}} - 2 \mathbb{i} \symbf k^\omega_{\symup{z}} & \mathbb{i} \symbf k_{\symup{y}} \\ \mathbb{i} \symbf k_{\symup{x}} & \mathbb{i} \symbf k_{\symup{y}} & 0\end{pmatrix} ~.
\end{equation}

若所考虑的电场组分,不参与晶体内的非线性光学过程,只在晶体中线性衍射、被晶体线性吸收/放大等,则 Eq.(\ref{eq:2-25}) 退化为无非线性驱动源的形式,其中:
\begin{equation} \label{eq:2-28}
	\left\{\ \begin{aligned} \bar{\bar{\symbf L}}^\omega \ket{\symbf g^{\omega}_z} &= \symbf 0 \\ \bra{\symbf k^\omega} \bar{\bar{\symbf{\varepsilon}}}_{\symup{r}}^{(1)\omega} \ket{\symbf g^{\omega}_z} &= 0 \end{aligned}\right. ~,
\end{equation}

若所考虑的电场组分,参与晶体内的非线性光学过程,那么这些组分由 Eq.(\ref{eq:2-25}) 描述;然而,若这些组分的频率不低于太赫兹,则它们一般是晶体中的行波,所以一般也全程参与晶体内的线性光学过程\myHyperFootnote{从来源上考虑,方程右侧的非线性驱动源,由三阶及以上张量及与之相互作用的电磁场组分组成;而二阶及以上的张量,在微观层面上的空间分布,是互相独立且局域的;所以非线性驱动源是互相独立且局域的;因此一旦这些源,产生并辐射出了电磁场的某些时空组分,则这些组分在晶体内的后续衍射与叠加,不再与产生它们的驱动源相关,而只受主宰线性光学过程的,晶体本身的二阶张量限制。},于是其同时也受 Eq.(\ref{eq:2-28}) 制约,甚至因此还必须优先满足它,以至于这些参与非线性光学过程的组分,也存在本征值、本征向量(偏振态),且由 Eq.(\ref{eq:2-28}) 决定。

二者的结合,导致参与晶体内非线性光学过程的太赫兹波段以上的组分,需要满足:
\begin{equation} \label{eq:2-29}
	\left\{\ \begin{aligned} \displaystyle{\frac{\partial}{\partial z}} \bar{\bar{\symbf V}}^\omega \ket{\symbf g^{\omega}_z} \mathbb{e}^{\mathbb{i} k^\omega_{\symup{z}} z} &= k^{2}_{0\omega}\ \bar{\bar{\symbf{\mu}}}_{\symup{r}}^{(1)\omega} \ket{\symbf Q^{{\symup{NL}},\omega}_z} \\ \displaystyle{\frac{\partial}{\partial z}} \bra{\symbf{\varepsilon}_{\symup{rz}}^{(1)\omega}} \ket{\symbf g^{\omega}_z} \mathbb{e}^{\mathbb{i} k^\omega_{\symup{z}} z} &= -\braket{\symbf{\mathsfit{N}}}{\symbf Q^{{\symup{NL}},\omega}_z} \end{aligned}\right. ~.
\end{equation}

综合来看,晶体中的非线性光学过程,如果所有波段均在太赫兹波段以上,则每个单色行波均须同时满足以下三个方程中的任意两者:\cref{eq:2-25,eq:2-28,eq:2-29},且一般使用后两者。

%\section{纯电各向异性介质中线性光学过程的解析解}
%\label{纯电各向异性介质中线性光学过程的解析解}
%\section{纯电各向异性介质中线性光学均匀平面波解析解}
\section{\protect\hyperlink{chap:\thesection}{纯电各向异性介质中线性光学均匀平面波解析解}}
\addtocontents{toc}{\protect\linkdest{chap:\thesection}}
\label{纯电各向异性介质中线性光学均匀平面波解析解}

在得到了涵盖了线性光学、非线性光学中,绝大部分主题、内容、现象(仅线性光学就包括但不限于:广义斯涅尔定律、广义菲涅尔方程、各向同性/单/双轴的线/圆双折射、负折射、自然/法拉第旋光/旋电/旋磁性、线/圆二向色性、双轴/手性/吸收/增益的各向异性、低/高阶/特征色散、相/群速度、晶体取向/泵浦方向、衍射、走离、锥光干涉、横向/纵向旋轨耦合、所有高阶线性衍射效应如圆锥衍射等\cite{mcleodVectorFourierOptics2014})的,电场混频方程组们 \cref{eq:2-28,eq:2-29} 后,接下来就需要求解它们。

然而,即使只考虑线性光学过程,在电磁线性均匀的条件下,且只限于纯电各向异性介质中,电场所满足的 Eq.(\ref{eq:2-28}) 的 4 个共享 $\symbf k_{\symup{\rho}}$ 的双折射双反射数值本征解\myHyperFootnote{直到特征方程都是解析的;但剩下的本质上只能数值求解,包括计算和区分两对正/反向本征值、本征向量,因此只有少数特殊情况可解析;而由于 Berry、Brenier 还给出了本征值、本征偏振态的解析形式,所以相比之下,Abdulhalim 还推导得不够彻底;尽管如此,这仍是包含反射的解析极限。},直到上世纪末(1999 年),才由 Abdulhalim 得到 \cite{abdulhalimAnalyticFormulaeRefractive1998};而 Eq.(\ref{eq:2-28}) 在任一方向上的 2 个解析本征解,直到本世纪初(2003 年),才由 Berry\cite{berryOpticalSingularitiesBirefringent2003} 得到\myHyperFootnote{尽管 Berry 得到的是 $\symbf d^{\omega}_z$ 的本征解,但其与 $\symbf g^{\omega}_z$ 共享本征值,且本征模也能相互转换(见下文)。};并同样由 Berry 于 2005 年拓展到电磁双各向异性\cite{berryOpticalSingularitiesBianisotropic2005};即使到目前(2025 年)为止,这也是晶体中线性光学的解析形式,所能走到的最远、最深、最高处,是理论上的皇冠、自然界的遗产;2000 年的光学史,200 年的近代光学史,人类文明一路发展到这,才能说我们大概懂光了、大概懂材料的光学性质了,尽管仍有一些问题没明晰\cite{lakhtakiaWhenDoesChoice2007,nelsonDerivingTransmissionReflection1995},且更广阔的晶体空间\cite{berryOpticalSingularitiesBianisotropic2005}、更丰富的复合材料\cite{mackayElectromagneticAnisotropyBianisotropy2019,mackayModernAnalyticalElectromagnetic2020}亟待探索。

尽管如此,由于 Berry 的重心,是晶体本身所给出的,单色均匀平面波的本征解;也就是说,不论是否向晶体内泵浦电磁场,都内禀不变的,晶体自身的性质;因此还剩下一些工作,诸如非均匀甚至非平面波泵浦(傅立叶光学),及其出入晶体两端面的边界条件、实验室 - 主轴坐标系变换等一些后续工作,需要跟进并补充完整。即使缺少一些拼图,“圣杯”已被以 Berry 为首的前辈们拿到\cite{olyslagerElectromagneticsExoticMedia2002}。

2016 年,Brenier 采用 Gerardin 等人的技术\cite{gerardinConditionsVoigtWave2001},通过以倒空间的各 $\mathup{Re} \left[ \hat{\symbf k}^\omega \right]$ 方向为 z 轴正向的三维笛卡尔直角动态/相对坐标系,也独立得到了纯电各向异性介质中 Eq.(\ref{eq:2-28}) 的电场解析本征解\cite{brenierLasingConicalDiffraction2016};该动态坐标系对于研究晶体中的线性光学,比 Berry 所采用的 $\hat{\symbf k}$ 空间的介电主轴球坐标系\myHyperFootnote{一种倒空间的静态/绝对坐标系,其下各张量元形式简洁且反映材料对称性,因此被大多数研究者采用。},更有优势\myHyperFootnote{以各 $\hat{\symbf k}^\omega$ 方向为 z 轴正向,可充分利用光场是横波的特性;同时,其动态坐标系通过牺牲所有分量非零,以将三阶张量简化为二阶张量,可进一步简化对线性光学的解析本征解的推导。};但可惜的是,所有的非介电主轴系都不适合以矢量的方式,研究晶体中的非线性光学\myHyperFootnote{Brenier 采取的 $\mathup{z}$ 轴依赖 $\mathup{Re} \left[ \hat{\symbf k}^\omega \right]$ 的动态坐标系,直接对张量而非矢量进行坐标变换,会使三阶、四阶非零张量元多至 27 个、81 个,使得对矢量非线性光学过程的计算更复杂。}。

尽管如此,由于 Brenier 利用傅立叶光学的思想,向晶体里泵浦非平面波,同时结合了端面边界条件\myHyperFootnote{尽管 Brenier 的边界条件因只是在坐标系变换而未用到 $\symbf g^{\omega}_z$ 的切向分量守恒而在严格意义上不正确。},以及实验室坐标系($\mathcal{Z}$ 系)与其 $\mathup{Re} \left[ \hat{\symbf k}^\omega \right]$ 坐标系的转换,使其理论,相比 Berry,更进一步贴近实验现象。

由于 Eq.(\ref{eq:2-28}) 特别是其中的波动方程的解析解难于获得,退而求其次,人们也尝试数值求解它。2014 年,McLeod 等人继承和发展了 Abdulhalim 等人的工作,在各向异性材料的矢量傅立叶光学综述中,将笛卡尔坐标系下的 ${\symbf k}^\omega = \symbf k_{\symup{\rho}} + \symbf k^\omega_{\symup{z}}$ 代入 Eq.(\ref{eq:2-28}) 中的波动方程,并设行列式为零,得出单色 Booker 四次曲面特征方程\cite{mcleodVectorFourierOptics2014};在无限空间的边界条件下,无需散度方程,直接数值求解 Booker 四次特征方程会得到四个本征解,分别拥有两个正向、两个反向的能流密度或群速度,对应折/透射与反射回同一媒介的两组行波,且四个反/透射波共享相同的横向波矢 $\symbf k_{\symup{\rho}}$,自动满足端面边界的相位连续条件。尽管该四个解共处于(界面同侧)同一个半无限空间,但这不妨碍 Eq.(\ref{eq:2-28}) 中的波动方程允许有介电常数 $\bar{\bar{\symbf{\varepsilon}}}_{\symup{r}z}^{\prime\omega}$ 不连续\myHyperFootnote{注意 $\bar{\bar{\symbf{\varepsilon}}}_{\symup{r}z}^{\prime\omega} = \bar{\bar{\symbf{\varepsilon}}}_{\symup{r}}^{\prime\omega} \cdot \mathup{step} \left( z - z_0 \right)$ 仍只乘上了 $\mathup{z}$ 向一维单位阶跃函数(且 $z_0$ 处未定义\cite{chenWavevectorspaceMethodWave1993,nelsonDerivingTransmissionReflection1995}),其他处仍不含 $\symbf{r} $。}处($z = z_0$)对应的交界面存在\cite{chenWavevectorspaceMethodWave1993,nelsonDerivingTransmissionReflection1995},并绕过散度方程\myHyperFootnote{不论散度方程 $\nabla \cdot \widetilde{\symbf D} = {\widetilde \rho}_{\symup{f}}$ 形式如何,均不会使用到它;那么也就是不仅允许两介质交界面上有自由电荷 ${\widetilde \rho}_{\symup{f}} \neq 0$,甚至允许各媒质体内有自由电荷源。},得到符合相位连续性边界条件、且符合任意 $\bar{\bar{\symbf{\varepsilon}}}_{\symup{r}}^{\omega}$ 材料本构关系的 4 个本征解。

%\subsection{复波矢的两种表示形式}
\subsection{\protect\hyperlink{chap:\thesubsection}{复波矢的两种表示形式}}
\addtocontents{toc}{\protect\linkdest{chap:\thesubsection}}
\label{复波矢的两种表示形式}

除非晶体无线/圆二向色性即透明(无吸收\myHyperFootnote{准确地说,是在材料的透明窗口内(导体中 $\symbf{J}^\omega_{{\symup{f}}z}$ 对应的经典带内跃迁之红外吸收、激光晶体中 $\symbf P^\omega_z$ 对应的量子带间跃迁之受激吸收,都在透明窗口外),且远离各种形式的小吸收峰。}或增益),否则在晶体内,单色平面波波矢 ${\symbf k}^\omega = \mathup{Re} \left[ \symbf k^\omega \right] + \mathbb{i} \cdot \mathup{Im} \left[ \symbf k^\omega \right]$ 一般是复/双矢量,其实虚部分别由一个实矢量构成;等价地说,其 $\mathup x,\mathup y,\mathup z$ 三分量均为复标量;而端面边界条件的关键,就在于单色平面波复波矢 ${\symbf k}^\omega \in \mathbb{C}^3$ 在两种不等价形式 $\symbf k_{\symup{\rho}} + \symbf k^\omega_{\symup{z}}$ 与 $k^\omega \hat{\symbf k}$ 间的转换。

复波矢的第一种表示形式 ${\symbf k}^\omega \left( \symbf k_{\symup{\rho}} \right) = \symbf k_{\symup{\rho}} + \symbf k^\omega_{\symup{z}} \left( \symbf k_{\symup{\rho}} \right)$ 中的 $\symbf k_{\symup{\rho}} \in \mathbb{R}^2$
是实的,可作为空域傅立叶正变换 Eq.(\ref{eq:2-18a}) 中实轴上的积分变量,因此与傅立叶光学兼容;另外,将其代入 Eq.(\ref{eq:2-28}) 中的波动方程后,可数值求解出 4 个复本征值 $k^\omega_{\symup{z}} \in \mathbb{C}$,这 4 个复标量场 $k^\omega_{\symup{z}} \left( \symbf k_{\symup{\rho}} \right)$ 的 8 个实虚部,共用同一个二维实矢量场 $\symbf k_{\symup{\rho}} \in \mathbb{R}^2$ 作为因变量,因此自动满足端面边界上的(复矢量场 $\symbf E^{\omega}_z$ 的)相位连续性条件。

但另一个随之而来的问题是,此时每个平面波的复波矢 ${\symbf k}^\omega = \symbf k_{\symup{\rho}} + \mathup{Re} \left[ \symbf k^\omega_{\symup{z}} \right] + \mathbb{i} \cdot \mathup{Im} \left[ \symbf k^\omega_{\symup{z}} \right]$ 没有单一的方向 $\hat{\symbf k}^\omega \in \mathbb{R}^3$ 可言(无法用某一实矢量表示):其 $\mathup z$ 分量(即本征值) $k^\omega_{\symup{z}} \in \mathbb{C}$ 是复的,但 $\mathup x,\mathup y$ 分量 $\{k_{\symup{x}}, k_{\symup{y}}\} \in \mathbb{R}$ 却是实的,导致相应平面波的相位传播与振幅衰减(增加),分别沿着不同方向的两个实矢量 $\mathup{Re} \left[ \symbf k^\omega \right] = \symbf k_{\symup{\rho}} + \mathup{Re} \left[ \symbf k^\omega_{\symup{z}} \right]$ 、 $\mathup{Im} \left[ \symbf k^\omega \right] = \mathup{Im} \left[ \symbf k^\omega_{\symup{z}} \right]$;并且此时 $\hat{\symbf k}^\omega := {\symbf k}^\omega \big/ k^\omega \in \mathbb{C}^3$ 的三分量均是复的。这也就意味着 Abdulhalim 在其两个开创性工作\cite{abdulhalimAnalyticFormulaeRefractive1998,abdulhalimExactMatrixMethod1999}中的角度相关陈述,只在无二向色性且无手性的普通双轴晶体内才有效,因为只有在这种条件下,标量场 $k^\omega_{\symup{z}} \left( \symbf k_{\symup{\rho}} \right) \in \mathbb{R}$ 才是实的,才可用某一实矢量 $\hat{\symbf k}^\omega \in \mathbb{R}^3$ 表示一个平面波的相位传播方向。

第二个继承下来的问题是,${\symbf k}^\omega \left( \symbf k_{\symup{\rho}} \right)$ 形式下,复波矢的傅立叶积分 Eq.(\ref{eq:2-18a}) 中的单色平面波 $\symbf G^{\omega}_{\symbf r} := \symbf G^{\omega}_z \cdot {\mathbb{e}^{\mathbb{i} \symbf k_{\symup{\rho}} \cdot \symbf \rho}} = \symbf g^{\omega} \cdot \mathbb{e}^{\mathbb{i} {\symbf k}^\omega \cdot \symbf r}$ 的等振幅面的法向总是 $\symup{z}$ 向的,对应 $\mathup{Im} \left[ \symbf k^\omega \right] = \mathup{Im} \left[ \symbf k^\omega_{\symup{z}} \right]$;也就是说,如果晶体有吸收/增益,则拥有 ${\symbf k}^\omega \left( \symbf k_{\symup{\rho}} \right)$ 形式复波矢的、晶体内任何传播方向(等相位面法向)的单色平面波,复振幅均只沿 $z$ 方向衰减(或增加),这需要进一步说明缘由。

Mcleod 等人采取了与 Abdulhalim 直接求解线性光学麦氏一阶微分方程组\myHyperFootnote{很像是从空间频率域入手的,但实际上 Abdulhalim 仍处理的是空域单色变振幅平面波,且采取的复波矢 ${\symbf k}^\omega$,形式上更像是 Berry 的 $k^\omega \hat{\symbf k}$,而不是 Mcleod 等人的 $\symbf k_{\symup{\rho}} + \symbf k^\omega_{\symup{z}}$,尽管特征方程与 Mcleod 等人一致。}等价的方法,将复波矢 ${\symbf k}^\omega \left( \symbf k_{\symup{\rho}} \right)$ 的这种形式代入 Eq.(\ref{eq:2-28}) 中的波动方程,并设行列式为零,二者均同时解决了全部本征值问题和部分边界条件问题;缺点是大部分情况下,只能数值求解所得的 Booker 四次特征方程,且需筛选出前/后向解,以致没有解析/显示/封闭表达式,就无法直接看出本征解物理意义的数学对应。

复波矢的第二种表示形式 ${\symbf k}^\omega \left( \hat{\symbf k} \right) = k^\omega \left( \hat{\symbf k} \right) \hat{\symbf k}$ 中的 $\hat{\symbf k} \in \mathbb{R}^3$ 对应 $\hat{\symbf k}$ 空间中某一方向的实单位矢量 $\hat{\symbf k} = \hat{k}_{\symup{x}} \symbfup{e}_{\symup{x}} + \hat{k}_{\symup{y}} \symbfup{e}_{\symup{y}} + \hat{k}_{\symup{z}} \symbfup{e}_{\symup{z}} =: \symbfup{e}_{\symup{r}}$,表示复波矢 ${\symbf k}^\omega$ 的相位传播方向 $\mathup{Re} \left[ \symbf k^\omega \right] = \mathup{Re} \left[ k^\omega \right] \hat{\symbf k}$,与振幅衰减(增益)方向 $\mathup{Im} \left[ \symbf k^\omega \right] = \mathup{Im} \left[ k^\omega \right] \hat{\symbf k}$ 相同,且由同一个实矢量决定,并且该实矢量(暂时\myHyperFootnote{在 Berry 所考虑的、无端面边界条件的情况下,复/双波矢的等振幅面与等相位面,才垂直于同一个实矢量,并且与频率无关。})与 $\omega$ 无关。将其代入 Eq.(\ref{eq:2-28}) 中的波动方程,并使用散度方程降秩后,可解析地解出 2 个复本征值 $k^\omega \in \mathbb{C}$,这 2 个复标量 $k^\omega \left( \hat{\symbf k} \right)$ 共用同一个三维实矢量\myHyperFootnote{但只有两个分量是独立的,与 Mcleod 等人形式的复波矢的独立分量个数相同。} $\hat{\symbf k} \in \mathbb{R}^3$ 作为因变量,因此 2 个本征波的 4 个等振幅面、等相位面,都共用相同的实矢量 $\hat{\symbf k} \in \mathbb{R}^3$ 作为法向。

但一个严重的问题便是:此时每个平面波复波矢 ${\symbf k}^\omega = \mathup{Re} \left[ k^\omega \right] \hat{\symbf k} + \mathup{Im} \left[ k^\omega \right] \hat{\symbf k}$ 的 $\mathup x,\mathup y$ 分量 $\{k_{\symup{x}}, k_{\symup{y}}\} = \{k^\omega \hat{k}_{\symup{x}}, k^\omega \hat{k}_{\symup{y}}\} \in \mathbb{C}$ 都不再是实的,无法作为空域傅立叶正变换 Eq.(\ref{eq:2-18a}) 中实轴上的积分变量,因此与傅立叶光学不兼容;而复波矢 ${\symbf k}^\omega \left( \hat{\symbf k} \right)$ 的 $\mathup z$ 分量 $k^\omega \left( \hat{\symbf k} \right) \hat{k}_{\symup{z}} \in \mathbb{C}$ 尽管也是复的,但与另一种形式 ${\symbf k}^\omega \left( \symbf k_{\symup{\rho}} \right)$ 下的 $\mathup z$ 分量 $k^\omega_{\symup{z}} \left( \symbf k_{\symup{\rho}} \right)$ 从自变量到因变量都不同;并且,对于方向 $\hat{\symbf k}$ 相同但复本征值 $k^\omega$ 不同的两个本征波,对应的两个复波矢的 $k_{\symup{x}}, k_{\symup{y}}$ 分量是不同的,因此不满足(复矢量场 $\symbf E^{\omega}_z$ 的)端面相位连续性边界条件,也即不是横向波矢相同的两个折射光。

第二个接踵而至的问题是,采取 ${\symbf k}^\omega \left( \hat{\symbf k} \right)$ 形式的单色平面波 $\symbf G^{\omega}_{\symbf r} = \symbf g^{\omega} \cdot \mathbb{e}^{\mathbb{i} {\symbf k}^\omega \cdot \symbf r}$ 的等振幅面的法向 $\mathup{Re} \left[ k^\omega \right] \hat{\symbf k}$ 与等相位面的法向 $\mathup{Im} \left[ k^\omega \right] \hat{\symbf k}$ 恒同向,吸收/增益介质中果真如此吗?这需要给出进一步的理由。

Berry 选择了 $k^\omega \left( \hat{\symbf k} \right) \hat{\symbf k}$ 形式的复波矢 ${\symbf k}^\omega \left( \hat{\symbf k} \right)$ 代入 Eq.(\ref{eq:2-28}) 中的波动方程和散度方程;由于 Berry 使用了散度方程,因此同样的方法无法处理带电荷源的介质;另外,须额外处理边界问题也是个缺点;但优点就在于本征值问题可解析。

因此,从以上两点来看,复波矢 ${\symbf k}^\omega$ 的这两种形式 $\symbf k_{\symup{\rho}} + \symbf k^\omega_{\symup{z}} \neq k^\omega \hat{\symbf k}$ 是彻底不等价的。但二者点乘 $\symbf{r}$ 后却可能相等:$\left( \symbf k_{\symup{\rho}} + \symbf k^\omega_{\symup{z}} \right) \cdot \symbf{r} = k^\omega \hat{\symbf k} \cdot \symbf{r}$,因此二者仍有相互转换的机会。另一方面,复波矢 ${\symbf k}^\omega$ 的两种表达方式 $\symbf k_{\symup{\rho}} + \symbf k^\omega_{\symup{z}}$ 与 $k^\omega \hat{\symbf k}$ 分别被 Mcleod 等人与 Berry 采取,而 Mcleod 等人与 Berry 分别从数值与解析的角度,都独立正确,这也说明二者之间有隐藏的联系尚未发现:我们的目标就是给 Berry 的解析解加上广义斯涅尔定律以及电场 $\symbf E^{\omega}_z$ 切向连续的边界条件,以得到 Mcleod 等人的四个傅立叶形式解中的两个透射解的解析形式;或者说,在 Mcleod 等人的基础上,完成两个透射解从数值到解析的过渡,亦即在 Berry 的基础上,完成边界条件从无到有、从平面波到任意单色场即傅立叶光学的过渡。

复波矢 ${\symbf k}^\omega$ 取不同形式时,算符 $\braket{\symbf k^\omega}{\symbf k^\omega}$、 $\ketbra{\symbf k^\omega}{\symbf k^\omega}$,及后续 Eq.(\ref{eq:2-31}) 中散度 $\braket{\symbf k^\omega}{\symbf d^\omega}$ 的形式也不同,导致波动方程中相关的线性、非线性算符 Eq.(\cref{eq:2-26,eq:2-27}),以及所有的散度方程,都需要相应地取不同形式。

以 $\braket{\symbf k^\omega}{\symbf k^\omega}$ 为例:在 $\symbf k_{\symup{\rho}} + \symbf k^\omega_{\symup{z}}$ 形式下,$\braket{\symbf k^\omega}{\symbf k^\omega}$ 最多只能写为 $k^2_\omega = k_{\symup{x}}^2 + k_{\symup{y}}^2 + k_{\omega {\symup{z}}}^2$,尽管它仍可以在形式上继续写作 $k^2_\omega = k_{\symup{x}}^2 + k_{\symup{y}}^2 + k_{\omega {\symup{z}}}^2 =: n^2_\omega k^{2}_{0\omega}$,但此时 $k^\omega, n^\omega$ 只与该形式下的本征值 $k^\omega_{\symup{z}}$ 间接相关,并且因自变量为 $\symbf k_{\symup{\rho}}$ 而不是单一方向的函数,进而与下述 $n^\omega \left( \hat{\symbf k} \right)$ 无关且无法互相转换;而在 $k^\omega \hat{\symbf k}$ 形式下,它不仅可直接进一步写作 $\displaystyle{ k^2_\omega =: n^2_\omega k^{2}_{0\omega} }$,而且此时 $k^\omega, n^\omega$ 与该形式下的本征值 $1 \big/ n^\omega$ 直接相关,并且因自变量为 $\hat{\symbf k}$ 而是单一方向的函数;因此只有当复波矢 ${\symbf k}^\omega$ 取 Berry 所采取的 $k^\omega \hat{\symbf k}$ 形式时,线性算符取 Eq.(\ref{eq:2-26b}) 中那样的最终形式才合理;否则 $\symbf k^\omega = \symbf k_{\symup{\rho}} + \symbf k^\omega_{\symup{z}}$ 对应的线性算符只能取 Eq.(\ref{eq:2-26a}) 的形式。

再以 $\braket{\symbf k^\omega}{\symbf d^\omega} = 0$ 为例:在 $\symbf k_{\symup{\rho}} + \symbf k^\omega_{\symup{z}}$ 形式下,它写作 $\braket{\symbf k^\omega}{\symbf d^\omega} = k_{\symup{x}} d^{\omega}_{\symup{x}} + k_{\symup{y}} d^{\omega}_{\symup{y}} + k^{\omega}_{\symup{z}} d^{\omega}_{\symup{z}} = 0$,其中除了 $\{k_{\symup{x}}, k_{\symup{y}}\} \in \mathbb{R}$ 是实的外,其他场分量 $\{k^{\omega}_{\symup{z}}, d^{\omega}_{\symup{xyz}}\} \in \mathbb{C}$ 全是复的;而在 $k^\omega \hat{\symbf k}$ 形式下,有 $\braket{\symbf k^\omega}{\symbf d^\omega} \big/ k^\omega = \hat{k}_{\symup{x}} d^{\omega}_{\symup{x}} + \hat{k}_{\symup{y}} d^{\omega}_{\symup{y}} + \hat{k}_{\symup{z}} d^{\omega}_{\symup{z}} = 0$,其中 $\{\hat{k}_{\symup{xyz}}\} \in \mathbb{R}$ 是实的,$\{d^{\omega}_{\symup{xyz}}\} \in \mathbb{C}$ 是复的。现假设两种 ${\symbf k}^\omega$ 形式下,二者 $\braket{\symbf k^\omega}{\symbf d^\omega}$ 中的 $d^{\omega}_{\symup{xyz}}$ 之比 $\in \mathbb{C}$ 为某一复常数、$k^{\omega}_{\symup{xy}} \big/ \hat{k}^{\omega}_{\symup{xy}} \in \mathbb{R}$ 为某一实常数,可推出 $k_{\symup{z}} \big/ \hat{k}^{\omega}_{\symup{z}} \in \mathbb{C}$ 为某一实常数,矛盾;因此要么 $k^{\omega}_{\symup{x}} \big/ \hat{k}^{\omega}_{\symup{x}} \neq k^{\omega}_{\symup{y}} \big/ \hat{k}^{\omega}_{\symup{y}}$ 不成比例,或者两种 ${\symbf k}^\omega$ 形式的 Eq.(\ref{eq:2-28}) 的本征向量 $\symbf d^{\omega}$ 不同;从结果来看,是支持后者的\cite{mcleodVectorFourierOptics2014},尽管前者也无所谓对错。

该 \ref{复波矢的两种表示形式} 小节的两种 ${\symbf k}^\omega$ 形式均处于同一坐标系下。但 $\symbf k_{\symup{\rho}} + \symbf k^\omega_{\symup{z}}$ 形式,原则上更适合以平行 $\symbfup{e}_{\symup{z}}$ 的方向为晶体端面法向的倒空间 $\mathcal{Z}$ 系,并且最好是三维直角/笛卡尔的,如果要将其与傅立叶光学结合,即如 \ref{电磁线性均匀各向异性介质中的电场时空谱} 小节的内容所示的话;而 $k^\omega \hat{\symbf k}$ 形式更适合搭配倒空间球坐标系,并且最好是晶体的介电主轴坐标系,以减少代数推导难度,且使物理内涵在数学上体现得更直接和本质。但原则上两种 ${\symbf k}^\omega$ 形式及与之相关的 $\bar{\bar{\symbf{\varepsilon}}}_{\symup{r}}^{\omega}$ 等,都可分别是任意坐标系下的;尽管如此,仍必须采用坐标系,以获取所有细节,而没有数学上技巧更先进无坐标法 \cite{matosAnisotropyTensorsNovel2007,matosConicalRefractionGeneralized2011,chenCoordinatefreeApproachWave1981,chenCoordinateFreeApproach1982}如几何代数法\cite{matosAnisotropyTensorsNovel2007,matosConicalRefractionGeneralized2011}解析得更彻底,尽管对其理论的具体实施也需要建立坐标系。

接下来的章节,均处在电磁线性均匀条件下(除边界处不连续外),但之后的非线性角谱理论,将允许对折射率进行微扰,而稍微突破这一限制。

%\subsection{实验室坐标系与晶体坐标系的矢量张量变换}
\subsection{\protect\hyperlink{chap:\thesubsection}{实验室坐标系与晶体坐标系的矢量张量变换}}
\addtocontents{toc}{\protect\linkdest{chap:\thesubsection}}
\label{实验室坐标系与晶体坐标系的矢量张量变换}

在无电流源\myHyperFootnote{电导率非零的情况,另定义 $\bar{\bar{\symbf{\eta}}}^{\prime\omega} := \bar{\bar{\symbf{\varepsilon}}}^{\prime-1}_{\symup{r}\omega}$ 替换掉 Eq.(\ref{eq:2-31}) 中的 $\bar{\bar{\symbf{\eta}}}^\omega$ 即可,其他步骤完全相同。}($\symbf{J}^\omega_{{\symup{f}}z} \equiv \symbf{0}$、$\bar{\bar{\symbf{\sigma}}}^{\omega} \equiv \bar{\bar{\symbf{0}}}$、$\bar{\bar{\symbf{\varepsilon}}}_{\symup{r}}^{\prime\omega} \equiv \bar{\bar{\symbf{\varepsilon}}}_{\symup{r}}^{\omega}$)、纯电各向异性($\bar{\bar{\symbf{\mu}}}_{\symup{r}}^{\omega} = \bar{\bar{\symbfup{I}}}$)条件\myHyperFootnote{线性光学部分,省略 $\bar{\bar{\symbf{\mu}}}_{\symup{r}}^{(1)\omega}, \bar{\bar{\symbf{\varepsilon}}}_{\symup{r}}^{\prime(1)\omega}, \bar{\bar{\symbf{\varepsilon}}}_{\symup{r}}^{(1)\omega}, \bar{\bar{\symbf{\eta}}}^{(1)\omega}$ 的上标 $"(1)"$,变为 $\bar{\bar{\symbf{\mu}}}_{\symup{r}}^{\omega}, \bar{\bar{\symbf{\varepsilon}}}_{\symup{r}}^{\prime\omega}, \bar{\bar{\symbf{\varepsilon}}}_{\symup{r}}^{\omega}, \bar{\bar{\symbf{\eta}}}^{\omega}$ 。}下\label{con:1},Eq.(\ref{eq:2-28}) 描述的线性光学过程,退化为\myHyperFootnote{可以看出 Eq.(\ref{eq:2-28}) 只控制了 $\ket{\symbf g^{\omega}_z}$ 的偏振态,因此无论 $\ket{\symbf g^{\omega}}$ 是否随 $z$ 变化,都满足 Eq.(\ref{eq:2-28})。这里并非先入为主地认为线性光学部分 $\ket{\symbf g^{\omega}}$ 与 $z$ 无关,而是从 Eq.(\ref{eq:2-28}) 的算符中不含 $z$ 就可见一斑。}:
\begin{equation} \label{eq:2-30}
	\left\{\ \begin{aligned} \left( \bar{\bar{\symbfup{I}}} - \ketbra{\hat{\symbf k}^\omega}{\hat{\symbf k}^\omega} - \frac{ \bar{\bar{\symbf{\varepsilon}}}_{\symup{r}}^{\omega} }{ n^2_\omega } \right) \ket{\symbf g^{\omega}} &= \symbf 0 \\ \bra{\hat{\symbf k}^\omega} \bar{\bar{\symbf{\varepsilon}}}_{\symup{r}}^{\omega} \ket{\symbf g^{\omega}} &= 0 \end{aligned}\right. ~,
\end{equation}
代入电位移场时空谱的偏振复矢量 $\mathcal F \left[ \symbf D^{\omega}_z \right] \big/ \mathbb{e}^{\mathbb{i} k^\omega_{\symup{z}} z} =: \symbf d^{\omega} = \bar{\bar{\symbf{\varepsilon}}}_{\symup{r}}^{\omega} \cdot \symbf g^{\omega} $,并使用 \ref{复波矢的两种表示形式} 小节中复波矢的 Berry 版表达式 ${\symbf k}^\omega \left( \hat{\symbf k} \right) = k^\omega \left( \hat{\symbf k} \right) \hat{\symbf k}$,且不再引入狄拉克符号\myHyperFootnote{这里不再使用量子力学中的狄拉克符号,原因是左态矢符号可能会引起伴随向量和左特征向量之间的混淆:在当晶体同时是手性且二向色性时,二者有区别,且这种区别很重要\cite{berryOpticalSingularitiesBirefringent2003}。},有:
\begin{equation} \label{eq:2-31}
	\left\{\ \begin{aligned} \left( \bar{\bar{\symbfup{I}}} - \hat{\symbf k} \hat{\symbf k} \right) \cdot \bar{\bar{\symbf{\eta}}}^{\omega} \cdot \symbf d^{\omega} &= \frac{ 1 }{ n^2_\omega } \symbf d^{\omega} \\ \hat{\symbf k} \cdot \symbf d^{\omega} &= 0 \end{aligned}\right. ~.
\end{equation}

该方程的成立,与坐标系的选择无关;但由于它起源自 Eq.(\ref{eq:2-18b}),而其中的空域二维傅立叶变换一般在 $\mathcal{Z}$ 系下执行,因此从 Eq.(\ref{eq:2-18b}) 到 Eq.(\ref{eq:2-31}) 间,以及后续不带下划线的所有矢量、张量,及他们的分量,都默认是 $\mathcal{Z}$ 系(下)的。即使在 $\mathcal{Z}$ 系下,方程组 Eq.(\ref{eq:2-31}) 仍有解析解,但其表达式是复杂的,导致物理意义难以捕捉;这归根到底是未选取最大程度简化材料本构张量表达式和体现其微观结构对称性的坐标系的缘故,为此先定义:
\begin{equation} \label{eq:2-32}
	\left\{\ \begin{aligned} \bar{\bar{\symbf{\varepsilon}}}_{\symup{r}}^{\omega} &:= \bar{\bar{\symbf{\epsilon}}}^{\omega} + \mathbb{i} \cdot {\symbf {\beta}^\omega} \times \\ \bar{\bar{\symbf{\eta}}}^\omega &:= \bar{\bar{\symbf{u}}}^\omega + \mathbb{i} \cdot {\symbf {\alpha}^\omega} \times = \bar{\bar{\symbf{\varepsilon}}}^{-1}_{\symup{r}\omega} \end{aligned}\right. ~,
\end{equation}
然后选取倒 $\left( \hat{\symbf k} \right)$ 空间中 $\bar{\bar{\symbf{\epsilon}}}_{\symup{R}}^{\omega} := \symup{Re} \left[ \bar{\bar{\symbf{\epsilon}}}^{\omega} \right]$ 的介电主轴系 $\bar{\bar{\underline{\symbf{\epsilon}}}}_{\symup{R}}^{\omega}$ 为晶体坐标系($\mathcal{C}$ 系)\myHyperFootnote{即三维直角 $\mathcal{C}$ 系的 $\symup{x}, \symup{y}, \symup{z}$ 轴分别为 $\bar{\bar{\underline{\symbf{\epsilon}}}}_{\symup{R}}^{\omega}$ 的 $\symup{xx}, \symup{yy}, \symup{zz}$ 张量元所对应的 $\symup{x}, \symup{y}, \symup{z}$ 即 $\symup{1}, \symup{2}, \symup{3}$ 或 $\symup{a}, \symup{b}, \symup{c}$ 或 $\symup{p}, \symup{m}, \symup{g}$ 椭球三主轴 $\bar{\underline{\symbfup{e}}}_{\Yup}:= \bar{\underline{\symbfup{e}}}_{\symup{xyz}} = \begin{pmatrix} \bar{\underline{\symbfup{e}}}_{\symup{x}} & \bar{\underline{\symbfup{e}}}_{\symup{y}} & \bar{\underline{\symbfup{e}}}_{\symup{z}} \end{pmatrix}^{\symup{T}}$;且这三个主轴方向上的折射率值的相对大小符合 Eq.(\ref{eq:2-34})。},与 Berry 的 $\bar{\bar{\symbf{u}}}_{\symup{R}}^{\omega} := \symup{Re} \left[ \bar{\bar{\symbf{u}}}^{\omega} \right]$ 主轴系 $\bar{\bar{\underline{\symbf{u}}}}_{\symup{R}}^{\omega}$ 有区别但不大,且并不严格按照其 $\underline{u}_1^\omega \geq \underline{u}_2^\omega \geq \underline{u}_3^\omega$ 即 $\underline{\epsilon}_{\symup{Ra}}^\omega \leq \underline{\epsilon}_{\symup{Rb}}^\omega \leq \underline{\epsilon}_{\symup{Rc}}^\omega$ 的规定(对于双轴晶体与 Berry 是一致的;但单轴并不一致):
\begin{equation} \label{eq:2-33}
	\left\{\ \begin{aligned} \text{neg.}&\ \text{uniaxial}\!\!\!\!\!\!&&:\ \underline{n}^{\omega}_{\symup{c}} = \underline{n}^{\omega}_{\symup{b}} > \underline{n}^{\omega}_{\symup{a}}\  \left( \underline{n}^{\omega}_{\symup{o}} = \underline{n}^{\omega}_{\symup{c}} \right) \\ \text{pos.}&\ \text{uniaxial}\!\!\!\!\!\!&&:\ \underline{n}^{\omega}_{\symup{c}} > \underline{n}^{\omega}_{\symup{b}} = \underline{n}^{\omega}_{\symup{a}}\ \left( \underline{n}^{\omega}_{\symup{o}} = \underline{n}^{\omega}_{\symup{a}} \right) \\ &\ \ \ \text{biaxial}\!\!\!\!\!\!&&:\ \underline{n}^{\omega}_{\symup{c}} > \underline{n}^{\omega}_{\symup{b}} > \underline{n}^{\omega}_{\symup{a}} \\ \text{neg.}&\ \ \ \!\!\!\!\!\!&&:\ \underline{n}^{\omega}_{\symup{c}} - \underline{n}^{\omega}_{\symup{b}} < \underline{n}^{\omega}_{\symup{b}} - \underline{n}^{\omega}_{\symup{a}} \\ \text{pos.}&\ \ \ \!\!\!\!\!\!&&:\ \underline{n}^{\omega}_{\symup{c}} - \underline{n}^{\omega}_{\symup{b}} > \underline{n}^{\omega}_{\symup{b}} - \underline{n}^{\omega}_{\symup{a}} \end{aligned}\right. ~,
\end{equation}
而是出于方便数值实验、兼容主流晶体光学、介质型双曲材料、非线性光学的考虑,选择了下述定义(其中允许 $\underline{\epsilon}^{\omega}_{\symup{Rc}} < 0$,以至 $\underline{n}^{\omega}_{\symup{c}}$ 可为纯虚数;但须有 $\underline{\epsilon}^{\omega}_{\symup{Ra}},\underline{\epsilon}^{\omega}_{\symup{Rb}} > 0$):
\begin{equation} \label{eq:2-34}
	\left\{\ \begin{aligned} \text{neg.}&\ \text{uniaxial}\!\!\!\!\!\!&&:\ \left| \underline{n}^{\omega}_{\symup{c}} \right| < \underline{n}^{\omega}_{\symup{b}} = \underline{n}^{\omega}_{\symup{a}} = \underline{n}^{\omega}_{\symup{o}} \\ \text{pos.}&\ \text{uniaxial}\!\!\!\!\!\!&&:\ \left| \underline{n}^{\omega}_{\symup{c}} \right| > \underline{n}^{\omega}_{\symup{b}} = \underline{n}^{\omega}_{\symup{a}} = \underline{n}^{\omega}_{\symup{o}} \\ &\ \ \ \text{biaxial}\!\!\!\!\!\!&&:\ \left| \underline{n}^{\omega}_{\symup{c}} \right| > \underline{n}^{\omega}_{\symup{b}} > \underline{n}^{\omega}_{\symup{a}} \\ \text{neg.}&\ \ \ \text{biaxial}\!\!\!\!\!\!&&:\ \left| \underline{n}^{\omega}_{\symup{c}} \right| - \underline{n}^{\omega}_{\symup{b}} < \underline{n}^{\omega}_{\symup{b}} - \underline{n}^{\omega}_{\symup{a}} \\ \text{pos.}&\ \ \ \text{biaxial}\!\!\!\!\!\!&&:\ \left| \underline{n}^{\omega}_{\symup{c}} \right| - \underline{n}^{\omega}_{\symup{b}} > \underline{n}^{\omega}_{\symup{b}} - \underline{n}^{\omega}_{\symup{a}} \end{aligned}\right. ~,
\end{equation}
以上均属于 $\bar{\bar{\underline{\symbf{\epsilon}}}}_{\symup{R}}^{\omega}, \bar{\bar{\underline{\symbf{u}}}}_{\symup{R}}^{\omega}$ 主对角元素,描述了(各向同性、单/双轴)晶体的线双折射。

定义了 $\mathcal{C}$ 系后,规定 Eq.(\ref{eq:2-31}) 之内且包括之外的所有矢量、张量,及他们的分量,只要由 $\mathcal{C}$ 系度量或本身就是 $\mathcal{C}$ 系的组分(如 $\underline{\symbfup{e}}_{\circleddash}:= \underline{\symbfup{e}}_{{\symup{r\theta\phi}}}$、$\underline{\symbfup{e}}_{\Yup}:= \underline{\symbfup{e}}_{{\symup{xyz}}}$ \myHyperFootnote{定义列向量 $\underline{\symbfup{e}}_{\circleddash}:= \underline{\symbfup{e}}_{{\symup{r\theta\phi}}} = \begin{pmatrix} \underline{\symbfup{e}}_{\symup{r}} & \underline{\symbfup{e}}_{\symup{\theta}} & \underline{\symbfup{e}}_{\symup{\phi}} \end{pmatrix}^{\symup{T}}$、$\underline{\symbfup{e}}_{\Yup}:= \underline{\symbfup{e}}_{\symup{xyz}} = \begin{pmatrix} \underline{\symbfup{e}}_{\symup{x}} & \underline{\symbfup{e}}_{\symup{y}} & \underline{\symbfup{e}}_{\symup{z}} \end{pmatrix}^{\symup{T}}$}),则在符号上均带一个下划线\myHyperFootnote{以表示其(各分量值)在 $\mathcal{C}$ 系下;若不带下划线,则默认是在 $\mathcal{Z}$ 系下,鉴于仍处于之前傅立叶光学框架下(即 Eq.(\ref{eq:2-18b}) 中的物理量没带下划线)的缘故;尽管 Eq.(\ref{eq:2-31}) 及其中各量在任何坐标系下都成立。},如 $\bar{\bar{\symbf{\varepsilon}}}_{\symup{r}}^{\omega}, \bar{\bar{\symbf{\eta}}}^\omega, \symbfup{e}_{\circleddash}, \symbfup{e}_{\Yup} \rightarrow \bar{\bar{\underline{\symbf{\varepsilon}}}}_{\symup{r}}^{\omega}, \bar{\bar{\underline{\symbf{\eta}}}}^{\omega}, \underline{\symbfup{e}}_{\circleddash}, \underline{\symbfup{e}}_{\Yup}$ 等;另外,设定 $\mathcal{C}$ 系在未旋转前与 $\mathcal{Z}$ 系重合,但根据晶体取向及切割的具体情况,$\mathcal{C}$ 系可相对 $\mathcal{Z}$ 系绕原点任意三维旋转,记作 $\underline{\symbfup{e}}_{{\symup{i'j'k'}}} = \bar{\bar{\underline{R}}} \cdot \symbfup{e}_{{\symup{ijk}}}$,以正交变换前后均是笛卡尔坐标系的三个单位矢量(${\symup{i'j'k'}} = {\symup{ijk}} = {\symup{xyz}} = \Yup$) $\underline{\symbfup{e}}_{\Yup} = \bar{\bar{\underline{R}}}_{\Yup} \cdot \symbfup{e}_{\Yup}$ \myHyperFootnote{$\bar{\bar{\underline{R}}}_{\Yup}$ 带双上划线但不加粗,表示形式上仍是二阶张量,但每个分量是标量;$\bar{\bar{\underline{R}}}_{\Yup}$ 的下划线表示从 $\mathcal{Z}$ 系到 $\mathcal{C}$ 系;$\bar{\bar{\underline{R}}}_{\Yup}$ 的下标 $\Yup$ 表示从笛卡尔到笛卡尔,是两个 $"\Yup\Yup"$ 的缩写。}为例,即
\begin{equation} \label{eq:2-35}
    \begin{pmatrix} \underline{\symbfup{e}}_{{\symup{x}}} \\ \underline{\symbfup{e}}_{{\symup{y}}} \\ \underline{\symbfup{e}}_{{\symup{z}}} \end{pmatrix} := \bar{\bar{\underline{R}}}_{\Yup} \cdot \begin{pmatrix} \symbfup{e}_{{\symup{x}}} \\ \symbfup{e}_{{\symup{y}}} \\ \symbfup{e}_{{\symup{z}}} \end{pmatrix} ~,
\end{equation}
其中旋转矩阵 $\bar{\bar{\underline{R}}}_{\Yup}$ 是个实正交矩阵\myHyperFootnote{满足 $\bar{\bar{\underline{R}}}_{\Yup}^{\symup{T}} = \bar{\bar{\underline{R}}}_{\Yup}^{-1} =: \bar{\bar{R}}_{\Yup}$,且其下各行/列向量两两间标准正交,也标准复正交(因为是实的)。},属于酉矩阵(但是实的\myHyperFootnote{但无需也不能是复的酉矩阵($\bar{\bar{\underline{R}}}_{\Yup}^{\dag} = \bar{\bar{\underline{R}}}_{\Yup}^{-1}$),因为变换前后 2 组坐标轴单位矢量及其所属空间是实的 $\mathbb{R}^3$。}),所以既能对代表 3 个坐标轴方向的单位实向量 $\symbfup{e}_{\Yup}$ 们实施旋转操作,也能旋转其对应的 3 个复标量(伪矢量/基系数) ${\bar{\nu}}_{\Yup}$ 们\myHyperFootnote{带个上划线但不加粗,表示形式上仍是列向量,但每个分量是标量:$\bar{\nu} = \begin{pmatrix} {\nu}_{\symup{i}} & {\nu}_{\symup{j}} & {\nu}_{\symup{k}} \end{pmatrix}^{\symup{T}}$,这样的伪矢量方便用于表述 $\symbf{\nu}_{\Yup}^{\symup{T}} = \bar{\nu}_{\Yup}^{\symup{T}} \cdot \symbfup{e}_{\Yup}$、计算机中的一维数组等;如果加粗了,则至少是一矢量、列向量;即使没有上划线,其各分量也至少是矢量:$\symbf{\nu} = \begin{pmatrix} \symbf{\nu}_{\symup{i}} & \symbf{\nu}_{\symup{j}} & \symbf{\nu}_{\symup{k}} \end{pmatrix}^{\symup{T}}$;另外,${\symbf{\nu}}$ 也代表任一复矢量,包括 $\hat{\symbf{k}}, {\symbf{g}}^{\omega}, {\symbf{d}}^{\omega}$ 等。},或分别单独对 ${\bar{\nu}}_{\Yup}$ 的实虚部进行旋转操作:根据矢量场的实在性,所继承的坐标变换不变性 $\underline{\symbf{\nu}}_{\Yup} = \symbf{\nu}_{\Yup}$\myHyperFootnote{场作为实体本身,与为描述场而采用的(数学)语言无关;另,这里本该但无需使用转置符号 $\symbf{\nu}_{\Yup}^{\symup{T}} = \bar{\nu}_{\Yup}^{\symup{T}} \cdot \symbfup{e}_{\Yup}$。},有 $\bar{\underline{\nu}}_{\Yup}^{\symup{T}} \cdot \underline{\symbfup{e}}_{\Yup} = \bar{\nu}_{\Yup}^{\symup{T}} \cdot \symbfup{e}_{\Yup}$,将 Eq.(\ref{eq:2-33}) 代入其内得 $\bar{\underline{\nu}}_{\Yup}^{\symup{T}} \cdot \bar{\bar{\underline{R}}}_{\Yup} \cdot \symbfup{e}_{\Yup} = \bar{\nu}_{\Yup}^{\symup{T}} \cdot \symbfup{e}_{\Yup}$,于是有 $\bar{\underline{\nu}}_{\Yup}^{\symup{T}} \cdot \bar{\bar{\underline{R}}}_{\Yup} = \bar{\nu}_{\Yup}^{\symup{T}}$,两侧取转置得 $\bar{\bar{R}}_{\Yup} \cdot \bar{\underline{\nu}}_{\Yup} = \bar{\nu}_{\Yup}$,左乘 $\bar{\bar{\underline{R}}}_{\Yup} := \bar{\bar{R}}_{\Yup}^{\symup{-1}} = \bar{\bar{R}}_{\Yup}^{\symup{T}}$,即有 $\bar{\underline{\nu}}_{\Yup} = \bar{\bar{\underline{R}}}_{\Yup} \cdot \bar{\nu}_{\Yup}$,也即
\begin{equation} \label{eq:2-36}
    \begin{pmatrix} \underline{\nu}_{{\symup{x}}} \\ \underline{\nu}_{{\symup{y}}} \\ \underline{\nu}_{{\symup{z}}} \end{pmatrix} := \bar{\bar{\underline{R}}}_{\Yup} \cdot \begin{pmatrix} {\nu}_{{\symup{x}}} \\ {\nu}_{{\symup{y}}} \\ {\nu}_{{\symup{z}}} \end{pmatrix} ~.
\end{equation}

一般地,$\symbfup{e}_{\Yup} \rightarrow \underline{\symbfup{e}}_{\Yup}$ 的旋转矩阵 $\bar{\bar{\underline{R}}}_{\Yup}$ 有多种形式,最常见的为潜在有万向节死锁(Gimbal Lock)问题的欧拉角(Euler Angles)方法,其次还有轴-角(Axis-Angle)法,以及四元数(Quaternion)表示。

但是,为避免可能存在的万向节死锁问题,以及更贴近球坐标系,这里未采取上述任何方法,转而采用球面三角学相关知识\myHyperFootnote{主要使用了:边的余弦定理、边角正弦定理、边的五元素公式。},先得到
\begin{align} \label{eq:2-37}
    \begin{pmatrix} \underline{\theta}_{\symbf{\nu}} \\ \underline{\phi}_{\symbf{\nu}} \end{pmatrix} &:= \bar{\bar{\underline{\mathcal{R}}}}_{\circleddash} \cdot \begin{pmatrix} \theta_{\symbf{\nu}} \\ \phi_{\symbf{\nu}} \end{pmatrix} \\ &= \begin{pmatrix} \arccos \left[ \cos{\theta_{\mathcal{C}}} \cos{\theta_{\symbf{\nu}}} + \sin{\theta_{\mathcal{C}}} \sin{\theta_{\symbf{\nu}}} \cos{\left( \phi_{\symbf{\nu}} - \phi_{\mathcal{C}} \right)} \right] \\ \text{arctan2} \left[ \sin{\left( \phi_{\symbf{\nu}} - \phi_{\mathcal{C}} \right)}, \cos{\theta_{\mathcal{C}}} \cos{\left( \phi_{\symbf{\nu}} - \phi_{\mathcal{C}} \right)} - \sin{\theta_{\mathcal{C}}} \cot{\theta_{\symbf{\nu}}} \right] - \underline{\phi}_{\mathcal{C}} \end{pmatrix} ~, \notag
\end{align}
其中采用了四象限反正切 $\text{arctan2} \left[ y,x \right]$ 保留完整的方位角信息;而$\bar{\bar{\underline{\mathcal{R}}}}_{\circleddash}$ 表示:球坐标的 $\mathcal{Z}$ 系下,任意某实/复矢量 $\symbf{\nu}_{\circleddash}$ 的单位方向实矢量\myHyperFootnote{如果是复矢量,则指其实部或虚部的单位方向实矢量。}的极角 $\theta_{\symbf{\nu}}$、方位角 $\phi_{\symbf{\nu}}$,到其在球坐标的 $\mathcal{C}$ 系下的 $\underline{\theta}_{\symbf{\nu}}, \underline{\phi}_{\symbf{\nu}}$ 的变换矩阵;如 Eq.(\ref{eq:2-35}) 所示,它没有显示表达式,以至于不是个矩阵,所以不是个正交矩阵,更不是个旋转矩阵,因此只是一个等价于二阶张量的算符或函数(用花体表示),除了需要输入输出的参数 $\theta_{\symbf{\nu}}, \phi_{\symbf{\nu}}, \underline{\theta}_{\symbf{\nu}}, \underline{\phi}_{\symbf{\nu}}$ 外,它只与 $\mathcal{C}$ 系相对 $\mathcal{Z}$ 系的极角 $\theta_{\mathcal{C}}$、方位角 $\phi_{\mathcal{C}}$,以及其相对于自己的自转角 $\underline{\phi}_{\mathcal{C}}$ 有关;注意,类似欧拉角法中的顺规,这三个转角执行的先后顺序有多种,但采取某种后就固定不变:规定当 $\theta_{\mathcal{C}},\phi_{\mathcal{C}},\underline{\phi}_{\mathcal{C}} = 0,0,0$ 时,$\mathcal{C}$ 系与 $\mathcal{Z}$ 系重合;然后 $\mathcal{C}$ 系的 $\underline{\symup{z}}$ 轴先绕 $\mathcal{Z}$ 系的 $\symup{y}$ 轴,沿 $\mathcal{Z}$ 系的 $\symup{z} \rightarrow \symup{x}$ 方向,相对于 $\mathcal{Z}$ 系的 $\symup{z}$ 轴旋转 $\theta_{\mathcal{C}}$;接着 $\mathcal{C}$ 系的 $\underline{\symup{z}} - \underline{\symup{x}}$ 平面绕 $\mathcal{Z}$ 系的 $\symup{z}$ 轴,沿 $\mathcal{Z}$ 系的 $\symup{x} \rightarrow \symup{y}$ 方向,旋转 $\phi_{\mathcal{C}}$;最后 $\mathcal{C}$ 系绕自己的 $\underline{\symup{z}}$ 轴,沿 $\mathcal{C}$ 系的 $\underline{\symup{x}} \rightarrow \underline{\symup{y}}$ 方向,自转 $\underline{\phi}_{\mathcal{C}}$。

可以证明,这种方法即使在非连续步进的条件下,也没有万向节死锁问题:因为混用了内旋(右乘/本地/动态/自转)与外旋(左乘/世界/静态/公转):先绕两次世界坐标系 $\mathcal{Z}$ 公转,再绕本地坐标系 $\mathcal{C}$ 自转;而不像大多数欧拉角方法,三次都绕本地坐标系 $\mathcal{C}$ ,或三次都绕世界坐标系 $\mathcal{Z}$ 自转,导致同一个姿态可能对应多组不同的欧拉角,不唯一就无法反解、信息冗余导致秩降低至 2,丢失了 1 个自由度。而同一个姿态,通过该球面三角的方法,能唯一确定一组 $\theta_{\mathcal{C}},\phi_{\mathcal{C}},\underline{\phi}_{\mathcal{C}}$,信息没有缺失/不足或冗余,尽管似乎根据姿态反解 $\theta_{\mathcal{C}},\phi_{\mathcal{C}},\underline{\phi}_{\mathcal{C}}$ 仍略有困难\myHyperFootnote{其实并不困难,即使没有 $\bar{\bar{\underline{\mathcal{R}}}}_{\circleddash}$ 的显示表达式:仍然通过球面三角的方法,原则上可以已知 $\theta_{\symbf{\nu}}, \phi_{\symbf{\nu}}, \underline{\theta}_{\symbf{\nu}}, \underline{\phi}_{\symbf{\nu}}$,反向求解出 $\theta_{\mathcal{C}},\phi_{\mathcal{C}},\underline{\phi}_{\mathcal{C}}$。但没必要:只需正向($\theta_{\symbf{\nu}}, \phi_{\symbf{\nu}} + \theta_{\mathcal{C}},\phi_{\mathcal{C}},\underline{\phi}_{\mathcal{C}} \rightarrow \underline{\theta}_{\symbf{\nu}}, \underline{\phi}_{\symbf{\nu}}$ 或 $\underline{\theta}_{\symbf{\nu}}, \underline{\phi}_{\symbf{\nu}} + \theta_{\mathcal{C}},\phi_{\mathcal{C}},\underline{\phi}_{\mathcal{C}} \rightarrow \theta_{\symbf{\nu}}, \phi_{\symbf{\nu}}$)一一对应即可,反向映射的函数除了用于证明是双射外,暂时没有其他帮助。}。

继 Eq.(\ref{eq:2-37}) 之后,还需要将其中表示球坐标系方向的伪矢量 $\bar{\circleddash}_{\symbf{\nu}} := \begin{pmatrix} \theta_{\symbf{\nu}} & \phi_{\symbf{\nu}} \end{pmatrix}^{\symup{T}}$,与 Eq.(\ref{eq:2-36}) 中伪矢量 ${\bar{\nu}}_{\Yup}$ 的单位形式 ${\hat{\nu}}_{\Yup} = {\bar{\nu}}_{\Yup} \big/ \left| {\bar{\nu}}_{\Yup} \right|$ 相互转换:
\begin{subequations} \label{eq:2-38}
	\belowdisplayskip=10pt
	\begin{align}
		\begin{pmatrix} \theta_{\symbf{\nu}} \\ \phi_{\symbf{\nu}} \end{pmatrix} &:= \bar{\bar{\mathcal{T}}}_{\circleddash} \cdot \begin{pmatrix} \hat{\nu}_{{\symup{x}}} \\ \hat{\nu}_{{\symup{y}}} \\ \hat{\nu}_{{\symup{z}}} \end{pmatrix} = \begin{pmatrix} \arccos{\hat{\nu}_{{\symup{z}}}} \\ \text{arctan2}\left[\hat{\nu}_{{\symup{y}}}, \hat{\nu}_{{\symup{x}}}\right] \end{pmatrix} \label{eq:2-38a}~, \\
		\begin{pmatrix} \hat{\nu}_{{\symup{x}}} \\ \hat{\nu}_{{\symup{y}}} \\ \hat{\nu}_{{\symup{z}}} \end{pmatrix} &:= \bar{\bar{\mathcal{T}}}_{\Yup} \cdot \begin{pmatrix} \theta_{\symbf{\nu}} \\ \phi_{\symbf{\nu}} \end{pmatrix} = \begin{pmatrix} \sin{\theta_{\symbf{\nu}}} \cos{\phi_{\symbf{\nu}}} \\ \sin{\theta_{\symbf{\nu}}} \sin{\phi_{\symbf{\nu}}} \\ \cos{\theta_{\symbf{\nu}}} \end{pmatrix} \label{eq:2-38b}~,
	\end{align}
\end{subequations}
注意 Eq(\ref{eq:2-38}) 在两个坐标系 $\mathcal{C,Z}$ 下都分别单独适用\myHyperFootnote{并不涉及坐标系间的转换:这已经由 Eq.(\ref{eq:2-37}) 描述;只是坐标系内的变换。},即在 $\mathcal{Z}$ 系下有:$\bar{\circleddash}_{\symbf{\nu}} = \bar{\bar{\mathcal{T}}}_{\circleddash} \cdot {\bar{\nu}}_{\Yup}$、${\bar{\nu}}_{\Yup} = \bar{\bar{\mathcal{T}}}_{\Yup} \cdot \bar{\circleddash}_{\symbf{\nu}}$,而在 $\mathcal{C}$ 系下有:$\bar{\underline{\circleddash}}_{\symbf{\nu}} = \bar{\bar{\underline{\mathcal{T}}}}_{\circleddash} \cdot {\bar{\underline{\nu}}}_{\Yup}$、${\bar{\underline{\nu}}}_{\Yup} = \bar{\bar{\underline{\mathcal{T}}}}_{\Yup} \cdot \bar{\underline{\circleddash}}_{\symbf{\nu}}$,且两对二阶\myHyperFootnote{双上划线表示(输入输出类似)二阶张量: $\bar{\bar{\mathcal{T}}}_{\circleddash}$ 是 $2 \times 3$ 维;$\bar{\bar{\mathcal{T}}}_{\Yup}$ 是 $3 \times 2$ 维(输入 2 个,输出 3 个参数)。}算符函数 $\bar{\bar{\mathcal{T}}}_{\circleddash} \cdot \bar{\bar{\mathcal{T}}}_{\Yup} = \bar{\bar{\symup{I}}}$、$\bar{\bar{\underline{\mathcal{T}}}}_{\circleddash} \cdot \bar{\bar{\underline{\mathcal{T}}}}_{\Yup} = \bar{\bar{\symup{I}}}$ 每对内部互为逆。

给出两类算符函数 $\mathcal{R}, \mathcal{T}$ 后,就可写出 Eq.(\ref{eq:2-36}) 中旋转矩阵 $\bar{\bar{\underline{R}}}_{\Yup}$ 的表达式了:
\begin{equation} \label{eq:2-39}
	\bar{\bar{\underline{R}}}_{\Yup} = \bar{\bar{\underline{\mathcal{T}}}}_{\Yup} \cdot \bar{\bar{\underline{\mathcal{R}}}}_{\circleddash} \cdot \bar{\bar{\mathcal{T}}}_{\circleddash} ~,
\end{equation}
一般的实验条件下,$\mathcal{C}$ 系相对于 $\mathcal{Z}$ 系的姿态是确定的,因此 $\bar{\bar{\underline{R}}}_{\Yup}$ 是不变的;那么针对不同的实伪向量(场) ${\bar{\nu}}_{\Yup}$,不必每次都调用函数过程 $\bar{\bar{\underline{\mathcal{T}}}}_{\Yup} \cdot \bar{\bar{\underline{\mathcal{R}}}}_{\circleddash} \cdot \bar{\bar{\mathcal{T}}}_{\circleddash} \cdot {\bar{\nu}}_{\Yup}$ 以获得 ${\bar{\underline{\nu}}}_{\Yup}$;而只需要先进行一次函数过程,即将 $\bar{\bar{\underline{\mathcal{T}}}}_{\Yup} \cdot \bar{\bar{\underline{\mathcal{R}}}}_{\circleddash} \cdot \bar{\bar{\mathcal{T}}}_{\circleddash} \cdot$ 作用于 $\bar{\bar{\symup{I}}}$ 以获得 $\bar{\bar{\underline{R}}}_{\Yup}$ 的 9 个分量值并记录下来,再用记录在内存中\myHyperFootnote{由于只是个张量而不是个张量场,所以也不怎么占内存,只是个 9 元素的二维复数组。}的 $\bar{\bar{\underline{R}}}_{\Yup}$ 作用于其他伪矢量(场) ${\bar{\nu}}_{\Yup}$(注意这不是个函数操作,只是矩阵乘法,因此很快),即可获得它们在 $\mathcal{C}$ 系下的值(分布)${\bar{\underline{\nu}}}_{\Yup} = \bar{\bar{\underline{R}}}_{\Yup} \cdot {\bar{\nu}}_{\Yup}$,其中在内存中记录的 $3 \times 3$ 二维实数数组为:%\bar{\bar{\underline{R}}}_{\Yup} = \bar{\bar{\underline{\mathcal{T}}}}_{\Yup} \cdot \bar{\bar{\underline{\mathcal{R}}}}_{\circleddash} \cdot \bar{\bar{\mathcal{T}}}_{\circleddash} \cdot \bar{\bar{\symup{I}}}
\begin{equation} \label{eq:2-40}
	\bar{\bar{\underline{R}}}_{\Yup} = \bar{\bar{\underline{R}}}_{\Yup} \cdot \bar{\bar{\symup{I}}} = \bar{\bar{\underline{\mathcal{T}}}}_{\Yup} \cdot \bar{\bar{\underline{\mathcal{R}}}}_{\circleddash} \cdot \bar{\bar{\mathcal{T}}}_{\circleddash} \cdot \bar{\bar{\symup{I}}} ~,
\end{equation}
从程序的角度,该式 Eq.(\ref{eq:2-40}) 相对于 Eq.(\ref{eq:2-39}) 的区别,就在于 Eq.(\ref{eq:2-39}) 是个定义好的,尚未被调用、未传入参数的函数,而 Eq.(\ref{eq:2-40}) 是向函数内传入参数 $"\bar{\bar{\symup{I}}}"$ 后,经运算并返回的结果,这是以极少空间换大量时间的技巧之一。

定义好了 $\mathcal{C}$ 系、熟悉了坐标基以及基系数(矢量各分量)在 $\mathcal{Z,C}$ 系间与系内的变换后,才能继续讨论 Eq.(\ref{eq:2-32}) 内的张量相关内容。在 Eq.(\ref{eq:2-32}) 的 $\bar{\bar{\symbf{\eta}}}^\omega = \bar{\bar{\symbf{\varepsilon}}}^{-1}_{\symup{r}\omega} := \bar{\bar{\symbf{u}}}^\omega + \mathbb{i} \cdot {\symbf {\alpha}^\omega} \times$ 中,${\symbf {\alpha}^\omega}$ 为一轴矢量,称为光学活性矢量:对于自然光学活性,${\symbf {\alpha}^\omega} := \bar{\bar{\symbf{\gamma}}}^\omega \cdot \hat{\symbf k}$;对于法拉第光学活性,${\symbf {\alpha}^\omega} := \bar{\bar{\symbf{\gamma}}}^\omega \cdot {\symbfup H}_{\symup{ex}}$(在任何坐标系下 ${\symbf {\alpha}^\omega} \times$ 都是反对称的\myHyperFootnote{二阶反对称张量等价于某一轴矢量的叉乘且秩为二\cite{landauCHAPTERXIELECTROMAGNETIC1984}(剩一个特征值为零,相应特征向量即为轴矢量)。}),其中 ${\symbfup H}_{\symup{ex}}$ 为外磁场、$\bar{\bar{\symbf{\gamma}}}^\omega$ 是二阶光学活性张量(在任何系下 $\bar{\bar{\symbf{\gamma}}}^\omega$ 都是对称的\myHyperFootnote{根据 Takagi 分解定理,复对称矩阵 $\bar{\bar{\symbf{\gamma}}}$ 总能酉相似对角化 $\bar{\bar{\symbf{U}}}^\dag \bar{\bar{\symbf{\gamma}}} \bar{\bar{\symbf{U}}} = \bar{\bar{\symbf{U}}}^{-1} \bar{\bar{\symbf{\gamma}}} \bar{\bar{\symbf{U}}} = \bar{\bar{\symbf{\Lambda}}}$ 为一个非负实对角阵。}),$\bar{\bar{\symbf{\gamma}}}^\omega$ 的实部 $\bar{\bar{\symbf{\gamma}}}^\omega_{\symup{R}}$ 描述了材料的旋光/手性/光学活性/圆双折射性质,虚部 $\bar{\bar{\symbf{\gamma}}}^\omega_{\symup{I}}$ 则对应圆二向色性,与描述线二向色性的 $\bar{\bar{\symbf{u}}}^\omega$ 的虚部 $\bar{\bar{\symbf{u}}}_{\symup{I}}^\omega$,共同描述材料的二向色性(在任何系下 $\bar{\bar{\symbf{u}}}_{\symup{I}}^\omega$ 也是对称的\cite{XieQuanMianHuiYiWas}),因此晶体空间一共有 $15=3+6+6$ 个材料参数自由度\myHyperFootnote{按理应该是 $21=3+6\times2+6$ 个才对,如果 $\bar{\bar{\symbf{\gamma}}}^\omega_{\symup{I}}$ 独立于 $\bar{\bar{\symbf{\gamma}}}^\omega_{\symup{R}}$ 地,像 $\bar{\bar{\symbf{u}}}^\omega_{\symup{I}}$ 一样也有 6 个自由度的话;可能 Berry 考虑了晶体对称性对 $\bar{\bar{\symbf{\gamma}}}^\omega$ 的限制\cite{haussuhlPhysicalPropertiesCrystals2007}。},与 Berry 在其纯电各向异性的理论工作\cite{berryOpticalSingularitiesBirefringent2003}中的定义一致。

当 $\symup{Im} \left[ \bar{\bar{\symbf{u}}}^{\omega} \right] =: \bar{\bar{\symbf{u}}}_{\symup{I}}^{\omega} << \bar{\bar{\symbf{u}}}_{\symup{R}}^{\omega}$ 且 $\left| \symbf{\alpha}^\omega \right| << 1$ 时\myHyperFootnote{或在 $\symup{Im} \left[ \bar{\bar{\symbf{\epsilon}}}^{\omega} \right] =: \bar{\bar{\symbf{\epsilon}}}_{\symup{I}}^{\omega} << \bar{\bar{\symbf{\epsilon}}}_{\symup{R}}^{\omega}$ 且 $\left| \symbf{\beta}^\omega \right| << 1$ 的条件下。},可给出 $\bar{\bar{\symbf{u}}}^{\omega}$ 与 $\bar{\bar{\symbf{\epsilon}}}^{\omega}$,以及 $\symbf{\alpha}^{\omega}$ 与 $\symbf{\beta}^{\omega}$ 间的转换关系:
\begin{subequations} \label{eq:2-41}
	\begin{align}
		&\begin{cases} \label{eq:2-41a}
			\bar{\bar{\symbf{u}}}^{\omega} = \bar{\bar{\symbf{\epsilon}}}_{\omega}^{-1} \\ \symbf{\alpha}^{\omega} = \displaystyle{ \frac{\bar{\bar{\symbf{\epsilon}}}^{\omega}}{\det \left[ \bar{\bar{\symbf{\epsilon}}}^{\omega} \right]} } \cdot \symbf{\beta}^{\omega}  ~,
		\end{cases}\\
		&\begin{cases} \label{eq:2-41b}
			\bar{\bar{\symbf{\epsilon}}}^{\omega} = \bar{\bar{\symbf{u}}}_{\omega}^{-1} \\ \symbf{\beta}^{\omega} = \displaystyle{ \frac{\bar{\bar{\symbf{u}}}^{\omega}}{\det \left[ \bar{\bar{\symbf{u}}}^{\omega} \right]} } \cdot \symbf{\alpha}^{\omega}  ~.
		\end{cases}
	\end{align}
\end{subequations}

注意,不像矢量,我们不关注张量的分量在 $\mathcal{Z,C}$ 系之间的坐标变换关系(但 \ref{晶体/实验室坐标系下线性光学过程的解析解} 小节会关注张量在同一系内的直角-球坐标变换关系)。一方面是因为只需要对矢量或张量二者之一变换即可,且对各自的变换是等价的;另一方面是因为对张量的变换相对于矢量,更复杂、耗时,甚至导致非零张量元增多(比如通常在 $\mathcal{C} \rightarrow \mathcal{Z}$ 系后),以至于在二阶及以上的矢量非线性光学中,对三阶及以上的高阶张量元的旋转变换,是不明智甚至不切实际的。

至此,已经完成了对晶体坐标系 $\mathcal{C}$ 系的定义,以及对基矢量、复矢量在 $\mathcal{Z,C}$ 系间旋转变换和系内直角-球坐标系的坐标变换、正逆介电张量在系内的转换的介绍。接下来工作,就基于 $\mathcal{C}$ 系下的 Eq.(\ref{eq:2-31}),开始对其进行解析求解。

%\subsection{晶体/实验室坐标系下线性光学过程的解析解}
\subsection{\protect\hyperlink{chap:\thesubsection}{晶体/实验室坐标系下线性光学过程的解析解}}
\addtocontents{toc}{\protect\linkdest{chap:\thesubsection}}
\label{晶体/实验室坐标系下线性光学过程的解析解}

在(直角或球坐标的)晶体坐标系 $\mathcal{C}$ 系下,Eq.(\ref{eq:2-31}) 变为:
\begin{equation} \label{eq:2-42}
	\left\{\ \begin{aligned} \left( \bar{\bar{\symbfup{I}}} - \hat{\underline{\symbf k}} \hat{\underline{\symbf k}} \right) \cdot \bar{\bar{\underline{\symbf{\eta}}}}^{\omega} \cdot \underline{\symbf d}^{\omega} &= \frac{ 1 }{ n^2_\omega } \underline{\symbf d}^{\omega} \\ \hat{\underline{\symbf k}} \cdot \underline{\symbf{d}}^{\omega} &= 0 \end{aligned}\right. ~,
\end{equation}
接着,定义 $\mathcal{C}$ 系下的球-柱-直角坐标系的三基矢
\begin{equation} \label{eq:2-43}
    \underline{\symbfup{e}}_{\circleddash\obar\Yup}^{\symup{T}} := \begin{pmatrix} \underline{\symbfup{e}}_{\circleddash} & \underline{\symbfup{e}}_{\obar} & \underline{\symbfup{e}}_{\Yup} \end{pmatrix} := \begin{pmatrix} \underline{\symbfup{e}}_{{\symup{r}}} & \underline{\symbfup{e}}_{{\symup{\rho}}} & \underline{\symbfup{e}}_{{\symup{x}}} \\ \underline{\symbfup{e}}_{{\symup{\theta}}} & \underline{\symbfup{e}}_{{\symup{z}}} & \underline{\symbfup{e}}_{{\symup{y}}} \\ \underline{\symbfup{e}}_{{\symup{\phi}}} & \underline{\symbfup{e}}_{{\symup{\phi}}} & \underline{\symbfup{e}}_{{\symup{z}}} \end{pmatrix}
\end{equation}
及其由 $\underline{\theta},\underline{\phi}$ 表示的坐标变换 $\underline{\symbfup{e}}_{\circleddash} := \bar{\bar{\underline{T}}}_{\circleddash\obar} \cdot \underline{\symbfup{e}}_{\obar} := \bar{\bar{\underline{T}}}_{\circleddash\obar} \cdot \bar{\bar{\underline{T}}}_{\obar\Yup} \cdot \underline{\symbfup{e}}_{\Yup} =: \bar{\bar{\underline{T}}}_{\circleddash\Yup} \cdot \underline{\symbfup{e}}_{\Yup}$ 关系:
\begin{subequations} \label{eq:2-44}
	\abovedisplayskip=0pt
	\belowdisplayskip=10pt
	\begin{align}
		\bar{\bar{\underline{T}}}_{\circleddash\Yup} &= \bar{\bar{\underline{T}}}_{\circleddash\obar} \cdot \bar{\bar{\underline{T}}}_{\obar\Yup} \label{eq:2-44a}\\ &= \begin{pmatrix} \sin{\underline{\theta}} & \cos{\underline{\theta}} & 0 \\ \cos{\underline{\theta}} & -\sin{\underline{\theta}} & 0 \\ 0 & 0 & 1 \end{pmatrix} \begin{pmatrix} \cos{\underline{\phi}} & \sin{\underline{\phi}} & 0 \\ 0 & 0 & 1 \\ -\sin{\underline{\phi}} & \cos{\underline{\phi}} & 0 \end{pmatrix} \label{eq:2-44b} \\ &= \begin{pmatrix} \sin{\underline{\theta}}\cos{\underline{\phi}} & \sin{\underline{\theta}}\sin{\underline{\phi}} & \cos{\underline{\theta}} \\ \cos{\underline{\theta}}\cos{\underline{\phi}} & \cos{\underline{\theta}}\sin{\underline{\phi}} & -\sin{\underline{\theta}} \\ -\sin{\underline{\phi}} & \cos{\underline{\phi}} & 0 \end{pmatrix} \label{eq:2-44c}~,
	\end{align}
\end{subequations}
尽管其中 $\bar{\underline{\circleddash}} := \begin{pmatrix} \underline{\theta} & \underline{\phi} \end{pmatrix}^{\symup{T}}$ 可代表球坐标 $\mathcal{C}$ 系下任何实向量的方向,但如无矢量角标,则默认其代表 $\hat{\symbf k}$ 的方向 $\bar{\underline{\circleddash}} = \bar{\underline{\circleddash}}_{\hat{\symbf k}} := \begin{pmatrix} \underline{\theta}_{\hat{\symbf k}} & \underline{\phi}_{\hat{\symbf k}} \end{pmatrix}^{\symup{T}}$,正如也默认 $\mathcal{Z}$ 系下 $\bar{\circleddash} = \bar{\circleddash}_{\hat{\symbf k}}$。注意 Eq.(\ref{eq:2-44a}) 默认了以 $\bar{\underline{\circleddash}}$ 为自变量,即省略了 $\bar{\bar{\underline{T}}}^{\circleddash}_{\circleddash\Yup} = \bar{\bar{\underline{T}}}^{\circleddash}_{\circleddash\obar} \cdot \bar{\bar{\underline{T}}}^{\circleddash}_{\obar\Yup}$ 中的角标 $\circleddash$。

可以看出 Eq.(\ref{eq:2-44c}) 也是个旋转矩阵、正交矩阵,因此 $\underline{\symbfup{e}}_{\Yup} = \bar{\bar{\underline{T}}}_{\Yup\circleddash} \cdot \underline{\symbfup{e}}_{\circleddash} = \bar{\bar{\underline{T}}}_{\circleddash\Yup}^{\symup{T}} \cdot \underline{\symbfup{e}}_{\circleddash}$,于是根据 Eq.(\ref{eq:2-35}) 到 Eq.(\ref{eq:2-36}) 的逻辑(场的实在性/连续性),由基系数构成的伪向量,同样满足基向量的旋转变换,则有 
\begin{equation} \label{eq:2-45}
	\bar{\underline{d}}^{\omega}_{\Yup} = \bar{\bar{\underline{T}}}_{\Yup\circleddash} \cdot \bar{\underline{d}}^{\omega}_{\circleddash} = \bar{\bar{\underline{T}}}_{\circleddash\Yup}^{\symup{T}} \cdot \bar{\underline{d}}^{\omega}_{\circleddash} ~,
\end{equation}
将其代入直角坐标系下的 Eq.(\ref{eq:2-42}) 中的波动方程
\begin{equation} \label{eq:2-46}
	\left( \bar{\bar{\symbfup{I}}} - \hat{\underline{\symbf k}}_{\Yup} \hat{\underline{\symbf k}}_{\Yup} \right) \cdot \bar{\bar{\underline{\symbf{\eta}}}}^{\omega}_{\Yup} \cdot \underline{\symbf d}^{\omega}_{\Yup} = \frac{ 1 }{ n^2_\omega } \underline{\symbf d}^{\omega}_{\Yup} ~,
\end{equation}
的伪矢量、伪张量(即 $\symup{x,y,z}$ 三分量的标量方程组)形式
\begin{equation} \label{eq:2-47}
	\left( \bar{\bar{\symup{I}}} - \hat{\underline{k}}_{\Yup} \hat{\underline{k}}_{\Yup} \right) \cdot \bar{\bar{\underline{\eta}}}^{\omega}_{\Yup} \cdot \bar{\underline{d}}^{\omega}_{\Yup} = \frac{ 1 }{ n^2_\omega } \bar{\underline{d}}^{\omega}_{\Yup} ~,
\end{equation}
并用 $\bar{\bar{\underline{T}}}_{\circleddash\Yup}$ 左乘 Eq.(\ref{eq:2-47}) 两侧,得
\begin{equation} \label{eq:2-48}
	\bar{\bar{\underline{T}}}_{\circleddash\Yup} \cdot \left( \bar{\bar{\symup{I}}} - \hat{\underline{k}}_{\Yup} \hat{\underline{k}}_{\Yup} \right) \cdot \bar{\bar{\underline{\eta}}}^{\omega}_{\Yup} \cdot \bar{\bar{\underline{T}}}_{\circleddash\Yup}^{\symup{T}} \cdot \bar{\underline{d}}^{\omega}_{\circleddash} = \frac{ 1 }{ n^2_\omega } \bar{\underline{d}}^{\omega}_{\circleddash} ~,
\end{equation}
此即 Berry 三维二秩方阵 $\bar{\bar{\underline{M}}}^{\omega}_{\circleddash}$ 所对应的特征方程
\begin{equation} \label{eq:2-49}
	\bar{\bar{\underline{M}}}^{\omega}_{\circleddash} \cdot \bar{\underline{d}}^{\omega}_{\circleddash} = \frac{ 1 }{ n^2_\omega } \bar{\underline{d}}^{\omega}_{\circleddash} ~,
\end{equation}
其中,三维方阵
\begin{subequations} \label{eq:2-50}
	\begin{align}
		\bar{\bar{\underline{M}}}^{\omega}_{\circleddash} &:= \bar{\bar{\underline{T}}}_{\circleddash\Yup} \cdot \bar{\bar{\underline{M}}}^{\omega}_{\Yup} \cdot \bar{\bar{\underline{T}}}_{\circleddash\Yup}^{\symup{T}} \label{eq:2-50a}~,\\ \bar{\bar{\underline{M}}}^{\omega}_{\Yup} &:= \left( \bar{\bar{\symup{I}}} - \hat{\underline{k}}_{\Yup} \hat{\underline{k}}_{\Yup} \right) \cdot \bar{\bar{\underline{\eta}}}^{\omega}_{\Yup} \label{eq:2-50b}~.
	\end{align}
\end{subequations}

拜 $\underline{\symbf{d}}^{\omega}$ 的横向性即 Eq.(\ref{eq:2-42}) 中散度方程 $\hat{\underline{\symbf k}} \cdot \underline{\symbf{d}}^{\omega} =: \underline{\symbfup{e}}_{\symup{r}} \cdot \underline{\symbf{d}}^{\omega}_{\circleddash} = \underline{d}^{\omega}_{\symup{r}} = 0$ \myHyperFootnote{其中用到了场的实在性所要求的 $\underline{\symbf{d}}^{\omega} = \underline{\symbf{d}}^{\omega}_{\Yup} = \underline{\symbf{d}}^{\omega}_{\circleddash}$。}所赐,球坐标的 $\mathcal{C}$ 系下, $\underline{\symbf{d}}^{\omega}_{\circleddash} = \bar{\underline{d}}_{\circleddash}^{\omega\symup{T}} \cdot \underline{\symbfup{e}}_{\circleddash} = \underline{d}^{\omega}_{\symup{\theta}} \underline{\symbfup{e}}_{\symup{\theta}} + \underline{d}^{\omega}_{\symup{\phi}} \underline{\symbfup{e}}_{\symup{\phi}}$ 是无径向分量的。利用这一点,将三维特征向量
\begin{equation} \label{eq:2-51}
	\bar{\underline{d}}^{\omega}_{\circleddash} = \begin{pmatrix} 0 & \bar{\underline{d}}^{\omega\symup{T}}_{\symup{\theta\phi}} \end{pmatrix}^{\symup{T}} = \begin{pmatrix} 0 \\ \bar{\underline{d}}^{\omega}_{\symup{\theta\phi}} \end{pmatrix} := \begin{pmatrix} 0 \\ \underline{d}^{\omega}_{\symup{\theta}} \\ \underline{d}^{\omega}_{\symup{\phi}} \end{pmatrix} ~,
\end{equation}
代入 Eq.(\ref{eq:2-49}) 可得
\begin{equation} \label{eq:2-52}
	\bar{\bar{\underline{M}}}^{\omega}_{\circleddash} \cdot \begin{pmatrix} 0 \\ \bar{\underline{d}}^{\omega}_{\symup{\theta\phi}} \end{pmatrix} = \frac{ 1 }{ n^2_\omega } \begin{pmatrix} 0 \\ \bar{\underline{d}}^{\omega}_{\symup{\theta\phi}} \end{pmatrix} ~,
\end{equation}
即可得到 Berry 二维满秩(即横向)方阵 $\bar{\bar{\underline{m}}}^{\omega}_{\circleddash}$ 所对应的特征方程
\begin{equation} \label{eq:2-53}
	\bar{\bar{\underline{m}}}^{\omega}_{\symup{\theta\phi}} \cdot \bar{\underline{d}}^{\omega}_{\symup{\theta\phi}} = \frac{ 1 }{ n^2_\omega } \bar{\underline{d}}^{\omega}_{\symup{\theta\phi}} ~,
\end{equation}
其中,横向的(transverse)二维满秩方阵 $\bar{\bar{\underline{m}}}^{\omega}_{\symup{\theta\phi}}$ 和二维特征向量 $\bar{\underline{d}}^{\omega}_{\symup{\theta\phi}}$ 定义为:
\begin{equation} \label{eq:2-54}
	\bar{\bar{\underline{m}}}^{\omega}_{\symup{\theta\phi}} := \text{trans} \left[ \underline{\bar{\bar{M}}}^{\omega}_{\circleddash} \right] = \begin{pmatrix} \underline{M}^{\omega}_{\symup{\theta\theta}} & \underline{M}^{\omega}_{\symup{\theta\phi}} \\ \underline{M}^{\omega}_{\symup{\phi\theta}} & \underline{M}^{\omega}_{\symup{\phi\phi}} \end{pmatrix},\ \ \ \ \ \  \bar{\underline{d}}^{\omega}_{\symup{\theta\phi}} := \text{trans} \left[ \bar{\underline{d}}^{\omega}_{\circleddash} \right] = \begin{pmatrix} \underline{d}^{\omega}_{\symup{\theta}} \\ \underline{d}^{\omega}_{\symup{\phi}} \end{pmatrix} ~.
\end{equation}

为方便地表示 Eq.(\ref{eq:2-54}) 中的横向方阵 $\bar{\bar{\underline{m}}}^{\omega}_{\symup{\theta\phi}}$,如图 \ref{fig:2-1} 所示:
\begin{figure}[h]
	\belowdisplayskip=0pt
	\centering
	\includegraphics[totalheight=3in]{./figures/2.1.png}
	\caption{\label{fig:2-1} 从 $\theta,\phi$ 方向的单位球面 $\symbf{s}$ 上一点,到 $X, Y$ 赤道平面上的向量 $\symbf{R}$ 的南极立体投影;阴影单位圆盘是北半球的投影\cite{berryOpticalSingularitiesBirefringent2003}。Berry 图中的 $\theta, \phi, \symbf{s}, X, Y, \symbf{R}$ 对应这里的 $\underline{\theta}, \underline{\phi}, \hat{\underline{\symbf k}}, \underline{X}, \underline{Y}, \underline{\symbf{R}}$。}
\end{figure}
引入以单位 $\hat{\underline{\symbf k}}$ 球面的南极 $- \underline{\symbfup{e}}_{{\symup{z}}}$ 到球面上一点 $\hat{\underline{\symbf k}}$ 的射线 $\hat{\underline{\symbf k}} + \underline{\symbfup{e}}_{{\symup{z}}}$ 与赤道平面交点为赤平面向量 $\underline{\symbf{R}} := \hat{\underline{\symbf k}}_{\symup{\rho}} \big/ \left( 1 + \hat{\underline{k}}_{\symup{z}} \right)$ 终点的“南极赤平投影 (south-pole stereographic projection)”平面极坐标系。其中,赤平面向量 $\underline{\symbf{R}} = \bar{\underline{R}}^{\symup{T}} \cdot \underline{\symbfup{e}}_{\perp}$ 可表示为 $\bar{\underline{R}}$ 或 $\hat{\underline{k}}_{\Yup}$ 或 $\bar{\underline{\circleddash}}$ 的函数:
\begin{subequations} \label{eq:2-55}
\abovedisplayskip=0pt
\begin{align}
	\underline{\symbf{R}}^{\symup{T}} = \begin{pmatrix} \underline{X} & \underline{Y} \end{pmatrix} \begin{pmatrix} \underline{\symbfup{e}}_{{\symup{x}}} \\ \underline{\symbfup{e}}_{{\symup{y}}} \end{pmatrix} &:= \frac{1}{1 + \hat{\underline{k}}_{\symup{z}}} \cdot \hat{\underline{\symbf k}}_{\symup{\rho}}^{\symup{T}} := \frac{1}{1 + \hat{\underline{k}}_{\symup{z}}} \cdot \begin{pmatrix} \hat{\underline{k}}_{\symup{x}} & \hat{\underline{k}}_{\symup{y}} \end{pmatrix} \begin{pmatrix} \underline{\symbfup{e}}_{{\symup{x}}} \\ \underline{\symbfup{e}}_{{\symup{y}}} \end{pmatrix}  \label{eq:2-55a}\\ \bar{\underline{R}}^{\symup{T}} := \begin{pmatrix} \underline{X} & \underline{Y} \end{pmatrix} &:= \frac{1}{1 + \hat{\underline{k}}_{\symup{z}}} \begin{pmatrix} \hat{\underline{k}}_{\symup{x}} & \hat{\underline{k}}_{\symup{y}} \end{pmatrix} \label{eq:2-55b}\\ &= \sqrt{\frac{1 - \hat{\underline{k}}_{\symup{z}}}{1 + \hat{\underline{k}}_{\symup{z}}}} \begin{pmatrix} \cos \underline{\phi} & \sin \underline{\phi} \end{pmatrix} = \tan \frac{ \underline{\theta} }{ 2 } \begin{pmatrix} \cos \underline{\phi} & \sin \underline{\phi} \end{pmatrix} \label{eq:2-55c} ~,
\end{align}
\end{subequations}
等价地,也可用同样含两个独立实参的复标量 $\underline{Z} \left(\underline{X},\underline{Y}\right)$ 表示实矢量 $\underline{\symbf{R}}\left(\underline{X},\underline{Y}\right)$:
\begin{equation} \label{eq:2-56}
	\abovedisplayskip=10pt
	\belowdisplayskip=10pt
	\underline{Z} = \underline{X} + \mathbb{i} \cdot \underline{Y} ~,
\end{equation}
可以证明,$\underline{\symbf{R}}, \underline{Z}$ 满足镜面对称 $\hat{\underline{k}}_{\symup{z}} \rightarrow -\hat{\underline{k}}_{\symup{z}}$ 变换关系,以及其所导出的轴对称 $\hat{\underline{k}}_{\symup{\rho}} \rightarrow -\hat{\underline{k}}_{\symup{\rho}}$、中心对称 $\hat{\underline{k}} \rightarrow -\hat{\underline{k}}$ 变换关系
\begin{subequations} \label{eq:2-57}
\abovedisplayskip=10pt
\belowdisplayskip=10pt
\begin{align}
	\underline{\symbf{R}} \left( -\hat{\underline{k}}_{\symup{z}} \right) &= \frac{ \underline{\symbf{R}} \left( \hat{\underline{k}}_{\symup{z}} \right)}{ \underline{R}^2 \left( \hat{\underline{k}}_{\symup{z}} \right) } ~,\hspace{2.6em} \underline{Z} \left( -\hat{\underline{k}}_{\symup{z}} \right) \hspace{-6.8em}&&= \frac{ 1 }{ \underline{Z}^* \left( \hat{\underline{k}}_{\symup{z}} \right) } \label{eq:2-57a} ~,\\
	\underline{\symbf{R}} \left( -\hat{\underline{k}}_{\symup{\rho}} \right) &= -\underline{\symbf{R}} \left( \hat{\underline{k}}_{\symup{\rho}} \right),\hspace{2em} \underline{Z} \left( -\hat{\underline{k}}_{\symup{\rho}} \right) \hspace{-6.8em}&&= -\underline{Z} \left( \hat{\underline{k}}_{\symup{\rho}} \right) \label{eq:2-57b} ~,\\ \underline{\symbf{R}} \left( -\hat{\underline{k}} \right) &= -\frac{ \underline{\symbf{R}} \left( \hat{\underline{k}} \right)}{ \underline{R}^2 \left( \hat{\underline{k}} \right) } ~,\hspace{2.7em} \underline{Z} \left( -\hat{\underline{k}} \right) \hspace{-6.8em}&&= -\frac{ 1 }{ \underline{Z}^* \left( \hat{\underline{k}} \right) } \label{eq:2-57c} ~,
\end{align}
\end{subequations}
以致北半球的 $\left| \underline{\symbf{R}} \right| = \left| \underline{Z} \right| < 1$,对应南半球的 $\left| \underline{\symbf{R}} \right| = \left| \underline{Z} \right| > 1$;赤道上 $\left| \underline{\symbf{R}} \right| = \left| \underline{Z} \right| = 1$。

利用 \cref{eq:2-55,eq:2-56} 的 $\bar{\underline{R}}, \underline{Z}$ 表示,Eq.(\ref{eq:2-54}) 中横向方阵 $\bar{\bar{\underline{m}}}^{\omega}_{\symup{\theta\phi}}$ 变为
\begin{equation} \label{eq:2-58}
	\abovedisplayskip=10pt
	\bar{\bar{\underline{m}}}^{\omega}_{\symup{\theta\phi}} =  \frac{\bar{\bar{\underline{R}}}_{\symup{\phi Y}} \cdot \bar{\bar{\underline{U}}}_{\symup{YL}} \cdot \begin{pmatrix} \underline{\mathcal{Q}}^{\omega} + \underline{\mathcal{G}}^{\omega} & \underline{\mathcal{P}}^{\omega}_{2} \\ \underline{\mathcal{P}}^{\omega}_{1} & \underline{\mathcal{Q}}^{\omega} - \underline{\mathcal{G}}^{\omega} \end{pmatrix} \cdot \bar{\bar{\underline{U}}}_{\symup{LY}} \cdot \bar{\bar{\underline{R}}}_{\symup{Y\phi}}}{2\left(1+\underline{R}^{2}\right)^{2}} ~,
\end{equation}
其中,三个坐标变换酉矩阵
\begin{subequations} \label{eq:2-59}
	\begin{align}
		\bar{\bar{\underline{R}}}_{\symup{Y \phi}} &= \bar{\bar{\underline{R}}}_{\symup{\phi Y}}^{\symup{T}} := \begin{pmatrix} \cos{\underline{\phi}} & - \sin{\underline{\phi}} \\ \sin{\underline{\phi}} & \cos{\underline{\phi}} \end{pmatrix} \label{eq:2-59a}~,\\ \bar{\bar{\underline{U}}}_{\symup{LY}} &= \bar{\bar{\underline{U}}}_{\symup{YL}}^{\symup{\dag}} := \frac{1}{\sqrt{2}} \begin{pmatrix} 1 & - \mathbb{i} \\ 1 & \mathbb{i} \end{pmatrix} \label{eq:2-59b}~,\\ \bar{\bar{\underline{U}}}_{\symup{L \phi}} &:= \bar{\bar{\underline{U}}}_{\symup{LY}} \cdot \bar{\bar{\underline{R}}}_{\symup{Y \phi}} = \bar{\bar{\underline{U}}}_{\symup{\phi L}}^{\symup{\dag}} = \frac{1}{\sqrt{2}} \begin{pmatrix} 1 \big/ \mathbb{e}^{\mathbb{i}\underline{\phi}} & -\mathbb{i} \big/ \mathbb{e}^{\mathbb{i}\underline{\phi}} \\ 1 \cdot \mathbb{e}^{\mathbb{i}\underline{\phi}} & \mathbb{i} \cdot \mathbb{e}^{\mathbb{i}\underline{\phi}} \end{pmatrix} \label{eq:2-59c}~,
	\end{align}
\end{subequations}
以及四个多项式
\begin{subequations} \label{eq:2-60}
\begin{align}
	\underline{\mathcal{Q}}^{\omega}\left(\underline{\symbf{R}}, \bar{\bar{\underline{\symbf{u}}}}^{\omega}\right) =\ &\left(1+\underline{R}^{4}\right)\left(\underline{u}^{\omega}_{\symup{x x}}+\underline{u}^{\omega}_{\symup{y y}}\right)+2\left(\underline{Y}^{2}-\underline{X}^{2}\right)\left(\underline{u}^{\omega}_{\symup{x x}}-\underline{u}^{\omega}_{\symup{y y}}\right) \label{eq:2-60a}\\ \ & +4 \underline{R}^{2} \underline{u}^{\omega}_{\symup{z z}}-8 \underline{X} \underline{Y} \underline{u}^{\omega}_{\symup{x y}}-4 \underline{X}\left(1-\underline{R}^{2}\right) \underline{u}^{\omega}_{\symup{x z}}-4 \underline{Y}\left(1-\underline{R}^{2}\right) \underline{u}^{\omega}_{\symup{y z}} \notag~,\\
	\underline{\mathcal{P}}^{\omega}_{1}\left(\underline{Z}, \bar{\bar{\underline{\symbf{u}}}}^{\omega}\right) =\ &\underline{\mathcal{P}}^{\omega}_{1}\left(\underline{Z}, \bar{\bar{\underline{\symbf{u}}}}_{\symup{R}}^{\omega}\right) + \mathbb{i} \cdot \underline{\mathcal{P}}^{\omega}_{1}\left(\underline{Z}, \bar{\bar{\underline{\symbf{u}}}}_{\symup{I}}^{\omega}\right) \label{eq:2-60b}\\ =\ &\left(1+\underline{Z}^{4}\right)\left(\underline{u}^{\omega}_{\symup{x x}}-\underline{u}^{\omega}_{\symup{y y}}\right)+2 \underline{Z}^{2}\left(2 \underline{u}^{\omega}_{\symup{z z}}-\underline{u}^{\omega}_{\symup{x x}}-\underline{u}^{\omega}_{\symup{y y}}\right) \notag\\ \ &+2 \mathbb{i} \cdot \left(1-\underline{Z}^{4}\right) \underline{u}^{\omega}_{\symup{x y}}-4 \underline{Z}\left(1-\underline{Z}^{2}\right) \underline{u}^{\omega}_{\symup{x z}}-4 \mathbb{i} \cdot \underline{Z}\left(1+\underline{Z}^{2}\right) \underline{u}^{\omega}_{\symup{y z}} \notag ~,\\
	\underline{\mathcal{P}}^{\omega}_{2}\left(\underline{Z}, \bar{\bar{\underline{\symbf{u}}}}^{\omega}\right) =\ &{\underline{\mathcal{P}}^{\omega*}_{1}\left(\underline{Z}, \bar{\bar{\underline{\symbf{u}}}}^{\omega*}\right)=\underline{\mathcal{P}}^{\omega*}_{1}\left(\underline{Z}, \bar{\bar{\underline{\symbf{u}}}}_{\symup{R}}^{\omega}\right)+\mathbb{i} \cdot \underline{\mathcal{P}}^{\omega*}_{1}\left(\underline{Z}, \bar{\bar{\underline{\symbf{u}}}}_{\symup{I}}^{\omega}\right) } \label{eq:2-60c}\\ =\ &\left(1+\underline{Z}^{* 4}\right)\left(\underline{u}^{\omega}_{\symup{x x}}-\underline{u}^{\omega}_{\symup{y y}}\right)+2 \underline{Z}^{* 2}\left(2 \underline{u}^{\omega}_{\symup{z z}}-\underline{u}^{\omega}_{\symup{x x}}-\underline{u}^{\omega}_{\symup{y y}}\right) \notag\\ \ &-2 \mathbb{i} \cdot \left(1-\underline{Z}^{* 4}\right) \underline{u}^{\omega}_{\symup{x y}}-4 \underline{Z}^{*}\left(1-\underline{Z}^{* 2}\right) \underline{u}^{\omega}_{\symup{x z}}+4 \mathbb{i} \cdot \underline{Z}^{*}\left(1+\underline{Z}^{* 2}\right) \underline{u}^{\omega}_{\symup{y z}} \notag ~,\\
	\underline{\mathcal{G}}^{\omega}\left(\underline{\symbf{R}}, \bar{\bar{\underline{\symbf{\gamma}}}}^{\omega}\right) =\ &2\left(1+\underline{R}^{2}\right)^{2} {\underline{\symbf {\alpha}}^\omega} \cdot \underline{\hat{\symbf k}} \label{eq:2-60d}\\ = \ &2\left[4 \underline{X}^{2} \underline{\gamma}^{\omega}_{\symup{x x}}+4 \underline{Y}^{2} \underline{\gamma}^{\omega}_{\symup{y y}}+\left(1-\underline{R}^{2}\right)^{2} \underline{\gamma}^{\omega}_{\symup{z z}}\right. \notag\\ \ &\left.\quad+8 \underline{X} \underline{Y} \underline{\gamma}^{\omega}_{\symup{x y}}+4\left(1-\underline{R}^{2}\right)\left(\underline{X} \underline{\gamma}^{\omega}_{\symup{x z}}+\underline{Y} \underline{\gamma}^{\omega}_{\symup{y z}}\right)\right] \notag~,
\end{align}
\end{subequations}
其中,Eq.(\ref{eq:2-60d}) 用到了(实际也可以不用):
\begin{equation} \label{eq:2-61}
	\hat{\underline{k}}^{\symup{T}} = \begin{pmatrix} \hat{\underline{k}}_{\symup{x}} & \hat{\underline{k}}_{\symup{y}} & \hat{\underline{k}}_{\symup{z}} \end{pmatrix} = \frac{ 1 }{ 1+\underline{R}^2 } \begin{pmatrix} 2\underline{X} & 2\underline{Y} & 1-\underline{R}^2 \end{pmatrix} ~.
\end{equation}

由 $\bar{\underline{R}}, \underline{Z}$ 表示的横向特征方程 Eq.(\ref{eq:2-58}) 的 2 个特征值为:
\begin{equation} \label{eq:2-62}
	\frac{ 1 }{ n^2_\omega } \Bigg|_\pm = \frac{ \underline{\mathcal{Q}}^{\omega} \pm \underline{\Delta}^\omega }{ 2\left( 1+\underline{R}^2 \right)^2 } ~,\ \ \ \ \ \  \underline{\Delta}^\omega := \sqrt{\underline{\mathcal{P}}^{\omega}_1 \underline{\mathcal{P}}^{\omega}_2 + \underline{\mathcal{G}}_{\omega}^2} ~,
\end{equation}
因此 $\mathcal{C}$ 系下 $\hat{\underline{\symbf k}}$(或 $\mathcal{Z}$ 系下 $\hat{\symbf k}$)方向的两个折射率为:
\begin{equation} \label{eq:2-63}
	n^{\omega}_{\pm} = \frac{ \sqrt{ 2 }\left( 1+\underline{R}^2 \right) }{ \sqrt{\underline{\mathcal{Q}}^{\omega} \mp \underline{\Delta}^\omega} } ~,
\end{equation}
其中,外层对复数的平方根,恒选择实部为正的分支\myHyperFootnote{python 中 $"\left(\cdot\right)**\ 0.5"$ 默认即返回实部为正的分支。};且虚部为正对应衰减\myHyperFootnote{前提是单色平面波采用 $\symbf g^{\omega} \cdot \mathbb{e}^{\mathbb{i} {\symbf k}^\omega \cdot \symbf r}$ 记法,且 $\displaystyle{ k^\omega =: n^\omega \omega \big/ {\symup{c}}} =: n^\omega k^\omega_{0} $ 。}。

Eq.(\ref{eq:2-58}) 中的横向方阵 $\bar{\bar{\underline{m}}}^{\omega}_{\symup{\theta\phi}} \left( \bar{\underline{R}}, \underline{Z} \right)$ 的 2 个特征向量 $\bar{\underline{d}}^{\omega\pm}_{\symup{\theta\phi}}$ 的 2 种表示形式  $\bar{\underline{d}}^{\omega\pm}_{\symup{1\theta\phi}},\bar{\underline{d}}^{\omega\pm}_{\symup{2\theta\phi}}$ 可用 Eq.(\ref{eq:2-59c}) 表示为(暂不涉及归一化)
\begin{subequations} \label{eq:2-64}
\begin{align}
	\begin{pmatrix} \bar{\underline{d}}^{\omega\pm}_{\symup{1\theta\phi}} & \bar{\underline{d}}^{\omega\pm}_{\symup{2\theta\phi}} \end{pmatrix} &= \sqrt{2} \cdot \bar{\bar{\underline{U}}}_{\symup{\phi L}} \cdot \begin{pmatrix} \bar{\underline{d}}^{\omega\pm}_{\symup{1RL}} & \bar{\underline{d}}^{\omega\pm}_{\symup{2RL}} \end{pmatrix} \label{eq:2-64a}\\ &= \begin{pmatrix} 1 \cdot \mathbb{e}^{\mathbb{i}\underline{\phi}} & 1 \big/ \mathbb{e}^{\mathbb{i}\underline{\phi}} \\ \mathbb{i} \cdot \mathbb{e}^{\mathbb{i}\underline{\phi}} & -\mathbb{i} \big/ \mathbb{e}^{\mathbb{i}\underline{\phi}} \end{pmatrix} \begin{pmatrix} \pm \underline{\Delta}^\omega + \underline{\mathcal{G}}^{\omega} & \underline{\mathcal{P}}^{\omega}_{2} \\ \underline{\mathcal{P}}^{\omega}_{1} & \pm \underline{\Delta}^\omega - \underline{\mathcal{G}}^{\omega} \end{pmatrix} \label{eq:2-64b}~,
\end{align}
\end{subequations}
其中,$\bar{\underline{d}}^{\omega\pm}_{\symup{1\theta\phi}} \mathop{//} \bar{\underline{d}}^{\omega\pm}_{\symup{2\theta\phi}}$ 继承了左右旋基下,横向方阵
\begin{equation} \label{eq:2-65}
	\bar{\bar{\underline{m}}}^{\omega}_{\symup{RL}} := \bar{\bar{\underline{U}}}_{\symup{L \phi}} \cdot \bar{\bar{\underline{m}}}^{\omega}_{\symup{\theta\phi}} \cdot \bar{\bar{\underline{U}}}_{\symup{\phi L}}
\end{equation}
的 2 个特征向量 $\bar{\underline{d}}^{\omega\pm}_{\symup{RL}}$ 的 2 种表示形式
\begin{equation} \label{eq:2-66}
	\begin{pmatrix} \bar{\underline{d}}^{\omega\pm}_{\symup{1RL}} & \bar{\underline{d}}^{\omega\pm}_{\symup{2RL}} \end{pmatrix} = \begin{pmatrix} \pm \underline{\Delta}^\omega + \underline{\mathcal{G}}^{\omega} & \underline{\mathcal{P}}^{\omega}_{2} \\ \underline{\mathcal{P}}^{\omega}_{1} & \pm \underline{\Delta}^\omega - \underline{\mathcal{G}}^{\omega} \end{pmatrix}
\end{equation}
的平行性 $\bar{\underline{d}}^{\omega\pm}_{\symup{1RL}} \mathop{//} \bar{\underline{d}}^{\omega\pm}_{\symup{2RL}}$,即两个列向量按行排,所构成的方阵行列式为零:
\begin{equation} \label{eq:2-67}
	\belowdisplayskip=16pt
	\det \left[ \begin{matrix} \bar{\underline{d}}^{\omega\pm}_{\symup{1\theta\phi}} & \bar{\underline{d}}^{\omega\pm}_{\symup{2\theta\phi}} \end{matrix} \right] = \det \left[ \begin{matrix} \bar{\underline{d}}^{\omega\pm}_{\symup{1RL}} & \bar{\underline{d}}^{\omega\pm}_{\symup{2RL}} \end{matrix} \right] = 0 ~.
\end{equation}

另外,由于属于同一特征值的特征向量的线性组合,也是该特征值对应的特征向量,因此有 $\bar{\underline{d}}^{\omega\pm}_{\symup{\theta\phi}}$ 的第 $3, 4$ 种表示形式:
\begin{subequations} \label{eq:2-68}
\begin{align}
	&\begin{pmatrix} \bar{\underline{d}}^{\omega\pm}_{\symup{3\theta\phi}} & \bar{\underline{d}}^{\omega\pm}_{\symup{4\theta\phi}} \end{pmatrix} = \sqrt{2} \begin{pmatrix} \bar{\underline{d}}^{\omega\pm}_{\symup{1\theta\phi}} & \bar{\underline{d}}^{\omega\pm}_{\symup{2\theta\phi}} \end{pmatrix} \cdot \bar{\bar{\underline{U}}}_{\symup{L \phi}} = 2 \bar{\bar{\underline{U}}}_{\symup{\phi L}} \cdot \begin{pmatrix} \bar{\underline{d}}^{\omega\pm}_{\symup{1RL}} & \bar{\underline{d}}^{\omega\pm}_{\symup{2RL}} \end{pmatrix} \cdot \bar{\bar{\underline{U}}}_{\symup{L \phi}} \label{eq:2-68a}\\ &= \begin{pmatrix} 1 \cdot \mathbb{e}^{\mathbb{i}\underline{\phi}} & 1 \big/ \mathbb{e}^{\mathbb{i}\underline{\phi}} \\ \mathbb{i} \cdot \mathbb{e}^{\mathbb{i}\underline{\phi}} & -\mathbb{i} \big/ \mathbb{e}^{\mathbb{i}\underline{\phi}} \end{pmatrix} \begin{pmatrix} \pm \underline{\Delta}^\omega + \underline{\mathcal{G}}^{\omega} & \underline{\mathcal{P}}^{\omega}_{2} \\ \underline{\mathcal{P}}^{\omega}_{1} & \pm \underline{\Delta}^\omega - \underline{\mathcal{G}}^{\omega} \end{pmatrix} \begin{pmatrix} 1 \big/ \mathbb{e}^{\mathbb{i}\underline{\phi}} & -\mathbb{i} \big/ \mathbb{e}^{\mathbb{i}\underline{\phi}} \\ 1 \cdot \mathbb{e}^{\mathbb{i}\underline{\phi}} & \mathbb{i} \cdot \mathbb{e}^{\mathbb{i}\underline{\phi}} \end{pmatrix} \label{eq:2-68b}\\ &= \begin{pmatrix} \pm 2\underline{\Delta}^\omega + \underline{\mathcal{P}}^{\omega}_{1} \big/ \mathbb{e}^{2\mathbb{i}\underline{\phi}} + \underline{\mathcal{P}}^{\omega}_{2} \cdot \mathbb{e}^{2\mathbb{i}\underline{\phi}} & \mathbb{i} \left( -2\underline{\mathcal{G}}^\omega - \underline{\mathcal{P}}^{\omega}_{1} \big/ \mathbb{e}^{2\mathbb{i}\underline{\phi}} + \underline{\mathcal{P}}^{\omega}_{2} \cdot \mathbb{e}^{2\mathbb{i}\underline{\phi}} \right) \\ \mathbb{i} \left( 2\underline{\mathcal{G}}^\omega - \underline{\mathcal{P}}^{\omega}_{1} \big/ \mathbb{e}^{2\mathbb{i}\underline{\phi}} + \underline{\mathcal{P}}^{\omega}_{2} \cdot \mathbb{e}^{2\mathbb{i}\underline{\phi}} \right) & \pm 2\underline{\Delta}^\omega - \underline{\mathcal{P}}^{\omega}_{1} \big/ \mathbb{e}^{2\mathbb{i}\underline{\phi}} - \underline{\mathcal{P}}^{\omega}_{2} \cdot \mathbb{e}^{2\mathbb{i}\underline{\phi}} \end{pmatrix} \label{eq:2-68c}~,
\end{align}
\end{subequations}
并且 $\bar{\underline{d}}^{\omega\pm}_{\symup{\theta\phi}}$ 的这 $4$ 种形式因 $\bar{\underline{d}}^{\omega\pm}_{\symup{1\theta\phi}} \mathop{//} \bar{\underline{d}}^{\omega\pm}_{\symup{2\theta\phi}} \mathop{//} \bar{\underline{d}}^{\omega\pm}_{\symup{3\theta\phi}} \mathop{//} \bar{\underline{d}}^{\omega\pm}_{\symup{4\theta\phi}}$ 而都是等价的了;但单独每个,在特定的晶体方向上,都存在数值奇点\myHyperFootnote{有的在 $\symup{c}$ 轴(主轴系的 $\symup{z}$ 轴,非光轴)附近,有的在赤道($\bot\ \hat{c} = \symbfup{e}_{\symup{c}} = \underline{\symbfup{e}}_{\symup{z}}$)附近。}。因此,在实际的数值实验中,这 4 种形式的 $\bar{\underline{d}}^{\omega\pm}_{\symup{\theta\phi}}$ 都会在不同情况下被用到,以规避可能的数值奇点。

引入复归一化算符/函数 $\mathcal{N}$ 的运算规则:
\begin{equation} \label{eq:2-69}
	\hat{\underline{d}}^{\omega} := \mathcal{N} \left[ \bar{\underline{d}}^{\omega} \right] := \mathcal{N} \cdot \bar{\underline{d}}^{\omega}_{\Yup} := \frac{ \bar{\underline{d}}^{\omega} }{ \left| \bar{\underline{d}}^{\omega} \right| } := \frac{ \bar{\underline{d}}^{\omega} }{ \sqrt{ \bar{\underline{d}}_{\omega}^{\dag} \bar{\underline{d}}_{\omega} } } = \frac{ \bar{\underline{d}}^{\omega} }{ \sqrt{ \underline{d}_{\symup{i}}^{\omega*} \underline{d}_{\symup{i}}^{\omega} } } ~,
\end{equation}
将其 Eq.(\ref{eq:2-69}) 作用于 Eq.(\ref{eq:2-45}) 两侧后,由于算符 $\mathcal{N}$ 与 $\bar{\bar{\underline{T}}}_{\circleddash\Yup}$ 的对易性,有:
\begin{subequations} \label{eq:2-70}
	\begin{align}
		\hat{\underline{d}}^{\omega}_{\Yup} &= \mathcal{N} \left[ \bar{\underline{d}}^{\omega}_{\Yup} \right] = \mathcal{N} \left[ \bar{\bar{\underline{T}}}_{\Yup\circleddash} \cdot \bar{\underline{d}}^{\omega}_{\circleddash} \right] = \bar{\bar{\underline{T}}}_{\Yup\circleddash} \cdot \mathcal{N} \left[ \bar{\underline{d}}^{\omega}_{\circleddash} \right] \label{eq:2-70a}\\ &= \mathcal{N} \cdot \bar{\underline{d}}^{\omega}_{\Yup} = \mathcal{N} \cdot \bar{\bar{\underline{T}}}_{\Yup\circleddash} \cdot \bar{\underline{d}}^{\omega}_{\circleddash} = \bar{\bar{\underline{T}}}_{\Yup\circleddash} \cdot \mathcal{N} \cdot \bar{\underline{d}}^{\omega}_{\circleddash} = \bar{\bar{\underline{T}}}_{\Yup\circleddash} \cdot \hat{\underline{d}}^{\omega}_{\circleddash} \label{eq:2-70b}~,
	\end{align}
\end{subequations}
将 Eq.(\ref{eq:2-51}) 代入 Eq.(\ref{eq:2-70b}),得三维笛卡尔 $\mathcal{C}$ 系下的复归一化偏振态:
\begin{equation} \label{eq:2-71}
	\hat{\underline{d}}^{\omega}_{\Yup} = \bar{\bar{\underline{T}}}_{\Yup\circleddash} \cdot \begin{pmatrix} 0 \\ \hat{\underline{d}}^{\omega}_{\symup{\theta\phi}} \end{pmatrix} ~,
\end{equation}
其中 $\bar{\bar{\underline{T}}}_{\Yup\circleddash}^{\symup T}$ 可以不取以 $\bar{\underline{\circleddash}}$ 为自变量的 Eq.(\ref{eq:2-44a}) 即 $\bar{\bar{\underline{T}}}^{\circleddash}_{\circleddash\Yup} = \bar{\bar{\underline{T}}}^{\circleddash}_{\circleddash\obar} \cdot \bar{\bar{\underline{T}}}^{\circleddash}_{\obar\Yup}$,而是将 Eq.(\ref{eq:2-38a}) 即 $\bar{\underline{\circleddash}} = \bar{\bar{\underline{\mathcal{T}}}}_{\circleddash} \cdot \bar{\underline{k}}_{\Yup}$ 代入 Eq.(\ref{eq:2-44a}),取以 ${\bar{\underline{k}}}_{\Yup}$ 为自变量的
\begin{subequations} \label{eq:2-72}
\begin{align}
	%	\abovedisplayskip=10pt
	\bar{\bar{\underline{T}}}^{\Yup}_{\circleddash\Yup} &= \bar{\bar{\underline{T}}}^{\Yup}_{\circleddash\obar} \cdot \bar{\bar{\underline{T}}}^{\Yup}_{\obar\Yup} \label{eq:2-72a}\\ &= \begin{pmatrix} \hat{\underline{k}}_{\symup{\rho}} & \hat{\underline{k}}_{\symup{z}} & 0 \\ \hat{\underline{k}}_{\symup{z}} & -\hat{\underline{k}}_{\symup{\rho}} & 0 \\ 0 & 0 & 1 \end{pmatrix} \begin{pmatrix} \hat{\underline{k}}_{\symup{x}} \big/ \hat{\underline{k}}_{\symup{\rho}} & \hat{\underline{k}}_{\symup{y}} \big/ \hat{\underline{k}}_{\symup{\rho}} & 0 \\ 0 & 0 & 1 \\ -\hat{\underline{k}}_{\symup{y}} \big/ \hat{\underline{k}}_{\symup{\rho}} & \hat{\underline{k}}_{\symup{x}} \big/ \hat{\underline{k}}_{\symup{\rho}} & 0 \end{pmatrix} \label{eq:2-72b}\\ &= \begin{pmatrix} \hat{\underline{k}}_{\symup{x}} & \hat{\underline{k}}_{\symup{y}} & \hat{\underline{k}}_{\symup{z}} \\ \hat{\underline{k}}_{\symup{x}} \hat{\underline{k}}_{\symup{z}} \big/ \hat{\underline{k}}_{\symup{\rho}} & \hat{\underline{k}}_{\symup{y}} \hat{\underline{k}}_{\symup{z}} \big/ \hat{\underline{k}}_{\symup{\rho}} & -\hat{\underline{k}}_{\symup{\rho}} \\ -\hat{\underline{k}}_{\symup{y}} \big/ \hat{\underline{k}}_{\symup{\rho}} & \hat{\underline{k}}_{\symup{x}} \big/ \hat{\underline{k}}_{\symup{\rho}} & 0 \end{pmatrix} = \begin{pmatrix} \hat{\underline{k}}_{\Yup}^{\symup{T}} \\ \left[ \left( \bar{\underline{\symup{e}}}_{\symup{z}} \times \hat{\underline{k}}_{\Yup} \right) \times \hat{\underline{k}}_{\Yup} \big/ \hat{\underline{k}}_{\symup{\rho}} \right] ^{\symup{T}} \\ \left( \bar{\underline{\symup{e}}}_{\symup{z}} \times \hat{\underline{k}}_{\Yup} \big/ \hat{\underline{k}}_{\symup{\rho}} \right) ^{\symup{T}} \end{pmatrix} \label{eq:2-72c}~,
\end{align}
\end{subequations}
其中用到了
\begin{equation} \label{eq:2-73}
	\begin{pmatrix} \underline{\symbfup{e}}_{{\symup{r}}} \\ \underline{\symbfup{e}}_{{\symup{\theta}}} \\ \underline{\symbfup{e}}_{{\symup{\phi}}} \end{pmatrix} = \begin{pmatrix} \hat{\underline{\symbf k}} \\ \underline{\symbfup{e}}_{{\symup{\phi}}} \times \underline{\symbfup{e}}_{{\symup{r}}} \\ \mathcal{N} \left[ \underline{\symbfup{e}}_{{\symup{z}}} \times \underline{\symbfup{e}}_{{\symup{r}}} \right] \end{pmatrix} ~,
\end{equation}
这样对 Eq.(\ref{eq:2-71}) 的运算可简写作笛卡尔的外积/叉积/矢量积 $\times$ 形式,便于调用 np.cross 等库。

将 $\bar{\underline{d}}^{\omega\pm}_{\symup{\theta\phi}}$ 的 4 种形式如 Eq.(\ref{eq:2-64b}) 或 Eq.(\ref{eq:2-68c}) 中的某一种代入 Eq.(\ref{eq:2-71}),即得本征偏振态 $\bar{\underline{d}}^{\omega\pm}$ 在笛卡尔的 $\mathcal{C}$ 系下的三分量:
\begin{equation} \label{eq:2-74}
	\abovedisplayskip=16pt
	\belowdisplayskip=16pt
	\hat{\underline{d}}^{\omega\pm}_{\Yup} = \bar{\bar{\underline{T}}}^{\Yup}_{\Yup\circleddash} \cdot \begin{pmatrix} 0 \\ \hat{\underline{d}}^{\omega\pm}_{\symup{\theta\phi}} \end{pmatrix} ~,
\end{equation}
其 Eq.(\ref{eq:2-74}) 亦可写作“ 3 分量 $\leftarrow$ 2 分量”的形式
\begin{equation} \label{eq:2-75}
	\abovedisplayskip=16pt
	\belowdisplayskip=16pt
	\hat{\underline{d}}^{\omega\pm}_{\Yup} = \bar{\bar{\underline{\mathscr{T}}}}^{\Yup}_{\Yup\circleddash} \cdot \hat{\underline{d}}^{\omega\pm}_{\symup{\theta\phi}} ~,
\end{equation}
其中,“mathscr” 花体算符 $\bar{\bar{\underline{\mathscr{T}}}}^{\Yup}_{\Yup\circleddash}$ 类似 Eq.(\ref{eq:2-38b}) 中的 “mathcal” 花体算符 $\bar{\bar{\mathcal{T}}}_{\Yup}$,接收 2 个参数,返回 3 个参数;但与之不同的是,“mathscr” 字体的算符是矩阵,但 “mathcal” 字体的算符不是:
\begin{equation} \label{eq:2-76}
	\abovedisplayskip=16pt
	\belowdisplayskip=16pt
	\bar{\bar{\underline{\mathscr{T}}}}^{\Yup}_{\Yup\circleddash} := \text{trans} \left[ \bar{\bar{\underline{T}}}^{\Yup \symup{T}}_{\circleddash\Yup} \right] = \begin{bmatrix} \left( \bar{\underline{\symup{e}}}_{\symup{z}} \times \hat{\underline{k}}_{\Yup} \right) \times \hat{\underline{k}}_{\Yup} & \bar{\underline{\symup{e}}}_{\symup{z}} \times \hat{\underline{k}}_{\Yup} \end{bmatrix} \big/ \hat{\underline{k}}_{\symup{\rho}} ~.
\end{equation}

在得到了电位移矢量在笛卡尔的 $\mathcal{C}$ 系下的归一化本征偏振态 Eq.(\ref{eq:2-75}) 后,可通过本构关系,得出电场的本征偏振态:
\begin{subequations} \label{eq:2-77}
\begin{align}
	\abovedisplayskip=16pt
	\belowdisplayskip=16pt
	\bar{\underline{g}}^{\omega} &\propto \bar{\bar{\underline{\eta}}}^{\omega} \cdot \bar{\underline{d}}^{\omega} \label{eq:2-77a} \\ &= \bar{\bar{\underline{u}}}^\omega \cdot \bar{\underline{d}}^{\omega} + \mathbb{i} \cdot {\bar{\underline{\alpha}}^\omega} \times \bar{\underline{d}}^{\omega}  \label{eq:2-77b} ~,
\end{align}
\end{subequations}
其中,对于磁光本构关系,若 Eq.(\ref{eq:2-32}) 中第二式更基本,则可直接通过 Eq.(\ref{eq:2-77b}) 计算;反之,若 Eq.(\ref{eq:2-32}) 中第一式更基本,则可代入近似公式 Eq.(\ref{eq:2-41a}) 到 Eq.(\ref{eq:2-77b}) 中,或直接代入 Eq.(\ref{eq:2-32}) 一式的逆 $\bar{\bar{\underline{\eta}}}^{\omega} = \bar{\bar{\underline{\varepsilon}}}_{\symup{r\omega}}^{-1} = \left( \bar{\bar{\underline{\epsilon}}}^{\omega} + \mathbb{i} \cdot {\bar{\underline {\beta}}^\omega} \times \right)^{-1}$ 到 Eq.(\ref{eq:2-77a}) 中,以得到未归一化笛卡尔的(因为张量分量一般是笛卡尔的)电场本征偏振模 $\bar{\underline{g}}^{\omega}$:
\begin{equation} \label{eq:2-78}
	\abovedisplayskip=16pt
	\belowdisplayskip=16pt
	\bar{\underline{g}}^{\omega\pm}_{\Yup} \propto \bar{\bar{\underline{\eta}}}^{\omega} \cdot \hat{\underline{d}}^{\omega\pm}_{\Yup} ~,
\end{equation}
接着用 Eq.(\ref{eq:2-69}) 中的 $\mathcal{N} \cdot$ 作用于 Eq.(\ref{eq:2-78}),且将 Eq.(\ref{eq:2-75}) 代入其中,得笛卡尔的 $\mathcal{C}$ 系下归一化的电场偏振态:
\begin{subequations} \label{eq:2-79}
\begin{align}
	\abovedisplayskip=16pt
	\belowdisplayskip=16pt
	\hat{\underline{g}}^{\omega\pm}_{\Yup} &= \mathcal{N} \cdot \bar{\bar{\underline{\eta}}}^{\omega} \cdot \hat{\underline{d}}^{\omega\pm}_{\Yup} \label{eq:2-79a} \\&= \mathcal{N} \cdot \bar{\bar{\underline{\eta}}}^{\omega} \cdot \bar{\bar{\underline{\mathscr{T}}}}^{\Yup}_{\Yup\circleddash} \cdot \hat{\underline{d}}^{\omega\pm}_{\symup{\theta\phi}} \label{eq:2-79b}~.
\end{align}
\end{subequations}

有了笛卡尔系下的电位移场、电场偏振态 Eq.(\ref{eq:2-75})、Eq.(\ref{eq:2-79b}) 复矢量场,接着可自行定义函数,以获取他们的主轴(长/短轴)实单位矢量、主轴转角、椭偏度等信息。由于函数略微比较复杂,且不属于核心结论\myHyperFootnote{信息都在偏振态里面了,这里的函数只是将其中的一些属性提炼出来,锦上添花而已。},这里不作展示。

另外,对 Eq.(\ref{eq:2-1}) 中电场旋度方程两侧时空傅立叶变换 $\bar{k} \times \bar{g}^{\omega} = \omega \bar{b}^{\omega} $,利用磁场本构关系 $\bar{h}^{\omega} = \bar{b}^{\omega} \big/ {\symup{\mu}}_0$,也可以给出单色定振幅平面波的其他场的信息,如磁感应场$\bar{b}^{\omega}$、磁场 $\bar{h}^{\omega}$、能流密度场 $\bar{s}^{\omega}$ 在 $\mathcal{C}$ 系下的单位复矢量、走离角 $\underline{\alpha}^{\omega}_{\pm}$:
\begin{subequations} \label{eq:2-80}
\begin{align}
	\hat{\underline{h}}^{\omega\pm}_{\Yup} = \hat{\underline{b}}^{\omega\pm}_{\Yup} &= \hat{\underline{k}}_{\Yup} \times \hat{\underline{g}}^{\omega\pm}_{\Yup} \label{eq:2-80a}~, \\ \hat{\underline{s}}^{\omega\pm}_{\Yup} &= \hat{\underline{g}}^{\omega\pm}_{\Yup} \times \hat{\underline{h}}^{\omega\pm}_{\Yup} \label{eq:2-80b}~, \\ \cos \underline{\alpha}^{\omega}_{\pm} &= \hat{\underline{s}}^{\omega\pm}_{\Yup} \cdot \hat{\underline{k}}_{\Yup} \label{eq:2-80c}~,
\end{align}
\end{subequations}
但注意,在介质有手性的情况下,\cref{eq:2-80b,eq:2-80c} 不成立\cite{nelsonDerivingTransmissionReflection1995};在二向色性晶体中,需重新考虑走离角 Eq.(\ref{eq:2-80c}) 的定义;另外,如果光学活性被归类于电磁间的交叉耦合\cite{berryOpticalSingularitiesBianisotropic2005},磁场本构关系 $\bar{b}^{\omega} = {\symup{\mu}}_0 \bar{h}^{\omega}$ 将被改写,导致 Eq.(\ref{eq:2-80a}) 也需作出相应改变。

另外,为满足纳入傅立叶光学 Eq.(\ref{eq:2-18a}) 范畴的必要条件之一,可利用事先已储存在内存中的 Eq.(\ref{eq:2-40}) 的逆 $\bar{\bar{R}}_{\Yup}^{-1} = \bar{\bar{\underline{R}}}_{\Yup}^{\symup{T}}$,将电位移场 Eq.(\ref{eq:2-75})、电场 Eq.(\ref{eq:2-79b})、磁感应场和磁场 Eq.(\ref{eq:2-80a})、能流密度场 Eq.(\ref{eq:2-80b}) 的归一化本征模,从 $\mathcal{C}$ 系下的 $\hat{\underline{d}}^{\omega\pm}_{\Yup}, \hat{\underline{g}}^{\omega\pm}_{\Yup}, \hat{\underline{b}}^{\omega\pm}_{\Yup}, \hat{\underline{h}}^{\omega\pm}_{\Yup}, \hat{\underline{s}}^{\omega\pm}_{\Yup}$ 分别变换到 $\mathcal{Z}$ 系下 $\hat{d}^{\omega\pm}_{\Yup}, \hat{g}^{\omega\pm}_{\Yup}, \hat{b}^{\omega\pm}_{\Yup}, \hat{h}^{\omega\pm}_{\Yup}, \hat{s}^{\omega\pm}_{\Yup}$:
\begin{subequations} \label{eq:2-81}
\begin{align}
	\hat{d}^{\omega\pm}_{\Yup} &= \bar{\bar{R}}_{\Yup} \cdot \hat{\underline{d}}^{\omega\pm}_{\Yup} = \bar{\bar{R}}_{\Yup} \cdot \bar{\bar{\underline{\mathscr{T}}}}^{\Yup}_{\Yup\circleddash} \cdot \hat{\underline{d}}^{\omega\pm}_{\symup{\theta\phi}} \label{eq:2-81a}~,\\ \hat{g}^{\omega\pm}_{\Yup} &= \bar{\bar{R}}_{\Yup} \cdot \hat{\underline{g}}^{\omega\pm}_{\Yup} = \bar{\bar{R}}_{\Yup} \cdot \mathcal{N} \cdot \bar{\bar{\underline{\eta}}}^{\omega} \cdot \bar{\bar{\underline{\mathscr{T}}}}^{\Yup}_{\Yup\circleddash} \cdot \hat{\underline{d}}^{\omega\pm}_{\symup{\theta\phi}}\label{eq:2-81b}~,\\ \hat{b}^{\omega\pm}_{\Yup} &= \bar{\bar{R}}_{\Yup} \cdot \hat{\underline{b}}^{\omega\pm}_{\Yup} = \bar{\bar{R}}_{\Yup} \cdot \hat{\underline{k}}_{\Yup} \times \mathcal{N} \cdot \bar{\bar{\underline{\eta}}}^{\omega} \cdot \bar{\bar{\underline{\mathscr{T}}}}^{\Yup}_{\Yup\circleddash} \cdot \hat{\underline{d}}^{\omega\pm}_{\symup{\theta\phi}} = \label{eq:2-81c}\\ \hat{h}^{\omega\pm}_{\Yup} &= \hat{k}_{\Yup} \times \hat{g}^{\omega\pm}_{\Yup} = \hat{k}_{\Yup} \times \bar{\bar{R}}_{\Yup} \cdot \mathcal{N} \cdot \bar{\bar{\underline{\eta}}}^{\omega} \cdot \bar{\bar{\underline{\mathscr{T}}}}^{\Yup}_{\Yup\circleddash} \cdot \hat{\underline{d}}^{\omega\pm}_{\symup{\theta\phi}} \label{eq:2-81d}~,\\
	\hat{s}^{\omega\pm}_{\Yup} &= \bar{\bar{R}}_{\Yup} \cdot \hat{\underline{s}}^{\omega\pm}_{\Yup} = \bar{\bar{R}}_{\Yup} \cdot \hat{\underline{g}}^{\omega\pm}_{\Yup} \times \hat{\underline{h}}^{\omega\pm}_{\Yup} = \hat{g}^{\omega\pm}_{\Yup} \times \hat{h}^{\omega\pm}_{\Yup} \label{eq:2-81e}~.
\end{align}
\end{subequations}

由于我们关注的是横跨各向同性与纯电各向异性这两种介质的透射情况,因此,除了晶体内的 $\mathcal{Z}$ 系偏振态外,我们也额外给出晶体外的 $\mathcal{Z}$ 系本征模:即为单轴晶体式起/检偏器服务的、在空气/真空而非介质中的 $\symup{HV}$ 三维偏振态($\leftarrow \uparrow\ \backsim \symbfup{e}_{\symup{xy}} = \symup{HV}$),所对应的 $\mathcal{Z}$ 系下的三维电场相关的归一化本征模
\begin{subequations} \label{eq:2-82}
\begin{align}
	\hat{d}^{\omega \leftarrow}_{\Yup} = \hat{g}^{\omega \leftarrow}_{\Yup} &:= \mathcal{N} \left[ \left( \hat{k}_{\Yup} \times \symbfup{e}_{\symup{x}} \right) \times \hat{k}_{\Yup} \right] \label{eq:2-82a}~,\\ \hat{d}^{\omega \uparrow}_{\Yup} = \hat{g}^{\omega \uparrow}_{\Yup} &:= \mathcal{N} \left[ \left( \hat{k}_{\Yup} \times \symbfup{e}_{\symup{y}} \right) \times \hat{k}_{\Yup} \right] \label{eq:2-82b}~,
\end{align}
\end{subequations}
以用与晶体内本征模相同的三维形式,描述各向同性介质内的偏振态\myHyperFootnote{在电场切向连续性边界条件处,需要三维偏振场作为输入;再加上起/检偏器本质上是强各向异性吸收晶体,会引入偏振串扰;于是设计了这样的三维偏振态,它只在起检偏时,才与物理实在对应。},同时引入理想偏振片对线偏光场的起偏或检偏,所导致的场的偏振分量间的串扰效应\cite{zhangNonparaxialIdealizedPolarizer2018}。相应的其他场量可用 \cref{eq:2-81c,eq:2-81d,eq:2-81e} 得出,这里不再赘述。

至此,Berry 所选择的复波矢形式 ${\symbf k}^\omega \left( \hat{\symbf k} \right) = k^\omega \left( \hat{\symbf k} \right) \hat{\symbf k}$ 带来了 $\mathcal{C,Z}$ 两系下彻底解析的本征解,包括 Eq.(\ref{eq:2-63}) 中的折射率 $n^{\omega}_{\pm}$、\cref{eq:2-75,eq:2-79b,eq:2-80a,eq:2-80b} 中笛卡尔的 $\mathcal{C}$ 系归一化本征模 $\hat{\underline{d}}^{\omega\pm}_{\Yup}, \hat{\underline{g}}^{\omega\pm}_{\Yup}, \hat{\underline{b}}^{\omega\pm}_{\Yup}, \hat{\underline{h}}^{\omega\pm}_{\Yup}, \hat{\underline{s}}^{\omega\pm}_{\Yup}$、Eq.(\ref{eq:2-81}) 中笛卡尔的 $\mathcal{Z}$ 系归一化本征向量 $\hat{d}^{\omega\pm}_{\Yup}, \hat{g}^{\omega\pm}_{\Yup}, \hat{b}^{\omega\pm}_{\Yup}, \hat{h}^{\omega\pm}_{\Yup}, \hat{s}^{\omega\pm}_{\Yup}$;此外,还附加自定义了 2 个空气中起检偏器相关的、笛卡尔的 $\mathcal{Z}$ 系归一化偏振态 $\hat{g}^{\omega \leftarrow\hspace{-0.525em}\uparrow}_{\Yup}$。
% $\hat{g}^{\omega \leftharpoondown\hspace{-0.525em}\upharpoonleft}_{\Yup}$

其中,对应同一个 $\hat{\symbf k}$ 的成对特征向量/基础解系的线性叠加,组成了矢量各向异性无源波动方程/亥姆霍兹方程 Eq.(\ref{eq:2-31}) 晶体所允许的随 $\hat{\symbf k}$ 变化的所有可能的偏振态;以 $\mathcal{Z}$ 系下的电场(其他类似)单色均匀\myHyperFootnote{“均匀”是指相位传播方向与复振幅衰减方向相同,“非均匀”则指等相位面与等振幅面的法向不同。}平面波的复振幅为例,即
\begin{equation} \label{eq:2-83}
	\bar{g}^{\omega}_{\Yup} \left( \hat{\symbf k} \right) = \overline{{\mathtt{g}}^{\omega}_{\pm}}^{\symup{T}} \cdot \overline{\hat{g}^{\omega\pm}_{\Yup}} := \begin{pmatrix} {\mathtt{g}}^{\omega}_{+} & {\mathtt{g}}^{\omega}_{-} \end{pmatrix} \begin{pmatrix} \hat{g}^{\omega+}_{\Yup} \\ \hat{g}^{\omega-}_{\Yup} \end{pmatrix} = \sum_{\pm} {\mathtt{g}}^{\omega}_{\pm} \hat{g}^{\omega\pm}_{\Yup} ~,
\end{equation}
将其乘以本征值 $n^{\omega}_{\pm}$ 相关的相位部分,则构成 Eq.(\ref{eq:2-31}) 在 $\hat{\symbf k}$ 方向的电场行波解
\begin{equation} \label{eq:2-84}
	\bar{G}^{\omega}_{\Yup \symbf r} \left( \hat{\symbf k} \right) = \overline{\bar{g}^{\omega\pm}_{\Yup}}^{\symup{T}} \cdot \overline{\mathbb{e}^{\mathbb{i} k^{\omega}_{0} n^{\omega}_{\pm} \hat{\symbf k} \cdot \symbf{r}}} := \begin{pmatrix} \bar{g}^{\omega+}_{\Yup} & \bar{g}^{\omega-}_{\Yup} \end{pmatrix} \begin{pmatrix} \mathbb{e}^{\mathbb{i} k^{\omega}_{0} n^{\omega}_{+} \hat{\symbf k} \cdot \symbf{r}} \\ \mathbb{e}^{\mathbb{i} k^{\omega}_{0} n^{\omega}_{-} \hat{\symbf k} \cdot \symbf{r}} \end{pmatrix} = \sum_{\pm} {\mathtt{g}}^{\omega}_{\pm} \hat{g}^{\omega\pm}_{\Yup} \cdot \mathbb{e}^{\mathbb{i} k^{\omega}_{0} n^{\omega}_{\pm} \hat{\symbf k} \cdot \symbf{r}} ~,
\end{equation}
其中的 $\overline{\bar{g}^{\omega\pm}_{\Yup}}^{\symup{T}} := \begin{pmatrix} \bar{g}^{\omega+}_{\Yup} & \bar{g}^{\omega-}_{\Yup} \end{pmatrix}$ 不同于 $\symbf{g}^{\omega}_{\Yup} = \bar{g}^{\omega\symup{T}}_{\Yup} \cdot \symbfup{e}_{\Yup}$ 中的 $\bar{g}^{\omega\symup{T}}_{\Yup} := \begin{pmatrix} g^{\omega}_{\symup{x}} & g^{\omega}_{\symup{y}} & g^{\omega}_{\symup{z}} \end{pmatrix}$。注意,$\bar{g}^{\omega}_{\Yup}, \bar{G}^{\omega}_{\Yup z}, \symbf{g}^{\omega}_{\Yup}, \symbf{G}^{\omega}_{\Yup z} = \bar{G}^{\omega\symup{T}}_{\Yup z} \cdot \symbfup{e}_{\Yup}$ 覆盖了单态/非叠加态的旧有设定,如 \cref{eq:2-22,eq:2-23,eq:2-77,eq:2-78} 等,且 $\bar{G}^{\omega}_{\Yup \symbf r}, \symbf{G}^{\omega}_{\Yup \symbf r}$ 因含 $x,y$ 而不属于时空谱 $\bar{G}^{\omega}_{\Yup z}, \symbf{G}^{\omega}_{\Yup z}$。

然而,这 $\pm$ 两个本征解和行波解,因共享同一个倒空间方向 $\hat{\symbf k}$,而不共享同一个横向波矢 $\symbf k_{\symup{\rho}}$;同时也导致 $k_{\symup x}, k_{\symup y}$ 在广义情况下是复的、无法构成横向实空间频率域,以纳入 Eq.(\ref{eq:2-18a}) 傅立叶光学的范畴;因此接下来,需从 $\mathcal{Z}$ 系下的波矢、偏振态着手,考虑两材料端面间的广义斯涅尔、广义菲涅尔边界条件,以将 Berry 的 $\mathcal{C}$ 系解析解,从纯理论拓展至 $\mathcal{Z}$ 系下有边界条件的更贴近实验的情况。

%\section{纯电各向异性与透明介质间的端面边界条件}
%\label{纯电各向异性与透明介质间的端面边界条件}
%\section{纯电各向异性介质中傅立叶线性光学解析解}
\section{\protect\hyperlink{chap:\thesection}{纯电各向异性介质中傅立叶线性光学解析解}}
\addtocontents{toc}{\protect\linkdest{chap:\thesection}}
\label{纯电各向异性介质中傅立叶线性光学解析解}

一般而言,电磁场的边界条件包括 2 个内容:斯涅尔定律、菲涅尔公式;作为麦氏方程组 Eq.(\ref{eq:2-1}) 的积分形式的 2 个结果性等价表述,二者共同描述了在材料性质的不连续处,电磁场各分量的连续性。但如果材料有吸收/增益\cite{berryOpticalSingularitiesBirefringent2003}或手性,熟悉的斯涅尔定律和菲涅耳公式将不再有效,需要拓展至更广义的情况。

广义菲涅耳透/反射系数,一般根据两种材料边界处切向电场和磁场 $\symbf E^{\omega}_z, \symbf H^{\omega}_z$ 的连续性得出\cite{abdulhalimExactMatrixMethod1999,mcleodVectorFourierOptics2014,wangComplexRayTracing2008a,wangComplexRayTracing2008};但对于光学活性晶体,这种方法不正确:因为在该情况下 $\symbf H^{\omega}_z$ 的切向分量不守恒,而只能采用 $\symbf B^{\omega}_z$ 的法向分量守恒,毕竟 $\symbf E^{\omega}_z, \symbf B^{\omega}_z$ 是基本场,而 $\symbf D^{\omega}_z, \symbf H^{\omega}_z$ 还与本构关系中的 $\symbf P^{\omega}_z, \symbf M^{\omega}_z$ 甚至电四极化强度\myHyperFootnote{尽管电(偶)极化强度 $\symbf P^{\omega}_z$ 只属于电动力学中电多极矩展开的一阶项,它仍还可继续展开出关于 $\symbf E^{\omega}_z$ 的非线性部分,正如 $\nabla \cdot \bar{\bar{\symbf Q}}^{\omega}_z$ 也一样:电多极矩的展开与其关于电场的非线性展开,二者是独立的。}的散度 $\nabla \cdot \bar{\bar{\symbf Q}}^{\omega}_z$ 有关\cite{nelsonDerivingTransmissionReflection1995};以致于在手性晶体中,$\symbf H^{\omega}_z$ 的切向分量、$\symbf D^{\omega}_z$ 的法向分量不再守恒,能流密度矢量 $\symbf S^{\omega}_z$ 也无法再写作 $\symbf E^{\omega}_z \times \symbf H^{\omega}_z$。这一切都因为光学活性是非局域的,其本质需要由量子力学和电动力学解释\cite{eimerlQuantumElectrodynamicsOptical1988,nelsonDerivingTransmissionReflection1995}。

广义菲涅尔系数/公式/方程,除了根据麦氏方程组 Eq.(\ref{eq:2-1}) 的积分形式所对应的 4 个边值关系中的其中 2 个导出,也可无需施加边界条件导出\cite{chenWavevectorspaceMethodWave1993,nelsonDerivingTransmissionReflection1995}。并且,在 Nelson 看来,不仅不需要,而且不能施加这样的边界条件\cite{nelsonDerivingTransmissionReflection1995}:因为旋光情况下 $\nabla \cdot \bar{\bar{\symbf Q}}^{\omega}_z$ 必须考虑且纳入 $\symbf D^{\omega}_z$ 内,而电四极化强度的存在,对光与物质相互作用的贡献,会导致通常的 Maxwell 边界条件的 $\symbf D^{\omega}_z$ 的法向连续性、$\symbf H^{\omega}_z$ 的切向连续性被打破,即使材料分界面上无自由电荷源\cite{nelsonDerivingTransmissionReflection1995}。

因此,由于无法应用 $\symbf E^{\omega}_z, \symbf H^{\omega}_z$ 的连续性、只能应用 $\symbf E^{\omega}_z, \symbf B^{\omega}_z$ 的连续性,而 $\symbf B^{\omega}_z$ 的法向连续性不容易具体实施、本质上也不能像这样从实空间,而应从波矢空间\cite{chenWavevectorspaceMethodWave1993}的角度获取透/反射系数的缘故,且基于大部分情况下材料前后端面镀了增透膜以至于对于单色光,透射率均接近于 1 的客观事实,以及我们只关注透射解不关注反射解的初衷,考虑到该问题的复杂性和争议性\cite{mcleodVectorFourierOptics2014},我们退而求其次地选择了只考虑电场 $\symbf E^{\omega}_z$ 的切向连续性。

对于忽略反射的透射解,该方法正确且完备:只需要考虑实/倒空间的 $\symbf E^{\omega}_z$ 的切向连续性,即可完全确定传输矩阵(场),无需且不能再多此一举地,考虑 $\symbf H^{\omega}_z$ 的切向连续性或 $\symbf B^{\omega}_z$ 的法向连续性;另外,该“部分正确的”边界条件与 Berry 的两个前向本征解\cite{berryOpticalSingularitiesBirefringent2003}、Mcleod 傅立叶光学矢量索墨菲边界条件的倒空间形式\cite{mcleodVectorFourierOptics2014}兼容,三者一起,共同在数学上紧凑/集成化地给出:线性波动光学本征透射解的彻底解析/封闭形式和通式,体现在程序上即显示出准确且高速的特征。另外,在电场 $\symbf E^{\omega}_z$ 切向连续的边界条件下,除了能计算高透射率下的双透射场,原则上也能计算高反射率下的双反射场。

除了确定透/反射系数之外,边界条件还用于获得折/反射角,包括方位角、极角;同时还用于确定“何为同一束单色平面波”的判定准则,及其连续性边值关系:相同的实横向空间频率,是折/反射前后对应同一束(但可能有 $\symup{x,y}$ 双分量而是个矢量场)入射单色平面波的标志,光场的自变量即实横向波矢 $\symbf k_{\symup{\rho}} \in \mathbb{R}$ 的跨界面连续性,体现了“它们是同一束单色平面波”:基于场分量连续的要求,场的相位也必须是连续的,这主要体现在横向波矢\myHyperFootnote{两侧的\{横向波矢 → 横向相位梯度 → 横向相位梯度的积分 → 横向相位\}处处相等。}的连续,如前述 \ref{复波矢的两种表示形式} 小节中复波矢 ${\symbf k}^\omega \left( \symbf k_{\symup{\rho}} \right)$ 的 $\symbf k_{\symup{\rho}} + \symbf k^\omega_{\symup{z}} \left( \symbf k_{\symup{\rho}} \right)$ 形式所示:2 个透射解、2 个反射解,与入射光场的 $\symup{x,y}$ 双分量的对应空间频率分量,一共 6 个单色平面波组分,共享相同的实横向波矢 $\symbf k_{\symup{\rho}} \in \mathbb{R}$。鉴于我们只关注入射矢量光场的 2 个透射解,因此相互作用的场组分一般减少到 4 个,且入 - 透射关系由 $2 \times 2$ 传输矩阵相关联。

这被称为广义斯涅尔定律。尽管广义斯涅尔定律与广义菲涅尔公式是并列关系,都是边值关系的两个分支,我们选择先介绍广义斯涅尔定律。原因是在傅立叶光学中,要想考虑每个空间频率的场的场分量连续关系即倒空间的广义菲涅尔公式,必须先找到入 - 透射侧的空间频率分量,即相同单色平面波组分的自变量对应关系,也就必须先介绍广义斯涅尔定律。

%\subsection{纯电各向异性与透明介质间的广义斯涅尔定律}
\subsection{\protect\hyperlink{chap:\thesubsection}{纯电各向异性与透明介质间的广义斯涅尔定律}}
\addtocontents{toc}{\protect\linkdest{chap:\thesubsection}}
\label{纯电各向异性与透明介质间的广义斯涅尔定律}

在 \ref{复波矢的两种表示形式} 小节中提到,对于复波矢 ${\symbf k}^\omega$,若其采取
\begin{subequations} \label{eq:2-85}
\abovedisplayskip=10pt
\belowdisplayskip=10pt
\begin{align}
	{\symbf k}^\omega \left( \hat{\symbf k} \right) &= k^\omega \left( \hat{\symbf k} \right) \hat{\symbf k} \hspace{-7.5em}&&= k^\omega_{0} n^\omega \left( \hat{\symbf k} \right) \hat{\symbf k} \label{eq:2-85a} \\ &= \left( k^\omega_{\symup{R}} + \mathbb{i} \cdot k^\omega_{\symup{I}} \right) \hat{\symbf k} \hspace{-7.5em}&&= k^\omega_{0} \left( n^\omega_{\symup{R}} + \mathbb{i} \cdot n^\omega_{\symup{I}} \right) \hat{\symbf k} \label{eq:2-85b}
\end{align}
\end{subequations}
的形式,则默认沿 $\hat{\symbf k}$ 衰减/增益;而若其采取
\begin{subequations} \label{eq:2-86}
\abovedisplayskip=10pt
\belowdisplayskip=10pt
\begin{align}
	{\symbf k}^\omega \left( \symbf k_{\symup{\rho}} \right) &= \symbf k_{\symup{\rho}} + \symbf k^\omega_{\symup{z}} \left( \symbf k_{\symup{\rho}} \right) \label{eq:2-86a}\\ &= \left( \symbf k_{\symup{\rho}} + \symbf k^\omega_{\symup{zR}} \right) + \mathbb{i} \cdot \symbf k^\omega_{\symup{zI}} \label{eq:2-86b}\\ &= \left( k_{\symup{x}} \symbfup{e}_{\symup{x}} + k_{\symup{y}} \symbfup{e}_{\symup{y}} + k^\omega_{\symup{zR}} \symbfup{e}_{\symup{z}} \right) + \mathbb{i} \cdot k^\omega_{\symup{zI}} \symbfup{e}_{\symup{z}} \label{eq:2-86c}
\end{align}
\end{subequations}
的形式,则默认沿 $z$ 向衰减/增益;这两者分别是 Berry 平面波解析解、Mcloed 傅立叶光学数值解的基础;然而,二者在一开始均没有给出理由。因此,为解释这两者的来源,以及架起他们间的桥梁,不加任何假设/最广义地,只能事先认为复波矢 ${\symbf k}^\omega$ 是个双矢量,其等相位/振幅面法向,分别沿 $\hat{\symbf k}_{\mathfrak{r}}^\omega, \hat{\symbf k}_{\mathfrak{i}}^\omega$ 向:
\begin{subequations} \label{eq:2-87}
\abovedisplayskip=10pt
\belowdisplayskip=10pt
\begin{align}
	{\symbf k}^\omega \left( \hat{\symbf k}_{\mathfrak{r}}^\omega, \hat{\symbf k}_{\mathfrak{i}}^\omega \right) &= {\symbf k}^\omega_{\mathfrak{r}} + \mathbb{i} \cdot {\symbf k}^\omega_{\mathfrak{i}} \label{eq:2-87a}\\&= k^\omega_{\mathfrak{r}} \hat{\symbf k}_{\mathfrak{r}}^\omega + \mathbb{i} \cdot k^\omega_{\mathfrak{i}} \hat{\symbf k}_{\mathfrak{i}}^\omega \label{eq:2-87b}\\&=: k^\omega_{0} \left( N^\omega \hat{\symbf k}_{\mathfrak{r}}^\omega + \mathbb{i} \cdot K^\omega \hat{\symbf k}_{\mathfrak{i}}^\omega \right) ~, \label{eq:2-87c}
\end{align}
\end{subequations}
其中,$N^\omega, K^\omega$ 分别称为表观/有效折射率实/虚部。

\cref{eq:2-85,eq:2-86,eq:2-87} 中这三种形式的复波矢 ${\symbf k}^\omega$,都在形式上满足 Eq.(\ref{eq:2-31}) 中的波动方程:
\begin{equation} \label{eq:2-88}
	\left[ \bar{\bar{\symbf D}}^{\omega} - \frac{ \bar{\bar{\symbfup{I}}} }{ n^2_\omega } \right] \cdot \symbf d^{\omega} := \left[ \left( \bar{\bar{\symbfup{I}}} - \hat{\symbf k}^{\omega} \hat{\symbf k}^{\omega} \right) \cdot \bar{\bar{\symbf{\eta}}}^{\omega} - \frac{ \bar{\bar{\symbfup{I}}} }{ n^2_\omega } \right] \cdot \symbf d^{\omega} = \symbf 0 ~,
\end{equation}
前提是,三种形式的复波矢 ${\symbf k}^\omega$ 均处在这样的定义下:
\begin{subequations} \label{eq:2-89}
\begin{align}
	\hat{\symbf k}^\omega := \frac{ {\symbf k}^\omega }{ k^\omega } := \frac{ {\symbf k}^\omega }{ \sqrt{{\symbf k}^\omega \cdot {\symbf k}^\omega} } \label{eq:2-89a}~,\\ n^\omega := \frac{ k^\omega }{ k^\omega_{0} } = \frac{ \sqrt{{\symbf k}^\omega \cdot {\symbf k}^\omega} }{ k^\omega_{0} } \label{eq:2-89b}~.
\end{align}
\end{subequations}

一般地,晶体所允许的偏振态 $\symbf d^{\omega}$ 是 Eq.(\ref{eq:2-88}) 的非平凡(非零)解;为此, Eq.(\ref{eq:2-88}) 左侧方阵 $\bar{\bar{\symbf D}}^{\omega} - \bar{\bar{\symbfup{I}}} \big/ n^2_\omega$ 的行列式必须为零:
\begin{equation} \label{eq:2-90}
	\det \left[ \bar{\bar{\symbf D}}^{\omega} - \frac{ \bar{\bar{\symbfup{I}}} }{ n^2_\omega } \right] = 0 ~,
\end{equation}
而这一般\myHyperFootnote{对于最广义的表示形式 Eq.(\ref{eq:2-87}),未见资料尝试过去得到其相应的特征方程。}会产生关于 ${\symbf k}^\omega$ 的两个独立自变量(如 $\theta_{\hat{\symbf k}}, \phi_{\hat{\symbf k}}$ 或 $k_{\symup x}, k_{\symup y}$ 等)的四次方程\cite{abdulhalimExactMatrixMethod1999,olyslagerElectromagneticsExoticMedia2002,mcleodVectorFourierOptics2014},其 4 个标量根即 4 个特征值(如 $n^{\omega}$ 或 $k^{\omega}_{\symup z}$ 等);将其代回 $\bar{\bar{\symbf D}}^{\omega} - \bar{\bar{\symbfup{I}}} \big/ n^2_\omega$ 后,找其零特征值对应的特征向量(即满足三元一次/线性齐次不定方程组 Eq.(\ref{eq:2-88}) 的解\cite{XieQuanMianHuiYiWas}),即得相应的 4 个本征偏振态 $\symbf d^{\omega}$,这 4 个本征解不一定两两成对地有中心反演对称性\cite{mcleodVectorFourierOptics2014,berryOpticalSingularitiesBirefringent2003}或关于分界面的镜面对称性,并且不一定能帮助 $\symbf d^{\omega}$ 满足散度方程,因为根本没用到散度方程。

因此,Eq.(\ref{eq:2-85a}) 即 ${\symbf k}^\omega \left( \hat{\symbf k} \right)$ 形式的 Berry 版的 $\mathcal{Z}$ 系本征解 \cref{eq:2-63,eq:2-81a},只是额外附加了 Eq.(\ref{eq:2-31}) 中 $\symbf d^{\omega}$ 的横向性 和 Eq.(\ref{eq:2-85a}) 这 2 个条件的 Eq.(\ref{eq:2-88}) 的 4 个解中,$n^{\omega}_{\pm} >0$ 的 2 个倒空间的共线 $\left( \hat{\symbf k} \right)$ 前向解 ${n^{\omega}_{\pm} \left( \hat{\symbf k} \right), \symbf d^{\omega}_{\pm}} \left( \hat{\symbf k} \right)$。

然而,为符合端面边界的相位连续条件,2 个透射解不能是共线 $\left( \hat{\symbf k} \right)$ 的,必须是共用横向波矢 $\symbf k_{\symup{\rho}}$ 的;并且为满足傅立叶光学的要求,横向波矢 $\symbf k_{\symup{\rho}}$ 还必须是实的;为此,必须找到 Eq.(\ref{eq:2-81b}) 形式 ${\symbf k}^\omega \left( \symbf k_{\symup{\rho}} \right) = \symbf k_{\symup{\rho}} + \symbf k^\omega_{\symup{z}} \left( \symbf k_{\symup{\rho}} \right)$ 复波矢的解析解。

现在着手将 Berry 本征解过渡到傅立叶光学形式:由于需要将 Eq.(\ref{eq:2-90}) 的标量根代入到 Eq.(\ref{eq:2-88}) 中,才能获得其本征向量,因此 Eq.(\ref{eq:2-88}) 的本征向量只与 Eq.(\ref{eq:2-90}) 的标量根的取值一一对应,而与其具体形式(自变量是什么)无关;而 Eq.(\ref{eq:2-90}) 的标量根的傅立叶光学即 Mcleod 版本为 $k^\omega_{\symup{z}}$,Berry 版本为 Eq.(\ref{eq:2-62})。

据此假设:在相同条件下,满足 Eq.(\ref{eq:2-90}) 且值相同的 ${\symbf k}^\omega \cdot {\symbf k}^\omega$,总对应相同的特征向量(们\myHyperFootnote{可能存在简并:多个特征向量对应同一个特征值;比如对于 Berry 版本征解,特征值总是反/对极(中心对称)不变的,而特征向量并不是反极不变的\cite{berryOpticalSingularitiesBirefringent2003},则同一个特征值总有共线反向的两个不同特征向量。}),不论 ${\symbf k}^\omega$ 采取 \cref{eq:2-85,eq:2-86,eq:2-87} 中的何种形式。在这样的假设下,由于 Berry 本征解满足 Eq.(\ref{eq:2-31}),因此本征值 $n^{\omega}_{\pm}$ 对应的 $k^{2}_{\omega\pm} = k^{2}_{0\omega} n^{2}_{\omega\pm}$ 一定满足 Eq.(\ref{eq:2-90}),且其本征向量 ${\symbf d}^{\omega}_{\pm}$ 一定满足 Eq.(\ref{eq:2-88});于是,根据假设,同样在 Eq.(\ref{eq:2-31}) 的散度方程的约束下,所有值为 $k^{2}_{0\omega} n^{2}_{\omega+}$(或 $k^{2}_{0\omega} n^{2}_{\omega-}$)的
\begin{subequations} \label{eq:2-91}
\abovedisplayskip=6pt
\belowdisplayskip=6pt
\begin{align}
	k^{2}_{\omega} &= {\symbf k}^\omega \cdot {\symbf k}^\omega \label{eq:2-91a}\\ &= k^{2}_{0\omega} n^{2}_{\omega} = k^{2}_{0\omega} \left( n^{2}_{\omega \symup{R}} - n^{2}_{\omega \symup{I}} + 2 \mathbb{i} n^\omega_{\symup{R}} n^\omega_{\symup{I}} \right) \label{eq:2-91b} \\&= k^{2}_{\symup{\rho}} + \left( k^{\omega2}_{\symup{zR}} - k^{\omega2}_{\symup{zI}} + 2 \mathbb{i} k^\omega_{\symup{zR}} k^\omega_{\symup{zI}} \right) \label{eq:2-91c}\\&= k^{2}_{0\omega} \left( N^{2}_{\omega} - K^{2}_{\omega} + 2 \mathbb{i} N^\omega K^\omega \hat{\symbf k}_{\mathfrak{r}}^\omega \cdot \hat{\symbf k}_{\mathfrak{i}}^\omega \right) \label{eq:2-91d} ~,
\end{align}
\end{subequations}
不论 ${\symbf k}^\omega$ 是什么形式,都对应/拥有相同的偏振态,且都平行于 Berry 的特征向量 ${\symbf d}^{\omega}_{+}$(或 ${\symbf d}^{\omega}_{-}$),或其反/对极特征向量;尽管从数学上无法排除相应特征向量的反/对极特征向量的可能性,但物理上允许排除该可能性,因为理应与 ${\symbf k}^\omega$ 对应,而不是 $- {\symbf k}^\omega$,不论 ${\symbf k}^\omega$ 对应透射还是反射的方向。因此,假设在同一介质中,不同形式的本征值 \cref{eq:2-91a,eq:2-91b,eq:2-91c,eq:2-91d} 都对应同一个本征向量。

比较 Eq.(\ref{eq:2-91b}) 与 Eq.(\ref{eq:2-91d}) 的实虚部,可得:
\begin{equation} \label{eq:2-92}
	\abovedisplayskip=12pt
	\belowdisplayskip=12pt
	\left\{\ \begin{aligned} N^{2}_{\omega} - K^{2}_{\omega} &= n^{2}_{\omega \symup{R}} - n^{2}_{\omega \symup{I}} \\ N^\omega K^\omega &= n^\omega_{\symup{R}} n^\omega_{\symup{I}} \big/ \hat{\symbf k}_{\mathfrak{r}}^\omega \cdot \hat{\symbf k}_{\mathfrak{i}}^\omega \end{aligned}\right. ~,
\end{equation}
定义恒定相位面、恒定振幅面法向单位矢量 $\hat{\symbf k}_{\mathfrak{r}}^\omega, \hat{\symbf k}_{\mathfrak{i}}^\omega$ 间“不均匀夹角 $\beta^\omega$”:
\begin{equation} \label{eq:2-93}
	\abovedisplayskip=16pt
	\belowdisplayskip=16pt
	\beta^\omega := \left< \hat{\symbf k}_{\mathfrak{r}}^\omega, \hat{\symbf k}_{\mathfrak{i}}^\omega \right> ~,
\end{equation}
联立 Eq.(\ref{eq:2-92}) 中的两个方程,解关于 $N^\omega$ 的一元四次方程得\cite{changRayTracingAbsorbing2005}:
\begin{equation} \label{eq:2-94}
	\abovedisplayskip=20pt
	\belowdisplayskip=12pt
	\left\{\ \begin{aligned} N^{\omega} &= \pm \sqrt{ \frac{ 1 }{ 2 } \left[ n^{2}_{\omega \symup{R}} - n^{2}_{\omega \symup{I}} + \sqrt{ \left( n^{2}_{\omega \symup{R}} - n^{2}_{\omega \symup{I}} \right)^2 + 4 \left( n^\omega_{\symup{R}} n^\omega_{\symup{I}} \big/ \cos \beta^\omega \right)^2 } \right] } \\ K^\omega &= \frac{ n^\omega_{\symup{R}} n^\omega_{\symup{I}} }{ \cos \beta^\omega \cdot N^\omega } \end{aligned}\right. ~.
\end{equation}

在两介质的交界面附近,麦氏方程组 Eq.(\ref{eq:2-1}) 的积分形式所导出的,总场法/切向的连续性,要求其在折/反射前后的相位连续,即要求每个单色平面波的相位连续,这导致同一个单色平面波,在折/反射前后的切向波矢连续:
\begin{equation} \label{eq:2-95}
	\abovedisplayskip=16pt
	\belowdisplayskip=16pt
	\symbfup{n} \times {\symbf k}^\omega_{\mathsf{i}} = \symbfup{n} \times {\symbf k}^\omega_{\mathsf{r}} = \symbfup{n} \times {\symbf k}^\omega_{\mathsf{t}} ~,
\end{equation}
特别地,在 $\mathcal{Z}$ 系下,对于以 $\symbfup{n} = \symbfup{e}_{\symup{z}}$ 为法向的交界面,有:
\begin{equation} \label{eq:2-96}
	\abovedisplayskip=16pt
	\belowdisplayskip=16pt
	{\symbf k}^{\mathsf{i}\omega}_{\symup{\rho}} = {\symbf k}^{\mathsf{r}\omega}_{\symup{\rho}} = {\symbf k}^{\mathsf{t}\omega}_{\symup{\rho}} ~,
\end{equation}
由于 ${\symbf k}^\omega_{\symup{\rho}} = {\symbf k}^\omega_{\symup{\rho}\mathfrak{r}} + \mathbb{i} \cdot {\symbf k}^\omega_{\symup{\rho}\mathfrak{i}}$ 实虚部对应相等,即有
\begin{equation} \label{eq:2-97}
	\abovedisplayskip=16pt
	\belowdisplayskip=16pt
	\left\{\ \begin{aligned} {\symbf k}^{\mathsf{i}\omega}_{\symup{\rho}\mathfrak{r}} &= {\symbf k}^{\mathsf{r}\omega}_{\symup{\rho}\mathfrak{r}} = {\symbf k}^{\mathsf{t}\omega}_{\symup{\rho}\mathfrak{r}} \\ {\symbf k}^{\mathsf{i}\omega}_{\symup{\rho}\mathfrak{i}} &= {\symbf k}^{\mathsf{r}\omega}_{\symup{\rho}\mathfrak{i}} = {\symbf k}^{\mathsf{t}\omega}_{\symup{\rho}\mathfrak{i}} \end{aligned}\right. ~,
\end{equation}
其 $\symup{x,y}$ 两分量分别相等,即要求模与方位角分别相等:
\begin{equation} \label{eq:2-98}
	\abovedisplayskip=16pt
	\belowdisplayskip=6pt
	\left\{\ \begin{aligned} {k}^{\mathsf{i}\omega}_{\mathfrak{r}} \sin \theta^{\mathsf{i}\omega}_{\mathfrak{r}} &= {k}^{\mathsf{r}\omega}_{\mathfrak{r}} \sin \theta^{\mathsf{r}\omega}_{\mathfrak{r}} = {k}^{\mathsf{t}\omega}_{\mathfrak{r}} \sin \theta^{\mathsf{t}\omega}_{\mathfrak{r}} \\ {k}^{\mathsf{i}\omega}_{\mathfrak{i}} \sin \theta^{\mathsf{i}\omega}_{\mathfrak{i}} &= {k}^{\mathsf{r}\omega}_{\mathfrak{i}} \sin \theta^{\mathsf{r}\omega}_{\mathfrak{i}} = {k}^{\mathsf{t}\omega}_{\mathfrak{i}} \sin \theta^{\mathsf{t}\omega}_{\mathfrak{i}} \end{aligned}\right. ~,
\end{equation}
\begin{equation} \label{eq:2-99}
	\abovedisplayskip=6pt
	\left\{\ \begin{aligned} \phi^{\omega}_{\mathfrak{r}} &:= \phi^{\mathsf{i}\omega}_{\mathfrak{r}} = \phi^{\mathsf{r}\omega}_{\mathfrak{r}} = \phi^{\mathsf{t}\omega}_{\mathfrak{r}} \\ \phi^{\omega}_{\mathfrak{i}} &:= \phi^{\mathsf{i}\omega}_{\mathfrak{i}} = \phi^{\mathsf{r}\omega}_{\mathfrak{i}} = \phi^{\mathsf{t}\omega}_{\mathfrak{i}} \end{aligned}\right. ~,
\end{equation}
其中,像 Eq(\ref{eq:2-44}) 一样,若未标明 $\theta, \phi$ 的矢量角标,则默认其角标为 $\hat{\symbf k}_{\mathfrak{r}}^{\mathsf{i}\omega},\hat{\symbf k}_{\mathfrak{r}}^{\mathsf{r}\omega},\hat{\symbf k}_{\mathfrak{r}}^{\mathsf{t}\omega}$, $\hat{\symbf k}_{\mathfrak{i}}^{\mathsf{i}\omega},\hat{\symbf k}_{\mathfrak{i}}^{\mathsf{r}\omega},\hat{\symbf k}_{\mathfrak{i}}^{\mathsf{t}\omega}$ 中的一个与之相对应的。

方程 Eq.(\ref{eq:2-98}) 中不包含反射的透射-入射对,被称为广义斯涅尔定律:
\begin{equation} \label{eq:2-100}
	\abovedisplayskip=12pt
	\belowdisplayskip=12pt
	\left\{\ \begin{aligned} {k}^{\omega\mathfrak{r}}_{\mathsf{t}} \sin \theta^{\omega\mathfrak{r}}_{\mathsf{t}} &= {k}^{\omega\mathfrak{r}}_{\mathsf{i}} \sin \theta^{\omega\mathfrak{r}}_{\mathsf{i}} \\ {k}^{\omega\mathfrak{i}}_{\mathsf{t}} \sin \theta^{\omega\mathfrak{i}}_{\mathsf{t}} &= {k}^{\omega\mathfrak{i}}_{\mathsf{i}} \sin \theta^{\omega\mathfrak{i}}_{\mathsf{i}} \end{aligned}\right. ~,
\end{equation}
将复波矢 ${\symbf k}^\omega$ 的第 3 种形式 Eq(\ref{eq:2-87}) 代入其中得
\begin{equation} \label{eq:2-101}
	\abovedisplayskip=12pt
	\belowdisplayskip=12pt
	\left\{\ \begin{aligned} N^{\omega}_{\mathsf{t}} \sin \vartheta^{\omega}_{\mathsf{t}} &= N^{\omega}_{\mathsf{i}} \sin \vartheta^{\omega}_{\mathsf{i}} \\ K^{\omega}_{\mathsf{t}} \sin \psi^{\omega}_{\mathsf{t}} &= K^{\omega}_{\mathsf{i}} \sin \psi^{\omega}_{\mathsf{i}} \end{aligned}\right. ~,
\end{equation}
其中为减少角标数量,定义
\begin{equation} \label{eq:2-102}
	\abovedisplayskip=12pt
	\belowdisplayskip=6pt
	\left\{\ \begin{aligned} \vartheta &:= \theta_{\mathfrak{r}} \\ \psi &:= \theta_{\mathfrak{i}} \end{aligned}\right. ~.
\end{equation}

设入射光所在介质 $\mathsf{i}$ 无吸收,即 $n^{\mathsf{i}\omega}_{\symup{I}} \left( \hat{\symbf k}_{\mathsf{i}} \right) \equiv 0$、$n^{\omega}_{\mathsf{i}} \equiv n^{\mathsf{i}\omega}_{\symup{R}}$,代入 Eq.(\ref{eq:2-94}) 得:
\begin{equation} \label{eq:2-103}
	\abovedisplayskip=12pt
	\belowdisplayskip=12pt
	\left\{\ \begin{aligned} N^{\omega}_{\mathsf{i}} &= n^{\omega}_{\mathsf{i}} \\ K^{\omega}_{\mathsf{i}} &= 0 \end{aligned}\right. ~,
\end{equation}
可知此时介质 $\mathsf{i}$ 中的复波矢 ${\symbf k}^\omega_{\mathsf{i}}$ 是个实矢量,没有虚部,可以由单一方向 
\begin{equation} \label{eq:2-104}
	\abovedisplayskip=12pt
	\belowdisplayskip=12pt
	\hat{\symbf k}_{\mathsf{i}} = \hat{\symbf k}_{\mathfrak{r}}^{\mathsf{i}\omega} = \mathcal{N} \left[ \symbf k^{\mathsf{i}}_{\symup{\rho}} + \symbf k^{\mathsf{i}\omega}_{\symup{z}} \right] =: \hat{\symbf k}^\omega_{\mathsf{i}} = \hat{\symbf k}^\omega_{\mathsf{i}} \left( \vartheta^{\omega}_{\mathsf{i}}, \phi^{\mathsf{i}\omega}_{\mathfrak{r}} \right) = \hat{\symbf k}^\omega_{\mathsf{i}} \left( \theta^{\omega}_{\mathsf{i}}, \phi \right)
\end{equation}
确定,以至于 $n^{\omega}_{\mathsf{i}}$ 只是 Eq.(\ref{eq:2-104}) 的函数,根据类似 Eq.(\ref{eq:2-38a}) 的规则,$n^{\omega}_{\mathsf{i}}$ 也是
\begin{subequations} \label{eq:2-105}
\abovedisplayskip=8pt
\belowdisplayskip=8pt
\begin{align}
	\theta^{\omega}_{\mathsf{i}} \left( n^{\omega}_{\mathsf{i}} \right) &= \arcsin \left( k^{\mathsf{i}}_{\symup{\rho}} \big/ k^{\omega}_{\mathsf{i}} \right) = \arcsin \frac{ k^{\mathsf{i}}_{\symup{\rho}} }{ k^{\omega}_{0} n^{\omega}_{\mathsf{i}} } \label{eq:2-105a}~,\\ \phi &= \text{arctan2} \left[k^{\mathsf{i}}_{\symup{y}}, k^{\mathsf{i}}_{\symup{x}}\right] \label{eq:2-105b}
\end{align}
\end{subequations}
的函数 $n^{\omega}_{\mathsf{i}} = n^{\omega}_{\mathsf{i}} \left( \theta^{\omega}_{\mathsf{i}}, \phi \right) = n^{\omega}_{\mathsf{i}} \left[ \theta^{\omega}_{\mathsf{i}} \left( n^{\omega}_{\mathsf{i}} \right), \phi \right] = n^{\omega}_{\mathsf{i}} \left( \arcsin \left[ k^{\mathsf{i}}_{\symup{\rho}} \big/ \left( k^{\omega}_{0} n^{\omega}_{\mathsf{i}} \right) \right], \phi \right) $,以至于折射率/波矢分量/方向满足以下超越方程关系\cite{grechinFourierSpaceMethod2014,mcleodVectorFourierOptics2014}或递归表达式:
\begin{equation} \label{eq:2-106}
	\abovedisplayskip=12pt
	\belowdisplayskip=12pt
	n^{\omega}_{\mathsf{i}} = n^{\omega}_{\mathsf{i}} \left[ \theta^{\omega}_{\mathsf{i}} \left( n^{\omega}_{\mathsf{i}} \right), \phi \right] ~,
\end{equation}
在横向空间频率/实波矢网格/实二维数组 $k^{\mathsf{i}}_{\symup{x}}, k^{\mathsf{i}}_{\symup{y}}$ 已知且确定,导致标量场 $\phi$ 以及 $k^{\mathsf{i}}_{\symup{\rho}}$ 已知且不变的情况下,可通过递归/迭代\cite{grechinFourierSpaceMethod2014}求解 Eq.(\ref{eq:2-106}):
\begin{equation} \label{eq:2-107}
	\abovedisplayskip=12pt
	\belowdisplayskip=12pt
	n^{\mathsf{i}\omega}_{2} = n^{\mathsf{i}\omega}_{2} \left( \theta^{\mathsf{i}\omega}_{1} \left[ n^{\mathsf{i}\omega}_{1} \left( 0, 0 \right) \right], \phi \right) ~,
\end{equation}
其中,第 1 次代入的初始极角和方位角,规定为界面法向 $\symbfup{n} = \symbfup{e}_{\symup{z}}$ 即 $\mathcal{Z}$ 系 $\symup{z}$ 向的 
\begin{equation} \label{eq:2-108}
	\abovedisplayskip=12pt
	\belowdisplayskip=12pt
	\theta^{\mathsf{i}\omega}_{0}, \phi^{\mathsf{i}}_{0} = 0, 0 ~,
\end{equation}
对应的第 1 次迭代所产生的折射率,为介质 $\mathsf{i}$ 内 $\symbfup{e}_{\symup{z}}$ 向、各向同性的折射率 
\begin{equation} \label{eq:2-109}
	\abovedisplayskip=13pt
	\belowdisplayskip=13pt
	n^{\mathsf{i}\omega}_{1} \left( \theta^{\mathsf{i}\omega}_{0}, \phi^{\mathsf{i}}_{0} \right) = n^{\mathsf{i}\omega}_{1} \left( 0, 0 \right) ~,
\end{equation}
以用其代表所有方向上的平均折射率,然后将该平均折射率代入 Eq.(\ref{eq:2-106}),即得第 2 次迭代的结果 Eq.(\ref{eq:2-107}),该折射率是各向异性的标量场。一般只需要迭代至第 2 次,结果已足够精确,因为试验发现第 3 次递归产生的 Eq.(\ref{eq:2-106}) 相对于第 2 次,几乎没变化。当然,如有需要,也可迭代 $i > 2$ 次:
\begin{equation} \label{eq:2-110}
	\abovedisplayskip=13pt
	\belowdisplayskip=13pt
	n^{\mathsf{i}\omega}_{i+1} = n^{\mathsf{i}\omega}_{i+1} \left( \theta^{\mathsf{i}\omega}_{i} \left[ n^{\mathsf{i}\omega}_{i} \left( \theta^{\mathsf{i}\omega}_{i-1}, \phi \right) \right], \phi \right) ~,
\end{equation}
一般地,在晶体前端面,入射侧介质 $\mathsf{i}$ 是各向同性的,此时 $n^{\omega}_{\mathsf{i}}$ 不是 $\theta^{\omega}_{\mathsf{i}}, \phi$ 的函数,于是介质 $\mathsf{i}$ 内入射光实波矢 ${\symbf k}^\omega_{\mathsf{i}}$ 的方位角、极角直接由 Eq(\ref{eq:2-105}) 确定;只有在透/折射侧介质 $\mathsf{t}$ 内,因其多为电各向异性的缘故,$n^{\omega}_{\mathsf{t}}$ 才是 $\theta^{\omega}_{\mathsf{t}}, \phi_{\mathsf{t}}$ 的函数,才需要用到 2 次迭代的 Eq.(\ref{eq:2-107}) 或 $i > 2$ 次迭代的 Eq.(\ref{eq:2-110})。

但若考虑晶体后端面,入射侧介质 $\mathsf{i}$ 通常是各向异性的晶体,透射侧介质 $\mathsf{t}$ 通常是各向同性的空气。此时,透射光实波矢 ${\symbf k}^\omega_{\mathsf{t}}$ 的 $\theta^{\omega}_{\mathsf{t}}, \phi_{\mathsf{t}}$ 直接由 Eq(\ref{eq:2-105}) 确定;而入射光实波矢 ${\symbf k}^\omega_{\mathsf{i}}$ 的折射率,需用 Eq.(\ref{eq:2-107}) 或 Eq.(\ref{eq:2-110}) 迭代,之后才能用 Eq(\ref{eq:2-105}) 确定 ${\symbf k}^\omega_{\mathsf{i}}$ 的 $\theta^{\omega}_{\mathsf{i}}, \phi$ 即 $\theta^{\mathsf{i}\omega}_{2}, \phi$ 或 $\theta^{\mathsf{i}\omega}_{i+1}, \phi$。

此外,将 Eq.(\ref{eq:2-103}) 代入广义斯涅尔定律 Eq.(\ref{eq:2-101}) 得:
\begin{equation} \label{eq:2-111}
	\abovedisplayskip=13pt
	\belowdisplayskip=13pt
	\left\{\ \begin{aligned} N^{\omega}_{\mathsf{t}} \sin \vartheta^{\omega}_{\mathsf{t}} &= n^{\omega}_{\mathsf{i}} \sin \vartheta^{\omega}_{\mathsf{i}} \\ K^{\omega}_{\mathsf{t}} \sin \psi^{\omega}_{\mathsf{t}} &= 0 \end{aligned}\right. ~.
\end{equation}

先考察上述 Eq.(\ref{eq:2-111}) 的第 2 个方程:当透射光所处的介质 $\mathsf{t}$ 有吸收即 $K^{\omega}_{\mathsf{t}} \neq 0$ 时,吸收方向 $\hat{\symbf k}_{\mathfrak{i}}^{\mathsf{t}\omega}$ 必然是沿界面法向 $\symbfup{n} = \symbfup{e}_{\symup{z}}$ 和 $\mathcal{Z}$ 系 $\symup{z}$ 向的,即 $\psi^{\omega}_{\mathsf{t}} = 0$ 推出
\begin{equation} \label{eq:2-112}
	\abovedisplayskip=13pt
	\belowdisplayskip=13pt
	\hat{\symbf k}_{\mathfrak{i}}^{\mathsf{t}\omega} \equiv \symbfup{e}_{\symup{z}} ~,
\end{equation}
这就直接给出了 Mcleod 的复波矢形式 Eq(\ref{eq:2-86}) 的合理性,尽管这要求分界面某侧的材料无吸收 Eq(\ref{eq:2-103});接着,将 Eq.(\ref{eq:2-112}) 代入 Eq.(\ref{eq:2-93}) 将导出
\begin{subequations} \label{eq:2-113}
\abovedisplayskip=6pt
\belowdisplayskip=6pt
\begin{align}
	\beta^\omega_{\mathsf{t}} &= \left< \hat{\symbf k}_{\mathfrak{r}}^{\mathsf{t}\omega}, \hat{\symbf k}_{\mathfrak{i}}^{\mathsf{t}\omega} \right> \label{eq:2-113a} \\ &= \left< \hat{\symbf k}_{\mathfrak{r}}^{\mathsf{t}\omega}, \symbfup{e}_{\symup{z}} \right> = \vartheta^{\omega}_{\mathsf{t}} \label{eq:2-113b}~,
\end{align}
\end{subequations}
更进一步地,$\psi^{\omega}_{\mathsf{t}} = 0$ 还将导致 Eq.(\ref{eq:2-99}) 中的 $\phi^{\omega}_{\mathfrak{i}} = \phi^{\mathsf{i}\omega}_{\mathfrak{i}} = \phi^{\mathsf{r}\omega}_{\mathfrak{i}} = \phi^{\mathsf{t}\omega}_{\mathfrak{i}}$ 取任何值都没有意义,因此允许我们无需考虑入/折/反射复波矢虚部实向量 ${\symbf k}_{\mathfrak{i}}^{\omega}$ 的方位角,只需考虑其实部实向量 ${\symbf k}_{\mathfrak{r}}^{\omega}$ 的方位角:
\begin{equation} \label{eq:2-114}
	\abovedisplayskip=10pt
	\belowdisplayskip=10pt
	\varphi^{\omega} := \varphi^{\omega}_{\mathsf{i}} = \varphi^{\omega}_{\mathsf{r}} = \varphi^{\omega}_{\mathsf{t}} ~,
\end{equation}
其中,像 Eq.(\ref{eq:2-102}) 一样,用特殊的字符 $\varphi$ 来简约地特指剩下的实部 ${\symbf k}_{\mathfrak{r}}^{\omega}$ 的方位角 $\phi_{\mathfrak{r}}$,以减少角标数量:
\begin{equation} \label{eq:2-115}
	\abovedisplayskip=8pt
	\belowdisplayskip=8pt
	\varphi := \phi_{\mathfrak{r}} ~.
\end{equation}

接着,将 Eq.(\ref{eq:2-113a}) 即 $\beta^\omega_{\mathsf{t}} = \vartheta^{\omega}_{\mathsf{t}}$ 代入取值为正(对应朝 $\symbfup{e}_{\symup{z}}$ 指向的半无限空间传播的透射波)的 Eq.(\ref{eq:2-94}) 中的有效折射率实部 $N^{\omega}_{\mathsf{t}} = N^{\omega}_{\mathsf{t}} \left( n^{\omega}_{\mathsf{t}}, \beta^{\omega}_{\mathsf{t}} \right) = N^{\omega}_{\mathsf{t}} \left[ n^{\omega}_{\mathsf{t}} \left( \hat{\symbf k}_{\mathsf{t}} \right), \vartheta^{\omega}_{\mathsf{t}} \right]$,再代入某侧材料无吸收 Eq(\ref{eq:2-103}) 的广义斯涅尔定律 Eq.(\ref{eq:2-111}) 的第 1 个方程(这一步不是必须的,见 Eq.(\ref{eq:2-123})),得:
\begin{equation} \label{eq:2-116}
	\abovedisplayskip=10pt
	\belowdisplayskip=10pt
	N^{\omega}_{\mathsf{t}} \left[ n^{\omega}_{\mathsf{t}} \left( \hat{\symbf k}_{\mathsf{t}} \right), \vartheta^{\omega}_{\mathsf{t}} \right] \sin \vartheta^{\omega}_{\mathsf{t}} = n^{\omega}_{\mathsf{i}} \sin \vartheta^{\omega}_{\mathsf{i}} ~,
\end{equation}
该方程左侧同时是 $\hat{\symbf k}_{\mathsf{t}}, \vartheta^{\omega}_{\mathsf{t}}$ 的函数,因此,对于方程右侧给定的 $n^{\omega}_{\mathsf{i}}, \vartheta^{\omega}_{\mathsf{i}}$,方程 Eq.(\ref{eq:2-116}) 并不能给出唯一确定的 $\vartheta^{\omega}_{\mathsf{t}}$ 与之对应:因为还与 $\hat{\symbf k}_{\mathsf{t}}$ 有关,而复波矢第一种形式 Eq.(\ref{eq:2-85a}) 的自变量 $\hat{\symbf k}_{\mathsf{t}}$ 尚未与第三种形式 Eq.(\ref{eq:2-87c}) 中单位实波矢 $\hat{\symbf k}_{\mathfrak{r}}^{\mathsf{t}\omega}$ 及其极角 $\vartheta^{\omega}_{\mathsf{t}}$ 相关联。

对此,我们只能在三种复波矢形式的虚部均远小于实部的条件\label{con:2}下(一般的晶体满足该条件,即使现象上强吸收,虚部也不会太大,见下文中的图;另外,由于我们关注透射解的演化,强吸收或高反射的情况本身就被排除在外,因为此时透射解传播不了多远,就耗散殆尽了,即使查看也是平庸的零解):
\begin{subequations} \label{eq:2-117}
\abovedisplayskip=10pt
\belowdisplayskip=10pt
\begin{alignat}{4}
	&n^\omega_{\symup{I}} &&\ll n^\omega_{\symup{R}} \hspace{2.0em} &&\text{or} \hspace{2.0em} n^{2}_{\omega \symup{R}} - n^{2}_{\omega \symup{I}} &&\gg 2 n^\omega_{\symup{R}} n^\omega_{\symup{I}} \label{eq:2-117a}~,\\ &k^\omega_{\symup{zI}} &&\ll \sqrt{ k^{2}_{\symup{\rho}} + k^{\omega2}_{\symup{zR}} } \hspace{2.0em} &&\text{or} \hspace{2.0em} k^{2}_{\symup{\rho}} + k^{\omega2}_{\symup{zR}} - k^{\omega2}_{\symup{zI}} &&\gg 2 k^\omega_{\symup{zR}} k^\omega_{\symup{zI}} \label{eq:2-117b}~,\\ &K^{\omega} &&\ll N^{\omega} \hspace{2.0em} &&\text{or} \hspace{2.0em} N^{2}_{\omega} - K^{2}_{\omega} &&\gg 2 N^\omega K^\omega \hat{\symbf k}_{\mathfrak{r}}^\omega \cdot \hat{\symbf k}_{\mathfrak{i}}^\omega \label{eq:2-117c}~,
\end{alignat}
\end{subequations}
才能认为三种形式的复波矢 ${\symbf k}^\omega$,类似 Eq.(\ref{eq:2-104}) 地,从双矢量接近退化为单一且同一方向
\begin{equation} \label{eq:2-118}
	\hat{\symbf k}^\omega := \hat{\symbf k} = \hat{\symbf k}_{\mathfrak{r}}^\omega = \mathcal{N} \left[ \symbf k_{\symup{\rho}} + \symbf k^\omega_{\symup{zR}} \right] ~,
\end{equation}
甚至同一个(模也相同的)实矢量(因为无衰减的单色平面波只有一个实波矢):
\begin{equation} \label{eq:2-119}
	\abovedisplayskip=10pt
	\belowdisplayskip=10pt
	{\symbf k}^\omega := k^\omega_{0} n^\omega_{\symup{R}} \hat{\symbf k} = k^\omega_{0} N^\omega \hat{\symbf k}_{\mathfrak{r}}^\omega = \symbf k_{\symup{\rho}} + \symbf k^\omega_{\symup{zR}} ~,
\end{equation}
并且这些实向量因互相平行,而共享极角、方位角 Eq(\ref{eq:2-105}):
\begin{subequations} \label{eq:2-120}
\abovedisplayskip=10pt
\belowdisplayskip=10pt
\begin{align}
	\theta^{\omega} = \vartheta^{\omega} &= \arcsin \left( k_{\symup{\rho}} \big/ k^\omega_{\mathfrak{r}} \right) = \arcsin \frac{ k_{\symup{\rho}} }{ k^{\omega}_{0} N^{\omega} } \label{eq:2-120a}~,\\ \phi = \varphi &= \text{arctan2} \left[k_{\symup{y}}, k_{\symup{x}}\right] \label{eq:2-120b} ~,
\end{align}
\end{subequations}
其中 $\theta^{\omega} = \vartheta^{\omega}$ 是 $n^\omega_{\symup{R}} = N^{\omega}$ 的函数,且与 $\hat{\symbf k} = \hat{\symbf k}_{\mathfrak{r}}^\omega$ 一样,均还额外是 $\omega$ 的函数\myHyperFootnote{注意:在复波矢第一种形式 Eq.(\ref{eq:2-85a}) 定义之初 ${\symbf k}^\omega \left( \hat{\symbf k} \right) = k^\omega \left( \hat{\symbf k} \right) \hat{\symbf k}$ 中,$\hat{\symbf k}$ 并不是 $\omega$ 的函数。},是 \cref{eq:2-118,eq:2-119} 共同作用的结果;而 Eq.(\ref{eq:2-120b}) 中的 $\phi = \varphi$ 以及 Eq.(\ref{eq:2-105b}) 中的 $\phi$,与 $\omega$ 无关,是 \cref{eq:2-86,eq:2-119} 的要求。

注意,\cref{eq:2-118,eq:2-119,eq:2-120} 在任何满足 Eq(\ref{eq:2-117}) 即吸收不太强的介质中都成立,不一定需要是透射介质 $\mathfrak{t}$。接下来,假设 Eq.(\ref{eq:2-116}) 左侧对应的透射介质中各物理量满足 Eq(\ref{eq:2-117}) 条件。

于是,将该条件下的结论 $\hat{\symbf k} = \hat{\symbf k}_{\mathfrak{r}}^\omega$ 代入 Eq.(\ref{eq:2-116}),并利用 Eq.(\ref{eq:2-120}),将其写成纯角度(极角、方位角)的函数,得
\begin{subequations} \label{eq:2-121}
\begin{align}
	N^{\omega}_{\mathsf{t}} \left[ n^{\omega}_{\mathsf{t}} \left( \hat{\symbf k}^{\mathsf{t}\omega}_{\mathfrak{r}} \right), \vartheta^{\omega}_{\mathsf{t}} \right] \sin \vartheta^{\omega}_{\mathsf{t}} &= n^{\omega}_{\mathsf{i}} \sin \vartheta^{\omega}_{\mathsf{i}} \label{eq:2-121a} \\ N^{\omega}_{\mathsf{t}} \left[ n^{\omega}_{\mathsf{t}} \left( \vartheta^{\omega}_{\mathsf{t}}, \varphi_{\mathsf{t}} \right), \vartheta^{\omega}_{\mathsf{t}} \right] \sin \vartheta^{\omega}_{\mathsf{t}} &= n^{\omega}_{\mathsf{i}} \sin \vartheta^{\omega}_{\mathsf{i}} \label{eq:2-121b}~,
\end{align}
\end{subequations}
将广义斯涅尔定律导出的复波矢实部的方位角连续性方程 Eq.(\ref{eq:2-114}) 即 $\varphi := \varphi_{\mathsf{t}} = \varphi_{\mathsf{i}}$ 代入上式 Eq.(\ref{eq:2-121b}),得
\begin{equation} \label{eq:2-122}
	N^{\omega}_{\mathsf{t}} \left[ n^{\omega}_{\mathsf{t}} \left( \vartheta^{\omega}_{\mathsf{t}}, \varphi \right), \vartheta^{\omega}_{\mathsf{t}} \right] \sin \vartheta^{\omega}_{\mathsf{t}} = n^{\omega}_{\mathsf{i}} \sin \vartheta^{\omega}_{\mathsf{i}} ~,
\end{equation}
相比 Eq.(\ref{eq:2-116}),该方程未知量只有一个 $\vartheta^{\omega}_{\mathsf{t}}$,因此对于确定的 $\vartheta^{\omega}_{\mathsf{i}}, \varphi$,有唯一解。

尽管如此,递归/迭代求解 Eq.(\ref{eq:2-122}) 并不明智:因为现成的方案已由 \cref{eq:2-105,eq:2-106,eq:2-107,eq:2-108,eq:2-109,eq:2-110} 给出:现只需像 Eq.(\ref{eq:2-106}) 一样,考虑 Eq.(\ref{eq:2-122}) 左侧的表观/有效折射率实部:
\begin{equation} \label{eq:2-123}
	\abovedisplayskip=10pt
	\belowdisplayskip=13pt
	N^{\omega}_{\mathsf{t}} = N^{\omega}_{\mathsf{t}} \left[ n^{\omega}_{\mathsf{t}} \left( \theta^{\omega}_{\mathsf{t}}, \phi \right), \theta^{\omega}_{\mathsf{t}} \right] ~,
\end{equation}
其中,用到了透射侧弱吸收 Eq(\ref{eq:2-117}) 条件下的 Eq.(\ref{eq:2-120}) 中的 $\theta^{\omega}_{\mathsf{t}} = \vartheta^{\omega}_{\mathsf{t}}, \phi = \varphi$,且有类似 Eq.(\ref{eq:2-105}) 的
\begin{subequations} \label{eq:2-124}
\abovedisplayskip=8pt
\belowdisplayskip=10pt
\begin{align}
	\theta^{\omega}_{\mathsf{t}} \left( N^{\omega}_{\mathsf{t}} \right) &= \arcsin \left( k^{\mathsf{t}}_{\symup{\rho}} \big/ k^{{\mathsf{t}}\omega}_{\mathfrak{r}} \right) = \arcsin \frac{ k^{\mathsf{t}}_{\symup{\rho}} }{ k^{\omega}_{0} N^{\omega}_{\mathsf{t}} } \label{eq:2-124a}~,\\ \phi &= \text{arctan2} \left[k^{\mathsf{t}}_{\symup{y}}, k^{\mathsf{t}}_{\symup{x}}\right] \label{eq:2-124b} ~,
\end{align}
\end{subequations}
据广义斯涅尔定律要求的复波矢实部的连续性 Eq.(\ref{eq:2-97}) 即 ${\symbf k}^{\mathsf{i}\omega}_{\symup{\rho}\mathfrak{r}} = {\symbf k}^{\mathsf{r}\omega}_{\symup{\rho}\mathfrak{r}} = {\symbf k}^{\mathsf{t}\omega}_{\symup{\rho}\mathfrak{r}}$,以及分界面两侧介质均弱吸收 Eq(\ref{eq:2-117}) 条件下的 Eq.(\ref{eq:2-119}),有
\begin{equation} \label{eq:2-125}
	\abovedisplayskip=13pt
	\belowdisplayskip=13pt
	{\symbf k}^{\mathsf{i}}_{\symup{\rho}} = {\symbf k}^{\mathsf{r}}_{\symup{\rho}} = {\symbf k}^{\mathsf{t}}_{\symup{\rho}} ~,
\end{equation}
即有与 \cref{eq:2-98,eq:2-99} 等价的
\begin{subequations} \label{eq:2-126}
\abovedisplayskip=8pt
\belowdisplayskip=13pt
\begin{align}
	&\hspace{1.1em} k_{\symup{\rho}} = k^{\mathsf{i}}_{\symup{\rho}} = k^{\mathsf{r}}_{\symup{\rho}} = k^{\mathsf{t}}_{\symup{\rho}} \label{eq:2-126a}~,\\ &\left\{\ \begin{aligned} k_{\symup{x}} = k^{\mathsf{i}}_{\symup{x}} = k^{\mathsf{r}}_{\symup{x}} = k^{\mathsf{t}}_{\symup{x}} \\ k_{\symup{y}} = k^{\mathsf{i}}_{\symup{y}} = k^{\mathsf{r}}_{\symup{y}} = k^{\mathsf{t}}_{\symup{y}} \end{aligned}\right. \label{eq:2-126b}~,
\end{align}
\end{subequations}
以至于 Eq(\ref{eq:2-124}) 变为\myHyperFootnote{其中略微用到了 Eq.(\ref{eq:2-119}) 中的 ${\symbf k}^\omega := {\symbf k}_{\mathfrak{r}}^\omega$。}:
\begin{subequations} \label{eq:2-127}
\abovedisplayskip=8pt
\belowdisplayskip=10pt
\begin{align}
	\theta^{\omega}_{\mathsf{t}} \left( N^{\omega}_{\mathsf{t}} \right) &= \arcsin \left( k_{\symup{\rho}} \big/ k^{\omega}_{\mathsf{t}} \right) = \arcsin \frac{ k_{\symup{\rho}} }{ k^{\omega}_{0} N^{\omega}_{\mathsf{t}} } \label{eq:2-127a}~,\\ \phi &= \text{arctan2} \left[k_{\symup{y}}, k_{\symup{x}}\right] \label{eq:2-127b} ~,
\end{align}
\end{subequations}
同时,Eq.(\ref{eq:2-105}) 也变为:
\begin{subequations} \label{eq:2-128}
\abovedisplayskip=8pt
\belowdisplayskip=10pt
\begin{align}
	\theta^{\omega}_{\mathsf{i}} \left( n^{\omega}_{\mathsf{i}} \right) &= \arcsin \left( k_{\symup{\rho}} \big/ k^{\omega}_{\mathsf{i}} \right) = \arcsin \frac{ k_{\symup{\rho}} }{ k^{\omega}_{0} n^{\omega}_{\mathsf{i}} } \label{eq:2-128a}~,\\ \phi &= \text{arctan2} \left[k_{\symup{y}}, k_{\symup{x}}\right] \label{eq:2-128b}~.
\end{align}
\end{subequations}
将 Eq.(\ref{eq:2-127a}) 代入 Eq.(\ref{eq:2-123}),即得到类似 Eq.(\ref{eq:2-106}) 的递归表达式:
\begin{equation} \label{eq:2-129}
	N^{\omega}_{\mathsf{t}} = N^{\omega}_{\mathsf{t}} \left\{ n^{\omega}_{\mathsf{t}} \left[ \theta^{\omega}_{\mathsf{t}} \left( N^{\omega}_{\mathsf{t}} \right), \phi \right], \theta^{\omega}_{\mathsf{t}} \left( N^{\omega}_{\mathsf{t}} \right) \right\} ~,
\end{equation}
接着,仿照 \cref{eq:2-106,eq:2-107,eq:2-108,eq:2-109,eq:2-110} 的步骤,递归求解 Eq.(\ref{eq:2-129})。

第 1 次迭代,所代入的极角和方位角,仍规定为 $\mathcal{Z}$ 系 $\symup{z}$ 向所对应的
\begin{equation} \label{eq:2-130}
	\theta^{\mathsf{t}\omega}_{0}, \phi_{0} = 0, 0 ~,
\end{equation}
第 1 次迭代,产生介质 $\mathsf{t}$ 内 $\symbfup{e}_{\symup{z}}$ 向各向同性折射率(定标量):
\begin{equation} \label{eq:2-131}
	N^{\mathsf{t}\omega}_{1} = N^{\mathsf{t}\omega}_{1} \left[ n^{\mathsf{t}\omega}_{1} \left( 0, 0 \right), 0 \right] ~,
\end{equation}
第 2 次迭代,产生为较精确的各向异性折射率(标量场):
%N^{\mathsf{t}\omega}_{2} = N^{\mathsf{t}\omega}_{2} \Big\{ &n^{\mathsf{t}\omega}_{2} \left[ \theta^{\mathsf{t}\omega}_{1} \left( N^{\mathsf{t}\omega}_{1} \left[ n^{\mathsf{t}\omega}_{1} \left( 0, 0 \right), 0 \right] \right), \phi \right] \\&\theta^{\mathsf{t}\omega}_{1} \left( N^{\mathsf{t}\omega}_{1} \left[ n^{\mathsf{t}\omega}_{1} \left( 0, 0 \right), 0 \right] \right) \Big\} \label{eq:2-132b}~
\begin{subequations} \label{eq:2-132}
\begin{align}
	N^{\mathsf{t}\omega}_{2} &= N^{\mathsf{t}\omega}_{2} \left\{ n^{\mathsf{t}\omega}_{2} \left[ \theta^{\mathsf{t}\omega}_{1} \left( N^{\mathsf{t}\omega}_{1} \right), \phi \right], \theta^{\mathsf{t}\omega}_{1} \left( N^{\mathsf{t}\omega}_{1} \right) \right\} \label{eq:2-132a} \\ &= N^{\mathsf{t}\omega}_{2} \left\{ n^{\mathsf{t}\omega}_{2} \left[ \theta^{\mathsf{t}\omega}_{1} \left( N^{\mathsf{t}\omega}_{1} \left[ n^{\mathsf{t}\omega}_{1} \left( 0, 0 \right), 0 \right] \right), \phi \right], \theta^{\mathsf{t}\omega}_{1} \left( N^{\mathsf{t}\omega}_{1} \left[ n^{\mathsf{t}\omega}_{1} \left( 0, 0 \right), 0 \right] \right) \right\} \label{eq:2-132b}~,
\end{align}
\end{subequations}
第 $i > 2$ 次迭代,产生更精确的各向异性折射率(标量场):
\begin{equation} \label{eq:2-133}
	N^{\mathsf{t}\omega}_{i} = N^{\mathsf{t}\omega}_{i} \left\{ n^{\mathsf{t}\omega}_{i} \left[ \theta^{\mathsf{t}\omega}_{i-1} \left( N^{\mathsf{t}\omega}_{i-1} \right), \phi \right], \theta^{\mathsf{t}\omega}_{i-1} \left( N^{\mathsf{t}\omega}_{i-1} \right) \right\} ~.
\end{equation}
另外,迭代出 $N^{\mathsf{t}\omega}_{i}$ 后,通过 Eq(\ref{eq:2-127a}) 可以得到
\begin{equation} \label{eq:2-134}
	\theta^{\mathsf{t}\omega}_{i} \left( N^{\mathsf{t}\omega}_{i} \right) = \arcsin \frac{ k_{\symup{\rho}} }{ k^{\omega}_{0} N^{\mathsf{t}\omega}_{i} } ~,
\end{equation}
比如得到 $N^{\mathsf{t}\omega}_{2}$ 后,则可得到 $\theta^{\mathsf{t}\omega}_{2} \left( N^{\mathsf{t}\omega}_{2} \right) = \arcsin \left[ k_{\symup{\rho}} \big/ \left( k^{\omega}_{0} N^{\mathsf{t}\omega}_{2} \right) \right]$;一般就使用到第 2 次迭代结果 $N^{\mathsf{t}\omega}_{2}, \theta^{\mathsf{t}\omega}_{2}$:只有在泵浦紧聚焦,或材料强各向异性的情况下,才需要考虑 $i > 2$ 次迭代;否则迭代收敛得很快,以至于只需要迭代 2 次即可,即直接使用 Eq(\ref{eq:2-132b}) 及 Eq(\ref{eq:2-134})。

当然,与 $N^{\omega}_{\mathsf{t}}$ 迭代式 Eq(\ref{eq:2-133}) 即 $N^{\mathsf{t}\omega}_{i} \left( N^{\mathsf{t}\omega}_{i-1} \right)$ 对应地,将 Eq(\ref{eq:2-123}) 或 Eq(\ref{eq:2-133}) 代入 Eq(\ref{eq:2-134}),也有 $\theta^{\omega}_{\mathsf{t}}$ 迭代式 $\theta^{\mathsf{t}\omega}_{i} \left( \theta^{\mathsf{t}\omega}_{i-1} \right)$:
\begin{equation} \label{eq:2-135}
	\theta^{\mathsf{t}\omega}_{i} = \arcsin \frac{ k_{\symup{\rho}} }{ k^{\omega}_{0} N^{\mathsf{t}\omega}_{i} \left[ n^{\mathsf{t}\omega}_{i} \left( \theta^{\mathsf{t}\omega}_{i-1}, \phi \right), \theta^{\mathsf{t}\omega}_{i-1} \right] } ~,
\end{equation}
但若先采用该式 Eq(\ref{eq:2-135}) 产生 $\theta^{\mathsf{t}\omega}_{i}$,再通过 Eq(\ref{eq:2-123}) 产生 $N^{\mathsf{t}\omega}_{i+1}$,二者 $\theta^{\mathsf{t}\omega}_{i}, N^{\mathsf{t}\omega}_{i+1}$ 的迭代层级将不对应;所以一般先使用 Eq(\ref{eq:2-132b}),再用 Eq(\ref{eq:2-134}),以使 $N^{\mathsf{t}\omega}_{i}, \theta^{\mathsf{t}\omega}_{i}$ 迭代层级对应。

在通过 \cref{eq:2-127b,eq:2-133,eq:2-134} 分别得到了指定空间频率 $k_{\symup{x}}, k_{\symup{y}}$ 对应的 $\phi, N^{\omega}_{\mathsf{t}}, \theta^{\omega}_{\mathsf{t}}$ 之后,还需得到与 $N^{\omega}_{\mathsf{t}}$ 相关的 Mcleod 形式的复本征值 $k^{\mathsf{t}\omega}_{\symup{z}}$ 和复波矢 ${\symbf k}^\omega_{\mathsf{t}} \left( \symbf k_{\symup{\rho}}^{\mathsf{t}} \right)$,以纳入傅立叶光学:通过查看 ${\symbf k}^\omega \cdot {\symbf k}^\omega$ 相同以共享本征向量的条件 \cref{eq:2-91c,eq:2-91d},对比实虚部得:
\begin{equation} \label{eq:2-136}
	\left\{\ \begin{aligned} k^{2}_{\symup{\rho}} + \left( k^{\omega2}_{\symup{zR}} - k^{\omega2}_{\symup{zI}} \right) &= k^{2}_{0\omega} \left( N^{2}_{\omega} - K^{2}_{\omega} \right) \\ k^\omega_{\symup{zR}} k^\omega_{\symup{zI}} &= N^\omega K^\omega \hat{\symbf k}_{\mathfrak{r}}^\omega \cdot \hat{\symbf k}_{\mathfrak{i}}^\omega \end{aligned}\right. ~,
\end{equation}
将入射侧介质 $\mathsf{i}$ 无吸收 Eq(\ref{eq:2-103}) 条件下,所推出的 Eq(\ref{eq:2-112}) 虚部实向量沿实验室坐标系 $\mathcal{Z}$ 系 $\symup{z}$ 向 $\hat{\symbf k}_{\mathfrak{i}}^{\mathsf{t}\omega} \equiv \symbfup{e}_{\symup{z}}$,以及 Eq(\ref{eq:2-113a}) 和 透射侧弱吸收 Eq(\ref{eq:2-117}) 条件下的 Eq.(\ref{eq:2-120}) 中的 $\theta^{\omega}_{\mathsf{t}} = \vartheta^{\omega}_{\mathsf{t}}$ 代入 Eq(\ref{eq:2-136}),得
\begin{equation} \label{eq:2-137}
	\left\{\ \begin{aligned} k^{\mathsf{t}2}_{\symup{\rho}} + \left( k^{\mathsf{t}\omega2}_{\symup{zR}} - k^{\mathsf{t}\omega2}_{\symup{zI}} \right) &= k^{2}_{0\omega} \left( N^{\omega2}_{\mathsf{t}} - K^{\omega2}_{\mathsf{t}} \right) \\ k^{\mathsf{t}\omega}_{\symup{zR}} k^{\mathsf{t}\omega}_{\symup{zI}} &= k^{2}_{0\omega} N^{\omega}_{\mathsf{t}} K^{\omega}_{\mathsf{t}} \cos \theta^{\omega}_{\mathsf{t}} \end{aligned}\right. ~,
\end{equation}
很容易看出,该方程的 $\symbfup{e}_{\symup{z}}$ 向传播解为
\begin{subequations} \label{eq:2-138}
	\begin{numcases}{}
		k^{\mathsf{t}\omega}_{\symup{zR}} = k^{\omega}_{0} N^{\omega}_{\mathsf{t}} \cos \theta^{\omega}_{\mathsf{t}} = \sqrt{ k^{2}_{0\omega} N^{2}_{\mathsf{t}\omega} - k^{2}_{\mathsf{t}\symup{\rho}} } \label{eq:2-138a}~, \\ k^{\mathsf{t}\omega}_{\symup{zI}} = k^{\omega}_{0} K^{\omega}_{\mathsf{t}} \label{eq:2-138b}~,
	\end{numcases}
\end{subequations}

此外,可以有更普适地推导:将入射侧介质 $\mathsf{i}$ 无吸收 Eq(\ref{eq:2-103}) 条件下,所推出的 Eq(\ref{eq:2-112}) 虚部实向量沿实验室坐标系 $\mathcal{Z}$ 系 $\symup{z}$ 向 $\hat{\symbf k}_{\mathfrak{i}}^{\mathsf{t}\omega} \equiv \symbfup{e}_{\symup{z}}$ 代入 Eq(\ref{eq:2-87c}):
\begin{equation} \label{eq:2-139}
	{\symbf k}^\omega \left( \hat{\symbf k}_{\mathfrak{r}}^\omega \right) = k^\omega_{0} \left( N^\omega \hat{\symbf k}_{\mathfrak{r}}^\omega + \mathbb{i} \cdot K^\omega \symbfup{e}_{\symup{z}} \right) ~,
\end{equation}
由于虚部的实向量均为 $\symbfup{e}_{\symup{z}}$,假设此时的复波矢的第三个形式即 Eq(\ref{eq:2-139}),与第二个形式即 Eq(\ref{eq:2-86b}),三个分量、实虚部,都分别相等,即完全指代同一个复波矢:
\begin{equation} \label{eq:2-140}
	{\symbf k}^\omega \left( \hat{\symbf k}_{\mathfrak{r}}^\omega \right) = {\symbf k}^\omega \left( \symbf k_{\symup{\rho}} \right) ~,
\end{equation}
该假设满足 ${\symbf k}^\omega \cdot {\symbf k}^\omega$ 相同以共享本征向量的条件 \cref{eq:2-91c,eq:2-91d},并且有
\begin{subequations} \label{eq:2-141}
\abovedisplayskip=0pt
\belowdisplayskip=10pt
\begin{align}
	k^\omega_{0} N^\omega \hat{\symbf k}_{\mathfrak{r}}^\omega &= \symbf k_{\symup{\rho}} + \symbf k^\omega_{\symup{zR}} \label{eq:2-141a}~, \\ k^\omega_{0} K^\omega &= k^\omega_{\symup{zI}} \label{eq:2-141b}~,
\end{align}
\end{subequations}
其中,Eq(\ref{eq:2-141a}) 与透射侧介质 $\mathsf{t}$ 弱吸收 Eq(\ref{eq:2-117}) 条件所推出的 Eq(\ref{eq:2-119}) 等价,并且上述两个方程 Eq(\ref{eq:2-141}) 的模,满足关系 Eq(\ref{eq:2-138});说明至少在该条件,以及入射侧介质 $\mathsf{i}$ 无吸收 Eq(\ref{eq:2-103}) 条件下,可以有复波矢的第三形式与第二形式等价的假设;尽管没有该侧介质弱吸收 Eq(\ref{eq:2-117}) 条件,只需另一侧介质无吸收 Eq(\ref{eq:2-103}),或许也有上述 Eq(\ref{eq:2-141}) 成立,但我们不做过多的假设。

将 Eq(\ref{eq:2-138}) 代入 $k^\omega_{\symup{z}} = k^\omega_{\symup{zR}} + \mathbb{i} \cdot k^\omega_{\symup{zI}}$,并省略透射侧下标 $\mathsf{t}$,即得所在侧介质弱吸收 Eq(\ref{eq:2-117})、另一侧介质无吸收 Eq(\ref{eq:2-103}) 条件下的傅立叶光学复本征值
\begin{subequations} \label{eq:2-142}
\begin{align}
	k^\omega_{\symup{z}} \left( \symbf k_{\symup{\rho}} \right) &= \sqrt{ k^{2}_{0\omega} N^{2}_{\omega} \left( \symbf k_{\symup{\rho}} \right) - k^{2}_{\symup{\rho}} } + \mathbb{i} k^{\omega}_{0} K^{\omega} \left( \symbf k_{\symup{\rho}} \right) \label{eq:2-142a} \\ &\xrightarrow[]{\symup{NA}_{0}^{\omega} := k_{\symup{\rho}} \big/ k^{\omega}_{0}} k^{\omega}_{0} \left( \sqrt{ N^{2}_{\omega} - \symup{NA}_{0}^{\omega2} } + \mathbb{i} \cdot K^{\omega} \right) \label{eq:2-142b}~,
\end{align}
\end{subequations}
其中 $\symup{NA}_{0}^{\omega} = k_{\symup{\rho}} \big/ k^{\omega}_{0}$ 为真空/空气中的数值孔径;将上述 Eq(\ref{eq:2-142}) 代入复波矢的第二种形式,即 Mcleod 形式 Eq(\ref{eq:2-86}),得傅立叶光学复波矢
\begin{subequations} \label{eq:2-143}
\begin{align}
	{\symbf k}^\omega \left( \symbf k_{\symup{\rho}} \right) &= \symbf k_{\symup{\rho}} + k^\omega_{\symup{z}} \left( \symbf k_{\symup{\rho}} \right) \cdot \symbfup{e}_{\symup{z}} \label{eq:2-143a} \\ &= \symbf k_{\symup{\rho}} + \sqrt{ k^{2}_{0\omega} N^{2}_{\omega} \left( \symbf k_{\symup{\rho}} \right) - k^{2}_{\symup{\rho}} } \cdot \symbfup{e}_{\symup{z}} + \mathbb{i} k^{\omega}_{0} K^{\omega} \left( \symbf k_{\symup{\rho}} \right) \cdot \symbfup{e}_{\symup{z}} \label{eq:2-143b} \\ &\xrightarrow[]{\symbfup{NA}_{0}^{\omega} := \symbf k_{\symup{\rho}} \big/ k^{\omega}_{0}} k^{\omega}_{0} \left[ \symbfup{NA}_{0}^{\omega} + \left( \sqrt{ N^{2}_{\omega} - \symup{NA}_{0}^{\omega2} } + \mathbb{i} \cdot K^{\omega} \right) \symbfup{e}_{\symup{z}} \right] \label{eq:2-143c}~,
\end{align}
\end{subequations}
该复波矢是 $\mathcal{Z}$ 系下横向实波矢 $\symbf k_{\symup{\rho}}$ 的函数,因而满足傅立叶光学的所有要求;同时它又是 $\phi, \theta^{\omega}$ 的函数,所以与 Berry 的解析解 \cref{eq:2-63,eq:2-81} 兼容;此外,它还包含复波矢的最广义形式 Eq(\ref{eq:2-87}) 的实虚部 $N^{\omega}, K^{\omega}$,并且符合广义斯涅尔定律 \cref{eq:2-111,eq:2-125}。

因此,该复波矢集成了 \cref{eq:2-85,eq:2-86,eq:2-87} 三种形式,并因满足 Eq(\ref{eq:2-91}) 而直接对应 Berry 版本征向量 Eq(\ref{eq:2-81})、本征值 Eq(\ref{eq:2-63});再加上广义斯涅尔定律 \cref{eq:2-111,eq:2-125},以至于折射前后,所有自变量 $\symbf k_{\symup{\rho}}$ 相同的场量(因变量)均属于同一个空间频率的单色平面电磁波(的各项属性)。

以上广义斯涅尔定律是结合了各向同性(无需各向异性\cite{wangComplexRayTracing2008a,wangComplexRayTracing2008})吸收介质中的复射线追踪\cite{changRayTracingAbsorbing2005},以及 Berry 单色均匀平面波本征解\cite{berryOpticalSingularitiesBirefringent2003}、Mcleod 矢量各向异性傅立叶光学\cite{mcleodVectorFourierOptics2014}的结果。

%\subsection{纯电各向异性与无吸收介质间的电场切向连续条件}
\subsection{\protect\hyperlink{chap:\thesubsection}{纯电各向异性与无吸收介质间的电场切向连续条件}}
\addtocontents{toc}{\protect\linkdest{chap:\thesubsection}}
\label{纯电各向异性与无吸收介质间的电场切向连续条件}

我们在 \ref{纯电各向异性介质中傅立叶线性光学解析解} 节初提到:一方面,由于实空间磁场 $\symbf H^{\omega}_z$ 切向连续边值关系在涉及旋光材料时失效,且原则上不应使用任何实空间的边值关系\cite{chenWavevectorspaceMethodWave1993,nelsonDerivingTransmissionReflection1995};另一方面,我们只关注材料界面高透射率(但材料体块/内部不一定透明)情况下的透射场,并只计算 Berry 的两个正向传播的偏振态,以服务于传输矩阵和转移矩阵的同时降低计算量\myHyperFootnote{尽管可以,但不额外计算两个反向传播的偏振态,以获取反射系数和更精确的透射系数。},因此只需且只能使用电场 $\symbf E^{\omega}_z$ 的切向连续性。

要想实空间电场 $\symbf E^{\omega}_z \left( \symbf{\rho} \right)$ 即 Eq(\ref{eq:2-18a}) 的切向连续,一个充分条件是 $\symbf E^{\omega}_z \left( \symbfup{\rho} \right)$ 的任何一个空间频率分量(时空谱) $\symbf G^{\omega}_z \left( \symbf k_{\symup{\rho}} \right)$ 即 Eq(\ref{eq:2-18b}) 的切向连续,也即要求共享横向实波矢 $\symbf k_{\symup{\rho}}$ 的电场时空谱的复振幅 $\symbf g^{\omega} \left( \symbf k_{\symup{\rho}} \right) = \symbf G^{\omega}_z \left( \symbf k_{\symup{\rho}} \right) \big/ \mathbb{e}^{\mathbb{i} k^\omega_{\symup{z}} \left( \symbf k_{\symup{\rho}} \right) z}$ 切向连续。

然而,根据 \ref{晶体/实验室坐标系下线性光学过程的解析解} 小节末的行波解 Eq(\ref{eq:2-84}),电场时空谱不能再写成单个行波 $\symbf G^{\omega}_z = \symbf g^{\omega} \cdot \mathbb{e}^{\mathbb{i} k^\omega_{\symup{z}} z}$ 的形式:因为对于 $\hat{\symbf k}$ 空间的任何一个方向 $\hat{\symbf k}$,都有两个单色均匀平面本征行波 $\bar{G}^{\omega\pm}_{\Yup \symbf r} \left( \hat{\symbf k} \right) = {\mathtt{g}}^{\omega}_{\pm} \hat{g}^{\omega\pm}_{\Yup} \cdot \mathbb{e}^{\mathbb{i} k^{\omega}_{0} n^{\omega}_{\pm} \hat{\symbf k} \cdot \symbf{r}}$ 满足 Eq.(\ref{eq:2-31});以致于类似地,对于任何一个实横向波矢 ${\symbf k}_{\symup{\rho}}$,都有两个待定系数 ${\mathtt{g}}^{\omega}_{\pm} \left( {\symbf k}_{\symup{\rho}} \right)$ 的单色非均匀平面本征行波
\begin{subequations} \label{eq:2-144}
	\abovedisplayskip=16pt
	\belowdisplayskip=12pt
	\begin{align}
		\bar{G}^{\omega\pm}_{\Yup \symbf r} \left( {\symbf k}_{\symup{\rho}} \right) &= \bar{G}^{\omega\pm}_{\Yup z} \cdot \mathbb{e}^{\mathbb{i} \symbf k_{\symup{\rho}} \cdot {\symbf \rho}} = \bar{g}^{\omega\pm}_{\Yup} \cdot \mathbb{e}^{\mathbb{i} {\symbf k}^\omega_{\pm} \left( \symbf k_{\symup{\rho}} \right) \cdot {\symbf r}} \xrightarrow[]{\text{Eq.(\ref{eq:2-83})}} {\mathtt{g}}^{\omega}_{\pm} \hat{g}^{\omega\pm}_{\Yup} \cdot \mathbb{e}^{\mathbb{i} {\symbf k}^\omega_{\pm} \left( \symbf k_{\symup{\rho}} \right) \cdot {\symbf r}} \label{eq:2-144a}\\ &\xrightarrow[]{\text{Eq.(\ref{eq:2-142a}) or Eq.(\ref{eq:2-143b})}} {\mathtt{g}}^{\omega}_{\pm} \hat{g}^{\omega\pm}_{\Yup} \cdot \mathbb{e}^{\mathbb{i} \left( \symbf k_{\symup{\rho}} \cdot {\symbf \rho} + \sqrt{ k^{2}_{0\omega} N^{2}_{\omega} - k^{2}_{\symup{\rho}} } z \right) - k^{\omega}_{0} K^{\omega} z} \label{eq:2-144b}~,
	\end{align}
\end{subequations}
满足 Eq.(\ref{eq:2-31})。其中,待定的两个系数 ${\mathtt{g}}^{\omega}_{\pm} \left( {\symbf k}_{\symup{\rho}} \right)$,需要且只可通过电场 $\symbf E^{\omega}_z$ 的切向连续边界条件求解。

因此,Eq.(\ref{eq:2-31}) 的电场时空谱傅立叶光学解析解
\begin{equation} \label{eq:2-145}
	\bar{G}^{\omega}_{z} = \sum_{\pm} \bar{G}^{\omega\pm}_{z} = \sum_{\pm} \bar{G}^{\omega\pm}_{\symbf r} \big/ \mathbb{e}^{\mathbb{i} \symbf k_{\symup{\rho}} \cdot {\symbf \rho}} = \overline{\bar{g}^{\omega}_{\pm}}^{\symup{T}} \cdot \overline{\mathbb{e}^{ \mathbb{i} k^{\omega\pm}_{\symup{z}} z}} ~,
\end{equation}
当其自变量在实横向空间频率 $\symbf k_{\symup{\rho}}$ 域内,且因变量在笛卡尔坐标系下时,变为:
\begin{subequations} \label{eq:2-146}
\begin{align}
	\bar{G}^{\omega}_{\Yup z} \left( {\symbf k}_{\symup{\rho}} \right) &= \overline{\bar{g}^{\omega\pm}_{\Yup}}^{\symup{T}} \cdot \overline{\mathbb{e}^{ \mathbb{i} k^{\omega\pm}_{\symup{z}} z}} \hspace{-1.0em}&&= \overline{\bar{g}^{\omega\pm}_{\Yup}}^{\symup{T}} \cdot \overline{\mathbb{e}^{ - k^{\omega}_{0} K^{\omega}_{\pm} z + \mathbb{i} \sqrt{ k^{2}_{0\omega} N^{2}_{\omega\pm} - k^{2}_{\symup{\rho}} } z}} \label{eq:2-146a}\\&= \sum_{\pm} {\mathtt{g}}^{\omega}_{\pm} \hat{g}^{\omega\pm}_{\Yup} \cdot \mathbb{e}^{ \mathbb{i} k^{\omega\pm}_{\symup{z}} z} \hspace{-1.0em}&&= \sum_{\pm} {\mathtt{g}}^{\omega}_{\pm} \mathbb{e}^{ - k^{\omega}_{0} K^{\omega}_{\pm} z} \cdot \hat{g}^{\omega\pm}_{\Yup} \mathbb{e}^{\mathbb{i} \sqrt{ k^{2}_{0\omega} N^{2}_{\omega\pm} - k^{2}_{\symup{\rho}} } z} \label{eq:2-146b}~,
\end{align}
\end{subequations}
出于前述满足 Eq(\ref{eq:2-91}) 条件的缘故,该非均匀时空谱解 $\bar{G}^{\omega}_{\Yup z} \left( {\symbf k}_{\symup{\rho}} \right)$ 与 $\mathcal Z$ 系下 Berry 均匀平面行波解 Eq(\ref{eq:2-84}) 即 $\bar{G}^{\omega}_{\Yup \symbf r} \left( \hat{\symbf k} \right)$ 共享 Berry 本征值 $n^{\omega}_{\pm}$、归一化本征向量 $\hat{g}^{\omega\pm}_{\Yup}$ 的表达式(计算方法),但不共享它们的值(计算结果),这也是 Mcleod 等人所提到的球坐标系下的本征方程,与直角坐标系下的 Booker 四次方程,从本征值到本征向量都不等价\cite{mcleodVectorFourierOptics2014}的原因,以及这两种特征方程的本征解的联系所在。

尽管看上去 Eq(\ref{eq:2-146}) 直接使用了 Eq(\ref{eq:2-83}) 的复振幅/偏振态/本征向量 $\bar{g}^{\omega\pm}_{\Yup} = {\mathtt{g}}^{\omega}_{\pm} \hat{g}^{\omega\pm}_{\Yup}$,此处 Eq(\ref{eq:2-146}) 的本征向量 $\hat{g}^{\omega\pm}_{\Yup} \left( {\symbf k}_{\symup{\rho}} \right)$ 的系数 ${\mathtt{g}}^{\omega}_{\pm} \left( {\symbf k}_{\symup{\rho}} \right)$ 却与 \cref{eq:2-83,eq:2-84} 中的 ${\mathtt{g}}^{\omega}_{\pm} \left( \hat{\symbf k} \right)$ 没有关联,导致时空谱 Eq(\ref{eq:2-146}) 的复振幅 ${\mathtt{g}}^{\omega}_{\pm} \left( {\symbf k}_{\symup{\rho}} \right) \hat{g}^{\omega\pm}_{\Yup} \left( {\symbf k}_{\symup{\rho}} \right)$ 也无关/独立于 \cref{eq:2-83,eq:2-84} 中的 ${\mathtt{g}}^{\omega}_{\pm} \left( \hat{\symbf k} \right) \hat{g}^{\omega\pm}_{\Yup} \left( \hat{\symbf k} \right)$,因为那里并没有边界条件约束 Eq(\ref{eq:2-83}),以至于 \cref{eq:2-83,eq:2-84} 的 ${\mathtt{g}}^{\omega}_{\pm} \left( \hat{\symbf k} \right)$ 和相应 ${\mathtt{g}}^{\omega}_{\pm} \left( \hat{\symbf k} \right) \hat{g}^{\omega\pm}_{\Yup} \left( \hat{\symbf k} \right)$ 的取值任意;而这里 Eq(\ref{eq:2-146}) 的 ${\mathtt{g}}^{\omega}_{\pm} \left( {\symbf k}_{\symup{\rho}} \right)$ 与对应的 ${\mathtt{g}}^{\omega}_{\pm} \left( {\symbf k}_{\symup{\rho}} \right) \hat{g}^{\omega\pm}_{\Yup} \left( {\symbf k}_{\symup{\rho}} \right)$ 却受到电场 $\symbf E^{\omega}_z$ 切向连续边界条件的制约。

因此 Eq(\ref{eq:2-146}) 与 Eq(\ref{eq:2-84}) 从本征值 $n^{\omega}_{\pm}$,到本征向量 $\bar{g}^{\omega\pm}_{\Yup}$,再到归一化本征向量 $\hat{g}^{\omega\pm}_{\Yup}$ 的系数 ${\mathtt{g}}^{\omega}_{\pm}$,以及这些元素所构成的通解 $\bar{G}^{\omega}_{\Yup z}$ 或 $\bar{G}^{\omega}_{\Yup \symbf r}$,尽管它们可以在符号上相同,但却完全不等价,无论是这些因变量还是自变量 $\left( \hat{\symbf k}\ \text{or}\ \symbf k_{\symup{\rho}} \right)$,包括其含义与其值。

另外,Eq(\ref{eq:2-146}) 相对于 Eq(\ref{eq:2-84}) 的优势,就在于尽管 Eq(\ref{eq:2-84}) 中的平面波解 $\bar{G}^{\omega}_{\Yup \symbf r} \left( \hat{\symbf k} \right)$ 在形式上也可以有 $\bar{G}^{\omega}_{\Yup z} \left( \hat{\symbf k} \right) = \bar{G}^{\omega}_{\Yup \symbf r} \left( \hat{\symbf k} \right) \big/ \mathbb{e}^{\mathbb{i} \symbf k_{\symup{\rho}} \cdot {\symbf \rho}}$,但其所产生的该 $\bar{G}^{\omega}_{\Yup z} \left( \hat{\symbf k} \right)$ 仍不是实横向空间频率 $\symbf k_{\symup{\rho}}$ 域的时空谱,因为它还与正空间的 $\symbf \rho$ 有关,而不单纯仅是实 $\symbf k_{\symup{\rho}}$ 的函数,或者其因是复 $\symbf k_{\symup{\rho}}$ 的函数而无法纳入傅立叶光学;而 Eq(\ref{eq:2-146}) 的 $\bar{G}^{\omega}_{\Yup z} \left( {\symbf k}_{\symup{\rho}} \right)$ 却满足傅立叶光学对空间频率/自变量、频谱函数/因变量的要求。

有了 Eq(\ref{eq:2-146}) 这个各向异性材料中的矢量傅立叶光学解析解后,才能开始处理边界条件问题。假设各向异性平板材料的两个边界端面分别在 $z=0,L$ 处,那么 Eq(\ref{eq:2-146}) 一般在 $z \in \left( 0, L \right)$ 内适用;尽管在更一般的材料中,即 $z \notin \left[ 0, L \right]$ 外 Eq(\ref{eq:2-146}) 也适用,但由于普通的实验条件下,晶体外多为真空/空气/油等各向同性媒质,此时 $z \notin \left[ 0, L \right]$ 外的 Eq(\ref{eq:2-146}) 退化为各向同性矢量亥姆霍兹方程的矢量角谱解,且方程组 Eq(\ref{eq:2-31}) 退化为同一个散度方程
\begin{equation} \label{eq:2-147}
	\abovedisplayskip=10pt
	\belowdisplayskip=10pt
	\symbf k^{\omega} \cdot \symbf g^{\omega} = 0 ~,
\end{equation}
它没有满足电位移场散度为零所对应的 Eq(\ref{eq:2-31}) 的偏振态解 Eq(\ref{eq:2-83}),只有满足电场散度为零即 Eq(\ref{eq:2-147}) 的三分量解\myHyperFootnote{这也是为什么我们要给晶体外加个三维偏振态解 Eq(\ref{eq:2-82}) 的原因,尽管多只在检偏时使用之。}
\begin{equation} \label{eq:2-148}
	\bar{g}^{\omega}_{\Yup} = \begin{pmatrix} g^{\omega}_{\symup{x}} \\ g^{\omega}_{\symup{y}} \\ g^{\omega}_{\symup{z}} \end{pmatrix} = \begin{pmatrix} g^{\omega}_{\symup{x}} \\ g^{\omega}_{\symup{y}} \\ - \symbf{k}_{\symup{\rho}} \cdot \symbf{g}^{\omega}_{\symup{\rho}} \big/ k^{\omega}_{\symup{z}} \end{pmatrix} := \begin{pmatrix} g^{\omega}_{\symup{x}} \\ g^{\omega}_{\symup{y}} \\ - \symbf{k}^{\symup{T}}_{\perp} \cdot \symbf{g}^{\omega}_{\perp} \big/ \sqrt{ k^{2}_{0\omega} n^{2}_{\omega} - k^{2}_{\symup{\rho}} } \end{pmatrix} ~,
\end{equation}
相应的行波解,从 Eq(\ref{eq:2-146}) 退化为
\begin{equation} \label{eq:2-149}
	\bar{G}^{\omega}_{\Yup z} \left( {\symbf k}_{\symup{\rho}} \right) = \bar{g}^{\omega}_{\Yup} \cdot \mathbb{e}^{ \mathbb{i} k^\omega_{\symup{z}} z} = \begin{pmatrix} g^{\omega}_{\symup{x}} \\ g^{\omega}_{\symup{y}} \\ - \bar{k}^{\symup{T}}_{\perp} \cdot \bar{g}^{\omega}_{\perp} \big/ \sqrt{ k^{2}_{0\omega} n^{2}_{\omega} - k^{2}_{\symup{\rho}} } \end{pmatrix} \cdot \mathbb{e}^{ \mathbb{i} \sqrt{ k^{2}_{0\omega} n^{2}_{\omega} - k^{2}_{\symup{\rho}} } z} ~,
\end{equation}
其中,$\bar{g}^{\omega\symup{T}}_{\perp} = \begin{pmatrix} g^{\omega}_{\symup{x}} & g^{\omega}_{\symup{y}} \end{pmatrix}$ 是任意的,等价于 Eq(\ref{eq:2-146}) 的复振幅 $\overline{\bar{g}^{\omega\pm}_{\Yup}}^{\symup{T}} = \begin{pmatrix} \bar{g}^{\omega+}_{\Yup} & \bar{g}^{\omega-}_{\Yup} \end{pmatrix}$,二者通过边界条件联系;前者若作为泵浦,则 $g^{\omega}_{\symup{x}}, g^{\omega}_{\symup{y}}$ 可任取 2 个标量场,但后者的 2 个系数 ${\mathtt{g}}^{\omega}_{+}, {\mathtt{g}}^{\omega}_{-}$ 一般须已知前者,再加上边界条件一起才能联合确定;并且,各向异性解 Eq(\ref{eq:2-146}) 与各向同性解 Eq(\ref{eq:2-149}) 的系数自由度相同,都是 2,因此都属于矢量光场,二者间通过 $2 \times 2$ 矩阵相联系;此外,其中定义了 $\symbf{k}_{\symup{\rho}} := \symbf{k}^{\symup{T}}_{\perp} = \bar{k}^{\symup{T}}_{\perp} \cdot {\symbfup{e}}_{\perp}$,同理 $\symbf{g}^{\omega}_{\symup{\rho}} := \symbf{g}^{\omega\symup{T}}_{\perp} = \bar{g}^{\omega\symup{T}}_{\perp} \cdot {\symbfup{e}}_{\perp}$,且 ${\symbfup{e}}^{\symup{T}}_{\perp} = \begin{pmatrix} {\symbfup{e}}_{\symup{x}} & {\symbfup{e}}_{\symup{y}} \end{pmatrix}$。

考虑各向异性长方体形材料入射端面 $z=0$ 处的电场 $\symbf E^{\omega}_z$ 切向连续边界条件的时机已成熟。设 $z<0$ 的入射场即泵浦时空谱 Eq(\ref{eq:2-149}) 的两个横向系数 $g^{\mathsf{i}\omega}_{\symup{x}}, g^{\mathsf{i}\omega}_{\symup{y}}$ 是已知的,那么泵浦在 $z<0$ 半无限大空间的入射介质 $\mathsf{i}$ 内的演化是已知的,并且在 $z \to 0^-$ 处的横向复振幅分布 $\bar{G}^{\mathsf{i}\omega}_{\perp 0^-} \to \bar{g}^{\mathsf{i}\omega}_{\perp}$ 是已知的,以至于 $z=0$ 处的电场 $\symbf E^{\omega}_z$ 切向连续边界条件可以由转移矩阵 $\overline{\hat{g}^{\mathsf{t}\omega\pm}_{\perp}}^{\symup{T}}$ 联系:
\begin{subequations} \label{eq:2-150}
	\abovedisplayskip=6pt
	\belowdisplayskip=10pt
	\begin{align}
		\bar{E}^{\mathsf{i}\omega}_{\perp 0^-} &= \bar{E}^{\mathsf{t}\omega}_{\perp 0^+} \label{eq:2-150a}\\ \bar{G}^{\mathsf{i}\omega}_{\perp 0^-} &= \bar{G}^{\mathsf{t}\omega}_{\perp 0^+} \label{eq:2-150b}\\ \begin{pmatrix} g^{\mathsf{i}\omega}_{\symup{x}} \\ g^{\mathsf{i}\omega}_{\symup{y}} \end{pmatrix} = \bar{g}^{\mathsf{i}\omega}_{\perp} &= \bar{g}^{\mathsf{t}\omega}_{\perp} = \sum_{\pm} {\mathtt{g}}^{\mathsf{t}\omega}_{\pm} \begin{pmatrix} \hat{g}^{\mathsf{t}\omega\pm}_{\symup{x}} \\ \hat{g}^{\mathsf{t}\omega\pm}_{\symup{y}} \end{pmatrix} = \begin{pmatrix} \hat{g}^{\mathsf{t}\omega+}_{\symup{x}} & \hat{g}^{\mathsf{t}\omega-}_{\symup{x}} \\ \hat{g}^{\mathsf{t}\omega+}_{\symup{y}} & \hat{g}^{\mathsf{t}\omega-}_{\symup{y}} \end{pmatrix} \begin{pmatrix} {\mathtt{g}}^{\mathsf{t}\omega}_{+} \\ {\mathtt{g}}^{\mathsf{t}\omega}_{-} \end{pmatrix} \label{eq:2-150c}~,
	\end{align}
\end{subequations}
其中,用到了复振幅 Eq(\ref{eq:2-83}) 相关的
\begin{subequations} \label{eq:2-151}
	\abovedisplayskip=10pt
	\belowdisplayskip=10pt
	\begin{align}
		\bar{g}^{\omega}_{\perp} &= \left[ \bar{g}^{\omega}_{\Yup} \right]_{\perp} = \left[ \sum_{\pm} {\mathtt{g}}^{\omega}_{\pm} \hat{g}^{\omega\pm}_{\Yup} \right]_{\perp} = \sum_{\pm} {\mathtt{g}}^{\omega}_{\pm} \left[ \hat{g}^{\omega\pm}_{\Yup} \right]_{\perp} \label{eq:2-151a}\\ &=: \sum_{\pm} {\mathtt{g}}^{\omega}_{\pm} \hat{g}^{\omega\pm}_{\perp} = \sum_{\pm} {\mathtt{g}}^{\omega}_{\pm} \begin{pmatrix} \hat{g}^{\omega\pm}_{\symup{x}} \\ \hat{g}^{\omega\pm}_{\symup{y}} \end{pmatrix} \label{eq:2-151b}\\ &= \overline{{\mathtt{g}}^{\omega}_{\pm}}^{\symup{T}} \cdot \overline{\hat{g}^{\omega\pm}_{\perp}} = \begin{pmatrix} {\mathtt{g}}^{\omega}_{+} & {\mathtt{g}}^{\omega}_{-} \end{pmatrix} \begin{pmatrix} \hat{g}^{\omega+}_{\perp} \\ \hat{g}^{\omega-}_{\perp} \end{pmatrix} \label{eq:2-151c}\\ &= \overline{\hat{g}^{\omega\pm}_{\perp}}^{\symup{T}} \cdot \overline{{\mathtt{g}}^{\omega}_{\pm}} = \begin{pmatrix} \hat{g}^{\omega+}_{\perp} & \hat{g}^{\omega-}_{\perp} \end{pmatrix} \begin{pmatrix} {\mathtt{g}}^{\omega}_{+} \\ {\mathtt{g}}^{\omega}_{-} \end{pmatrix} = \begin{pmatrix} \hat{g}^{\omega+}_{\symup{x}} & \hat{g}^{\omega-}_{\symup{x}} \\ \hat{g}^{\omega+}_{\symup{y}} & \hat{g}^{\omega-}_{\symup{y}} \end{pmatrix} \begin{pmatrix} {\mathtt{g}}^{\omega}_{+} \\ {\mathtt{g}}^{\omega}_{-} \end{pmatrix} \label{eq:2-151d}~,
	\end{align}
\end{subequations}
注意,定义了 $\hat{g}^{\omega\pm}_{\perp} := \text{trans} \left[ \hat{g}^{\omega\pm}_{\Yup} \right] := \left[ \hat{g}^{\omega\pm}_{\Yup} \right]_{\perp}$ 这种容易引起岐义的表示:它既可以表示归一化二维偏振态 $\mathcal{N} \cdot \left[ \hat{g}^{\omega\pm}_{\Yup} \right]_{\perp}$,也可以表示归一化三维偏振态的 $\perp$ 分量 $\left[ \hat{g}^{\omega\pm}_{\Yup} \right]_{\perp}$,后者作为二维偏振态,并未归一化;上述 $\hat{g}^{\omega\pm}_{\symup{x}},\hat{g}^{\omega\pm}_{\symup{y}}$ 是后者 $\left[ \hat{g}^{\omega\pm}_{\Yup} \right]_{\perp}$ 而非前者的 $\symup{x,y}$ 分量,即我们采用的是第二种定义。

求解二元一次方程组 Eq(\ref{eq:2-150b}) 即可得透射侧介质 $\mathsf{t}$ 内的归一化三维本征向量 $\hat{g}^{\mathsf{t}\omega\pm}_{\Yup}$ 的基系数
\begin{subequations} \label{eq:2-152}
	\abovedisplayskip=8pt
	\belowdisplayskip=10pt
	\begin{align}
		\begin{pmatrix} {\mathtt{g}}^{\mathsf{t}\omega}_{+} \\ 	{\mathtt{g}}^{\mathsf{t}\omega}_{-} \end{pmatrix} &= \begin{pmatrix} \hat{g}^{\mathsf{t}\omega+}_{\symup{x}} & \hat{g}^{\mathsf{t}\omega-}_{\symup{x}} \\ \hat{g}^{\mathsf{t}\omega+}_{\symup{y}} & \hat{g}^{\mathsf{t}\omega-}_{\symup{y}} \end{pmatrix}^{-1} \begin{pmatrix} g^{\mathsf{i}\omega}_{\symup{x}} \\ g^{\mathsf{i}\omega}_{\symup{y}} \end{pmatrix} \label{eq:2-152a}\\ &= \frac{ \begin{pmatrix} \hat{g}^{\mathsf{t}\omega-}_{\symup{y}} & -\hat{g}^{\mathsf{t}\omega-}_{\symup{x}} \\ -\hat{g}^{\mathsf{t}\omega+}_{\symup{y}} & \hat{g}^{\mathsf{t}\omega+}_{\symup{x}} \end{pmatrix} \begin{pmatrix} g^{\mathsf{i}\omega}_{\symup{x}} \\ g^{\mathsf{i}\omega}_{\symup{y}} \end{pmatrix} }{\hat{g}^{\mathsf{t}\omega+}_{\symup{x}} \hat{g}^{\mathsf{t}\omega-}_{\symup{y}} - \hat{g}^{\mathsf{t}\omega-}_{\symup{x}} \hat{g}^{\mathsf{t}\omega+}_{\symup{y}}} \label{eq:2-152b}~,
	\end{align}
\end{subequations}
以后就可以用(通过求解晶体本征向量而)所有元素均已知的转移矩阵 $\overline{\hat{g}^{\mathsf{t}\omega\pm}_{\perp}}^{\symup{T}}$ 及其逆 $\overline{\hat{g}^{\mathsf{t}\omega\pm}_{\perp}}^{-\symup{T}}$,轻易实现复正交的 $\symup{HV}$ 基的系数 $\bar{g}^{\mathsf{i}\omega}_{\perp}$,与不一定复正交(晶体有吸收时)的 2 个晶体本征向量的系数 $\overline{{\mathtt{g}}^{\mathsf{t}\omega}_{\pm}}$,即 $\bar{g}^{\mathsf{i}\omega}_{\perp}, \overline{{\mathtt{g}}^{\mathsf{t}\omega}_{\pm}}$ 之间的相互转换。

这种因“电场 $\symbf E^{\omega}_z$ 切向连续性”所导致的基系数间的转换,甚至不一定是 $\mathsf{i}, \mathsf{t}$ 介质间的:可以进一步拓展到晶体内部,即同一个 $\mathsf{t}$ 介质内,每个 $k_{\symup{x}}, k_{\symup{y}}$ 切片/断面/横截面的 $\bar{g}^{\mathsf{t}\omega}_{\perp}, \overline{{\mathtt{g}}^{\mathsf{t}\omega}_{\pm}}$ 转换,用于统一输出 $\symup{HV}$ 基系数,以便绘制矢量光场强度或检偏后的强度或相位分布图;这是因为“电场 $\symbf E^{\omega}_z$ 切向连续性”不仅在端面边界适用,在晶体内部、外部的任何一个 $z=z_0$ 面也适用。

因此更广义地,在任何同一个介质中,有类似 Eq(\ref{eq:2-150b}) 或 Eq(\ref{eq:2-151d}) 的 $\bar{g}^{\omega}_{\perp} = \overline{\hat{g}^{\omega\pm}_{\perp}}^{\symup{T}} \cdot \overline{{\mathtt{g}}^{\omega}_{\pm}}$,以及 Eq(\ref{eq:2-152a}) 的 $\overline{{\mathtt{g}}^{\omega}_{\pm}} = \overline{\hat{g}^{\omega\pm}_{\perp}}^{-\symup{T}} \cdot \bar{g}^{\omega}_{\perp}$,以用转移矩阵 $\overline{\hat{g}^{\omega\pm}_{\perp}}^{\symup{T}}$ 及其逆 $\overline{\hat{g}^{\omega\pm}_{\perp}}^{-\symup{T}}$,描述同一个矢量光场的两对基系数间的转换,我们将其总结在这里:
\begin{subequations} \label{eq:2-153}
	\abovedisplayskip=10pt
	\belowdisplayskip=10pt
	\begin{align}
		\bar{g}^{\omega}_{\perp} &= \overline{\hat{g}^{\omega\pm}_{\perp}}^{\symup{T}} \cdot \overline{{\mathtt{g}}^{\omega}_{\pm}} \label{eq:2-153a}~,\\ \overline{{\mathtt{g}}^{\omega}_{\pm}} &= \overline{\hat{g}^{\omega\pm}_{\perp}}^{-\symup{T}} \cdot \bar{g}^{\omega}_{\perp} \label{eq:2-153b}~.
	\end{align}
\end{subequations}

由于电场 $\symbf E^{\omega}_z$ 切向连续的要求,电场初始复振幅不仅在边界上连续,在晶体内也保持 $\bar{g}^{\mathsf{t}\omega\pm}_{\Yup} = {\mathtt{g}}^{\mathsf{t}\omega}_{\pm} \hat{g}^{\mathsf{t}\omega\pm}_{\Yup}$ 不变,再加上晶体归一化本征向量 $\hat{g}^{\mathsf{t}\omega\pm}_{\Yup}$ 与场无关而由晶体决定,导致晶体本征向量的基系数 $\overline{{\mathtt{g}}^{\mathsf{t}\omega}_{\pm}} = \overline{\hat{g}^{\mathsf{t}\omega\pm}_{\perp}}^{-\symup{T}} \cdot \bar{g}^{\mathsf{i}\omega}_{\perp}$ 在同一个晶体内保持不变,因此我们将 Eq(\ref{eq:2-146a}) 的 $\perp$ 部分,对 $\overline{{\mathtt{g}}^{\omega}_{\pm}}$ 进行进一步的拆写
\begin{subequations} \label{eq:2-154}
	\begin{align}
		\bar{G}^{\omega}_{\perp z} = \sum_{\pm} {\mathtt{g}}^{\omega}_{\pm} \hat{g}^{\omega\pm}_{\perp} \mathbb{e}^{ \mathbb{i} k^{\omega\pm}_{\symup{z}} z} &= \overline{\hat{g}^{\omega\pm}_{\perp} \mathbb{e}^{ \mathbb{i} k^{\omega\pm}_{\symup{z}} z}}^{\symup{T}} \cdot \overline{{\mathtt{g}}^{\omega}_{\pm}} \label{eq:2-154a}\\&= \overline{\hat{g}^{\omega\pm}_{\perp}}^{\symup{T}} \cdot \overline{\overline{\mathbb{e}^{ \mathbb{i} k^{\omega\pm}_{\symup{z}} z}}} \cdot \overline{{\mathtt{g}}^{\omega}_{\pm}} \label{eq:2-154b}~,
	\end{align}
\end{subequations}
其中,定义了传播矩阵,为一对角矩阵
\begin{equation} \label{eq:2-155}
	\overline{\overline{\mathbb{e}^{ \mathbb{i} k^{\omega\pm}_{\symup{z}} z}}} := \text{diag} \left(\mathbb{e}^{ \mathbb{i} k^{\omega+}_{\symup{z}} z}, \mathbb{e}^{ \mathbb{i} k^{\omega-}_{\symup{z}} z}  \right) = \begin{pmatrix} \mathbb{e}^{ \mathbb{i} k^{\omega+}_{\symup{z}} z} & 0 \\ 0 & \mathbb{e}^{ \mathbb{i} k^{\omega-}_{\symup{z}} z} \end{pmatrix}~.
\end{equation}

接着,将由泵浦 $\bar{g}^{\mathsf{i}\omega}_{\perp}$、晶体 $\overline{\hat{g}^{\mathsf{t}\omega\pm}_{\perp}}^{-\symup{T}}$ 和电场 $\symbf E^{\omega}_z$ 切向连续性 Eq(\ref{eq:2-153b}) 决定的 $\overline{{\mathtt{g}}^{\mathsf{t}\omega}_{\pm}} = \overline{\hat{g}^{\mathsf{t}\omega\pm}_{\perp}}^{-\symup{T}} \cdot \bar{g}^{\mathsf{i}\omega}_{\perp}$ 代入晶体 $\mathsf{t}$ 内的 Eq(\ref{eq:2-154b}),即可得晶体内 $z \in \left( 0, L \right)$ 的电场时空谱的 $\perp$ 部分:
\begin{subequations} \label{eq:2-156}
	\begin{align}
		\bar{G}^{\mathsf{t}\omega}_{\perp z} &:= \overline{\hat{g}^{\mathsf{t}\omega\pm}_{\perp}}^{\symup{T}} \cdot \overline{{\mathtt{G}}^{\mathsf{t}\omega\pm}_{z}} \label{eq:2-156a}\\&:= \overline{\hat{g}^{\mathsf{t}\omega\pm}_{\perp}}^{\symup{T}} \cdot \overline{\overline{\mathbb{e}^{ \mathbb{i} k^{\mathsf{t}\omega\pm}_{\symup{z}} z}}} \cdot \overline{{\mathtt{g}}^{\mathsf{t}\omega}_{\pm}} \label{eq:2-156b}\\&= \overline{\hat{g}^{\mathsf{t}\omega\pm}_{\perp}}^{\symup{T}} \cdot \overline{\overline{\mathbb{e}^{ \mathbb{i} k^{\mathsf{t}\omega\pm}_{\symup{z}} z}}} \cdot \overline{\hat{g}^{\mathsf{t}\omega\pm}_{\perp}}^{-\symup{T}} \cdot \bar{g}^{\mathsf{i}\omega}_{\perp} \label{eq:2-156c}\\&=: \overline{\hat{g}^{\mathsf{t}\omega\pm}_{\perp}}^{\symup{T}} \cdot \bar{\bar{\mathcal{M}}}_z \left( k^{\mathsf{t}\omega\pm}_{\symup{z}}, \hat{g}^{\mathsf{t}\omega\pm}_{\perp} \right) \cdot \bar{G}^{\mathsf{i}\omega}_{\perp 0^-} \label{eq:2-156d}~,
	\end{align}
\end{subequations}
其中,仿照 Eq(\ref{eq:2-151d}) 的 $\bar{g}^{\omega}_{\perp} = \overline{\hat{g}^{\omega\pm}_{\perp}}^{\symup{T}} \cdot \overline{{\mathtt{g}}^{\omega}_{\pm}}$,定义了同一介质内的
\begin{subequations} \label{eq:2-157}
	\begin{align}
		\bar{G}^{\omega}_{\perp z} &:= \overline{\hat{g}^{\omega\pm}_{\perp}}^{\symup{T}} \cdot \overline{{\mathtt{G}}^{\omega\pm}_{z}} \label{eq:2-157a}~,\\ \overline{{\mathtt{G}}^{\omega\pm}_{z}} &:= \overline{\overline{\mathbb{e}^{ \mathbb{i} k^{\omega\pm}_{\symup{z}} z}}} \cdot \overline{{\mathtt{g}}^{\omega}_{\pm}} \label{eq:2-157b}~;\\ \bar{\bar{\mathcal{M}}}_z &:= \overline{\overline{\mathbb{e}^{ \mathbb{i} k^{\omega\pm}_{\symup{z}} z}}} \cdot \overline{\hat{g}^{\omega\pm}_{\perp}}^{-\symup{T}} \label{eq:2-157c}~,
	\end{align}
\end{subequations}
其中,\cref{eq:2-151d,eq:2-157a} 在三分量 $\Yup$ 版本下仍成立;Eq(\ref{eq:2-157c}) 中的 $\bar{\bar{\mathcal{M}}}_z = \bar{\bar{\mathcal{M}}}_z \left( k^{\omega\pm}_{\symup{z}}, \hat{g}^{\omega\pm}_{\perp} \right)$ 作为传播矩阵 $\overline{\overline{\mathbb{e}^{ \mathbb{i} k^{\omega\pm}_{\symup{z}} z}}}$ 与转移矩阵(的逆) $\overline{\hat{g}^{\omega\pm}_{\perp}}^{-\symup{T}}$ 的乘积,称为传输矩阵;此外,利用了 Eq(\ref{eq:2-150}) 将 $\bar{g}^{\mathsf{i}\omega}_{\perp}$ 写作了 $\bar{G}^{\mathsf{i}\omega}_{\perp 0^-}$。

Eq(\ref{eq:2-156a}) 的得来过程也可这样描述:像 \cref{eq:2-152a,eq:2-153b,eq:2-156b} 调用储存在内存中的转移矩阵 $\overline{\hat{g}^{\mathsf{t}\omega\pm}_{\perp}}^{\symup{T}}$ 的逆矩阵 $\overline{\hat{g}^{\mathsf{t}\omega\pm}_{\perp}}^{-\symup{T}}$ 一样,再次调用 \cref{eq:2-150b,eq:2-151d,eq:2-153a} 中所使用到的转移矩阵 $\overline{\hat{g}^{\mathsf{t}\omega\pm}_{\perp}}^{\symup{T}}$,将其作用于 Eq(\ref{eq:2-157b}),可以得到类似 Eq(\ref{eq:2-153a}) 的 Eq(\ref{eq:2-156a})。最终所得的 Eq(\ref{eq:2-156}),即为晶体内电场时空谱在 $\symup{HV}$ 基下的系数 $g^{\mathsf{t}\omega}_{\symup{x}z}, g^{\mathsf{t}\omega}_{\symup{y}z}$,也就是晶体内电场时空谱的 $\perp$ 分量随 $z$ 的演化表达式/算法。

更一般地,在同一个均匀材料中,任意两个纵向位置 $z, z_0$ 处的横截面上的场 $\bar{g}^{\omega}_{\Yup}$ 的切向 $\bar{g}^{\omega}_{\perp}$ 即 Eq(\ref{eq:2-156d}),可以写成如下形式
\begin{subequations} \label{eq:2-158}
	\abovedisplayskip=10pt
	\belowdisplayskip=10pt
	\begin{align}
		\bar{G}^{\omega}_{\perp z} &= \mathcal{G} \cdot \bar{G}^{\omega}_{\perp z_0} \label{eq:2-158a}~,\\ \mathcal{G} \left( k^{\omega\pm}_{\symup{z}}, \hat{g}^{\omega\pm}_{\perp} \right) &:= \overline{\hat{g}^{\omega\pm}_{\perp}}^{\symup{T}} \cdot \overline{\overline{\mathbb{e}^{ \mathbb{i} k^{\omega\pm}_{\symup{z}} \left( z - z_0 \right)}}} \cdot \overline{\hat{g}^{\omega\pm}_{\perp}}^{-\symup{T}} \label{eq:2-158b}\\ &=: \overline{\hat{g}^{\omega\pm}_{\perp}}^{\symup{T}} \cdot \bar{\bar{\mathcal{M}}}_z \left( k^{\omega\pm}_{\symup{z}}, \hat{g}^{\omega\pm}_{\perp} \right) \label{eq:2-158c}~,
	\end{align}
\end{subequations}
其中,定义了任意 $\bar{\bar{\symbf{\varepsilon}}}_{\symup{r}}^{\omega}$ 材料的传递函数(矩阵)$\mathcal{G} = \mathcal{G} \left( k^{\omega\pm}_{\symup{z}}, \hat{g}^{\omega\pm}_{\perp} \right)$,它由材料的 2 个本征值 $k^{\omega\pm}_{\symup{z}}$ 和 2 个归一化本征向量的 $\symup{x,y}$ 分量 $\hat{g}^{\omega\pm}_{\perp}$ 共同决定。但是,$\mathcal{G}$ 中的 2 个本征向量 $\bar{g}^{\omega\pm}_{\Yup}$ 甚至不需要归一化:因为复振幅由两部分构成,当本征向量/基较大时,基系数自动就变小了,以保证电场 $\symbf E^{\omega}_z$ 切向连续 Eq(\ref{eq:2-150}),以至于传递函数(矩阵)$\mathcal{G}$ 可以是非归一化本征向量的函数:
\begin{subequations} \label{eq:2-159}
	\abovedisplayskip=10pt
	\belowdisplayskip=10pt
	\begin{align}
		\mathcal{G} \left( k^{\omega\pm}_{\symup{z}}, \bar{g}^{\omega\pm}_{\perp} \right) &= 	\overline{\bar{g}^{\omega\pm}_{\perp}}^{\symup{T}} \cdot \overline{\overline{\mathbb{e}^{ \mathbb{i} k^{\omega\pm}_{\symup{z}} \left( z - z_0 \right)}}} \cdot \overline{\bar{g}^{\omega\pm}_{\perp}}^{-\symup{T}} \label{eq:2-159a}\\ &=: \overline{\hat{g}^{\omega\pm}_{\perp}}^{\symup{T}} \cdot \bar{\bar{\mathcal{M}}}_z \left( k^{\omega\pm}_{\symup{z}}, \bar{g}^{\omega\pm}_{\perp} \right) \label{eq:2-159b}~,
	\end{align}
\end{subequations}
另外,基于该原因和晶体可能有吸收等的缘故,传递函数矩阵 $\mathcal{G}$ 不一定也不需要是酉矩阵。它的非零副对角项的大小,衡量了 $\bar{G}^{\omega}_{\symup{x} z}, \bar{G}^{\omega}_{\symup{y} z_0}$ 和 $\bar{G}^{\omega}_{\symup{y} z}, \bar{G}^{\omega}_{\symup{x} z_0}$ 各自内部间的转换功率,即表征了各向异性材料与光场相互作用,所导致的光场自身的旋轨耦合效应。

横向分量 $\bar{G}^{\omega}_{\perp z}$ 的计算,只需用到 Eq(\ref{eq:2-156}) 所导出的 Eq(\ref{eq:2-158});纵向分量 $G^{\omega}_{\symup{z}z}$,可根据 Eq(\ref{eq:2-156}) 的 $\symup{z}$ 分量版本获得:
\begin{subequations} \label{eq:2-160}
	\abovedisplayskip=10pt
	\belowdisplayskip=10pt
	\begin{align}
		G^{\omega}_{\symup{z}z} &= \overline{\hat{g}^{\omega\pm}_{\symup{z}}}^{\symup{T}} \cdot \overline{{\mathtt{G}}^{\omega\pm}_{z}} = \sum_{\pm} \hat{g}^{\omega\pm}_{\symup{z}} {\mathtt{G}}^{\omega\pm}_{z} \label{eq:2-160a}\\ &= \overline{\hat{g}^{\omega\pm}_{\symup{z}}}^{\symup{T}} \cdot \overline{\overline{\mathbb{e}^{ \mathbb{i} k^{\omega\pm}_{\symup{z}} \left( z - z_0 \right)}}} \cdot \overline{{\mathtt{G}}^{\omega\pm}_{z_0}} = \sum_{\pm} \hat{g}^{\omega\pm}_{\symup{z}} \mathbb{e}^{ \mathbb{i} k^{\omega\pm}_{\symup{z}} \left( z - z_0 \right)} {\mathtt{G}}^{\omega\pm}_{z_0} \label{eq:2-160b}\\&= \overline{\hat{g}^{\omega\pm}_{\symup{z}}}^{\symup{T}} \cdot \overline{\overline{\mathbb{e}^{ \mathbb{i} k^{\omega\pm}_{\symup{z}} \left( z - z_0 \right)}}} \cdot \overline{\hat{g}^{\omega\pm}_{\perp}}^{-\symup{T}} \cdot \bar{G}^{\omega}_{\perp z_0} = \overline{\hat{g}^{\omega\pm}_{\symup{z}}}^{\symup{T}} \cdot \bar{\bar{\mathcal{M}}}_z \left( k^{\omega\pm}_{\symup{z}}, \hat{g}^{\omega\pm}_{\perp} \right) \cdot \bar{G}^{\omega}_{\perp z_0} \label{eq:2-160c}~,
	\end{align}
\end{subequations}
提取 Eq(\ref{eq:2-160}) 与 Eq(\ref{eq:2-158}) 的公有部分,即可合成为三分量版本:
\begin{subequations} \label{eq:2-161}
	\abovedisplayskip=10pt
	\belowdisplayskip=10pt
	\begin{align}
		\bar{G}^{\omega}_{\Yup z} &= \overline{\hat{g}^{\omega\pm}_{\Yup}}^{\symup{T}} \cdot \overline{{\mathtt{G}}^{\omega\pm}_{z}} \label{eq:2-161a}\\ &= \overline{\hat{g}^{\omega\pm}_{\Yup}}^{\symup{T}} \cdot \overline{\overline{\mathbb{e}^{ \mathbb{i} k^{\omega\pm}_{\symup{z}} \left( z - z_0 \right)}}} \cdot \overline{{\mathtt{G}}^{\omega\pm}_{z_0}} \label{eq:2-161b}\\&= \overline{\hat{g}^{\omega\pm}_{\Yup}}^{\symup{T}} \cdot \overline{\overline{\mathbb{e}^{ \mathbb{i} k^{\omega\pm}_{\symup{z}} \left( z - z_0 \right)}}} \cdot \overline{\hat{g}^{\omega\pm}_{\perp}}^{-\symup{T}} \cdot \bar{G}^{\omega}_{\perp z_0} \label{eq:2-161c}\\&= \overline{\hat{g}^{\omega\pm}_{\Yup}}^{\symup{T}} \cdot \bar{\bar{\mathcal{M}}}_z \left( k^{\omega\pm}_{\symup{z}}, \hat{g}^{\omega\pm}_{\perp} \right) \cdot \bar{G}^{\omega}_{\perp z_0} \label{eq:2-161d}~,
	\end{align}
\end{subequations}
其中,\cref{eq:2-158,eq:2-160,eq:2-161} 中的所有三维、二维归一化向量 $\hat{g}^{\omega\pm}_{\Yup}, \hat{g}^{\omega\pm}_{\perp}$,可同时选择任意相同的复归一化标准,或更直接地都不归一化也行 $\bar{g}^{\omega\pm}_{\Yup}, \bar{g}^{\omega\pm}_{\perp}$;并且可选择之前于 Eq(\ref{eq:2-82}) 定义的,用于晶体外起检偏的三维实基矢 $\hat{g}^{\omega \leftarrow\hspace{-0.525em}\uparrow}_{\Yup}, \hat{g}^{\omega \leftarrow\hspace{-0.525em}\uparrow}_{\perp}$,以查看非理想检偏器的偏振串扰效应\cite{zhangNonparaxialIdealizedPolarizer2018}。从输入 $\symup{xy}$ 两分量的 $\hat{g}^{\omega\pm}_{\perp}$、输出 $\symup{xyz}$ 三分量的 $\hat{g}^{\omega\pm}_{\Yup}$ 输入-输出关系来看,信息并没有减少,反而输出的信息比输入的信息还多,但这是因为由于散度方程的限制所导致的 Eq(\ref{eq:2-160}),即时空谱的 $\symup{z}$ 分量并不独立于 $\symup{xy}$ 分量的缘故。

至此,纯电各向异性介质内 Eq(\ref{eq:2-161}),以及其与无吸收各向同性介质间 Eq(\ref{eq:2-156}) 的电场 $\symbf E^{\omega}_z$ 切向连续边界条件,已经被完整地实施。后者也可以继续拓展到无吸收各向异性介质与纯电各向异性介质间的电场 $\symbf E^{\omega}_z$ 切向连续边界条件,只需要更改 Eq(\ref{eq:2-150}) 左侧即可;这不是件难事,只不过实际实验中用处较少/局限,条件较苛刻(比如多层电各向异性介质膜的情形),因此不再此体现。

%\subsection{各向异性矢量傅立叶线性光学的函数调用栈}
\subsection{\protect\hyperlink{chap:\thesubsection}{各向异性矢量傅立叶线性光学的函数调用栈}}
\addtocontents{toc}{\protect\linkdest{chap:\thesubsection}}
\label{各向异性矢量傅立叶线性光学的函数调用栈}

在第 \pageref{con:1} 页条件下的电各向异性材料中,电场时空谱 Eq(\ref{eq:2-18b}) 满足波动和散度方程组 Eq(\ref{eq:2-30});其解析解,即矢量傅立叶线性光学的单色电场空间频率域解,最广义地为 Eq(\ref{eq:2-161c}) 的基非归一化版本:
\begin{equation} \label{eq:2-162}
	\abovedisplayskip=10pt
	\belowdisplayskip=10pt
		\bar{G}^{\omega}_{\Yup z} = \overline{\bar{g}^{\omega\pm}_{\Yup}}^{\symup{T}} \cdot \overline{\overline{\mathbb{e}^{ \mathbb{i} k^{\omega\pm}_{\symup{z}} \left( z - z_0 \right)}}} \cdot \overline{\bar{g}^{\omega\pm}_{\perp}}^{-\symup{T}} \cdot \bar{G}^{\omega}_{\perp z_0}
\end{equation}
其中,输入的电场时空谱 $\bar{G}^{\omega}_{\Yup z}$ 的三个分量、输出的 $\bar{G}^{\omega}_{\perp z_0}$ 的两个分量标量场,作为二维横向空间频率/实波矢 $k_{\symup{x}},k_{\symup{y}}$ 的函数,分别与 $z_0$ 处输入平面 - $z$ 处输出平面的实/正/$\symbf{r}$ 空间的二维实坐标 $x,y$ 分布的电矢量场 $\bar{E}^{\omega}_{\Yup z_0}, \bar{E}^{\omega}_{\perp z}$,通过正逆 $\symbf \rho \leftrightarrow \symbf k_{\symup{\rho}}$ 二维空间傅立叶变换 \cref{eq:2-18b,eq:2-18a} 相联系。

为得到 Eq(\ref{eq:2-162}) 的返回值 $\bar{G}^{\omega}_{\Yup z}$,除传入已知的参数 $\bar{G}^{\omega}_{\perp z_0}$ 外,方程 Eq(\ref{eq:2-162}) 中只剩与晶体有关的 2 个场量需要确定:双本征值 $k^{\omega\pm}_{\symup{z}}$ 由 Eq(\ref{eq:2-142}) 确定,双本征向量 $\bar{g}^{\omega\pm}_{\Yup}$ 由 Eq(\ref{eq:2-81b}) 确定 —— 但要想确定双本征向量 $\bar{g}^{\omega\pm}_{\Yup}$,由于 Eq(\ref{eq:2-81b}) 中的球坐标的 $\mathcal C$ 系下的本征向量 $\bar{\underline{d}}^{\omega\pm}_{\symup{\theta\phi}}$ 即 \cref{eq:2-66,eq:2-68},是 $\mathcal C$ 系下的方位角、极角 $\underline{\theta}, \underline{\phi}$ 的函数;所以要想确定 $\bar{g}^{\omega\pm}_{\Yup}$ 就必须先确定 $\bar{\underline{d}}^{\omega\pm}_{\symup{\theta\phi}}$ 以及其自变量 $\symbf k_{\symup{\rho}}$ 所对应的 $\mathcal C$ 系非均匀二维数组 $\underline{\theta}^{\omega\pm}, \underline{\phi}^{\omega\pm}$。

然而,最先已知的仅有 $\mathcal Z$ 系均匀实二维网格 $k_{\symup{x}},k_{\symup{y}}$,因此必须先确定 $k_{\symup{x}},k_{\symup{y}}$ 所对应的 $\mathcal Z$ 系非均匀实二维数组 $\theta^{\omega\pm}, \phi^{\omega\pm}$,再利用 Eq(\ref{eq:2-37}) 中的球面三角坐标变换 $\bar{\bar{\underline{\mathcal{R}}}}_{\circleddash}$ 将其变换到 $\mathcal C$ 系下 $\underline{\theta}^{\omega\pm}, \underline{\phi}^{\omega\pm}$,才能接着按顺序计算 $\bar{\underline{d}}^{\omega\pm}_{\symup{\theta\phi}}, \bar{g}^{\omega\pm}_{\Yup}$。

利用角度迭代式 Eq(\ref{eq:2-135}),配合 $\mathcal Z \to \mathcal C$ 的球面三角坐标变换式 Eq(\ref{eq:2-37}),可以首先得到以 $k_{\symup{x}},k_{\symup{y}}$ 为自变量的 $\mathcal Z$ 系非均匀实二维数组 $\theta^{\pm}_\omega, \phi^{\pm}_\omega$;中途还可以得到 $k_{\symup{x}},k_{\symup{y}}$ 对应的有效折射率 Eq(\ref{eq:2-132b}) 即 $N^{\pm}_{\omega}$,并作为参数传入 Eq(\ref{eq:2-142}) 获得用于 Eq(\ref{eq:2-162}) 中的传播(矩阵)$\overline{\overline{\mathbb{e}^{ \mathbb{i} k^{\omega\pm}_{\symup{z}} \left( z - z_0 \right)}}}$ 的两对复本征值 $k^{\omega\pm}_{\symup{z}}$。

有了复波矢 z 分量 $k^{\omega\pm}_{\symup{z}}$ 以及初始已知的横向空间频率二维实数组 $k_{\symup{x}},k_{\symup{y}}$,根据 Eq(\ref{eq:2-119}),一方面可以取复波矢 $k^{\omega\pm}_{\symup{z}}$ 的实部 $k^{\omega\pm}_{\symup{zR}}$,使用 Eq(\ref{eq:2-38a}) 来计算 $\mathcal Z$ 系下的方位角、极角 $\theta^{\omega\pm}, \phi^{\omega\pm}$,并将其代入 Eq(\ref{eq:2-37}) 以得到 $\bar{\bar{\underline{\mathcal{R}}}}_{\circleddash}$,再代入 Eq(\ref{eq:2-40}) 以得到笛卡尔的 $\mathcal Z \to \mathcal C$ 系 $\symup{xyz}$ 三分量旋转矩阵 $\bar{\bar{\underline{R}}}_{\Yup}$;另一方面可将复波矢 $k^{\omega\pm}_{\symup{z}}$ 实部 $k^{\omega\pm}_{\symup{zR}}$ 与 $k_{\symup{x}},k_{\symup{y}}$ 一起代入 Eq(\ref{eq:2-76}) 后,将结果再与 $\bar{\bar{\underline{R}}}_{\Yup}$ 一起代入 Eq(\ref{eq:2-81b}),以返回 $\mathcal Z$ 系下的两个本征向量 $\bar{g}^{\omega\pm}_{\Yup}$,构成 Eq(\ref{eq:2-162}) 中的转移矩阵 $\overline{\bar{g}^{\omega\pm}_{\Yup}}^{\symup{T}}$ 及其逆矩阵 $\overline{\bar{g}^{\omega\pm}_{\Yup}}^{-\symup{T}}$。各向异性材料中的矢量傅立叶线性光学的主要框架/流程图/函数调用栈如下图 \ref{fig:2-2} 所示:
\begin{figure}[htbp]
%	\abovedisplayskip=0pt
	\belowdisplayskip=0pt
	\makebox[\textwidth][c]{\includegraphics[width=1.2\textwidth]{./figures/2.2.png}}
%	\center{\includegraphics[width=20cm]{./figures/2.2.png}}
	\caption{\label{fig:2-2} 各向异性材料中的矢量傅立叶线性光学的函数调用栈。}
\end{figure}

%\section{纯电各向异性介质中非线性光学过程的时空谱耦合波方程组}
\section{\protect\hyperlink{chap:\thesection}{纯电各向异性介质中非线性光学过程的时空谱耦合波方程组}}
\addtocontents{toc}{\protect\linkdest{chap:\thesection}}
\label{纯电各向异性介质中非线性光学过程的时空谱耦合波方程组}

只有在各向异性材料中的矢量傅立叶线性光学 Eq(\ref{eq:2-28}) 的本征值与本征向量均已知的基础上,才能继续考虑各向异性材料中的标量或矢量非线性光学过程:因为参与光学非线性混频的电场中,至少有一个成分,是晶体中的行波,而这些标量时空谱\footnote{从这一节开始,为了省略字数,不含衍射项或只有复振幅的时空谱也被简称做“时空谱”,注意他们大多没有含衍射项。} $\left\{ {\mathtt{g}}^{\omega_i\pm}_{z} \right\}$ 或矢量时空谱 $\left\{ \bar{g}^{\omega_i\pm}_{z} \right\}$ 对象们,既参与非线性的相互耦合过程,又同时在进行各自相互独立的线性衍射过程;对应于从数学上,他们不仅需要满足有源、非齐次的 Eq(\ref{eq:2-29}) 中的波动方程们,所构成的非线性耦合波动方程组,更需要首先满足各自的 Eq(\ref{eq:2-28}) 中的无源、齐次的线性波动方程。

由于二阶非线性系数相比一阶小很多,在非线性参数过程中的泵浦们,对材料的作用不是很强的前提下,可以认为非线性波动方程右侧的波源项,只对各矢量时空谱 $\left\{ \bar{g}^{\omega_i\pm}_{z} = {\mathtt{g}}^{\omega_i\pm}_{z} \cdot \hat{g}^{\omega_i\pm} \right\}$ 的复振幅,即各标量时空谱 $\left\{ {\mathtt{g}}^{\omega_i\pm}_{z} \right\}$ 的振幅和相位有作用,且假设只通过波动方程而非散度方程对 $\left\{ {\mathtt{g}}^{\omega_i\pm}_{z} \right\}$ 作用,而对各偏振态 $\left\{ \hat{g}^{\omega_i\pm} \right\}$ 的作用有限且微弱,因此接下来可以首先忽略 Eq(\ref{eq:2-29}) 中的散度方程,认为其对矢量时空谱 $\left\{ \bar{g}^{\omega_i\pm}_{z} \right\}$,包括其偏振态和复振幅,均无作用。

%\subsection{连续光和频、脉冲光倍频的时空谱耦合波方程}
\subsection{\protect\hyperlink{chap:\thesubsection}{连续光和频、脉冲光倍频的时空谱耦合波方程}}
\addtocontents{toc}{\protect\linkdest{chap:\thesubsection}}
\label{连续光和频、脉冲光倍频的时空谱耦合波方程}

在第 \pageref{con:1} 页处的限制条件,以及线性光学本征值、本征向量的约束下,Eq(\ref{eq:2-29}) 中的波动方程变为
\begin{equation} \label{eq:2-163}
%	\abovedisplayskip=10pt
%	\belowdisplayskip=10pt
	\displaystyle{\frac{\partial}{\partial z}} \bar{\bar{V}}^\omega \bar{g}^{\omega\pm}_z = k^{2}_{0\omega} {\bar{Q}^{{\symup{NL}},\omega}_z} \mathbb{e}^{-\mathbb{i} k^{\omega\pm}_{\symup{z}} z} ~,
\end{equation}
将 Eq(\ref{eq:2-27}) 代入上式 Eq(\ref{eq:2-163}),可得
\begin{equation} \label{eq:2-164}
	%	\abovedisplayskip=10pt
	%	\belowdisplayskip=10pt
	\displaystyle{\frac{\partial}{\partial z}} \begin{pmatrix} \displaystyle{- \frac{\partial}{\partial z}} - 2 \mathbb{i} k^{\omega\pm}_{\symup{z}} & 0 & \mathbb{i} k_{\symup{x}} \\ 0 & \displaystyle{- \frac{\partial}{\partial z}} - 2 \mathbb{i} k^{\omega\pm}_{\symup{z}} & \mathbb{i} k_{\symup{y}} \\ \mathbb{i} k_{\symup{x}} & \mathbb{i} k_{\symup{y}} & 0 \end{pmatrix} \bar{g}^{\omega\pm}_z = k^{2}_{0\omega} \frac{\bar{Q}^{{\symup{NL}},\omega}_z}{\mathbb{e}^{\mathbb{i} k^{\omega\pm}_{\symup{z}} z}} ~,
\end{equation}
对 $\bar{g}^{\omega\pm}_z$ 的三分量应用缓变振幅近似,即忽略上式 Eq(\ref{eq:2-164}) 关于 $z$ 坐标的二阶偏导项,得到
\begin{equation} \label{eq:2-165}
	%	\abovedisplayskip=10pt
	%	\belowdisplayskip=10pt
	\begin{pmatrix} - 2 k^{\omega\pm}_{\symup{z}} & 0 & k_{\symup{x}} \\ 0 & - 2 k^{\omega\pm}_{\symup{z}} & k_{\symup{y}} \\ k_{\symup{x}} & k_{\symup{y}} & 0 \end{pmatrix} \displaystyle{\frac{\partial \bar{g}^{\omega\pm}_z}{\partial z}} = - \mathbb{i} k^{2}_{0\omega} \frac{\bar{Q}^{{\symup{NL}},\omega}_z}{\mathbb{e}^{\mathbb{i} k^{\omega\pm}_{\symup{z}} z}} ~,
\end{equation}
对上述三元一次方程组应用 Cramer's Rule,发现其系数行列式
\begin{equation} \label{eq:2-166}
	%	\abovedisplayskip=10pt
	%	\belowdisplayskip=10pt
	\det \begin{pmatrix} - 2 k^{\omega\pm}_{\symup{z}} & 0 & k_{\symup{x}} \\ 0 & - 2 k^{\omega\pm}_{\symup{z}} & k_{\symup{y}} \\ k_{\symup{x}} & k_{\symup{y}} & 0 \end{pmatrix} = 2 k^{\omega\pm}_{\symup{z}} \left( k^2_{\symup{x}} + k^2_{\symup{y}} \right) ~,
\end{equation}
在非正(垂直)出射晶体端面,即 $\symbf{k}_{\symup{\rho}} \neq \symbf{0}$ 的情况下,是非零的,此时方程组 Eq(\ref{eq:2-165}) 具有唯一解
\begin{subequations} \label{eq:2-167}
	\abovedisplayskip=10pt
	\belowdisplayskip=10pt
	\begin{align}
		\displaystyle{\frac{\partial g^{\omega\pm}_{\symup{x} z}}{\partial z}} &= - \mathbb{i} k^{2}_{0\omega} \frac{ \det \begin{pmatrix} Q^{{\symup{NL}},\omega}_{\symup{x} z} & 0 & k_{\symup{x}} \\ Q^{{\symup{NL}},\omega}_{\symup{y} z} & - 2 k^{\omega\pm}_{\symup{z}} & k_{\symup{y}} \\ Q^{{\symup{NL}},\omega}_{\symup{z} z} & k_{\symup{y}} & 0 \end{pmatrix} }{ 2 k^{\omega\pm}_{\symup{z}} \left( k^2_{\symup{x}} + k^2_{\symup{y}} \right) \mathbb{e}^{\mathbb{i} k^{\omega\pm}_{\symup{z}} z}} \label{eq:2-167a}~, \\ \displaystyle{\frac{\partial g^{\omega\pm}_{\symup{y} z}}{\partial z}} &= - \mathbb{i} k^{2}_{0\omega} \frac{ \det \begin{pmatrix} - 2 k^{\omega\pm}_{\symup{z}} & Q^{{\symup{NL}},\omega}_{\symup{x} z} & k_{\symup{x}} \\ 0 & Q^{{\symup{NL}},\omega}_{\symup{y} z} & k_{\symup{y}} \\ k_{\symup{x}} & Q^{{\symup{NL}},\omega}_{\symup{z} z} & 0 \end{pmatrix} }{ 2 k^{\omega\pm}_{\symup{z}} \left( k^2_{\symup{x}} + k^2_{\symup{y}} \right) \mathbb{e}^{\mathbb{i} k^{\omega\pm}_{\symup{z}} z}} \label{eq:2-167b}~, \\ \displaystyle{\frac{\partial g^{\omega\pm}_{\symup{z} z}}{\partial z}} &= - \mathbb{i} k^{2}_{0\omega} \frac{ \det \begin{pmatrix} - 2 k^{\omega\pm}_{\symup{z}} & 0 & Q^{{\symup{NL}},\omega}_{\symup{x} z} \\ 0 & - 2 k^{\omega\pm}_{\symup{z}} & Q^{{\symup{NL}},\omega}_{\symup{y} z} \\ k_{\symup{x}} & k_{\symup{y}} & Q^{{\symup{NL}},\omega}_{\symup{z} z} \end{pmatrix} }{ 2 k^{\omega\pm}_{\symup{z}} \left( k^2_{\symup{x}} + k^2_{\symup{y}} \right) \mathbb{e}^{\mathbb{i} k^{\omega\pm}_{\symup{z}} z}} \label{eq:2-167c}~,
	\end{align}
\end{subequations}
也即
\begin{equation} \label{eq:2-168}
	%	\abovedisplayskip=10pt
	%	\belowdisplayskip=10pt
	\displaystyle{\frac{\partial \bar{g}^{\omega\pm}_{z}}{\partial z}} = - \mathbb{i} k^{2}_{0\omega} \frac{ \begin{pmatrix} k_{\symup{x}} \left( k_{\symup{y}} Q^{{\symup{NL}},\omega}_{\symup{y} z} + 2 k^{\omega\pm}_{\symup{z}} Q^{{\symup{NL}},\omega}_{\symup{z} z} \right) - k^2_{\symup{y}} Q^{{\symup{NL}},\omega}_{\symup{x} z} \\ k_{\symup{y}} \left( k_{\symup{x}} Q^{{\symup{NL}},\omega}_{\symup{x} z} + 2 k^{\omega\pm}_{\symup{z}} Q^{{\symup{NL}},\omega}_{\symup{z} z} \right) - k^2_{\symup{x}} Q^{{\symup{NL}},\omega}_{\symup{y} z} \\ 2 k^{\omega\pm}_{\symup{z}} \left( k_{\symup{y}} Q^{{\symup{NL}},\omega}_{\symup{y} z} + 2 k^{\omega\pm}_{\symup{z}} Q^{{\symup{NL}},\omega}_{\symup{z} z} \right) + 2 k^{\omega\pm}_{\symup{z}} k_{\symup{x}} Q^{{\symup{NL}},\omega}_{\symup{x} z} \end{pmatrix} }{ 2 k^{\omega\pm}_{\symup{z}} \left( k^2_{\symup{x}} + k^2_{\symup{y}} \right) \mathbb{e}^{\mathbb{i} k^{\omega\pm}_{\symup{z}} z}} ~,
\end{equation}
亦即
\begin{subequations} \label{eq:2-169}
	\abovedisplayskip=10pt
	\belowdisplayskip=10pt
	\begin{align}
		\displaystyle{\frac{\partial g^{\omega\pm}_{\symup{x} z}}{\partial z}} &= - \mathbb{i} k^{2}_{0\omega} \frac{ k_{\symup{y}} \left( k_{\symup{x}} Q^{{\symup{NL}},\omega}_{\symup{y} z} - k_{\symup{y}} Q^{{\symup{NL}},\omega}_{\symup{x} z} \right) + 2 k^{\omega\pm}_{\symup{z}} k_{\symup{x}} Q^{{\symup{NL}},\omega}_{\symup{z} z} }{ 2 k^{\omega\pm}_{\symup{z}} \left( k^2_{\symup{x}} + k^2_{\symup{y}} \right) \mathbb{e}^{\mathbb{i} k^{\omega\pm}_{\symup{z}} z}} \label{eq:2-169a}~, \\ \displaystyle{\frac{\partial g^{\omega\pm}_{\symup{y} z}}{\partial z}} &= - \mathbb{i} k^{2}_{0\omega} \frac{ k_{\symup{x}} \left( k_{\symup{y}} Q^{{\symup{NL}},\omega}_{\symup{x} z} - k_{\symup{x}} Q^{{\symup{NL}},\omega}_{\symup{y} z} \right) + 2 k^{\omega\pm}_{\symup{z}} k_{\symup{y}} Q^{{\symup{NL}},\omega}_{\symup{z} z} }{ 2 k^{\omega\pm}_{\symup{z}} \left( k^2_{\symup{x}} + k^2_{\symup{y}} \right) \mathbb{e}^{\mathbb{i} k^{\omega\pm}_{\symup{z}} z}} \label{eq:2-169b}~, \\ \displaystyle{\frac{\partial g^{\omega\pm}_{\symup{z} z}}{\partial z}} &= - \mathbb{i} k^{2}_{0\omega} \frac{ \left( k_{\symup{x}} Q^{{\symup{NL}},\omega}_{\symup{x} z} + k_{\symup{y}} Q^{{\symup{NL}},\omega}_{\symup{y} z} \right) + 2 k^{\omega\pm}_{\symup{z}} Q^{{\symup{NL}},\omega}_{\symup{z} z} }{ \left( k^2_{\symup{x}} + k^2_{\symup{y}} \right) \mathbb{e}^{\mathbb{i} k^{\omega\pm}_{\symup{z}} z}} \label{eq:2-169c}~,
	\end{align}
\end{subequations}
然而,该组混频解因含矢量非线性卷积 $\bar{Q}^{{\symup{NL}},\omega}_{z}$ 的三分量而计算较为麻烦,且对特定方向如正出射晶体端面($\symbf{k}_{\symup{\rho}} = \symbf{0}$)的空间频率组分无效。

对此,我们假设所考虑的时间频率 $\omega$ 组分,横向空间频率不太高 $k_{\symup{\rho}} < k^{\omega\pm}_{\symup{zR}}$;并且所混频出的 $\bar{g}^{\omega\pm}_z$ 的 $\symup{z}$ 分量 $g^{\omega\pm}_{\symup{z} z}$,相比其横向 $\perp$ 分量 $g^{\omega\pm}_{\perp z}$ 而言,振幅不太高,即 $\left| g^{\omega\pm}_{\symup{z} z} \right| < \left| g^{\omega\pm}_{\perp z} \right|$,在这种条件下,Eq(\ref{eq:2-165}) 可以退化为二维/横向的形式
\begin{subequations} \label{eq:2-170}
	\abovedisplayskip=10pt
	\belowdisplayskip=10pt
	\begin{align}
		\begin{pmatrix} - 2 k^{\omega\pm}_{\symup{z}} & 0 \\ 0 & - 2 k^{\omega\pm}_{\symup{z}} \end{pmatrix} \displaystyle{\frac{\partial \bar{g}^{\omega\pm}_{\perp z}}{\partial z}} &= - \mathbb{i} k^{2}_{0\omega} \frac{\bar{Q}^{{\symup{NL}},\omega}_{\perp z}}{\mathbb{e}^{\mathbb{i} k^{\omega\pm}_{\symup{z}} z}} \label{eq:2-170a} \\ \displaystyle{\frac{\partial \bar{g}^{\omega\pm}_{\perp z}}{\partial z}} &= \mathbb{i} k^{2}_{0\omega} \frac{\bar{Q}^{{\symup{NL}},\omega}_{\perp z}}{2 k^{\omega\pm}_{\symup{z}} \mathbb{e}^{\mathbb{i} k^{\omega\pm}_{\symup{z}} z}} \label{eq:2-170b}~,
	\end{align}
\end{subequations}
将每一个参与相互作用的时间频率组分 $\omega$ 所对应的 Eq(\ref{eq:2-170}) 组合起来,即得到实验室坐标系 $\mathcal{Z}$ 系下的矢量时空谱耦合波方程组;它既可以是相干脉冲光连续谱 ${\omega}$ 对应的无穷个矢量时空谱耦合波方程,也可以是有限个连续光 ${\omega_i}$ 对应的有限个矢量时空谱耦合波方程;也可以是单一频率的光,其自己参与的如四波混频或折射率微调制等,所对应的单个矢量时空谱耦合波方程。

将 $\bar{g}^{\omega\pm}_{\perp z} = {\mathtt{g}}^{\omega\pm}_{z} \cdot \hat{g}^{\omega\pm}_{\perp}$ 代入矢量非线性波动方程 Eq(\ref{eq:2-170}),并两侧点乘 $\hat{g}^{\omega\pm}_{\perp}$ 的共轭转置 $\hat{g}^{\omega\pm\dag}_{\perp}$,可得:
\begin{subequations} \label{eq:2-171}
	\abovedisplayskip=10pt
	\belowdisplayskip=10pt
	\begin{align}
		\hat{g}^{\omega\pm}_{\perp} \displaystyle{\frac{\partial {\mathtt{g}}^{\omega\pm}_{z}}{\partial z}} &= \mathbb{i} k^{2}_{0\omega} \frac{\bar{Q}^{{\symup{NL}},\omega}_{\perp z}}{2 k^{\omega\pm}_{\symup{z}} \mathbb{e}^{\mathbb{i} k^{\omega\pm}_{\symup{z}} z}} \label{eq:2-171a} \\ \displaystyle{\frac{\partial {\mathtt{g}}^{\omega\pm}_{z}}{\partial z}} &= \mathbb{i} k^{2}_{0\omega} \frac{\hat{g}^{\omega\pm\dag}_{\perp} \cdot \bar{Q}^{{\symup{NL}},\omega}_{\perp z}}{\hat{g}^{\omega\pm\dag}_{\perp} \cdot \hat{g}^{\omega\pm}_{\perp} 2 k^{\omega\pm}_{\symup{z}} \mathbb{e}^{\mathbb{i} k^{\omega\pm}_{\symup{z}} z}} \label{eq:2-171b}~,
	\end{align}
\end{subequations}
其中 Eq(\ref{eq:2-171b}) 即为标量时空谱所需满足的矢量非线性波动方程;注意并没有规定 $\hat{g}^{\omega\pm}_{\perp}$ 是二维还是三维归一化后的,因此 $\hat{g}^{\omega\pm\dag}_{\perp} \cdot \hat{g}^{\omega\pm}_{\perp}$ 的值不一定为 $1$;并且甚至可以对 Eq(\ref{eq:2-171a}) 左侧随便点乘一个二维复向量 $\bar{c}_{\perp}$,只需保证 Eq(\ref{eq:2-171b}) 的分母中的 $\bar{c}_{\perp} \cdot \hat{g}^{\omega\pm}_{\perp}$ 非零即可。

在 $\bar{g}^{\omega\pm}_z$ 的 $\symup{z}$ 分量 $g^{\omega\pm}_{\symup{z} z}$ 以及 $\bar{Q}^{{\symup{NL}},\omega}_{z}$ 的 $\symup{z}$ 分量 $Q^{{\symup{NL}},\omega}_{\symup{z} z}$ 都较小的前提下,上式 Eq(\ref{eq:2-171b}) 可以进一步写做 $\mathcal{Z}$ 系甚至 $\mathcal{C}$ 系下的笛卡尔三分量形式
\begin{subequations} \label{eq:2-172}
	\abovedisplayskip=10pt
	\belowdisplayskip=10pt
	\begin{align}
		\displaystyle{\frac{\partial {\mathtt{g}}^{\omega\pm}_{z}}{\partial z}} &= \mathbb{i} k^{2}_{0\omega} \frac{\hat{g}^{\omega\pm\dag}_{\Yup} \cdot \bar{Q}^{{\symup{NL}},\omega}_{\Yup z}}{ \hat{g}^{\omega\pm\dag}_{\Yup} \cdot \hat{g}^{\omega\pm}_{\Yup} 2 k^{\omega\pm}_{\symup{z}} \mathbb{e}^{\mathbb{i} k^{\omega\pm}_{\symup{z}} z} } \label{eq:2-172a} \\ &= \mathbb{i} k^{2}_{0\omega} \frac{ \left( \bar{\bar{R}}_{\Yup} \cdot \hat{\underline{g}}^{\omega\pm}_{\Yup} \right)^{\dag} \cdot \left( \bar{\bar{R}}_{\Yup} \cdot \bar{\underline{Q}}^{{\symup{NL}},\omega}_{\Yup z} \right) }{ \left( \bar{\bar{R}}_{\Yup} \cdot \hat{\underline{g}}^{\omega\pm}_{\Yup} \right)^{\dag} \cdot \left( \bar{\bar{R}}_{\Yup} \cdot \hat{\underline{g}}^{\omega\pm}_{\Yup} \right) 2 k^{\omega\pm}_{\symup{z}} \mathbb{e}^{\mathbb{i} k^{\omega\pm}_{\symup{z}} z}} \label{eq:2-172b} \\ &= \mathbb{i} k^{2}_{0\omega} \frac{ \hat{\underline{g}}^{\omega\pm\dag}_{\Yup} \cdot \left( \bar{\bar{R}}^{\dag}_{\Yup} \cdot \bar{\bar{R}}_{\Yup} \right) \cdot \bar{\underline{Q}}^{{\symup{NL}},\omega}_{\Yup z} }{ \hat{\underline{g}}^{\omega\pm\dag}_{\Yup} \cdot \left( \bar{\bar{R}}^{\dag}_{\Yup} \cdot \bar{\bar{R}}_{\Yup} \right) \cdot \hat{\underline{g}}^{\omega\pm}_{\Yup} 2 k^{\omega\pm}_{\symup{z}} \mathbb{e}^{\mathbb{i} k^{\omega\pm}_{\symup{z}} z} } \label{eq:2-172c} \\ &= \mathbb{i} k^{2}_{0\omega} \frac{ \hat{\underline{g}}^{\omega\pm\dag}_{\Yup} \cdot \bar{\underline{Q}}^{{\symup{NL}},\omega}_{\Yup z} }{ \hat{\underline{g}}^{\omega\pm\dag}_{\Yup} \cdot \hat{\underline{g}}^{\omega\pm}_{\Yup} 2 k^{\omega\pm}_{\symup{z}} \mathbb{e}^{\mathbb{i} k^{\omega\pm}_{\symup{z}} z} } \label{eq:2-172d}~,
	\end{align}
\end{subequations}
其中,用到了 $\bar{\bar{R}}^{\dag}_{\Yup} \cdot \bar{\bar{R}}_{\Yup} = \bar{\bar{R}}_{\Yup}^{\symup{T}} \cdot \bar{\bar{R}}_{\Yup} = 1$,以将其中所有矢量的坐标变换到 $\mathcal{C}$ 系下。注意,Eq(\ref{eq:2-172a}) 与 Eq(\ref{eq:2-172d}) 和 Eq(\ref{eq:2-171b}) 一样,均属于矢量非线性波动方程,尽管方程左侧是复振幅标量场 ${\mathtt{g}}^{\omega\pm}_{z}$ 的 $z$ 向偏导:因为右侧的 $\bar{\underline{Q}}^{{\symup{NL}},\omega}_{\Yup z}$ 是矢量的;并且复振幅标量场 ${\mathtt{g}}^{\omega\pm}_{z}$ 一旦已知,在乘以偏振态 $\hat{g}^{\omega\pm}_{\Yup}$ 后,可直接转换为矢量时空谱三分量 $\bar{g}^{\omega\pm}_{z} = {\mathtt{g}}^{\omega\pm}_{z} \cdot \hat{g}^{\omega\pm}_{\Yup}$。

对于以脉冲光倍频为主\myHyperFootnote{若 $\omega_{\symup{p}}, \omega$ 分别在单个泵浦光脉冲中心频率 $\Omega_{\symup{p}} = \Omega \big/ 2$ 及其产生的 $2\omega_{\symup{p}}$ 倍频光脉冲中心频率 $\Omega = 2\Omega_{\symup{p}}$ 附近,且 $\omega = 2\omega_{\symup{p}} > 0$,则该式代表单脉冲光倍频过程。}、以脉冲光整流后续级联电光效应\cite{jangMulticycleTerahertzPulse2020}为辅\myHyperFootnote{若 $\omega, \omega_{\symup{THz}}$ 分别在单个泵浦光脉冲中心频率 $\Omega \gg \Omega_{\symup{THz}}$ 及其产生的 THz 脉冲的中心频率 $\Omega_{\symup{THz}} \ll \Omega$ 附近,且 $\omega \gg \omega_{\symup{THz}} > 0$,则该式代表脉冲光整流后续级联电光效应过程。}的 $\omega' + \left( \omega-\omega' \right) \to \omega > 0$\myHyperFootnote{在映射到物理过程时,默认各频率为正;但在数学积分中可为负。}二阶非线性频率上转换过程,波动方程 Eq(\ref{eq:2-171b}) 与 Eq(\ref{eq:2-172a}) 右侧非线性波源项 $\bar{Q}^{{\symup{NL}},\omega}_{z} = \mathcal F \left[ \bar{P}^{{\symup{NL}},\omega}_z \right] \big/ {\symup{\varepsilon_0}}$ 进一步限定为
\begin{subequations} \label{eq:2-173}
	\abovedisplayskip=10pt
	\belowdisplayskip=10pt
	\begin{align}
		\bar{Q}^{(2)\omega}_{z} &= \mathcal F \left[ \bar{P}^{(2)\omega}_z \right] \big/ {\symup{\varepsilon_0}} \label{eq:2-173a} \\ &= \mathcal F \left[ \bar{\chi}^{(2)\omega}_{ {\symup{\mu}}_{12}z} E^\omega_{{\symup{\mu}}_1 z}\ \widetilde *\ E^\omega_{{\symup{\mu}}_2 z} \right] \label{eq:2-173b} \\ &= \mathcal F \left[ \bar{\chi}^{(2)\omega}_{ {\symup{\mu}}_{12}z} \right] * \mathcal F \left[ E^\omega_{{\symup{\mu}}_1 z}\ \widetilde *\ E^\omega_{{\symup{\mu}}_2 z} \right] \label{eq:2-173c}~,
	\end{align}
\end{subequations}
注意,其中简写了角标的 $\bar{\chi}^{(2)\omega}_{ {\symup{\mu}}_{12} z} := \bar{\chi}^{(2)\omega}_{ {\symup{\mu}}_1 {\symup{\mu}}_2z}$ 的三分量方向以 $\mathcal{Z}$ 系衡量,所以一般有 $27$ 个非零分量,这不是希望看到的;因此实际计算的时候,会像 Eq(\ref{eq:2-172d}) 一样,连同相关的量一起,转换到 $\mathcal{C}$ 系下,以利用晶体对称性将 $\mathcal{C}$ 系下的 $\bar{\underline{\chi}}^{(2)\omega}_{ {\symup{\mu}}_{12}z}$ 的非零独立分量压缩到个位数:
\begin{subequations} \label{eq:2-174}
	\abovedisplayskip=10pt
	\belowdisplayskip=10pt
	\begin{align}
		\bar{\underline{Q}}^{(2)\omega}_{z} &= \mathcal F \left[ \bar{\underline{P}}^{(2)\omega}_z \right] \big/ {\symup{\varepsilon_0}} \label{eq:2-174a} \\ &= \mathcal F \left[ \bar{\underline{\chi}}^{(2)\omega}_{ {\symup{\mu}}_{12}z} \underline{E}^\omega_{{\symup{\mu}}_1 z}\ \widetilde *\ \underline{E}^\omega_{{\symup{\mu}}_2 z} \right] \label{eq:2-174b} \\ &= \mathcal F \left[ \bar{\underline{\chi}}^{(2)\omega}_{ {\symup{\mu}}_{12}z} \right] * \mathcal F \left[ \underline{E}^\omega_{{\symup{\mu}}_1 z}\ \widetilde *\ \underline{E}^\omega_{{\symup{\mu}}_2 z} \right] \label{eq:2-174c}~,
	\end{align}
\end{subequations}
注意,二维空域傅立叶变换 $\mathcal F \left[ \cdot \right]$ 及其所有相关自变量 $\symbf{k}_{\symup{\rho}}, \symbf{\rho}$,以及后续的三维空域傅立叶变换 $\mathcal F_{\symup{3D}} \left[ \cdot \right]$ 及其所有相关自变量 $k_{\symup{xyz}}, xyz$(斜体),仍都处在 $\mathcal Z$ 系下;这属于矢量场的三分量标量场,随空域变化的,坐标自变量,与矢量场三分量 $\symup{xyz}$(直体)在 $\mathcal C$ 系下,二者并不相关、互相独立,因而可以分别适用两套坐标系,这并不冲突。

对于非脉冲/连续谱,而是两个独立单色波长的和频\myHyperFootnote{尽管双泵浦的光强可能不大,这里仍不说“上转换”:因为“上转换”过程一般是“一强一弱”双光泵浦生成弱 $\omega_3$,以至于参与混频的三波中,有两束光不满足泵浦未耗尽近似条件,因此只要有“上转换”则必有“下转换”过程发生(能量从 $\omega_3$ 回流到其中一个弱泵浦中),于是不可避免地涉及三波混频时空谱耦合波方程组中的至少 2 个方程,然而这里只给出了 1 个“上转换”过程的方程,因此这里只能代表/指和频过程。} 或上转换出第三个波长的 $\omega_1 + \omega_2 \to \omega_3$ 过程,以该离散的特殊情况为例,Eq(\ref{eq:2-174}) 变为
\begin{subequations} \label{eq:2-175}
	\abovedisplayskip=13pt
	\belowdisplayskip=13pt
	\begin{align}
		\bar{\underline{Q}}^{(2)\omega_3}_{z} &= \mathcal F \left[ \bar{\underline{P}}^{(2)\omega_3}_z \right] \big/ {\symup{\varepsilon_0}} \label{eq:2-175a} \\ &= \mathcal F \left[ \bar{\underline{\chi}}^{(2)\omega_3}_{ {\symup{\mu}}_{12}z} \underline{E}^{\omega_1}_{{\symup{\mu}}_1 z} \cdot \underline{E}^{\omega_2}_{{\symup{\mu}}_2 z} \right] \label{eq:2-175b} \\ &= \mathcal F \left[ \bar{\underline{\chi}}^{(2)\omega_3}_{ {\symup{\mu}}_{12}z} \right] * \mathcal F \left[ \underline{E}^{\omega_1}_{{\symup{\mu}}_1 z} \cdot \underline{E}^{\omega_2}_{{\symup{\mu}}_2 z} \right] \label{eq:2-175c} \\ &= \mathcal F^{-1}_{z} \left[ \mathcal F_{\symup{3D}} \left[ \bar{\underline{\chi}}^{(2)\omega_3}_{ {\symup{\mu}}_{12}z} \right] \right] * \mathcal F \left[ \underline{E}^{\omega_1}_{{\symup{\mu}}_1 z} \right] * \mathcal F \left[ \underline{E}^{\omega_2}_{{\symup{\mu}}_2 z} \right] \label{eq:2-175d} \\ &= \mathcal F^{-1}_{z} \left[ \mathcal F_{\symup{3D}} \left[ \bar{\underline{\chi}}^{(2)\omega_3}_{ {\symup{\mu}}_{12}z} \right] * \underline{G}^{\omega_1}_{{\symup{\mu}}_1 z} * \underline{G}^{\omega_2}_{{\symup{\mu}}_2 z} \right] \label{eq:2-175e}~,
	\end{align}
\end{subequations}
同样作为三阶张量场\myHyperFootnote{也即对于 27 个分量,每个分量,都是互相独立的 3 维标量场(空间分布为 $\mathcal{Z}$ 系下 $\symbf{r}$ 的函数)。},设其中二阶非线性系数 $\bar{\bar{\underline{\chi}}}^{(2)\omega_3}_{z}$ 的调制函数 $\bar{\bar{\underline{M}}}^{\omega_3}_z$ 满足
\begin{equation} \label{eq:2-176}
	\abovedisplayskip=13pt
	\belowdisplayskip=13pt
	\underline{\chi}^{(2)\omega_3}_{{\symup{\mu}}_{312}z} = \underline{\chi}^{(2)\omega_3}_{{\symup{\mu}}_{312}} \cdot \underline{M}^{\omega_3}_{{\symup{\mu}}_{312}z} ~,
\end{equation}
于是 Eq(\ref{eq:2-175e}) 继续写为
\begin{subequations} \label{eq:2-177}
	\abovedisplayskip=13pt
	\belowdisplayskip=13pt
	\begin{align}
		\underline{Q}^{(2)\omega_3}_{{\symup{\mu}}_{3} z} &= \mathcal F^{-1}_{z} \left[ \mathcal F_{\symup{3D}} \left[ \underline{\chi}^{(2)\omega_3}_{{\symup{\mu}}_{312}z} \right] * \underline{G}^{\omega_1}_{{\symup{\mu}}_1 z} * \underline{G}^{\omega_2}_{{\symup{\mu}}_2 z} \right] \label{eq:2-177a} \\ &= \underline{\chi}^{(2)\omega_3}_{{\symup{\mu}}_{312}} \cdot \mathcal F^{-1}_{z} \left[ \mathcal F_{\symup{3D}} \left[ \underline{M}^{\omega_3}_{{\symup{\mu}}_{312}z} \right] * \underline{G}^{\omega_1}_{{\symup{\mu}}_1 z} * \underline{G}^{\omega_2}_{{\symup{\mu}}_2 z} \right] \label{eq:2-177b} \\ &=: \underline{\chi}^{(2)\omega_3}_{{\symup{\mu}}_{312}} \cdot \mathcal F^{-1}_{z} \left[ \underline{C}^{\omega_3}_{{\symup{\mu}}_{312}} * \underline{G}^{\omega_1}_{{\symup{\mu}}_1 z} * \underline{G}^{\omega_2}_{{\symup{\mu}}_2 z} \right] \label{eq:2-177c}~,
	\end{align}
\end{subequations}
其中,定义了三维 $\symbf{k}$ 空间的倒格波系数(关于 $k_{\symup{xy}}, g_{\symup{z}}$ 的三阶张量场)
\begin{equation} \label{eq:2-178}
	\abovedisplayskip=13pt
	\belowdisplayskip=13pt
	\underline{C}^{\omega_3}_{{\symup{\mu}}_{312}} := \mathcal F_{\symup{3D}} \left[ \underline{M}^{\omega_3}_{{\symup{\mu}}_{312}z} \right] ~.
\end{equation}

若指定作为非线性波源的双单色时空谱的偏振态序/排列(而非组合),则有
\begin{subequations} \label{eq:2-179}
	\abovedisplayskip=13pt
	\belowdisplayskip=13pt
	\begin{align}
		\underline{Q}^{(2)\omega_3\pm\pm}_{{\symup{\mu}}_{3} z} &= \underline{\chi}^{(2)\omega_3}_{{\symup{\mu}}_{312}} \cdot \mathcal F^{-1}_{z} \left[ \underline{C}^{\omega_3}_{{\symup{\mu}}_{312}} * \underline{G}^{\omega_1\pm}_{{\symup{\mu}}_1 z} * \underline{G}^{\omega_2\pm}_{{\symup{\mu}}_2 z} \right] \label{eq:2-179a} \\ &= \underline{\chi}^{(2)\omega_3}_{{\symup{\mu}}_{312}} \cdot \mathcal F^{-1}_{z} \left[ \underline{C}^{\omega_3}_{{\symup{\mu}}_{312}} * \left( \hat{\underline{g}}^{\omega_1\pm}_{{\symup{\mu}}_1} {\mathtt{G}}^{\omega_1\pm}_{z} \right) * \left( \hat{\underline{g}}^{\omega_2\pm}_{{\symup{\mu}}_2} {\mathtt{G}}^{\omega_2\pm}_{z} \right) \right] \label{eq:2-179b} \\ &= \underline{\chi}^{(2)\omega_3}_{{\symup{\mu}}_{312}} \hat{\underline{g}}^{\omega_1\pm}_{{\symup{\mu}}_1} \hat{\underline{g}}^{\omega_2\pm}_{{\symup{\mu}}_2} \cdot \mathcal F^{-1}_{z} \left[ \underline{C}^{\omega_3}_{{\symup{\mu}}_{312}} * {\mathtt{G}}^{\omega_1\pm}_{z} * {\mathtt{G}}^{\omega_2\pm}_{z} \right] \label{eq:2-179c}~,
	\end{align}
\end{subequations}
其中,Eq(\ref{eq:2-179b}) $\to$ Eq(\ref{eq:2-179c}) 的条件是作为非线性波源项的双泵浦 $\underline{G}^{\omega_1}_{{\symup{\mu}}_1 z}, \underline{G}^{\omega_2}_{{\symup{\mu}}_2 z}$ 的偏振态 $\hat{\underline{g}}^{\omega_1}_{{\symup{\mu}}_1}, \hat{\underline{g}}^{\omega_2}_{{\symup{\mu}}_2}$ 固定,不是横向空间频率 $\symbf{k}_{\symup{\rho}}$ 的函数,不随方向改变,但这一般是不成立的:\ref{各向异性矢量傅立叶线性光学的函数调用栈} 小节的线性光学部分末已经给出结论,电各向异性材料中,电磁波的偏振态是 $\symbf{k}_{\symup{\rho}}$ 的函数,并且已在较大程度上完全解析;尽管如此,为了不止步于 Eq(\ref{eq:2-179b}),以及为了得到标量非线性时空谱耦合波方程,我们在“参与构成非线性波源的所有行波均为偏振态固定的标量场”,即“标量非线性波源”的假设\label{con:3}下,从 Eq(\ref{eq:2-179b}) 开始继续向后推导。

若不指定两个时空谱源的偏振态排列,则有
\begin{equation} \label{eq:2-180}
	\abovedisplayskip=13pt
	\belowdisplayskip=13pt
	\underline{Q}^{(2)\omega_3}_{{\symup{\mu}}_{3} z} = \sum_{\pm\pm} \underline{Q}^{(2)\omega_3\pm\pm}_{{\symup{\mu}}_{3} z} ~,
\end{equation}
另外,利用 matlab 的 $"{.*}"$ 语法(矩阵对应元素相乘),这里也可以用 $"{.\cdot}"$ 表示哈达马积/对应元素积 $\odot$,则 Eq(\ref{eq:2-179c}) 可以写成列向量的形式 %一般人会用 \odot 或 \circ,但为了省下这两个符号,我用了 $\ {.\cdot}\ $
\begin{equation} \label{eq:2-181}
	\abovedisplayskip=13pt
	\belowdisplayskip=13pt
	\bar{\underline{Q}}^{(2)\omega_3\pm\pm}_{z} = \bar{\underline{\chi}}^{(2)\omega_3}_{{\symup{\mu}}_{12}} \hat{\underline{g}}^{\omega_1\pm}_{{\symup{\mu}}_1} \hat{\underline{g}}^{\omega_2\pm}_{{\symup{\mu}}_2} \odot \mathcal F^{-1}_{z} \left[ \bar{\underline{C}}^{\omega_3}_{{\symup{\mu}}_{12}} * {\mathtt{G}}^{\omega_1\pm}_{z} * {\mathtt{G}}^{\omega_2\pm}_{z} \right] ~,
\end{equation}
更清楚地,可以将偏振态 $\pm$ 替换成字符 $\symup{p}_{123}$;并将 $\omega_{123}$ 以右下角标 $123$ 的形式出现:
\begin{equation} \label{eq:2-182}
	\abovedisplayskip=13pt
	\belowdisplayskip=13pt
	\bar{\underline{Q}}^{(2){\symup{p}}_{12}}_{3z} = \bar{\underline{\chi}}^{(2)}_{3{\symup{\mu}}_{12}} \hat{\underline{g}}^{\symup{p}_{1}}_{1{\symup{\mu}}_1} \hat{\underline{g}}^{\symup{p}_{2}}_{2{\symup{\mu}}_2} \odot \mathcal F^{-1}_{z} \left[ \bar{\underline{C}}_{3{\symup{\mu}}_{12}} * {\mathtt{G}}^{\symup{p}_{1}}_{1z} * {\mathtt{G}}^{\symup{p}_{2}}_{2z} \right] ~,
\end{equation}
相应的 Eq(\ref{eq:2-180}) 便可应用爱因斯坦求和约定,并在形式上与 Eq(\ref{eq:2-182}) 相同:
\begin{equation} \label{eq:2-183}
	\abovedisplayskip=13pt
	\belowdisplayskip=13pt
	\bar{\underline{Q}}^{(2)}_{3z} = \bar{\underline{\chi}}^{(2)}_{3{\symup{\mu}}_{12}} \hat{\underline{g}}^{\symup{p}_{1}}_{1{\symup{\mu}}_1} \hat{\underline{g}}^{\symup{p}_{2}}_{2{\symup{\mu}}_2} \odot \mathcal F^{-1}_{z} \left[ \bar{\underline{C}}_{3{\symup{\mu}}_{12}} * {\mathtt{G}}^{\symup{p}_{1}}_{1z} * {\mathtt{G}}^{\symup{p}_{2}}_{2z} \right] ~.
\end{equation}

将 Eq(\ref{eq:2-182}) 代入 Eq(\ref{eq:2-172d}),并指定参与混频的三个时空谱的偏振态排列,可得
\begin{subequations} \label{eq:2-184}
	\abovedisplayskip=13pt
	\belowdisplayskip=13pt
	\begin{align}
		\displaystyle{\frac{\partial {\mathtt{g}}^{\symup{p}_{312}}_{3z}}{\partial z}} &= \mathbb{i} k^{2}_{03} \frac{ \hat{\underline{g}}^{\symup{p}_{3}\dag}_{3} \cdot \bar{\underline{Q}}^{(2)\symup{p}_{12}}_{3z} }{ \hat{\underline{g}}^{\symup{p}_{3}\dag}_{3} \cdot \hat{\underline{g}}^{\symup{p}_{3}}_{3} 2 k^{\symup{p}_{3}}_{3\symup{z}} \mathbb{e}^{\mathbb{i} k^{\symup{p}_{3}}_{3\symup{z}} z} } \label{eq:2-184a} \\ &= \mathbb{i} k^{2}_{03} \frac{ \hat{\underline{g}}^{\symup{p}_{3}\dag}_{3} \cdot \bar{\underline{\chi}}^{(2)}_{3{\symup{\mu}}_{12}} \hat{\underline{g}}^{\symup{p}_{1}}_{1{\symup{\mu}}_1} \hat{\underline{g}}^{\symup{p}_{2}}_{2{\symup{\mu}}_2} \odot \mathcal F^{-1}_{z} \left[ \bar{\underline{C}}_{3{\symup{\mu}}_{12}} * {\mathtt{G}}^{\symup{p}_{1}}_{1z} * {\mathtt{G}}^{\symup{p}_{2}}_{2z} \right] }{ \hat{\underline{g}}^{\symup{p}_{3}\dag}_{3} \cdot \hat{\underline{g}}^{\symup{p}_{3}}_{3} 2 k^{\symup{p}_{3}}_{3\symup{z}} \mathbb{e}^{\mathbb{i} k^{\symup{p}_{3}}_{3\symup{z}} z} } \label{eq:2-184b} \\ &= \mathbb{i} k^{2}_{03} \frac{ \hat{\underline{g}}^{\symup{p}_{3}*}_{3} \odot \bar{\underline{\chi}}^{(2)}_{3{\symup{\mu}}_{12}} \hat{\underline{g}}^{\symup{p}_{1}}_{1{\symup{\mu}}_1} \hat{\underline{g}}^{\symup{p}_{2}}_{2{\symup{\mu}}_2} \odot \mathcal F^{-1}_{z} \left[ \bar{\underline{C}}_{3{\symup{\mu}}_{12}} * {\mathtt{G}}^{\symup{p}_{1}}_{1z} * {\mathtt{G}}^{\symup{p}_{2}}_{2z} \right] }{ \hat{\underline{g}}^{\symup{p}_{3}*}_{3} \odot \hat{\underline{g}}^{\symup{p}_{3}}_{3} 2 k^{\symup{p}_{3}}_{3\symup{z}} \mathbb{e}^{\mathbb{i} k^{\symup{p}_{3}}_{3\symup{z}} z} } \label{eq:2-184c} \\ &= \mathbb{i} k^{2}_{03} \frac{ \hat{\underline{g}}^{\symup{p}_{3}*}_{3{{\symup{\mu}}_{3}}} \underline{\chi}^{(2)}_{3{\symup{\mu}}_{312}} \hat{\underline{g}}^{\symup{p}_{1}}_{1{\symup{\mu}}_1} \hat{\underline{g}}^{\symup{p}_{2}}_{2{\symup{\mu}}_2} \mathcal F^{-1}_{z} \left[ \underline{C}_{3{\symup{\mu}}_{312}} * {\mathtt{G}}^{\symup{p}_{1}}_{1z} * {\mathtt{G}}^{\symup{p}_{2}}_{2z} \right] }{ \hat{\underline{g}}^{\symup{p}_{3}*}_{3{{\symup{\mu}}_{3}}} \hat{\underline{g}}^{\symup{p}_{3}}_{3{{\symup{\mu}}_{3}}} 2 k^{\symup{p}_{3}}_{3\symup{z}} \mathbb{e}^{\mathbb{i} k^{\symup{p}_{3}}_{3\symup{z}} z} } \label{eq:2-184d}~,
	\end{align}
\end{subequations}
在调制三阶定张量 $\underline{\chi}^{(2)}_{3{\symup{\mu}}_{312}}$ 的三阶张量场 $\underline{C}_{3{\symup{\mu}}_{312}}$ 退化为一个标量调制场 $C_{3}$ 乘以一个与 ${\symup{\mu}}_{312}$ 有关的三阶定张量 $\symup{C}_{3{\symup{\mu}}_{312}}$,即
\begin{equation} \label{eq:2-185}
	\abovedisplayskip=13pt
	\belowdisplayskip=13pt
	\underline{C}_{3{\symup{\mu}}_{312}} = \symup{C}_{3{\symup{\mu}}_{312}} C_{3}
\end{equation}
的前提下,定义第 \pageref{con:3} 页的“标量非线性波源”条件下的有效非线性系数
\begin{equation} \label{eq:2-186}
	\abovedisplayskip=13pt
	\belowdisplayskip=13pt
	\chi^{(2)\symup{p}_{312}}_{3\symup{eff}} := \hat{\underline{g}}^{\symup{p}_{3}*}_{3{{\symup{\mu}}_{3}}} \symup{C}_{3{\symup{\mu}}_{312}} \underline{\chi}^{(2)}_{3{\symup{\mu}}_{312}} \hat{\underline{g}}^{\symup{p}_{1}}_{1{\symup{\mu}}_1} \hat{\underline{g}}^{\symup{p}_{2}}_{2{\symup{\mu}}_2} ~,
\end{equation}
则指定了三个偏振态排列 $\symup{p}_{312}$ 的 Eq(\ref{eq:2-184d}) 变为
\begin{equation} \label{eq:2-187}
	\abovedisplayskip=13pt
	\belowdisplayskip=13pt
	\displaystyle{\frac{\partial {\mathtt{g}}^{\symup{p}_{312}}_{3z}}{\partial z}} = \mathbb{i} k^{2}_{03} \frac{ \chi^{(2)\symup{p}_{312}}_{3\symup{eff}} \mathcal F^{-1}_{z} \left[ C_{3} * {\mathtt{G}}^{\symup{p}_{1}}_{1z} * {\mathtt{G}}^{\symup{p}_{2}}_{2z} \right] }{ \left| \hat{\underline{g}}^{\symup{p}_{3}}_{3} \right|^2 2 k^{\symup{p}_{3}}_{3\symup{z}} \mathbb{e}^{\mathbb{i} k^{\symup{p}_{3}}_{3\symup{z}} z} } ~,
\end{equation}
若只指定 $\omega_3$ 的偏振态 $\symup{p}_{3}$,则表达式与上式 Eq(\ref{eq:2-187}) 相同(类似 Eq(\ref{eq:2-183}) 之于 Eq(\ref{eq:2-182})),但注意对 $\symup{p}_{12}$ 使用了爱因斯坦求和约定:
\begin{equation} \label{eq:2-188}
	\abovedisplayskip=13pt
	\belowdisplayskip=13pt
	\displaystyle{\frac{\partial {\mathtt{g}}^{\symup{p}_{3}}_{3z}}{\partial z}} = \mathbb{i} k^{2}_{03} \frac{ \chi^{(2)\symup{p}_{312}}_{3\symup{eff}} \mathcal F^{-1}_{z} \left[ C_{3} * {\mathtt{G}}^{\symup{p}_{1}}_{1z} * {\mathtt{G}}^{\symup{p}_{2}}_{2z} \right] }{ \left| \hat{\underline{g}}^{\symup{p}_{3}}_{3} \right|^2 2 k^{\symup{p}_{3}}_{3\symup{z}} \mathbb{e}^{\mathbb{i} k^{\symup{p}_{3}}_{3\symup{z}} z} } ~,
\end{equation}
注意,上式暂不能对 $\symup{p}_{3}$ 使用爱因斯坦求和约定,因为取值不同的 $\symup{p}_{3}$ 所对应的标量场 ${\mathtt{g}}^{\symup{p}_{3}}_{3z}$,所对应的偏振态 $\pm$ 不同,不满足干涉条件,不能直接相加;它需要配合 $\hat{g}^{\symup{p}_{3}}_{3}$,并且在表示双偏振态 $\pm$ 叠加出的 $\bar{g}_{3z}$ 总场时,才能使用爱因斯坦求和约定,如 $\bar{g}_{3z} = {\mathtt{g}}^{\symup{p}_{3}}_{3z} \hat{g}^{\symup{p}_{3}}_{3}$。

式 Eq(\ref{eq:2-188}) 即为各向异性材料中 $\omega_{3}$ 对应的傅立叶标量和频或上转换耦合波方程,其配合 Eq(\ref{eq:2-186}) 和矢量场偏振态 $\hat{g}^{\symup{p}_{3}}_{3}$ 即成为矢量和频或上转换时空谱耦合波方程,但受 Eq(\ref{eq:2-186}) 限制,2 个非线性波源仍是标量场,因此还不是最广义的矢量过程:如果参与混频的三波都为矢量光(倒空间偏振态有分布且不均匀),则 Eq(\ref{eq:2-188}) 的分子相关部分需要退化至 Eq(\ref{eq:2-179b});除此之外,对于更长波长的 $\omega_{12}$ 还分别有 2 个标/矢量差频或下转换时空谱耦合波方程,三者共同构成标/矢量三波混频时空谱耦合波方程组,以描述有弱光参与的上转换、下转换过程。

现将 \cref{eq:2-175,eq:2-176,eq:2-177,eq:2-178,eq:2-179,eq:2-180,eq:2-181,eq:2-182,eq:2-183,eq:2-184,eq:2-185,eq:2-186,eq:2-187,eq:2-188} 的离散波长/离散谱/连续波的特殊情况,拓展到连续谱/脉冲光的广义情况。替换掉相关变量和算符后,Eq(\ref{eq:2-174c}) 变为描述脉冲光倍频或光整流后续级联电光效应\cite{jangMulticycleTerahertzPulse2020}的,时间频率域谱内和频/差频过程的下述方程:
\begin{subequations} \label{eq:2-189}
	\abovedisplayskip=13pt
	\belowdisplayskip=13pt
	\begin{align}
		\bar{\underline{Q}}^{(2)\omega}_{z} &= \mathcal F \left[ \bar{\underline{\chi}}^{(2)\omega}_{ {\symup{\mu}}_{12}z} \right] * \mathcal F \left[ \underline{E}^\omega_{{\symup{\mu}}_1 z}\ \widetilde *\ \underline{E}^\omega_{{\symup{\mu}}_2 z} \right]  \label{eq:2-189a} \\ &= \mathcal F^{-1}_{z} \left[ \mathcal F_{\symup{3D}} \left[ \bar{\underline{\chi}}^{(2)\omega}_{ {\symup{\mu}}_{12}z} \right] \right] * \mathcal F \left[ \underline{E}^{\omega_1}_{{\symup{\mu}}_1 z} \right] \widetilde \circledast\ \mathcal F \left[ \underline{E}^{\omega_2}_{{\symup{\mu}}_2 z} \right] \label{eq:2-189b} \\ &= \mathcal F^{-1}_{z} \left[ \mathcal F_{\symup{3D}} \left[ \bar{\underline{\chi}}^{(2)\omega}_{ {\symup{\mu}}_{12}z} \right] * \underline{G}^{\omega}_{{\symup{\mu}}_1 z} \widetilde \circledast\ \underline{G}^{\omega}_{{\symup{\mu}}_2 z} \right] \label{eq:2-189c}~,
	\end{align}
\end{subequations}
其中,定义了 $\widetilde \circledast := \begin{smallmatrix} \widetilde * \\ * \end{smallmatrix}$,从上到下阅读这两个符号中的任意一个,即表示三重积分号(内)从左到右先时间频率域一维卷积,后空间频率域二维卷积;而若从内到外读三重积分号(内的被积函数) or 算符 $"\widetilde \circledast"$,则反之:先空间频率域二维卷积,后时间频率域一维卷积,这也是相应程序中 for 循环从内到外层的计算顺序。通俗地说,需按以下顺序执行积分:$"\widetilde \circledast"$ 的 $\symbf{k}_{\symup{\rho}}$ 域 $\to$ $\symbf{k}_{\symup{\rho}}$ 域的 $"*"$ $\to$ $q_{\symup{z}}$ 域的 $F^{-1}_{z} \left[ \cdot \right]$ $\to$ $"\widetilde \circledast"$ 的 $\omega$ 域(即 $\omega$ 域的 $"\ \widetilde *\ "$)。

当调制函数 $\bar{\underline{M}}^{\omega}_{{\symup{\mu}}_{12}z}$ 退化一个标量调制场 $M^{\omega}_{z}$ 乘以一个三阶定张量 $\bar{\underline{\symup{M}}}^{\omega}_{{\symup{\mu}}_{12}}$ 后,将倒格波系数 $\bar{\underline{C}}^{\omega}_{{\symup{\mu}}_{12}}$ 所正比的空域三维分布的二阶非线性系数 $\bar{\underline{\chi}}^{(2)\omega}_{ {\symup{\mu}}_{12}z}$ 的三维傅立叶变换
\begin{subequations} \label{eq:2-190}
	\abovedisplayskip=13pt
	\belowdisplayskip=13pt
	\begin{align}
		\mathcal F_{\symup{3D}} \left[ \bar{\underline{\chi}}^{(2)\omega}_{ {\symup{\mu}}_{12}z} \right] &= \bar{\underline{\chi}}^{(2)\omega}_{{\symup{\mu}}_{12}} \odot \bar{\underline{C}}^{\omega}_{{\symup{\mu}}_{12}} = \bar{\underline{\chi}}^{(2)\omega}_{{\symup{\mu}}_{12}} \odot \bar{\underline{\symup{C}}}^{\omega}_{{\symup{\mu}}_{12}} C^{\omega} \label{eq:2-190a}~, \\ \text{where}\ \ \ \ \ \ \bar{\underline{\chi}}^{(2)\omega}_{{\symup{\mu}}_{12}z} &= \bar{\underline{\chi}}^{(2)\omega}_{{\symup{\mu}}_{12}} \odot \bar{\underline{M}}^{\omega}_{{\symup{\mu}}_{12}z} = \bar{\underline{\chi}}^{(2)\omega}_{{\symup{\mu}}_{12}} \odot \bar{\underline{\symup{M}}}^{\omega}_{{\symup{\mu}}_{12}} M^{\omega}_{z} \label{eq:2-190b}~,
	\end{align}
\end{subequations}
代入 Eq(\ref{eq:2-189c}),并遍历或指定非线性波源的偏振态序列,可得
\begin{subequations} \label{eq:2-191}
	\abovedisplayskip=13pt
	\belowdisplayskip=13pt
	\begin{align}
		\bar{\underline{Q}}^{(2)\omega}_{z} &= \bar{\underline{\chi}}^{(2)\omega}_{{\symup{\mu}}_{12}} \odot \bar{\underline{\symup{C}}}^{\omega}_{{\symup{\mu}}_{12}} \mathcal F^{-1}_{z} \left[ C^{\omega} * \underline{G}^{\omega}_{{\symup{\mu}}_1 z} \widetilde \circledast\ \underline{G}^{\omega}_{{\symup{\mu}}_2 z} \right] \label{eq:2-191a} \\ &= \bar{\underline{\symup{C}}}^{\omega}_{{\symup{\mu}}_{12}} \odot \bar{\underline{\chi}}^{(2)\omega}_{{\symup{\mu}}_{12}} \mathcal F^{-1}_{z} \left[ C^{\omega} * \left( \hat{\underline{g}}^{\omega{\symup{p}}_{1}}_{{\symup{\mu}}_1} {\mathtt{G}}^{\omega{\symup{p}}_{1}}_{z} \right) \widetilde \circledast \left( \hat{\underline{g}}^{\omega{\symup{p}}_{2}}_{{\symup{\mu}}_2} {\mathtt{G}}^{\omega{\symup{p}}_{2}}_{z} \right) \right] \label{eq:2-191b} \\ &= \bar{\underline{\symup{C}}}^{\omega}_{{\symup{\mu}}_{12}} \odot \bar{\underline{\chi}}^{(2)\omega}_{{\symup{\mu}}_{12}} \hat{\underline{g}}^{\omega{\symup{p}}_{1}}_{{\symup{\mu}}_1} \ \widetilde *\ \hat{\underline{g}}^{\omega{\symup{p}}_{2}}_{{\symup{\mu}}_2} \mathcal F^{-1}_{z} \left[ C^{\omega} * {\mathtt{G}}^{\omega{\symup{p}}_{1}}_{z} \widetilde \circledast\ {\mathtt{G}}^{\omega{\symup{p}}_{2}}_{z} \right] \label{eq:2-191c}~, \\ \bar{\underline{Q}}^{(2)\omega{\symup{p}}_{12}}_{z} &= \bar{\underline{\symup{C}}}^{\omega}_{{\symup{\mu}}_{12}} \odot \bar{\underline{\chi}}^{(2)\omega}_{{\symup{\mu}}_{12}} \hat{\underline{g}}^{\omega{\symup{p}}_{1}}_{{\symup{\mu}}_1} \ \widetilde *\ \hat{\underline{g}}^{\omega{\symup{p}}_{2}}_{{\symup{\mu}}_2} \mathcal F^{-1}_{z} \left[ C^{\omega} * {\mathtt{G}}^{\omega{\symup{p}}_{1}}_{z} \widetilde \circledast\ {\mathtt{G}}^{\omega{\symup{p}}_{2}}_{z} \right] \label{eq:2-191d}~,
	\end{align}
\end{subequations}
注意,不论正上标带 $"\sim"$ 的符号有多少个(这里有两个:$"\widetilde *, \widetilde \circledast"$),只对这些符号所作用的最左(这里即 $\hat{\underline{g}}^{\omega{\symup{p}}_{1}}_{{\symup{\mu}}_1}$)到最右(这里即 ${\mathtt{G}}^{\omega{\symup{p}}_{2}}_{z}$)之间的部分作为被积函数,在时间频率维度做一次(而不是多次)一维卷积积分。此外,这里的 Eq.(\ref{eq:2-191b}) $\to$ Eq.(\ref{eq:2-191c}) 也使用了第 \pageref{con:3} 页的“标量非线性波源”条件。% $"\ \widetilde {}\ "$

将 Eq.(\ref{eq:2-191d}) 代入 Eq.(\ref{eq:2-172d}),可得
\begin{subequations} \label{eq:2-192}
	\abovedisplayskip=13pt
	\belowdisplayskip=13pt
	\begin{align}
		\displaystyle{\frac{\partial {\mathtt{g}}^{\omega\symup{p}_{312}}_{z}}{\partial z}} &= \mathbb{i} k^{2}_{0\omega} \frac{ \hat{\underline{g}}^{\omega\symup{p}_{3}\dag} \cdot \bar{\underline{Q}}^{(2)\omega\symup{p}_{12}}_{z} }{ \hat{\underline{g}}^{\omega\symup{p}_{3}\dag} \cdot \hat{\underline{g}}^{\omega\symup{p}_{3}} 2 k^{\omega\symup{p}_{3}}_{\symup{z}} \mathbb{e}^{\mathbb{i} k^{\omega\symup{p}_{3}}_{\symup{z}} z} } \label{eq:2-192a} \\ &= \mathbb{i} k^{2}_{0\omega} \frac{ \hat{\underline{g}}^{\omega\symup{p}_{3}*} \odot \bar{\underline{\symup{C}}}^{\omega}_{{\symup{\mu}}_{12}} \odot \bar{\underline{\chi}}^{(2)\omega}_{{\symup{\mu}}_{12}} \hat{\underline{g}}^{\omega{\symup{p}}_{1}}_{{\symup{\mu}}_1} \ \widetilde *\ \hat{\underline{g}}^{\omega{\symup{p}}_{2}}_{{\symup{\mu}}_2} \mathcal F^{-1}_{z} \left[ C^{\omega} * {\mathtt{G}}^{\omega{\symup{p}}_{1}}_{z} \widetilde \circledast\ {\mathtt{G}}^{\omega{\symup{p}}_{2}}_{z} \right] }{ \hat{\underline{g}}^{\omega\symup{p}_{3}*} \odot \hat{\underline{g}}^{\omega\symup{p}_{3}} 2 k^{\omega\symup{p}_{3}}_{\symup{z}} \mathbb{e}^{\mathbb{i} k^{\omega\symup{p}_{3}}_{\symup{z}} z} } \label{eq:2-192b} \\ &= \mathbb{i} k^{2}_{0\omega} \frac{ \hat{\underline{g}}^{\omega\symup{p}_{3}*}_{{\symup{\mu}}_{3}} \underline{\symup{C}}^{\omega}_{{\symup{\mu}}_{312}} \underline{\chi}^{(2)\omega}_{{\symup{\mu}}_{312}} \hat{\underline{g}}^{\omega\symup{p}_{1}}_{{\symup{\mu}}_1} \ \widetilde *\ \hat{\underline{g}}^{\omega\symup{p}_{2}}_{{\symup{\mu}}_2} \mathcal F^{-1}_{z} \left[ C^{\omega} * {\mathtt{G}}^{\omega{\symup{p}}_{1}}_{z} \widetilde \circledast\ {\mathtt{G}}^{\omega{\symup{p}}_{2}}_{z} \right] }{ \hat{\underline{g}}^{\omega\symup{p}_{3}*}_{{{\symup{\mu}}_{3}}} \hat{\underline{g}}^{\omega\symup{p}_{3}}_{{{\symup{\mu}}_{3}}} 2 k^{\omega\symup{p}_{3}}_{\symup{z}} \mathbb{e}^{\mathbb{i} k^{\omega\symup{p}_{3}}_{\symup{z}} z} } \label{eq:2-192c} \\ &=: \mathbb{i} k^{2}_{0\omega} \frac{ \widetilde{\chi}^{(2)\omega\symup{p}_{312}}_{\symup{eff}} \mathcal F^{-1}_{z} \left[ C^{\omega} * {\mathtt{G}}^{\omega{\symup{p}}_{1}}_{z} \widetilde \circledast\ {\mathtt{G}}^{\omega{\symup{p}}_{2}}_{z} \right] }{ \left| \hat{\underline{g}}^{\omega\symup{p}_{3}} \right|^2 2 k^{\omega\symup{p}_{3}}_{\symup{z}} \mathbb{e}^{\mathbb{i} k^{\omega\symup{p}_{3}}_{\symup{z}} z} } \label{eq:2-192d}~,
	\end{align}
\end{subequations}
其中,定义了第 \pageref{con:3} 页“标量非线性波源”条件下的脉冲光倍频过程的有效非线性系数
\begin{equation} \label{eq:2-193}
	\abovedisplayskip=13pt
	\belowdisplayskip=13pt
	\widetilde{\chi}^{(2)\omega\symup{p}_{312}}_{\symup{eff}} := \hat{\underline{g}}^{\omega\symup{p}_{3}*}_{{\symup{\mu}}_{3}} \underline{\symup{C}}^{\omega}_{{\symup{\mu}}_{312}} \underline{\chi}^{(2)\omega}_{{\symup{\mu}}_{312}} \hat{\underline{g}}^{\omega\symup{p}_{1}}_{{\symup{\mu}}_1} \ \widetilde *\ \hat{\underline{g}}^{\omega\symup{p}_{2}}_{{\symup{\mu}}_2} ~,
\end{equation}
其 $\chi$ 头上的一个波浪符号 $"\sim"$,表示 $\widetilde{\chi}^{(2)\omega\symup{p}_{312}}_{\symup{eff}}$ 整体作为被积函数,处在时间频率域的一维卷积积分内。

遍历非线性波源的偏振态序列,即得指定频率且指定偏振模的倍频脉冲光的倒空间矢量偏振态 $\hat{g}^{\omega\symup{p}_{3}}$ 所对应的标量复振幅系数 ${\mathtt{g}}^{\omega\symup{p}_{3}}_{z}$ 的 $z$ 向变化率
\begin{subequations} \label{eq:2-194}
	\abovedisplayskip=13pt
	\belowdisplayskip=13pt
	\begin{align}
		\displaystyle{\frac{\partial {\mathtt{g}}^{\omega\symup{p}_{3}}_{z}}{\partial z}} &= \mathbb{i} k^{2}_{0\omega} \frac{ \hat{\underline{g}}^{\omega\symup{p}_{3}\dag} \cdot \bar{\underline{Q}}^{(2)\omega}_{z} }{ \hat{\underline{g}}^{\omega\symup{p}_{3}\dag} \cdot \hat{\underline{g}}^{\omega\symup{p}_{3}} 2 k^{\omega\symup{p}_{3}}_{\symup{z}} \mathbb{e}^{\mathbb{i} k^{\omega\symup{p}_{3}}_{\symup{z}} z} } \label{eq:2-194a} \\ &= \mathbb{i} k^{2}_{0\omega} \frac{ \widetilde{\chi}^{(2)\omega\symup{p}_{312}}_{\symup{eff}} \mathcal F^{-1}_{z} \left[ C^{\omega} * {\mathtt{G}}^{\omega{\symup{p}}_{1}}_{z} \widetilde \circledast\ {\mathtt{G}}^{\omega{\symup{p}}_{2}}_{z} \right] }{ \left| \hat{\underline{g}}^{\omega\symup{p}_{3}} \right|^2 2 k^{\omega\symup{p}_{3}}_{\symup{z}} \mathbb{e}^{\mathbb{i} k^{\omega\symup{p}_{3}}_{\symup{z}} z} } \label{eq:2-194b}~,
	\end{align}
\end{subequations}
上式 Eq(\ref{eq:2-194}) 即为各向异性材料中光波段单色 $\omega$ 对应的傅立叶标量脉冲光倍频或光整流后续级联电光效应耦合波方程,其配合 Eq(\ref{eq:2-193}) 和 $\hat{g}^{\omega\symup{p}_{3}}$ 即变成右侧标量、左侧矢量的版本;若进一步回退到分子含 Eq(\ref{eq:2-191b}) 的情形,则进化至可完整描述三个脉冲矢量光场间的混频;除此之外,对于 THz 波段的 $\omega$,还可能有 1 个标/矢量脉冲光整流时空谱耦合波方程,二者共同构成标/矢量脉冲光谱内(自)混频时空谱耦合波方程组,以描述单个脉冲光谱内混频(上/下转换)。
  
%\subsection{三波混频、脉冲光整流的时空谱耦合波方程组}
\subsection{\protect\hyperlink{chap:\thesubsection}{三波混频、脉冲光整流的时空谱耦合波方程组}}
\addtocontents{toc}{\protect\linkdest{chap:\thesubsection}}
\label{三波混频、脉冲光整流的时空谱耦合波方程组}
% 有些人就会说,这俩是一回事么?嘿嘿,那是你没看穿,看穿就是一回事。

考虑脉冲光谱内(自)差频即光整流过程,若不纳入其后续的级联电光效应,则该二阶非线性过程的频率守恒方程\myHyperFootnote{也可写成加/和的形式 $\left( \omega'+\omega \right) - \omega' \to \omega > 0$,但有时减/差更贴近卷积或相关运算的数学定义。}为 $ \omega' - \left( \omega'-\omega \right) \to \omega > 0$;对于该下转换过程,波动方程 Eq(\ref{eq:2-172d}) 右侧非线性波源项 $\bar{\underline{Q}}^{{\symup{NL}},\omega}_{z} = \mathcal F \left[ \bar{\underline{P}}^{{\symup{NL}},\omega}_z \right] \big/ {\symup{\varepsilon_0}}$ 变为
\begin{subequations} \label{eq:2-195}
	\abovedisplayskip=10pt
	\belowdisplayskip=10pt
	\begin{align}
		\bar{\underline{Q}}^{(2)\omega}_{z} &= \mathcal F \left[ \bar{\underline{P}}^{(2)\omega}_z \right] \big/ {\symup{\varepsilon_0}} \label{eq:2-195a} \\ &= \mathcal F \left[ \bar{\underline{\chi}}^{(2)\omega*}_{ {\symup{\mu}}_{32}z} \underline{E}^\omega_{{\symup{\mu}}_3 z}\ \widetilde *\ \underline{E}^{-\omega*}_{{\symup{\mu}}_2 z} \right] \label{eq:2-195b} \\ &= \mathcal F \left[ \bar{\underline{\chi}}^{(2)\omega*}_{ {\symup{\mu}}_{32}z} \right] * \mathcal F \left[ \underline{E}^\omega_{{\symup{\mu}}_3 z}\ \widetilde *\ \underline{E}^{-\omega*}_{{\symup{\mu}}_2 z} \right] \label{eq:2-195c} \\ &= \mathcal F^{-1}_{z} \left[ \mathcal F_{\symup{3D}} \left[ \bar{\underline{\chi}}^{(2)\omega*}_{ {\symup{\mu}}_{32}z} \right] \right] * \mathcal F \left[ \underline{E}^{\omega}_{{\symup{\mu}}_3 z} \right] \widetilde \circledast\ \mathcal F \left[ \underline{E}^{-\omega*}_{{\symup{\mu}}_2 z} \right] \label{eq:2-195d} \\ &= \mathcal F^{-1}_{z} \left[ \mathcal F_{\symup{3D}} \left[ \bar{\underline{\chi}}^{(2)\omega*}_{ {\symup{\mu}}_{32}z} \right] * \underline{G}^{\omega}_{{\symup{\mu}}_3 z} \widetilde \circledast\ \underline{G}^{-\omega*}_{{\symup{\mu}}_2 z} \left( - \symbf{k}_{\symup{\rho}} \right) \right] \label{eq:2-195e} \\ &= \mathcal F^{-1}_{z} \left[ \mathcal F_{\symup{3D}} \left[ \bar{\underline{\chi}}^{(2)\omega*}_{ {\symup{\mu}}_{32}z} \right] * \underline{G}^{\omega}_{{\symup{\mu}}_2 z} \widetilde \circledcirc\ \underline{G}^{\omega}_{{\symup{\mu}}_3 z} \right] \label{eq:2-195f}~,
	\end{align}
\end{subequations}
其中,类似 $"*"$ 之于 $"\widetilde \circledast"$ 地,定义了 $\symbf{k}_{\symup{\rho}}$ 域互相关算符 $"\circ"$,以及 $\omega, \symbf{k}_{\symup{\rho}}$ 域的互相关算符 $"\widetilde \circledcirc"$;当互相关与卷积同时出现时,为了省略对互相关运算及其对象的括号,要求同为 $\symbf{k}_{\symup{\rho}}$ 域或 $\omega$ 域的互相关和卷积,互相关的优先级高于卷积。另外,Eq(\ref{eq:2-195}) 中用到了如下关系:
\begin{subequations} \label{eq:2-196}
	\abovedisplayskip=10pt
	\belowdisplayskip=10pt
	\begin{align}
		\mathcal F \left[ \underline{E}^{-\omega*}_{{\symup{\mu}}_2 z} \right] &= \underline{G}^{-\omega*}_{{\symup{\mu}}_2 z} \left( - \symbf{k}_{\symup{\rho}} \right) \label{eq:2-196a}~, \\ \mathcal F \left[ \underline{E}^\omega_{{\symup{\mu}}_3 z}\ \widetilde *\ \underline{E}^{-\omega*}_{{\symup{\mu}}_2 z} \right] &= \underline{G}^{\omega}_{{\symup{\mu}}_2 z} \widetilde \circledcirc\ \underline{G}^{\omega}_{{\symup{\mu}}_3 z} \label{eq:2-196b}~,
	\end{align}
\end{subequations}
这纯粹是数学关系,以 $\symbf{k}_{\symup{\rho}}$ 域为例,上述 Eq(\ref{eq:2-196}) 只是下述更一般关系的应用:
\begin{subequations} \label{eq:2-197}
	\abovedisplayskip=10pt
	\belowdisplayskip=10pt
	\begin{align}
		\mathcal F \left[ B^* \right] &= \left\{ \mathcal F \left[ B \right] \Big|_{- \symbf{k}_{\symup{\rho}}} \right\}^* = \mathcal F^* \left[ B \right] \Big|_{- \symbf{k}_{\symup{\rho}}} \label{eq:2-197a}~, \\ \mathcal F \left[ A \cdot B^* \right] &= \mathcal F \left[ A \right] * \mathcal F^* \left[ B \right] \Big|_{- \symbf{k}_{\symup{\rho}}} \label{eq:2-197b} \\ &= \mathcal F^* \left[ B \right] \Big|_{- \symbf{k}_{\symup{\rho}}} * \mathcal F \left[ A \right] =: \mathcal F \left[ B \right] \circ \mathcal F \left[ A \right] \label{eq:2-197c}~,
	\end{align}
\end{subequations}
其中,以 $\symbf{k}_{\symup{\rho}}$ 域为例,定义了互相关这个二元/双目算符
\begin{subequations} \label{eq:2-198}
	\abovedisplayskip=10pt
	\belowdisplayskip=10pt
	\begin{align}
		A \circ B := A^* \left( - \symbf{k}_{\symup{\rho}} \right) * B &= \iint_{-\infty}^{+\infty} A^* \left( \symbf{k}'_{\symup{\rho}} - \symbf{k}_{\symup{\rho}} \right) \cdot B \left( \symbf{k}'_{\symup{\rho}} \right) {\mathbb{d}{k'_{\symup{x}}} \mathbb{d}{k'_{\symup{y}}}} \label{eq:2-198a} \\ &= \iint_{-\infty}^{+\infty} A^* \left( \symbf{k}'_{\symup{\rho}} \right) \cdot B \left( \symbf{k}'_{\symup{\rho}} + \symbf{k}_{\symup{\rho}} \right) {\mathbb{d}{k'_{\symup{x}}} \mathbb{d}{k'_{\symup{y}}}} \label{eq:2-198b}~, 
	\end{align}
\end{subequations}
不像卷积,互相关不满足交换律:
\begin{subequations} \label{eq:2-199}
	\abovedisplayskip=10pt
	\belowdisplayskip=10pt
	\begin{align}
		A \circ B &= A^* \left( - \symbf{k}_{\symup{\rho}} \right) * B \label{eq:2-199a} \\ &= \left[ A * B^* \left( - \symbf{k}_{\symup{\rho}} \right) \right]^* \left( - \symbf{k}_{\symup{\rho}} \right) \label{eq:2-199b} \\ &= \left[ B^* \left( - \symbf{k}_{\symup{\rho}} \right) * A \right]^* \left( - \symbf{k}_{\symup{\rho}} \right) = \left[ B \circ A \right]^* \left( - \symbf{k}_{\symup{\rho}} \right) \label{eq:2-199c}~, 
	\end{align}
\end{subequations}
互相关也不满足结合律
\begin{subequations} \label{eq:2-200}
	\abovedisplayskip=10pt
	\belowdisplayskip=10pt
	\begin{align}
		\left( A \circ B \right) \circ C &= \left( A \circ B \right)^* \left( - \symbf{k}_{\symup{\rho}} \right) * C \label{eq:2-200a} \\ &= B \circ A * C \label{eq:2-200b} \\ &= B^* \left( - \symbf{k}_{\symup{\rho}} \right) * A * C \label{eq:2-200c} \\ &= A * B^* \left( - \symbf{k}_{\symup{\rho}} \right) * C \label{eq:2-200d} \\ &\neq A^* \left( - \symbf{k}_{\symup{\rho}} \right) * B^* \left( - \symbf{k}_{\symup{\rho}} \right) * C \label{eq:2-200e} \\ &= \left( A * B \right) \circ C \label{eq:2-200f} \\ &= A^* \left( - \symbf{k}_{\symup{\rho}} \right) * \left( B \circ C \right) = A \circ \left( B \circ C \right) \label{eq:2-200g}~. 
	\end{align}
\end{subequations}

利用 Eq(\ref{eq:2-197a}) 的规则,Eq(\ref{eq:2-195f}) 中的 $\mathcal F_{\symup{3D}} \left[ \bar{\underline{\chi}}^{(2)\omega*}_{ {\symup{\mu}}_{32}z} \right]$ 也可以写成类似 Eq(\ref{eq:2-196a}) 的形式,但倒格波的频率 $\omega$ 不再取负,因为这里不含 $\omega$ 域卷积 $"\ \widetilde *\ "$ 或互相关 $"\ \widetilde \circ\ "$,只有 $\symbf{k}_{\symup{\rho}}$ 域卷积 $"*"$ 或互相关 $"\circ"$:
\begin{subequations} \label{eq:2-201}
	\abovedisplayskip=13pt
	\belowdisplayskip=13pt
	\begin{align}
		\mathcal F_{\symup{3D}} \left[ \bar{\underline{\chi}}^{(2)\omega*}_{ {\symup{\mu}}_{32}z} \right] &= \bar{\underline{\chi}}^{(2)\omega*}_{{\symup{\mu}}_{32}} \odot \bar{\underline{C}}^{\omega*}_{{\symup{\mu}}_{32}} \left( - \symbf{k}_{\symup{q}} \right) = \bar{\underline{\chi}}^{(2)\omega*}_{{\symup{\mu}}_{32}} \odot \bar{\underline{\symup{C}}}^{\omega*}_{{\symup{\mu}}_{32}} C^{\omega*} \left( - \symbf{k}_{\symup{q}} \right) \label{eq:2-201a}~, \\ \text{where}\ \ \ \ \ \ \bar{\underline{\chi}}^{(2)\omega*}_{{\symup{\mu}}_{32}z} &= \bar{\underline{\chi}}^{(2)\omega*}_{{\symup{\mu}}_{32}} \odot \bar{\underline{M}}^{\omega*}_{{\symup{\mu}}_{32}z} = \bar{\underline{\chi}}^{(2)\omega*}_{{\symup{\mu}}_{32}} \odot \bar{\underline{\symup{M}}}^{\omega*}_{{\symup{\mu}}_{32}} M^{\omega*}_{z} \label{eq:2-201b}~,
	\end{align}
\end{subequations}
其中 $\symbf{k}_{\symup{q}} = \symbf{k}_{\symup{\rho}} + \symbf{q}_{\symup{z}}$ 与时间频率 $\omega$ 无关,且横向空间频率 $\symbf{k}_{\symup{\rho}}$ 暂因卷积而与混频光场共用,但纵向空间频率 $\symbf{q}_{\symup{z}}$ 是结构自身的,独立于泵浦;于是 Eq(\ref{eq:2-195f}) 变为
\begin{subequations} \label{eq:2-202}
	\abovedisplayskip=10pt
	\belowdisplayskip=10pt
	\begin{align}
		\bar{\underline{Q}}^{(2)\omega}_{z} &= \bar{\underline{\chi}}^{(2)\omega*}_{{\symup{\mu}}_{32}} \odot \bar{\underline{\symup{C}}}^{\omega*}_{{\symup{\mu}}_{32}} \mathcal F^{-1}_{z} \left[ C^{\omega*} \left( - \symbf{k}_{\symup{q}} \right) * \underline{G}^{\omega}_{{\symup{\mu}}_2 z} \widetilde \circledcirc\ \underline{G}^{\omega}_{{\symup{\mu}}_3 z} \right] \label{eq:2-202a} \\ &= \bar{\underline{\chi}}^{(2)\omega*}_{{\symup{\mu}}_{32}} \odot \bar{\underline{\symup{C}}}^{\omega*}_{{\symup{\mu}}_{32}} \mathcal F^{-*}_{z} \left[ C^{\omega} \circ \underline{G}^{\omega}_{{\symup{\mu}}_2 z} \widetilde \circledcirc\ \underline{G}^{\omega}_{{\symup{\mu}}_3 z} \right] \label{eq:2-202b}~,
	\end{align}
\end{subequations}
其中,使用了 Eq(\ref{eq:2-198}),并且定义了核函数 ${\mathbb{e}^{-\mathbb{i} q_{\symup{z}} \cdot z}}$ 共轭于 $\mathcal F^{-1}_{z}$ 的核函数 ${\mathbb{e}^{\mathbb{i} q_{\symup{z}} \cdot z}}$ 的空域 $z$ 向一维傅立叶逆变换:
\begin{equation} \label{eq:2-203}
	\abovedisplayskip=13pt
	\belowdisplayskip=13pt
	\mathcal F^{-*}_{z} \left[ \cdot \right] := \iint_{-\infty}^{+\infty} \cdot\  {\mathbb{e}^{-\mathbb{i} q_{\symup{z}} \cdot z}} ~ {\mathbb{d}{q_{\symup{z}}}} ~.
\end{equation}

注意,规定 Eq(\ref{eq:2-202b}) 中的运算符在 $\symbf{k}_{\symup{\rho}}$ 域上的优先级 $"\widetilde \circledcirc" > "\circ"$,以省略对 $"\widetilde \circledcirc"$ 及其作用对象的括号;但在 $\omega$ 域上 $"\widetilde \circledcirc" < "\circ"$,因此有 $"\widetilde \circledcirc"$ 的 $\symbf{k}_{\symup{\rho}}$ 域 $>$ $"\circ"$ 的 $\symbf{k}_{\symup{\rho}}$ 域 $>$ $q_{\symup{z}}$ 域的 $F^{-*}_{z} \left[ \cdot \right]$ $>$ $"\widetilde \circledcirc"$ 的 $\omega$ 域(即 $\omega$ 域的 $"\ \widetilde \circ\ "$)。此外,之后我们会提到,Eq(\ref{eq:2-202b}) 中的 $"\widetilde \circledcirc"$ 中的 $"\ \widetilde \circ\ "$ 和 $"\circ"$,二者均会用到 2 条互相关运算规则 Eq(\ref{eq:2-198a}) 和 Eq(\ref{eq:2-198b}),并且在不同的情况下使用不同的规则,这一点对于理解下转换版本的非线性卷积过程的横纵波矢匹配以及频率守恒方程至关重要。

接着,Eq(\ref{eq:2-202b}) 继续写作
\begin{subequations} \label{eq:2-204}
	\abovedisplayskip=13pt
	\belowdisplayskip=13pt
	\begin{align}
		\bar{\underline{Q}}^{(2)\omega}_{z} \asymp \bar{\underline{Q}}^{(2)\omega{\symup{p}}_{32}}_{z} &= \bar{\underline{\symup{C}}}^{\omega*}_{{\symup{\mu}}_{32}} \odot \bar{\underline{\chi}}^{(2)\omega*}_{{\symup{\mu}}_{32}} \mathcal F^{-*}_{z} \left[ C^{\omega} \circ \left( \hat{\underline{g}}^{\omega{\symup{p}}_{2}}_{{\symup{\mu}}_2} {\mathtt{G}}^{\omega{\symup{p}}_{2}}_{z} \right) \widetilde \circledcirc \left( \hat{\underline{g}}^{\omega{\symup{p}}_{3}}_{{\symup{\mu}}_3} {\mathtt{G}}^{\omega{\symup{p}}_{3}}_{z} \right) \right] \label{eq:2-204a} \\ &= \bar{\underline{\symup{C}}}^{\omega*}_{{\symup{\mu}}_{32}} \odot \bar{\underline{\chi}}^{(2)\omega*}_{{\symup{\mu}}_{32}} \hat{\underline{g}}^{\omega{\symup{p}}_{2}}_{{\symup{\mu}}_2}\ \widetilde \circ\ \hat{\underline{g}}^{\omega{\symup{p}}_{3}}_{{\symup{\mu}}_3} \mathcal F^{-*}_{z} \left[ C^{\omega} \circ {\mathtt{G}}^{\omega{\symup{p}}_{2}}_{z} \widetilde \circledcirc\ {\mathtt{G}}^{\omega{\symup{p}}_{3}}_{z} \right] \label{eq:2-204b}~,
	\end{align}
\end{subequations}
其中,$"\asymp"$ 符号表示“数学表达式相同”(但一般含义不同)。注意,这里的 Eq.(\ref{eq:2-204a}) $\to$ Eq.(\ref{eq:2-204b}) 也使用了第 \pageref{con:3} 页的“标量非线性波源”条件。

相应的脉冲光整流时空谱耦合波方程变为
\begin{equation} \label{eq:2-205}
	\abovedisplayskip=13pt
	\belowdisplayskip=13pt
	\displaystyle{\frac{\partial {\mathtt{g}}^{\omega\symup{p}_{1}}_{z}}{\partial z}} \asymp \displaystyle{\frac{\partial {\mathtt{g}}^{\omega\symup{p}_{132}}_{z}}{\partial z}} = \mathbb{i} k^{2}_{0\omega} \frac{ \widetilde{\chi}^{(2)\omega\symup{p}_{132}}_{\symup{eff}} \mathcal F^{-*}_{z} \left[ C^{\omega} \circ {\mathtt{G}}^{\omega{\symup{p}}_{2}}_{z} \widetilde \circledcirc\ {\mathtt{G}}^{\omega{\symup{p}}_{3}}_{z} \right] }{ \left| \hat{\underline{g}}^{\omega\symup{p}_{1}} \right|^2 2 k^{\omega\symup{p}_{1}}_{\symup{z}} \mathbb{e}^{\mathbb{i} k^{\omega\symup{p}_{1}}_{\symup{z}} z} } ~,
\end{equation}
其中,定义了第 \pageref{con:3} 页“标量非线性波源”条件下的脉冲光整流过程的有效非线性系数
\begin{subequations} \label{eq:2-206}
	\abovedisplayskip=13pt
	\belowdisplayskip=13pt
	\begin{align}
		\widetilde{\chi}^{(2)\omega\symup{p}_{132}}_{\symup{eff}} &:= \hat{\underline{g}}^{\omega\symup{p}_{1}*} \odot \bar{\underline{\symup{C}}}^{\omega*}_{{\symup{\mu}}_{32}} \odot \bar{\underline{\chi}}^{(2)\omega*}_{{\symup{\mu}}_{32}} \hat{\underline{g}}^{\omega{\symup{p}}_{2}}_{{\symup{\mu}}_2}\ \widetilde \circ\ \hat{\underline{g}}^{\omega{\symup{p}}_{3}}_{{\symup{\mu}}_3} \label{eq:2-206a} \\ &= \hat{\underline{g}}^{\omega\symup{p}_{1}*}_{{\symup{\mu}}_{1}} \underline{\symup{C}}^{\omega*}_{{\symup{\mu}}_{132}} \underline{\chi}^{(2)\omega*}_{{\symup{\mu}}_{132}} \hat{\underline{g}}^{\omega\symup{p}_{2}}_{{\symup{\mu}}_2} \ \widetilde \circ\ \hat{\underline{g}}^{\omega\symup{p}_{3}}_{{\symup{\mu}}_3} \label{eq:2-206b}~.
	\end{align}
\end{subequations}

对比脉冲光整流过程的 \cref{eq:2-204,eq:2-205,eq:2-206} 以及脉冲光倍频过程的 \cref{eq:2-191,eq:2-192,eq:2-193,eq:2-194},可以发现一些卷积(上转换) → 互相关(下转换)的符号替换规律,将该规律应用到离散波长之三波混频上转换过程的相关表达式 \cref{eq:2-182,eq:2-183,eq:2-184,eq:2-185,eq:2-186,eq:2-187,eq:2-188},即可得到第 \pageref{con:3} 页的“标量非线性波源”条件下的三波混频下转换过程的时空谱耦合波方程:
\begin{subequations} \label{eq:2-207}
	\abovedisplayskip=13pt
	\belowdisplayskip=13pt
	\begin{align}
		\displaystyle{\frac{\partial {\mathtt{g}}^{\symup{p}_{1}}_{1z}}{\partial z}} \asymp \displaystyle{\frac{\partial {\mathtt{g}}^{\symup{p}_{132}}_{1z}}{\partial z}} &= \mathbb{i} k^{2}_{01} \frac{ {\chi}^{(2)\symup{p}_{132}}_{1\symup{eff}} \mathcal F^{-*}_{z} \left[ C_1 \circ \left( {\mathtt{G}}^{{\symup{p}}_{2}}_{2z} \circ {\mathtt{G}}^{{\symup{p}}_{3}}_{3z} \right) \right] }{ \left| \hat{\underline{g}}^{\symup{p}_{1}}_1 \right|^2 2 k^{\symup{p}_{1}}_{1\symup{z}} \mathbb{e}^{\mathbb{i} k^{\symup{p}_{1}}_{1\symup{z}} z} } \label{eq:2-207a}~, \\ {\chi}^{(2)\symup{p}_{132}}_{1\symup{eff}} &:= \hat{\underline{g}}^{\symup{p}_{1}*}_1 \odot \bar{\underline{\symup{C}}}^{*}_{1{\symup{\mu}}_{32}} \odot \bar{\underline{\chi}}^{(2)*}_{1{\symup{\mu}}_{32}} \hat{\underline{g}}^{{\symup{p}}_{3}}_{3{\symup{\mu}}_3} \hat{\underline{g}}^{{\symup{p}}_{2}*}_{2{\symup{\mu}}_2} \label{eq:2-207b} \\ &= \hat{\underline{g}}^{\symup{p}_{1}*}_{1{\symup{\mu}}_{1}} \underline{\symup{C}}^{*}_{1{\symup{\mu}}_{132}} \underline{\chi}^{(2)*}_{1{\symup{\mu}}_{132}} \hat{\underline{g}}^{{\symup{p}}_{3}}_{3{\symup{\mu}}_3} \hat{\underline{g}}^{{\symup{p}}_{2}*}_{2{\symup{\mu}}_2} \label{eq:2-207c}~,
	\end{align}
\end{subequations}
注意,由于互相关不满足结合律,因此需指定 Eq(\ref{eq:2-207a}) 中的两个空域互相关的计算顺序,以至于需要括号,并且符合 Eq(\ref{eq:2-200g}) 的结合顺序。另外,也可以将 \cref{eq:2-191,eq:2-192,eq:2-193,eq:2-194} 到 \cref{eq:2-182,eq:2-183,eq:2-184,eq:2-185,eq:2-186,eq:2-187,eq:2-188} \cref{eq:2-191,eq:2-192,eq:2-193,eq:2-194} 的符号替换规律,作用于 \cref{eq:2-204,eq:2-205,eq:2-206} 以得到 Eq(\ref{eq:2-207})。

对于离散波长的情况,将和频过程对应的单个 \cref{eq:2-187,eq:2-188} 与差频过程对应的 2 个 Eq(\ref{eq:2-207}) 组合到一起,即得三波混频时空谱耦合波方程组
\begin{subequations} \label{eq:2-208}
	\abovedisplayskip=13pt
	\belowdisplayskip=13pt
	\begin{align}
		\displaystyle{\frac{\partial {\mathtt{g}}^{\symup{p}_{3}}_{3z}}{\partial z}} \asymp \displaystyle{\frac{\partial {\mathtt{g}}^{\symup{p}_{312}}_{3z}}{\partial z}} &= \mathbb{i} k^{2}_{03} \frac{ \chi^{(2)\symup{p}_{312}}_{3\symup{eff}} \mathcal F^{-1}_{z} \left[ C_{3} * {\mathtt{G}}^{\symup{p}_{1}}_{1z} * {\mathtt{G}}^{\symup{p}_{2}}_{2z} \right] }{ \left| \hat{\underline{g}}^{\symup{p}_{3}}_{3} \right|^2 2 k^{\symup{p}_{3}}_{3\symup{z}} \mathbb{e}^{\mathbb{i} k^{\symup{p}_{3}}_{3\symup{z}} z} } \label{eq:2-208a}~, \\ \displaystyle{\frac{\partial {\mathtt{g}}^{\symup{p}_{1}}_{1z}}{\partial z}} \asymp \displaystyle{\frac{\partial {\mathtt{g}}^{\symup{p}_{132}}_{1z}}{\partial z}} &= \mathbb{i} k^{2}_{01} \frac{ {\chi}^{(2)\symup{p}_{132}}_{1\symup{eff}} \mathcal F^{-*}_{z} \left[ C_1 \circ \left( {\mathtt{G}}^{{\symup{p}}_{2}}_{2z} \circ {\mathtt{G}}^{{\symup{p}}_{3}}_{3z} \right) \right] }{ \left| \hat{\underline{g}}^{\symup{p}_{1}}_1 \right|^2 2 k^{\symup{p}_{1}}_{1\symup{z}} \mathbb{e}^{\mathbb{i} k^{\symup{p}_{1}}_{1\symup{z}} z} } \label{eq:2-208b}~, \\ \displaystyle{\frac{\partial {\mathtt{g}}^{\symup{p}_{2}}_{2z}}{\partial z}} \asymp \displaystyle{\frac{\partial {\mathtt{g}}^{\symup{p}_{231}}_{2z}}{\partial z}} &= \mathbb{i} k^{2}_{02} \frac{ {\chi}^{(2)\symup{p}_{231}}_{2\symup{eff}} \mathcal F^{-*}_{z} \left[ C_2 \circ \left( {\mathtt{G}}^{{\symup{p}}_{1}}_{1z} \circ {\mathtt{G}}^{{\symup{p}}_{3}}_{3z} \right) \right] }{ \left| \hat{\underline{g}}^{\symup{p}_{2}}_2 \right|^2 2 k^{\symup{p}_{2}}_{2\symup{z}} \mathbb{e}^{\mathbb{i} k^{\symup{p}_{2}}_{2\symup{z}} z} } \label{eq:2-208c}~,
	\end{align}
\end{subequations}
其中,定义了第 \pageref{con:3} 页“标量非线性波源”条件下的各波段的有效非线性系数
\begin{subequations} \label{eq:2-209}
	\abovedisplayskip=13pt
	\belowdisplayskip=13pt
	\begin{align}
		\chi^{(2)\symup{p}_{312}}_{3\symup{eff}} &:= \hat{\underline{g}}^{\symup{p}_{3}*}_{3{{\symup{\mu}}_{3}}} \symup{C}_{3{\symup{\mu}}_{312}} \underline{\chi}^{(2)}_{3{\symup{\mu}}_{312}} \hat{\underline{g}}^{\symup{p}_{1}}_{1{\symup{\mu}}_1} \hat{\underline{g}}^{\symup{p}_{2}}_{2{\symup{\mu}}_2} \label{eq:2-209a}~, \\ {\chi}^{(2)\symup{p}_{132}}_{1\symup{eff}} &:= \hat{\underline{g}}^{\symup{p}_{1}*}_{1{\symup{\mu}}_{1}} \underline{\symup{C}}^{*}_{1{\symup{\mu}}_{132}} \underline{\chi}^{(2)*}_{1{\symup{\mu}}_{132}} \hat{\underline{g}}^{\symup{p}_{3}}_{3{\symup{\mu}}_3} \hat{\underline{g}}^{\symup{p}_{2}*}_{2{\symup{\mu}}_2} \label{eq:2-209b}~, \\ {\chi}^{(2)\symup{p}_{231}}_{2\symup{eff}} &:= \hat{\underline{g}}^{\symup{p}_{2}*}_{2{\symup{\mu}}_{2}} \underline{\symup{C}}^{*}_{2{\symup{\mu}}_{231}} \underline{\chi}^{(2)*}_{2{\symup{\mu}}_{231}} \hat{\underline{g}}^{\symup{p}_{3}}_{3{\symup{\mu}}_3} \hat{\underline{g}}^{\symup{p}_{1}*}_{1{\symup{\mu}}_1} \label{eq:2-209c}~.
	\end{align}
\end{subequations}

这里给出三个简化条件\label{con:4}:其一,假设调制函数 $\bar{\bar{\underline{M}}}^{\omega}_z$ 和倒格波系数 $\bar{\bar{\underline{C}}}^{\omega} := \mathcal F_{\symup{3D}} \left[ \bar{\bar{\underline{M}}}^{\omega}_z \right]$ 均与频率 $\omega$ 无关,但 $\bar{\bar{\underline{\chi}}}^{(2)\omega}$ 仍与频率 $\omega$ 有关;其二,省略右下角标中表示频率 $\omega_i$ 的数字 $i$;其三,在 $\symbf{k}_{\symup{\rho}}$ 域,默认先计算光场间的互相关,再计算与倒格波系数的互相关,或从右到左计算两个互相关,以省略括号;在这三点规则下,三波混频的时空谱耦合波方程组简写为
\begin{subequations} \label{eq:2-210}
	\abovedisplayskip=13pt
	\belowdisplayskip=13pt
	\begin{align}
		\displaystyle{\frac{\partial {\mathtt{g}}^{\symup{p}_{3}}_{z}}{\partial z}} \asymp \displaystyle{\frac{\partial {\mathtt{g}}^{\symup{p}_{312}}_{z}}{\partial z}} &= \mathbb{i} k^{2}_{03} \frac{ \chi^{(2)\symup{p}_{312}}_{\symup{eff}} \mathcal F^{-1}_{z} \left[ C * {\mathtt{G}}^{\symup{p}_{1}}_{z} * {\mathtt{G}}^{\symup{p}_{2}}_{z} \right] }{ \left| \hat{\underline{g}}^{\symup{p}_{3}} \right|^2 2 k^{\symup{p}_{3}}_{\symup{z}} \mathbb{e}^{\mathbb{i} k^{\symup{p}_{3}}_{\symup{z}} z} } \label{eq:2-210a}~, \\ \displaystyle{\frac{\partial {\mathtt{g}}^{\symup{p}_{1}}_{z}}{\partial z}} \asymp \displaystyle{\frac{\partial {\mathtt{g}}^{\symup{p}_{132}}_{z}}{\partial z}} &= \mathbb{i} k^{2}_{01} \frac{ {\chi}^{(2)\symup{p}_{132}}_{\symup{eff}} \mathcal F^{-*}_{z} \left[ C \circ  {\mathtt{G}}^{{\symup{p}}_{2}}_{z} \circ {\mathtt{G}}^{{\symup{p}}_{3}}_{z} \right] }{ \left| \hat{\underline{g}}^{\symup{p}_{1}} \right|^2 2 k^{\symup{p}_{1}}_{\symup{z}} \mathbb{e}^{\mathbb{i} k^{\symup{p}_{1}}_{\symup{z}} z} } \label{eq:2-210b}~, \\ \displaystyle{\frac{\partial {\mathtt{g}}^{\symup{p}_{2}}_{z}}{\partial z}} \asymp \displaystyle{\frac{\partial {\mathtt{g}}^{\symup{p}_{231}}_{z}}{\partial z}} &= \mathbb{i} k^{2}_{02} \frac{ {\chi}^{(2)\symup{p}_{231}}_{\symup{eff}} \mathcal F^{-*}_{z} \left[ C \circ {\mathtt{G}}^{{\symup{p}}_{1}}_{z} \circ {\mathtt{G}}^{{\symup{p}}_{3}}_{z} \right] }{ \left| \hat{\underline{g}}^{\symup{p}_{2}} \right|^2 2 k^{\symup{p}_{2}}_{\symup{z}} \mathbb{e}^{\mathbb{i} k^{\symup{p}_{2}}_{\symup{z}} z} } \label{eq:2-210c}~,
	\end{align}
\end{subequations}
其中,定义了第 \pageref{con:3} 页“标量非线性波源”条件下的各波段的有效非线性系数
\begin{subequations} \label{eq:2-211}
	\abovedisplayskip=13pt
	\belowdisplayskip=13pt
	\begin{align}
		\chi^{(2)\symup{p}_{312}}_{\symup{eff}} &:= \hat{\underline{g}}^{\symup{p}_{3}*}_{{{\symup{\mu}}_{3}}} \symup{C}_{{\symup{\mu}}_{312}} \underline{\chi}^{(2)}_{{\symup{\mu}}_{312}} \hat{\underline{g}}^{\symup{p}_{1}}_{{\symup{\mu}}_1} \hat{\underline{g}}^{\symup{p}_{2}}_{{\symup{\mu}}_2} \label{eq:2-211a}~, \\ {\chi}^{(2)\symup{p}_{132}}_{\symup{eff}} &:= \hat{\underline{g}}^{\symup{p}_{1}*}_{{\symup{\mu}}_{1}} \underline{\symup{C}}^{*}_{{\symup{\mu}}_{132}} \underline{\chi}^{(2)*}_{{\symup{\mu}}_{132}} \hat{\underline{g}}^{\symup{p}_{3}}_{{\symup{\mu}}_3} \hat{\underline{g}}^{\symup{p}_{2}*}_{{\symup{\mu}}_2} \label{eq:2-211b}~, \\ {\chi}^{(2)\symup{p}_{231}}_{\symup{eff}} &:= \hat{\underline{g}}^{\symup{p}_{2}*}_{{\symup{\mu}}_{2}} \underline{\symup{C}}^{*}_{{\symup{\mu}}_{231}} \underline{\chi}^{(2)*}_{{\symup{\mu}}_{231}} \hat{\underline{g}}^{\symup{p}_{3}}_{{\symup{\mu}}_3} \hat{\underline{g}}^{\symup{p}_{1}*}_{{\symup{\mu}}_1} \label{eq:2-211c}~.
	\end{align}
\end{subequations}

同样,对于连续谱/脉冲光的情况,将脉冲光谱内上转换对应的单个 \cref{eq:2-192,eq:2-193,eq:2-194} 与脉冲光整流对应的 2 个(或 1 个) Eq(\ref{eq:2-205}) 组合到一起,即得脉冲光整流及其后续级联电光效应组成的时空谱耦合波方程组
\begin{subequations} \label{eq:2-212}
	\abovedisplayskip=13pt
	\belowdisplayskip=13pt
	\begin{align}
		\displaystyle{\frac{\partial {\mathtt{g}}^{\omega\symup{p}_{3}}_{z}}{\partial z}} \asymp \displaystyle{\frac{\partial {\mathtt{g}}^{\omega\symup{p}_{312}}_{z}}{\partial z}} &= \mathbb{i} k^{2}_{0\omega} \frac{ \widetilde{\chi}^{(2)\omega\symup{p}_{312}}_{\symup{eff}} \mathcal F^{-1}_{z} \left[ C^{\omega} * {\mathtt{G}}^{\omega{\symup{p}}_{1}}_{z} \widetilde \circledast\ {\mathtt{G}}^{\omega{\symup{p}}_{2}}_{z} \right] }{ \left| \hat{\underline{g}}^{\omega\symup{p}_{3}} \right|^2 2 k^{\omega\symup{p}_{3}}_{\symup{z}} \mathbb{e}^{\mathbb{i} k^{\omega\symup{p}_{3}}_{\symup{z}} z} } \label{eq:2-212a}~, \\ \displaystyle{\frac{\partial {\mathtt{g}}^{\omega\symup{p}_{1}}_{z}}{\partial z}} \asymp \displaystyle{\frac{\partial {\mathtt{g}}^{\omega\symup{p}_{132}}_{z}}{\partial z}} &= \mathbb{i} k^{2}_{0\omega} \frac{ \widetilde{\chi}^{(2)\omega\symup{p}_{132}}_{\symup{eff}} \mathcal F^{-*}_{z} \left[ C^{\omega} \circ {\mathtt{G}}^{\omega{\symup{p}}_{2}}_{z} \widetilde \circledcirc\ {\mathtt{G}}^{\omega{\symup{p}}_{3}}_{z} \right] }{ \left| \hat{\underline{g}}^{\omega\symup{p}_{1}} \right|^2 2 k^{\omega\symup{p}_{1}}_{\symup{z}} \mathbb{e}^{\mathbb{i} k^{\omega\symup{p}_{1}}_{\symup{z}} z} } \label{eq:2-212b}~, \\ \displaystyle{\frac{\partial {\mathtt{g}}^{\omega\symup{p}_{2}}_{z}}{\partial z}} \asymp \displaystyle{\frac{\partial {\mathtt{g}}^{\omega\symup{p}_{231}}_{z}}{\partial z}} &= \mathbb{i} k^{2}_{0\omega} \frac{ \widetilde{\chi}^{(2)\omega\symup{p}_{231}}_{\symup{eff}} \mathcal F^{-*}_{z} \left[ C^{\omega} \circ {\mathtt{G}}^{\omega{\symup{p}}_{1}}_{z} \widetilde \circledcirc\ {\mathtt{G}}^{\omega{\symup{p}}_{3}}_{z} \right] }{ \left| \hat{\underline{g}}^{\omega\symup{p}_{2}} \right|^2 2 k^{\omega\symup{p}_{2}}_{\symup{z}} \mathbb{e}^{\mathbb{i} k^{\omega\symup{p}_{2}}_{\symup{z}} z} } \label{eq:2-212c}~,
	\end{align}
\end{subequations}
其中,定义了第 \pageref{con:3} 页“标量非线性波源”条件下的各波段的有效非线性系数
\begin{subequations} \label{eq:2-213}
	\abovedisplayskip=13pt
	\belowdisplayskip=13pt
	\begin{align}
		\widetilde{\chi}^{(2)\omega\symup{p}_{312}}_{\symup{eff}} &:= \hat{\underline{g}}^{\omega\symup{p}_{3}*}_{{\symup{\mu}}_{3}} \underline{\symup{C}}^{\omega}_{{\symup{\mu}}_{312}} \underline{\chi}^{(2)\omega}_{{\symup{\mu}}_{312}} \hat{\underline{g}}^{\omega\symup{p}_{1}}_{{\symup{\mu}}_1} \ \widetilde *\ \hat{\underline{g}}^{\omega\symup{p}_{2}}_{{\symup{\mu}}_2} \label{eq:2-213a}~, \\ \widetilde{\chi}^{(2)\omega\symup{p}_{132}}_{\symup{eff}} &:= \hat{\underline{g}}^{\omega\symup{p}_{1}*}_{{\symup{\mu}}_{1}} \underline{\symup{C}}^{\omega*}_{{\symup{\mu}}_{132}} \underline{\chi}^{(2)\omega*}_{{\symup{\mu}}_{132}} \hat{\underline{g}}^{\omega\symup{p}_{2}}_{{\symup{\mu}}_2} \ \widetilde \circ\ \hat{\underline{g}}^{\omega\symup{p}_{3}}_{{\symup{\mu}}_3} \label{eq:2-213b}~, \\ \widetilde{\chi}^{(2)\omega\symup{p}_{231}}_{\symup{eff}} &:= \hat{\underline{g}}^{\omega\symup{p}_{2}*}_{{\symup{\mu}}_{2}} \underline{\symup{C}}^{\omega*}_{{\symup{\mu}}_{231}} \underline{\chi}^{(2)\omega*}_{{\symup{\mu}}_{231}} \hat{\underline{g}}^{\omega\symup{p}_{1}}_{{\symup{\mu}}_1} \ \widetilde \circ\ \hat{\underline{g}}^{\omega\symup{p}_{3}}_{{\symup{\mu}}_3} \label{eq:2-213c}~.
	\end{align}
\end{subequations}

同样,上述脉冲光整流时空谱耦合波方程组,也可以利用第 \pageref{con:4} 页中前述三个规则,简写为:
\begin{subequations} \label{eq:2-214}
	\abovedisplayskip=13pt
	\belowdisplayskip=13pt
	\begin{align}
		\displaystyle{\frac{\partial {\mathtt{g}}^{\symup{p}_{3}}_{z}}{\partial z}} \asymp \displaystyle{\frac{\partial {\mathtt{g}}^{\symup{p}_{312}}_{z}}{\partial z}} &= \mathbb{i} k^{2}_{0} \frac{ \widetilde{\chi}^{(2)\symup{p}_{312}}_{\symup{eff}} \mathcal F^{-1}_{z} \left[ C^{} * {\mathtt{G}}^{{\symup{p}}_{1}}_{z} \widetilde \circledast\ {\mathtt{G}}^{{\symup{p}}_{2}}_{z} \right] }{ \left| \hat{\underline{g}}^{\symup{p}_{3}} \right|^2 2 k^{\symup{p}_{3}}_{\symup{z}} \mathbb{e}^{\mathbb{i} k^{\symup{p}_{3}}_{\symup{z}} z} } \label{eq:2-214a}~, \\ \displaystyle{\frac{\partial {\mathtt{g}}^{\symup{p}_{1}}_{z}}{\partial z}} \asymp \displaystyle{\frac{\partial {\mathtt{g}}^{\symup{p}_{132}}_{z}}{\partial z}} &= \mathbb{i} k^{2}_{0} \frac{ \widetilde{\chi}^{(2)\symup{p}_{132}}_{\symup{eff}} \mathcal F^{-*}_{z} \left[ C^{} \circ {\mathtt{G}}^{{\symup{p}}_{2}}_{z} \widetilde \circledcirc\ {\mathtt{G}}^{{\symup{p}}_{3}}_{z} \right] }{ \left| \hat{\underline{g}}^{\symup{p}_{1}} \right|^2 2 k^{\symup{p}_{1}}_{\symup{z}} \mathbb{e}^{\mathbb{i} k^{\symup{p}_{1}}_{\symup{z}} z} } \label{eq:2-214b}~, \\ \displaystyle{\frac{\partial {\mathtt{g}}^{\symup{p}_{2}}_{z}}{\partial z}} \asymp \displaystyle{\frac{\partial {\mathtt{g}}^{\symup{p}_{231}}_{z}}{\partial z}} &= \mathbb{i} k^{2}_{0} \frac{ \widetilde{\chi}^{(2)\symup{p}_{231}}_{\symup{eff}} \mathcal F^{-*}_{z} \left[ C^{} \circ {\mathtt{G}}^{{\symup{p}}_{1}}_{z} \widetilde \circledcirc\ {\mathtt{G}}^{{\symup{p}}_{3}}_{z} \right] }{ \left| \hat{\underline{g}}^{\symup{p}_{2}} \right|^2 2 k^{\symup{p}_{2}}_{\symup{z}} \mathbb{e}^{\mathbb{i} k^{\symup{p}_{2}}_{\symup{z}} z} } \label{eq:2-214c}~,
	\end{align}
\end{subequations}
其中,定义了第 \pageref{con:3} 页“标量非线性波源”条件下的各波段的有效非线性系数
\begin{subequations} \label{eq:2-215}
	\abovedisplayskip=13pt
	\belowdisplayskip=13pt
	\begin{align}
		\widetilde{\chi}^{(2)\symup{p}_{312}}_{\symup{eff}} &:= \hat{\underline{g}}^{\symup{p}_{3}*}_{{\symup{\mu}}_{3}} \underline{\symup{C}}^{}_{{\symup{\mu}}_{312}} \underline{\chi}^{(2)}_{{\symup{\mu}}_{312}} \hat{\underline{g}}^{\symup{p}_{1}}_{{\symup{\mu}}_1} \ \widetilde *\ \hat{\underline{g}}^{\symup{p}_{2}}_{{\symup{\mu}}_2} \label{eq:2-215a}~, \\ \widetilde{\chi}^{(2)\symup{p}_{132}}_{\symup{eff}} &:= \hat{\underline{g}}^{\symup{p}_{1}*}_{{\symup{\mu}}_{1}} \underline{\symup{C}}^{*}_{{\symup{\mu}}_{132}} \underline{\chi}^{(2)*}_{{\symup{\mu}}_{132}} \hat{\underline{g}}^{\symup{p}_{2}}_{{\symup{\mu}}_2} \ \widetilde \circ\ \hat{\underline{g}}^{\symup{p}_{3}}_{{\symup{\mu}}_3} \label{eq:2-215b}~, \\ \widetilde{\chi}^{(2)\symup{p}_{231}}_{\symup{eff}} &:= \hat{\underline{g}}^{\symup{p}_{2}*}_{{\symup{\mu}}_{2}} \underline{\symup{C}}^{*}_{{\symup{\mu}}_{231}} \underline{\chi}^{(2)*}_{{\symup{\mu}}_{231}} \hat{\underline{g}}^{\symup{p}_{1}}_{{\symup{\mu}}_1} \ \widetilde \circ\ \hat{\underline{g}}^{\symup{p}_{3}}_{{\symup{\mu}}_3} \label{eq:2-215c}~.
	\end{align}
\end{subequations}

注意,对于脉冲光/连续谱的时空谱耦合波方程组,后两个下转换方程可以合并为一个,或只保留其中任意一个,但代价是同一个下转换方程可能会横跨光脉冲和太赫兹脉冲共 2 个脉冲的频谱。

此外,对于太赫兹波段的行波,对其在晶体内的线性和非线性光学过程,起主要贡献的,除了晶体内的电子外,还有不可忽略的晶格振动/声子,甚至由后者主导,导致太赫兹波段的一阶线性系数(即折射率)和二阶非线性系数实部起伏非常大(通常是升高),并且均常常还伴随有不小的虚部,分别对应电子对太赫兹波的线性吸收和多光子吸收等。

并且,即使不在离子晶体中(但一般需要非中心对称的晶体,以提供正负电中心分离的周期性环境),如果材料对太赫兹波段的电磁波,有原子的集体运动的强响应,并因此带来光子与声子的强耦合,通常不再能以单纯的光波/电磁波来理解晶体中的行波,而是一种半光-半物质的混合:声子极化激元/极化子/电磁耦子;此时,除了晶体中高速运动的电子(上至飞秒)作为贡献太赫兹波的直接来源和次波波源外,晶体中略低速集体振动的正负电中心(上至皮秒)也是直接产生源和次波波源之一,并且这种集体振动还会影响原子分布密度,进而通过影响折射率来影响太赫兹的产生、叠加和传播,以至于太赫兹波的偏振态是复杂的,并且光子、声子几乎无法解耦,尽管形式上仍可以单独研究电磁场,毕竟它就只由麦氏方程组描述,且只有几个物理量作为属性。

由于声子在太赫兹波段扮演的角色不容忽视,在研究晶体中的太赫兹波时,麦氏方程组的本构关系中,电极化强度 $\widetilde{\symbf P}$ 不仅与 $\widetilde{\symbf E}$ 有关,需要加上黄昆方程中的正负电中心相对位移 $\widetilde{\symbf W}$ 相关项,才能从宏观上描述晶体内的电磁场演化/传播,以及 $\widetilde{\symbf W}$ 自身的动力学过程:$\widetilde{\symbf W}$ 的变化率与 $\widetilde{\symbf W}$ 自身和 $\widetilde{\symbf E}$ 均有关。而本构方程的较大变化则将首先导致线性光学本征解的性质发生根本改变,再影响非线性光学过程,以至于又是另一个故事了,不属于本文的主线,所以不再讨论这个支线;但这肯定是另一场令人激动的旅程,可惜我不能同时去涉足,金黄色的树林里分出的这第二条路。

%\subsection{一次电光效应、折射率微扰势散射的时空谱耦合波方程}
\subsection{\protect\hyperlink{chap:\thesubsection}{一次电光效应、折射率微扰势散射的时空谱耦合波方程}}
\addtocontents{toc}{\protect\linkdest{chap:\thesubsection}}
\label{一次电光效应、折射率微扰势散射的时空谱耦合波方程}

在所有类型的二阶非线性过程中,除了上述连续波三波混频和脉冲光整流之外,涉及自耦合\myHyperFootnote{即自身的变化率与自身的值相关。}的动力学过程,还有典型的射频/微波波段的一次电光效应以及折射率微调制介质中的线性散射过程。

以一次电光效应为例,在 ${\symbf{G}}^{\omega{\symup{p}}_{1}}_{z}, {\symbf{G}}^{\omega{\symup{p}}_{2}}_{z}$ 中任选一个,将脉冲光整流时空谱耦合波方程组 Eq(\ref{eq:2-212}) 中,每个方程右侧波源项对应的该低频光场(的复振幅和偏振态),全修改为直流、交流电场\myHyperFootnote{即使调制频率很高,也远低于光场频率,甚至远低于太赫兹频率,因此它们的电场性质非常不同。}(的对应物),即可得到一次电光效应时空谱耦合波方程。

对于交流行波电场或交/直流驻波电场\myHyperFootnote{交/直流 = 是/否含时;行/驻波 = 含空,且每个时刻的空域电场分布,类似无限大均匀平面波,其电场分布通常(在参与混频的其他行波的波矢方向上)具有一维周期性。},其理想化的一次电光效应时空谱耦合波方程在形式上大体与 Eq(\ref{eq:2-212}) 相同:复振幅 ${\mathtt{G}}^{\omega{\symup{p}}_{1}}_{z}, {\mathtt{G}}^{\omega{\symup{p}}_{2}}_{z}$ 形式上保持不变;只是在有效非线性系数 Eq(\ref{eq:2-213}) 中,行波场的偏振态 $\hat{\underline{g}}^{\omega\symup{p}_{1}}_{{\symup{\mu}}_1}$ 或 $\hat{\underline{g}}^{\omega\symup{p}_{2}}_{{\symup{\mu}}_2}$ 不再受晶体制约,而由行波电极形状及其上加载的时空波形确定,以至于不再需要上标 $\symup{p}_{1}, \symup{p}_{2}$,而写为 $\hat{\underline{g}}^{\omega}_{{\symup{\mu}}_1}$ 或 $\hat{\underline{g}}^{\omega}_{{\symup{\mu}}_2}$ 形式,表示偏振态只有一个。

对于空间分布更一般的交/直流电场,由于参与混频的该右侧波源并非可自由传播的光场/电磁场,而是局域的纯电场(不含磁场),导致这种电场不存在任何方向的行波复波矢 $\symbf{k}^\omega$,但在某一个或多个维度上,可能存在 2 个共线相反的倒格矢,因此一般情况下,交/直流电场时空谱 ${\mathtt{G}}^{\omega{\symup{p}}_{1}}_{z}$ 或 ${\mathtt{G}}^{\omega{\symup{p}}_{2}}_{z}$ 应调整为不含 $z$ 向传递函数,且偏振模唯一的 ${\mathtt{g}}^{\omega}_1$ 或 ${\mathtt{g}}^{\omega}_2$,即不再大写、不再有斜下标 $z$、斜上标 ${\symup{p}}_{1}, {\symup{p}}_{2}$。

更一般地,由于电极板产生的纯电场 $\underline{E}^\omega_{{\symup{\mu}}_1 z}$,除了相比光电场可能更复杂外,关键在于其不是行波,以至于不再满足晶体或空气中的线性矢量衍射规律,因此延续对其二维傅立叶变换的意义不大:类似空域分布固定的二阶非线性系数 ${\underline{\chi}}^{(2)\omega}_{ {\symup{\mu}}_{312}z}$,需要三维傅立叶变换才合理。

因此,如果一次电光效应的波动方程左侧电场偏振态,仍仅由线性光学确定(对于波动方程右侧波源含射频/微波波源的电光效应,这一般是不成立的),则对于普遍意义上的交/直流电场,波动方程需要从 Eq(\ref{eq:2-174b}) 开始,换一种路线推导,即将二阶非线性系数与电极产生的纯电场的乘积 ${\underline{\chi}}^{(2)\omega}_{ {\symup{\mu}}_{312}z} \underline{E}^{\widetilde{\omega}}_{{\symup{\mu}}_1 z}$ 视为整体:
\begin{subequations} \label{eq:2-216}
	\abovedisplayskip=10pt
	\belowdisplayskip=10pt
	\begin{align}
		{\underline{Q}}^{(2)\omega}_{{\symup{\mu}}_{3}z} &= \mathcal F \left[ {\underline{\chi}}^{(2)\omega}_{ {\symup{\mu}}_{312}z} \underline{E}^\omega_{{\symup{\mu}}_1 z}\ \widetilde *\ \underline{E}^\omega_{{\symup{\mu}}_2 z} \right] \label{eq:2-216a} \\ &= \mathcal F \left[ {\underline{\chi}}^{(2)\omega}_{ {\symup{\mu}}_{312}z} \underline{E}^{\widetilde{\omega}}_{{\symup{\mu}}_1 z} \right] \widetilde \circledast\ \mathcal F \left[ \underline{E}^{\widetilde{\omega}}_{{\symup{\mu}}_2 z} \right]  \label{eq:2-216b} \\ &=\mathcal F^{-1}_{z} \left[ \mathcal F_{\symup{3D}} \left[ {\underline{\chi}}^{(2)\omega}_{ {\symup{\mu}}_{312}z} \underline{E}^{\widetilde{\omega}}_{{\symup{\mu}}_1 z} \right] \right] \widetilde \circledast\ \underline{G}^{\widetilde{\omega}}_{{\symup{\mu}}_2 z} \label{eq:2-216c} \\ &=\mathcal F^{-1}_{z} \left[ \mathcal F_{\symup{3D}} \left[ {\underline{\chi}}^{(2)\omega}_{ {\symup{\mu}}_{312}z} \underline{E}^{\widetilde{\omega}}_{{\symup{\mu}}_1 z} \right] \widetilde \circledast\ \underline{G}^{\widetilde{\omega}}_{{\symup{\mu}}_2 z} \right] \label{eq:2-216d}~,
	\end{align}
\end{subequations}
其中,相对于 Eq(\ref{eq:2-174c}) 或 Eq(\ref{eq:2-189a}) 中的 $\underline{E}^\omega_{{\symup{\mu}}_1 z}$,上式 Eq(\ref{eq:2-216b}) 中的 $\underline{E}^{\widetilde{\omega}}_{{\symup{\mu}}_1 z}$ 的 $\omega$ 头上多了一个波浪符号 $"\sim"$,表示算符 $"\widetilde \circledast"$ 的作用范围只触及 $\underline{E}^{\widetilde{\omega}}_{{\symup{\mu}}_1 z}$ 而不包含 ${\underline{\chi}}^{(2)\omega}_{ {\symup{\mu}}_{312}z}$ 的同时,$\underline{E}^{\widetilde{\omega}}_{{\symup{\mu}}_1 z}$ 的角频率在时间频率域卷积积分内与 ${\underline{\chi}}^{(2)\omega}_{ {\symup{\mu}}_{312}z}$ 不同,但积分后相同。这是一种难以理解的表示,但除此之外几乎没有其他更好的办法:$\underline{E}^{\widetilde{\omega}}_{{\symup{\mu}}_1 z}$ 是两个算符 $\mathcal F_{\symup{3D}} \left[ \cdot \right], \widetilde \circledast$ 所各自作用的元素所构成的集合的交集,且该交集只有 $\underline{E}^{\widetilde{\omega}}_{{\symup{\mu}}_1 z}$ 这一个元素,并且 $\underline{E}^{\widetilde{\omega}}_{{\symup{\mu}}_1 z}$ 在时间频率域的自变量(即角频率)与 ${\underline{\chi}}^{(2)\omega}_{ {\symup{\mu}}_{312}z}$ 不同。

经 \cref{eq:2-190,eq:2-193} 和第 \pageref{con:4} 页的首个假设,Eq(\ref{eq:2-216d}) 变为
\begin{subequations} \label{eq:2-217}
	\abovedisplayskip=13pt
	\belowdisplayskip=13pt
	\begin{align}
		{\underline{Q}}^{(2)\omega}_{{\symup{\mu}}_{3}z} \asymp {\underline{Q}}^{(2)\omega{\symup{p}}_{32}}_{{\symup{\mu}}_{3}z} &= {\underline{\chi}}^{(2)\omega}_{{\symup{\mu}}_{312}} {\underline{\symup{C}}}_{{\symup{\mu}}_{312}} \mathcal F^{-1}_{z} \left[ C \circledast \underline{g}^{\omega}_{{\symup{\mu}}_1} \widetilde \circledast\ \underline{G}^{\omega}_{{\symup{\mu}}_2 z} \right] \label{eq:2-217a} \\ &= {\underline{\symup{C}}}_{{\symup{\mu}}_{312}} {\underline{\chi}}^{(2)\omega}_{{\symup{\mu}}_{312}} \mathcal F^{-1}_{z} \left[ C \circledast \left( \hat{\underline{g}}^{\omega}_{{\symup{\mu}}_1} {\mathtt{g}}^{\omega}_1 \right) \widetilde \circledast \left( \hat{\underline{g}}^{\omega{\symup{p}}_{2}}_{{\symup{\mu}}_2} {\mathtt{G}}^{\omega{\symup{p}}_{2}}_{z} \right) \right] \label{eq:2-217b} \\ &= {\underline{\symup{C}}}_{{\symup{\mu}}_{312}} {\underline{\chi}}^{(2)\omega}_{{\symup{\mu}}_{312}} \hat{\underline{g}}^{\omega}_{{\symup{\mu}}_1} \ \widetilde *\ \hat{\underline{g}}^{\omega{\symup{p}}_{2}}_{{\symup{\mu}}_2} \mathcal F^{-1}_{z} \left[ C \circledast {\mathtt{g}}^{\omega}_1 \ \widetilde \circledast\ {\mathtt{G}}^{\omega{\symup{p}}_{2}}_{z} \right] \label{eq:2-217c}~, 
	\end{align}
\end{subequations}
这里不再需要使用 $\widetilde \omega$,因为根据第 \pageref{con:4} 页的首个假设,$C^\omega$ 不再与 $\omega$ 相关,以至于消除了歧义。此外,上式 Eq(\ref{eq:2-217}) 的积分顺序如下:$"\circledast"$ 的 $\symbf{k}_{\symup{q}}$ 域 $\to$ $"\widetilde \circledast"$ 的 $\symbf{k}_{\symup{\rho}}$ 域(即 $\symbf{k}_{\symup{\rho}}$ 域的 $"*"$) $\to$ $q_{\symup{z}}$ 域的 $F^{-1}_{z} \left[ \cdot \right]$ $\to$ $"\widetilde \circledast"$ 的 $\omega$ 域(即 $\omega$ 域的 $"\ \widetilde *\ "$)。并且,Eq.(\ref{eq:2-217b}) $\to$ Eq.(\ref{eq:2-217c}) 对电场和光场也均分别使用了第 \pageref{con:3} 页的“标量非线性波源”条件。

将 Eq(\ref{eq:2-217}) 代入 Eq.(\ref{eq:2-172d}) 即有
\begin{subequations} \label{eq:2-218}
	\abovedisplayskip=13pt
	\belowdisplayskip=13pt
	\begin{align}
		\displaystyle{\frac{\partial {\mathtt{g}}^{\omega\symup{p}_{3}}_{z}}{\partial z}} \asymp \displaystyle{\frac{\partial {\mathtt{g}}^{\omega\symup{p}_{32}}_{z}}{\partial z}} &= \mathbb{i} k^{2}_{0\omega} \frac{ \widetilde{\chi}^{(2)\omega\symup{p}_{32}}_{\symup{eff}} \mathcal F^{-1}_{z} \left[ C \circledast {\mathtt{g}}^{\omega}_1 \ \widetilde \circledast\ {\mathtt{G}}^{\omega{\symup{p}}_{2}}_{z} \right] }{ \left| \hat{\underline{g}}^{\omega\symup{p}_{3}} \right|^2 2 k^{\omega\symup{p}_{3}}_{\symup{z}} \mathbb{e}^{\mathbb{i} k^{\omega\symup{p}_{3}}_{\symup{z}} z} } \label{eq:2-218a}~, \\ \text{where}\ \ \ \ \ \  \widetilde{\chi}^{(2)\omega\symup{p}_{32}}_{\symup{eff}} &:= \hat{\underline{g}}^{\omega\symup{p}_{3}*}_{{\symup{\mu}}_{3}} {\underline{\symup{C}}}_{{\symup{\mu}}_{312}} {\underline{\chi}}^{(2)\omega}_{{\symup{\mu}}_{312}} \hat{\underline{g}}^{\omega}_{{\symup{\mu}}_1} \ \widetilde *\ \hat{\underline{g}}^{\omega{\symup{p}}_{2}}_{{\symup{\mu}}_2} \label{eq:2-218b}~.
	\end{align}
\end{subequations}

原则上可以继续写出另外两个下转换的方程,并令二阶非线性系数均一分布,且将电极电场简化,以得到更一般或更具代表性的一次电光效应时空谱耦合波方程;然而,同样是二阶非线性过程,直流/射频/微波波段的电光效应却不能由 Eq(\ref{eq:2-218}) 描述:由于参与混频的电极电场频率大多处于直流/射频/微波波段,远低于太赫兹、近红外、可见光波段,导致直流/射频/微波波段与光波段的混频结果,大多只是在以直流/射频/微波频率改变光波段的偏振态,而并未产生新频率的光(包括电光频率梳)。

因此,要想研究直流/射频/微波波段的(一次/二次)电光效应,不应回退至 Eq(\ref{eq:2-174b}),而应直接回退至 Eq(\ref{eq:2-25})。为此,假设方程右侧非线性波源的影响,只改变方程左侧电场的倒空间偏振态,而对其复振幅无影响\label{con:5},以至于 ${\symbf g^{\omega}_z}$ 退化为 ${\symbf g^{\omega}}$ 并且所有 $z$ 向偏导项消失;则在第 \pageref{con:1} 页的条件下,Eq(\ref{eq:2-25}) 变为:
\begin{equation} \label{eq:2-219}
	\left\{\ \begin{aligned} \bar{\bar{\symbf L}}^\omega \cdot {\symbf g^{\omega}} &= k^{2}_{0\omega} {\symbf Q^{{\symup{NL}},\omega}_z} \big/ \mathbb{e}^{\mathbb{i} k^\omega_{\symup{z}} z} \\ {\symbf k^\omega} \cdot \bar{\bar{\symbf{\varepsilon}}}_{\symup{r}}^{(1)\omega} \cdot {\symbf g^{\omega}} &= \mathbb{i} {\symbf{\mathsfit{N}}} \cdot {\symbf Q^{{\symup{NL}},\omega}_z} \big/ \mathbb{e}^{\mathbb{i} k^\omega_{\symup{z}} z} \end{aligned}\right. ~,
\end{equation}
其中 ${\symbf Q^{{\symup{NL}},\omega}_z} \big/ \mathbb{e}^{\mathbb{i} k^\omega_{\symup{z}} z}$ 以及 ${\symbf{\mathsfit{N}}} \cdot {\symbf Q^{{\symup{NL}},\omega}_z} \big/ \mathbb{e}^{\mathbb{i} k^\omega_{\symup{z}} z}$ 整体必须不含 $z$,以使方程组的解 ${\symbf g^{\omega}}$ 不含 $z$;要想前者 ${\symbf Q^{{\symup{NL}},\omega}_z} \big/ \mathbb{e}^{\mathbb{i} k^\omega_{\symup{z}} z}$ 不含 $z$,则要求 ${\symbf Q^{{\symup{NL}},\omega}_z}$ 的每一个分量都含且只含 $\mathbb{e}^{\mathbb{i} k^\omega_{\symup{z}} z}$ 这唯一一个与 $z$ 有关的因子,有了该条件后,后者 ${\symbf{\mathsfit{N}}} \cdot {\symbf Q^{{\symup{NL}},\omega}_z} \big/ \mathbb{e}^{\mathbb{i} k^\omega_{\symup{z}} z}$ 便自动不含 $z$,并且算符 ${\symbf{\mathsfit{N}}}$ 退化为 $\mathbb{i} {\symbf k^\omega}$。在第 \pageref{con:5} 页的该条件下,Eq(\ref{eq:2-219}) 退化为
\begin{subequations} \label{eq:2-220}
	\abovedisplayskip=13pt
	\belowdisplayskip=13pt
	\begin{align}
		\left\{\ \begin{aligned} \left( \bar{\bar{\symbf L}}^\omega - k^{2}_{0\omega} {\bar{\bar{\symbf Q}}^{{\symup{NL}},\omega}} \right) \cdot {\symbf g^{\omega}} &= \symbf 0 \\ {\symbf k^\omega} \cdot \left( \bar{\bar{\symbf{\varepsilon}}}_{\symup{r}}^{(1)\omega} + {\bar{\bar{\symbf Q}}^{{\symup{NL}},\omega}} \right) \cdot {\symbf g^{\omega}} &= 0 \end{aligned}\right. ~ \label{eq:2-220a}~, \\ \text{where}\ \ \ \ \ \ {\symbf Q^{{\symup{NL}},\omega}_z} =: {\bar{\bar{\symbf Q}}^{{\symup{NL}},\omega}} \cdot {\symbf g^{\omega}} \mathbb{e}^{\mathbb{i} k^\omega_{\symup{z}} z} \label{eq:2-220b}~.
	\end{align}
\end{subequations}

以一次电光效应为例,要想 Eq(\ref{eq:2-216}) 中的 ${\underline{Q}}^{(2)\omega}_{{\symup{\mu}}_{3}z} \big/ \mathbb{e}^{\mathbb{i} k^\omega_{\symup{z}} z}$ 整体不含 $z$,则必须要求 ${\underline{\chi}}^{(2)\omega}_{ {\symup{\mu}}_{312}z} = {\underline{\chi}}^{(2)\omega}_{ {\symup{\mu}}_{312}}$ 以及 $\underline{E}^\omega_{{\symup{\mu}}_1 z} = \underline{E}^\omega_{{\symup{\mu}}_1}$ 不含 $z$,以至于 $\mathcal F_{\symup{3D}} \left[ \cdot \right], "\circledast"$ 分别退化为 $\mathcal F \left[ \cdot \right], "*"$,且不再需要 $\mathcal F^{-1}_z \left[ \cdot \right]$;并且相位因子/传递函数 $\mathbb{e}^{\mathbb{i} k^\omega_{\symup{z}} z}$ 只由 $\underline{G}^{\omega}_{{\symup{\mu}}_2 z}$ 提供,以至于 $\underline{G}^{\omega\pm}_{{\symup{\mu}}_3 z}$ 与 $\underline{G}^{\omega\pm}_{{\symup{\mu}}_2 z}$ 共享本征值 $k^{\omega\pm}_{\symup{z}}$,甚至共享偏振态及其斜上角标 ${\symup{p}}_{3} = {\symup{p}}_{2} = \pm$(但不共享 $\symup{xyz}$ 三分量及其斜下角标 ${\symup{\mu}}_{3} \neq {\symup{\mu}}_{2}$);同时,$\underline{E}^\omega_{{\symup{\mu}}_1} = \underline{E}_{{\symup{\mu}}_1}$ 不含 $\omega$,导致 $"\ \widetilde *\ ,\widetilde \circledast"$ 也分别退化为 $"\cdot, *"$,且不再需要 $"\widetilde \omega"$ 头上的波浪号:
\begin{subequations} \label{eq:2-221}
	\abovedisplayskip=13pt
	\belowdisplayskip=13pt
	\begin{align}
		{\underline{Q}}^{(2)\omega}_{{\symup{\mu}}_{3}z} \asymp {\underline{Q}}^{(2)\omega\pm}_{{\symup{\mu}}_{3}z} &= \mathcal F \left[ {\underline{\chi}}^{(2)\omega}_{ {\symup{\mu}}_{312}} \underline{E}_{{\symup{\mu}}_1} \cdot \underline{E}^\omega_{{\symup{\mu}}_2 z} \right] \label{eq:2-221a} \\ &= \mathcal F \left[ {\underline{\chi}}^{(2)\omega}_{ {\symup{\mu}}_{312}} \right] * \mathcal F \left[ \underline{E}_{{\symup{\mu}}_1} \right] * \mathcal F \left[ \underline{E}^{\omega}_{{\symup{\mu}}_2 z} \right]  \label{eq:2-221b} \\ &= {\underline{\chi}}^{(2)\omega}_{{\symup{\mu}}_{312}} {\underline{\symup{C}}}_{{\symup{\mu}}_{312}} \cdot C * \underline{g}_{{\symup{\mu}}_1} \!\! * \underline{G}^{\omega}_{{\symup{\mu}}_2 z} \label{eq:2-221c} \\ &= {\underline{\symup{C}}}_{{\symup{\mu}}_{312}} {\underline{\chi}}^{(2)\omega}_{{\symup{\mu}}_{312}} \cdot C * \left( \hat{\underline{\symup{g}}}_{{\symup{\mu}}_1} {\mathtt{g}}_1 \right) * \left( \hat{\underline{g}}^{\omega\pm}_{{\symup{\mu}}_2} {\mathtt{G}}^{\omega\pm}_{z} \right) \label{eq:2-221d} \\ &= {\underline{\symup{C}}}_{{\symup{\mu}}_{312}} {\underline{\chi}}^{(2)\omega}_{{\symup{\mu}}_{312}} \hat{\underline{\symup{g}}}_{{\symup{\mu}}_1} \hat{\underline{g}}^{\omega\pm}_{{\symup{\mu}}_2} \cdot C * {\mathtt{g}}_1 * {\mathtt{G}}^{\omega\pm}_{z} \label{eq:2-221e} \\ &= {\underline{\symup{C}}}_{{\symup{\mu}}_{312}} {\underline{\chi}}^{(2)\omega}_{{\symup{\mu}}_{312}} \hat{\underline{\symup{g}}}_{{\symup{\mu}}_1} \hat{\underline{g}}^{\omega\pm}_{{\symup{\mu}}_2} \cdot C * {\mathtt{g}}_1 * {\mathtt{g}}^{\omega\pm} \mathbb{e}^{\mathbb{i} k^{\omega\pm}_{\symup{z}} z} \label{eq:2-221f} \\ &= {\underline{\symup{C}}}_{{\symup{\mu}}_{312}} {\underline{\chi}}^{(2)\omega}_{{\symup{\mu}}_{312}} \hat{\underline{\symup{g}}}_{{\symup{\mu}}_1} \!\! \cdot C * {\mathtt{g}}_1 * \bar{\underline{g}}^{\omega\pm}_{{\symup{\mu}}_2} \mathbb{e}^{\mathbb{i} k^{\omega\pm}_{\symup{z}} z} \label{eq:2-221g}~,
	\end{align}
\end{subequations}
将 Eq.(\ref{eq:2-221g}) 与 Eq.(\ref{eq:2-220b}) 对比,可得 $\mathcal C$ 系下二阶张量二维卷积算符 ${\bar{\bar{\symbf Q}}^{(2)\omega}}$
\begin{equation} \label{eq:2-222}
	{\underline{Q}^{(2)\omega}_{{\symup{\mu}}_{32}}} = {\underline{\symup{C}}}_{{\symup{\mu}}_{312}} {\underline{\chi}}^{(2)\omega}_{{\symup{\mu}}_{312}} \hat{\underline{\symup{g}}}_{{\symup{\mu}}_1} \!\! \cdot C * {\mathtt{g}}_1 * ~,
\end{equation}
其中,Eq.(\ref{eq:2-221g}) 只对电场使用了第 \pageref{con:3} 页的“标量非线性波源”条件,以将准静态射频/微波场的偏振态 $\hat{\underline{\symup{g}}}_{{\symup{\mu}}_1} \!\!$ 从 $\underline{g}_{{\symup{\mu}}_1} \!\!$ 中提取出来。

如果 ${\underline{\chi}}^{(2)\omega}_{ {\symup{\mu}}_{312}}$ 以及 $\underline{E}^\omega_{{\symup{\mu}}_1}$ 在空域横向上与 $x,y$ 无关,即晶体内二阶非线性系数与直流/射频/微波电场都无横向空域调制、处在均一背景下,则 $C, {\mathtt{g}}_1$ 均为 $k_{\symup{x}},k_{\symup{y}}$ 域的 $\delta$ 函数,此时 Eq.(\ref{eq:2-222}) 中的 ${{Q}^{(2)\omega}_{{\symup{\mu}}_{32}}}$ 不再含二维空间频率域卷积,退化为二阶张量算符
\begin{equation} \label{eq:2-223}
	{\underline{Q}^{(2)\omega}_{{\symup{\mu}}_{32}}} = {\underline{\symup{C}}}_{{\symup{\mu}}_{312}} {\underline{\chi}}^{(2)\omega}_{{\symup{\mu}}_{312}} \bar{\underline{\symup{g}}}_{{\symup{\mu}}_1} ~,
\end{equation}
其中,$C$ 内的定常量 $\symup{C}$ 合并到三阶定张量 ${\underline{\symup{C}}}_{{\symup{\mu}}_{312}}$ 中,且 ${\mathtt{g}}_1$ 内的定常量 ${\mathsf{g}}_1$ 乘以 $\hat{\underline{\symup{g}}}_{{\symup{\mu}}_1}$ 后,合并到定矢量 $\bar{\underline{\symup{g}}}_{{\symup{\mu}}_1} \!\! = \hat{\underline{\symup{g}}}_{{\symup{\mu}}_1} {\mathsf{g}}_1$ 中。

将 Eq.(\ref{eq:2-26}) 代入一阶电光效应的波动方程和散度方程 Eq.(\ref{eq:2-220a}),并转换至 $\mathcal C$ 系形式
\begin{equation} \label{eq:2-224}
	\left\{\ \begin{aligned} \left( \bar{\bar{\symup{I}}} - \hat{\underline{k}}^\omega \otimes \hat{\underline{k}}^\omega - \frac{ \bar{\bar{\underline{\varepsilon}}}_{\symup{r}}^{(1)\omega} + {\bar{\bar{\underline{Q}}}^{(2)\omega}} }{ n^2_\omega } \right) \bar{\underline{g}}^{\omega} &= \bar{0} \\ \hat{\underline{k}}^\omega \cdot \left( \bar{\bar{\underline{\varepsilon}}}_{\symup{r}}^{(1)\omega} + {\bar{\bar{\underline{Q}}}^{(2)\omega}} \right) \cdot \bar{\underline{g}}^{\omega} &= 0 \end{aligned}\right. ~,
\end{equation}
可以看出,对晶体施加外电场后,晶体内的电位移矢量 $\bar{\underline{d}}^{\omega} = \bar{\bar{\underline{\varepsilon}}}_{\symup{r}}^{\omega} \cdot \bar{\underline{g}}^{\omega}$ 也不再横向 $\perp \hat{\underline{k}}^\omega$,而是加上 ${\bar{\bar{\underline{Q}}}^{\omega}} \cdot \bar{\underline{g}}^{\omega}$ 后(注意,从此省略了 $\bar{\bar{\underline{\varepsilon}}}_{\symup{r}}^{\omega}, {\bar{\bar{\underline{Q}}}^{\omega}}$ 斜上角标的阶数),所构成的新复合电位移矢量
\begin{subequations} \label{eq:2-225}
	\abovedisplayskip=13pt
	\belowdisplayskip=13pt
	\begin{align}
		\bar{\underline{d}}^{\omega}_Q &:= \left( \bar{\bar{\underline{\varepsilon}}}_{\symup{r}}^{\omega} + {\bar{\bar{\underline{Q}}}^{\omega}} \right) \cdot \bar{\underline{g}}^{\omega} \label{eq:2-225a} \\ &=: \bar{\bar{\underline{\varepsilon}}}_{\symup{r}Q}^{\omega} \cdot \bar{\underline{g}}^{\omega} \label{eq:2-225b}~,
	\end{align}
\end{subequations}
才能继续保持横向;接着,在后续利用 Eq.(\ref{eq:2-119}) 之前,利用 Berry 版复波矢表达式 Eq.(\ref{eq:2-85a}) 将 Eq.(\ref{eq:2-224}) 改造为
\begin{subequations} \label{eq:2-226}
	\abovedisplayskip=13pt
	\belowdisplayskip=13pt
	\begin{align}
		&\left\{\ \begin{aligned} \left( \bar{\bar{\symup{I}}} - \hat{\underline{k}} \hat{\underline{k}} \right) \cdot \bar{\bar{\underline{\eta}}}^{\omega}_Q \cdot \bar{\underline{d}}^{\omega}_Q &= \frac{ 1 }{ n^2_\omega } \bar{\underline{d}}^{\omega}_Q \\ \hat{\underline{k}} \cdot \bar{\underline{d}}^{\omega}_Q &= 0 \end{aligned}\right. \label{eq:2-226a}~, \\ &\ \ \ \ \ \ \ \ \text{where}\ \ \ \ \ \ \bar{\bar{\underline{\eta}}}^{\omega}_Q := \bar{\bar{\underline{\varepsilon}}}_{\symup{r}Q}^{\omega-1} \label{eq:2-226b}~,
	\end{align}
\end{subequations}
其中,所有的物理量都在 $\bar{\bar{\symbf{\epsilon}}}_{\symup{R}}^{\omega} := \symup{Re} \left[ \bar{\bar{\symbf{\epsilon}}}^{\omega} \right]$ 的主轴系 $\bar{\bar{\underline{\symbf{\epsilon}}}}_{\symup{R}}^{\omega}$ 下定义;然而在该 $\mathcal C$ 系下,$\bar{\bar{\underline{\symbf{\epsilon}}}}_{\symup{R}Q}^{\omega} = \underline{\bar{\bar{\symbf{\epsilon}}}_{\symup{R}}^{\omega} + {\bar{\bar{Q}}^{\omega}}}$ 可能存在非主对角分量,甚至不再是对称张量,但仍可(二次)主轴化,对应加电场后折射率椭球会旋转和伸缩;因此,为了直接沿用相同的算法,上述方程的坐标系应从未加电场前的 $\mathcal C$ 系,(再次)转换到加电场后的介电主轴 $\undertilde{\mathcal C}$ 系中,对应的 Eq.(\ref{eq:2-226a}) 变为
\begin{equation} \label{eq:2-227}
	\left\{\ \begin{aligned} \left( \bar{\bar{\symup{I}}} - \hat{\undertilde{k}} \hat{\undertilde{k}} \right) \cdot \bar{\bar{\undertilde{\eta}}}^{\omega}_Q \cdot \bar{\undertilde{d}}^{\omega}_Q &= \frac{ 1 }{ n^2_\omega } \bar{\undertilde{d}}^{\omega}_Q \\ \hat{\undertilde{k}} \cdot \bar{\undertilde{d}}^{\omega}_Q &= 0 \end{aligned}\right. ~,
\end{equation}
其中,波浪号下划线表示一阶及以上的张量各分量由 $\undertilde{\mathcal C}$ 系度量。

该方程组可用图 \ref{fig:2-2} 的解析各向异性矢量傅立叶线性光学流程求解;重点反而落在 $\mathcal C \rightarrow \undertilde{\mathcal C}$ 系的变换矩阵(不一定是旋转矩阵),左乘前述 $\mathcal Z \rightarrow \mathcal C$ 系间的旋转矩阵 $\bar{\bar{\underline{R}}}_{\Yup}$,最终得到 $\mathcal Z \rightarrow \undertilde{\mathcal C}$ 系的变换矩阵;这不是一件难事,但同样超出了本文的范围,留给后续工作跟进。

此外,想要完整地描述电光效应,即便参与相互作用最低频率的电磁场,不在太赫兹波段,即使在直流/射频/微波波段,甚至直流电场,也需要耦合波方程组,而不是一个方程,以至于还需要光整流这个“反电光效应”以产生低频场\cite{ShiFeiXianXingGuangXue2012}才完整;但在太赫兹波段以下,这个低频场一般在方程右侧作为源,而不是在方程左侧作为被产生的对象,因此这里我们不关注直流/射频/微波波段电光效应所对应的光整流效应,因此只需要一个波动方程。

从 Eq.(\ref{eq:2-216}) 一路走来,到 Eq.(\ref{eq:2-227}) 为止,这一段以二阶非线性光学为起点的旅程,终点却收敛到一阶的线性光学了:参与混频的某个电场频率远低于其他电场频率,使得电光效应对光波段的影响,起于非线性、止于线性。

然而,恰恰有一些看似是线性光学的过程,却反而由非线性波动方程统治;接下来我们就举一个这样的与上述电光效应完全相反的例子:泵浦受到在空 $\symbf{r}$ 域上受到微扰/调制的材料的折射率的影响,所发生的线性散射过程。该过程就始于线性、但终于非线性。

要导出该线性散射过程的非线性波动方程,需回退至第 \pageref{con:1} 页条件下的最早的 Eq.(\ref{eq:2-16}) 的无非线性波源 $\symbf P^{{\symup{NL}},\omega}_z = \symbf{0}$ 版本:
\begin{equation} \label{eq:2-228}
	\left\{\ \begin{aligned} \nabla \times \left( \nabla \times \symbf E^{\omega}_z \right) - k^{2}_{0\omega} \overset{\rightharpoonup\!\!\!\! \rightharpoonup}{\symbf{\varepsilon}_{\symup{r}}}^{(1)\omega}_{z} \cdot \symbf E^{\omega}_z &= \symbf 0 \\ \nabla \cdot \left( \overset{\rightharpoonup\!\!\!\! \rightharpoonup}{\symbf{\varepsilon}_{\symup{r}}}^{(1)\omega}_{z} \cdot \symbf E^{\omega}_z \right) &= 0 \end{aligned}\right. ~,
\end{equation}
该方程组多用于处理线性光子晶体内的电磁场本征模问题,尽管更常见的是将其合二为一地写做单个只关于磁场的主方程,而不需要任何散度方程\cite{sakodaOpticalPropertiesPhotonic2005,joannopoulosPhotonicCrystalsMolding2008};但在这里将避免这种形式,转而采取弱折射率/介电系数调制条件
\begin{subequations} \label{eq:2-229}
	\abovedisplayskip=13pt
	\belowdisplayskip=13pt
	\begin{align}
		\bar{\bar{\varepsilon}}^{\omega}_{\symup{r}z} &= \bar{\bar{\varepsilon}}^{\omega}_{\symup{r}} + \bar{\bar{\backepsilon}}^{\omega}_{z} \label{eq:2-229a}~, \\ \text{where}\ \ \ \ \ \ \bar{\bar{\backepsilon}}^{\omega}_{z} &\ll \bar{\bar{\varepsilon}}^{\omega}_{\symup{r}} \label{eq:2-229b}~,
	\end{align}
\end{subequations}
下的散射势(势散射)形式\cite{bornPrinciplesOptics60th2019,gerkeAperiodicVolumeOptics2010}:
\begin{equation} \label{eq:2-230}
	\left\{\ \begin{aligned} \nabla \times \left( \nabla \times \bar{E}^{\omega}_z \right) - k^{2}_{0\omega} \bar{\bar{\varepsilon}}^{\omega}_{\symup{r}} \cdot \bar{E}^{\omega}_z &= k^{2}_{0\omega} \bar{\bar{\backepsilon}}^{\omega}_{z} \cdot \bar{E}^{\omega}_z \\ \nabla \cdot \left( \bar{\bar{\varepsilon}}^{\omega}_{\symup{r}} \cdot \bar{E}^{\omega}_z \right) &= - \nabla \cdot \left( \bar{\bar{\backepsilon}}^{\omega}_{z} \cdot \bar{E}^{\omega}_z \right) \end{aligned}\right. ~,
\end{equation}
由于几乎找不到该势散射的微-积分方程的封闭解(更不用说折射率调制程度更高的、更一般意义上的光子晶体了);因此,通常的办法是,采用一阶或多阶玻恩近似\cite{bornPrinciplesOptics60th2019}条件,即认为波动方程右侧非线性波源项中的 $\bar{E}^{\omega}_z$ 不同于方程左侧的 $\bar{E}^{\omega}_z$,二者有因果/先后顺序关系:先有右侧的入射场或第 $n-1$ 次散射场,后有左侧第 $n$ 次散射场;左侧慢半拍,对应左侧是结果,右侧是原因;同时,二者也有强弱关系,即一般右侧的源比左侧的新增场更强,而作为结果的左侧的新增场,将作为下一个波动方程右侧的源,继续产生下一个左侧新增场:
\begin{equation} \label{eq:2-231}
	\left\{\ \begin{aligned} \nabla \times \left( \nabla \times \bar{E}^{\omega}_{nz} \right) - k^{2}_{0\omega} \bar{\bar{\varepsilon}}^{\omega}_{\symup{r}} \cdot \bar{E}^{\omega}_{nz} &= k^{2}_{0\omega} \bar{\bar{\backepsilon}}^{\omega}_{z} \cdot \bar{E}^{\omega}_{(n-1)z} \\ \nabla \cdot \left( \bar{\bar{\varepsilon}}^{\omega}_{\symup{r}} \cdot \bar{E}^{\omega}_{nz} \right) &= 0 \end{aligned}\right. ~,
\end{equation}
其中,对散射方程的处理不同于波动方程:右侧既不含较弱的 $\bar{E}^{\omega}_{nz}$ 自身,也不含较强的源 $\bar{E}^{\omega}_{(n-1)z}$,而是 $0$;原因是当 Eq.(\ref{eq:2-229b}) 的条件满足时 Eq.(\ref{eq:2-230}) 的散度方程右侧约为零,因此几乎不影响电位移矢量场 $\bar{\bar{\varepsilon}}^{\omega}_{\symup{r}} \cdot \bar{E}^{\omega}_z$ 的横向性,也就不影响各阶次的弱散射电磁场的电位移矢量 $\bar{\bar{\varepsilon}}^{\omega}_{\symup{r}} \cdot \bar{E}^{\omega}_{nz}$ 的横向性,其中总电场
\begin{equation} \label{eq:2-232}
	\bar{E}^{\omega}_z = \sum_{n=0}^{\infty} \bar{E}^{\omega}_{nz} = \sum_{n=1}^{\infty} \bar{E}^{\omega}_{\left( n - 1 \right)z}
\end{equation}
为入射电场 $\bar{E}^{\omega}_{0z}$ 与各阶散射电场 $\bar{E}^{\omega}_{nz} \left( n \geq 1 \right)$ 的线性叠加。

利用 Eq.(\ref{eq:2-20}),通过类似 \cref{eq:2-21,eq:2-22,eq:2-23} 的操作,可以将折射率微扰介质中的第 $n$ 级/阶玻恩近似下的线性散射过程 Eq.(\ref{eq:2-231}) 中的空域非线性波动方程,过渡到空间频率域,其表达式在形式上与 Eq.(\ref{eq:2-163}) 一致:
\begin{subequations} \label{eq:2-233}
	\abovedisplayshortskip=13pt
	\abovedisplayskip=13pt
	\begin{align}
		\displaystyle{\frac{\partial}{\partial z}} \bar{\bar{V}}^\omega \bar{g}^{\omega\pm}_{nz} = k^{2}_{0\omega} {\bar{Q}^{(2)\omega}_{(n-1)z}} \mathbb{e}^{-\mathbb{i} k^{\omega\pm}_{\symup{z}} z} \label{eq:2-233a}~, \\ \text{where}\ \ \ \ \ \ {\bar{Q}^{(2)\omega}_{(n-1)z}} = \mathcal F \left[ \bar{\bar{\backepsilon}}^{\omega}_{z} \cdot \bar{E}^{\omega}_{(n-1)z} \right] \label{eq:2-233b}~, 
	\end{align}
\end{subequations}
并且最终有与 Eq.(\ref{eq:2-172d}) 形式相同的
\begin{equation} \label{eq:2-234}
	\displaystyle{\frac{\partial {\mathtt{g}}^{\omega{\symup{p}}_{n}}_{z}}{\partial z}} = \mathbb{i} k^{2}_{0\omega} \frac{ \hat{\underline{g}}^{\omega{\symup{p}}_{n}\dag} \cdot \bar{\underline{Q}}^{(2)\omega}_{(n-1)z} }{ \hat{\underline{g}}^{\omega{\symup{p}}_{n}\dag} \cdot \hat{\underline{g}}^{\omega{\symup{p}}_{n}} 2 k^{\omega{\symup{p}}_{n}}_{\symup{z}} \mathbb{e}^{\mathbb{i} k^{\omega{\symup{p}}_{n}}_{\symup{z}} z} } ~,
\end{equation}
其中,$\mathcal C$ 系下的二阶非线性波源项 ${\bar{\underline{Q}}^{(2)\omega}_{(n-1)z}}$ 具象为:
\begin{subequations} \label{eq:2-235}
	\abovedisplayskip=13pt
	\belowdisplayskip=13pt
	\begin{align}
		{\underline{Q}}^{(2)\omega}_{{\symup{\mu}}_{n}z} \asymp {\underline{Q}}^{(2)\omega{\symup{p}}_{n-1}}_{{\symup{\mu}}_{n}z} &= \mathcal F \left[ \underline{\backepsilon}^{\omega}_{{\symup{\mu}}_{n,n-1}z} \cdot \underline{E}^{\omega}_{{\symup{\mu}}_{n-1}z} \right] \label{eq:2-235a} \\ &= \mathcal F_z^{-1} \left[ \mathcal F_{\symup{3D}} \left[ \underline{\backepsilon}^{\omega}_{{\symup{\mu}}_{n,n-1}z} \right] \right] * \mathcal F \left[ \underline{E}^{\omega}_{{\symup{\mu}}_{n-1}z} \right]  \label{eq:2-235b} \\ &= \mathcal F_z^{-1} \left[ \mathcal F_{\symup{3D}} \left[ \underline{\backepsilon}^{\omega}_{{\symup{\mu}}_{n,n-1}z} \right] * \mathcal F \left[ \underline{E}^{\omega}_{{\symup{\mu}}_{n-1}z} \right] \right] \label{eq:2-235c} \\ &= {\underline{\backepsilon}}^{\omega}_{{\symup{\mu}}_{n,n-1}} \mathcal F_z^{-1} \left[ C * \underline{G}^{\omega}_{{\symup{\mu}}_{n-1} z} \right] \label{eq:2-235d} \\ &= {\underline{\backepsilon}}^{\omega}_{{\symup{\mu}}_{n,n-1}} \mathcal F_z^{-1} \left[ C * \left( \hat{\underline{g}}^{\omega{\symup{p}}_{n-1}}_{{\symup{\mu}}_{n-1}} {\mathtt{G}}^{\omega{\symup{p}}_{n-1}}_{z} \right) \right] \label{eq:2-235e} \\ &= {\underline{\backepsilon}}^{\omega}_{{\symup{\mu}}_{n,n-1}} \hat{\underline{g}}^{\omega{\symup{p}}_{n-1}}_{{\symup{\mu}}_{n-1}} \mathcal F_z^{-1} \left[ C * {\mathtt{G}}^{\omega{\symup{p}}_{n-1}}_{z} \right] \label{eq:2-235f}~,
	\end{align}
\end{subequations}
其中,仍采用了第 \pageref{con:3} 页的“标量非线性波源”条件,以实现 Eq.(\ref{eq:2-235e}) $\to$ Eq.(\ref{eq:2-235f}) 的偏振态解耦;并且,为了简化,仍在第 \pageref{con:4} 页的第一个条件下,假设了可分离变量型的微扰介电系数张量:
\begin{subequations} \label{eq:2-236}
	\abovedisplayskip=13pt
	\belowdisplayskip=13pt
	\begin{align}
		\mathcal F_{\symup{3D}} \left[ \underline{\backepsilon}^{\omega}_{{\symup{\mu}}_{n,n-1}z} \right] := \mathcal F_{\symup{3D}} \left[ \underline{\backepsilon}^{\omega}_{{\symup{\mu}}_{n,n-1}} M_z \right] &= \underline{\backepsilon}^{\omega}_{{\symup{\mu}}_{n,n-1}} \mathcal F_{\symup{3D}} \left[ M_z \right] \label{eq:2-236a} \\ &=: \underline{\backepsilon}^{\omega}_{{\symup{\mu}}_{n,n-1}} C \label{eq:2-236b}~.
	\end{align}
\end{subequations}

将 Eq.(\ref{eq:2-235f}) 代入 Eq.(\ref{eq:2-234}),即得折射率微扰诱导的势散射过程的时空谱自耦合波动方程:
\begin{subequations} \label{eq:2-237}
	\abovedisplayskip=13pt
	\belowdisplayskip=13pt
	\begin{align}
		\displaystyle{\frac{\partial {\mathtt{g}}^{\omega\symup{p}_{n}}_{z}}{\partial z}} \asymp \displaystyle{\frac{\partial {\mathtt{g}}^{\omega\symup{p}_{n,n-1}}_{z}}{\partial z}} &= \mathbb{i} k^{2}_{0\omega} \frac{ {\backepsilon}^{\omega\symup{p}_{n,n-1}}_{\symup{eff}} \mathcal F_z^{-1} \left[ C * {\mathtt{G}}^{\omega{\symup{p}}_{n-1}}_{z} \right] }{ \left| \hat{\underline{g}}^{\omega\symup{p}_{n}} \right|^2 2 k^{\omega\symup{p}_{n}}_{\symup{z}} \mathbb{e}^{\mathbb{i} k^{\omega\symup{p}_{n}}_{\symup{z}} z} } \label{eq:2-237a}~, \\ \text{where}\ \ \ \ \ \ {\backepsilon}^{\omega\symup{p}_{n,n-1}}_{\symup{eff}} &:= \hat{\underline{g}}^{\omega\symup{p}_{n}*}_{{\symup{\mu}}_{n}} {\underline{\backepsilon}}^{\omega}_{{\symup{\mu}}_{n,n-1}} \hat{\underline{g}}^{\omega{\symup{p}}_{n-1}}_{{\symup{\mu}}_{n-1}} \label{eq:2-237b}~.
	\end{align}
\end{subequations}

以经常计算的一阶散射场 $(n = 1)$ 为例(折射率调制程度越弱 Eq.(\ref{eq:2-229b}),一级/阶玻恩近似越准),上式变为
\begin{subequations} \label{eq:2-238}
	\abovedisplayskip=13pt
	\belowdisplayskip=13pt
	\begin{align}
		\displaystyle{\frac{\partial {\mathtt{g}}^{\omega\symup{p}_{1}}_{z}}{\partial z}} \asymp \displaystyle{\frac{\partial {\mathtt{g}}^{\omega\symup{p}_{10}}_{z}}{\partial z}} &= \mathbb{i} k^{2}_{0\omega} \frac{ {\backepsilon}^{\omega\symup{p}_{10}}_{\symup{eff}} \mathcal F_z^{-1} \left[ C * {\mathtt{G}}^{\omega{\symup{p}}_{0}}_{z} \right] }{ \left| \hat{\underline{g}}^{\omega\symup{p}_{1}} \right|^2 2 k^{\omega\symup{p}_{1}}_{\symup{z}} \mathbb{e}^{\mathbb{i} k^{\omega\symup{p}_{1}}_{\symup{z}} z} } \label{eq:2-238a}~, \\ \text{where}\ \ \ \ \ \ {\backepsilon}^{\omega\symup{p}_{10}}_{\symup{eff}} &:= \hat{\underline{g}}^{\omega\symup{p}_{1}*}_{{\symup{\mu}}_{1}} {\underline{\backepsilon}}^{\omega}_{{\symup{\mu}}_{10}} \hat{\underline{g}}^{\omega{\symup{p}}_{0}}_{{\symup{\mu}}_{0}} \label{eq:2-238b}~,
	\end{align}
\end{subequations}
在后续的章节中,会给出统治该线性散射过程的 Eq.(\ref{eq:2-238}) 的非线性卷积解,其本质为波矢完美匹配条件下的非线性卷积过程,因此该线性光学过程(不涉及频率转换),在数学上并没有那么线性,在性质上与以前的 Eq.(\ref{eq:2-216}) 到 Eq.(\ref{eq:2-227}) 间的电光效应过程,完全相反。

光折变效应、多层薄膜、飞秒激光直写等机理或技术,均可导致材料的折射率起伏变化,甚至连带对二阶及高阶非线性系数进行调制;但需要注意的是,上述 Eq.(\ref{eq:2-238}) 的适用条件为 Eq.(\ref{eq:2-229b});除了 Eq.(\ref{eq:2-229b}) 这个条件外,实际还需满足:通光方向上的调制区域不太厚(以降低多重散射概率),这第二个条件。

在这两个条件下,Eq.(\ref{eq:2-238}) 的非线性卷积解以及更进一步的非线性角谱/傅立叶光学解,与严格耦合波(Rigorous Coupled Wave Analysis, RCWA)/傅里叶模态法(Fourier Modal Method, FMM)的结果,别无二致;并且能快速、准确地处理二维和三维折射率弱调制的情况,为三维线性全息\cite{gerkeAperiodicVolumeOptics2010},以及将来线性、非线性系数同时调制\cite{chenQuasiphasematchingdivisionMultiplexingHolography2021b,gerkeAperiodicVolumeOptics2010}所带来的强耦合、新现象,提供前所未有的高效算法和解决方案。

除了上述二阶非线性过程外,三阶及以上的非线性效应所对应的耦合波方程(组),也均可以按照上述模板,从正空间而非倒空间、非晶体光学、不带偏振态和折射率分布、平面波近似、傍轴近似、无线性/非线性系数调制的情况下,被“格式化”为彻底同时解决了上述所有问题的非线性傅立叶光学框架下的时空谱耦合波方程组。

以三阶非线性为例,通过照搬第 \ref{三波混频、脉冲光整流的时空谱耦合波方程组} 小节的推导,则可获得连续/脉冲光四波混频的时空谱耦合波方程组;而通过复制第 \ref{一次电光效应、折射率微扰势散射的时空谱耦合波方程} 小节的形式,则可以得到二次电光效应、时/空域自/散聚焦、自/交叉相位调制的时空谱耦合波方程。然而,我们不再重复上述过程,以得到三维三阶及以上非线性波动方程,在时间频率和空间频率域的时空谱耦合波方程。

这是因为:一方面,由于三阶及以上非线性光学过程至少需要四个电场参与混频,导致可能会计算四个及以上矢量光的本征值问题,以至于即使基于 \ref{纯电各向异性介质中傅立叶线性光学解析解} 节彻底解析的傅立叶线性光学模型,上述偏振态的计算量也会随着混频场数量线性升高,尽管这并不是问题(一般三个矢量光场的偏振态计算在 10 s 以内)。

最主要的 2 个限制是:其一,从后续对非线性卷积解的线性卷积化的经验上来看,任何带耦合的时空谱耦合波方程组,最终都会收敛到一类(非线性)卷积型高维微-积分方程组,它们的解析解对于目前的人类而言遥不可及(截止 2025 年,人类甚至只能得到少数特殊形式的一维积分方程的封闭解);除此之外,第二个困难是,即使材料在线性和非线性系数上都是均匀的,材料本身也可通过正逆压电、弹光、电光、声光、布里渊散射、拉曼散射等效应,直接或间接地以不同程度参与光-物质耦合;以至于晶体内实际进行的,不仅只有光-光相互作用的参数过程;而一旦涉及光-物质的非线性耦合,宏观层面的三维空间中的时空解析解是肉眼可见地屈指可数:几乎是虚位以待。

因此,光-光耦合、光-物质耦合中,任何一种耦合的存在,都将导致三维非线性波动方程组不仅从空域上无法得到解析解,在对其进行二维或以上的傅立叶变换以转化为偏微-积分方程组后,仍然只能终结于非线性卷积方程组解,而无法继续简化为线性卷积方程组,进而继续简化为线性傅立叶变换解;然而非线性卷积的计算量仍然很大,以至于作为非线性卷积的孩子的人类,将被智子锁定般地,对于彻底理解造物主最强大的工具:全局非线性之非线性卷积本身,彻底无缘。

鲁迅曾说,绝望就是把美好的东西撕破给人看。之后我们会深深地体会到这一点:不带耦合的单个高维非线性卷积的傅立叶变换解是可以漂亮地存在的;但问题就是,上帝只给你看这幅画的冰山一角,其他广袤的区域故作保留、鸦雀无声般地闭口不谈;这种戛然而止的余音袅袅、这种万赖此都寂,但余钟磬音的渐行渐远,让我感到彻头彻尾的造物弄人。

知不可乎骤得,托遗响于悲风。

%%\subsection{连续光、脉冲光四波混频的时空谱耦合波方程组}
%\subsection{\protect\hyperlink{chap:\thesubsection}{连续/脉冲光四波混频的时空谱耦合波方程组}}
%\addtocontents{toc}{\protect\linkdest{chap:\thesubsection}}
%\label{连续/脉冲光四波混频的时空谱耦合波方程组}
%
%%\subsection{二次电光效应、时/空域自/散聚焦、自/交叉相位调制的时空谱耦合波方程}
%\subsection{\protect\hyperlink{chap:\thesubsection}{二次电光效应、时/空域自/散聚焦、自/交叉相位调制的时空谱耦合波方程}}
%\addtocontents{toc}{\protect\linkdest{chap:\thesubsection}}
%\label{二次电光效应、时/空域自/散聚焦、自/交叉相位调制的时空谱耦合波方程}

%\section{纯电各向异性介质中无耦合非线性光学过程的解析解}
\section{\protect\hyperlink{chap:\thesection}{纯电各向异性介质中无耦合非线性光学过程的解析解}}
\addtocontents{toc}{\protect\linkdest{chap:\thesection}}
\label{纯电各向异性介质中无耦合非线性光学过程的解析解}

%\subsection{三波混频、折射率微扰势散射的非线性卷积解}
\subsection{\protect\hyperlink{chap:\thesubsection}{三波混频、折射率微扰势散射的非线性卷积解}}
\addtocontents{toc}{\protect\linkdest{chap:\thesubsection}}
\label{三波混频、折射率微扰势散射的非线性卷积解}

只有在得到了 \ref{纯电各向异性介质中非线性光学过程的时空谱耦合波方程组} 节中的各标量/矢量时空谱耦合波方程组之后,才能继续尝试着对它们进行求解(否则连可求解的对象都不存在),以得到纯电各向异性介质中各非线性光学过程的时空谱解析解。

现以第 \pageref{con:3} 页的“标量非线性波源”条件、第 \pageref{con:4} 页的前两个条件,以及双泵浦 ${\mathtt{G}}^{\symup{p}_{1}}_{z}, {\mathtt{G}}^{\symup{p}_{2}}_{z}$ 未耗尽近似(Non-depleted Pump Approximation, NPA),即双泵浦的复振幅 ${\mathtt{g}}^{\symup{p}_{1}}, {\mathtt{g}}^{\symup{p}_{2}}$ 与 $z$ 无关、双泵浦只在晶体内线性衍射 ${\mathtt{G}}^{\symup{p}_{1}}_{z} = {\mathtt{g}}^{\symup{p}_{1}} \mathbb{e}^{\mathbb{i} k^{\symup{p}_{1}}_{\symup{z}} z }, {\mathtt{G}}^{\symup{p}_{2}}_{z} = {\mathtt{g}}^{\symup{p}_{2}} \mathbb{e}^{\mathbb{i} k^{\symup{p}_{2}}_{\symup{z}} z }$,条件下\label{con:6}的和频时空谱耦合波方程 Eq.(\ref{eq:2-188}) 或 Eq.(\ref{eq:2-210a}) 为例:
\begin{subequations} \label{eq:2-239}
	\abovedisplayskip=13pt
	\belowdisplayskip=13pt
	\begin{align}
		\displaystyle{\frac{\partial {\mathtt{g}}^{\symup{p}_{3(12)}}_{z}}{\partial z}} &= \mathbb{i} k^{2}_{03} \frac{ \chi^{(2)\symup{p}_{312}}_{\symup{eff}} \mathcal F^{-1}_{z} \left[ C * {\mathtt{G}}^{\symup{p}_{1}}_{z} * {\mathtt{G}}^{\symup{p}_{2}}_{z} \right] }{ \left| \hat{\underline{g}}^{\symup{p}_{3}} \right|^2 2 k^{\symup{p}_{3}}_{\symup{z}} \mathbb{e}^{\mathbb{i} k^{\symup{p}_{3}}_{\symup{z}} z} } \label{eq:2-239a}~, \\ \text{where}\ \ \ \ \ \ \chi^{(2)\symup{p}_{312}}_{\symup{eff}} &:= \hat{\underline{g}}^{\symup{p}_{3}*}_{{{\symup{\mu}}_{3}}} \symup{C}_{{\symup{\mu}}_{312}} \underline{\chi}^{(2)}_{{\symup{\mu}}_{312}} \hat{\underline{g}}^{\symup{p}_{1}}_{{\symup{\mu}}_1} \hat{\underline{g}}^{\symup{p}_{2}}_{{\symup{\mu}}_2} \label{eq:2-239b}~,
	\end{align}
\end{subequations}
其中,${\mathtt{g}}^{\symup{p}_{3(12)}}_{z}$ 表示 ${\mathtt{g}}^{\symup{p}_{3}}_{z} \asymp {\mathtt{g}}^{\symup{p}_{312}}_{z}$ 中的某一个。

将 Eq.(\ref{eq:2-239a}) 中的非线性卷积相关项展开,即有
\begin{subequations} \label{eq:2-240}
	\abovedisplayskip=13pt
	\belowdisplayskip=13pt
	\begin{align}
		\hspace{-1em} \frac{ \mathcal F^{-1}_{z} \left[ C * {\mathtt{G}}^{\symup{p}_{1}}_{z} * {\mathtt{G}}^{\symup{p}_{2}}_{z} \right] }{ \mathbb{e}^{\mathbb{i} k^{\symup{p}_{3}}_{\symup{z}} z} } = \iiint C \left( \symbf q \right) \iint &{\mathtt{G}}^{\symup{p}_{1}}_{0} \left( \symbf k_{1\symup{\rho}} \right) {\mathtt{G}}^{\symup{p}_{2}}_{0} \left( \symbf k_{2\symup{\rho}} \right) \cdot \mathbb{e}^{\mathbb{i} \Delta k^{\symup{p}_{123}}_{3\symup{z}} z} ~ \mathbb{d}{\symbf k_{1\symup{\rho}}} \mathbb{d}{\symbf q} \label{eq:2-240a}~, \\ \hspace{-1em} \text{where}\ \ \ \ \text{6-D} \ \ \ \ \Delta k^{\symup{p}_{123}}_{3\symup{z}} &:= k^{\symup{p}_{12}}_{3\symup{z}} - k^{\symup{p}_{3}}_{\symup{z}} \label{eq:2-240b}~, \\ \hspace{-1em} \text{in which}\ \ \ \ \text{6-D} \ \ \ \ \ \ \ \ k^{\symup{p}_{12}}_{3\symup{z}} &:= k^{\symup{p}_{1}}_{\symup{z}} \left( \symbf k_{1\symup{\rho}} \right) + k^{\symup{p}_{2}}_{\symup{z}} \left( \symbf k_{2\symup{\rho}} \right) + q_{\symup{z}} \label{eq:2-240c}~, \\ \hspace{-1em} \text{and}\ \ \ \ \text{6-D} \ \ \ \ \ \ \ \ \ \symbf k_{2\symup{\rho}} &:= \symbf k_{3\symup{\rho}} - \symbf k_{1\symup{\rho}} - \symbf q_{\symup{\rho}} \label{eq:2-240d}~, \\ \hspace{-1em} \text{where}\ \ \ \ \text{2-D} \ \ \ \ \ \ \ \ \ \symbf k_{3\symup{\rho}} &:= \symbf k_{\symup{\rho}} \label{eq:2-240e}~,
	\end{align}
\end{subequations}
其中,对 $\symbf k_{1\symup{\rho}}$ 的积分,也可换作是对 $\symbf k_{2\symup{\rho}}$ 的积分,只需将上述 Eq.(\ref{eq:2-240d}) 更改为 $\symbf k_{1\symup{\rho}} := \symbf k_{3\symup{\rho}} - \symbf k_{2\symup{\rho}} - \symbf q_{\symup{\rho}}$ 即可,其他不必作改动。此外,$n\text{-D}$ 表示自变量是 $n$ 维的。

将 Eq.(\ref{eq:2-240a}) 代入 Eq.(\ref{eq:2-239a}) 后,从 Eq.(\ref{eq:2-239a}) 的两侧,对 $z$ 沿 $0 \to z$ 积分,便可得第 \pageref{con:6} 页条件下,和频时空谱耦合波方程的非线性卷积解:
\begin{subequations} \label{eq:2-241}
	\abovedisplayskip=13pt
	\belowdisplayskip=13pt
	\begin{align}
		\hspace{-0.8em} {\mathtt{g}}^{\symup{p}_{3(12)}}_{z} &= \Upsilon^{\symup{p}_{312}}_{3} \iiint C \left( \symbf q \right) \iint {\mathtt{G}}^{\symup{p}_{1}}_{0} \left( \symbf k_{1\symup{\rho}} \right) {\mathtt{G}}^{\symup{p}_{2}}_{0} \left( \symbf k_{2\symup{\rho}} \right) \frac{ \mathbb{e}^{\mathbb{i} \Delta k^{\symup{p}_{123}}_{3\symup{z}} z} - 1 }{ \Delta k^{\symup{p}_{123}}_{3\symup{z}} } ~ \mathbb{d}{\symbf k_{1\symup{\rho}}} \mathbb{d}{\symbf q} \label{eq:2-241a} \\ \hspace{-0.8em} &= \Upsilon^{\symup{p}_{312}}_{3z} \iiint C \left( \symbf q \right) \iint {\mathtt{G}}^{\symup{p}_{1}}_{0} \left( \symbf k_{1\symup{\rho}} \right) {\mathtt{G}}^{\symup{p}_{2}}_{0} \left( \symbf k_{2\symup{\rho}} \right) \text{since} \left( \frac{ \Delta k^{\symup{p}_{123}}_{3\symup{z}} z }{ 2 } \right) \mathbb{d}{\symbf k_{1\symup{\rho}}} \mathbb{d}{\symbf q} \label{eq:2-241b}~, \\ \hspace{-0.8em} \Upsilon^{\symup{p}_{312}}_{3z} &:= \mathbb{i} z \cdot \Upsilon^{\symup{p}_{312}}_{3} := \mathbb{i} z \cdot \frac{ \chi^{(2)\symup{p}_{312}}_{\symup{eff}} k^{2}_{03} }{ \left| \hat{\underline{g}}^{\symup{p}_{3}} \right|^2 2 k^{\symup{p}_{3}}_{\symup{z}} }~, \ \ \ \ \chi^{(2)\symup{p}_{312}}_{\symup{eff}} := \hat{\underline{g}}^{\symup{p}_{3}*}_{{{\symup{\mu}}_{3}}} \symup{C}_{{\symup{\mu}}_{312}} \underline{\chi}^{(2)}_{{\symup{\mu}}_{312}} \hat{\underline{g}}^{\symup{p}_{1}}_{{\symup{\mu}}_1} \hat{\underline{g}}^{\symup{p}_{2}}_{{\symup{\mu}}_2} \label{eq:2-241c}~,
	\end{align}
\end{subequations}
其中,定义了
\begin{equation} \label{eq:2-242}
	\text{since} \left( x \right) := \text{sinc} \left( x \right) \mathbb{e}^{\mathbb{i} x } ~.
\end{equation}

类似地,也能得到第 \pageref{con:6} 页条件下,差频时空谱耦合波方程 Eq.(\ref{eq:2-210b}):
\begin{subequations} \label{eq:2-243}
	\abovedisplayskip=13pt
	\belowdisplayskip=13pt
	\begin{align}
		\displaystyle{\frac{\partial {\mathtt{g}}^{\symup{p}_{1(32)}}_{z}}{\partial z}} &= \mathbb{i} k^{2}_{01} \frac{ {\chi}^{(2)\symup{p}_{132}}_{\symup{eff}} \mathcal F^{-*}_{z} \left[ C \circ  {\mathtt{G}}^{{\symup{p}}_{2}}_{z} \circ {\mathtt{G}}^{{\symup{p}}_{3}}_{z} \right] }{ \left| \hat{\underline{g}}^{\symup{p}_{1}} \right|^2 2 k^{\symup{p}_{1}}_{\symup{z}} \mathbb{e}^{\mathbb{i} k^{\symup{p}_{1}}_{\symup{z}} z} } \label{eq:2-243a}~, \\ \text{where}\ \ \ \ \ \ {\chi}^{(2)\symup{p}_{132}}_{\symup{eff}} &:= \hat{\underline{g}}^{\symup{p}_{1}*}_{{\symup{\mu}}_{1}} \underline{\symup{C}}^{*}_{{\symup{\mu}}_{132}} \underline{\chi}^{(2)*}_{{\symup{\mu}}_{132}} \hat{\underline{g}}^{\symup{p}_{3}}_{{\symup{\mu}}_3} \hat{\underline{g}}^{\symup{p}_{2}*}_{{\symup{\mu}}_2} \label{eq:2-243b}~,
	\end{align}
\end{subequations}
的非线性卷积/互相关解:
\begin{subequations} \label{eq:2-244}
	\abovedisplayskip=13pt
	\belowdisplayskip=13pt
	\begin{align}
		\hspace{-1em} {\mathtt{g}}^{\symup{p}_{1(32)}}_{z} &= \Upsilon^{\symup{p}_{132}}_{1} \iiint C^* \left( \symbf q \right) \iint {\mathtt{G}}^{\symup{p}_{3}}_{0} \left( \symbf k_{3\symup{\rho}} \right) {\mathtt{G}}^{\symup{p}_{2}*}_{0} \left( \symbf k_{2\symup{\rho}} \right) \frac{ \mathbb{e}^{\mathbb{i} \Delta k^{\symup{p}_{321}}_{1\symup{z}} z} - 1 }{ \Delta k^{\symup{p}_{321}}_{1\symup{z}} } ~ \mathbb{d}{\symbf k_{3\symup{\rho}}} \mathbb{d}{\symbf q} \label{eq:2-244a} \\ \hspace{-1em} &= \Upsilon^{\symup{p}_{132}}_{1z} \iiint C^* \left( \symbf q \right) \iint {\mathtt{G}}^{\symup{p}_{3}}_{0} \left( \symbf k_{3\symup{\rho}} \right) {\mathtt{G}}^{\symup{p}_{2}*}_{0} \left( \symbf k_{2\symup{\rho}} \right) \text{since} \left( \frac{ \Delta k^{\symup{p}_{321}}_{1\symup{z}} z }{ 2 } \right) \mathbb{d}{\symbf k_{3\symup{\rho}}} \mathbb{d}{\symbf q} \label{eq:2-244b}~, \\ \hspace{-1em} \Upsilon^{\symup{p}_{132}}_{1z} &:= \mathbb{i} z \cdot \Upsilon^{\symup{p}_{132}}_{1} := \mathbb{i} z \cdot \frac{ \chi^{(2)\symup{p}_{132}}_{\symup{eff}} k^{2}_{01} }{ \left| \hat{\underline{g}}^{\symup{p}_{1}} \right|^2 2 k^{\symup{p}_{1}}_{\symup{z}} }~, \ \ \ \ {\chi}^{(2)\symup{p}_{132}}_{\symup{eff}} := \hat{\underline{g}}^{\symup{p}_{1}*}_{{\symup{\mu}}_{1}} \underline{\symup{C}}^{*}_{{\symup{\mu}}_{132}} \underline{\chi}^{(2)*}_{{\symup{\mu}}_{132}} \hat{\underline{g}}^{\symup{p}_{3}}_{{\symup{\mu}}_3} \hat{\underline{g}}^{\symup{p}_{2}*}_{{\symup{\mu}}_2} \label{eq:2-244c}~, 
	\end{align}
\end{subequations}
其中,定义了该过程的 $z$ 向波矢失配量:
\begin{subequations} \label{eq:2-245}
	\abovedisplayskip=13pt
	\belowdisplayskip=13pt
	\begin{align}
		\text{6-D}\ \ \ \ \Delta k^{\symup{p}_{321}}_{1\symup{z}} &:= k^{\symup{p}_{32}}_{1\symup{z}} - k^{\symup{p}_{1}}_{\symup{z}} \label{eq:2-245a}~, \\ \text{in which}\ \ \ \ \text{6-D} \ \ \ \ \ \ \ \ k^{\symup{p}_{32}}_{1\symup{z}} &:= k^{\symup{p}_{3}}_{\symup{z}} \left( \symbf k_{3\symup{\rho}} \right) - k^{\symup{p}_{2}}_{\symup{z}} \left( \symbf k_{2\symup{\rho}} \right) - q_{\symup{z}} \label{eq:2-245b}~, \\ \text{and}\ \ \ \ \text{6-D}\ \ \ \ \ \ \ \ \ \symbf k_{2\symup{\rho}} &:= \symbf k_{3\symup{\rho}} - \symbf k_{1\symup{\rho}} - \symbf q_{\symup{\rho}} \label{eq:2-245c}~, \\ \text{where}\ \ \ \ \text{2-D}\ \ \ \ \ \ \ \ \ \symbf k_{1\symup{\rho}} &:= \symbf k_{\symup{\rho}} \label{eq:2-245d}~,
	\end{align}
\end{subequations}

有了和/差频时空谱的非线性卷积/互相关解,便可以写出三波混频时空谱的非线性卷积/互相关解:
\begin{subequations} \label{eq:2-246}
	\abovedisplayskip=13pt
	\belowdisplayskip=0pt
	\begin{align}
		\!\!{\mathtt{g}}^{\symup{p}_{3(12)}}_{z} &= \Upsilon^{\symup{p}_{312}}_{3} \iiint C \left( \symbf q \right) \iint {\mathtt{G}}^{\symup{p}_{1}}_{0} \left( \symbf k_{1\symup{\rho}}^{\text{\three}} \right) {\mathtt{G}}^{\symup{p}_{2}}_{0} \left( \symbf k_{2\symup{\rho}}^{\text{\textcircled{3}\one}} \right) \frac{ \mathbb{e}^{\mathbb{i} \Delta k^{\symup{p}_{123}}_{3\symup{z}} z} - 1 }{ \Delta k^{\symup{p}_{123}}_{3\symup{z}} } ~ \mathbb{d}{\symbf k_{1\symup{\rho}}^{\text{\three}}} \mathbb{d}{\symbf q} \label{eq:2-246a}~, \\ \!\!{\mathtt{g}}^{\symup{p}_{1(32)}}_{z} &= \Upsilon^{\symup{p}_{132}}_{1} \iiint C^* \left( \symbf q \right) \iint {\mathtt{G}}^{\symup{p}_{3}}_{0} \left( \symbf k_{3\symup{\rho}}^\text{\one} \right) {\mathtt{G}}^{\symup{p}_{2}*}_{0} \left( \symbf k_{2\symup{\rho}}^{\text{\three\textcircled{1}}} \right) \frac{ \mathbb{e}^{\mathbb{i} \Delta k^{\symup{p}_{321}}_{1\symup{z}} z} - 1 }{ \Delta k^{\symup{p}_{321}}_{1\symup{z}} } ~ \mathbb{d}{\symbf k_{3\symup{\rho}}^\text{\one}} \mathbb{d}{\symbf q} \label{eq:2-246b}~, \\ \!\!{\mathtt{g}}^{\symup{p}_{2(31)}}_{z} &= \Upsilon^{\symup{p}_{231}}_{2} \iiint C^* \left( \symbf q \right) \iint {\mathtt{G}}^{\symup{p}_{3}}_{0} \left( \symbf k_{3\symup{\rho}}^\text{\two} \right) {\mathtt{G}}^{\symup{p}_{1}*}_{0} \left( \symbf k_{1\symup{\rho}}^{\text{\three\textcircled{2}}} \right) \frac{ \mathbb{e}^{\mathbb{i} \Delta k^{\symup{p}_{312}}_{2\symup{z}} z} - 1 }{ \Delta k^{\symup{p}_{312}}_{2\symup{z}} } ~ \mathbb{d}{\symbf k_{3\symup{\rho}}^\text{\two}} \mathbb{d}{\symbf q} \label{eq:2-246c}~, 
	\end{align}
\end{subequations}
\begin{subequations} \label{eq:2-247}
	\abovedisplayskip=0pt
	\belowdisplayskip=0pt
	\begin{align}
		\Delta k^{\symup{p}_{123}}_{3\symup{z}} &:= k^{\symup{p}_{12}}_{3\symup{z}} - k^{\symup{p}_{3}}_{\symup{z}} := k^{\symup{p}_{1}}_{\symup{z}} \left( \symbf k_{1\symup{\rho}}^{\text{\three}} \right) + k^{\symup{p}_{2}}_{\symup{z}} \left( \symbf k_{2\symup{\rho}}^{\text{\textcircled{3}\one}} \right) + q_{\symup{z}} - k^{\symup{p}_{3}}_{\symup{z}} \label{eq:2-247a}~, \\ \Delta k^{\symup{p}_{321}}_{1\symup{z}} &:= k^{\symup{p}_{32}}_{1\symup{z}} - k^{\symup{p}_{1}}_{\symup{z}} := k^{\symup{p}_{3}}_{\symup{z}} \left( \symbf k_{3\symup{\rho}}^\text{\one} \right) - k^{\symup{p}_{2}}_{\symup{z}} \left( \symbf k_{2\symup{\rho}}^{\text{\three\textcircled{1}}} \right) - q_{\symup{z}} - k^{\symup{p}_{1}}_{\symup{z}} \label{eq:2-247b}~, \\ \Delta k^{\symup{p}_{312}}_{2\symup{z}} &:= k^{\symup{p}_{31}}_{2\symup{z}} - k^{\symup{p}_{2}}_{\symup{z}} := k^{\symup{p}_{3}}_{\symup{z}} \left( \symbf k_{3\symup{\rho}}^\text{\two} \right) - k^{\symup{p}_{1}}_{\symup{z}} \left( \symbf k_{1\symup{\rho}}^{\text{\three\textcircled{2}}} \right) - q_{\symup{z}} - k^{\symup{p}_{2}}_{\symup{z}} \label{eq:2-247c}~,
	\end{align}
\end{subequations}
\begin{subequations} \label{eq:2-248}
	\abovedisplayskip=0pt
	\belowdisplayskip=0pt
	\begin{align}
		\symbf k_{2\symup{\rho}}^{\text{\textcircled{3}\one}} &:= \symbf k_{3\symup{\rho}} - \symbf k_{1\symup{\rho}}^{\text{\three}} - \symbf q_{\symup{\rho}} := \symbf k_{\symup{\rho}} - \symbf k_{1\symup{\rho}}^{\text{\three}} - \symbf q_{\symup{\rho}} \label{eq:2-248a}~, \\ \symbf k_{2\symup{\rho}}^{\text{\three\textcircled{1}}} &:= \symbf k_{3\symup{\rho}}^\text{\one} - \symbf k_{1\symup{\rho}} - \symbf q_{\symup{\rho}} := \symbf k_{3\symup{\rho}}^\text{\one} - \symbf k_{\symup{\rho}} - \symbf q_{\symup{\rho}} \label{eq:2-248b}~, \\ \symbf k_{1\symup{\rho}}^{\text{\three\textcircled{2}}} &:= \symbf k_{3\symup{\rho}}^\text{\two} - \symbf k_{2\symup{\rho}} - \symbf q_{\symup{\rho}} := \symbf k_{3\symup{\rho}}^\text{\two} - \symbf k_{\symup{\rho}} - \symbf q_{\symup{\rho}} \label{eq:2-248c}~, 
	\end{align}
\end{subequations}
\begin{subequations} \label{eq:2-249}
	\abovedisplayskip=0pt
	\belowdisplayskip=13pt
	\begin{align}
		\hspace{-0.6em} \Upsilon^{\symup{p}_{312}}_{3z} &:= \mathbb{i} z \cdot \Upsilon^{\symup{p}_{312}}_{3} := \mathbb{i} z \cdot \frac{ \chi^{(2)\symup{p}_{312}}_{\symup{eff}} k^{2}_{03} }{ \left| \hat{\underline{g}}^{\symup{p}_{3}} \right|^2 2 k^{\symup{p}_{3}}_{\symup{z}} }~, &\chi^{(2)\symup{p}_{312}}_{\symup{eff}} := \hat{\underline{g}}^{\symup{p}_{3}*}_{{{\symup{\mu}}_{3}}} \symup{C}_{{\symup{\mu}}_{312}} \underline{\chi}^{(2)}_{{\symup{\mu}}_{312}} \hat{\underline{g}}^{\symup{p}_{1}}_{{\symup{\mu}}_1} \hat{\underline{g}}^{\symup{p}_{2}}_{{\symup{\mu}}_2} \label{eq:2-249a}~, \\ \hspace{-0.6em} \Upsilon^{\symup{p}_{132}}_{1z} &:= \mathbb{i} z \cdot \Upsilon^{\symup{p}_{132}}_{1} := \mathbb{i} z \cdot \frac{ \chi^{(2)\symup{p}_{132}}_{\symup{eff}} k^{2}_{01} }{ \left| \hat{\underline{g}}^{\symup{p}_{1}} \right|^2 2 k^{\symup{p}_{1}}_{\symup{z}} }~, &{\chi}^{(2)\symup{p}_{132}}_{\symup{eff}} := \hat{\underline{g}}^{\symup{p}_{1}*}_{{\symup{\mu}}_{1}} \underline{\symup{C}}^{*}_{{\symup{\mu}}_{132}} \underline{\chi}^{(2)*}_{{\symup{\mu}}_{132}} \hat{\underline{g}}^{\symup{p}_{3}}_{{\symup{\mu}}_3} \hat{\underline{g}}^{\symup{p}_{2}*}_{{\symup{\mu}}_2} \label{eq:2-249b}~, \\ \hspace{-0.6em} \Upsilon^{\symup{p}_{231}}_{2z} &:= \mathbb{i} z \cdot \Upsilon^{\symup{p}_{231}}_{2} := \mathbb{i} z \cdot \frac{ \chi^{(2)\symup{p}_{231}}_{\symup{eff}} k^{2}_{02} }{ \left| \hat{\underline{g}}^{\symup{p}_{2}} \right|^2 2 k^{\symup{p}_{2}}_{\symup{z}} }~, &{\chi}^{(2)\symup{p}_{231}}_{\symup{eff}} := \hat{\underline{g}}^{\symup{p}_{2}*}_{{\symup{\mu}}_{2}} \underline{\symup{C}}^{*}_{{\symup{\mu}}_{231}} \underline{\chi}^{(2)*}_{{\symup{\mu}}_{231}} \hat{\underline{g}}^{\symup{p}_{3}}_{{\symup{\mu}}_3} \hat{\underline{g}}^{\symup{p}_{1}*}_{{\symup{\mu}}_1} \label{eq:2-249c}~, 
	\end{align}
\end{subequations}
其中,Eq.(\ref{eq:2-246}) 也可写作下述含 $\text{sinc}$ 的形式:
\begin{subequations} \label{eq:2-250}
	\abovedisplayskip=13pt
	\belowdisplayskip=13pt
	\begin{align}
		\hspace{-1em} {\mathtt{g}}^{\symup{p}_{3(12)}}_{z} &= \Upsilon^{\symup{p}_{312}}_{3z} \iiint C_{\symbf q} \iint {\mathtt{G}}^{\symup{p}_{1}}_{0} \left( \symbf k_{1\symup{\rho}}^{\text{\three}} \right) {\mathtt{G}}^{\symup{p}_{2}}_{0} \left( \symbf k_{2\symup{\rho}}^{\text{\textcircled{3}\one}} \right) \text{since} \left( \frac{ \Delta k^{\symup{p}_{123}}_{3\symup{z}} z }{ 2 } \right) \mathbb{d}{\symbf k_{1\symup{\rho}}^{\text{\three}}} ~ \mathbb{d}{\symbf q} \label{eq:2-250a}~, \\ \hspace{-1em} {\mathtt{g}}^{\symup{p}_{1(32)}}_{z} &= \Upsilon^{\symup{p}_{132}}_{1z} \iiint C^*_{\symbf q} \iint {\mathtt{G}}^{\symup{p}_{3}}_{0} \left( \symbf k_{3\symup{\rho}}^\text{\one} \right) {\mathtt{G}}^{\symup{p}_{2}*}_{0} \left( \symbf k_{2\symup{\rho}}^{\text{\three\textcircled{1}}} \right) \text{since} \left( \frac{ \Delta k^{\symup{p}_{321}}_{1\symup{z}} z }{ 2 } \right) \mathbb{d}{\symbf k_{3\symup{\rho}}^\text{\one}} \mathbb{d}{\symbf q} \label{eq:2-250b}~, \\ \hspace{-1em} {\mathtt{g}}^{\symup{p}_{2(31)}}_{z} &= \Upsilon^{\symup{p}_{231}}_{2z} \iiint C^*_{\symbf q} \iint {\mathtt{G}}^{\symup{p}_{3}}_{0} \left( \symbf k_{3\symup{\rho}}^\text{\two} \right) {\mathtt{G}}^{\symup{p}_{1}*}_{0} \left( \symbf k_{1\symup{\rho}}^{\text{\three\textcircled{2}}} \right) \text{since} \left( \frac{ \Delta k^{\symup{p}_{312}}_{2\symup{z}} z }{ 2 } \right) \mathbb{d}{\symbf k_{3\symup{\rho}}^\text{\two}} \mathbb{d}{\symbf q} \label{eq:2-250c}~, 
	\end{align}
\end{subequations}
其中,对于拥有单个黑色背景圆圈符号作为角标的对象,表示被积变量;其圆圈内的白色数字表示计算的是哪个频率的时空谱。对于拥有两个圆圈符号作为角标的对象,其中黑色背景圆圈符号内的白色数字本身,表示被积变量的频率;而从左到右读这两个圆圈符号内的数字,表示相应的空间频率做差。

可以看到,三波混频时空谱的非线性卷积/互相关解 Eq.(\ref{eq:2-246}) 的内涵是极其丰富的,以至于为避免多个含义对应同一个字符,必须要创造额外的字符,以详尽地表达其中的全部信息量。此外,由于存在光-光耦合,理论上 Eq.(\ref{eq:2-246}) 没有显示/封闭/解析解,只可数值求解。

然而,与和频时空谱的非线性卷积解 Eq.(\ref{eq:2-241})、差频时空谱的互相关解 Eq.(\ref{eq:2-244}) 一样,由于不存在光-光耦合或光-物质耦合,折射率微扰势散射的非线性卷积解也可以进一步化为线性卷积和傅立叶变换。因此在这里额外给出第 \pageref{con:3} 页的“标量非线性波源”条件、第 \pageref{con:4} 页的第一个条件,以及第 $n$ 级/阶玻恩近似条件下的,折射率微扰诱导的势散射过程的时空谱自耦合波动方程:
\begin{subequations} \label{eq:2-251}
	\abovedisplayskip=13pt
	\belowdisplayskip=13pt
	\begin{align}
		\displaystyle{\frac{\partial {\mathtt{g}}^{\omega\symup{p}_{n(,n-1)}}_{z}}{\partial z}} &= \mathbb{i} k^{2}_{0\omega} \frac{ {\backepsilon}^{\omega\symup{p}_{n,n-1}}_{\symup{eff}} \mathcal F_z^{-1} \left[ C * {\mathtt{G}}^{\omega{\symup{p}}_{n-1}}_{z} \right] }{ \left| \hat{\underline{g}}^{\omega\symup{p}_{n}} \right|^2 2 k^{\omega\symup{p}_{n}}_{\symup{z}} \mathbb{e}^{\mathbb{i} k^{\omega\symup{p}_{n}}_{\symup{z}} z} } \label{eq:2-251a}~, \\ \text{where}\ \ \ \ \ \ {\backepsilon}^{\omega\symup{p}_{n,n-1}}_{\symup{eff}} &:= \hat{\underline{g}}^{\omega\symup{p}_{n}*}_{{\symup{\mu}}_{n}} {\underline{\backepsilon}}^{\omega}_{{\symup{\mu}}_{n,n-1}} \hat{\underline{g}}^{\omega{\symup{p}}_{n-1}}_{{\symup{\mu}}_{n-1}} \label{eq:2-251b}~,
	\end{align}
\end{subequations}
的非线性卷积解:
\begin{subequations} \label{eq:2-252}
	\abovedisplayskip=13pt
	\belowdisplayskip=13pt
	\begin{align}
		\hspace{-1em} {\mathtt{g}}^{\symup{p}_{n(,n-1)}}_{z} &= \Upsilon^{\symup{p}_{n,n-1}}_{n} \iiint C \left( \symbf q \right) {\mathtt{G}}^{\symup{p}_{n-1}}_{0} \left( \symbf k_{n-1,\symup{\rho}} \right) \frac{ \mathbb{e}^{\mathbb{i} \Delta k^{\symup{p}_{n-1,n}}_{n\symup{z}} z} - 1 }{ \Delta k^{\symup{p}_{n-1,n}}_{n\symup{z}} } ~ \mathbb{d}{\symbf q} \label{eq:2-252a} \\ \hspace{-1em} &= \Upsilon^{\symup{p}_{n,n-1}}_{nz} \iiint C \left( \symbf q \right) {\mathtt{G}}^{\symup{p}_{n-1}}_{0} \left( \symbf k_{n-1,\symup{\rho}} \right) \text{since} \left( \frac{ \Delta k^{\symup{p}_{n-1,n}}_{n\symup{z}} z }{ 2 } \right) \mathbb{d}{\symbf q} \label{eq:2-252b}~, \\ \hspace{-1em} \Upsilon^{\symup{p}_{n,n-1}}_{nz} &:= \mathbb{i} z \cdot \Upsilon^{\symup{p}_{n,n-1}}_{n} := \mathbb{i} z \cdot \frac{ {\backepsilon}^{\omega\symup{p}_{n,n-1}}_{\symup{eff}} k^{2}_{0\omega} }{ \left| \hat{\underline{g}}^{\omega\symup{p}_{n}} \right|^2 2 k^{\omega\symup{p}_{n}}_{\symup{z}} }~, \ \ {\backepsilon}^{\omega\symup{p}_{n,n-1}}_{\symup{eff}} := \hat{\underline{g}}^{\omega\symup{p}_{n}*}_{{\symup{\mu}}_{n}} {\underline{\backepsilon}}^{\omega}_{{\symup{\mu}}_{n,n-1}} \hat{\underline{g}}^{\omega{\symup{p}}_{n-1}}_{{\symup{\mu}}_{n-1}} \label{eq:2-252c}~,
	\end{align}
\end{subequations}
其中,定义了该过程的 $z$ 向波矢失配量:
\begin{subequations} \label{eq:2-253}
	\abovedisplayskip=13pt
	\belowdisplayskip=13pt
	\begin{align}
		\text{4-D} \ \ \ \ \Delta k^{\symup{p}_{n-1,n}}_{n\symup{z}} &:= k^{\symup{p}_{n-1}}_{n\symup{z}} - k^{\symup{p}_{n}}_{\symup{z}} \label{eq:2-253a}~, \\ \text{in which}\ \ \ \ \text{4-D} \hspace{2.18em} k^{\symup{p}_{n-1}}_{n\symup{z}} &:= k^{\symup{p}_{n-1}}_{\symup{z}} \left( \symbf k_{n-1,\symup{\rho}} \right) + q_{\symup{z}} \label{eq:2-253b}~, \\ \text{and}\ \ \ \ \text{4-D} \hspace{1.75em} \symbf k_{n-1,\symup{\rho}} &:= \symbf k_{n\symup{\rho}} - \symbf q_{\symup{\rho}} \label{eq:2-253c}~, \\ \text{where}\ \ \ \ \text{2-D} \hspace{2.8em} \symbf k_{n\symup{\rho}} &:= \symbf k_{\symup{\rho}} \label{eq:2-253d}~,
	\end{align}
\end{subequations}
可以看出,对于折射率调制诱导的,第 $n$ 阶线性散射过程而言,如果所考虑的散射场与泵浦源的偏振方向相同($\symup{p}_{n} = \symup{p}_{n-1}$),则通光方向($z$ 向)无折射率调制($q_{\symup{z}} = 0$)的线性散射过程,反而在该方向($z$ 向)是波矢匹配的,与一般的非线性过程恰恰相反:无二阶非线性系数倒格矢辅助的非线性过程一般是失配的。因此,尽管折射率调制诱导的线性散射过程本质上是非线性过程,它与非线性过程在匹配/失配条件上,却完全对立/相互交换。

可以看出,上述波矢匹配都优先保证横向($\symup{\rho}$ 向或 $\perp$ 向)匹配(如 \cref{eq:2-240d,eq:2-245c,eq:2-248,eq:2-253c}),波矢失配只发生在纵向($z$ 向)上(如 \cref{eq:2-240b,eq:2-245a,eq:2-247,eq:2-253a}),因此只能计算布拉格或拉曼奈斯衍射型非线性过程,而无法计算优先保证纵向匹配的切伦科夫型非线性过程;究其原因,应出在之前对空域三维波动方程 Eq.(\ref{eq:2-19}) 进行二维傅立叶变换 $\mathcal F \left[ \cdot \right]$ 的缘故(即 Eq.(\ref{eq:2-20}));如果当时不这么做,转而对其进行 $z$ 向傅立叶变换 $\mathcal F_z \left[ \cdot \right]$,或许可以导出切伦科夫型非线性卷积解,但这么做既在物理上不易解释,同时也在数学上更复杂。因此,由于非线性切伦科夫过程的固有难处,我没有再尝试用所提到的方式处理波动方程,希望有人试着走走这条荆棘之路。

另外,\cref{eq:2-241,eq:2-244,eq:2-246,eq:2-250,eq:2-252} 这些均为前向解,也有复共轭解(表面上对前向解两侧取复共轭;本质上需对实空间波动方程取复共轭),或反向传播解(驱动源或生成的电磁场中任何组分,都可能是反向传播的),留给读者自行尝试。

%\subsection{和/差频、折射率微扰势散射的匹配型非线性角谱解}
\subsection{\protect\hyperlink{chap:\thesubsection}{和/差频、折射率微扰势散射的匹配型非线性角谱解}}
\addtocontents{toc}{\protect\linkdest{chap:\thesubsection}}
\label{和/差频、折射率微扰势散射的匹配型非线性角谱解}

有了无耦合非线性过程 \cref{eq:2-241,eq:2-244,eq:2-252} 的非线性卷积解,接下来将尝试将其转化为线性卷积,进而将其过渡到傅立叶变换,以彻底解析该非线性卷积过程的同时,结合现代快速傅立叶变换(Fast Fourier Transform, FFT)算法,在尽量不牺牲精度的前提下,大幅加快对上述非线性卷积过程的实际计算速度。

由于大多数非线性过程被当作频率转换的工具,自然期望该过程的转换效率较高;因此相比失配解,人们更关注波矢/相位匹配情况下的解析解。那么仍先以和频为例,一步步给出其匹配型非线性角谱解。

数学上,总存在一共正整数 $J \in \mathbb{N}^+$ 对合适的系数对 $a_j, b_j$ 所构成的集合 $\left\{ a_j, b_j \right\}$,加上直流偏置 $a_0$,一共 $2J + 1$ 个待定系数,使得可以将 $\text{sinc}, \text{since}$ 函数在该自变量范围内
\begin{equation} \label{eq:2-254}
	\left| x \right| \leq \left( J+1 \right) \pi ~,
\end{equation}
展开为“分数阶非正交余弦基和弦级数”(注意,使用了爱因斯坦求和约定,以遍历角/哑标 $j$ 甚至遍历 $\pm$ 并求和):
\begin{subequations} \label{eq:2-255}
	\abovedisplayskip=13pt
	\belowdisplayskip=13pt
	\begin{align}
		\hspace{-1em} \text{sinc}\left( x \right) = a_0 + \sum\limits^J_{j=1} {{a_j}\cos \left( \displaystyle{ {\frac{x}{{{b_j}}}} } \right)} &= a_0 + \frac{ a_j }{ 2 } \left( \mathbb{e}^{\mathbb{i}{\frac{x}{b_j}}} + \mathbb{e}^{-\mathbb{i}{\frac{x}{b_j}}} \right) = a_0 + \frac{ a_j }{ 2 } \mathbb{e}^{\pm \mathbb{i} {\frac{x}{b_j}}} \label{eq:2-255a}~, \\ \hspace{-1em} \text{since}\left( x \right) = \text{sinc}\left( x \right) \mathbb{e}^{\mathbb{i}{x}} &= \left( a_0 + \frac{ a_j }{ 2 } \mathbb{e}^{\pm \mathbb{i} {\frac{x}{b_j}}} \right) \mathbb{e}^{\mathbb{i}{x}} = a_0 \mathbb{e}^{\mathbb{i}{x}} + \frac{ a_j }{ 2 } \mathbb{e}^{\mathbb{i} {\frac{b_j \pm 1}{b_j}} x} \label{eq:2-255b}~,
	\end{align}
\end{subequations}
其中,直流偏置/背景 $a_0$ 与交流系数对 $a_j, b_j$ 们,满足如下关系:
\begin{subequations} \label{eq:2-256}
	\abovedisplayskip=13pt
	\belowdisplayskip=13pt
	\begin{align}
		&a_0 + \sum\limits^J_{j=1} a_j = \sum\limits^J_{j=0} a_j = 1 \label{eq:2-256a}~, \\
		&\begin{cases}
			a_0 = 0 ~, &\quad \text{for} \ \ \ J = 1 \\
			a_0 \neq 0 ~, &\quad \text{for} \ \ \ J \geq 2 \\
		\end{cases} \label{eq:2-256b}~, \\
		&\begin{cases}
			a_{j_2} > a_{j_1} ~, &\quad \text{for} \ \ \ j_2 > j_1 \\
			b_{j_2} > b_{j_1} ~, &\quad \text{for} \ \ \ j_2 > j_1 \\
		\end{cases} \label{eq:2-256c}~, \\
		&\begin{cases}
			a_j < 10^{-4} ~, &\quad \text{for} \ \ \ J \geq 5 \ \ \ \text{and} \ \ \ 1 \leq j 	\leq \text{discard}(J) \\
			b_j > 1 ~, &\quad \text{for} \ \ \ a_j \geq 10^{-4} \\
		\end{cases} \label{eq:2-256d}~,
	\end{align}
\end{subequations}
由于 $b_j > 1$ 以及它在分母上这一点,它具有“波长”或“弦长”的性质,因此也称该级数 Eq.(\ref{eq:2-255a}) 为“和弦级数”;此外,求和阶数 $J$ 对应的,包含 $\mathbb{e}$ 指数项和 $a_0$ 项的,总的有效($ a_j \geq 10^{-4} $)求和项数为:
\begin{equation} \label{eq:2-257}
	c_J J := 2 \left[ J - \text{discard} \left( J \right) \right] + \text{int} \left( J \geq 2 \right) \in \left( J, 2J + 1 \right] ~,
\end{equation}
其中,discard 表示“因幅值 $a_j$ 太小而可忽略(的余弦 $\cos$ 项)”;$\text{discard}(J)$ 表示共 $J$ 项对 $\cos$ 的求和中,这样的、可忽略的余弦 $\cos$ 项的个数;并且,$\text{int}$ 表示将布尔值转化为整型。

指定横向阶数 $J = 4$ 的分数阶非正交余弦基和弦级数,如下图 \ref{fig:2-3} 所示:
\begin{figure}[htbp]
	\abovedisplayskip=0pt
	\belowdisplayskip=0pt
	\makebox[\textwidth][c]{\includegraphics[width=1\textwidth]{./figures/2.3.png}}
	%	\center{\includegraphics[width=20cm]{./figures/2.3.png}}
	\caption{\label{fig:2-3} 目标 $\text{sinc}$ 函数与 $J = 4$ 的和弦级数及其余弦谱。}
\end{figure}

不同横向阶数 $J$ 的和弦级数的 $a_0, \left\{ a_j, b_j \right\}$ 系/参数表,如下表 \ref{tab:2-1} 所示:
\begin{table}[h!]
\caption{\label{tab:2-1} $J \in \left[ 1, 10 \right]$ 的和弦级数的 $a_0, \left\{ a_j, b_j \right\}$ 参数表。}
\resizebox{1.0\linewidth}{!}{  % 宽度不超文本宽度
	\begin{tabular}{c|cccccccccc}
		\toprule[2pt]
		$J$  & 1      & 2      & 3      & 4      & 5      & 6      & 7      & 8      & 9      & 10     \\ \midrule[1.2pt]
		$a_0$  & 0 & 0.2436 & 0.1819 & 0.1466 & 0.1349 & 0.1137 & 0.0970 & 0.0921 & 0.0817 & 0.0798 \\ \midrule
		$a_1$  & 1 & 0.3010 & 0.1695 & 0.1021 & $\cancel{< 10^{-4}}$      & $\cancel{< 10^{-4}}$      & $\cancel{< 10^{-4}}$      & $\cancel{< 10^{-4}}$      & $\cancel{< 10^{-4}}$      & $\cancel{< 10^{-4}}$      \\
		$a_2$  &        & 0.4554 & 0.2984 & 0.2048 & 0.1300 & 0.0866 & 0.0648 & $\cancel{< 10^{-4}}$      & $\cancel{< 10^{-4}}$      & $\cancel{< 10^{-4}}$      \\
		$a_3$  &        &        & 0.3503 & 0.2606 & 0.2172 & 0.1608 & 0.1245 & 0.0764 & 0.0572 & $\cancel{< 10^{-4}}$      \\
		$a_4$  &        &        &        & 0.2859 & 0.2519 & 0.1979 & 0.1578 & 0.1353 & 0.1058 & 0.0550 \\
		$a_5$  &        &        &        &        & 0.2660 & 0.2162 & 0.1765 & 0.1604 & 0.1318 & 0.1107 \\
		$a_6$  &        &        &        &        &        & 0.2248 & 0.1870 & 0.1730 & 0.1463 & 0.1361 \\
		$a_7$  &        &        &        &        &        &        & 0.1924 & 0.1797 & 0.1549 & 0.1479 \\
		$a_8$  &        &        &        &        &        &        &        & 0.1831 & 0.1599 & 0.1541 \\
		$a_9$  &        &        &        &        &        &        &        &        & 0.1625 & 0.1574 \\
		$a_{10}$ &        &        &        &        &        &        &        &        &        & 0.1591 \\ \midrule[1.2pt]
		$b_1$  & 1.7321 & 1.1486 & 1.0745 & 1.0422 & $\cancel{0.6083}$ & $\cancel{0.5747}$ & $\cancel{0.3598}$ & $\cancel{0.1991}$ & $\cancel{0.1242}$ & $\cancel{0.0299}$ \\
		$b_2$  &        & 2.0964 & 1.4534 & 1.2470 & 1.0572 & 1.0363 & 1.0267 & $\cancel{0.7408}$ & $\cancel{0.3171}$ & $\cancel{0.3042}$ \\
		$b_3$  &        &        & 2.7821 & 1.7670 & 1.3061 & 1.1943 & 1.1408 & 1.0315 & 1.0237 & $\cancel{0.5960}$ \\
		$b_4$  &        &        &        & 3.4388 & 1.8920 & 1.5249 & 1.3628 & 1.1643 & 1.1201 & 1.0217 \\
		$b_5$  &        &        &        &        & 3.7238 & 2.2339 & 1.7676 & 1.4094 & 1.2942 & 1.1201 \\
		$b_6$  &        &        &        &        &        & 4.4146 & 2.6077 & 1.8443 & 1.5799 & 1.3027 \\
		$b_7$  &        &        &        &        &        &        & 5.1690 & 2.7360 & 2.0748 & 1.6001 \\
		$b_8$  &        &        &        &        &        &        &        & 5.4395 & 3.0829 & 2.1112 \\
		$b_9$  &        &        &        &        &        &        &        &        & 6.1337 & 3.1469 \\
		$b_{10}$ &        &        &        &        &        &        &        &        &        & 6.2726 \\ \bottomrule[2pt]
	\end{tabular}
}
\end{table}

则一方面将 Eq.(\ref{eq:2-255b}) 的级数/交流/余弦部分 $a_j\big/2 \cdot \mathbb{e}^{\mathbb{i} {\frac{b_j \pm 1}{b_j}} x}$,代入和频时空谱耦合波方程的非线性卷积解 Eq.(\ref{eq:2-241b}),得到 ${\mathtt{g}}^{\symup{p}_{3(12)}}_{z;\cos}$ 的交流项 ${\mathtt{g}}^{\symup{p}_{3(12)}}_{z;\text{AC}}$:
\begin{subequations} \label{eq:2-258}
	\abovedisplayskip=13pt
	\belowdisplayskip=13pt
	\begin{align}
		& {\mathtt{g}}^{\symup{p}_{3(12)}}_{z;\text{AC}} = \Upsilon^{\symup{p}_{312}}_{3z} \iiint C_{\symbf q} \iint {\mathtt{G}}^{\symup{p}_{1}}_{10} {\mathtt{G}}^{\symup{p}_{2}}_{20} \cdot \text{since} \left( \frac{ \Delta k^{\symup{p}_{123}}_{3\symup{z}} z }{ 2 } \right) \mathbb{d}{\symbf k_{1\symup{\rho}}} \mathbb{d}{\symbf q} \label{eq:2-258a} \\ &\xrightarrow[]{\text{AC\ part}} \Upsilon^{\symup{p}_{312}}_{3z} \iiint C_{\symbf q} \iint {\mathtt{G}}^{\symup{p}_{1}}_{10} {\mathtt{G}}^{\symup{p}_{2}}_{20} \cdot \frac{ a_j }{ 2 } \mathbb{e}^{\mathbb{i} \frac{b_j \pm 1}{b_j} \frac{ \Delta k^{\symup{p}_{123}}_{3\symup{z}} z }{ 2 }} ~ \mathbb{d}{\symbf k_{1\symup{\rho}}} \mathbb{d}{\symbf q} \label{eq:2-258b} \\ & = \Upsilon^{\symup{p}_{312}}_{3z} \frac{ a_j }{ 2 } \mathbb{e}^{\mathbb{i} \frac{b_j \pm 1}{b_j} \frac{ - k^{\symup{p}_{3}}_{\symup{z}} z }{ 2 }} \iiint C_{\symbf q} \mathbb{e}^{\mathbb{i} \frac{b_j \pm 1}{b_j} \frac{ q_{\symup{z}} z }{ 2 }} \iint {\mathtt{G}}^{\symup{p}_{1}}_{10} \mathbb{e}^{\mathbb{i} \frac{b_j \pm 1}{b_j} \frac{ k^{\symup{p}_{1}}_{1\symup{z}} z }{ 2 }} {\mathtt{G}}^{\symup{p}_{2}}_{20} \mathbb{e}^{\mathbb{i} \frac{b_j \pm 1}{b_j} \frac{ k^{\symup{p}_{2}}_{2\symup{z}} z }{ 2 }} ~ \mathbb{d}{\symbf k_{1\symup{\rho}}} \mathbb{d}{\symbf q} \label{eq:2-258c} \\ & = \Upsilon^{\symup{p}_{312}}_{3z} \frac{ a_j }{ 2 } \mathbb{e}^{ - \mathbb{i} \frac{b_j \pm 1}{b_j} \frac{ k^{\symup{p}_{3}}_{\symup{z}} z }{ 2 }} \iiint C_{\symbf q} \iint {\mathtt{G}}^{\symup{p}_{1}}_{1,\frac{b_j \pm 1}{b_j} \frac{ z }{ 2 }} {\mathtt{G}}^{\symup{p}_{2}}_{2,\frac{b_j \pm 1}{b_j} \frac{ z }{ 2 }} ~ \mathbb{d}{\symbf k_{1\symup{\rho}}} \mathbb{e}^{\mathbb{i} \frac{b_j \pm 1}{b_j} \frac{ q_{\symup{z}} z }{ 2 }} ~ \mathbb{d}{\symbf q} \label{eq:2-258d} \\ & = \Upsilon^{\symup{p}_{312}}_{3z} \frac{ a_j }{ 2 } \mathbb{e}^{ - \mathbb{i} k^{\symup{p}_{3}}_{\symup{z}} \frac{b_j \pm 1}{b_j} \frac{ z }{ 2 }} \int C_{3q_{\symup{z}}} * {\mathtt{G}}^{\symup{p}_{1}}_{3,\frac{b_j \pm 1}{b_j} \frac{ z }{ 2 }} * {\mathtt{G}}^{\symup{p}_{2}}_{3,\frac{b_j \pm 1}{b_j} \frac{ z }{ 2 }} \mathbb{e}^{\mathbb{i} q_{\symup{z}} \frac{b_j \pm 1}{b_j} \frac{ z }{ 2 }} ~ \mathbb{d}{q_{\symup{z}}} \label{eq:2-258e} \\ & = \Upsilon^{\symup{p}_{312}}_{3z} \frac{ a_j }{ 2 } \mathbb{e}^{ - \mathbb{i} k^{\symup{p}_{3}}_{\symup{z}} \frac{b_j \pm 1}{b_j} \frac{ z }{ 2 }} \int \mathcal F \left[ \mathcal F_{z} \left[ M_{z} \right] {\mathtt{E}}^{\symup{p}_{1}}_{\frac{b_j \pm 1}{b_j} \frac{ z }{ 2 }} {\mathtt{E}}^{\symup{p}_{2}}_{\frac{b_j \pm 1}{b_j} \frac{ z }{ 2 }} \right] \mathbb{e}^{\mathbb{i} q_{\symup{z}} \frac{b_j \pm 1}{b_j} \frac{ z }{ 2 }} ~ \mathbb{d}{q_{\symup{z}}} \label{eq:2-258f} \\ & =: \frac{ a_j }{ 2 } {\mathtt{g}}^{\symup{p}_{3(12)}}_{\frac{b_j \pm 1}{b_j} \frac{ z }{ 2 };\text{form}} \label{eq:2-258g}~,
	\end{align}
\end{subequations}
另一方面,将 Eq.(\ref{eq:2-255b}) 的直流部分 $a_0 \mathbb{e}^{\mathbb{i}{x}}$ 代入 Eq.(\ref{eq:2-241b}),也可类似地得到 ${\mathtt{g}}^{\symup{p}_{3(12)}}_{z;\cos}$ 的直流项 ${\mathtt{g}}^{\symup{p}_{3(12)}}_{z;\text{DC}}$:
\begin{subequations} \label{eq:2-259}
	\abovedisplayskip=13pt
	\belowdisplayskip=13pt
	\begin{align}
		{\mathtt{g}}^{\symup{p}_{3(12)}}_{z;\text{DC}} &= \Upsilon^{\symup{p}_{312}}_{3z} a_0 \iiint C_{\symbf q} \iint {\mathtt{G}}^{\symup{p}_{1}}_{10} {\mathtt{G}}^{\symup{p}_{2}}_{20} \cdot \mathbb{e}^{\mathbb{i} \frac{ \Delta k^{\symup{p}_{123}}_{3\symup{z}} z }{ 2 }} ~ \mathbb{d}{\symbf k_{1\symup{\rho}}} \mathbb{d}{\symbf q} \label{eq:2-259a} \\ & = \Upsilon^{\symup{p}_{312}}_{3z} a_0 \mathbb{e}^{ - \mathbb{i} k^{\symup{p}_{3}}_{\symup{z}}  \frac{ z }{ 2 }} \int C_{3q_{\symup{z}}} * {\mathtt{G}}^{\symup{p}_{1}}_{3 \frac{ z }{ 2 }} * {\mathtt{G}}^{\symup{p}_{2}}_{3 \frac{ z }{ 2 }} \mathbb{e}^{\mathbb{i} q_{\symup{z}}  \frac{ z }{ 2 }} ~ \mathbb{d}{q_{\symup{z}}} \label{eq:2-259b} \\ & = \Upsilon^{\symup{p}_{312}}_{3z} a_0 \mathbb{e}^{ - \mathbb{i} k^{\symup{p}_{3}}_{\symup{z}} \frac{ z }{ 2 }} \int \mathcal F \left[ \mathcal F_{z} \left[ M_{z} \right] {\mathtt{E}}^{\symup{p}_{1}}_{ \frac{ z }{ 2 }} {\mathtt{E}}^{\symup{p}_{2}}_{ \frac{ z }{ 2 }} \right] \mathbb{e}^{\mathbb{i} q_{\symup{z}} \frac{ z }{ 2 }} ~ \mathbb{d}{q_{\symup{z}}} \label{eq:2-259c} \\ & = a_0 {\mathtt{g}}^{\symup{p}_{3(12)}}_{\frac{ z }{ 2 };\text{form}} \label{eq:2-259d}~,
	\end{align}
\end{subequations}
接着,结合直/交流项,得到双泵浦 ${\mathtt{G}}^{\symup{p}_{1}}_{z}, {\mathtt{G}}^{\symup{p}_{2}}_{z}$ 未耗尽近似条件下的,和频时空谱(无衍射复振幅)匹配型非线性角谱解
\begin{subequations} \label{eq:2-260}
	\abovedisplayskip=13pt
	\belowdisplayskip=13pt
	\begin{align}
		{\mathtt{g}}^{\symup{p}_{3(12)}}_{z;\cos} &= {\mathtt{g}}^{\symup{p}_{3(12)}}_{z;\text{DC}} + {\mathtt{g}}^{\symup{p}_{3(12)}}_{z;\text{AC}} = a_0 {\mathtt{g}}^{\symup{p}_{3(12)}}_{\frac{ z }{ 2 };\text{form}} + \frac{ a_j }{ 2 } {\mathtt{g}}^{\symup{p}_{3(12)}}_{\frac{b_j \pm 1}{b_j} \frac{ z }{ 2 };\text{form}} \label{eq:2-260a} \\ &= \Upsilon^{\symup{p}_{312}}_{3z} \int \begin{pmatrix} a_0 ~ \mathcal F \left[ \mathcal F_{z} \left[ M_{z} \right] {\mathtt{E}}^{\symup{p}_{1}}_{ \frac{ z }{ 2 }} {\mathtt{E}}^{\symup{p}_{2}}_{ \frac{ z }{ 2 }} \right] \mathbb{e}^{\mathbb{i} \left( q_{\symup{z}} - k^{\symup{p}_{3}}_{\symup{z}} \right) \frac{ z }{ 2 }} \\ + \displaystyle{\frac{ a_j }{ 2 }} ~ \mathcal F \left[ \mathcal F_{z} \left[ M_{z} \right] {\mathtt{E}}^{\symup{p}_{1}}_{\frac{b_j \pm 1}{b_j} \frac{ z }{ 2 }} {\mathtt{E}}^{\symup{p}_{2}}_{\frac{b_j \pm 1}{b_j} \frac{ z }{ 2 }} \right] \mathbb{e}^{\mathbb{i} \left( q_{\symup{z}} - k^{\symup{p}_{3}}_{\symup{z}} \right) \frac{b_j \pm 1}{b_j} \frac{ z }{ 2 }} \end{pmatrix} \mathbb{d}{q_{\symup{z}}} \label{eq:2-260b}~,
	\end{align}
\end{subequations}
最后,得完整的、带衍射项 $\mathbb{e}^{\mathbb{i} k^{\symup{p}_{3}}_{\symup{z}} z }$ 的和频时空谱匹配型(复振幅)非线性角谱解
\begin{subequations} \label{eq:2-261}
	\abovedisplayskip=13pt
	\belowdisplayskip=13pt
	\begin{align}
		\hspace{-0.9em} {\mathtt{G}}^{\symup{p}_{3(12)}}_{z;\cos} &= {\mathtt{g}}^{\symup{p}_{3(12)}}_{z;\cos} \mathbb{e}^{\mathbb{i} k^{\symup{p}_{3}}_{\symup{z}} z } = {\mathtt{G}}^{\symup{p}_{3(12)}}_{z;\text{DC}} + {\mathtt{G}}^{\symup{p}_{3(12)}}_{z;\text{AC}} =: a_0 {\mathtt{G}}^{\symup{p}_{3(12)}}_{\frac{ z }{ 2 };\text{form}} + \frac{ a_j }{ 2 } {\mathtt{G}}^{\symup{p}_{3(12)}}_{\frac{b_j \pm 1}{b_j} \frac{ z }{ 2 };\text{form}} \label{eq:2-261a} \\ \hspace{-0.9em} &= \Upsilon^{\symup{p}_{312}}_{3z} \int \begin{pmatrix} a_0 ~ \mathcal F \left[ \mathcal F_{z} \left[ M_{z} \right] {\mathtt{E}}^{\symup{p}_{1}}_{ \frac{ z }{ 2 }} {\mathtt{E}}^{\symup{p}_{2}}_{ \frac{ z }{ 2 }} \right] \mathbb{e}^{\mathbb{i} \left( q_{\symup{z}} + k^{\symup{p}_{3}}_{\symup{z}} \right) \frac{ z }{ 2 }} \\ + \displaystyle{\frac{ a_j }{ 2 }} ~ \mathcal F \left[ \mathcal F_{z} \left[ M_{z} \right] {\mathtt{E}}^{\symup{p}_{1}}_{\frac{b_j \pm 1}{b_j} \frac{ z }{ 2 }} {\mathtt{E}}^{\symup{p}_{2}}_{\frac{b_j \pm 1}{b_j} \frac{ z }{ 2 }} \right] \mathbb{e}^{\mathbb{i} \left( q_{\symup{z}} \frac{b_j \pm 1}{b_j} + k^{\symup{p}_{3}}_{\symup{z}} \frac{b_j \mp 1}{b_j} \right) \frac{ z }{ 2 }} \end{pmatrix} \mathbb{d}{q_{\symup{z}}} \label{eq:2-261b}~,
	\end{align}
\end{subequations}
其中,从 Eq.(\ref{eq:2-258a}) 开始,作为自变量的横向空间频率,均缩并到斜右下标中(注意区分于之前从 Eq.(\ref{eq:2-182}) $\to$ Eq.(\ref{eq:2-209}),斜右下标表示角频率这一假定),比如 ${\mathtt{G}}^{\symup{p}_{2}}_{20} := {\mathtt{G}}^{\symup{p}_{2}}_{0} \left( \symbf k_{2\symup{\rho}} \right), k^{\symup{p}_{1}}_{1\symup{z}} := k^{\symup{p}_{1}}_{\symup{z}} \left( \symbf k_{1\symup{\rho}} \right)$;并且,Eq.(\ref{eq:2-258b}) $\to$ Eq.(\ref{eq:2-258c}) 这一步,利用了 \cref{eq:2-240b,eq:2-240c,eq:2-247a} 所对应的 $\Delta k^{\symup{p}_{123}}_{3\symup{z}} := k^{\symup{p}_{12}}_{3\symup{z}} - k^{\symup{p}_{3}}_{\symup{z}} := k^{\symup{p}_{1}}_{1\symup{z}} + k^{\symup{p}_{2}}_{2\symup{z}} + q_{\symup{z}} - k^{\symup{p}_{3}}_{\symup{z}}$;此外,Eq.(\ref{eq:2-258d}) $\to$ Eq.(\ref{eq:2-258e}) 这一步,一方面形式上可以对标量场做傅立叶变换,另一方面为了加快线性卷积计算速度,定义了:
\begin{equation} \label{eq:2-262}
	{\mathtt{E}}_{z} := \mathcal F^{-1} \left[ {\mathtt{G}}_{z} \right] ~.
\end{equation}

同样,对于差频过程,将 Eq.(\ref{eq:2-255b}) 代入差频时空谱耦合波方程的非线性卷积/互相关解 Eq.(\ref{eq:2-244b}),有 ${\mathtt{g}}^{\symup{p}_{1(32)}}_{z;\cos}$ 的交流项 ${\mathtt{g}}^{\symup{p}_{1(32)}}_{z;\text{AC}}$:
\begin{subequations} \label{eq:2-263}
	\abovedisplayskip=13pt
	\belowdisplayskip=13pt
	\begin{align}
		&\hspace{-1.2em} {\mathtt{g}}^{\symup{p}_{1(32)}}_{z;\text{AC}} = \Upsilon^{\symup{p}_{132}}_{1z} \iiint C^*_{\symbf q} \iint {\mathtt{G}}^{\symup{p}_{3}}_{30} {\mathtt{G}}^{\symup{p}_{2}*}_{20} \cdot \text{since} \left( \frac{ \Delta k^{\symup{p}_{321}}_{1\symup{z}} z }{ 2 } \right) \mathbb{d}{\symbf k_{3\symup{\rho}}} \mathbb{d}{\symbf q} \label{eq:2-263a} \\ &\hspace{-1.2em} \xrightarrow[]{\text{AC\ part}} \Upsilon^{\symup{p}_{132}}_{1z} \iiint C^*_{\symbf q} \iint {\mathtt{G}}^{\symup{p}_{3}}_{30} {\mathtt{G}}^{\symup{p}_{2}*}_{20} \cdot \frac{ a_j }{ 2 } \mathbb{e}^{\mathbb{i} \frac{b_j \pm 1}{b_j} \frac{ \Delta k^{\symup{p}_{321}}_{1\symup{z}} z }{ 2 }} ~ \mathbb{d}{\symbf k_{3\symup{\rho}}} \mathbb{d}{\symbf q} \label{eq:2-263b} \\ &\hspace{-1.2em} = \Upsilon^{\symup{p}_{132}}_{1z} \frac{ a_j }{ 2 } \mathbb{e}^{\mathbb{i} \frac{b_j \pm 1}{b_j} \frac{ - k^{\symup{p}_{1}}_{\symup{z}} z }{ 2 }} \iiint C^*_{\symbf q} \mathbb{e}^{\mathbb{i} \frac{b_j \pm 1}{b_j} \frac{ - q_{\symup{z}} z }{ 2 }} \iint {\mathtt{G}}^{\symup{p}_{3}}_{30} \mathbb{e}^{\mathbb{i} \frac{b_j \pm 1}{b_j} \frac{ k^{\symup{p}_{3}}_{3\symup{z}} z }{ 2 }} {\mathtt{G}}^{\symup{p}_{2}*}_{20} \mathbb{e}^{\mathbb{i} \frac{b_j \pm 1}{b_j} \frac{ - k^{\symup{p}_{2}}_{2\symup{z}} z }{ 2 }} ~ \mathbb{d}{\symbf k_{3\symup{\rho}}} \mathbb{d}{\symbf q} \label{eq:2-263c} \\ &\hspace{-1.2em} = \Upsilon^{\symup{p}_{132}}_{1z} \frac{ a_j }{ 2 } \mathbb{e}^{ - \mathbb{i} \frac{b_j \pm 1}{b_j} \frac{ k^{\symup{p}_{1}}_{\symup{z}} z }{ 2 }} \iiint C^*_{\symbf q} \iint {\mathtt{G}}^{\symup{p}_{3}}_{3,\frac{b_j \pm 1}{b_j} \frac{ z }{ 2 }} {\mathtt{G}}^{\symup{p}_{2}*}_{2,\frac{b_j \pm 1}{b_j} \frac{ z }{ 2 }} ~ \mathbb{d}{\symbf k_{3\symup{\rho}}} \mathbb{e}^{ - \mathbb{i} \frac{b_j \pm 1}{b_j} \frac{ q_{\symup{z}} z }{ 2 }} ~ \mathbb{d}{\symbf q} \label{eq:2-263d} \\ &\hspace{-1.2em} = \Upsilon^{\symup{p}_{132}}_{1z} \frac{ a_j }{ 2 } \mathbb{e}^{ - \mathbb{i} k^{\symup{p}_{1}}_{\symup{z}} \frac{b_j \pm 1}{b_j} \frac{ z }{ 2 }} \int C^*_{1q_{\symup{z}}} \circ \left( {\mathtt{G}}^{\symup{p}_{2}}_{1,\frac{b_j \pm 1}{b_j} \frac{ z }{ 2 }} \circ {\mathtt{G}}^{\symup{p}_{3}}_{1,\frac{b_j \pm 1}{b_j} \frac{ z }{ 2 }} \right) \mathbb{e}^{- \mathbb{i} q_{\symup{z}} \frac{b_j \pm 1}{b_j} \frac{ z }{ 2 }} ~ \mathbb{d}{q_{\symup{z}}} \label{eq:2-263e} \\ &\hspace{-1.2em} = \Upsilon^{\symup{p}_{132}}_{1z} \frac{ a_j }{ 2 } \mathbb{e}^{ - \mathbb{i} k^{\symup{p}_{1}}_{\symup{z}} \frac{b_j \pm 1}{b_j} \frac{ z }{ 2 }} \int \mathcal F \left[ \mathcal F_{z} \left[ M^*_{z} \right] {\mathtt{E}}^{\symup{p}_{3}}_{\frac{b_j \pm 1}{b_j} \frac{ z }{ 2 }} {\mathtt{E}}^{\symup{p}_{2}*}_{\frac{b_j \pm 1}{b_j} \frac{ z }{ 2 }} \right] \mathbb{e}^{ - \mathbb{i} q_{\symup{z}} \frac{b_j \pm 1}{b_j} \frac{ z }{ 2 }} ~ \mathbb{d}{q_{\symup{z}}} \label{eq:2-263f} \\ &\hspace{-1.2em} =: \frac{ a_j }{ 2 } {\mathtt{g}}^{\symup{p}_{1(32)}}_{\frac{b_j \pm 1}{b_j} \frac{ z }{ 2 };\text{form}} \label{eq:2-263g}~,
	\end{align}
\end{subequations}
以及 ${\mathtt{g}}^{\symup{p}_{1(32)}}_{z;\cos}$ 的直流项 ${\mathtt{g}}^{\symup{p}_{1(32)}}_{z;\text{DC}}$:
\begin{subequations} \label{eq:2-264}
	\abovedisplayskip=13pt
	\belowdisplayskip=13pt
	\begin{align}
		{\mathtt{g}}^{\symup{p}_{1(32)}}_{z;\text{DC}} &= \Upsilon^{\symup{p}_{132}}_{1z} a_0 \iiint C^*_{\symbf q} \iint {\mathtt{G}}^{\symup{p}_{3}}_{30} {\mathtt{G}}^{\symup{p}_{2}*}_{20} \cdot \mathbb{e}^{\mathbb{i} \frac{ \Delta k^{\symup{p}_{123}}_{3\symup{z}} z }{ 2 }} ~ \mathbb{d}{\symbf k_{3\symup{\rho}}} \mathbb{d}{\symbf q} \label{eq:2-264a} \\ &= \Upsilon^{\symup{p}_{132}}_{1z} a_0 \mathbb{e}^{ - \mathbb{i} k^{\symup{p}_{1}}_{\symup{z}} \frac{ z }{ 2 }} \int C^*_{1q_{\symup{z}}} \circ \left( {\mathtt{G}}^{\symup{p}_{2}}_{1 \frac{ z }{ 2 }} \circ {\mathtt{G}}^{\symup{p}_{3}}_{1 \frac{ z }{ 2 }} \right) \mathbb{e}^{- \mathbb{i} q_{\symup{z}}  \frac{ z }{ 2 }} ~ \mathbb{d}{q_{\symup{z}}} \label{eq:2-264b} \\ &= \Upsilon^{\symup{p}_{132}}_{1z} a_0 \mathbb{e}^{ - \mathbb{i} k^{\symup{p}_{1}}_{\symup{z}} \frac{ z }{ 2 }} \int \mathcal F \left[ \mathcal F_{z} \left[ M^*_{z} \right] {\mathtt{E}}^{\symup{p}_{3}}_{ \frac{ z }{ 2 }} {\mathtt{E}}^{\symup{p}_{2}*}_{ \frac{ z }{ 2 }} \right] \mathbb{e}^{ - \mathbb{i} q_{\symup{z}} \frac{ z }{ 2 }} ~ \mathbb{d}{q_{\symup{z}}} \label{eq:2-264c} \\ &= a_0 {\mathtt{g}}^{\symup{p}_{1(32)}}_{\frac{ z }{ 2 };\text{form}} \label{eq:2-264d}~,
	\end{align}
\end{subequations}
于是,结合直流项、交流项,得到双泵浦 ${\mathtt{G}}^{\symup{p}_{3}}_{z}, {\mathtt{G}}^{\symup{p}_{2}}_{z}$ 未耗尽近似条件下的,差频时空谱(无衍射复振幅)匹配型非线性角谱解
\begin{subequations} \label{eq:2-265}
	\abovedisplayskip=13pt
	\belowdisplayskip=13pt
	\begin{align}
		{\mathtt{g}}^{\symup{p}_{1(32)}}_{z;\cos} &= {\mathtt{g}}^{\symup{p}_{1(32)}}_{z;\text{DC}} + {\mathtt{g}}^{\symup{p}_{1(32)}}_{z;\text{AC}} = a_0 {\mathtt{g}}^{\symup{p}_{1(32)}}_{\frac{ z }{ 2 };\text{form}} + \frac{ a_j }{ 2 } {\mathtt{g}}^{\symup{p}_{1(32)}}_{\frac{b_j \pm 1}{b_j} \frac{ z }{ 2 };\text{form}} \label{eq:2-265a} \\ &= \Upsilon^{\symup{p}_{132}}_{1z} \int \begin{pmatrix} a_0 ~ \mathcal F \left[ \mathcal F_{z} \left[ M^*_{z} \right] {\mathtt{E}}^{\symup{p}_{3}}_{ \frac{ z }{ 2 }} {\mathtt{E}}^{\symup{p}_{2}*}_{ \frac{ z }{ 2 }} \right] \mathbb{e}^{- \mathbb{i} \left( q_{\symup{z}} + k^{\symup{p}_{1}}_{\symup{z}} \right) \frac{ z }{ 2 }} \\ + \displaystyle{\frac{ a_j }{ 2 }} ~ \mathcal F \left[ \mathcal F_{z} \left[ M^*_{z} \right] {\mathtt{E}}^{\symup{p}_{3}}_{\frac{b_j \pm 1}{b_j} \frac{ z }{ 2 }} {\mathtt{E}}^{\symup{p}_{2}*}_{\frac{b_j \pm 1}{b_j} \frac{ z }{ 2 }} \right] \mathbb{e}^{ - \mathbb{i} \left( q_{\symup{z}} + k^{\symup{p}_{1}}_{\symup{z}} \right) \frac{b_j \pm 1}{b_j} \frac{ z }{ 2 }} \end{pmatrix} \mathbb{d}{q_{\symup{z}}} \label{eq:2-265b}~,
	\end{align}
\end{subequations}
最后,得完整的、带衍射项 $\mathbb{e}^{\mathbb{i} k^{\symup{p}_{1}}_{\symup{z}} z }$ 的差频时空谱匹配型(复振幅)非线性角谱解
\begin{subequations} \label{eq:2-266}
	\abovedisplayskip=13pt
	\belowdisplayskip=13pt
	\begin{align}
		\hspace{-1em} {\mathtt{G}}^{\symup{p}_{1(32)}}_{z;\cos} &= {\mathtt{g}}^{\symup{p}_{1(32)}}_{z;\cos} \mathbb{e}^{\mathbb{i} k^{\symup{p}_{1}}_{\symup{z}} z } = {\mathtt{G}}^{\symup{p}_{1(32)}}_{z;\text{DC}} + {\mathtt{G}}^{\symup{p}_{1(32)}}_{z;\text{AC}} =: a_0 {\mathtt{G}}^{\symup{p}_{1(32)}}_{\frac{ z }{ 2 };\text{form}} + \frac{ a_j }{ 2 } {\mathtt{G}}^{\symup{p}_{1(32)}}_{\frac{b_j \pm 1}{b_j} \frac{ z }{ 2 };\text{form}} \label{eq:2-266a} \\ \hspace{-1em} &= \Upsilon^{\symup{p}_{132}}_{1z} \int \begin{pmatrix} a_0 ~ \mathcal F \left[ \mathcal F_{z} \left[ M^*_{z} \right] {\mathtt{E}}^{\symup{p}_{3}}_{ \frac{ z }{ 2 }} {\mathtt{E}}^{\symup{p}_{2}*}_{ \frac{ z }{ 2 }} \right] \mathbb{e}^{ \mathbb{i} \left( k^{\symup{p}_{1}}_{\symup{z}} - q_{\symup{z}} \right) \frac{ z }{ 2 }} \\ + \displaystyle{\frac{ a_j }{ 2 }} ~ \mathcal F \left[ \mathcal F_{z} \left[ M^*_{z} \right] {\mathtt{E}}^{\symup{p}_{3}}_{\frac{b_j \pm 1}{b_j} \frac{ z }{ 2 }} {\mathtt{E}}^{\symup{p}_{2}*}_{\frac{b_j \pm 1}{b_j} \frac{ z }{ 2 }} \right] \mathbb{e}^{ \mathbb{i} \left( k^{\symup{p}_{1}}_{\symup{z}} \frac{b_j \mp 1}{b_j} - q_{\symup{z}} \frac{b_j \pm 1}{b_j} \right) \frac{ z }{ 2 }} \end{pmatrix} \mathbb{d}{q_{\symup{z}}} \label{eq:2-266b}~,
	\end{align}
\end{subequations}

类似地,对于折射率微扰势散射,将 Eq.(\ref{eq:2-255b}) 代入折射率微扰诱导的势散射过程的时空谱自耦合波动方程 Eq.(\ref{eq:2-252b}),得 ${\mathtt{g}}^{\symup{p}_{n(,n-1)}}_{z}$ 的交流项 ${\mathtt{g}}^{\symup{p}_{n(,n-1)}}_{z;\text{AC}}$:
\begin{subequations} \label{eq:2-267}
	\abovedisplayskip=13pt
	\belowdisplayskip=13pt
	\begin{align}
		{\mathtt{g}}^{\symup{p}_{n(,n-1)}}_{z;\text{AC}} &= \Upsilon^{\symup{p}_{n,n-1}}_{nz} \iiint C_{\symbf q} {\mathtt{G}}^{\symup{p}_{n-1}}_{n-1,0} \cdot \text{since} \left( \frac{ \Delta k^{\symup{p}_{n-1,n}}_{n\symup{z}} z }{ 2 } \right) \mathbb{d}{\symbf q} \label{eq:2-267a} \\ &\hspace{-2.2em} \xrightarrow[]{\text{AC\ part}} \Upsilon^{\symup{p}_{n,n-1}}_{nz} \iiint C_{\symbf q} {\mathtt{G}}^{\symup{p}_{n-1}}_{n-1,0} \cdot \frac{ a_j }{ 2 } \mathbb{e}^{\mathbb{i} \frac{b_j \pm 1}{b_j} \frac{ \Delta k^{\symup{p}_{n-1,n}}_{n\symup{z}} z }{ 2 }} ~ \mathbb{d}{\symbf q} \label{eq:2-267b} \\ &= \Upsilon^{\symup{p}_{n,n-1}}_{nz} \frac{ a_j }{ 2 } \mathbb{e}^{\mathbb{i} \frac{b_j \pm 1}{b_j} \frac{ - k^{\symup{p}_{n}}_{\symup{z}} z }{ 2 }} \iiint C_{\symbf q} \mathbb{e}^{\mathbb{i} \frac{b_j \pm 1}{b_j} \frac{ q_{\symup{z}} z }{ 2 }} {\mathtt{G}}^{\symup{p}_{n-1}}_{n-1,0} \mathbb{e}^{\mathbb{i} \frac{b_j \pm 1}{b_j} \frac{ k^{\symup{p}_{n-1}}_{n-1,\symup{z}} z }{ 2 }} ~ \mathbb{d}{\symbf q} \label{eq:2-267c} \\ &= \Upsilon^{\symup{p}_{n,n-1}}_{nz} \frac{ a_j }{ 2 } \mathbb{e}^{ - \mathbb{i} \frac{b_j \pm 1}{b_j} \frac{ k^{\symup{p}_{n}}_{\symup{z}} z }{ 2 }} \iiint C_{\symbf q} {\mathtt{G}}^{\symup{p}_{n-1}}_{n-1,\frac{b_j \pm 1}{b_j} \frac{ z }{ 2 }} \mathbb{e}^{\mathbb{i} \frac{b_j \pm 1}{b_j} \frac{ q_{\symup{z}} z }{ 2 }} ~ \mathbb{d}{\symbf q} \label{eq:2-267d} \\ &= \Upsilon^{\symup{p}_{n,n-1}}_{nz} \frac{ a_j }{ 2 } \mathbb{e}^{ - \mathbb{i} k^{\symup{p}_{n}}_{\symup{z}} \frac{b_j \pm 1}{b_j} \frac{ z }{ 2 }} \int C_{nq_{\symup{z}}} * {\mathtt{G}}^{\symup{p}_{n-1}}_{n,\frac{b_j \pm 1}{b_j} \frac{ z }{ 2 }} \mathbb{e}^{\mathbb{i} q_{\symup{z}} \frac{b_j \pm 1}{b_j} \frac{ z }{ 2 }} ~ \mathbb{d}{q_{\symup{z}}} \label{eq:2-267e} \\ &= \Upsilon^{\symup{p}_{n,n-1}}_{nz} \frac{ a_j }{ 2 } \mathbb{e}^{ - \mathbb{i} k^{\symup{p}_{n}}_{\symup{z}} \frac{b_j \pm 1}{b_j} \frac{ z }{ 2 }} \int \mathcal F \left[ \mathcal F_{z} \left[ M_{z} \right] {\mathtt{E}}^{\symup{p}_{n-1}}_{\frac{b_j \pm 1}{b_j} \frac{ z }{ 2 }} \right] \mathbb{e}^{\mathbb{i} q_{\symup{z}} \frac{b_j \pm 1}{b_j} \frac{ z }{ 2 }} ~ \mathbb{d}{q_{\symup{z}}} \label{eq:2-267f} \\ &=: \frac{ a_j }{ 2 } {\mathtt{g}}^{\symup{p}_{n(,n-1)}}_{\frac{b_j \pm 1}{b_j} \frac{ z }{ 2 };\text{form}} \label{eq:2-267g}~,
	\end{align}
\end{subequations}
以及 ${\mathtt{g}}^{\symup{p}_{n(,n-1)}}_{z;\cos}$ 的直流项 ${\mathtt{g}}^{\symup{p}_{n(,n-1)}}_{z;\text{DC}}$:
\begin{subequations} \label{eq:2-268}
	\abovedisplayskip=13pt
	\belowdisplayskip=13pt
	\begin{align}
		{\mathtt{g}}^{\symup{p}_{n(,n-1)}}_{z;\text{DC}} &= \Upsilon^{\symup{p}_{n,n-1}}_{nz} a_0 \iiint C_{\symbf q} {\mathtt{G}}^{\symup{p}_{n-1}}_{n-1,0} \cdot \mathbb{e}^{\mathbb{i} \frac{ \Delta k^{\symup{p}_{n-1,n}}_{n\symup{z}} z }{ 2 }} ~ \mathbb{d}{\symbf q} \label{eq:2-268a} \\ &= \Upsilon^{\symup{p}_{n,n-1}}_{nz} a_0 \mathbb{e}^{ - \mathbb{i} k^{\symup{p}_{n}}_{\symup{z}} \frac{ z }{ 2 }} \int C_{nq_{\symup{z}}} * {\mathtt{G}}^{\symup{p}_{n-1}}_{n \frac{ z }{ 2 }} \mathbb{e}^{\mathbb{i} q_{\symup{z}}  \frac{ z }{ 2 }} ~ \mathbb{d}{q_{\symup{z}}} \label{eq:2-268b} \\ &= \Upsilon^{\symup{p}_{n,n-1}}_{nz} a_0 \mathbb{e}^{ - \mathbb{i} k^{\symup{p}_{n}}_{\symup{z}} \frac{ z }{ 2 }} \int \mathcal F \left[ \mathcal F_{z} \left[ M_{z} \right] {\mathtt{E}}^{\symup{p}_{n-1}}_{ \frac{ z }{ 2 }} \right] \mathbb{e}^{\mathbb{i} q_{\symup{z}} \frac{ z }{ 2 }} ~ \mathbb{d}{q_{\symup{z}}} \label{eq:2-268c} \\ &= a_0 {\mathtt{g}}^{\symup{p}_{n(,n-1)}}_{\frac{ z }{ 2 };\text{form}} \label{eq:2-268d}~,
	\end{align}
\end{subequations}
然后,结合直流项、交流项,得到单泵浦 ${\mathtt{G}}^{\symup{p}_{n-1}}_{z}$ 未耗尽近似条件下的,折射率微扰诱导的势散射过程的时空谱(无衍射复振幅)匹配型非线性角谱解
\begin{subequations} \label{eq:2-269}
	\abovedisplayskip=13pt
	\belowdisplayskip=13pt
	\begin{align}
		{\mathtt{g}}^{\symup{p}_{n(,n-1)}}_{z;\cos} &= {\mathtt{g}}^{\symup{p}_{n(,n-1)}}_{z;\text{DC}} + {\mathtt{g}}^{\symup{p}_{n(,n-1)}}_{z;\text{AC}} = a_0 {\mathtt{g}}^{\symup{p}_{n(,n-1)}}_{\frac{ z }{ 2 };\text{form}} + \frac{ a_j }{ 2 } {\mathtt{g}}^{\symup{p}_{n(,n-1)}}_{\frac{b_j \pm 1}{b_j} \frac{ z }{ 2 };\text{form}} \label{eq:2-269a} \\ &= \Upsilon^{\symup{p}_{n,n-1}}_{nz} \int \begin{pmatrix} a_0 ~ \mathcal F \left[ \mathcal F_{z} \left[ M_{z} \right] {\mathtt{E}}^{\symup{p}_{n-1}}_{ \frac{ z }{ 2 }} \right] \mathbb{e}^{ \mathbb{i} \left( q_{\symup{z}} - k^{\symup{p}_{n}}_{\symup{z}} \right) \frac{ z }{ 2 }} \\ + \displaystyle{\frac{ a_j }{ 2 }} ~ \mathcal F \left[ \mathcal F_{z} \left[ M_{z} \right] {\mathtt{E}}^{\symup{p}_{n-1}}_{\frac{b_j \pm 1}{b_j} \frac{ z }{ 2 }} \right] \mathbb{e}^{ \mathbb{i} \left( q_{\symup{z}} - k^{\symup{p}_{n}}_{\symup{z}} \right) \frac{b_j \pm 1}{b_j} \frac{ z }{ 2 }} \end{pmatrix} \mathbb{d}{q_{\symup{z}}} \label{eq:2-269b}~,
	\end{align}
\end{subequations}
最后,得到完整的、带衍射项 $\mathbb{e}^{\mathbb{i} k^{\symup{p}_{n}}_{\symup{z}} z }$ 的折射率微扰诱导的势散射过程的时空谱(复振幅)匹配型非线性角谱解
\begin{subequations} \label{eq:2-270}
	\abovedisplayskip=13pt
	\belowdisplayskip=13pt
	\begin{align}
		\hspace{-0.5em} {\mathtt{G}}^{\symup{p}_{n(,n-1)}}_{z;\cos} &= {\mathtt{g}}^{\symup{p}_{n(,n-1)}}_{z;\cos} \mathbb{e}^{\mathbb{i} k^{\symup{p}_{n}}_{\symup{z}} z } = {\mathtt{G}}^{\symup{p}_{n(,n-1)}}_{z;\text{DC}} + {\mathtt{G}}^{\symup{p}_{n(,n-1)}}_{z;\text{AC}} =: a_0 {\mathtt{G}}^{\symup{p}_{n(,n-1)}}_{\frac{ z }{ 2 };\text{form}} + \frac{ a_j }{ 2 } {\mathtt{G}}^{\symup{p}_{n(,n-1)}}_{\frac{b_j \pm 1}{b_j} \frac{ z }{ 2 };\text{form}} \label{eq:2-270a} \\ &= \Upsilon^{\symup{p}_{n,n-1}}_{nz} \int \begin{pmatrix} a_0 ~ \mathcal F \left[ \mathcal F_{z} \left[ M_{z} \right] {\mathtt{E}}^{\symup{p}_{n-1}}_{ \frac{ z }{ 2 }} \right] \mathbb{e}^{ \mathbb{i} \left( q_{\symup{z}} + k^{\symup{p}_{n}}_{\symup{z}} \right) \frac{ z }{ 2 }} \\ + \displaystyle{\frac{ a_j }{ 2 }} ~ \mathcal F \left[ \mathcal F_{z} \left[ M_{z} \right] {\mathtt{E}}^{\symup{p}_{n-1}}_{\frac{b_j \pm 1}{b_j} \frac{ z }{ 2 }} \right] \mathbb{e}^{ \mathbb{i} \left( q_{\symup{z}} \frac{b_j \pm 1}{b_j} + k^{\symup{p}_{n}}_{\symup{z}} \frac{b_j \mp 1}{b_j} \right) \frac{ z }{ 2 }} \end{pmatrix} \mathbb{d}{q_{\symup{z}}} \label{eq:2-270b}~,
	\end{align}
\end{subequations}
可以看出,所有级数型非线性角谱解的线性衍射的相位部分,在某种程度上,是参与相互作用的波矢、倒格矢的算术平均,所构成的新混合波矢,在晶体内的衍射贡献的;原本的线性衍射相位的权重只占约一半左右。

在使用这一小节的级数解,对非线性过程进行实际计算时,不一定直接一步计算至晶体后端面,即步长不一定为晶体长度 $z \neq L$,而是可以每次只算 $z = {\Delta z}_i$,通过纵向迭代数 $I$ 次迭代,得到晶体后端面的时空谱分布;具体的纵向迭代式,可以是迭代型或求和型:
\begin{subequations} \label{eq:2-271}
	\abovedisplayskip=13pt
	\belowdisplayskip=13pt
	\begin{align}
		{\mathtt{g}}_{z} = {\mathtt{g}}_{z_I} &= {\mathtt{g}}_{z_{I-1}} \cdot \mathbb{e}^{\mathbb{i} k_{\symup{z}} {\Delta z}_{I}} + \mathbb{d} {\mathtt{g}}_{z_{I}} \label{eq:2-271a} \\ &= \left( {\mathtt{g}}_{z_{I-2}} \cdot \mathbb{e}^{\mathbb{i} k_{\symup{z}} {\Delta z}_{I-1}} + \mathbb{d} {\mathtt{g}}_{z_{I-1}} \right) \cdot \mathbb{e}^{\mathbb{i} k_{\symup{z}} {\Delta z}_{I}} + \mathbb{d} {\mathtt{g}}_{z_{I}} \label{eq:2-271b} \\ &= \sum^{I}_{i=1} \mathbb{d} {\mathtt{g}}_{z_{i}} \cdot \mathbb{e}^{\mathbb{i} k_{\symup{z}} \left( z_I - z_i \right)} = \mathbb{d} {\mathtt{g}}_{z_{i}} \cdot \mathbb{e}^{\mathbb{i} k_{\symup{z}} \left( z_I - z_i \right)} \label{eq:2-271c} ~,
	\end{align}
\end{subequations}
其中,爱因斯坦约定下的角标 $i \in \left[ 1, I \right]$,类似 Eq.(\ref{eq:2-255a}) 中的 $j \in \left[ 1, J \right]$;此外,规定了总距离(晶体长度)$z = z_I = L$ 与各步长 $z_i$ 的关系
\begin{subequations} \label{eq:2-272}
	\abovedisplayskip=13pt
	\belowdisplayskip=13pt
	\begin{align}
		z = z_I &= z_{I-1} + {\Delta z}_{I} \label{eq:2-272a} \\ &= \left( z_{I-2} + {\Delta z}_{I-1} \right) + {\Delta z}_{I} \label{eq:2-272b} \\ &= \sum^{I}_{i=1} {\Delta z}_{i} \label{eq:2-272c} ~, \\ \text{where}\ \ \ \ \ \ {\Delta z}_{1} &= z_1 - z_0 = z_1 \label{eq:2-272d} ~,
	\end{align}
\end{subequations}
一般地,如果晶体在通光方向,即纵向或 $z$ 向有以 $\Lambda_{\symup{z}}$ 为周期的一阶或二阶非线性系数调制 $M_z$,则要求每一个步长 ${\Delta z}_{i}$ 都被纵向调制周期 $\Lambda_{\symup{z}}$ 整除(定义 $"\Big|"$ 为“被整除”算符),即
\begin{equation} \label{eq:2-273}
	\Lambda_{\symup{z}} \Big| {\Delta z}_{i} ~, \ \ \ \text{for} \ \ \ i \in \left[ 1, I \right] ~,
\end{equation}
否则,可能会出现数值问题。

为了简化,前 $I-1$ 步或前 $I$ 步的步长可统一为 $\Delta z_i = \Delta z$,此时
\begin{subequations} \label{eq:2-274}
	\abovedisplayskip=13pt
	\belowdisplayskip=13pt
	\begin{align}
		\Delta z_i &= \begin{cases}
			\Delta z ~, &\quad \text{for} \ \ \ i \in \left[ 1, I-1 \right]\\
			z - \left( I-1 \right) \Delta z ~, &\quad \text{for} \ \ \ i = I \\
		\end{cases} \label{eq:2-274a} \\ I &= \lceil \frac{z}{\Delta z} \rceil \label{eq:2-274b} ~,
	\end{align}
\end{subequations}
其中,算符 $\lceil \cdot \rceil$ 表示向上取整。注意,上述 Eq.(\ref{eq:2-274a}) 也需要满足 Eq.(\ref{eq:2-273})。

在特定步长 $\Delta z$ 和具体匹配程度 $\left|\Delta k_{\symup{z}}\right|_{\max}$ 下,和弦级数 Eq.(\ref{eq:2-255b}) 的项 $J$ 在超过某一正整数后,计算结果将完全准确且保持不变;因此对于任何实际情况,$J$ 都是有限的(只要 $J$ 取得够大,足以覆盖任何情况,包括完全失配),对应 $J$ 的取值是某个正整数(半开集的下限为最佳,而不是更大或无限大),这也就意味着为保证该傅立叶变换解的准确性,$J$ 一般取满足
\begin{equation} \label{eq:2-275}
	\left| \frac{ \Delta k_{\symup{z}} \Delta z }{ 2 }  \right|_{\max} \leq \left( J+1 \right) \pi
\end{equation}
的最小正整数值
\begin{subequations} \label{eq:2-276}
	\abovedisplayskip=13pt
	\belowdisplayskip=13pt
	\begin{align}
		J = J_{\min} &= \lceil \left| \frac{ \Delta k_{\symup{z}} \Delta z }{ 2 \pi }  \right|_{\max} \rceil - 1 \label{eq:2-276a} \\ J \left( \left| \Delta k_{\symup{z}} \right|_{\max}, \Delta z \right) &= \lceil \frac{ \left| \Delta k_{\symup{z}} \right|_{\max} \Delta z }{ 2 \pi } \rceil - 1 \label{eq:2-276b} ~,
	\end{align}
\end{subequations}
可以看出,为同时保证精度和速度,“横向阶数”的最小值 $J = J_{\min}$ 与最大纵向失配量 $\left|\Delta k_{\symup{z}}\right|_{\max}$ 和步长 $\Delta z$ 这两个因素均有关。

当纵向波矢失配量 $\left|\Delta k_{\symup{z}}\right|$ 较大或步长 $\Delta z$ 较长时,$\left|\Delta k_{\symup{z}} \Delta z\right|_{\max}$ 较大,此时 $J = J_{\min}$ 较大,导致这种级数算法的求和项数 $2 \left[ J - \text{discard} \left( J \right) \right] + 1 = c_J J \in \left( J, 2J + 1 \right]$ 较多(实际横向求和数或比 $2J$ 略小:当 $J \geq 5$ 较大时,$c_J$ 在数值上接近 $1.\symup{x}_J$,见 Eq.(\ref{eq:2-256d}) 和下文),计算速度和效率下降;因此,这种级数傅立叶变换解,只能在匹配点附近,且晶体/步长不太长的情况下,实现对无耦合非线性卷积的快速、准确计算,这也是该算法被称为匹配型非线性角谱解的原因:失配情况下,计算量较大;尽管无论匹配还是失配,以及它们的程度如何,都能算,且能算准。

一般地,紧聚焦或波长、温度、角度不合适,会导致纵向波矢失配的最大值 $\left|\Delta k_{\symup{z}}\right|_{\max}$ 较大;那么,为保证纵向波矢失配积累的相位最大值 $\left|\Delta k_{\symup{z}} \Delta z\right|_{\max}$ 不大,步长 $\Delta z$ 必须较小(可能小于晶体长度 $z = L$),以使对每一步长的“横向阶数” $J = J_{\min}$ 较小;但若每一步的步长 $\Delta z$ 较小,则迭代完整个晶体长度,所需的纵向迭代数 $I$ 却又较大。可以发现,这两者间是制约关系,横向阶数 $J$ 小了,纵向迭代数 $I$ 就得大;因此,在保证精确且运算量最小(即 $J = J_{\min}$)的情况下,横纵二维(双层 for 循环)总迭代数
\begin{subequations} \label{eq:2-277}
	\abovedisplayskip=13pt
	\belowdisplayskip=13pt
	\begin{align}
		I \cdot c_J J &= \lceil \frac{z}{\Delta z} \rceil \cdot c_J \left( \lceil \frac{ \left| \Delta k_{\symup{z}} \right|_{\max} \Delta z }{ 2 \pi } \rceil - 1 \right) \label{eq:2-277a} \\ &\gtrsim c_J \left( \lceil \frac{ \left| \Delta k_{\symup{z}} \right|_{\max} z }{ 2 \pi } \rceil - \lceil \frac{z}{\Delta z} \rceil \right) \label{eq:2-277b} ~,
	\end{align}
\end{subequations}
这个横纵总迭代数 $I \cdot c_J J$ 即 Eq.(\ref{eq:2-277b}),在对晶体量级进行直接一步到位的计算,即步长等于晶体长度 $\Delta z = z$ 时,存在一个最大值
\begin{equation} \label{eq:2-278}
	\left( I \cdot c_J J \right)_{\max} \simeq c_J \left( \lceil \frac{ \left| \Delta k_{\symup{z}} \right|_{\max} z }{ 2 \pi } \rceil - 1 \right) ~,
\end{equation}
这是理所应当的:因为无论什么算法,在指定算法后,算法误差 $\text{Error} \left( I \right)$ 随着纵向步数/迭代数 $I$ 的线性增加,一般是指数级减小的;而随着横向阶数 $J$ 的线性增加,这一小节的级数算法的算法误差 $\text{Error} \left( I, J \right)$ 是平方倍减小的(对 $\left|\Delta k_{\symup{z}}\right|_{\max}$ 的限制最终会落脚于二维横向空间频率的视场范围);以至于,提升 $J$,没有提升相同倍数的 $I$ 的性价比高,所以导致指定精度下,存在一个性价比最低的最大总迭代数 $\left( I \cdot c_J J \right)_{\max}$ 即 Eq.(\ref{eq:2-278})。

此外,随纵向迭代数 $I$ 线性增加,计算量 $Time \left( I \right)$ 一般也线性增加,但算法误差 $\text{Error} \left( I \right)$ 一般指数级减小,这也是为什么再差的算法,只要使用 Eq.(\ref{eq:2-271c}) 迭代,基本都是准确的;这也是分步傅立叶算法(Split-step Fourier transform, SSF)从结果上看能成功实施的主要原因;因此,不同的算法之间好坏,一般用相同精度下的时间复杂度/计算量,或相同计算量下的精度来衡量,或比较不同的算法的效率与精度的乘积。

在保证精确的前提下,该小节的级数解的横纵总迭代数 $I \cdot c_J J$,也存在一个最小值
\begin{subequations} \label{eq:2-279}
	\abovedisplayskip=13pt
	\belowdisplayskip=13pt
	\begin{align}
		\left( I \cdot c_J J \right)_{\min} &= \lceil \frac{z}{\Delta z} \rceil \cdot 2 \label{eq:2-279a} \\ &= 2 \lceil \frac{\left| \Delta k_{\symup{z}} \right|_{\max} z}{2 \pi} \rceil \label{eq:2-279b} ~,
	\end{align}
\end{subequations}
此时,$J = 1$ 且 $c_J = 2$;这来源于 Eq.(\ref{eq:2-255a}) 中横向阶数 $J \geq 1$ 这个下限限制:
\begin{subequations} \label{eq:2-280}
	\abovedisplayskip=13pt
	\belowdisplayskip=13pt
	\begin{align}
		J = \lceil \frac{ \left| \Delta k_{\symup{z}} \right|_{\max} \Delta z }{ 2 \pi } \rceil - 1 &\geq 1 \label{eq:2-280a} \\ \lceil \frac{ \left| \Delta k_{\symup{z}} \right|_{\max} \Delta z }{ 2 \pi } \rceil &\geq 2 \label{eq:2-280b} \\ \frac{ \left| \Delta k_{\symup{z}} \right|_{\max} \Delta z }{ 2 \pi } &\geq 1 \label{eq:2-280c} \\ \Delta z &\geq \frac{ 2 \pi }{ \left| \Delta k_{\symup{z}} \right|_{\max} } \label{eq:2-280d} ~,
	\end{align}
\end{subequations}
所导致的步长最小值 $\Delta z_{\min} = 2 \pi \big/ \left| \Delta k_{\symup{z}} \right|_{\max}$ 代入横向迭代数 $c_J J = 2$ 最小情况下的 Eq.(\ref{eq:2-277a}) 所得;因为从 Eq.(\ref{eq:2-277b}) 可见,横纵总迭代数 $\left( I \cdot c_J J \right) \left( \Delta z \right)$ 大体是步长 $\Delta z$ 的单调函数;此时级数解本质上只含一个余弦 $\cos$ 函数,对于每一步长,在横向上只需通过欧拉公式算两次 $\mathbb{e}$ 指数即可。

因此,总的来说,横纵总迭代数 $I \cdot c_J J$ 的值域和对应的定义域如下:
\begin{subequations} \label{eq:2-281}
	\abovedisplayskip=13pt
	\belowdisplayskip=13pt
	\begin{alignat}{4}
		& \quad && 2 \lceil \frac{\left| \Delta k_{\symup{z}} \right|_{\max} z}{2 \pi} \rceil \leq \quad && \hspace{0.3em} I \cdot c_J J \quad && \lesssim c_J \left( \lceil \frac{ \left| \Delta k_{\symup{z}} \right|_{\max} z }{ 2 \pi } \rceil - 1 \right) \label{eq:2-281a}~, \\
		\text{for} & \quad && \hspace{2.1em} \frac{ 2 \pi }{ \left| \Delta k_{\symup{z}} \right|_{\max} } \leq && \hspace{1.1em} \Delta z && \leq z \label{eq:2-281b}~, \\
		\text{and for} & \quad && \hspace{5.4em} 1 \leq && \hspace{1.4em} J && \leq \lceil \frac{ \left| \Delta k_{\symup{z}} \right|_{\max} z }{ 2 \pi } \rceil - 1 \label{eq:2-281c}~, \\
		\text{and for} & \quad && \hspace{0.5em} \lceil \frac{\left| \Delta k_{\symup{z}} \right|_{\max} z}{2 \pi} \rceil \geq && \hspace{1.4em} I && \geq 1 \label{eq:2-281d}~,
	\end{alignat}
\end{subequations}
不过要注意,横纵总迭代数 $I \cdot c_J J$ 随 $J$ 的变化可能不是单调的,因为 $c_J$ 还与 $J$ 有关,且 $c_J$ 随 $J$ 的增加,不是单调的;尽管如此,$\text{discard} \left( J \right)$ 仍是单调不减的;此外,对 $\text{discard} \left( J \right)$ 的研究有助于揭开和弦级数 Eq.(\ref{eq:2-255a}) 通项的递推方法的解析形式。

在保证准确的同时,横纵总迭代数最少(效率最高)的条件下,对于该小节的级数解,步长有一个最小值 $\Delta z_{\min} = 2 \pi \big/ \left| \Delta k_{\symup{z}} \right|_{\max}$;如果此时 $\left| \Delta k_{\symup{z}} \right|_{\max}$ 又很大,则步长 $\Delta z = \Delta z_{\min}$ 会很小,以至于可能不满足 Eq.(\ref{eq:2-273}) 的纵向周期整除步长 $\Lambda_{\symup{z}} \Big| {\Delta z}$ 的条件,比如若此时 $\Lambda_{\symup{z}}$ 纵向周期又较大的话;然而 Eq.(\ref{eq:2-273}) 一般又是必须要满足的,因此级数解的纵向迭代有条件限制,需要找到无条件限制的纵向迭代解:接下来的迭代型非线性角谱解就可满足这一个需求。

%\subsection{和/差频、折射率微扰势散射的迭代型非线性角谱解}
\subsection{\protect\hyperlink{chap:\thesubsection}{和/差频、折射率微扰势散射的迭代型非线性角谱解}}
\addtocontents{toc}{\protect\linkdest{chap:\thesubsection}}
\label{和/差频、折射率微扰势散射的迭代型非线性角谱解}

从 Eq.(\ref{eq:2-276a}) 可见,上一节的级数型非线性角谱解,其精度同时与最大纵向失配量 $\left|\Delta k_{\symup{z}}\right|_{\max}$ 和步长 $\Delta z$ 这两个因素有关,以至于只有当波矢的纵向分量接近匹配的情况下,级数解的精度速度积才有优势;对于失配的情况,级数解的精度速度积优势不大;为克服这一个缺点,这一节将导出与纵向失配量 $\left|\Delta k_{\symup{z}}\right|$ 几乎无关的、只与步长 $\Delta z$ 有关的迭代型非线性角谱解。

通过异于 Eq.(\ref{eq:2-258}) 的另一途径,处理和频时空谱耦合波方程的非线性卷积解 Eq.(\ref{eq:2-241b}),可得到双泵浦 ${\mathtt{G}}^{\symup{p}_{1}}_{z}, {\mathtt{G}}^{\symup{p}_{2}}_{z}$ 未耗尽近似条件下的,和频时空谱(无衍射复振幅)迭代型非线性角谱解:
\begin{subequations} \label{eq:2-282}
	\abovedisplayskip=13pt
	\belowdisplayskip=13pt
	\begin{align}
		{\mathtt{g}}^{\symup{p}_{3(12)}}_{z;\text{sinc}} &= \Upsilon^{\symup{p}_{312}}_{3z} \iiint C_{\symbf q} \iint {\mathtt{G}}^{\symup{p}_{1}}_{10} {\mathtt{G}}^{\symup{p}_{2}}_{20} \cdot \text{sinc} \left( \frac{ \Delta k^{\symup{p}_{123}}_{3\symup{z}} z }{ 2 } \right) \mathbb{e}^{\mathbb{i} \frac{ \Delta k^{\symup{p}_{123}}_{3\symup{z}} z }{ 2 }} ~ \mathbb{d}{\symbf k_{1\symup{\rho}}} \mathbb{d}{\symbf q} \label{eq:2-282a} \\ &\hspace{-3.5em} \xrightarrow[\Delta k^{\symup{p}_{123}}_{3\symup{z}} \to \Delta k^{\symup{p}_{\overline{12}3}}_{3\symup{z}}]{z\ \text{is not so big}} \Upsilon^{\symup{p}_{312}}_{3z} \iiint C_{\symbf q} \iint {\mathtt{G}}^{\symup{p}_{1}}_{10} {\mathtt{G}}^{\symup{p}_{2}}_{20} \cdot \mathbb{e}^{\mathbb{i} \frac{ \Delta k^{\symup{p}_{123}}_{3\symup{z}} z }{ 2 }} ~ \mathbb{d}{\symbf k_{1\symup{\rho}}} \mathbb{d}{\symbf q} \cdot \text{sinc} \left( \frac{ \Delta k^{\symup{p}_{\overline{12}3}}_{3\symup{z}} z }{ 2 } \right) \label{eq:2-282b} \\ &\hspace{-3.5em} \xrightarrow[\text{ignore}\ a_0]{\text{Eq.(\ref{eq:2-259})}} \frac{{\mathtt{g}}^{\symup{p}_{3(12)}}_{z;\text{DC}}}{a_0} \cdot \text{sinc} \left( \frac{ \Delta k^{\symup{p}_{\overline{12}3}}_{3\symup{z}} z }{ 2 } \right) = {\mathtt{g}}^{\symup{p}_{3(12)}}_{\frac{ z }{ 2 };\text{form}} \cdot \text{sinc} \left( \frac{ \Delta k^{\symup{p}_{\overline{12}3}}_{3\symup{z}} z }{ 2 } \right) \label{eq:2-282c}~,
	\end{align}
\end{subequations}
以及和频时空谱(复振幅)迭代型非线性角谱解:
\begin{subequations} \label{eq:2-283}
	\abovedisplayskip=13pt
	\belowdisplayskip=13pt
	\begin{align}
		{\mathtt{G}}^{\symup{p}_{3(12)}}_{z;\text{sinc}} \xrightarrow[\text{ignore}\ a_0]{\text{Eq.(\ref{eq:2-261})}} &\ {\mathtt{G}}^{\symup{p}_{3(12)}}_{\frac{ z }{ 2 };\text{form}} \cdot \text{sinc} \left( \frac{ \Delta k^{\symup{p}_{\overline{12}3}}_{3\symup{z}} z }{ 2 } \right) \label{eq:2-283a}~, \\ \text{where}\ \ \ \ \text{2-D} \ \ \ \ \Delta k^{\symup{p}_{\overline{12}3}}_{3\symup{z}} :=&\ k^{\symup{p}_{\overline{12}}}_{3\symup{z}} - k^{\symup{p}_{3}}_{\symup{z}} := \overline{k}^{\symup{p}_{1}}_{\symup{z}} + \overline{k}^{\symup{p}_{2}}_{\symup{z}} + q_{\symup{z}} - k^{\symup{p}_{3}}_{\symup{z}} \label{eq:2-283b}~, \\ \text{in which}\ \ \ \ \text{0-D} \hspace{2.42em} \overline{k}^{\symup{p}_{i}}_{\symup{z}} :=&\ \frac{ \int \left| {\mathtt{G}}^{\symup{p}_{i}}_{0} \left( \symbf k_{\symup{\rho}} \right) \right|^2 k^{\symup{p}_{i}}_{\symup{z}} \left( \symbf k_{\symup{\rho}} \right) \mathbb{d}{\symbf k_{\symup{\rho}}} }{ \int \left| {\mathtt{G}}^{\symup{p}_{i}}_{0} \left( \symbf k_{\symup{\rho}} \right) \right|^2 \mathbb{d}{\symbf k_{\symup{\rho}}} } \label{eq:2-283c}~,
	\end{align}
\end{subequations}
其中,从 Eq.(\ref{eq:2-240d}) 来看,$k^{\symup{p}_{2}}_{\symup{z}}$ 本身是 $6$ 维的($\symbf k_{3\symup{\rho}}, \symbf k_{1\symup{\rho}}, \symbf q_{\symup{\rho}}$ 的函数),取平均后压缩至现在 $\overline{k}^{\symup{p}_{2}}_{\symup{z}}$ 的 $0$ 维;以至于 $\Delta k^{\symup{p}_{123}}_{3\symup{z}}$ 也从原来的 $6$ 维压缩至现在 $\Delta k^{\symup{p}_{\overline{12}3}}_{3\symup{z}}$ 的 $2$ 维。允许这样对 $k^{\symup{p}_{12}}_{3\symup{z}} \to k^{\symup{p}_{\overline{12}}}_{3\symup{z}}$ 取平均的原因是,在非紧聚焦的情况下,泵浦的倒空间发散角不大,且能量集中于某一 $k^{\symup{p}_{i}}_{\symup{z}}$ 值附近(但在 $\symbf k_{\symup{\rho}}$ 方面不一定集中)。

不对产生场的 $k^{\symup{p}_{3}}_{\symup{z}}$ 取平均,是因为本身就应该/需要、且可以保留其二维分布特征,同时因其不参与非线性卷积过程而可分离;所以综合来看,对 $k^{\symup{p}_{12}}_{3\symup{z}}$ 取平均但保留 $k^{\symup{p}_{3}}_{\symup{z}}$ 不变,是为了在不牺牲太大精度的前提下(保留 $k^{\symup{p}_{3}}_{\symup{z}}$ 不变),将非 $\mathbb{e}$ 指数部分,即 $\text{sinc}$ 部分,从非线性卷积中提取出来;这样略牺牲一点精度,但至少可化为线性卷积。

类似的过程,异于 Eq.(\ref{eq:2-263}) 地处理 Eq.(\ref{eq:2-244b}),可得到双泵浦 ${\mathtt{G}}^{\symup{p}_{3}}_{z}, {\mathtt{G}}^{\symup{p}_{2}}_{z}$ 未耗尽近似条件下的,差频时空谱(无衍射复振幅)迭代型非线性角谱解:
\begin{subequations} \label{eq:2-284}
	\abovedisplayskip=13pt
	\belowdisplayskip=13pt
	\begin{align}
		{\mathtt{g}}^{\symup{p}_{1(32)}}_{z;\text{sinc}} &= \Upsilon^{\symup{p}_{132}}_{1z} \iiint C^*_{\symbf q} \iint {\mathtt{G}}^{\symup{p}_{3}}_{30} {\mathtt{G}}^{\symup{p}_{2}*}_{20} \cdot \text{sinc} \left( \frac{ \Delta k^{\symup{p}_{321}}_{1\symup{z}} z }{ 2 } \right) \mathbb{e}^{\mathbb{i} \frac{ \Delta k^{\symup{p}_{321}}_{1\symup{z}} z }{ 2 }} ~ \mathbb{d}{\symbf k_{3\symup{\rho}}} \mathbb{d}{\symbf q} \label{eq:2-284a} \\ &\hspace{-3.5em} \xrightarrow[\Delta k^{\symup{p}_{321}}_{1\symup{z}} \to \Delta k^{\symup{p}_{\overline{32}1}}_{1\symup{z}}]{z\ \text{is not so big}} \Upsilon^{\symup{p}_{132}}_{1z} \iiint C^*_{\symbf q} \iint {\mathtt{G}}^{\symup{p}_{3}}_{30} {\mathtt{G}}^{\symup{p}_{2}*}_{20} \cdot \mathbb{e}^{\mathbb{i} \frac{ \Delta k^{\symup{p}_{321}}_{1\symup{z}} z }{ 2 }} ~ \mathbb{d}{\symbf k_{3\symup{\rho}}} \mathbb{d}{\symbf q} \cdot \text{sinc} \left( \frac{ \Delta k^{\symup{p}_{\overline{32}1}}_{1\symup{z}} z }{ 2 } \right) \label{eq:2-284b} \\ &\hspace{-3.5em} \xrightarrow[\text{ignore}\ a_0]{\text{Eq.(\ref{eq:2-264})}} \frac{{\mathtt{g}}^{\symup{p}_{1(32)}}_{z;\text{DC}}}{a_0} \cdot \text{sinc} \left( \frac{ \Delta k^{\symup{p}_{\overline{32}1}}_{1\symup{z}} z }{ 2 } \right) = {\mathtt{g}}^{\symup{p}_{1(32)}}_{\frac{ z }{ 2 };\text{form}} \cdot \text{sinc} \left( \frac{ \Delta k^{\symup{p}_{\overline{32}1}}_{1\symup{z}} z }{ 2 } \right) \label{eq:2-284c}~,
	\end{align}
\end{subequations}
以及差频时空谱(复振幅)迭代型非线性角谱解:
\begin{subequations} \label{eq:2-285}
	\abovedisplayskip=13pt
	\belowdisplayskip=13pt
	\begin{align}
		{\mathtt{G}}^{\symup{p}_{1(32)}}_{z;\text{sinc}} \xrightarrow[\text{ignore}\ a_0]{\text{Eq.(\ref{eq:2-266})}} &\ {\mathtt{G}}^{\symup{p}_{1(32)}}_{\frac{ z }{ 2 };\text{form}} \cdot \text{sinc} \left( \frac{ \Delta k^{\symup{p}_{\overline{32}1}}_{1\symup{z}} z }{ 2 } \right) \label{eq:2-285a} ~, \\ \text{where}\ \ \ \ \text{2-D}\ \ \ \ \Delta k^{\symup{p}_{\overline{32}1}}_{1\symup{z}} :=&\ k^{\symup{p}_{\overline{32}}}_{1\symup{z}} - k^{\symup{p}_{1}}_{\symup{z}} := \overline{k}^{\symup{p}_{3}}_{\symup{z}} - \overline{k}^{\symup{p}_{2}}_{\symup{z}} - q_{\symup{z}} - k^{\symup{p}_{1}}_{\symup{z}} \label{eq:2-285b} ~, \\ \text{in which} \ \ \ \ \text{0-D} \hspace{2.42em} \overline{k}^{\symup{p}_{i}}_{\symup{z}} :=&\ \frac{ \int \left| {\mathtt{G}}^{\symup{p}_{i}}_{0} \right|^2 k^{\symup{p}_{i}}_{\symup{z}} ~ \mathbb{d}{\symbf k_{\symup{\rho}}} }{ \int \left| {\mathtt{G}}^{\symup{p}_{i}}_{0} \right|^2 \mathbb{d}{\symbf k_{\symup{\rho}}} } \label{eq:2-285c}~.
	\end{align}
\end{subequations}

同样的手法,异于 Eq.(\ref{eq:2-267}) 地处理 Eq.(\ref{eq:2-252b}),可得到单泵浦 ${\mathtt{G}}^{\symup{p}_{n-1}}_{z}$ 未耗尽近似条件下的,折射率微扰诱导的势散射过程的时空谱(无衍射复振幅)迭代型非线性角谱解:
\begin{subequations} \label{eq:2-286}
	\abovedisplayskip=13pt
	\belowdisplayskip=13pt
	\begin{align}
		{\mathtt{g}}^{\symup{p}_{n(,n-1)}}_{z;\text{sinc}} &= \Upsilon^{\symup{p}_{n,n-1}}_{nz} \iiint C_{\symbf q} {\mathtt{G}}^{\symup{p}_{n-1}}_{n-1,0} \cdot \text{sinc} \left( \frac{ \Delta k^{\symup{p}_{n-1,n}}_{n\symup{z}} z }{ 2 } \right) \mathbb{e}^{\mathbb{i} \frac{ \Delta k^{\symup{p}_{n-1,n}}_{n\symup{z}} z }{ 2 }} ~ \mathbb{d}{\symbf q} \label{eq:2-286a} \\ &\hspace{-3.5em} \xrightarrow[\Delta k^{\symup{p}_{n-1,n}}_{n\symup{z}} \to \Delta k^{\symup{p}_{\overline{n-1},n}}_{n\symup{z}}]{z\ \text{is not so big}} \Upsilon^{\symup{p}_{n,n-1}}_{nz} \iiint C_{\symbf q} {\mathtt{G}}^{\symup{p}_{n-1}}_{n-1,0} \cdot \mathbb{e}^{\mathbb{i} \frac{ \Delta k^{\symup{p}_{n-1,n}}_{n\symup{z}} z }{ 2 }} ~ \mathbb{d}{\symbf q} \cdot \text{sinc} \left( \frac{ \Delta k^{\symup{p}_{\overline{n-1},n}}_{n\symup{z}} z }{ 2 } \right) \label{eq:2-286b} \\ &\hspace{-3.5em} \xrightarrow[\text{ignore}\ a_0]{\text{Eq.(\ref{eq:2-268})}} \frac{{\mathtt{g}}^{\symup{p}_{n(,n-1)}}_{z;\text{DC}}}{a_0} \cdot \text{sinc} \left( \frac{ \Delta k^{\symup{p}_{\overline{n-1},n}}_{n\symup{z}} z }{ 2 } \right) = {\mathtt{g}}^{\symup{p}_{n(,n-1)}}_{\frac{ z }{ 2 };\text{form}} \cdot \text{sinc} \left( \frac{ \Delta k^{\symup{p}_{\overline{n-1},n}}_{n\symup{z}} z }{ 2 } \right) \label{eq:2-286c}~,
	\end{align}
\end{subequations}
以及折射率微扰诱导的势散射过程的时空谱(复振幅)迭代型非线性角谱解:
\begin{subequations} \label{eq:2-287}
	\abovedisplayskip=13pt
	\belowdisplayskip=13pt
	\begin{align}
		\hspace{-1.8em} {\mathtt{G}}^{\symup{p}_{n(,n-1)}}_{z;\text{sinc}} \xrightarrow[\text{ignore}\ a_0]{\text{Eq.(\ref{eq:2-270})}} &\ {\mathtt{G}}^{\symup{p}_{n(,n-1)}}_{\frac{ z }{ 2 };\text{form}} \cdot \text{sinc} \left( \frac{ \Delta k^{\symup{p}_{\overline{n-1},n}}_{n\symup{z}} z }{ 2 } \right) \label{eq:2-287a}~, \\ \hspace{-1.8em} \text{where}\ \ \ \ \text{2-D}\ \ \ \ \Delta k^{\symup{p}_{\overline{n-1},n}}_{n\symup{z}} :=&\ \overline{k}^{\symup{p}_{n-1}}_{n\symup{z}} - k^{\symup{p}_{n}}_{\symup{z}} := \overline{k}^{\symup{p}_{n-1}}_{\symup{z}} + q_{\symup{z}} - k^{\symup{p}_{n}}_{\symup{z}} \label{eq:2-287b}~, \\ \hspace{-1.8em} \text{in which} \ \ \ \ \text{0-D} \hspace{2.2em} \overline{k}^{\symup{p}_{n-1}}_{\symup{z}} :=&\ \frac{ \sum_{\symbf k_{\symup{\rho}}} \left| {\mathtt{G}}^{\symup{p}_{n-1}}_{0} \right|^2 k^{\symup{p}_{n-1}}_{\symup{z}} }{ \sum_{\symbf k_{\symup{\rho}}} \left| {\mathtt{G}}^{\symup{p}_{n-1}}_{0} \right|^2 } \label{eq:2-287c}~.
	\end{align}
\end{subequations}

该小节的迭代型非线性角谱解的误差,几乎只与步长 $\Delta z$ 有关,与 $\left|\Delta k_{\symup{z}}\right|$ 是匹配还是失配几乎无关;并且对于每一步长,该算法在所有的 3 个算法中运算量最小,因此在任何情况下(不论匹配还是失配),该算法都是迭代时应采用的最佳算法,这也是称之为迭代型非线性角谱解的原因。

即便如此,由于该算法的误差随步长 $\Delta z$ 的增加比较明显且无法通过其他途径降低,迭代几乎是不可避免的,所以迭代型非线性角谱的计算效率,相对不会太高;但在精度速度积上,迭代型非线性角谱仍比分步傅立叶算法高。

在接近完全匹配 $\left|\Delta k_{\symup{z}}\right|_{\max} \to 0$ 时,在保证精度的情况下,匹配型/级数型非线性角谱解的步长也可以取得很大,允许纵向迭代数 $I$ 很小的同时,横向求和数 $c_J J$ 也可以不大,以至于在这些条件下,匹配型非线性角谱,比迭代型非线性角谱快。

在接近完全失配 $\left|\Delta k_{\symup{z}}\right|_{\min} \gg 0$ 时,匹配型、迭代型非线性角谱均无法兼顾效率与精度;然而,接下来的失配型非线性角谱解将填补这一不足。

%\subsection{和/差频、折射率微扰势散射的失配型非线性角谱解}
\subsection{\protect\hyperlink{chap:\thesubsection}{和/差频、折射率微扰势散射的失配型非线性角谱解}}
\addtocontents{toc}{\protect\linkdest{chap:\thesubsection}}
\label{和/差频、折射率微扰势散射的失配型非线性角谱解}

在保证精度的情况下,接近完全匹配 $\left|\Delta k_{\symup{z}}\right|_{\max} \to 0$ 时,适合用级数型/匹配型非线性角谱;部分匹配 $\left|\Delta k_{\symup{z}}\right|_{\min} \to 0$ 时,适合用迭代型非线性角谱;完全失配 $\left|\Delta k_{\symup{z}}\right|_{\min} \gg 0$ 时,除了用迭代型非线性角谱,还可以用速度精度积更高的失配型非线性角谱解,这一节就将对其进行推导。

现用不同于 \cref{eq:2-258,eq:2-282} 第 3 种方法,处理和频时空谱耦合波方程的非线性卷积解 Eq.(\ref{eq:2-241a}),可得到双泵浦 ${\mathtt{G}}^{\symup{p}_{1}}_{z}, {\mathtt{G}}^{\symup{p}_{2}}_{z}$ 未耗尽近似条件下的,和频时空谱(无衍射复振幅)失配型非线性角谱解:
\begin{subequations} \label{eq:2-288}
	\abovedisplayskip=13pt
	\belowdisplayskip=13pt
	\begin{align}
		{\mathtt{g}}^{\symup{p}_{3(12)}}_{z;\text{mis}} &= \Upsilon^{\symup{p}_{312}}_{3z} \iiint C_{\symbf q} \iint {\mathtt{G}}^{\symup{p}_{1}}_{10} {\mathtt{G}}^{\symup{p}_{2}}_{20} \cdot \frac{ \mathbb{e}^{\mathbb{i} \Delta k^{\symup{p}_{123}}_{3\symup{z}} z} - 1 }{ \Delta k^{\symup{p}_{123}}_{3\symup{z}} } ~ \mathbb{d}{\symbf k_{1\symup{\rho}}} \mathbb{d}{\symbf q} \label{eq:2-288a} \\ &\hspace{-3.5em} \xrightarrow[\Delta k^{\symup{p}_{123}}_{3\symup{z}} \to \Delta k^{\symup{p}_{\overline{12}3}}_{3\symup{z}}]{\left| \Delta k^{\symup{p}_{123}}_{3\symup{z}} \right|_{\min} \gg 0} \Upsilon^{\symup{p}_{312}}_{3z} \iiint C_{\symbf q} \iint {\mathtt{G}}^{\symup{p}_{1}}_{10} {\mathtt{G}}^{\symup{p}_{2}}_{20} \cdot \left( \mathbb{e}^{\mathbb{i} \Delta k^{\symup{p}_{123}}_{3\symup{z}} z} - 1 \right) \mathbb{d}{\symbf k_{1\symup{\rho}}} \mathbb{d}{\symbf q} \cdot \frac{ 1 }{ \Delta k^{\symup{p}_{\overline{12}3}}_{3\symup{z}} } \label{eq:2-288b} \\ &\hspace{-3.5em} \xrightarrow[\text{ignore}\ a_0]{\text{Eq.(\ref{eq:2-259})}} \frac{ {\mathtt{g}}^{\symup{p}_{3(12)}}_{z;\text{DC}} \Big|_{\frac{z}{2} \to z} - {\mathtt{g}}^{\symup{p}_{3(12)}}_{z;\text{DC}} \Big|_{\frac{z}{2} \to 0} }{ a_0 \Delta k^{\symup{p}_{\overline{12}3}}_{3\symup{z}} } = \frac{ {\mathtt{g}}^{\symup{p}_{3(12)}}_{2z;\text{DC}} - {\mathtt{g}}^{\symup{p}_{3(12)}}_{0;\text{DC}} }{ a_0 \Delta k^{\symup{p}_{\overline{12}3}}_{3\symup{z}} } = \frac{ {\mathtt{g}}^{\symup{p}_{3(12)}}_{z;\text{form}} - {\mathtt{g}}^{\symup{p}_{3(12)}}_{0;\text{form}} }{ \Delta k^{\symup{p}_{\overline{12}3}}_{3\symup{z}} } \label{eq:2-288c}~,
	\end{align}
\end{subequations}
以及和频时空谱(复振幅)失配型非线性角谱解:
\begin{subequations} \label{eq:2-289}
	\abovedisplayskip=13pt
	\belowdisplayskip=13pt
	\begin{align}
		{\mathtt{G}}^{\symup{p}_{3(12)}}_{z;\text{mis}} \xrightarrow[\text{ignore}\ a_0]{\text{Eq.(\ref{eq:2-261})}} &\ \frac{ {\mathtt{G}}^{\symup{p}_{3(12)}}_{z;\text{form}} - {\mathtt{G}}^{\symup{p}_{3(12)}}_{0;\text{form}} }{ \Delta k^{\symup{p}_{\overline{12}3}}_{3\symup{z}} } \label{eq:2-289a}~, \\ \text{where}\ \ \ \ \text{2-D} \ \ \ \ \Delta k^{\symup{p}_{\overline{12}3}}_{3\symup{z}} :=&\ k^{\symup{p}_{\overline{12}}}_{3\symup{z}} - k^{\symup{p}_{3}}_{\symup{z}} := \overline{k}^{\symup{p}_{1}}_{\symup{z}} + \overline{k}^{\symup{p}_{2}}_{\symup{z}} + q_{\symup{z}} - k^{\symup{p}_{3}}_{\symup{z}} \label{eq:2-289b}~, \\ \text{in which}\ \ \ \ \text{0-D} \hspace{2.42em} \overline{k}^{\symup{p}_{i}}_{\symup{z}} :=&\ \frac{ \int \left| {\mathtt{G}}^{\symup{p}_{i}}_{0} \right|^2 k^{\symup{p}_{i}}_{\symup{z}} ~ \mathbb{d}{\symbf k_{\symup{\rho}}} }{ \int \left| {\mathtt{G}}^{\symup{p}_{i}}_{0} \right|^2 \mathbb{d}{\symbf k_{\symup{\rho}}} } \label{eq:2-289c}~.
	\end{align}
\end{subequations}

类似地,不同于 \cref{eq:2-263,eq:2-284} 地处理 Eq.(\ref{eq:2-244a}),可得到双泵浦 ${\mathtt{G}}^{\symup{p}_{3}}_{z}, {\mathtt{G}}^{\symup{p}_{2}}_{z}$ 未耗尽近似条件下的,差频时空谱(无衍射复振幅)失配型非线性角谱解:
\begin{subequations} \label{eq:2-290}
	\abovedisplayskip=13pt
	\belowdisplayskip=13pt
	\begin{align}
		{\mathtt{g}}^{\symup{p}_{1(32)}}_{z;\text{mis}} &= \Upsilon^{\symup{p}_{132}}_{1z} \iiint C^*_{\symbf q} \iint {\mathtt{G}}^{\symup{p}_{3}}_{30} {\mathtt{G}}^{\symup{p}_{2}*}_{20} \cdot \frac{ \mathbb{e}^{\mathbb{i} \Delta k^{\symup{p}_{321}}_{1\symup{z}} z} - 1 }{ \Delta k^{\symup{p}_{321}}_{1\symup{z}} } ~ \mathbb{d}{\symbf k_{3\symup{\rho}}} \mathbb{d}{\symbf q} \label{eq:2-290a} \\ &\hspace{-3.5em} \xrightarrow[\Delta k^{\symup{p}_{321}}_{1\symup{z}} \to \Delta k^{\symup{p}_{\overline{32}1}}_{1\symup{z}}]{\left| \Delta k^{\symup{p}_{321}}_{1\symup{z}} \right|_{\min} \gg 0} \Upsilon^{\symup{p}_{132}}_{1z} \iiint C^*_{\symbf q} \iint {\mathtt{G}}^{\symup{p}_{3}}_{30} {\mathtt{G}}^{\symup{p}_{2}*}_{20} \cdot \left( \mathbb{e}^{\mathbb{i} \Delta k^{\symup{p}_{321}}_{1\symup{z}} z} - 1 \right) \mathbb{d}{\symbf k_{3\symup{\rho}}} \mathbb{d}{\symbf q} \cdot \frac{ 1 }{ \Delta k^{\symup{p}_{\overline{32}1}}_{1\symup{z}} } \label{eq:2-290b} \\ &\hspace{-3.5em} \xrightarrow[\text{ignore}\ a_0]{\text{Eq.(\ref{eq:2-264})}} \frac{ {\mathtt{g}}^{\symup{p}_{1(32)}}_{z;\text{form}} - {\mathtt{g}}^{\symup{p}_{1(32)}}_{0;\text{form}} }{ \Delta k^{\symup{p}_{\overline{32}1}}_{1\symup{z}} } \label{eq:2-290c}~,
	\end{align}
\end{subequations}
以及差频时空谱(复振幅)失配型非线性角谱解:
\begin{subequations} \label{eq:2-291}
	\abovedisplayskip=13pt
	\belowdisplayskip=13pt
	\begin{align}
		{\mathtt{G}}^{\symup{p}_{1(32)}}_{z;\text{mis}} \xrightarrow[\text{ignore}\ a_0]{\text{Eq.(\ref{eq:2-266})}} &\ \frac{ {\mathtt{G}}^{\symup{p}_{1(32)}}_{z;\text{form}} - {\mathtt{G}}^{\symup{p}_{1(32)}}_{0;\text{form}} }{ \Delta k^{\symup{p}_{\overline{32}1}}_{1\symup{z}} } \label{eq:2-291a} ~, \\ \text{where}\ \ \ \ \text{2-D}\ \ \ \ \Delta k^{\symup{p}_{\overline{32}1}}_{1\symup{z}} :=&\ k^{\symup{p}_{\overline{32}}}_{1\symup{z}} - k^{\symup{p}_{1}}_{\symup{z}} := \overline{k}^{\symup{p}_{3}}_{\symup{z}} - \overline{k}^{\symup{p}_{2}}_{\symup{z}} - q_{\symup{z}} - k^{\symup{p}_{1}}_{\symup{z}} \label{eq:2-291b} ~, \\ \text{in which} \ \ \ \ \text{0-D} \hspace{2.42em} \overline{k}^{\symup{p}_{i}}_{\symup{z}} :=&\ \frac{ \sum_{\symbf k_{\symup{\rho}}} \left| {\mathtt{G}}^{\symup{p}_{i}}_{0} \right|^2 k^{\symup{p}_{i}}_{\symup{z}} }{ \sum_{\symbf k_{\symup{\rho}}} \left| {\mathtt{G}}^{\symup{p}_{i}}_{0} \right|^2 } \label{eq:2-291c}~.
	\end{align}
\end{subequations}

同样地,异于 \cref{eq:2-267,eq:2-286} 地处理 Eq.(\ref{eq:2-252a}),可得到单泵浦 ${\mathtt{G}}^{\symup{p}_{n-1}}_{z}$ 未耗尽近似条件下的,折射率微扰诱导的势散射过程的时空谱(无衍射复振幅)失配型非线性角谱解:
\begin{subequations} \label{eq:2-292}
	\abovedisplayskip=13pt
	\belowdisplayskip=13pt
	\begin{align}
		{\mathtt{g}}^{\symup{p}_{n(,n-1)}}_{z;\text{mis}} &= \Upsilon^{\symup{p}_{n,n-1}}_{nz} \iiint C_{\symbf q} {\mathtt{G}}^{\symup{p}_{n-1}}_{n-1,0} \cdot \frac{ \mathbb{e}^{\mathbb{i} \Delta k^{\symup{p}_{n-1,n}}_{n\symup{z}} z} - 1 }{ \Delta k^{\symup{p}_{n-1,n}}_{n\symup{z}} } ~ \mathbb{d}{\symbf q} \label{eq:2-292a} \\ &\hspace{-3.5em} \xrightarrow[\Delta k^{\symup{p}_{n-1,n}}_{n\symup{z}} \to \Delta k^{\symup{p}_{\overline{n-1},n}}_{n\symup{z}}]{\left| \Delta k^{\symup{p}_{n-1,n}}_{n\symup{z}} \right|_{\min} \gg 0} \Upsilon^{\symup{p}_{n,n-1}}_{nz} \iiint C_{\symbf q} {\mathtt{G}}^{\symup{p}_{n-1}}_{n-1,0} \cdot \left( \mathbb{e}^{\mathbb{i} \Delta k^{\symup{p}_{n-1,n}}_{n\symup{z}} z} - 1 \right) \mathbb{d}{\symbf q} \cdot \frac{1}{\Delta k^{\symup{p}_{\overline{n-1},n}}_{n\symup{z}}} \label{eq:2-292b} \\ &\hspace{-3.5em} \xrightarrow[\text{ignore}\ a_0]{\text{Eq.(\ref{eq:2-268})}} \frac{ {\mathtt{g}}^{\symup{p}_{n(,n-1)}}_{z;\text{form}} - {\mathtt{g}}^{\symup{p}_{n(,n-1)}}_{0;\text{form}} }{ \Delta k^{\symup{p}_{\overline{n-1},n}}_{n\symup{z}} } \label{eq:2-292c}~,
	\end{align}
\end{subequations}
以及折射率微扰诱导的势散射过程的时空谱(复振幅)失配型非线性角谱解:
\begin{subequations} \label{eq:2-293}
	\abovedisplayskip=13pt
	\belowdisplayskip=13pt
	\begin{align}
		\hspace{-1.8em} {\mathtt{G}}^{\symup{p}_{n(,n-1)}}_{z;\text{mis}} \xrightarrow[\text{ignore}\ a_0]{\text{Eq.(\ref{eq:2-270})}} &\ \frac{ {\mathtt{G}}^{\symup{p}_{n(,n-1)}}_{z;\text{form}} - {\mathtt{G}}^{\symup{p}_{n(,n-1)}}_{0;\text{form}} }{ \Delta k^{\symup{p}_{\overline{n-1},n}}_{n\symup{z}} } \label{eq:2-293a}~, \\ \hspace{-1.8em} \text{where}\ \ \ \ \text{2-D}\ \ \ \ \Delta k^{\symup{p}_{\overline{n-1},n}}_{n\symup{z}} :=&\ \overline{k}^{\symup{p}_{n-1}}_{n\symup{z}} - k^{\symup{p}_{n}}_{\symup{z}} := \overline{k}^{\symup{p}_{n-1}}_{\symup{z}} + q_{\symup{z}} - k^{\symup{p}_{n}}_{\symup{z}} \label{eq:2-293b}~, \\ \hspace{-1.8em} \text{in which} \ \ \ \ \text{0-D} \hspace{2.2em} \overline{k}^{\symup{p}_{n-1}}_{\symup{z}} :=&\ \frac{ \int \left| {\mathtt{G}}^{\symup{p}_{n-1}}_{0} \right|^2 k^{\symup{p}_{n-1}}_{\symup{z}} ~ \mathbb{d}{\symbf k_{\symup{\rho}}} }{ \int \left| {\mathtt{G}}^{\symup{p}_{n-1}}_{0} \right|^2 \mathbb{d}{\symbf k_{\symup{\rho}}} } \label{eq:2-293c}~.
	\end{align}
\end{subequations}

该小节的失配型非线性角谱解的误差,几乎只与 $\left|\Delta k_{\symup{z}}\right|$ 是匹配还是失配有关,与步长 $\Delta z$ 几乎无关,与上一小节 \ref{和/差频、折射率微扰势散射的迭代型非线性角谱解} 的迭代型恰好相反;因此该算法是 3 个算法中,唯一一个纵向、横向均无需迭代的算法,因为即使纵向迭代了也等于自身(可以尝试推导一下);然而同时,该算法也是 3 个算法中,唯一一个存在奇点,且无法通过任何手段弥补之的算法:不仅在完全匹配 $\left|\Delta k_{\symup{z}}\right|_{\max} \to 0$,甚至在部分匹配 $\left|\Delta k_{\symup{z}}\right|_{\min} \to 0$ 时,也将因分母存在像素点们 $\{ \symbf k_{\symup{\rho}} \}$ 的值接近零而无法正常工作;因此,该算法只在彻底失配 $\left|\Delta k_{\symup{z}}\right|_{\min} \gg 0$ 时才能正常工作,一般会将条件作进一步限制,即失配量须大于一定程度,才能使用该失配解:
\begin{subequations} \label{eq:2-294}
	\abovedisplayskip=13pt
	\belowdisplayskip=13pt
	\begin{align}
		\Delta k_{\symup{z;\max}} < - &\Delta k_{\symup{z;\text{band}}} < 0 \ \ \ \ \text{or}\ \ \ \ \Delta k_{\symup{z;\min}} > \Delta k_{\symup{z;\text{band}}} > 0 \label{eq:2-294a}~, \\ \text{where}\ \ \ \ &\Delta k_{\symup{z;\text{band}}} := \Delta k_{\symup{z;\max}} - \Delta k_{\symup{z;\min}} > 0 \label{eq:2-294b}~, \\ \text{and for SFG}\ \ \ \ &\Delta k_{\symup{z;\max}} = \Delta k^{\symup{p}_{123}}_{3\symup{z;\max}} := k^{\symup{p}_{12}}_{3\symup{z;\max}} - k^{\symup{p}_{3}}_{\symup{z}} \label{eq:2-294c} \\ &\hphantom{\Delta k_{\symup{z;\max}} = \Delta k^{\symup{p}_{123}}_{3\symup{z;\max}}} := k^{\symup{p}_{1}}_{\symup{z;\max}} + k^{\symup{p}_{2}}_{\symup{z;\max}} + q_{\symup{z}} - k^{\symup{p}_{3}}_{\symup{z}} \label{eq:2-294d}~, \\ \text{and for SFG}\ \ \ \ &\Delta k_{\symup{z;\min}} = \Delta k^{\symup{p}_{123}}_{3\symup{z;\min}} := k^{\symup{p}_{12}}_{3\symup{z;\min}} - k^{\symup{p}_{3}}_{\symup{z}} \label{eq:2-294e} \\ &\hphantom{\Delta k_{\symup{z;\min}} = \Delta k^{\symup{p}_{123}}_{3\symup{z;\min}}} := k^{\symup{p}_{1}}_{\symup{z;\min}} + k^{\symup{p}_{2}}_{\symup{z;\min}} + q_{\symup{z}} - k^{\symup{p}_{3}}_{\symup{z}} \label{eq:2-294f}~, \\ \text{and for DFG}\ \ \ \ &\Delta k_{\symup{z;\max}} = \Delta k^{\symup{p}_{321}}_{1\symup{z;\max}} := k^{\symup{p}_{32}}_{1\symup{z;\max}} - k^{\symup{p}_{1}}_{\symup{z}} \label{eq:2-294g} \\ &\hphantom{\Delta k_{\symup{z;\min}} = \Delta k^{\symup{p}_{321}}_{1\symup{z;\min}}} := k^{\symup{p}_{3}}_{\symup{z;\max}} - k^{\symup{p}_{2}}_{\symup{z;\min}} - q_{\symup{z}} - k^{\symup{p}_{1}}_{\symup{z}} \label{eq:2-294h}~, \\ \text{and for DFG}\ \ \ \ &\Delta k_{\symup{z;\min}} = \Delta k^{\symup{p}_{321}}_{1\symup{z;\min}} := k^{\symup{p}_{32}}_{1\symup{z;\min}} - k^{\symup{p}_{1}}_{\symup{z}} \label{eq:2-294i} \\ &\hphantom{\Delta k_{\symup{z;\min}} = \Delta k^{\symup{p}_{321}}_{1\symup{z;\min}}} := k^{\symup{p}_{3}}_{\symup{z;\min}} - k^{\symup{p}_{2}}_{\symup{z;\max}} - q_{\symup{z}} - k^{\symup{p}_{1}}_{\symup{z}} \label{eq:2-294j}~,
	\end{align}
\end{subequations}
亦即:
\begin{equation} \label{eq:2-295}
	\Delta k_{\symup{z;\max}} < \frac{ \Delta k_{\symup{z;\min}} }{ 2 } < 0 \ \ \ \ \bigcup\ \ \ \ \Delta k_{\symup{z;\min}} > \frac{ \Delta k_{\symup{z;\max}} }{ 2 } > 0 ~,
\end{equation}
否则属于部分匹配/失配:
\begin{subequations} \label{eq:2-296}
	\abovedisplayskip=13pt
	\belowdisplayskip=13pt
	\begin{align}
		\left| \Delta k_{\symup{z}} \right|_{\max} \Delta z > 2 \left( J+1 \right) \pi\ \ \ \ &\bigcap \label{eq:2-296a} \\ \frac{ \Delta k_{\symup{z;\min}} }{ 2 } < \Delta k_{\symup{z;\max}} < 0 \ \ \ \ &\bigcap\ \ \ \ \frac{ \Delta k_{\symup{z;\max}} }{ 2 } > \Delta k_{\symup{z;\min}} > 0 \label{eq:2-296b}~,
	\end{align}
\end{subequations}
或接近完美匹配:
\begin{subequations} \label{eq:2-297}
	\abovedisplayskip=13pt
	\belowdisplayskip=13pt
	\begin{align}
		&\left| \Delta k_{\symup{z}} \right|_{\max} \Delta z \leq 2 \left( J+1 \right) \pi \label{eq:2-297a}~, \\ \text{where}\ \ \ \ &\left| \Delta k_{\symup{z}} \right|_{\max} := \max \left( \left| \Delta k_{\symup{z;\min}} \right|, \left| \Delta k_{\symup{z;\max}} \right| \right) \label{eq:2-297b}~,
	\end{align}
\end{subequations}
此时,要么使用 \ref{和/差频、折射率微扰势散射的迭代型非线性角谱解} 小节的迭代型非线性角谱,要么用 \ref{和/差频、折射率微扰势散射的匹配型非线性角谱解} 小节的匹配型非线性角谱,视 $\left|\Delta k_{\symup{z}}\right|$ 匹配程度而定。

至此,匹配、迭代、失配型的三种算法,及其所分别适用的三个条件 \cref{eq:2-275,eq:2-296,eq:2-295} 都已经给出;但最好不称之为“条件”,而是“区域”,因为即使是同一个 $\Delta k_{\symup{z}}$,由于其所对应的 $\left| \Delta k_{\symup{z}} \right|_{\max}, \Delta k_{\symup{z;\max}}, \Delta k_{\symup{z;\min}}$ 仍是二维分布的场,导致可能存在不同的 3 个横向空间频率区域 $\{ \symbf k_{\symup{\rho}} \}$,分别同时满足 \cref{eq:2-275,eq:2-296,eq:2-295} 这 3 个条件;因此同一个非线性过程,可能会用上 \ref{和/差频、折射率微扰势散射的匹配型非线性角谱解}、\ref{和/差频、折射率微扰势散射的迭代型非线性角谱解}、\ref{和/差频、折射率微扰势散射的失配型非线性角谱解} 三个小节的所有的 3 种算法,如果这三个条件所对应的横向空间频率区域 $\{ \symbf k_{\symup{\rho}} \}$ 的面积都非零的话;此外,这三种算法的适用区域,交集为空,并集为全,彼此在倒空间和谐并存,因此这三种算法一起,共同构成总的非线性角谱算法。

然而,由于迭代型算法相对其他两种算法,精度上的判据不够明确,且步长迈不大(必须迭代/精度不高),因此一般不混合使用 3 种算法,只混合使用 2 种算法:匹配型和失配型,对应的启动/实施条件变为:
\begin{subequations} \label{eq:2-298}
	\abovedisplayskip=13pt
	\belowdisplayskip=13pt
	\begin{align}
		\Delta k_{\symup{z;\max}} \leq \frac{2 \left( J+1 \right) \pi}{\Delta z} \ \ \ \ &\bigcap\ \ \ \ \Delta k_{\symup{z;\min}} > \frac{ \Delta k_{\symup{z;\max}} }{ 2 } \neq \emptyset \label{eq:2-298a}~, \\ \text{or}\ \ \ \ \Delta k_{\symup{z;\min}} \geq - \frac{2 \left( J+1 \right) \pi}{\Delta z} \ \ \ \ &\bigcap\ \ \ \ \Delta k_{\symup{z;\max}} < \frac{ \Delta k_{\symup{z;\min}} }{ 2 } \neq \emptyset \label{eq:2-298b}~,
	\end{align}
\end{subequations}
此类由 2 种非线性角谱算法所构成的综合性算法,一般不在纵向迭代中使用,也不进行纵向迭代,而是直接一步计算到晶体后端面 $\Delta z = z$,以至于该过程只存在横向求和;因此这里结合了 2 种非线性角谱算法的综合性算法,或者说根据倒空间 $\{ \symbf k_{\symup{\rho}} \}$ 区域自动判断算法类型的“智能算法”,可看做 \ref{和/差频、折射率微扰势散射的匹配型非线性角谱解} 小节匹配型非线性角谱算法的拓展:减少了其在失配情况下的高计算量负荷。


