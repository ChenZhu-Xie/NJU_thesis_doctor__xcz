\marginLeft[-2.4em]{chap:N/LCO}\chapter{晶体中的(非)线性光学过程}\label{chap:N/LCO}

\vspace*{-8.5em}

\marginLeft[-2.4em]{sec:maxwell}\section{\textcolor{Maroon}{Maxwell-Lorentz-Heaviside} 方程组:\textcolor{Maroon}{$\text{场} = f(\text{源})$}}\label{sec:maxwell}

该节从经典/现代的\textcolor{Maroon}{多极理论}视角,查看场关于源的 \textcolor{Maroon}{Maxwell-Lorentz-Heaviside} 方程组、源的\textcolor{Maroon}{守恒}律、场的\textcolor{Plum}{连续}性、场/源的表面/体分布。 --- 对应“物质告诉时空怎么弯曲,时空告诉物质怎么运动”的前半句:即“源告诉场怎么弯曲”。

\vspace*{-5.0em}

\marginLeft[-1.4em]{ssec:EHpJf}\subsection{$\bar{E},\bar{H}$ 双旋度、${\rho}_{\;\!\textcolor{Maroon}{\text{f}}}, \bar{J}_{\;\!\textcolor{Maroon}{\text{f}}}$ 裸源连续}\label{ssec:EHpJf}

相对观察者静止\Footnote{否则标/矢/张量场的 2 个实\textcolor{gray}{自变量}:时间$\mathcolor{gray}{t} \in \mathcolor{gray}{\mathbb{R}_{\textcolor{Maroon}{1}}}$和空间$\mathcolor{gray}{\bar{r}} \in \mathcolor{gray}{\bar{\mathbb{R}}_{\textcolor{Plum}{3}}}$(见\bref{hook:mathbb})将在其他参考系(见\bref{hook:uwav})的度量下相互混合,即$\mathcolor{gray}{\uwav{t}} \left( \mathcolor{gray}{t}, \mathcolor{gray}{\bar{r}} \right), \mathcolor{gray}{\uwav{\bar{r}}} \left( \mathcolor{gray}{t}, \mathcolor{gray}{\bar{r}} \right)$,并形成统一的$1+3$维黎曼时空/微分流形$\mathcolor{gray}{\uwav{\bar{x}}} \left( \mathcolor{gray}{\bar{x}} \right) \in \mathcolor{gray}{\bar{\mathbb{R}}_{\textcolor{Plum}{4}}}$;接着,定义四维位移矢量$\mathcolor{gray}{\bar{x}}$对四维标量固有时$\mathcolor{gray}{\uo{t}}$(见\bref{hook:uo})的导数:四维速度矢量$\bar{u}_{\;\!\mathcolor{gray}{\bar{x}}} := \mathbb{d} \mathcolor{gray}{\bar{x}} \big/ \mathbb{d} \mathcolor{gray}{\uo{t}} \in \bar{\mathbb{C}}_{\textcolor{Plum}{4}} ( \mathcolor{gray}{\bar{\mathbb{R}}_{\textcolor{Plum}{4}}} )$,并将总电流 $\bar{J}^{\;\!\mathcolor{gray}{t}}_{\;\!\mathcolor{gray}{z}} = {\rho}^{\;\!\mathcolor{gray}{t}}_{\;\!\mathcolor{gray}{z}} \bar{v}^{\;\!\mathcolor{gray}{t}}_{\;\!\mathcolor{gray}{z}}$ 与总电荷${\rho}^{\;\!\mathcolor{gray}{t}}_{\;\!\mathcolor{gray}{z}}$合并以扩展至四维$\bar{J}_{\;\! \mathcolor{gray}{\bar{x}}} = \uo{\rho}_{\;\! \mathcolor{gray}{\uo{\bar{x}}}} \bar{u}_{\;\! \mathcolor{gray}{\bar{x}}}  \in \bar{\mathbb{C}}_{\textcolor{Plum}{4}} ( \mathcolor{gray}{\bar{\mathbb{R}}_{\textcolor{Plum}{4}}} )$\cite{lakhtakiaCovariancesInvariancesMaxwell1995,chen-zhuChenZhuxieUndergraduate_courses2024};此外,\textcolor{NavyBlue}{基本场} $\bar{E}_{\;\!\mathcolor{gray}{\bar{x}}},\bar{B}_{\;\!\mathcolor{gray}{\bar{x}}}$ 也将相互耦合,即$\uwav{\bar{E}}_{\;\!\mathcolor{gray}{\uwav{\bar{x}}}} \left( \bar{E}_{\;\!\mathcolor{gray}{\bar{x}}},\bar{B}_{\;\!\mathcolor{gray}{\bar{x}}} \right), \uwav{\bar{B}}_{\;\!\mathcolor{gray}{\uwav{\bar{x}}}} \left( \bar{E}_{\;\!\mathcolor{gray}{\bar{x}}},\bar{B}_{\;\!\mathcolor{gray}{\bar{x}}} \right)$,并成为一个整体:2阶4维电磁张量场 $\uwav{\bar{\bar{F}}}_{\;\!\mathcolor{gray}{\uwav{\bar{x}}}} ( \bar{\bar{F}}_{\;\!\mathcolor{gray}{\bar{x}}} ) \in \bar{\bar{\mathbb{C}}}_{\textcolor{Plum}{\left[4 \times 4\right]}} ( \mathcolor{gray}{\bar{\mathbb{R}}_{\textcolor{Plum}{4}}} )$\cite{lakhtakiaCovariancesInvariancesMaxwell1995,berryOpticalSingularitiesBianisotropic2005,chen-zhuChenZhuxieUndergraduate_courses2024},以致材料的\textcolor{Maroon}{本构关系}将呈现固有\textcolor{PineGreen}{双各向异性}$\uwav{\bar{D}}_{\;\!\mathcolor{gray}{\uwav{\bar{x}}}} [ \uwav{\bar{E}}_{\;\!\mathcolor{gray}{\uwav{\bar{x}}}} \left( \bar{E}_{\;\!\mathcolor{gray}{\bar{x}}},\bar{B}_{\;\!\mathcolor{gray}{\bar{x}}} \right) ], \uwav{\bar{H}}_{\;\!\mathcolor{gray}{\uwav{\bar{x}}}} [ \uwav{\bar{B}}_{\;\!\mathcolor{gray}{\uwav{\bar{x}}}} \left( \bar{E}_{\;\!\mathcolor{gray}{\bar{x}}},\bar{B}_{\;\!\mathcolor{gray}{\bar{x}}} \right) ]$或$\uwav{\bar{\bar{G}}}_{\;\!\mathcolor{gray}{\uwav{\bar{x}}}} [ \uwav{\bar{\bar{F}}}_{\;\!\mathcolor{gray}{\uwav{\bar{x}}}} ( \bar{\bar{F}}_{\;\!\mathcolor{gray}{\bar{x}}} ) ] = \uwav{\bar{\bar{G}}}_{\;\!\mathcolor{gray}{\uwav{\bar{x}}}} [ \bar{\bar{G}}_{\;\!\mathcolor{gray}{\bar{x}}} ( \bar{\bar{F}}_{\;\!\mathcolor{gray}{\bar{x}}} ) ]$\cite{langeMultipoleTheoryHehl2015,hehlLinearMediaClassical2005,mackayElectromagneticAnisotropyBianisotropy2019,mackayModernAnalyticalElectromagnetic2020,lakhtakiaCovariancesInvariancesMaxwell1995}。此外,${\rho}^{\;\!\mathcolor{gray}{t}}_{\;\!\mathcolor{gray}{z}}$对应的束缚/自由=价带/导带电子的(有效)运动质量(或相对论速度)也会变大,运动轨迹改变\cite{boydNonlinearOptics2019},并因此同时对\textcolor{Plum}{线性}和\textcolor{Plum}{非线性}极化、磁化、自由电流$\bar{J}^{\;\!\mathcolor{gray}{t}}_{\;\!\textcolor{Maroon}{\text{f}}\mathcolor{gray}{z}} = {\rho}^{\;\!\mathcolor{gray}{t}}_{\;\!\textcolor{Maroon}{\text{f}}\mathcolor{gray}{z}} \bar{v}^{\;\!\mathcolor{gray}{t}}_{\;\!\textcolor{Maroon}{\text{f}}\mathcolor{gray}{z}}$产生额外影响。 --- 上述场景可发生在超快和强场\textcolor{NavyBlue}{泵浦}(及其通过多光子/隧穿电离产生的等离子体)中\cite{boydNonlinearOptics2019}、相对材料运动的\textcolor{Plum}{坐标系}下,甚至一直在发生在所有材料内(核库伦力导致的电子轨道运动:金的颜色\cite{boydNonlinearOptics2019}、汞常温液体\cite{pyykkoRelativisticEffectsChemistry2012}、铅酸电池的额外电压和稳定性\cite{ahujaRelativityLeadacidBattery2011};狄拉克方程描述的电子自旋运动、自旋-轨道、自旋-自旋耦合\cite{pyykkoRelativisticEffectsChemistry2012}:许多磁效应\cite{chen-zhuChenZhuxieUndergraduate_courses2024}),以至于可能(必)需要引入\textcolor{Maroon}{相对论(量子)电动力学} --- 电子一直在材料内运动,尽管材料本身相对观察者的参考系静止。此外,所有磁效应的本质和来源,似乎都可以\textcolor{NavyBlue}{纯电}的方式解释,见 \bref{ssec:EBpJ,ssec:PMQN}。} 的 3 维空间$\mathcolor{gray}{\bar{r}}$ \textcolor{Plum}{坐标系}下,在可能存在非零的时 $\mathcolor{gray}{t}$ 变自由电荷源和自由电流源${\rho}_{\;\!\textcolor{Maroon}{\text{f}}} \left( \mathcolor{gray}{\bar{r}}, \mathcolor{gray}{t} \right), \bar{J}_{\;\!\textcolor{Maroon}{\text{f}}} \left( \mathcolor{gray}{\bar{r}}, \mathcolor{gray}{t} \right)$\Footnote{对于符号约定,比如下标 $\textcolor{Maroon}{\text{f}}$ 的\textcolor{Maroon}{褐红色}及其含义 `\textcolor{Maroon}{free}',其定义见\bref{hook:Maroon};此外,在 \bref{hook:1bar} 中还约定:总使用 1 条上短横线 $\bar{~}$(而不是粗体)来表示矢量(如 $\bar{J}_{\;\!\textcolor{Maroon}{\text{f}}}$),以区别于 \bref{hook:0bar} 中定义的无上短横线的标量(如 ${\rho}_{\;\!\textcolor{Maroon}{\text{f}}}$);这使得即使是细体,也可以表示矢量和张量,见\bref{hook:thin}。本文不用粗体表示矢量或张量,见\bref{hook:bold}。} 的,相对静止的一般电磁介质内部,4 个空域时变\Footnote{指复矢量场 $\bar{E}^{\;\!\mathcolor{gray}{t}}_{\;\!\mathcolor{gray}{z}}, \bar{H}^{\;\!\mathcolor{gray}{t}}_{\;\!\mathcolor{gray}{z}}, \bar{D}^{\;\!\mathcolor{gray}{t}}_{\;\!\mathcolor{gray}{z}}, \bar{B}^{\;\!\mathcolor{gray}{t}}_{\;\!\mathcolor{gray}{z}}$ 均是四维时空 $\mathcolor{gray}{\bar{r}}, \mathcolor{gray}{t}$ 的函数,且因此一般意义上是\textcolor{gray}{复色} $\left\{ \mathcolor{gray}{\omega} \in \mathcolor{gray}{\mathbb{R}} \right\}$ 的复场 $\in \bar{\mathbb{C}}_{\textcolor{Plum}{3}} ( \mathcolor{gray}{\bar{\mathbb{R}}_{\textcolor{Maroon}{3+1}}} )$(认为这四者必须为实场\cite{boydNonlinearOptics2019}也没关系:它们对$\mathcolor{gray}{t}$的\textcolor{Plum}{傅立叶变换}\bref{eq:FT-tw}所得到的$\pm \omega$\textcolor{gray}{单色}子波是复共轭的,以至于\textcolor{Plum}{正负}\textcolor{gray}{角频率}的对应复子波求和后会消掉虚部,只剩下\textcolor{Plum}{实部}的余弦$\cos$实子波,因此在\textcolor{gray}{正}/\textcolor{gray}{倒空间}中的总/子场均是有\textcolor{NavyBlue}{物理意义}的实场),属于在视觉上占主导的\textcolor{Plum}{因变量},并用黑色(见 \bref{hook:black})的 \textit{斜体 oblique}(见 \bref{hook:oblique})表示;相对地,\textcolor{gray}{自变量}用视觉和含义上均更次要的\textcolor{gray}{灰色}来表示,见 \bref{hook:gray}。}\textcolor{gray}{复色}场 
$\bar{E}^{\;\!\mathcolor{gray}{t}}_{\;\!\mathcolor{gray}{z}}, \bar{H}^{\;\!\mathcolor{gray}{t}}_{\;\!\mathcolor{gray}{z}}, \bar{D}^{\;\!\mathcolor{gray}{t}}_{\;\!\mathcolor{gray}{z}}, \bar{B}^{\;\!\mathcolor{gray}{t}}_{\;\!\mathcolor{gray}{z}}$\Footnote{由于\textcolor{NavyBlue}{傅立叶光学}一般运行在平行平面间,约定下述表示相互等价:$\bar{E} \left( \mathcolor{gray}{\bar{r}}, \mathcolor{gray}{t} \right) = \bar{E}^{\;\!\mathcolor{gray}{t}}_{\;\!\mathcolor{gray}{\bar{r}}} = \bar{E}^{\;\!\mathcolor{gray}{t}}_{\;\!\mathcolor{gray}{z}} \left( \mathcolor{gray}{\bar{\rho}} \right)$,并因此经常省略面内\textcolor{gray}{自变量}:极径列向量 $\mathcolor{gray}{\bar{\rho}} := \left( \mathcolor{gray}{x},~ \mathcolor{gray}{y} \right)^{\mathsf{\textcolor{Plum}{T}}}$,以只写作朝 $\textcolor{Maroon}{+\symup{z}}$ 轴\textcolor{gray}{传播距离} $\mathcolor{gray}{z}$ 的函数 $\bar{E}^{\;\!\mathcolor{gray}{t}}_{\;\!\mathcolor{gray}{z}} := \bar{E}^{\;\!\mathcolor{gray}{t}}_{\;\!\mathcolor{gray}{z}} \left( \mathcolor{gray}{\bar{\rho}} \right)$。 ---  同样的规则也适用于其他场量(如 ${\rho}_{\;\!\textcolor{Maroon}{\text{f}}} \left( \mathcolor{gray}{\bar{r}}, \mathcolor{gray}{t} \right) \to {\rho}^{\;\!\mathcolor{gray}{t}}_{\;\!\textcolor{Maroon}{\text{f}}\mathcolor{gray}{z}}$),且适用于\textcolor{gray}{空间频率}域,见 \bref{hook:Xint}。},满足微分形式\Footnote{尽管是微分形式,仍然处于(相对的)宏观层面:典型的光\textcolor{gray}{波长} $1$um 是原子特征尺寸 $1\text{\r{A}} = 0.1$nm 的 $10^4$ 倍,因此 \bref{eq:Curl-EH,eq:Div-DB,eq:Div-em-f} 涉及的所有\textcolor{NavyBlue}{物理量}均是空间平均后的结果\cite{mackayElectromagneticAnisotropyBianisotropy2019};如果要引入\textcolor{Plum}{(非)线性}极/磁化强度/率随考虑区域尺度的缩放,则需要 \textcolor{Maroon}{Clausius-Mossotti equation} 或 \textcolor{Maroon}{Lorentz-Lorenz law} 的\textcolor{Plum}{局域}场修正\cite{boydNonlinearOptics2019}。}的麦氏方程组的 2 个旋度假设\Footnote{使用\textcolor{Plum}{爱因斯坦求和}约定(2 个相同指标相遇则主体求和,但 2 个 同侧角标不适用\cite{frankelGeometryPhysicsIntroduction2011}),定义了空域 3 维向量微分算子 $\mathcolor{gray}{\bar{\nabla}} := \mathcolor{gray}{\bar{\nabla}_{\bar{r}}} := \hat{\symup{e}}_{\mathcolor{gray}{\symup{\iota}}} \mathcolor{gray}{\nabla^\iota}$,其中 $\symup{\iota} = \symup{x,y,z}$,且 $\hat{\symup{e}}_{\symup{x}},\hat{\symup{e}}_{\symup{y}},\hat{\symup{e}}_{\symup{z}}$ 分别为 $\symup{x,y,z}$ 方向的单位定矢(“定矢$\hat{\symup{e}}$”也“直体”:见 \bref{hook:upright})。对于\textcolor{Plum}{旋度/叉乘/外积/叉积/矢量积}运算,还可写成轴矢量构成的反称二阶张量的矩阵乘积:$\mathcolor{gray}{\bar{\nabla} \times} \bar{v} = \mathcolor{gray}{\bar{\nabla}^\times \cdot} \bar{v}$\cite{ossikovskiConstitutiveRelationsOptically2021},并且矩阵乘积可以省略 `$\cdot$' 而写为 $\mathcolor{gray}{\bar{\nabla}^\times} \bar{v} = \mathcolor{gray}{\bar{\nabla}^\times \cdot} \bar{v}$。此外,定义了 \textcolor{Plum}{\text{Levi-Civita symbol}} $\epsilon^{\hphantom{\symup{\iota}\hat{1}}\hat{2}}_{\symup{\iota} \hat{1}}$、空域偏导算符 $\mathcolor{gray}{\nabla^{\hat{1}}} := \mathcolor{gray}{ \partial \mathcolor{black}{\underline{~~}} \big/ \partial \mathcolor{gray}{\hat{1}} }$、时域偏导算符 $\mathcolor{gray}{\nabla^t} \underline{~~} := \mathcolor{gray}{\partial \mathcolor{black}{\underline{~~}} \big/ \partial t}$。其中,$\symup{\iota},\hat{1},\hat{2}$ 均取遍 与之相同颜色的 直体 $\symup{x},\symup{y},\symup{z}$(如 $\epsilon^{\hphantom{\symup{\iota}\hat{1}}\hat{2}}_{\symup{\iota} \hat{1}} \to \epsilon^{\hphantom{\symup{x}\mathcolor{gray}{\symup{y}}}\symup{z}}_{\symup{x}\symup{y}}, \cdots$、$E^{\;\!\mathcolor{gray}{t}}_{\;\! \hat{2}\mathcolor{gray}{z}} \to E^{\;\!\mathcolor{gray}{t}}_{\;\! \symup{x} \mathcolor{gray}{z}}, \cdots$)或斜体 $x,y,z$(如 $\mathcolor{gray}{\nabla^{\hat{1}}} \to \mathcolor{gray}{\nabla^{x}}, \cdots$)。}:
\begin{subequations} \label{eq:Curl-EH}
	\abovedisplayskip=-8pt
\begin{align}
	\textcolor{Maroon}{\text{Faraday's law of electromagnetic induction}}\text{:}&\hspace{1.2em} \mathcolor{gray}{\bar{\nabla} \times} \bar{E}^{\;\!\mathcolor{gray}{t}}_{\;\!\mathcolor{gray}{z}} + \mathcolor{gray}{\nabla^t} \bar{B}^{\;\!\mathcolor{gray}{t}}_{\;\!\mathcolor{gray}{z}} \hspace{-0.9em} &&= - \bar{K}^{\;\!\mathcolor{gray}{t}}_{\;\!\textcolor{Maroon}{\text{f}}\mathcolor{gray}{z}} \label{eq:Curl-EK} \\ 
	&\epsilon^{\hphantom{\symup{\iota}\hat{1}}\hat{2}}_{\symup{\iota}\mathcolor{gray}{\hat{1}}} \mathcolor{gray}{\nabla^{\hat{1}}} E^{\;\!\mathcolor{gray}{t}}_{\;\! \hat{2}\mathcolor{gray}{z}} + \mathcolor{gray}{\nabla^t} B^{\;\!\mathcolor{gray}{t}}_{\;\! \symup{\iota}\mathcolor{gray}{z}} \hspace{-0.9em} &&= - \Xint{\mathcolor{gray}{-}}{10}{K}^{\;\!\mathcolor{gray}{t}}_{\;\! \textcolor{Maroon}{\text{f}} \symup{\iota}\mathcolor{gray}{z}}~, \label{eq:curl-EK} \\ 
	\textcolor{Maroon}{\text{Jefimenko's}} \to \textcolor{Maroon}{\text{Amp\`{e}re-Maxwell circuital law}}\text{:}&\hspace{1.2em} \mathcolor{gray}{\bar{\nabla} \times} \bar{H}^{\;\!\mathcolor{gray}{t}}_{\;\!\mathcolor{gray}{z}} - \mathcolor{gray}{\nabla^t} \bar{D}^{\;\!\mathcolor{gray}{t}}_{\;\!\mathcolor{gray}{z}} \hspace{-0.9em} &&= \bar{J}^{\;\!\mathcolor{gray}{t}}_{\;\!\textcolor{Maroon}{\text{f}}\mathcolor{gray}{z}} \label{eq:Curl-H} \\ 
	&\epsilon^{\hphantom{\symup{\iota}\hat{1}}\hat{2}}_{\symup{\iota}\mathcolor{gray}{\hat{1}}} \mathcolor{gray}{\nabla^{\hat{1}}} H^{\;\!\mathcolor{gray}{t}}_{\;\! \hat{2}\mathcolor{gray}{z}} - \mathcolor{gray}{\nabla^t} D^{\;\!\mathcolor{gray}{t}}_{\;\! \symup{\iota}\mathcolor{gray}{z}} \hspace{-0.9em} &&= J^{\;\!\mathcolor{gray}{t}}_{\;\! \textcolor{Maroon}{\text{f}} \symup{\iota}\mathcolor{gray}{z}}~. \label{eq:curl-H}
\end{align}
\end{subequations}
以及 2 个散度假设\Footnote{对于散度\&点积,也可以使用“行向量·列向量”的矩阵乘法,如 $\mathcolor{gray}{\bar{\nabla}^\intercal} \bar{v} = \mathcolor{gray}{\bar{\nabla} \cdot} \bar{v}$。此外,高阶散度/点积(如 \bref{eq:P-b} 中的 $\mathcolor{gray}{\bar{\nabla}} \mathcolor{gray}{\bar{\nabla} \colon}\! \bar{\bar{Q}}^{\;\!\mathcolor{gray}{t}}_{\;\!\mathcolor{gray}{z}}$)原则上也可用 $\mathcolor{gray}{\bar{\nabla} \cdot} ( \mathcolor{gray}{\bar{\nabla} \cdot} \bar{\bar{Q}}^{\;\!\mathcolor{gray}{t}}_{\;\!\mathcolor{gray}{z}} ) \neq \mathcolor{gray}{( \bar{\nabla} \cdot \bar{\nabla} )} \bar{\bar{Q}}^{\;\!\mathcolor{gray}{t}}_{\;\!\mathcolor{gray}{z}} = \mathcolor{gray}{\bar{\nabla}^2} \bar{\bar{Q}}^{\;\!\mathcolor{gray}{t}}_{\;\!\mathcolor{gray}{z}} = \mathcolor{gray}{\nabla^2} \bar{\bar{Q}}^{\;\!\mathcolor{gray}{t}}_{\;\!\mathcolor{gray}{z}}$ 或 $\mathcolor{gray}{\left( \bar{\nabla} \bar{\nabla} \right)^\intercal} \! \bar{\bar{Q}}^{\;\!\mathcolor{gray}{t}}_{\;\!\mathcolor{gray}{z}} = \mathcolor{gray}{\bar{\nabla}} ( \mathcolor{gray}{\bar{\nabla}^\intercal} \bar{\bar{Q}}^{\;\!\mathcolor{gray}{t}}_{\;\!\mathcolor{gray}{z}} )^\intercal$ 表示,但对于后者中高阶张量$\mathcolor{gray}{\bar{\nabla}} \mathcolor{gray}{\bar{\nabla}}$的转置需要指定(哪)两个\textcolor{Plum}{维度}。此外,\textcolor{Plum}{点积} $\cdot$ 也可视为对应元素积=哈达玛积 $\odot$ 后,再对所有元素求和。}:
\begin{subequations} \label{eq:Div-DB}
\begin{align}
	\textcolor{Maroon}{\text{Coulomb's}} \to \textcolor{Maroon}{\text{Gauss's law for electricity}}\text{:}&\hspace{0.5em} \mathcolor{gray}{\bar{\nabla} \cdot} \bar{D}^{\;\!\mathcolor{gray}{t}}_{\;\!\mathcolor{gray}{z}} \hspace{-3.8em} &&= {\rho}^{\;\!\mathcolor{gray}{t}}_{\;\!\textcolor{Maroon}{\text{f}}\mathcolor{gray}{z}} \label{eq:Div-D} \\ 
	&\hspace{0.5em} \mathcolor{gray}{\nabla^\iota} D^{\;\!\mathcolor{gray}{t}}_{\;\! \mathcolor{gray}{\symup{\iota}}\mathcolor{gray}{z}} \hspace{-3.8em} &&= {\rho}^{\;\!\mathcolor{gray}{t}}_{\;\!\textcolor{Maroon}{\text{f}}\mathcolor{gray}{z}}~, \label{eq:div-D} \\
	\textcolor{Maroon}{\text{Biot-Savart}} \to \textcolor{Maroon}{\text{Gauss's law for magnetism}}\text{:}&\hspace{0.5em} \mathcolor{gray}{\bar{\nabla} \cdot} \bar{B}^{\;\!\mathcolor{gray}{t}}_{\;\!\mathcolor{gray}{z}} \hspace{-3.8em} &&= {\kappa}^{\;\!\mathcolor{gray}{t}}_{\;\!\textcolor{Maroon}{\text{f}}\mathcolor{gray}{z}} \label{eq:Div-Bk} \\
	&\hspace{0.5em} \mathcolor{gray}{\nabla^\iota} B^{\;\!\mathcolor{gray}{t}}_{\;\! \mathcolor{gray}{\symup{\iota}}\mathcolor{gray}{z}} \hspace{-3.8em} &&= {\kappa}^{\;\!\mathcolor{gray}{t}}_{\;\!\textcolor{Maroon}{\text{f}}\mathcolor{gray}{z}}~. \label{eq:div-Bk}
\end{align}
\end{subequations}
其中,为\textcolor{Plum}{数学}形式对称(以方便引入相对论效应和检验其协变性\cite{lakhtakiaCovariancesInvariancesMaxwell1995,chen-zhuChenZhuxieUndergraduate_courses2024}),和\textcolor{NavyBlue}{物理}上不排除可能存在的磁单极子,除自由电(荷/流)源${\rho}^{\;\!\mathcolor{gray}{t}}_{\;\!\textcolor{Maroon}{\text{f}}\mathcolor{gray}{z}}, \bar{J}^{\;\!\mathcolor{gray}{t}}_{\;\!\textcolor{Maroon}{\text{f}}\mathcolor{gray}{z}}$外,还添加了自由磁(荷/流)源${\kappa}^{\;\!\mathcolor{gray}{t}}_{\;\!\textcolor{Maroon}{\text{f}}\mathcolor{gray}{z}}, \bar{K}^{\;\!\mathcolor{gray}{t}}_{\;\!\textcolor{Maroon}{\text{f}}\mathcolor{gray}{z}}$\cite{lakhtakiaCovariancesInvariancesMaxwell1995}。这 4 个自由源(体密度)项,满足 2 个\textcolor{Plum}{连续}性假设\cite{mackayElectromagneticAnisotropyBianisotropy2019,lakhtakiaCovariancesInvariancesMaxwell1995,chen-zhuChenZhuxieUndergraduate_courses2024}:
\begin{subequations} \label{eq:Div-em-f}
	\abovedisplayskip=-4pt
	\belowdisplayskip=10pt
\begin{align}
	\textcolor{Maroon}{\text{Continuity for free electric charge}}\text{:}&\hspace{0.5em} \mathcolor{gray}{\bar{\nabla} \cdot} \bar{J}^{\;\!\mathcolor{gray}{t}}_{\;\!\textcolor{Maroon}{\text{f}}\mathcolor{gray}{z}} + \mathcolor{gray}{\nabla^t} {\rho}^{\;\!\mathcolor{gray}{t}}_{\;\!\textcolor{Maroon}{\text{f}}\mathcolor{gray}{z}} \hspace{-5.3em} &&= 0 \label{eq:Div-e-f} \\ 
	&\hspace{0.5em} \mathcolor{gray}{\nabla^\iota} J^{\;\!\mathcolor{gray}{t}}_{\;\!\textcolor{Maroon}{\text{f}} \mathcolor{gray}{\symup{\iota}}\mathcolor{gray}{z}} + \mathcolor{gray}{\nabla^t} {\rho}^{\;\!\mathcolor{gray}{t}}_{\;\!\textcolor{Maroon}{\text{f}}\mathcolor{gray}{z}} \hspace{-5.3em} &&= 0~, \label{eq:div-e-f} \\ 
	\textcolor{Maroon}{\text{Continuity for free magnetic charge}}\text{:}&\hspace{0.5em} \mathcolor{gray}{\bar{\nabla} \cdot} \bar{K}^{\;\!\mathcolor{gray}{t}}_{\;\!\textcolor{Maroon}{\text{f}}\mathcolor{gray}{z}} + \mathcolor{gray}{\nabla^t} {\kappa}^{\;\!\mathcolor{gray}{t}}_{\;\!\textcolor{Maroon}{\text{f}}\mathcolor{gray}{z}} \hspace{-5.3em} &&= 0 \label{eq:Div-m-f} \\
	&\hspace{0.5em} \mathcolor{gray}{\nabla^\iota} \Xint{\mathcolor{gray}{-}}{10}{K}^{\;\!\mathcolor{gray}{t}}_{\;\!\textcolor{Maroon}{\text{f}} \mathcolor{gray}{\symup{\iota}}\mathcolor{gray}{z}} + \mathcolor{gray}{\nabla^t} {\kappa}^{\;\!\mathcolor{gray}{t}}_{\;\!\textcolor{Maroon}{\text{f}}\mathcolor{gray}{z}} \hspace{-5.3em} &&= 0~. \label{eq:div-m-f}
\end{align}
\end{subequations}
注意,对旋度 \bref{eq:Curl-EH} 两边取散度($\mathcolor{gray}{\bar{\nabla} \cdot}$),\textcolor{Plum}{连续}性 \bref{eq:Div-em-f} 可导出散度 \bref{eq:Div-DB},反之亦然\cite{lakhtakiaGenesisPostConstraint2004}。因此,2 条散度方程均不是必需的,可将其视为冗余\Footnote{并且不应简单地仅根据$\bar{D}^{\;\!\mathcolor{gray}{t}}_{\;\!\mathcolor{gray}{z}}, \bar{B}^{\;\!\mathcolor{gray}{t}}_{\;\!\mathcolor{gray}{z}}$的\textcolor{Plum}{横向}性,而将二者视为\textcolor{NavyBlue}{基本场}\cite{quesadaPhotonPairsNonlinear2022,berryOpticalSingularitiesBianisotropic2005}。但从场能量体密度变化率$\bar{E}^{\;\!\mathcolor{gray}{t}}_{\;\!\mathcolor{gray}{z}} \mathbb{d}\bar{D}^{\;\!\mathcolor{gray}{t}}_{\;\!\mathcolor{gray}{z}} +  \bar{H}^{\;\!\mathcolor{gray}{t}}_{\;\!\mathcolor{gray}{z}} \mathbb{d}\bar{B}^{\;\!\mathcolor{gray}{t}}_{\;\!\mathcolor{gray}{z}}$中含有 2 个旋度\bref{eq:Curl-EH}中对$\bar{D}^{\;\!\mathcolor{gray}{t}}_{\;\!\mathcolor{gray}{z}}, \bar{B}^{\;\!\mathcolor{gray}{t}}_{\;\!\mathcolor{gray}{z}}$的微分\cite{chen-zhuChenZhuxieUndergraduate_courses2024}、方便引入适用于\textcolor{Plum}{非线性}量子光学的正确的\textcolor{NavyBlue}{哈密顿量}\cite{quesadaPhotonPairsNonlinear2022},将$\bar{D}^{\;\!\mathcolor{gray}{t}}_{\;\!\mathcolor{gray}{z}}, \bar{B}^{\;\!\mathcolor{gray}{t}}_{\;\!\mathcolor{gray}{z}}$视为\textcolor{NavyBlue}{基本场}也有一定道理?但如此一来,电\textcolor{Plum}{非线性} $\bar{E}^{\;\!\mathcolor{gray}{t}}_{\;\!\mathcolor{gray}{z}} \left( \bar{D}^{\;\!\mathcolor{gray}{t}}_{\;\!\mathcolor{gray}{z}} \right) \slashed{\propto} \bar{D}^{\;\!\mathcolor{gray}{t}}_{\;\!\mathcolor{gray}{z}}$ 的显式表达式有悖常理。}。

\clearpage
%\XGap{-10em}
\vspace*{-8.5em}

\marginLeft[-2.4em]{ssec:EBpJ}\subsection{$\bar{E},\bar{B}$ 基本场、${\rho}, \bar{J}$ 总源连续}\label{ssec:EBpJ}

现代\textcolor{NavyBlue}{电磁学}/\textcolor{NavyBlue}{电动力学}追根溯源地\Footnote{从 $\bar{E}^{\;\!\mathcolor{gray}{t}}_{\;\!\mathcolor{gray}{z}}, \bar{B}^{\;\!\mathcolor{gray}{t}}_{\;\!\mathcolor{gray}{z}}$ 出发又回到 $\bar{E}^{\;\!\mathcolor{gray}{t}}_{\;\!\mathcolor{gray}{z}}, \bar{B}^{\;\!\mathcolor{gray}{t}}_{\;\!\mathcolor{gray}{z}}$,先升格又降格 $\bar{E}^{\;\!\mathcolor{gray}{t}}_{\;\!\mathcolor{gray}{z}}, \bar{B}^{\;\!\mathcolor{gray}{t}}_{\;\!\mathcolor{gray}{z}} \to \bar{D}^{\;\!\mathcolor{gray}{t}}_{\;\!\mathcolor{gray}{z}}, \bar{H}^{\;\!\mathcolor{gray}{t}}_{\;\!\mathcolor{gray}{z}} \to \bar{E}^{\;\!\mathcolor{gray}{t}}_{\;\!\mathcolor{gray}{z}}, \bar{B}^{\;\!\mathcolor{gray}{t}}_{\;\!\mathcolor{gray}{z}}$,绕了一个大弯。},将 $\bar{E}^{\;\!\mathcolor{gray}{t}}_{\;\!\mathcolor{gray}{z}}, \bar{B}^{\;\!\mathcolor{gray}{t}}_{\;\!\mathcolor{gray}{z}}$\Footnote{$\bar{D}^{\;\!\mathcolor{gray}{t}}_{\;\!\mathcolor{gray}{z}},\bar{H}^{\;\!\mathcolor{gray}{t}}_{\;\!\mathcolor{gray}{z}}$作为总/裸场$\bar{E}^{\;\!\mathcolor{gray}{t}}_{\;\!\mathcolor{gray}{z}},\bar{B}^{\;\!\mathcolor{gray}{t}}_{\;\!\mathcolor{gray}{z}}$分别加上(通过\bref{eq:Div-D})或减去(通过\bref{eq:Curl-H})束缚源所响应产生的(极化或磁化)场后的\textcolor{NavyBlue}{辅助场},分别代表自由/裸源${\rho}^{\;\!\mathcolor{gray}{t}}_{\;\!\textcolor{Maroon}{\text{f}}\mathcolor{gray}{z}}, \bar{J}^{\;\!\mathcolor{gray}{t}}_{\;\!\textcolor{Maroon}{\text{f}}\mathcolor{gray}{z}}$所对应的“净留/残余场”。$\bar{E}^{\;\!\mathcolor{gray}{t}}_{\;\!\mathcolor{gray}{z}},\bar{B}^{\;\!\mathcolor{gray}{t}}_{\;\!\mathcolor{gray}{z}}$是\textcolor{NavyBlue}{基本场},出于下述原因:其\textcolor{NavyBlue}{物理起源}是微观且明确的、可直接\textcolor{NavyBlue}{测量}、包含了所有的束缚和自由源产生的场、\textcolor{Maroon}{洛伦兹力公式}(普适至相对论情形)、\bref{eq:Curl-EK,eq:Div-Bk}的\textcolor{NavyBlue}{无源}特性及其导出的标矢势和四维势矢量、四维二阶电磁场张量\cite{chen-zhuChenZhuxieUndergraduate_courses2024}、无矛盾地推导和适用 \textcolor{Maroon}{Post} 约束\cite{lakhtakiaGenesisPostConstraint2004};同时也方便\textcolor{NavyBlue}{原子物理}中对 \textcolor{Maroon}{Larmor} 进动、\textcolor{Maroon}{Stern-Gerlach} 实验、\textcolor{Maroon}{Zeeman} 效应的表述\cite{chen-zhuChenZhuxieUndergraduate_courses2024},以及\textcolor{NavyBlue}{量子电动力学}中对磁光材料的拉氏量的处理\cite{nelsonLagrangianTreatmentMagnetic1994}。 --- 但是,选择$\bar{B}^{\;\!\mathcolor{gray}{t}}_{\;\!\mathcolor{gray}{z}}$而不是$\bar{H}^{\;\!\mathcolor{gray}{t}}_{\;\!\mathcolor{gray}{z}}$将不方便(准静)磁学,如铁磁性物质的磁滞回线的表述\cite{hillionBasicFieldElectromagnetism1996}。但此时将外场写作$\bar{B}_0$而不是$\bar{H}$似乎即可解决:正如电光效应的外加准静电场$\bar{E}_0$一样。}而不是 $\bar{E}^{\;\!\mathcolor{gray}{t}}_{\;\!\mathcolor{gray}{z}}, \bar{H}^{\;\!\mathcolor{gray}{t}}_{\;\!\mathcolor{gray}{z}}$\Footnote{相对论或手性的情形下,将$\bar{E}^{\;\!\mathcolor{gray}{t}}_{\;\!\mathcolor{gray}{z}}, \bar{H}^{\;\!\mathcolor{gray}{t}}_{\;\!\mathcolor{gray}{z}}$而不是$\bar{E}^{\;\!\mathcolor{gray}{t}}_{\;\!\mathcolor{gray}{z}}, \bar{B}^{\;\!\mathcolor{gray}{t}}_{\;\!\mathcolor{gray}{z}}$作为\textcolor{Maroon}{本构关系}的\textcolor{NavyBlue}{基本场},可能更有优势\cite{hillionBasicFieldElectromagnetism1996,lakhtakiaGenesisPostConstraint2004};此外,对于\textcolor{Maroon}{边界条件},进可采用$\bar{E}^{\;\!\mathcolor{gray}{t}}_{\;\!\mathcolor{gray}{z}},\bar{H}^{\;\!\mathcolor{gray}{t}}_{\;\!\mathcolor{gray}{z}}$切向\textcolor{Plum}{连续}\textcolor{Maroon}{边界条件}\cite{mcleodVectorFourierOptics2014},退可四维时空\textcolor{Plum}{傅立叶变换}\cite{chenWavevectorspaceMethodWave1993,chenWavePropagationExciton1993,nelsonDerivingTransmissionReflection1995}。还允许不关注\textcolor{NavyBlue}{微观物理起源}\cite{eimerlQuantumElectrodynamicsOptical1988,nelsonMechanismsDispersionCrystalline1989,boydNonlinearOptics2019,loudonPropagationElectromagneticEnergy1997,laxLinearNonlinearElectrodynamics1971},电场的\textcolor{Plum}{非局域}一阶\textcolor{PineGreen}{波矢}\textcolor{NavyBlue}{色散}也可直接放进\textcolor{Maroon}{本构关系}而无需额外处理\cite{berryOpticalSingularitiesBianisotropic2005}。但可能没法处理材料表面积累电荷(如铁电体的$\textcolor{Maroon}{+\symup{c}}$ 面)、表面电流\cite{chen-zhuChenZhuxieUndergraduate_courses2024}、表面光学活性\cite{nelsonMechanismsDispersionCrystalline1989}、许多磁致光学现象\cite{raabMultipoleTheoryElectromagnetism2004},尽管没有使用到任何散度方程/\textcolor{Plum}{横向}约束,已经很有吸引力了\cite{eimerlQuantumElectrodynamicsOptical1988,berryOpticalSingularitiesBirefringent2003,berryOpticalSingularitiesBianisotropic2005}。}视为\textcolor{NavyBlue}{基本场}\cite{hillionBasicFieldElectromagnetism1996,lakhtakiaGenesisPostConstraint2004,nelsonDerivingTransmissionReflection1995,hehlGentleIntroductionFoundations2000}。因此,2 个旋度方程 \bref{eq:Curl-EK}、\bref{eq:Curl-H} 应等效地写做:
\begin{subequations} \label{eq:Curl-EB}
\begin{align}
	\textcolor{Maroon}{\text{法拉第电磁感应定律}}\text{:}~~~~~~~ \mathcolor{gray}{\bar{\nabla} \times} \bar{E}^{\;\!\mathcolor{gray}{t}}_{\;\!\mathcolor{gray}{z}} + \mathcolor{gray}{\nabla^t} \bar{B}^{\;\!\mathcolor{gray}{t}}_{\;\!\mathcolor{gray}{z}} &= \bar{0} \label{eq:Curl-E} \\ 
	\epsilon^{\hphantom{\symup{\iota}\hat{1}}\hat{2}}_{\symup{\iota}\mathcolor{gray}{\hat{1}}} \mathcolor{gray}{\nabla^{\hat{1}}} E^{\;\!\mathcolor{gray}{t}}_{\;\! \hat{2}\mathcolor{gray}{z}} + \mathcolor{gray}{\nabla^t} B^{\;\!\mathcolor{gray}{t}}_{\;\! \symup{\iota}\mathcolor{gray}{z}} &= 0~, \label{eq:curl-E} \\
	\textcolor{Maroon}{\text{安倍环路定律}}\text{:}~~~~~~~~~ {\symup{\mu}}_0^{-1} \mathcolor{gray}{\bar{\nabla} \times} \bar{B}^{\;\!\mathcolor{gray}{t}}_{\;\!\mathcolor{gray}{z}} - {\symup{\varepsilon}}_0 \mathcolor{gray}{\nabla^t} \bar{E}^{\;\!\mathcolor{gray}{t}}_{\;\!\mathcolor{gray}{z}} &= \bar{J}^{\;\!\mathcolor{gray}{t}}_{\;\!\mathcolor{gray}{z}} \label{eq:Curl-B} \\ 
	{\symup{\mu}}_0^{-1} \epsilon^{\hphantom{\symup{\iota}\hat{1}}\hat{2}}_{\symup{\iota}\mathcolor{gray}{\hat{1}}} \mathcolor{gray}{\nabla^{\hat{1}}} B^{\;\!\mathcolor{gray}{t}}_{\;\! \hat{2}\mathcolor{gray}{z}} - {\symup{\varepsilon}}_0 \mathcolor{gray}{\nabla^t} E^{\;\!\mathcolor{gray}{t}}_{\;\! \symup{\iota}\mathcolor{gray}{z}} &= J^{\;\!\mathcolor{gray}{t}}_{\;\! \symup{\iota}\mathcolor{gray}{z}}~. \label{eq:curl-B}
\end{align}
\end{subequations}
相应地,2 个散度方程 \bref{eq:Div-D}、\bref{eq:Curl-B} 返璞归真为:
\begin{subequations} \label{eq:Div-EB}
\begin{align}
	\textcolor{Maroon}{\text{高斯定律(电)}}\text{:}~~~~~~~ {\symup{\varepsilon}}_0 \mathcolor{gray}{\bar{\nabla} \cdot} \bar{E}^{\;\!\mathcolor{gray}{t}}_{\;\!\mathcolor{gray}{z}} &= {\rho}^{\;\!\mathcolor{gray}{t}}_{\;\!\mathcolor{gray}{z}} \label{eq:Div-E} \\ 
	{\symup{\varepsilon}}_0 \mathcolor{gray}{\nabla^\iota} E^{\;\!\mathcolor{gray}{t}}_{\;\! \mathcolor{gray}{\symup{\iota}} \mathcolor{gray}{z}} &= {\rho}^{\;\!\mathcolor{gray}{t}}_{\;\!\mathcolor{gray}{z}}~, \label{eq:div-E} \\ 
	\textcolor{Maroon}{\text{高斯定律(磁)}}\text{:}~~~~~~~~~~ \mathcolor{gray}{\bar{\nabla} \cdot} \bar{B}^{\;\!\mathcolor{gray}{t}}_{\;\!\mathcolor{gray}{z}} &= 0 \label{eq:Div-B} \\ 
	\mathcolor{gray}{\nabla^\iota} B^{\;\!\mathcolor{gray}{t}}_{\;\! \mathcolor{gray}{\symup{\iota}} \mathcolor{gray}{z}} &= 0~. \label{eq:div-B} 
\end{align}
\end{subequations}
同理,总\textcolor{Maroon}{电荷} ${\rho}^{\;\!\mathcolor{gray}{t}}_{\;\!\mathcolor{gray}{z}}$ \textcolor{Maroon}{守恒}方程从自由电荷 ${\rho}^{\;\!\mathcolor{gray}{t}}_{\;\!\textcolor{Maroon}{\text{f}}\mathcolor{gray}{z}}$ 所满足的\textcolor{Plum}{连续}性 \bref{eq:Div-e-f} 升级为:
\begin{align}
	\textcolor{Maroon}{\text{电荷守恒定律}}\text{:}~~~~~~~ \mathcolor{gray}{\bar{\nabla} \cdot} \bar{J}^{\;\!\mathcolor{gray}{t}}_{\;\!\mathcolor{gray}{z}} + \mathcolor{gray}{\nabla^t} {\rho}^{\;\!\mathcolor{gray}{t}}_{\;\!\mathcolor{gray}{z}} &= 0 \label{eq:Div-e} \\ 
	\mathcolor{gray}{\nabla^\iota} J^{\;\!\mathcolor{gray}{t}}_{\;\! \mathcolor{gray}{\symup{\iota}} \mathcolor{gray}{z}} + \mathcolor{gray}{\nabla^t} {\rho}^{\;\!\mathcolor{gray}{t}}_{\;\!\mathcolor{gray}{z}} &= 0~. \label{eq:div-e} 
\end{align}
该 \bref{eq:Div-e} 即\textcolor{Maroon}{电荷} $Q$ \textcolor{Maroon}{守恒},与\textcolor{Maroon}{磁流} $\varPhi$ \textcolor{Maroon}{守恒}(即\textcolor{Maroon}{法拉第电磁感应定律} \bref{eq:Curl-E}、\textcolor{Maroon}{安倍-麦克斯韦环路定律} \bref{eq:Curl-B})一起\cite{hehlSpacetimeMetricLocal2006},经受住了大量的实验考验\cite{hehlGentleIntroductionFoundations2000},因此二者都是基本的\textcolor{Maroon}{物理定律}\cite{hehlRecentDevelopmentsPremetric2006,hehlFOUNDATIONSCLASSICALELECTRODYNAMICS}。

下 \bref{fig:EHDB} 中,2 条红\textcolor{Plum}{边}(连接了\textcolor{Plum}{顶点} $D,H$、$E,B$)、2 条蓝\textcolor{Plum}{边}(连接了\textcolor{Plum}{顶点} $E,H$、$D,B$)上,对应颜色(红/橘色、蓝色)的文字,分别给出了选择 4 种电磁场组合对 $\{ E,H$、$E,B$、$D,B$、$D,H \}$ 中的每一对作为\textcolor{NavyBlue}{基本场}的理由。

\begin{figure}[htbp!]
	\centering
	\includegraphics[width=0.8\textwidth]{D:/C2D/Desktop/article_fig/phd_thesis_fig/chapter-02/本构关系与边界条件-single-page.pdf}
	\biackcaption[\textbf{The rationale for selecting $E,B$ as the fundamental fields} --- evidenced by the extensive red and orange annotations along their corresponding edges --- \textbf{is far more substantial than that for the other three pairings}, i.e. $E,H$ and $D,B$ with blue edges \& texts, and $D,H$.]{-0.7em}{\textbf{$E,B$ 作为\textcolor{NavyBlue}{基本场}的理由}(大量的红/橘色文字理由\textcolor{Plum}{所在的边})\textbf{,远比其他 3 种组合}(2 条蓝色文字\textcolor{Plum}{边}的\textcolor{Plum}{端点} $E,H$ 和 $D,B$,以及 1 条少量红/橘色文字\textcolor{Plum}{对边} $D,H$)\textbf{更丰富}\\}{fig:EHDB}
\end{figure}

\vspace*{-5.5em}

\marginLeft[-1.5em]{ssec:PMQN}\subsection{$\bar{P},\bar{M}$ 极磁化、$\bar{\bar{Q}},\bar{\bar{N}}$ 多极理论}\label{ssec:PMQN}

在\textcolor{Maroon}{高斯定律(电)} \bref{eq:Div-E} 和\textcolor{Maroon}{电荷守恒}定律 \bref{eq:Div-e} 中,总电荷源 ${\rho}^{\;\!\mathcolor{gray}{t}}_{\;\!\mathcolor{gray}{z}}$ 分为自由电荷源 ${\rho}^{\;\!\mathcolor{gray}{t}}_{\;\!\textcolor{Maroon}{\text{f}}\mathcolor{gray}{z}}$ 和束缚电荷源 ${\rho}^{\;\!\mathcolor{gray}{t}}_{\;\!\textcolor{Maroon}{\text{b}}\mathcolor{gray}{z}}$\cite{langeMultipoleTheoryHehl2015,raabMultipoleTheoryElectromagnetism2004},即:
\abovedisplayskip=5pt
\belowdisplayskip=5pt
\begin{align} \label{eq:p=f+b}
	{\rho}^{\;\!\mathcolor{gray}{t}}_{\;\!\mathcolor{gray}{z}} = {\rho}^{\;\!\mathcolor{gray}{t}}_{\;\!\textcolor{Maroon}{\text{f}}\mathcolor{gray}{z}} + {\rho}^{\;\!\mathcolor{gray}{t}}_{\;\!\textcolor{Maroon}{\text{b}}\mathcolor{gray}{z}}~,
\end{align}
其中,对动态电荷源分布所产生的标势场 $\phi^{\;\!\mathcolor{gray}{t}}_{\;\!\mathcolor{gray}{z}}$,在场点 $\mathcolor{gray}{\bar{r}}$ 邻域(无须在源分布体系的平衡态附近),进行三元泰勒展开\Footnote{本文将\textcolor{Plum}{多极}、\textcolor{Plum}{非局域}、\textcolor{Plum}{非线性}展开系数,全纳入\textcolor{NavyBlue}{物理量}内,以降低心智负担。},得束缚电荷体密度 ${\rho}^{\;\!\mathcolor{gray}{t}}_{\;\!\textcolor{Maroon}{\text{b}}\mathcolor{gray}{z}}$ 的表达式\cite{raabMultipoleTheoryElectromagnetism2004,delangeTranslationalInvariancePost2012,chen-zhuChenZhuxieUndergraduate_courses2024}\Footnote{定义了二、三阶张量 $\bar{\bar{Q}},\bar{\bar{\bar{O}}}$,见 \bref{hook:2bar,hook:3bar}。}:
\begin{subequations}
	\abovedisplayskip=-10pt
\begin{align}
	{\rho}^{\;\!\mathcolor{gray}{t}}_{\;\!\textcolor{Maroon}{\text{b}}\mathcolor{gray}{z}} &= -\hspace{0.2em} \mathcolor{gray}{\bar{\nabla} \cdot} \left( \bar{P}^{\;\!\mathcolor{gray}{t}}_{\;\!\mathcolor{gray}{z}} - \mathcolor{gray}{\bar{\nabla} \cdot} \bar{\bar{Q}}^{\;\!\mathcolor{gray}{t}}_{\;\!\mathcolor{gray}{z}} + \mathcolor{gray}{\bar{\nabla}} \mathcolor{gray}{\bar{\nabla} \colon}\! \bar{\bar{\bar{O}}}^{\;\!\mathcolor{gray}{t}}_{\;\!\mathcolor{gray}{z}} - \cdots \right) \label{eq:P-b} \\
	&= -\hspace{0.2em} \mathcolor{gray}{\nabla^\iota} \left( P^{\;\!\mathcolor{gray}{t}}_{\;\! \mathcolor{gray}{\symup{\iota}} \mathcolor{gray}{z}} - \mathcolor{gray}{\nabla^{\hat{1}}} Q^{\;\!\mathcolor{gray}{t}}_{\;\! \mathcolor{gray}{\symup{\iota} \hat{1}} \mathcolor{gray}{z}} + \mathcolor{gray}{\nabla^{\hat{1}}} \mathcolor{gray}{\nabla^{\hat{2}}} O^{\;\!\mathcolor{gray}{t}}_{\;\! \mathcolor{gray}{\symup{\iota} \hat{1} \hat{2}} \mathcolor{gray}{z}} - \cdots \right)~. \label{eq:p-b}
\end{align}
\end{subequations}

同样,在安倍环路定律 \bref{eq:Curl-B} 和 \textcolor{Maroon}{电荷守恒}定律 \bref{eq:Div-e} 中,总电流源 $\bar{J}^{\;\!\mathcolor{gray}{t}}_{\;\!\mathcolor{gray}{z}}$ 内包含了自由项 $\bar{J}^{\;\!\mathcolor{gray}{t}}_{\;\!\textcolor{Maroon}{\text{f}}\mathcolor{gray}{z}}$ 和束缚项 $\bar{J}^{\;\!\mathcolor{gray}{t}}_{\;\!\textcolor{Maroon}{\text{b}}\mathcolor{gray}{z}}$\cite{langeMultipoleTheoryHehl2015,raabMultipoleTheoryElectromagnetism2004},而束缚源 $\bar{J}^{\;\!\mathcolor{gray}{t}}_{\;\!\textcolor{Maroon}{\text{b}}\mathcolor{gray}{z}}$ 又由电源 $\bar{J}^{\;\!\mathcolor{gray}{t}}_{\;\!\textcolor{Maroon}{\text{e}}\mathcolor{gray}{z}}$ 和磁源 $\bar{J}^{\;\!\mathcolor{gray}{t}}_{\;\!\textcolor{Maroon}{\text{m}}\mathcolor{gray}{z}}$ 构成,即:
\begin{subequations} \label{eq:j-fbem}
	\abovedisplayskip=2pt
	\belowdisplayskip=10pt
\begin{align}
	\bar{J}^{\;\!\mathcolor{gray}{t}}_{\;\!\mathcolor{gray}{z}} &= \bar{J}^{\;\!\mathcolor{gray}{t}}_{\;\!\textcolor{Maroon}{\text{f}}\mathcolor{gray}{z}} + \bar{J}^{\;\!\mathcolor{gray}{t}}_{\;\!\textcolor{Maroon}{\text{b}}\mathcolor{gray}{z}}~, \label{eq:j=f+b} \\ \bar{J}^{\;\!\mathcolor{gray}{t}}_{\;\!\textcolor{Maroon}{\text{b}}\mathcolor{gray}{z}} &= \bar{J}^{\;\!\mathcolor{gray}{t}}_{\;\!\textcolor{Maroon}{\text{e}}\mathcolor{gray}{z}} + \bar{J}^{\;\!\mathcolor{gray}{t}}_{\;\!\textcolor{Maroon}{\text{m}}\mathcolor{gray}{z}}~. \label{eq:b=e+m}
\end{align}
\end{subequations}
其中,考虑有限区域内\textcolor{Plum}{连续}分布的时变电流源所产生的矢势场 $\bar{A}^{\;\!\mathcolor{gray}{t}}_{\;\!\mathcolor{gray}{z}}$,等价于源全集中于其荷心时,在原点处的永久/自发\textcolor{Plum}{多极}矩和(受到外场如 $\bar{E}^{\;\!\mathcolor{gray}{t}}_{\;\!\mathcolor{gray}{z}}$ 和/或 $\bar{B}^{\;\!\mathcolor{gray}{t}}_{\;\!\mathcolor{gray}{z}}$、应力场 $\bar{T}^{\;\!\mathcolor{gray}{t}}_{\;\!\mathcolor{gray}{z}}$ 等后反作用出的)感应\textcolor{Plum}{多极}矩(的各项\textcolor{Plum}{多极}展开之和)之和,朝心外某一场点激发的矢势场\cite{raabMultipoleTheoryElectromagnetism2004,delangeTranslationalInvariancePost2012,chen-zhuChenZhuxieUndergraduate_courses2024},可分别得到束缚电源 $\bar{J}^{\;\!\mathcolor{gray}{t}}_{\;\!\textcolor{Maroon}{\text{e}}\mathcolor{gray}{z}}$ 和磁源 $\bar{J}^{\;\!\mathcolor{gray}{t}}_{\;\!\textcolor{Maroon}{\text{m}}\mathcolor{gray}{z}}$ 体密度:
\begin{subequations} \label{eq:J-em}
	\abovedisplayskip=8pt
	\belowdisplayskip=10pt
\begin{align}
	\bar{J}^{\;\!\mathcolor{gray}{t}}_{\;\!\textcolor{Maroon}{\text{e}}\mathcolor{gray}{z}} &= \mathcolor{gray}{\nabla^t} \left( \bar{P}^{\;\!\mathcolor{gray}{t}}_{\;\!\mathcolor{gray}{z}} - \mathcolor{gray}{\bar{\nabla} \cdot} \bar{\bar{Q}}^{\;\!\mathcolor{gray}{t}}_{\;\!\mathcolor{gray}{z}} + \mathcolor{gray}{\bar{\nabla}} \mathcolor{gray}{\bar{\nabla} \colon}\! \bar{\bar{\bar{O}}}^{\;\!\mathcolor{gray}{t}}_{\;\!\mathcolor{gray}{z}} - \cdots \right) \label{eq:J-e} \\ 
	&= \mathcolor{gray}{\nabla^t} \left( P^{\;\!\mathcolor{gray}{t}}_{\;\! \symup{\iota}\mathcolor{gray}{z}} - \mathcolor{gray}{\nabla^{\hat{1}}} Q^{\;\!\mathcolor{gray}{t}}_{\;\! \symup{\iota} \mathcolor{gray}{\hat{1}} \mathcolor{gray}{z}} + \mathcolor{gray}{\nabla^{\hat{1}}} \mathcolor{gray}{\nabla^{\hat{2}}} O^{\;\!\mathcolor{gray}{t}}_{\;\! \symup{\iota} \mathcolor{gray}{\hat{1}\hat{2}} \mathcolor{gray}{z}} - \cdots \right) \hat{\symup{e}}^i~, \label{eq:j-e} \\ 
	\bar{J}^{\;\!\mathcolor{gray}{t}}_{\;\!\textcolor{Maroon}{\text{m}}\mathcolor{gray}{z}} &= \mathcolor{gray}{\bar{\nabla} \times} \left( \bar{M}^{\;\!\mathcolor{gray}{t}}_{\;\!\mathcolor{gray}{z}} - \mathcolor{gray}{\bar{\nabla} \cdot} \bar{\bar{N}}^{\;\!\mathcolor{gray}{t}}_{\;\!\mathcolor{gray}{z}} + \cdots \right) \label{eq:J-m} \\ 
	&= \epsilon^{\hphantom{\symup{\iota}\hat{1}}\hat{2}}_{\symup{\iota}\mathcolor{gray}{\hat{1}}} \mathcolor{gray}{\nabla^{\hat{1}}} \left( M^{\;\!\mathcolor{gray}{t}}_{\;\! \hat{2}\mathcolor{gray}{z}} - \mathcolor{gray}{\nabla^{\hat{3}}} N^{\;\!\mathcolor{gray}{t}}_{\;\! \hat{2} \mathcolor{gray}{\hat{3}} \mathcolor{gray}{z}} + \cdots \right) \hat{\symup{e}}^i~. \label{eq:j-m}
\end{align}
\end{subequations}
其中,$\bar{P}^{\;\!\mathcolor{gray}{t}}_{\;\!\mathcolor{gray}{z}}, \bar{M}^{\;\!\mathcolor{gray}{t}}_{\;\!\mathcolor{gray}{z}}$ 为电/磁偶极矩,$\bar{\bar{Q}}^{\;\!\mathcolor{gray}{t}}_{\;\!\mathcolor{gray}{z}}, \bar{\bar{N}}^{\;\!\mathcolor{gray}{t}}_{\;\!\mathcolor{gray}{z}}$ 为电/磁四极矩、$\bar{\bar{\bar{O}}}^{\;\!\mathcolor{gray}{t}}_{\;\!\mathcolor{gray}{z}}$ 为电八极矩。

然而,在强度\Footnote{相邻阶\textcolor{Plum}{多极}矩的强度比大约为 $10^{-6} = \left( 2\symup{\pi} \big/ 1\symup{\mu}\text{m} \cdot 1\text{\r{A}} \right)^2$。}上,\textcolor{Plum}{多极}矩的正确分组应是\cite{grahamMultipoleSolutionMacroscopic2000}\Footnote{从同阶磁矩弱于电矩、电子受到的电场力是洛伦兹力的c$\big/v$倍的角度,物质 和 CCD(电荷耦合器件)对电场的响应比对同等量级的磁场的响应大(至少对于高频电磁场=光的相互作用)。然而,电四/磁偶极矩的强度和其造成的影响,不一定比电偶极矩的小\cite{raabMultipoleTheoryElectromagnetism2004,OriginDependenceMaterial},包括其表面效应、响应的\textcolor{Plum}{线性}和\textcolor{Plum}{非线性}行为\cite{bethuneOpticalQuadrupoleSumfrequency1976}。}:
\begin{subequations}
	\abovedisplayskip=8pt
	\belowdisplayskip=10pt
\begin{align}
	\bar{P}^{\;\!\mathcolor{gray}{t}}_{\;\!\mathcolor{gray}{z}} \hspace{1em} &\ll \hspace{1em} \bar{\bar{Q}}^{\;\!\mathcolor{gray}{t}}_{\;\!\mathcolor{gray}{z}}, \bar{M}^{\;\!\mathcolor{gray}{t}}_{\;\!\mathcolor{gray}{z}} \hspace{-2.5em} &&\ll \hspace{1em} \bar{\bar{\bar{O}}}^{\;\!\mathcolor{gray}{t}}_{\;\!\mathcolor{gray}{z}}, \bar{\bar{N}}^{\;\!\mathcolor{gray}{t}}_{\;\!\mathcolor{gray}{z}} \label{eq:PQMON} \\ 
	\textcolor{Maroon}{\text{电偶极矩}} \hspace{1em} &\ll \hspace{1em} \textcolor{Maroon}{\text{电偶、磁四极矩}} \hspace{-2.5em} &&\ll \hspace{1em} \textcolor{Maroon}{\text{电八、磁四极矩}} ~. 
\end{align}
\end{subequations}
相应地将束缚电流源 \bref{eq:b=e+m} 展开并整理为:
\begin{subequations}
	\abovedisplayskip=8pt
	\belowdisplayskip=10pt
\begin{align}
\bar{J}^{\;\!\mathcolor{gray}{t}}_{\;\!\textcolor{Maroon}{\text{b}}\mathcolor{gray}{z}} = \mathcolor{gray}{\nabla^t} \bar{P}^{\;\!\mathcolor{gray}{t}}_{\;\!\mathcolor{gray}{z}} &- \left( \mathcolor{gray}{\bar{\nabla} \cdot} \mathcolor{gray}{\nabla^t} \bar{\bar{Q}}^{\;\!\mathcolor{gray}{t}}_{\;\!\mathcolor{gray}{z}} - \mathcolor{gray}{\bar{\nabla} \times} \bar{M}^{\;\!\mathcolor{gray}{t}}_{\;\!\mathcolor{gray}{z}} \right) \label{eq:J-b} \\ &+ \left[ \mathcolor{gray}{\bar{\nabla}} \mathcolor{gray}{\bar{\nabla} \colon}\! \mathcolor{gray}{\nabla^t} \bar{\bar{\bar{O}}}^{\;\!\mathcolor{gray}{t}}_{\;\!\mathcolor{gray}{z}} - \mathcolor{gray}{\bar{\nabla} \times} \left( \mathcolor{gray}{\bar{\nabla} \cdot}  \bar{\bar{N}}^{\;\!\mathcolor{gray}{t}}_{\;\!\mathcolor{gray}{z}} \right) \right] - \cdots~, \\
J^{\;\!\mathcolor{gray}{t}}_{\;\!\textcolor{Maroon}{\text{b}} \symup{\iota} \mathcolor{gray}{z}} = \mathcolor{gray}{\nabla^t} P^{\;\!\mathcolor{gray}{t}}_{\;\! \symup{\iota}\mathcolor{gray}{z}} &- \mathcolor{gray}{\nabla^{\hat{1}}} \left( \mathcolor{gray}{\nabla^t} Q^{\;\!\mathcolor{gray}{t}}_{\;\! \symup{\iota} \mathcolor{gray}{\hat{1}} \mathcolor{gray}{z}} - \epsilon^{\hphantom{\symup{\iota}\hat{1}}\hat{2}}_{\symup{\iota}\mathcolor{gray}{\hat{1}}} M^{\;\!\mathcolor{gray}{t}}_{\;\! \hat{2}\mathcolor{gray}{z}} \right) \label{eq:j-b} \\ &+ \mathcolor{gray}{\nabla^{\hat{1}}} \mathcolor{gray}{\nabla^{\hat{3}}} \left[ \mathcolor{gray}{\nabla^t} O^{\;\!\mathcolor{gray}{t}}_{\;\! \symup{\iota} \mathcolor{gray}{\hat{1}\hat{3}} \mathcolor{gray}{z}} - \epsilon^{\hphantom{\symup{\iota}\hat{1}}\hat{2}}_{\symup{\iota}\mathcolor{gray}{\hat{1}}} N^{\;\!\mathcolor{gray}{t}}_{\;\! \hat{2} \mathcolor{gray}{\hat{3}} \mathcolor{gray}{z}} \right] - \cdots~.
\end{align}
\end{subequations}

\clearpage
%\XGap{-10em}
\vspace*{-7.5em}

\marginLeft[-1.3em]{ssec:step-delta}\subsection{${\rho}, \bar{J}$ 表面源、${\mathbb{1}},\delta$ 场源展开}\label{ssec:step-delta}

考虑在 $\mathcolor{gray}{z} = \mathcolor{gray}{0}$ 处面接触的两个半无限介质 \textcolor{Maroon}{0} 和介质 \textcolor{Maroon}{1},\textcolor{Plum}{输入}场$=$\textcolor{NavyBlue}{泵浦}的能流方向从介质 \textcolor{Maroon}{0} $\to$ 介质 \textcolor{Maroon}{1},且与接触面内法向夹成锐角\Footnote{这里隐式地定义了\textcolor{PineGreen}{实验(室)}\textcolor{Plum}{坐标系}(\textcolor{PineGreen}{$\mathcal{Z}$ 系})的 \textcolor{Plum}{z 轴正向}为板状材料\textcolor{Plum}{输入面的内法向},见 \bref{hook:Z_frame}。注意,本文大部分(\textcolor{Plum}{自}/\textcolor{Plum}{因})\textcolor{Plum}{变量},若未特殊说明,则均运行在\textcolor{PineGreen}{实验室}\textcolor{Plum}{坐标系}下。}。在上述\textcolor{Maroon}{初始条件}下,电荷/流源 ${\rho}^{\;\!\mathcolor{gray}{t}}_{\;\!\mathcolor{gray}{z}}, \bar{J}^{\;\!\mathcolor{gray}{t}}_{\;\!\mathcolor{gray}{z}}$ 中的每一个\Footnote{是将源 ${\rho}^{\;\!\mathcolor{gray}{t}}_{\;\!\mathcolor{gray}{z}}, \bar{J}^{\;\!\mathcolor{gray}{t}}_{\;\!\mathcolor{gray}{z}}$ 本身,而不是将构成源的底层元素们 --- 束缚电/磁\textcolor{Plum}{多极}矩强度 $\bar{P}^{\;\!\mathcolor{gray}{t}}_{\;\!\mathcolor{gray}{z}},\bar{\bar{Q}}^{\;\!\mathcolor{gray}{t}}_{\;\!\mathcolor{gray}{z}},\bar{\bar{\bar{O}}}^{\;\!\mathcolor{gray}{t}}_{\;\!\mathcolor{gray}{z}} ; \bar{M}^{\;\!\mathcolor{gray}{t}}_{\;\!\mathcolor{gray}{z}}, \bar{\bar{N}}^{\;\!\mathcolor{gray}{t}}_{\;\!\mathcolor{gray}{z}}$,展开成 ${\mathbb{1}},\delta$ 的函数。 ---  若将后者展开,则 ${\rho}^{\;\!\mathcolor{gray}{t}}_{\;\!\mathcolor{gray}{z}}$ 中也将产生 $\delta$ 项,并与 $\bar{J}^{\;\!\mathcolor{gray}{t}}_{\;\!\mathcolor{gray}{z}}$ 中的同层次 $\delta$ 项抵消,无法提供信息。} 都可以写成 2 个\textcolor{Plum}{阶跃函数} / \textcolor{Plum}{Heaviside 单位函数}
\abovedisplayskip=8pt
\belowdisplayskip=10pt
\begin{align} \label{eq:u}
	\mathbb{1}_{\mathcolor{gray}{z}} := \mathbb{1} \left( \mathcolor{gray}{z} \right) = ~\left\{~ \begin{aligned} 
		&1 &&, \mathcolor{gray}{z \mathcolor{black}{>} 0} \\ 
		&\textcolor{Plum}{\text{undefined}} &&, \mathcolor{gray}{z \mathcolor{black}{=} 0} \\
		&0 &&, \mathcolor{gray}{z \mathcolor{black}{<} 0} \end{aligned}\right. ~,
\end{align}
之和:即使用
\begin{subequations} \label{eq:u-01}
	\abovedisplayskip=0pt
	\belowdisplayskip=8pt
\begin{align}
	\leftindex_{\textcolor{Maroon}{1}} {\mathbb{1}}_{\mathcolor{gray}{z}} &= {\mathbb{1}}_{\mathcolor{gray}{z}} ~, \label{eq:u-1} \\ 
	\leftindex_{\textcolor{Maroon}{0}} {\mathbb{1}}_{\mathcolor{gray}{z}} &= {\mathbb{1}}_{ - \mathcolor{gray}{z} } ~, \label{eq:u-0}
\end{align}
\end{subequations}
将电荷/流源 $\rho_{\;\!\textcolor{Maroon}{\text{b}}}, J_{\;\!\textcolor{Maroon}{\text{b}}}$ 中的每一个(均暂记为 $X$)暂表达为:
\abovedisplayskip=6pt
\belowdisplayskip=8pt
\begin{align} \label{eq:X-01}
	X_{\mathcolor{gray}{z}} = \leftindex_{\textcolor{Maroon}{\mathsfit{z}}} {\mathbb{1}}_{\mathcolor{gray}{z}} \leftindex^{\textcolor{Maroon}{\mathsfit{z}}} X_{\mathcolor{gray}{z}} ~~~~, \text{其中} ~~~ \textcolor{Maroon}{\mathsfit{z}} = \textcolor{Maroon}{0,1} ~,
\end{align}
然后将二者代入由 \bref{eq:div-e,eq:div-e-f} 构建的\textcolor{Maroon}{束缚电荷} ${Q}^{\;\!\mathcolor{gray}{t}}_{\;\!\textcolor{Maroon}{\text{b}}\mathcolor{gray}{z}}$ \textcolor{Maroon}{守恒}\Footnote{束缚电荷 ${Q}^{\;\!\mathcolor{gray}{t}}_{\;\!\textcolor{Maroon}{\text{b}}\mathcolor{gray}{z}}$ 与 自由电荷 ${Q}^{\;\!\mathcolor{gray}{t}}_{\;\!\textcolor{Maroon}{\text{f}}\mathcolor{gray}{z}}$ 可以相互转化,因此各自单独而言在最广义上不(一定)守恒:通过价带/杂质能级到导带的跃迁,比如物质对光子的\textcolor{Plum}{线性}、\textcolor{Plum}{非线性}\textcolor{NavyBlue}{吸收}并产生光电压/流所对应的内/外光伏/电效应\cite{boydNonlinearOptics2019};因此 \bref{eq:div-e-b} 和 \bref{eq:div-e-f} 均不是定律,相比\textcolor{Maroon}{总电荷} ${Q}^{\;\!\mathcolor{gray}{t}}_{\;\!\mathcolor{gray}{z}}$ \textcolor{Maroon}{守恒} \bref{eq:div-e} 而言\cite{markelExternalInducedFree2018}。此外,外加/固有、感应/永久电荷,均可处于束缚/自由态,3 组概念互相独立\cite{markelExternalInducedFree2018}。这里暂时不考虑摩擦起电、外光电效应、激光加工之强场激发等离子体\cite{boydNonlinearOptics2019,dengTheoryElectrodynamicResponse2020}等非平衡/含驰豫现象,因此暂认为\textcolor{Maroon}{束缚、自由电荷分别独立守恒}。}中\cite{grahamMultipoleSolutionMacroscopic2000,delangeElectromagneticBoundaryConditions2013}:
\begin{subequations}
	\abovedisplayskip=6pt
	\belowdisplayskip=8pt
\begin{align}
	\mathcolor{gray}{\nabla^\iota} J^{\;\!\mathcolor{gray}{t}}_{\;\!\textcolor{Maroon}{\text{b}} \mathcolor{gray}{\symup{\iota}} \mathcolor{gray}{z}} &+ \mathcolor{gray}{\nabla^t} {\rho}^{\;\!\mathcolor{gray}{t}}_{\;\!\textcolor{Maroon}{\text{b}}\mathcolor{gray}{z}} &&\hspace{-2em}= 0 \label{eq:div-e-b} \\ 
	\mathcolor{gray}{\nabla^\iota} \left( \leftindex_{\textcolor{Maroon}{\mathsfit{z}}} {\mathbb{1}}_{\mathcolor{gray}{z}} \leftindex^{\textcolor{Maroon}{\mathsfit{z}}} \;\! J^{\;\!\mathcolor{gray}{t}}_{\;\!\textcolor{Maroon}{\text{b}} \mathcolor{gray}{\symup{\iota}} \mathcolor{gray}{z}} \right) &+ \mathcolor{gray}{\nabla^t} \left( \leftindex_{\textcolor{Maroon}{\mathsfit{z}}} {\mathbb{1}}_{\mathcolor{gray}{z}} \leftindex^{\textcolor{Maroon}{\mathsfit{z}}} {\rho}^{\;\!\mathcolor{gray}{t}}_{\;\!\textcolor{Maroon}{\text{b}}\mathcolor{gray}{z}} \right) &&\hspace{-2em}= 0 \label{eq:div-e-b-01} \\ 
	\leftindex_{\textcolor{Maroon}{\mathsfit{z}}} {\mathbb{1}}_{\mathcolor{gray}{z}} \left( \mathcolor{gray}{\nabla_x} \leftindex^{\textcolor{Maroon}{\mathsfit{z}}} \;\! J^{\;\!\mathcolor{gray}{t}}_{\;\!\textcolor{Maroon}{\text{b}} \mathcolor{gray}{\symup{x}} \mathcolor{gray}{z}} + \mathcolor{gray}{\nabla_y} \leftindex^{\textcolor{Maroon}{\mathsfit{z}}} \;\! J^{\;\!\mathcolor{gray}{t}}_{\;\!\textcolor{Maroon}{\text{b}} \mathcolor{gray}{\symup{y}} \mathcolor{gray}{z}} \right) + \mathcolor{gray}{\nabla_z} \left( \leftindex_{\textcolor{Maroon}{\mathsfit{z}}} {\mathbb{1}}_{\mathcolor{gray}{z}} \leftindex^{\textcolor{Maroon}{\mathsfit{z}}} \;\! J^{\;\!\mathcolor{gray}{t}}_{\;\!\textcolor{Maroon}{\text{b}} \mathcolor{gray}{\symup{z}} \mathcolor{gray}{z}} \right) &+ \leftindex_{\textcolor{Maroon}{\mathsfit{z}}} {\mathbb{1}}_{\mathcolor{gray}{z}} \mathcolor{gray}{\nabla^t} \leftindex^{\textcolor{Maroon}{\mathsfit{z}}} {\rho}^{\;\!\mathcolor{gray}{t}}_{\;\!\textcolor{Maroon}{\text{b}}\mathcolor{gray}{z}} &&\hspace{-2em}= 0~, \label{eq:div-e-b-01'}
\end{align}
\end{subequations}
注意到,在空域上对 $\leftindex_{\textcolor{Maroon}{\mathsfit{z}}} {\mathbb{1}}_{\mathcolor{gray}{z}} \leftindex^{\textcolor{Maroon}{\mathsfit{z}}} X_{\mathcolor{gray}{z}}$ 求 $\mathcolor{gray}{z}$ 向一阶或高阶偏导 $\mathcolor{gray}{\nabla_z}$,即
\begin{subequations}
	\abovedisplayskip=6pt
	\belowdisplayskip=8pt
\begin{align}
	\mathcolor{gray}{\nabla_z} \left( \leftindex_{\textcolor{Maroon}{\mathsfit{z}}} {\mathbb{1}}_{\mathcolor{gray}{z}} \leftindex^{\textcolor{Maroon}{\mathsfit{z}}} X_{\mathcolor{gray}{z}} \right) &= \leftindex_{\textcolor{Maroon}{\mathsfit{z}}} \;\! \delta_{\mathcolor{gray}{z}} \leftindex^{\textcolor{Maroon}{\mathsfit{z}}} X_{\mathcolor{gray}{z}} + \leftindex_{\textcolor{Maroon}{\mathsfit{z}}} {\mathbb{1}}_{\mathcolor{gray}{z}} \left( \mathcolor{gray}{\nabla_z} \leftindex^{\textcolor{Maroon}{\mathsfit{z}}} X_{\mathcolor{gray}{z}} \right) ~, \label{eq:partial_z-step} \\ 
	\mathcolor{gray}{\nabla_z^2} \left( \leftindex_{\textcolor{Maroon}{\mathsfit{z}}} {\mathbb{1}}_{\mathcolor{gray}{z}} \leftindex^{\textcolor{Maroon}{\mathsfit{z}}} X_{\mathcolor{gray}{z}} \right) &= \leftindex_{\textcolor{Maroon}{\mathsfit{z}}} \;\! \delta'_{\mathcolor{gray}{z}} \leftindex^{\textcolor{Maroon}{\mathsfit{z}}} X_{\mathcolor{gray}{z}} + 2 \leftindex_{\textcolor{Maroon}{\mathsfit{z}}} \;\! \delta_{\mathcolor{gray}{z}} \left( \mathcolor{gray}{\nabla_z} \leftindex^{\textcolor{Maroon}{\mathsfit{z}}} X_{\mathcolor{gray}{z}} \right) + \leftindex_{\textcolor{Maroon}{\mathsfit{z}}} {\mathbb{1}}_{\mathcolor{gray}{z}} \left( \mathcolor{gray}{\nabla_z^2} \leftindex^{\textcolor{Maroon}{\mathsfit{z}}} X_{\mathcolor{gray}{z}} \right) ~, \label{eq:partial_z^2-step}
\end{align}
\end{subequations}
会产生(带有单位 $\mathcolor{gray}{\symup{m}^{-1}}$ 的)$\delta_{\mathcolor{gray}{z}}$ 函数
\begin{subequations}
	\abovedisplayskip=8pt
	\belowdisplayskip=10pt
\begin{align}
	\leftindex_{\textcolor{Maroon}{1}} \;\! \delta_{\mathcolor{gray}{z}} &= \mathcolor{gray}{\nabla_z} \leftindex_{\textcolor{Maroon}{1}} {\mathbb{1}}_{\mathcolor{gray}{z}} = {\mathbb{1}'}_{\mathcolor{gray}{z}} = \delta_{\mathcolor{gray}{z}} ~, \label{eq:delta-1} \\ 
	\leftindex_{\textcolor{Maroon}{0}} \;\! \delta_{\mathcolor{gray}{z}} &= \leftindex_{\textcolor{Maroon}{0}} {\mathbb{1}'}_{\mathcolor{gray}{z}} = - \leftindex_{\textcolor{Maroon}{1}} {\mathbb{1}'}_{\mathcolor{gray}{z}} = -\hspace{0.2em} \delta_{\mathcolor{gray}{z}} ~, \label{eq:delta-0}
\end{align}
\end{subequations}
及其导数 $\delta'_{\mathcolor{gray}{z}}$(含单位 $\mathcolor{gray}{\symup{m}^{-2}}$)等。 ---  具体到 \bref{eq:div-e-b-01} 的情况,即对 $\bar{J}^{\;\!\mathcolor{gray}{t}}_{\;\!\textcolor{Maroon}{\text{b}}\mathcolor{gray}{z}}$ 的 $\mathcolor{gray}{z}$ 向散度 $\mathcolor{gray}{\nabla_z} \left( \leftindex_{\textcolor{Maroon}{\mathsfit{z}}} {\mathbb{1}}_{\mathcolor{gray}{z}} \leftindex^{\textcolor{Maroon}{\mathsfit{z}}} \;\! J^{\;\!\mathcolor{gray}{t}}_{\;\!\textcolor{Maroon}{\text{b}} \mathcolor{gray}{\symup{z}} \mathcolor{gray}{z}} \right)$,将产生 \bref{eq:partial_z-step} 中的 $\leftindex_{\textcolor{Maroon}{\mathsfit{z}}} \;\! \delta_{\mathcolor{gray}{z}} \leftindex^{\textcolor{Maroon}{\mathsfit{z}}} \;\! J^{\;\!\mathcolor{gray}{t}}_{\;\!\textcolor{Maroon}{\text{b}} \symup{z} \mathcolor{gray}{z}}$ 项,该项只能通过 ${\rho}^{\;\!\mathcolor{gray}{t}}_{\;\!\textcolor{Maroon}{\text{b}}\mathcolor{gray}{z}}$ 中也原始地就含有对应的表面 $\delta_{\mathcolor{gray}{z}}$ 项(如铁电体 $\textcolor{Maroon}{-\symup{c}}$ 面薄层内的束缚电子 ${\sigma}^{\;\!\mathcolor{gray}{t}}_{\;\!\textcolor{Maroon}{\text{b}}\mathcolor{gray}{z}} = {\mathcal{P}}^{\;\!\mathcolor{gray}{t}}_{\;\!\textcolor{Maroon}{\text{b}} \symup{z} \mathcolor{gray}{z}}$;它在单位上比 ${\rho}^{\;\!\mathcolor{gray}{t}}_{\;\!\textcolor{Maroon}{\text{b}}\mathcolor{gray}{z}}$ 多个 $\mathcolor{gray}{\symup{m}}$)以\textcolor{Plum}{吸收}/\textcolor{Plum}{消除}掉 \cite{grahamMultipoleSolutionMacroscopic2000},否则将违反 \bref{eq:div-e-b}\Footnote{这归根结底是因为  ---  在 \bref{eq:div-e-b} 中,对 ${\rho}^{\;\!\mathcolor{gray}{t}}_{\;\!\textcolor{Maroon}{\text{b}}\mathcolor{gray}{z}}$ 只有时间偏导 $\mathcolor{gray}{\nabla^t}$,然而对 $\bar{J}^{\;\!\mathcolor{gray}{t}}_{\;\!\textcolor{Maroon}{\text{b}}\mathcolor{gray}{z}}$ 却有空域偏导 $\mathcolor{gray}{\bar{\nabla}}$,特别是对 $\mathcolor{gray}{z}$ 的偏导 $\mathcolor{gray}{\nabla_z}$:这种\textcolor{Plum}{不对称性}最终会强制拓展 ${\rho}^{\;\!\mathcolor{gray}{t}}_{\;\!\textcolor{Maroon}{\text{b}}\mathcolor{gray}{z}}$ 的 \bref{eq:X-01} 并为之带来下一阶不\textcolor{Plum}{连续}/表面项。}。

此外,不仅 \bref{eq:X-01} 需要为 ${\rho}^{\;\!\mathcolor{gray}{t}}_{\;\!\textcolor{Maroon}{\text{b}}\mathcolor{gray}{z}}$ 拓展,以适应对 $\bar{J}^{\;\!\mathcolor{gray}{t}}_{\;\!\textcolor{Maroon}{\text{b}}\mathcolor{gray}{z}}$ 的 \bref{eq:X-01} 展开的一阶 $\mathcolor{gray}{z}$ 向偏导所生成的 $\delta_{\mathcolor{gray}{z}}$ 函数  ---  $\bar{J}^{\;\!\mathcolor{gray}{t}}_{\;\!\textcolor{Maroon}{\text{b}}\mathcolor{gray}{z}}$ 本身还有表面电流(如磁光材料),相应地 $\bar{J}^{\;\!\mathcolor{gray}{t}}_{\;\!\textcolor{Maroon}{\text{b}}\mathcolor{gray}{z}}$ 的原始 \bref{eq:X-01} 也应该添加一个 $\delta_{\mathcolor{gray}{z}}$ 函数;那么为满足 \bref{eq:X-01},对应 ${\rho}^{\;\!\mathcolor{gray}{t}}_{\;\!\textcolor{Maroon}{\text{b}}\mathcolor{gray}{z}}$ 的表达式会继续出现 $\delta'_{\mathcolor{gray}{z}}$ 项。因此需要同时将 ${\rho}^{\;\!\mathcolor{gray}{t}}_{\;\!\textcolor{Maroon}{\text{b}}\mathcolor{gray}{z}},\bar{J}^{\;\!\mathcolor{gray}{t}}_{\;\!\textcolor{Maroon}{\text{b}}\mathcolor{gray}{z}}$ 分别拓展 \bref{eq:X-01} 至:
\begin{subequations} \label{eq:e-b-01}
%	\abovedisplayskip=-3pt
%	\belowdisplayskip=12pt
\begin{align}
	J^{\;\!\mathcolor{gray}{t}}_{\;\!\textcolor{Maroon}{\text{b}} \symup{\iota}\mathcolor{gray}{z}} &= &&\hspace{-4.5em}\leftindex_{\textcolor{Maroon}{\mathsfit{z}}} {\mathbb{1}}_{\mathcolor{gray}{z}} \leftindex^{\textcolor{Maroon}{\mathsfit{z}}} \;\! J^{\;\!\mathcolor{gray}{t}}_{\;\!\textcolor{Maroon}{\text{b}} \symup{\iota}\mathcolor{gray}{z}} &&\hspace{-4.5em}+ \delta_{\mathcolor{gray}{z}} \leftindex_{\textcolor{Maroon}{\mathsfit{z}}} \;\! \leftindex^{\textcolor{Maroon}{\mathsfit{z}}}
	{\mathcal{K}}^{\;\!\mathcolor{gray}{t}}_{\;\!\textcolor{Maroon}{\text{b}} \symup{\iota}\symup{z}\mathcolor{gray}{z}} \leftindex^{\textcolor{Maroon}{\mathsfit{z}}} \;\! n_{\mathcolor{gray}{z}} &&\hspace{-4.5em} \\ 
	&= &&\hspace{-4.5em}\leftindex_{\textcolor{Maroon}{\mathsfit{z}}} {\mathbb{1}}_{\mathcolor{gray}{z}} \leftindex^{\textcolor{Maroon}{\mathsfit{z}}} \;\! J^{\;\!\mathcolor{gray}{t}}_{\;\!\textcolor{Maroon}{\text{b}} \symup{\iota}\mathcolor{gray}{z}} &&\hspace{-4.5em}- \leftindex_{\textcolor{Maroon}{\mathsfit{z}}} \;\! \delta_{\mathcolor{gray}{z}} \leftindex^{\textcolor{Maroon}{\mathsfit{z}}}
	{\mathcal{K}}^{\;\!\mathcolor{gray}{t}}_{\;\!\textcolor{Maroon}{\text{b}} \symup{\iota}\symup{z}\mathcolor{gray}{z}} ~, &&\hspace{-4.5em} \label{eq:j-b-01} \\
	{\rho}^{\;\!\mathcolor{gray}{t}}_{\;\!\textcolor{Maroon}{\text{b}}\mathcolor{gray}{z}} &= &&\hspace{-4.5em}\leftindex_{\textcolor{Maroon}{\mathsfit{z}}} {\mathbb{1}}_{\mathcolor{gray}{z}} \leftindex^{\textcolor{Maroon}{\mathsfit{z}}} {\rho}^{\;\!\mathcolor{gray}{t}}_{\;\!\textcolor{Maroon}{\text{b}}\mathcolor{gray}{z}} &&\hspace{-4.5em}+ \delta_{\mathcolor{gray}{z}} \leftindex_{\textcolor{Maroon}{\mathsfit{z}}} \;\! \leftindex^{\textcolor{Maroon}{\mathsfit{z}}} \;\! {\mathcal{P}}^{\;\!\mathcolor{gray}{t}}_{\;\!\textcolor{Maroon}{\text{b}} \symup{z} \mathcolor{gray}{z}} \leftindex^{\textcolor{Maroon}{\mathsfit{z}}} \;\! n_{\mathcolor{gray}{z}} &&\hspace{-4.5em}+ \delta'_{\mathcolor{gray}{z}} \leftindex_{\textcolor{Maroon}{\mathsfit{z}}} \;\! \leftindex^{\textcolor{Maroon}{\mathsfit{z}}} \;\! {\mathcal{Q}}^{\;\!\mathcolor{gray}{t}}_{\;\!\textcolor{Maroon}{\text{b}} \symup{z} \symup{z} \mathcolor{gray}{z}} \leftindex^{\textcolor{Maroon}{\mathsfit{z}}} \;\! n_{\mathcolor{gray}{z}} \\ 
	&= &&\hspace{-4.5em}\leftindex_{\textcolor{Maroon}{\mathsfit{z}}} {\mathbb{1}}_{\mathcolor{gray}{z}} \leftindex^{\textcolor{Maroon}{\mathsfit{z}}} {\rho}^{\;\!\mathcolor{gray}{t}}_{\;\!\textcolor{Maroon}{\text{b}}\mathcolor{gray}{z}} &&\hspace{-4.5em}- \leftindex_{\textcolor{Maroon}{\mathsfit{z}}} \;\! \delta_{\mathcolor{gray}{z}} \leftindex^{\textcolor{Maroon}{\mathsfit{z}}} \;\! {\mathcal{P}}^{\;\!\mathcolor{gray}{t}}_{\;\!\textcolor{Maroon}{\text{b}} \symup{z} \mathcolor{gray}{z}} &&\hspace{-4.5em}- \leftindex_{\textcolor{Maroon}{\mathsfit{z}}} \;\! \delta'_{\mathcolor{gray}{z}} \leftindex^{\textcolor{Maroon}{\mathsfit{z}}} \;\! {\mathcal{Q}}^{\;\!\mathcolor{gray}{t}}_{\;\!\textcolor{Maroon}{\text{b}} \symup{z} \symup{z} \mathcolor{gray}{z}} ~, \label{eq:p-b-01}
\end{align}
\end{subequations}
其中,定义了 介质 $\textcolor{Maroon}{\mathsfit{z}}$ 在 $\mathcolor{gray}{z} = \mathcolor{gray}{0}$ 面上的表面单位外法向量 $\leftindex^{\textcolor{Maroon}{\mathsfit{z}}} {\hat{n}}_{\mathcolor{gray}{z}}$ 与 $\hat{\symup{e}}_{\mathcolor{gray}{z}}$ 的\textcolor{Plum}{点积}
\begin{align} \label{eq:surf-normal-dot}
	\leftindex^{\textcolor{Maroon}{\mathsfit{z}}} \;\! n_{\mathcolor{gray}{z}} := - \symup{sgn} \left( \leftindex^{\textcolor{Maroon}{\mathsfit{z}}} \;\! \delta_{\mathcolor{gray}{z}} \right) = - \symup{sgn} \left( \leftindex^{\textcolor{Maroon}{\mathsfit{z}}} \;\! \delta'_{\mathcolor{gray}{z}} \right) = \cdots = \leftindex^{\textcolor{Maroon}{\mathsfit{z}}} {\hat{n}}_{\mathcolor{gray}{z}} \cdot \hat{\symup{e}}_{\mathcolor{gray}{z}}~.
\end{align}
为一步到位,\bref{eq:j-b-01,eq:p-b-01} 还可以进一步拓展至高一阶项:
\begin{subequations} \label{eq:e-b-01'}
\begin{align}
	J^{\;\!\mathcolor{gray}{t}}_{\;\!\textcolor{Maroon}{\text{b}} \symup{\iota}\mathcolor{gray}{z}} &= &&\hspace{-2em}\leftindex_{\textcolor{Maroon}{\mathsfit{z}}} {\mathbb{1}}_{\mathcolor{gray}{z}} \leftindex^{\textcolor{Maroon}{\mathsfit{z}}} \;\! J^{\;\!\mathcolor{gray}{t}}_{\;\!\textcolor{Maroon}{\text{b}} \symup{\iota}\mathcolor{gray}{z}} &&\hspace{-2em}- \leftindex_{\textcolor{Maroon}{\mathsfit{z}}} \;\! \delta_{\mathcolor{gray}{z}} \leftindex^{\textcolor{Maroon}{\mathsfit{z}}}
	{\mathcal{K}}^{\;\!\mathcolor{gray}{t}}_{\;\!\textcolor{Maroon}{\text{b}} \symup{\iota}\symup{z}\mathcolor{gray}{z}} &&\hspace{-2em}- \leftindex_{\textcolor{Maroon}{\mathsfit{z}}} \;\! \delta'_{\mathcolor{gray}{z}} \leftindex^{\textcolor{Maroon}{\mathsfit{z}}} \;\! {\mathcal{L}}^{\;\!\mathcolor{gray}{t}}_{\;\!\textcolor{Maroon}{\text{b}} \symup{\iota}\symup{z} \symup{z} \mathcolor{gray}{z}} ~, &&\hspace{-2em} \label{eq:j-b-01'} \\
	{\rho}^{\;\!\mathcolor{gray}{t}}_{\;\!\textcolor{Maroon}{\text{b}}\mathcolor{gray}{z}} &= &&\hspace{-2em}\leftindex_{\textcolor{Maroon}{\mathsfit{z}}} {\mathbb{1}}_{\mathcolor{gray}{z}} \leftindex^{\textcolor{Maroon}{\mathsfit{z}}} {\rho}^{\;\!\mathcolor{gray}{t}}_{\;\!\textcolor{Maroon}{\text{b}}\mathcolor{gray}{z}} &&\hspace{-2em}- \leftindex_{\textcolor{Maroon}{\mathsfit{z}}} \;\! \delta_{\mathcolor{gray}{z}} \leftindex^{\textcolor{Maroon}{\mathsfit{z}}} \;\! {\mathcal{P}}^{\;\!\mathcolor{gray}{t}}_{\;\!\textcolor{Maroon}{\text{b}} \symup{z} \mathcolor{gray}{z}} &&\hspace{-2em}- \leftindex_{\textcolor{Maroon}{\mathsfit{z}}} \;\! \delta'_{\mathcolor{gray}{z}} \leftindex^{\textcolor{Maroon}{\mathsfit{z}}} \;\! {\mathcal{Q}}^{\;\!\mathcolor{gray}{t}}_{\;\!\textcolor{Maroon}{\text{b}} \symup{z} \symup{z} \mathcolor{gray}{z}} &&\hspace{-2em}- \leftindex_{\textcolor{Maroon}{\mathsfit{z}}} \;\! \delta''_{\mathcolor{gray}{z}} \leftindex^{\textcolor{Maroon}{\mathsfit{z}}} \;\! {\mathcal{O}}^{\;\!\mathcolor{gray}{t}}_{\;\!\textcolor{Maroon}{\text{b}} \symup{z} \symup{z} \symup{z} \mathcolor{gray}{z}} ~, \label{eq:p-b-01'}
\end{align}
\end{subequations}
现将 \bref{eq:j-b-01',eq:p-b-01'} 代入 \bref{eq:div-e-b} 中\Footnote{利用 \bref{eq:div-e-b-01'} 和 \bref{eq:partial_z-step} 所导出的:$\mathcolor{gray}{\nabla^\iota} \left( \leftindex_{\textcolor{Maroon}{\mathsfit{z}}} {\mathbb{1}}_{\mathcolor{gray}{z}} \leftindex^{\textcolor{Maroon}{\mathsfit{z}}} \;\! J^{\;\!\mathcolor{gray}{t}}_{\;\!\textcolor{Maroon}{\text{b}} \mathcolor{gray}{\symup{\iota}} \mathcolor{gray}{z}} \right) = \leftindex_{\textcolor{Maroon}{\mathsfit{z}}} \;\! \delta_{\mathcolor{gray}{z}} \leftindex^{\textcolor{Maroon}{\mathsfit{z}}} \;\! J^{\;\!\mathcolor{gray}{t}}_{\;\!\textcolor{Maroon}{\text{b}} \symup{z} \mathcolor{gray}{z}} + \leftindex_{\textcolor{Maroon}{\mathsfit{z}}} {\mathbb{1}}_{\mathcolor{gray}{z}} \mathcolor{gray}{\nabla^\iota} \leftindex^{\textcolor{Maroon}{\mathsfit{z}}} \;\! J^{\;\!\mathcolor{gray}{t}}_{\;\!\textcolor{Maroon}{\text{b}} \mathcolor{gray}{\symup{\iota}} \mathcolor{gray}{z}}$。},同介质 $\textcolor{Maroon}{\mathsfit{z}}$ 内(另一侧介质是真空时也须成立),各同阶次内的\textcolor{Plum}{奇异}项必须分别(分层次)为零,因此有

\clearpage
\vspace*{-5.0em}

\begin{subequations} \label{eq:div-e-b-01-deltas}
	\abovedisplayskip=7pt
	\belowdisplayskip=8pt
\begin{align}
	{\delta}_{\mathcolor{gray}{z}} ~\textcolor{Maroon}{\text{项}}:&\hspace{1.0em} + \leftindex^{\textcolor{Maroon}{\mathsfit{z}}} {\delta}_{\mathcolor{gray}{z}} \leftindex^{\textcolor{Maroon}{\mathsfit{z}}} \;\! J^{\;\!\mathcolor{gray}{t}}_{\;\!\textcolor{Maroon}{\text{b}} \symup{z}\mathcolor{gray}{z}} &&\hspace{-1.5em}- \leftindex^{\textcolor{Maroon}{\mathsfit{z}}} {\delta}_{\mathcolor{gray}{z}} \left( \mathcolor{gray}{\nabla^\iota} \leftindex^{\textcolor{Maroon}{\mathsfit{z}}}
	{\mathcal{K}}^{\;\!\mathcolor{gray}{t}}_{\;\!\textcolor{Maroon}{\text{b}} \mathcolor{gray}{\symup{\iota}} \symup{z}\mathcolor{gray}{z}} \right. &&\hspace{-1.5em}+ \left. \mathcolor{gray}{\nabla^t} \leftindex^{\textcolor{Maroon}{\mathsfit{z}}} \;\! {\mathcal{P}}^{\;\!\mathcolor{gray}{t}}_{\;\!\textcolor{Maroon}{\text{b}} \symup{z} \mathcolor{gray}{z}} \right) &&\hspace{-1.5em}= 0~, \label{eq:div-e-b-01-delta} \\
	{\delta}'_{\mathcolor{gray}{z}} ~\textcolor{Maroon}{\text{项}}:&\hspace{1.0em} - \leftindex^{\textcolor{Maroon}{\mathsfit{z}}} \delta'_{\mathcolor{gray}{z}} \leftindex^{\textcolor{Maroon}{\mathsfit{z}}}
	{\mathcal{K}}^{\;\!\mathcolor{gray}{t}}_{\;\!\textcolor{Maroon}{\text{b}} \symup{z} \symup{z}\mathcolor{gray}{z}} &&\hspace{-1.5em}- \leftindex^{\textcolor{Maroon}{\mathsfit{z}}} \delta'_{\mathcolor{gray}{z}} \left( \mathcolor{gray}{\nabla^\iota} \leftindex^{\textcolor{Maroon}{\mathsfit{z}}} \;\! {\mathcal{L}}^{\;\!\mathcolor{gray}{t}}_{\;\!\textcolor{Maroon}{\text{b}} \mathcolor{gray}{\symup{\iota}} \symup{z} \symup{z} \mathcolor{gray}{z}} \right. &&\hspace{-1.5em}+ \left. \mathcolor{gray}{\nabla^t} \leftindex^{\textcolor{Maroon}{\mathsfit{z}}} \;\! {\mathcal{Q}}^{\;\!\mathcolor{gray}{t}}_{\;\!\textcolor{Maroon}{\text{b}} \symup{z} \symup{z} \mathcolor{gray}{z}} \right) &&\hspace{-1.5em}= 0~, \label{eq:div-e-b-01-delta'} \\
	{\delta}''_{\mathcolor{gray}{z}} ~\textcolor{Maroon}{\text{项}}:&\hspace{1.0em} - \leftindex^{\textcolor{Maroon}{\mathsfit{z}}} {\delta}''_{\mathcolor{gray}{z}} \leftindex^{\textcolor{Maroon}{\mathsfit{z}}} \;\! {\mathcal{L}}^{\;\!\mathcolor{gray}{t}}_{\;\!\textcolor{Maroon}{\text{b}} \symup{z} \symup{z} \symup{z} \mathcolor{gray}{z}} &&\hspace{-1.5em}- \leftindex^{\textcolor{Maroon}{\mathsfit{z}}} \delta''_{\mathcolor{gray}{z}} \left( \mathcolor{gray}{\nabla^\iota} \leftindex^{\textcolor{Maroon}{\mathsfit{z}}} \;\! \cdots \right. &&\hspace{-1.5em}+ \left. \mathcolor{gray}{\nabla^t} \leftindex^{\textcolor{Maroon}{\mathsfit{z}}} \;\! {\mathcal{O}}^{\;\!\mathcolor{gray}{t}}_{\;\!\textcolor{Maroon}{\text{b}} \symup{z} \symup{z} \symup{z} \mathcolor{gray}{z}} \right) &&\hspace{-1.5em}= 0~, \label{eq:div-e-b-01-delta''}
\end{align}
\end{subequations}
即有 下述关系,与介质无关地 成立(实际上利用到了 \bref{eq:Intdeltasum=0})\Footnote{由于 2 个下标相同,看上去 \bref{eq:div-e-b-01-delta'-conclusion} 中的 $\mathcolor{gray}{\nabla^\iota} {\mathcal{L}}^{\;\!\mathcolor{gray}{t}}_{\;\!\textcolor{Maroon}{\text{m}} \mathcolor{gray}{\symup{\iota}} \symup{z} \symup{z} \mathcolor{gray}{z}} = 0$ 但实则不然;\bref{eq:div-e-b-01-delta''-conclusion} 只考虑到电八/磁四级(忽略更高阶矩)。}:
\begin{subequations} \label{eq:div-e-b-01-delta-conclusions}
	\abovedisplayskip=7pt
	\belowdisplayskip=8pt
\begin{align}
	{\delta}_{\mathcolor{gray}{z}} ~\textcolor{Maroon}{\text{项}}:&\hspace{1.0em}  J^{\;\!\mathcolor{gray}{t}}_{\;\!\textcolor{Maroon}{\text{b}} \symup{z} \mathcolor{gray}{0}} \hspace{-1.7em}&&=\hspace{0.2em} + \hspace{0.2em} \left( \mathcolor{gray}{\nabla^\iota} {\mathcal{K}}^{\;\!\mathcolor{gray}{t}}_{\;\!\textcolor{Maroon}{\text{b}} \mathcolor{gray}{\symup{\iota}} \symup{z}\mathcolor{gray}{0}} \hspace{-2.5em}\right. &&\hspace{0.8em}+ \left. \mathcolor{gray}{\nabla^t} {\mathcal{P}}^{\;\!\mathcolor{gray}{t}}_{\;\!\textcolor{Maroon}{\text{b}} \symup{z} \mathcolor{gray}{0}} \right) \hspace{-1.6em}&&=+\hspace{0.2em} \mathcolor{gray}{\nabla^t} {\mathcal{P}}^{\;\!\mathcolor{gray}{t}}_{\;\!\textcolor{Maroon}{\text{b}} \symup{z} \mathcolor{gray}{0}} + \mathcolor{gray}{\nabla^\iota} {\mathcal{K}}^{\;\!\mathcolor{gray}{t}}_{\;\!\textcolor{Maroon}{\text{b}} \mathcolor{gray}{\symup{\iota}} \symup{z}\mathcolor{gray}{0}}~, \label{eq:div-e-b-01-delta-conclusion} \\
	{\delta}'_{\mathcolor{gray}{z}} ~\textcolor{Maroon}{\text{项}}:&\hspace{1.0em}
	{\mathcal{K}}^{\;\!\mathcolor{gray}{t}}_{\;\!\textcolor{Maroon}{\text{b}} \symup{z} \symup{z}\mathcolor{gray}{0}} \hspace{-1.7em}&&=\hspace{0.2em} - \hspace{0.2em} \left( \mathcolor{gray}{\nabla^\iota} {\mathcal{L}}^{\;\!\mathcolor{gray}{t}}_{\;\!\textcolor{Maroon}{\text{b}} \mathcolor{gray}{\symup{\iota}} \symup{z} \symup{z} \mathcolor{gray}{0}} \hspace{-2.5em}\right. &&\hspace{0.8em}+ \left. \mathcolor{gray}{\nabla^t} {\mathcal{Q}}^{\;\!\mathcolor{gray}{t}}_{\;\!\textcolor{Maroon}{\text{b}} \symup{z} \symup{z} \mathcolor{gray}{0}} \right) \hspace{-1.6em}&&=-\hspace{0.2em} \mathcolor{gray}{\nabla^t} {\mathcal{Q}}^{\;\!\mathcolor{gray}{t}}_{\;\!\textcolor{Maroon}{\text{b}} \symup{z} \symup{z} \mathcolor{gray}{0}} - \mathcolor{gray}{\nabla^\iota} {\mathcal{L}}^{\;\!\mathcolor{gray}{t}}_{\;\!\textcolor{Maroon}{\text{b}} \mathcolor{gray}{\symup{\iota}} \symup{z} \symup{z} \mathcolor{gray}{0}}~, \label{eq:div-e-b-01-delta'-conclusion} \\
	{\delta}''_{\mathcolor{gray}{z}} ~\textcolor{Maroon}{\text{项}}:&\hspace{1.0em} {\mathcal{L}}^{\;\!\mathcolor{gray}{t}}_{\;\!\textcolor{Maroon}{\text{b}} \symup{z} \symup{z} \symup{z} \mathcolor{gray}{0}} \hspace{-1.7em}&&=\hspace{0.2em} - \hspace{0.2em} \left( \mathcolor{gray}{\nabla^\iota} \cdots \hspace{-2.5em}\right. &&\hspace{0.8em}+ \left. \mathcolor{gray}{\nabla^t} {\mathcal{O}}^{\;\!\mathcolor{gray}{t}}_{\;\!\textcolor{Maroon}{\text{b}} \symup{z} \symup{z} \symup{z} \mathcolor{gray}{0}} \right) \hspace{-1.6em}&&=-\hspace{0.2em} \mathcolor{gray}{\nabla^t} {\mathcal{O}}^{\;\!\mathcolor{gray}{t}}_{\;\!\textcolor{Maroon}{\text{b}} \symup{z} \symup{z} \symup{z} \mathcolor{gray}{0}} - \mathcolor{gray}{\nabla^\iota} \cdots~. \label{eq:div-e-b-01-delta''-conclusion}
\end{align}
\end{subequations}
经典的\textcolor{Maroon}{多极理论} \cite{raabMultipoleTheoryElectromagnetism2004} 给出 \bref{eq:j-b-01',eq:p-b-01'} 中的各体/表面电荷/流项:
\begin{subequations} \label{eq:multipole}
	\abovedisplayskip=7pt
	\belowdisplayskip=8pt
\begin{align}
	&{\mathcal{O}}^{\;\!\mathcolor{gray}{t}}_{\;\!\textcolor{Maroon}{\text{b}} \symup{\iota}\hat{1}\hat{2} \mathcolor{gray}{z}} \hspace{-1em}&&=\hspace{0.2em} +~ O^{\;\!\mathcolor{gray}{t}}_{\;\! \symup{\iota}\hat{1}\hat{2}\mathcolor{gray}{z}} &&\hspace{-1.1em}- \cdots~, &&\hspace{0.3em} {\mathcal{L}}^{\;\!\mathcolor{gray}{t}}_{\;\!\textcolor{Maroon}{\text{m}} \symup{\iota}\hat{1}\hat{3} \mathcolor{gray}{z}} \hspace{-1em}&&=\hspace{0.2em} +~ \epsilon^{\hphantom{\symup{\iota}\hat{1}}\hat{2}}_{\symup{\iota} \hat{1}} N^{\;\!\mathcolor{gray}{t}}_{\;\! \hat{2}\hat{3} \mathcolor{gray}{z}} &&\hspace{-1.1em}- \cdots~, \label{eq:Ob-Lm} \\
	&{\mathcal{Q}}^{\;\!\mathcolor{gray}{t}}_{\;\!\textcolor{Maroon}{\text{b}} \symup{\iota}\hat{1} \mathcolor{gray}{z}} \hspace{-1em}&&=\hspace{0.2em} -~ Q^{\;\!\mathcolor{gray}{t}}_{\;\! \symup{\iota}\hat{1}\mathcolor{gray}{z}} &&\hspace{-1.1em}+ \mathcolor{gray}{\nabla^{\hat{2}}} {\mathcal{O}}^{\;\!\mathcolor{gray}{t}}_{\;\!\textcolor{Maroon}{\text{b}} \symup{\iota} \hat{1} \mathcolor{gray}{\hat{2}} \mathcolor{gray}{z}}~,  &&\hspace{0.3em} {\mathcal{K}}^{\;\!\mathcolor{gray}{t}}_{\;\!\textcolor{Maroon}{\text{m}} \symup{\iota}\hat{1} \mathcolor{gray}{z}} \hspace{-1em}&&=\hspace{0.2em} -~ \epsilon^{\hphantom{\symup{\iota}\hat{1}}\hat{2}}_{\symup{\iota} \hat{1}} M^{\;\!\mathcolor{gray}{t}}_{\;\! \hat{2}\mathcolor{gray}{z}} &&\hspace{-1.1em}+ \mathcolor{gray}{\nabla^{\hat{3}}} {\mathcal{L}}^{\;\!\mathcolor{gray}{t}}_{\;\!\textcolor{Maroon}{\text{m}} \symup{\iota} \hat{1} \mathcolor{gray}{\hat{3}} \mathcolor{gray}{z}}~, \label{eq:Qb-Km} \\
	&{\mathcal{P}}^{\;\!\mathcolor{gray}{t}}_{\;\!\textcolor{Maroon}{\text{b}} \symup{\iota} \mathcolor{gray}{z}} \hspace{-1em}&&=\hspace{0.2em} \hphantom{+}~ P^{\;\!\mathcolor{gray}{t}}_{\;\! \symup{\iota}\mathcolor{gray}{z}} &&\hspace{-1.1em}+ \mathcolor{gray}{\nabla^{\hat{1}}} {\mathcal{Q}}^{\;\!\mathcolor{gray}{t}}_{\;\!\textcolor{Maroon}{\text{b}} \symup{\iota} \mathcolor{gray}{\hat{1}} \mathcolor{gray}{z}}~, &&\hspace{0.3em} {J}^{\;\!\mathcolor{gray}{t}}_{\;\!\textcolor{Maroon}{\text{m}} \symup{\iota} \mathcolor{gray}{z}} \hspace{-1em}&&=\hspace{0.2em} &&\hspace{-1.1em}- \mathcolor{gray}{\nabla^{\hat{1}}} {\mathcal{K}}^{\;\!\mathcolor{gray}{t}}_{\;\!\textcolor{Maroon}{\text{m}} \symup{\iota} \mathcolor{gray}{\hat{1}} \mathcolor{gray}{z}}~, \label{eq:Pb-Jm} \\
	&{\rho}^{\;\!\mathcolor{gray}{t}}_{\;\!\textcolor{Maroon}{\text{b}} \mathcolor{gray}{z}} \hspace{-1em}&&=\hspace{0.2em} &&\hspace{-1.1em}- \mathcolor{gray}{\nabla^\iota} {\mathcal{P}}^{\;\!\mathcolor{gray}{t}}_{\;\!\textcolor{Maroon}{\text{b}} \mathcolor{gray}{\symup{\iota}} \mathcolor{gray}{z}}~, &&\hspace{0.3em} \hspace{-1em}&& &&\hspace{-1.1em} \label{eq:pb} \\
	&{J}^{\;\!\mathcolor{gray}{t}}_{\;\!\textcolor{Maroon}{\text{e}} \symup{\iota} \mathcolor{gray}{z}} \hspace{-1em}&&=\hspace{0.2em} &&\hspace{-1.1em}+ \mathcolor{gray}{\nabla^t} {\mathcal{P}}^{\;\!\mathcolor{gray}{t}}_{\;\!\textcolor{Maroon}{\text{b}} \symup{\iota} \mathcolor{gray}{z}}~, &&\hspace{0.3em} {J}^{\;\!\mathcolor{gray}{t}}_{\;\!\textcolor{Maroon}{\text{b}} \symup{\iota} \mathcolor{gray}{z}} \hspace{-1em}&&=\hspace{0.2em} ~ \hphantom{- \epsilon^{\hphantom{\symup{\iota}\hat{1}}\hat{2}}_{\symup{\iota} \hat{1}}} {J}^{\;\!\mathcolor{gray}{t}}_{\;\!\textcolor{Maroon}{\text{e}} \symup{\iota} \mathcolor{gray}{z}} &&\hspace{-1.1em}+ {J}^{\;\!\mathcolor{gray}{t}}_{\;\!\textcolor{Maroon}{\text{m}} \symup{\iota} \mathcolor{gray}{z}}~, \label{eq:Je-Jb} \\
	&{\mathcal{K}}^{\;\!\mathcolor{gray}{t}}_{\;\!\textcolor{Maroon}{\text{e}} \symup{\iota}\hat{1} \mathcolor{gray}{z}} \hspace{-1em}&&=\hspace{0.2em} &&\hspace{-1.1em}- \mathcolor{gray}{\nabla^t} {\mathcal{Q}}^{\;\!\mathcolor{gray}{t}}_{\;\!\textcolor{Maroon}{\text{b}} \symup{\iota}\hat{1} \mathcolor{gray}{z}}~, &&\hspace{0.3em} {\mathcal{K}}^{\;\!\mathcolor{gray}{t}}_{\;\!\textcolor{Maroon}{\text{b}} \symup{\iota}\hat{1} \mathcolor{gray}{z}} \hspace{-1em}&&=\hspace{0.2em} ~ \hphantom{- \epsilon^{\hphantom{\symup{\iota}\hat{1}}\hat{2}}_{\symup{\iota} \hat{1}}} {\mathcal{K}}^{\;\!\mathcolor{gray}{t}}_{\;\!\textcolor{Maroon}{\text{e}} \symup{\iota}\hat{1} \mathcolor{gray}{z}} &&\hspace{-1.1em}+ {\mathcal{K}}^{\;\!\mathcolor{gray}{t}}_{\;\!\textcolor{Maroon}{\text{m}} \symup{\iota}\hat{1} \mathcolor{gray}{z}}~, \label{eq:Ke-Kb} \\
	&{\mathcal{L}}^{\;\!\mathcolor{gray}{t}}_{\;\!\textcolor{Maroon}{\text{e}} \symup{\iota}\hat{1}\hat{3} \mathcolor{gray}{z}} \hspace{-1em}&&=\hspace{0.2em} &&\hspace{-1.1em}- \mathcolor{gray}{\nabla^t} {\mathcal{O}}^{\;\!\mathcolor{gray}{t}}_{\;\!\textcolor{Maroon}{\text{b}} \symup{\iota}\hat{1}\hat{3} \mathcolor{gray}{z}}~, &&\hspace{0.3em} {\mathcal{L}}^{\;\!\mathcolor{gray}{t}}_{\;\!\textcolor{Maroon}{\text{b}} \symup{\iota}\hat{1}\hat{3} \mathcolor{gray}{z}} \hspace{-1em}&&=\hspace{0.2em} ~ \hphantom{- \epsilon^{\hphantom{\symup{\iota}\hat{1}}\hat{2}}_{\symup{\iota} \hat{1}}} {\mathcal{L}}^{\;\!\mathcolor{gray}{t}}_{\;\!\textcolor{Maroon}{\text{e}} \symup{\iota}\hat{1}\hat{3} \mathcolor{gray}{z}} &&\hspace{-1.1em}+ {\mathcal{L}}^{\;\!\mathcolor{gray}{t}}_{\;\!\textcolor{Maroon}{\text{m}} \symup{\iota}\hat{1}\hat{3} \mathcolor{gray}{z}}~, \label{eq:Le-Lb}
\end{align}
\end{subequations}
可以验证,上述经典\textcolor{Maroon}{多极理论} \bref{eq:multipole} 并不能自动将每一阶/级\textcolor{Plum}{奇异}项消除,即无法自动满足 \bref{eq:div-e-b-01-delta-conclusions} 中的 3 个\textcolor{Plum}{奇异}层次 ${\delta}_{\mathcolor{gray}{z}},{\delta}'_{\mathcolor{gray}{z}},{\delta}''_{\mathcolor{gray}{z}}$ 对应的方程。

为验证这 2 种理论的一致性,将 \bref{eq:Le-Lb,eq:Ke-Kb,eq:Je-Jb} 展开
\begin{subequations} \label{eq:JKL}
	\abovedisplayskip=7pt
	\belowdisplayskip=8pt
\begin{align}
	{J}^{\;\!\mathcolor{gray}{t}}_{\;\!\textcolor{Maroon}{\text{b}} \symup{\iota} \mathcolor{gray}{z}} &=+\hspace{0.2em} \mathcolor{gray}{\nabla^t} {\mathcal{P}}^{\;\!\mathcolor{gray}{t}}_{\;\!\textcolor{Maroon}{\text{b}} \symup{\iota} \mathcolor{gray}{z}} - \mathcolor{gray}{\nabla^{\hat{1}}} {\mathcal{K}}^{\;\!\mathcolor{gray}{t}}_{\;\!\textcolor{Maroon}{\text{m}} \symup{\iota} \mathcolor{gray}{\hat{1}} \mathcolor{gray}{z}}~, \label{eq:Jb} \\
	{\mathcal{K}}^{\;\!\mathcolor{gray}{t}}_{\;\!\textcolor{Maroon}{\text{b}} \symup{\iota}\hat{1} \mathcolor{gray}{z}} &= -\hspace{0.2em} \mathcolor{gray}{\nabla^t} {\mathcal{Q}}^{\;\!\mathcolor{gray}{t}}_{\;\!\textcolor{Maroon}{\text{b}} \symup{\iota}\hat{1} \mathcolor{gray}{z}} + {\mathcal{K}}^{\;\!\mathcolor{gray}{t}}_{\;\!\textcolor{Maroon}{\text{m}} \symup{\iota}\hat{1} \mathcolor{gray}{z}}~, \label{eq:Kb} \\
	{\mathcal{L}}^{\;\!\mathcolor{gray}{t}}_{\;\!\textcolor{Maroon}{\text{b}} \symup{\iota}\hat{1}\hat{3} \mathcolor{gray}{z}} &= -\hspace{0.2em} \mathcolor{gray}{\nabla^t} {\mathcal{O}}^{\;\!\mathcolor{gray}{t}}_{\;\!\textcolor{Maroon}{\text{b}} \symup{\iota}\hat{1}\hat{3} \mathcolor{gray}{z}} + \epsilon^{\hphantom{\symup{\iota}\hat{1}}\hat{2}}_{\symup{\iota} \hat{1}} N^{\;\!\mathcolor{gray}{t}}_{\;\! \hat{2}\hat{3} \mathcolor{gray}{z}}~, \label{eq:Lb}
\end{align}
\end{subequations}
上述 {\one} 经典\textcolor{Maroon}{多极理论} \bref{eq:Lb} 的预言:${\mathcal{L}}^{\;\!\mathcolor{gray}{t}}_{\;\!\textcolor{Maroon}{\text{b}} \symup{z} \symup{z} \symup{z} \mathcolor{gray}{z}} = - \mathcolor{gray}{\nabla^t} {\mathcal{O}}^{\;\!\mathcolor{gray}{t}}_{\;\!\textcolor{Maroon}{\text{b}} \symup{z} \symup{z} \symup{z} \mathcolor{gray}{z}} + \epsilon^{\hphantom{\symup{z} \symup{z}}\hat{2}}_{\symup{z} \symup{z}} N^{\;\!\mathcolor{gray}{t}}_{\;\! \hat{2}\symup{z}\mathcolor{gray}{z}} = - \mathcolor{gray}{\nabla^t} {\mathcal{O}}^{\;\!\mathcolor{gray}{t}}_{\;\!\textcolor{Maroon}{\text{b}} \symup{z} \symup{z} \symup{z} \mathcolor{gray}{z}}$,恰好等于\textcolor{Plum}{奇异}\textcolor{Maroon}{边界条件} \bref{eq:div-e-b-01-delta''-conclusion} 所给出的 ${\mathcal{L}}^{\;\!\mathcolor{gray}{t}}_{\;\!\textcolor{Maroon}{\text{b}} \symup{z} \symup{z} \symup{z} \mathcolor{gray}{0}} = - \mathcolor{gray}{\nabla^t} {\mathcal{O}}^{\;\!\mathcolor{gray}{t}}_{\;\!\textcolor{Maroon}{\text{b}} \symup{z} \symup{z} \symup{z} \mathcolor{gray}{0}}$;{\two} 经典\textcolor{Maroon}{多极理论} \bref{eq:Kb} 的预言:${\mathcal{K}}^{\;\!\mathcolor{gray}{t}}_{\;\!\textcolor{Maroon}{\text{b}} \symup{z} \symup{z} \mathcolor{gray}{z}} = - \mathcolor{gray}{\nabla^t} {\mathcal{Q}}^{\;\!\mathcolor{gray}{t}}_{\;\!\textcolor{Maroon}{\text{b}} \symup{z} \symup{z} \mathcolor{gray}{z}} + {\mathcal{K}}^{\;\!\mathcolor{gray}{t}}_{\;\!\textcolor{Maroon}{\text{m}} \symup{z} \symup{z} \mathcolor{gray}{z}} = - \mathcolor{gray}{\nabla^t} {\mathcal{Q}}^{\;\!\mathcolor{gray}{t}}_{\;\!\textcolor{Maroon}{\text{b}} \symup{z} \symup{z} \mathcolor{gray}{z}}$,暂不等于\textcolor{Plum}{奇异}\textcolor{Maroon}{边界条件} \bref{eq:div-e-b-01-delta'-conclusion} 所导出的 ${\mathcal{K}}^{\;\!\mathcolor{gray}{t}}_{\;\!\textcolor{Maroon}{\text{b}} \symup{z} \symup{z} \mathcolor{gray}{0}} = - \left( \mathcolor{gray}{\nabla^t} {\mathcal{Q}}^{\;\!\mathcolor{gray}{t}}_{\;\!\textcolor{Maroon}{\text{b}} \symup{z} \symup{z} \mathcolor{gray}{0}} + \mathcolor{gray}{\nabla^\iota} {\mathcal{L}}^{\;\!\mathcolor{gray}{t}}_{\;\!\textcolor{Maroon}{\text{b}} \mathcolor{gray}{\symup{\iota}} \symup{z} \symup{z} \mathcolor{gray}{0}} \right)$;{\three} 经典\textcolor{Maroon}{多极理论} \bref{eq:Jb} 的预言:${J}^{\;\!\mathcolor{gray}{t}}_{\;\!\textcolor{Maroon}{\text{b}} \symup{z} \mathcolor{gray}{z}} = \mathcolor{gray}{\nabla^t} {\mathcal{P}}^{\;\!\mathcolor{gray}{t}}_{\;\!\textcolor{Maroon}{\text{b}} \symup{z} \mathcolor{gray}{z}} - \mathcolor{gray}{\nabla^{\hat{1}}} {\mathcal{K}}^{\;\!\mathcolor{gray}{t}}_{\;\!\textcolor{Maroon}{\text{m}} \symup{z} \mathcolor{gray}{\hat{1}} \mathcolor{gray}{z}} = \mathcolor{gray}{\nabla^t} {\mathcal{P}}^{\;\!\mathcolor{gray}{t}}_{\;\!\textcolor{Maroon}{\text{b}} \symup{z} \mathcolor{gray}{z}} + \mathcolor{gray}{\nabla^\iota} {\mathcal{K}}^{\;\!\mathcolor{gray}{t}}_{\;\!\textcolor{Maroon}{\text{m}} \mathcolor{gray}{\symup{\iota}} \symup{z} \mathcolor{gray}{z}}$,也暂不等于\textcolor{Plum}{奇异}\textcolor{Maroon}{边界条件} \bref{eq:div-e-b-01-delta-conclusion} 所导出的 ${J}^{\;\!\mathcolor{gray}{t}}_{\;\!\textcolor{Maroon}{\text{b}} \symup{z} \mathcolor{gray}{0}} = \mathcolor{gray}{\nabla^t} {\mathcal{P}}^{\;\!\mathcolor{gray}{t}}_{\;\!\textcolor{Maroon}{\text{b}} \symup{z} \mathcolor{gray}{0}} + \mathcolor{gray}{\nabla^\iota} {\mathcal{K}}^{\;\!\mathcolor{gray}{t}}_{\;\!\textcolor{Maroon}{\text{b}} \mathcolor{gray}{\symup{\iota}} \symup{z}\mathcolor{gray}{0}}$。

可见,\textcolor{Plum}{奇异}\textcolor{Maroon}{边界条件}的预测结果,似乎总比经典\textcolor{Maroon}{多极理论}的预测结果,多出至少一项极化电流表面/体密度项:比如在 ${\delta}_{\mathcolor{gray}{z}}$ 层次即 ${J}^{\;\!\mathcolor{gray}{t}}_{\;\!\textcolor{Maroon}{\text{b}} \symup{z} \mathcolor{gray}{z}}$ 中,会多出一项 $\mathcolor{gray}{\nabla^\iota} {\mathcal{K}}^{\;\!\mathcolor{gray}{t}}_{\;\!\textcolor{Maroon}{\text{e}} \mathcolor{gray}{\symup{\iota}} \symup{z}\mathcolor{gray}{0}} = - \mathcolor{gray}{\nabla^\iota} \mathcolor{gray}{\nabla^t} {\mathcal{Q}}^{\;\!\mathcolor{gray}{t}}_{\;\!\textcolor{Maroon}{\text{b}} \mathcolor{gray}{\symup{\iota}} \symup{z} \mathcolor{gray}{z}}$ 并与 $\mathcolor{gray}{\nabla^t} {\mathcal{P}}^{\;\!\mathcolor{gray}{t}}_{\;\!\textcolor{Maroon}{\text{b}} \symup{z} \mathcolor{gray}{z}}$ 中的对应项 $\mathcolor{gray}{\nabla^t} \mathcolor{gray}{\nabla^{\hat{2}}} {\mathcal{Q}}^{\;\!\mathcolor{gray}{t}}_{\;\!\textcolor{Maroon}{\text{b}} \symup{z} \mathcolor{gray}{\hat{2}} \mathcolor{gray}{z}}$ 约掉(由于\textcolor{Plum}{多极}矩的\textcolor{Plum}{置换对称性}\cite{raabMultipoleTheoryElectromagnetism2004});又比如会在 ${\delta}'_{\mathcolor{gray}{z}}$ 层次即 ${K}^{\;\!\mathcolor{gray}{t}}_{\;\!\textcolor{Maroon}{\text{b}} \symup{z} \symup{z} \mathcolor{gray}{z}}$ 中,会多出一项 $\mathcolor{gray}{\nabla^\iota} {\mathcal{L}}^{\;\!\mathcolor{gray}{t}}_{\;\!\textcolor{Maroon}{\text{e}} \mathcolor{gray}{\symup{\iota}} \symup{z} \symup{z} \mathcolor{gray}{0}} = - \mathcolor{gray}{\nabla^\iota} \mathcolor{gray}{\nabla^t} {\mathcal{O}}^{\;\!\mathcolor{gray}{t}}_{\;\!\textcolor{Maroon}{\text{b}} \mathcolor{gray}{\symup{\iota}} \symup{z} \symup{z} \mathcolor{gray}{0}}$ 并与 $\mathcolor{gray}{\nabla^t} {\mathcal{Q}}^{\;\!\mathcolor{gray}{t}}_{\;\!\textcolor{Maroon}{\text{b}} \symup{z} \symup{z} \mathcolor{gray}{z}}$ 中的对应项 $\mathcolor{gray}{\nabla^t} \mathcolor{gray}{\nabla^{\hat{2}}} {\mathcal{O}}^{\;\!\mathcolor{gray}{t}}_{\;\!\textcolor{Maroon}{\text{b}} \symup{z} \symup{z} \mathcolor{gray}{\hat{2}} \mathcolor{gray}{0}}$ 约掉(同样由于\textcolor{Plum}{多极}矩的\textcolor{Plum}{置换对称性})。此外,在 ${\delta}'_{\mathcolor{gray}{z}}$ 层次即 ${K}^{\;\!\mathcolor{gray}{t}}_{\;\!\textcolor{Maroon}{\text{b}} \symup{z} \symup{z} \mathcolor{gray}{z}}$ 中,还会多出一项 $\mathcolor{gray}{\nabla^\iota} {\mathcal{L}}^{\;\!\mathcolor{gray}{t}}_{\;\!\textcolor{Maroon}{\text{m}} \mathcolor{gray}{\symup{\iota}} \symup{z} \symup{z} \mathcolor{gray}{z}} = \epsilon^{\hphantom{\symup{\iota}\hat{1}}\hat{2}}_{\mathcolor{gray}{\symup{\iota}} \symup{z}} \mathcolor{gray}{\nabla^\iota} N^{\;\!\mathcolor{gray}{t}}_{\;\! \hat{2}\symup{z}\mathcolor{gray}{z}}$。

因此,不禁会像爱因斯坦一样发问:哪个理论错了? ---  答案是\textcolor{Plum}{奇异}\textcolor{Maroon}{边界条件}正确,经典\textcolor{Maroon}{多极理论}部分错误  ---  后者的磁化表面电流 ${\mathcal{K}}^{\;\!\mathcolor{gray}{t}}_{\;\!\textcolor{Maroon}{\text{m}} \symup{\iota}\hat{1} \mathcolor{gray}{z}},{\mathcal{L}}^{\;\!\mathcolor{gray}{t}}_{\;\!\textcolor{Maroon}{\text{m}} \symup{\iota}\hat{1}\hat{2} \mathcolor{gray}{z}}$,以及极化表面电荷/流项 ${\mathcal{P}}^{\;\!\mathcolor{gray}{t}}_{\;\!\textcolor{Maroon}{\text{b}} \symup{\iota} \mathcolor{gray}{z}},{\mathcal{Q}}^{\;\!\mathcolor{gray}{t}}_{\;\!\textcolor{Maroon}{\text{b}} \symup{\iota}\hat{1} \mathcolor{gray}{z}},{\mathcal{O}}^{\;\!\mathcolor{gray}{t}}_{\;\!\textcolor{Maroon}{\text{b}} \symup{\iota}\hat{1}\hat{2} \mathcolor{gray}{z}};{\mathcal{K}}^{\;\!\mathcolor{gray}{t}}_{\;\!\textcolor{Maroon}{\text{e}} \symup{\iota}\hat{1} \mathcolor{gray}{z}},{\mathcal{L}}^{\;\!\mathcolor{gray}{t}}_{\;\!\textcolor{Maroon}{\text{e}} \symup{\iota}\hat{1}\hat{2} \mathcolor{gray}{z}}$ 错了\Footnote{经典\textcolor{Maroon}{多极理论}给出的表面项错了,以及这样修改它们的原因,见 \textcolor{Maroon}{O.L. DE LANGE} 等人的著作\cite{grahamMultipoleSolutionMacroscopic2000,raabMultipoleTheoryElectromagnetism2004,delangeElectromagneticBoundaryConditions2013}。}(但极/磁化体电荷/流项 ${\rho}^{\;\!\mathcolor{gray}{t}}_{\;\!\textcolor{Maroon}{\text{b}} \mathcolor{gray}{z}};{J}^{\;\!\mathcolor{gray}{t}}_{\;\!\textcolor{Maroon}{\text{e}} \symup{\iota} \mathcolor{gray}{z}},{J}^{\;\!\mathcolor{gray}{t}}_{\;\!\textcolor{Maroon}{\text{m}} \symup{\iota} \mathcolor{gray}{z}}$ 均正确)。接着,在假设\textcolor{Plum}{奇异}\textcolor{Maroon}{边界条件}是正确的条件下,修正\textcolor{Maroon}{多极理论}。为满足\textcolor{Plum}{奇异}\textcolor{Maroon}{边界条件} \bref{eq:div-e-b-01-delta-conclusions},\textcolor{Maroon}{多极理论}必须满足以下层次关系:
\begin{subequations} \label{eq:multipole-JKL}
\begin{align}
	&{J}^{\;\!\mathcolor{gray}{t}}_{\;\!\textcolor{Maroon}{\text{e}} \symup{z} \mathcolor{gray}{z}} \hspace{-2.5em}&&=\hspace{0.2em} &&\hspace{-2.5em}+ \mathcolor{gray}{\nabla^t} {\mathcal{P}}^{\;\!\mathcolor{gray}{t}}_{\;\! \symup{z} \mathcolor{gray}{z}} + \mathcolor{gray}{\nabla^{\hat{1}}} {\mathcal{K}}^{\;\!\mathcolor{gray}{t}}_{\;\!\textcolor{Maroon}{\text{E}} \mathcolor{gray}{\hat{1}} \symup{z}\mathcolor{gray}{z}}~, &&\hspace{0.3em} {J}^{\;\!\mathcolor{gray}{t}}_{\;\!\textcolor{Maroon}{\text{m}} \symup{z} \mathcolor{gray}{z}} \hspace{-2.5em}&&=\hspace{0.2em} +~ \mathcolor{gray}{\nabla^{\hat{1}}} {\mathcal{K}}^{\;\!\mathcolor{gray}{t}}_{\;\!\textcolor{Maroon}{\text{M}} \mathcolor{gray}{\hat{1}} \symup{z} \mathcolor{gray}{z}}~, \label{eq:Je-JM} \\
	&{\mathcal{K}}^{\;\!\mathcolor{gray}{t}}_{\;\!\textcolor{Maroon}{\text{E}} \symup{z}\symup{z} \mathcolor{gray}{z}} \hspace{-2.5em}&&=\hspace{0.2em} &&\hspace{-2.5em}- \mathcolor{gray}{\nabla^t} {\mathcal{Q}}^{\;\!\mathcolor{gray}{t}}_{\;\! \symup{z}\symup{z} \mathcolor{gray}{z}} - \mathcolor{gray}{\nabla^{\hat{3}}} {\mathcal{L}}^{\;\!\mathcolor{gray}{t}}_{\;\!\textcolor{Maroon}{\text{E}} \mathcolor{gray}{\hat{3}} \symup{z}\symup{z}\mathcolor{gray}{z}}~, &&\hspace{0.3em} {\mathcal{K}}^{\;\!\mathcolor{gray}{t}}_{\;\!\textcolor{Maroon}{\text{M}} \symup{z}\symup{z} \mathcolor{gray}{z}} \hspace{-2.5em}&&=\hspace{0.2em} -~ \mathcolor{gray}{\nabla^{\hat{3}}} {\mathcal{L}}^{\;\!\mathcolor{gray}{t}}_{\;\!\textcolor{Maroon}{\text{M}} \mathcolor{gray}{\hat{3}} \symup{z} \symup{z} \mathcolor{gray}{z}}~, \label{eq:Ke-KM} \\
	&{\mathcal{L}}^{\;\!\mathcolor{gray}{t}}_{\;\!\textcolor{Maroon}{\text{E}} \symup{z}\symup{z}\symup{z} \mathcolor{gray}{z}} \hspace{-2.5em}&&=\hspace{0.2em} &&\hspace{-2.5em}- \mathcolor{gray}{\nabla^t} {\mathcal{O}}^{\;\!\mathcolor{gray}{t}}_{\;\! \symup{z}\symup{z}\symup{z} \mathcolor{gray}{z}} - \mathcolor{gray}{\nabla^m} \cdots~, &&\hspace{0.3em} {\mathcal{L}}^{\;\!\mathcolor{gray}{t}}_{\;\!\textcolor{Maroon}{\text{M}} \symup{z}\symup{z}\symup{z} \mathcolor{gray}{z}} \hspace{-2.5em}&&=\hspace{0.2em} -~ \mathcolor{gray}{\nabla^m} \cdots~, \label{eq:Le-LM}
\end{align}
\end{subequations}
其中,角标 $\textcolor{Maroon}{\text{E}},\textcolor{Maroon}{\text{M}}$ 是更新了的 $\textcolor{Maroon}{\text{e}},\textcolor{Maroon}{\text{m}}$。无角标 $\textcolor{Maroon}{\text{b}}$ 的 $\mathcal{P},\mathcal{Q},\mathcal{O}$ 分别是更新了的 $\mathcal{P}_{\;\!\textcolor{Maroon}{\text{b}}},\mathcal{Q}_{\;\!\textcolor{Maroon}{\text{b}}},\mathcal{O}_{\;\!\textcolor{Maroon}{\text{b}}}$。所有的这些修正量,重新适用于 \bref{eq:e-b-01',eq:div-e-b-01-deltas,eq:div-e-b-01-delta-conclusions}。

为保持极化体电荷/流公式 ${\rho}^{\;\!\mathcolor{gray}{t}}_{\;\!\textcolor{Maroon}{\text{b}} \mathcolor{gray}{z}};{J}^{\;\!\mathcolor{gray}{t}}_{\;\!\textcolor{Maroon}{\text{b}} \symup{\iota} \mathcolor{gray}{z}},{J}^{\;\!\mathcolor{gray}{t}}_{\;\!\textcolor{Maroon}{\text{e}} \symup{\iota} \mathcolor{gray}{z}},{J}^{\;\!\mathcolor{gray}{t}}_{\;\!\textcolor{Maroon}{\text{m}} \symup{\iota} \mathcolor{gray}{z}}$ 不变的同时,满足\textcolor{Maroon}{电荷守恒} \bref{eq:div-e-b},经典\textcolor{Maroon}{多极理论} \bref{eq:multipole},需要修正并进化为现代\textcolor{Plum}{奇异}版本:
\begin{subequations} \label{eq:multipole-modify}
	\abovedisplayskip=0pt
\begin{align}
	&{\mathcal{O}}^{\;\!\mathcolor{gray}{t}}_{\;\! \symup{\iota}\hat{1}\hat{2} \mathcolor{gray}{z}} \hspace{-2.0em}&&=\hspace{0.2em} {\mathcal{O}}^{\;\!\mathcolor{gray}{t}}_{\;\!\textcolor{Maroon}{\text{b}} \symup{\iota}\hat{1}\hat{2} \mathcolor{gray}{z}} &&\hspace{-2.0em}+ \cdots~, &&\hspace{0.3em} {\mathcal{L}}^{\;\!\mathcolor{gray}{t}}_{\;\!\textcolor{Maroon}{\text{M}} \symup{\iota}\hat{1}\hat{3} \mathcolor{gray}{z}} \hspace{-2.0em}&&=\hspace{0.2em} {\mathcal{L}}^{\;\!\mathcolor{gray}{t}}_{\;\!\textcolor{Maroon}{\text{m}} \symup{\iota}\hat{1}\hat{3} \mathcolor{gray}{z}} &&\hspace{-2.0em}- \cdots~, \label{eq:Ob-LM} \\
	&{\mathcal{Q}}^{\;\!\mathcolor{gray}{t}}_{\;\! \symup{\iota}\hat{1} \mathcolor{gray}{z}} \hspace{-2.0em}&&=\hspace{0.2em} {\mathcal{Q}}^{\;\!\mathcolor{gray}{t}}_{\;\!\textcolor{Maroon}{\text{b}} \symup{\iota}\hat{1} \mathcolor{gray}{z}} &&\hspace{-2.0em}+ \mathcolor{gray}{\nabla^{\hat{2}}} {\mathcal{O}}^{\;\!\mathcolor{gray}{t}}_{\;\!\textcolor{Maroon}{\text{b}} \symup{\iota} \hat{1} \mathcolor{gray}{\hat{2}} \mathcolor{gray}{z}}~,  &&\hspace{0.3em} {\mathcal{K}}^{\;\!\mathcolor{gray}{t}}_{\;\!\textcolor{Maroon}{\text{M}} \symup{\iota}\hat{1} \mathcolor{gray}{z}} \hspace{-2.0em}&&=\hspace{0.2em} {\mathcal{K}}^{\;\!\mathcolor{gray}{t}}_{\;\!\textcolor{Maroon}{\text{m}} \symup{\iota}\hat{1} \mathcolor{gray}{z}} &&\hspace{-2.0em}- \mathcolor{gray}{\nabla^{\hat{3}}} {\mathcal{L}}^{\;\!\mathcolor{gray}{t}}_{\;\!\textcolor{Maroon}{\text{m}} \mathcolor{gray}{\hat{3}} \symup{\iota}\hat{1} \mathcolor{gray}{z}}~, \label{eq:Qb-KM} \\
	&{\mathcal{P}}^{\;\!\mathcolor{gray}{t}}_{\;\! \symup{\iota} \mathcolor{gray}{z}} \hspace{-2.0em}&&=\hspace{0.2em} {\mathcal{P}}^{\;\!\mathcolor{gray}{t}}_{\;\!\textcolor{Maroon}{\text{b}} \symup{\iota} \mathcolor{gray}{z}} &&\hspace{-2.0em}+ \mathcolor{gray}{\nabla^{\hat{1}}} {\mathcal{Q}}^{\;\!\mathcolor{gray}{t}}_{\;\!\textcolor{Maroon}{\text{b}} \symup{\iota} \mathcolor{gray}{\hat{1}} \mathcolor{gray}{z}}~, &&\hspace{0.3em} {\mathcal{Q}}^{\;\!\mathcolor{gray}{t}}_{\;\!\textcolor{Maroon}{\text{b}} \symup{\iota}\hat{1} \mathcolor{gray}{z}} \hspace{-2.0em}&&=\hspace{0.2em} {\mathcal{Q}}^{\;\!\mathcolor{gray}{t}}_{\;\!\textcolor{Maroon}{\text{b}} \symup{\iota}\hat{1} \mathcolor{gray}{z}} &&\hspace{-2.0em}+ \mathcolor{gray}{\nabla^{\hat{2}}} {\mathcal{O}}^{\;\!\mathcolor{gray}{t}}_{\;\!\textcolor{Maroon}{\text{b}} \symup{\iota} \hat{1} \mathcolor{gray}{\hat{2}} \mathcolor{gray}{z}}~, \label{eq:Pb-QB} \\
	&{\mathcal{K}}^{\;\!\mathcolor{gray}{t}}_{\;\!\textcolor{Maroon}{\text{E}} \symup{\iota}\hat{1} \mathcolor{gray}{z}} \hspace{-2.0em}&&=\hspace{0.2em} &&\hspace{-2.0em}- \mathcolor{gray}{\nabla^t} {\mathcal{Q}}^{\;\!\mathcolor{gray}{t}}_{\;\!\textcolor{Maroon}{\text{b}} \symup{\iota}\hat{1} \mathcolor{gray}{z}}~, &&\hspace{0.3em} {\mathcal{K}}^{\;\!\mathcolor{gray}{t}}_{\;\! \symup{\iota}\hat{1} \mathcolor{gray}{z}} \hspace{-2.0em}&&=\hspace{0.2em} {\mathcal{K}}^{\;\!\mathcolor{gray}{t}}_{\;\!\textcolor{Maroon}{\text{E}} \symup{\iota}\hat{1} \mathcolor{gray}{z}} &&\hspace{-2.0em}+ {\mathcal{K}}^{\;\!\mathcolor{gray}{t}}_{\;\!\textcolor{Maroon}{\text{M}} \symup{\iota}\hat{1} \mathcolor{gray}{z}}~, \label{eq:KE-Kb} \\
	&{\mathcal{L}}^{\;\!\mathcolor{gray}{t}}_{\;\!\textcolor{Maroon}{\text{E}} \symup{\iota}\hat{1}\hat{3} \mathcolor{gray}{z}} \hspace{-2.0em}&&=\hspace{0.2em} &&\hspace{-2.0em}- \mathcolor{gray}{\nabla^t} {\mathcal{O}}^{\;\!\mathcolor{gray}{t}}_{\;\!\textcolor{Maroon}{\text{b}} \symup{\iota}\hat{1}\hat{3} \mathcolor{gray}{z}}~, &&\hspace{0.3em} {\mathcal{L}}^{\;\!\mathcolor{gray}{t}}_{\;\! \symup{\iota}\hat{1}\hat{3} \mathcolor{gray}{z}} \hspace{-2.0em}&&=\hspace{0.2em} {\mathcal{L}}^{\;\!\mathcolor{gray}{t}}_{\;\!\textcolor{Maroon}{\text{E}} \symup{\iota}\hat{1}\hat{3} \mathcolor{gray}{z}} &&\hspace{-2.0em}+ {\mathcal{L}}^{\;\!\mathcolor{gray}{t}}_{\;\!\textcolor{Maroon}{\text{M}} \symup{\iota}\hat{1}\hat{3} \mathcolor{gray}{z}}~, \label{eq:LE-Lb}
\end{align}
\end{subequations}
可以验证,上述修正后的\textcolor{Maroon}{多极理论} \bref{eq:multipole-modify} 满足\Footnote{提示,$\mathcolor{gray}{\nabla^\iota} \mathcolor{gray}{\nabla^{\hat{3}}} {\mathcal{L}}^{\;\!\mathcolor{gray}{t}}_{\;\!\textcolor{Maroon}{\text{m}} \mathcolor{gray}{\hat{3}} \mathcolor{gray}{\symup{\iota}} \hat{1} \mathcolor{gray}{z}} = \mathcolor{gray}{\nabla^{\hat{1}}} \mathcolor{gray}{\nabla^\iota} {\mathcal{L}}^{\;\!\mathcolor{gray}{t}}_{\;\!\textcolor{Maroon}{\text{m}} \mathcolor{gray}{\symup{\iota} \hat{1}} \hat{3} \mathcolor{gray}{z}} = 0$,但不是每项都为零,而是\textcolor{Plum}{爱因斯坦求和}后,两两抵消,一共三对。}\textcolor{Plum}{奇异}展开 \bref{eq:e-b-01'} 和 \textcolor{Maroon}{束缚电荷守恒} \bref{eq:div-e-b} 所导出的 \textcolor{Plum}{奇异}\textcolor{Maroon}{边界条件} \bref{eq:div-e-b-01-delta-conclusions} 即 \bref{eq:multipole-JKL}。此外,上述修正后的\textcolor{Maroon}{多极理论} \bref{eq:multipole-modify} 也可以从经典\textcolor{Maroon}{多极理论}自身(即回溯 \bref{eq:multipole} 的来源)出发\cite{raabMultipoleTheoryElectromagnetism2004,delangeElectromagneticBoundaryConditions2013} 以找到问题所在,并修补之以得到正果。可以验证,上述 \bref{eq:multipole-modify} 与文献\cite{delangeElectromagneticBoundaryConditions2013}的主要结果的绝大部分一致,除了与其公式 (6) 的最末一项符号相反。此问题应归结为文献\cite{delangeElectromagneticBoundaryConditions2013}的作者将那处的 $-$ 误打错成 $+$。

需要强调的是,\bref{eq:e-b-01'} 中的每一层次的\textcolor{Plum}{奇异}项,都含有低阶至高阶的\textcolor{Plum}{多极}矩(体密度/表面有限厚度的薄层内的面密度等)\Footnote{其中越高阶的\textcolor{Plum}{多极}矩,越需要更多的散度来降张量的阶,以保持在同一层次的\textcolor{Plum}{奇异}表面/体项内,各\textcolor{Plum}{多极}矩项的总张量阶次相同。};此外,各\textcolor{Plum}{奇异}项每多一个分量(张量每升一阶),就在单位上多乘一个 $\mathcolor{gray}{\symup{m}}$;配合相应的带有单位 $\mathcolor{gray}{\symup{m}^{-1}},\mathcolor{gray}{\symup{m}^{-2}},\cdots$ 的 $\delta_{\mathcolor{gray}{z}}$ 函数及其导数,\bref{eq:e-b-01'} 中各\textcolor{Plum}{奇异}项的(总)单位才保持不变。

总之,以上通过对比修正后的经典\textcolor{Maroon}{多极理论} \bref{eq:multipole-modify} 所导出的各阶表面电流密度 \bref{eq:multipole-JKL} 的 $\symup{z}$ 、 $\symup{z} \symup{z}$ 、 $\symup{z} \symup{z} \symup{z}$ 分量,与\textcolor{Plum}{奇异}\textcolor{Maroon}{边界条件} \bref{eq:e-b-01'} 配合 \bref{eq:div-e-b-01-deltas} 所要求的 \bref{eq:div-e-b-01-delta-conclusions},两个来源独立的假说却各自最终获得了相同的结果,因此同时验证了 \bref{eq:e-b-01'} 、 \bref{eq:multipole} 和 \bref{eq:multipole-modify} 的正确性。

此外,\bref{eq:e-b-01'} 的 另一点 显而易见的 正确性是,当 介质 \textcolor{Maroon}{1} 与介质 \textcolor{Maroon}{2} 的晶格结构和取向全同时(尽管它们处于不同的两个半空间 $\leftindex_{\textcolor{Maroon}{1}} {\mathbb{1}}_{\mathcolor{gray}{z}}$、$\leftindex_{\textcolor{Maroon}{2}} {\mathbb{1}}_{\mathcolor{gray}{z}}$),即当 \bref{eq:multipole} 中的各项(与 $\textcolor{Maroon}{\mathsfit{z}}$ 无关地;或者退一步只需要在 $\mathcolor{gray}{z \mathcolor{black}{=} 0}$ 处/左右\textcolor{Plum}{连续}即可)在介质 \textcolor{Maroon}{1} 与介质 \textcolor{Maroon}{2} 中完全一致时,\bref{eq:e-b-01'} 中的所有\textcolor{Plum}{奇异}/表面/接触项会自动约去并消失,甚至不需要 \bref{eq:div-e-b-01-delta-conclusions} 及\textcolor{Plum}{多极}矩理论(给出的显式 \bref{eq:multipole} 和 \bref{eq:multipole-modify})成立。 ---  这说明\textcolor{Plum}{奇异}项确实对应表面项,即只存在于体表面,而不是体内部。

如果\textcolor{Maroon}{自由电流}/\textcolor{Maroon}{荷} ${Q}^{\;\!\mathcolor{gray}{t}}_{\;\!\textcolor{Maroon}{\text{f}}\mathcolor{gray}{z}}$ \textcolor{Maroon}{也单独守恒}(但可能从束缚电源 ${Q}^{\;\!\mathcolor{gray}{t}}_{\;\!\textcolor{Maroon}{\text{b}}\mathcolor{gray}{z}}$ 转换过来),那么原则上 ${\rho}^{\;\!\mathcolor{gray}{t}}_{\;\!\textcolor{Maroon}{\text{f}}\mathcolor{gray}{z}},\bar{J}^{\;\!\mathcolor{gray}{t}}_{\;\!\textcolor{Maroon}{\text{f}}\mathcolor{gray}{z}}$ 也应需要以 \bref{eq:e-b-01'} 的形式展开,且各\textcolor{Plum}{奇异}层次间满足 \bref{eq:div-e-b-01-delta-conclusions},以遵循\textcolor{Maroon}{自由电荷守恒} \bref{eq:div-e-f};但对应的 \bref{eq:e-b-01'} 中的各阶自由体/表面电荷/流密度,不再有经典\textcolor{Maroon}{多极理论}的显式 \bref{eq:multipole}。因此,对于自由电流/荷 ${Q}^{\;\!\mathcolor{gray}{t}}_{\;\!\textcolor{Maroon}{\text{f}}\mathcolor{gray}{z}}$,原则上只需要将该节中的每条公式中各变量的下角标 \textcolor{Maroon}{\text{b}} 替换为 \textcolor{Maroon}{\text{f}} 即可。为清晰起见,给出 \bref{eq:e-b-01'} 的自由电荷版本:
\begin{subequations} \label{eq:e-f-01}
	\abovedisplayskip=5pt
\begin{align}
	J^{\;\!\mathcolor{gray}{t}}_{\;\!\textcolor{Maroon}{\text{f}} \symup{\iota}\mathcolor{gray}{z}} &= \leftindex_{\textcolor{Maroon}{\mathsfit{z}}} {\mathbb{1}}_{\mathcolor{gray}{z}} \leftindex^{\textcolor{Maroon}{\mathsfit{z}}} \;\! J^{\;\!\mathcolor{gray}{t}}_{\;\!\textcolor{Maroon}{\text{f}} \symup{\iota}\mathcolor{gray}{z}} + \delta_{\mathcolor{gray}{z}} \leftindex_{\textcolor{Maroon}{\mathsfit{z}}} \;\! \leftindex^{\textcolor{Maroon}{\mathsfit{z}}} \;\!
	{\alpha}^{\;\!\mathcolor{gray}{t}}_{\;\!\textcolor{Maroon}{\text{f}} \symup{\iota}\mathcolor{gray}{z}} ~, \label{eq:j-f-01} \\
	{\rho}^{\;\!\mathcolor{gray}{t}}_{\;\!\textcolor{Maroon}{\text{f}}\mathcolor{gray}{z}} &= \leftindex_{\textcolor{Maroon}{\mathsfit{z}}} {\mathbb{1}}_{\mathcolor{gray}{z}} \leftindex^{\textcolor{Maroon}{\mathsfit{z}}} {\rho}^{\;\!\mathcolor{gray}{t}}_{\;\!\textcolor{Maroon}{\text{f}}\mathcolor{gray}{z}} + \delta_{\mathcolor{gray}{z}} \leftindex_{\textcolor{Maroon}{\mathsfit{z}}} \;\! \leftindex^{\textcolor{Maroon}{\mathsfit{z}}} \;\! {\sigma}^{\;\!\mathcolor{gray}{t}}_{\;\!\textcolor{Maroon}{\text{f}} \mathcolor{gray}{z}} ~, \label{eq:p-f-01}
\end{align}
\end{subequations}
注意,由于(额外的)自由电荷不会在两个接触端面自动抵消(而是代数求和),且会因库伦力和在导带中“自由”而快速移动并最终平衡,因此上 \bref{eq:e-f-01} 与 \bref{eq:e-b-01'} 在 $\delta_{\mathcolor{gray}{z}}$ 函数的正负上略有不同。因此, \bref{eq:div-e-b-01-delta-conclusions} 的自由电荷版应写作
\abovedisplayskip=5pt
\begin{align} \label{eq:div-e-f-01-delta-conclusions}
	{\delta}_{\mathcolor{gray}{z}} ~\textcolor{Maroon}{\text{项}}:  \leftindex^{\textcolor{Maroon}{\mathsfit{z}}} \;\! J^{\;\!\mathcolor{gray}{t}}_{\;\!\textcolor{Maroon}{\text{f}} \symup{z} \mathcolor{gray}{0}} = \leftindex^{\textcolor{Maroon}{\mathsfit{z}}} \;\! n_{\mathcolor{gray}{z}} \left( \mathcolor{gray}{\nabla^t} {\sigma}^{\;\!\mathcolor{gray}{t}}_{\;\!\textcolor{Maroon}{\text{f}} \mathcolor{gray}{0}} + \mathcolor{gray}{\nabla^\iota} {\alpha}^{\;\!\mathcolor{gray}{t}}_{\;\!\textcolor{Maroon}{\text{f}} \mathcolor{gray}{\symup{\iota}} \mathcolor{gray}{0}} \right)~.
\end{align}
注意,该 \bref{eq:div-e-f-01-delta-conclusions} 没有更高阶的 ${\delta}_{\mathcolor{gray}{z}}$ 函数项:自由电流/荷 ${Q}^{\;\!\mathcolor{gray}{t}}_{\;\!\textcolor{Maroon}{\text{f}}\mathcolor{gray}{z}}$ 不需要 \bref{eq:div-e-b-01-delta'-conclusion} 及以上的\textcolor{Plum}{奇异}层次 \cite{dengTheoryElectrodynamicResponse2020,delangeElectromagneticBoundaryConditions2013}。此外,\bref{eq:div-e-f-01-delta-conclusions} 变得与材料表面法向的取向有关。由于无多重含义和字母冲突,表面自由电荷/流 ${\sigma}_{\;\!\textcolor{Maroon}{\text{f}}},{\alpha}_{\;\!\textcolor{Maroon}{\text{f}} \symup{\iota}}$ 均可省略下标 $\textcolor{Maroon}{\text{f}}$ 写作 ${\sigma},{\alpha}_{\;\! \symup{\iota}}$。

然而,为了减少符号数量,以及后续的形式统一,总可以将 $\leftindex^{\textcolor{Maroon}{\mathsfit{z}}} \;\!
{\alpha}^{\;\!\mathcolor{gray}{t}}_{\;\! \symup{\iota}\mathcolor{gray}{z}}, \leftindex^{\textcolor{Maroon}{\mathsfit{z}}} \;\! {\sigma}^{\;\!\mathcolor{gray}{t}}_{\;\! \mathcolor{gray}{z}}$ 重新定义为 $-\leftindex^{\textcolor{Maroon}{\mathsfit{z}}} \;\! n_{\mathcolor{gray}{z}} \leftindex^{\textcolor{Maroon}{\mathsfit{z}}} \;\!
{\alpha}^{\;\!\mathcolor{gray}{t}}_{\;\! \symup{\iota}\mathcolor{gray}{z}}, -\leftindex^{\textcolor{Maroon}{\mathsfit{z}}} \;\! n_{\mathcolor{gray}{z}} \leftindex^{\textcolor{Maroon}{\mathsfit{z}}} \;\! {\sigma}^{\;\!\mathcolor{gray}{t}}_{\;\! \mathcolor{gray}{z}}$,然后对 \bref{eq:e-f-01} 两边乘以 $1 = \left( - \leftindex^{\textcolor{Maroon}{\mathsfit{z}}} \;\! n_{\mathcolor{gray}{z}} \right) \left( - \leftindex^{\textcolor{Maroon}{\mathsfit{z}}} \;\! n_{\mathcolor{gray}{z}} \right)$,即得到新定义的 $\leftindex^{\textcolor{Maroon}{\mathsfit{z}}} \;\!
{\alpha}^{\;\!\mathcolor{gray}{t}}_{\;\! \symup{\iota}\mathcolor{gray}{z}}, \leftindex^{\textcolor{Maroon}{\mathsfit{z}}} \;\! {\sigma}^{\;\!\mathcolor{gray}{t}}_{\;\! \mathcolor{gray}{z}}$ 所满足的、与 \bref{eq:e-b-01'} 类似的形式
\begin{subequations} \label{eq:e-f-01'}
	\abovedisplayskip=5pt
\begin{align}
	J^{\;\!\mathcolor{gray}{t}}_{\;\!\textcolor{Maroon}{\text{f}} \symup{\iota}\mathcolor{gray}{z}} &= \leftindex_{\textcolor{Maroon}{\mathsfit{z}}} {\mathbb{1}}_{\mathcolor{gray}{z}} \leftindex^{\textcolor{Maroon}{\mathsfit{z}}} \;\! J^{\;\!\mathcolor{gray}{t}}_{\;\!\textcolor{Maroon}{\text{f}} \symup{\iota}\mathcolor{gray}{z}} + \leftindex_{\textcolor{Maroon}{\mathsfit{z}}} \;\! \delta_{\mathcolor{gray}{z}} \leftindex^{\textcolor{Maroon}{\mathsfit{z}}} \;\!
	{\alpha}^{\;\!\mathcolor{gray}{t}}_{\;\! \symup{\iota}\mathcolor{gray}{z}} ~, \label{eq:j-f-01'} \\
	{\rho}^{\;\!\mathcolor{gray}{t}}_{\;\!\textcolor{Maroon}{\text{f}} \mathcolor{gray}{z}} &= \leftindex_{\textcolor{Maroon}{\mathsfit{z}}} {\mathbb{1}}_{\mathcolor{gray}{z}} \leftindex^{\textcolor{Maroon}{\mathsfit{z}}} {\rho}^{\;\!\mathcolor{gray}{t}}_{\;\!\textcolor{Maroon}{\text{f}} \mathcolor{gray}{z}} + \leftindex_{\textcolor{Maroon}{\mathsfit{z}}} \;\! \delta_{\mathcolor{gray}{z}} \leftindex^{\textcolor{Maroon}{\mathsfit{z}}} \;\! {\sigma}^{\;\!\mathcolor{gray}{t}}_{\;\! \mathcolor{gray}{z}} ~, \label{eq:p-f-01'}
\end{align}
\end{subequations}
相应地,重新定义的 $\leftindex^{\textcolor{Maroon}{\mathsfit{z}}} \;\!
{\alpha}^{\;\!\mathcolor{gray}{t}}_{\;\! \symup{\iota}\mathcolor{gray}{z}}, \leftindex^{\textcolor{Maroon}{\mathsfit{z}}} \;\! {\sigma}^{\;\!\mathcolor{gray}{t}}_{\;\! \mathcolor{gray}{z}}$ 所满足的 \bref{eq:div-e-f-01-delta-conclusions} 变为
\abovedisplayskip=5pt
\begin{align} \label{eq:div-e-f-01-delta-conclusions'}
	{\delta}_{\mathcolor{gray}{z}} ~\textcolor{Maroon}{\text{项}}:  J^{\;\!\mathcolor{gray}{t}}_{\;\!\textcolor{Maroon}{\text{f}} \symup{z} \mathcolor{gray}{0}} = - \left( \mathcolor{gray}{\nabla^t} {\sigma}^{\;\!\mathcolor{gray}{t}}_{\;\! \mathcolor{gray}{0}} + \mathcolor{gray}{\nabla^\iota} {\alpha}^{\;\!\mathcolor{gray}{t}}_{\;\! \mathcolor{gray}{\symup{\iota}} \mathcolor{gray}{0}} \right)~.
\end{align}
\bref{eq:e-f-01'} 相比 \bref{eq:e-f-01},通过给 ${\delta}_{\mathcolor{gray}{z}}$ 函数引入 $\textcolor{Maroon}{\mathsfit{z}}$ 为代价,消除了 \bref{eq:div-e-f-01-delta-conclusions} 中的 $\textcolor{Maroon}{\mathsfit{z}}$ 以成为 \bref{eq:div-e-f-01-delta-conclusions'}。

事实上,对于“各阶\textcolor{Plum}{奇异}项之和为零”
\abovedisplayskip=5pt
\begin{align} \label{eq:deltasum=0}
	{\delta}_{\mathcolor{gray}{z}} f_{\mathcolor{gray}{z}} + {\delta}'_{\mathcolor{gray}{z}} g_{\mathcolor{gray}{z}} + 
	{\delta}''_{\mathcolor{gray}{z}} h_{\mathcolor{gray}{z}} = 0 ~,
\end{align}
有除“各阶\textcolor{Plum}{奇异}项分别为零”即 \bref{eq:div-e-b-01-deltas} 以外的结论存在。对 \bref{eq:deltasum=0}、$\mathcolor{gray}{z}$ 乘以之、$\mathcolor{gray}{z^2}$ 乘以之,分别沿 $\mathcolor{gray}{z}$ 做跨越过 $\mathcolor{gray}{z} = \mathcolor{gray}{0}$ 定积分,使用分部积分法\cite{delangeElectromagneticBoundaryConditions2013},有
\begin{subequations} \label{eq:Intdeltasum=0}
	\abovedisplayskip=6pt
\begin{align}
	f_{\mathcolor{gray}{0}} - g'_{\mathcolor{gray}{0}} + h''_{\mathcolor{gray}{0}} &= 0~, \hspace{1.2em} \Longrightarrow \hspace{1em} f_{\mathcolor{gray}{0}} = g'_{\mathcolor{gray}{0}} = h''_{\mathcolor{gray}{0}} = 0~, \label{eq:intdeltasum=0} \\
	- g_{\mathcolor{gray}{0}} + h'_{\mathcolor{gray}{0}} &= 0~, \hspace{1.2em} \Longrightarrow \hspace{1em} \hphantom{f_{\mathcolor{gray}{0}} =}~\;\! g_{\mathcolor{gray}{0}} = h'_{\mathcolor{gray}{0}} = 0~, \label{eq:intzdeltasum=0} \\
	h_{\mathcolor{gray}{0}} &= 0~, \hspace{1.2em} \Longrightarrow \hspace{1em} \hphantom{f_{\mathcolor{gray}{0}} = g_{\mathcolor{gray}{0}} =}~\;\! h_{\mathcolor{gray}{0}} = 0~, \label{eq:intzzdeltasum=0}
\end{align}
\end{subequations}
此外,起源互相独立的三个函数 $f_{\mathcolor{gray}{z}}$、$g_{\mathcolor{gray}{z}}$、$h_{\mathcolor{gray}{z}}$ 对 $\mathcolor{gray}{x,y}$ 的各阶导数 $\mathcolor{gray}{\nabla_x^n},\mathcolor{gray}{\nabla_y^n}$ 也满足 \bref{eq:deltasum=0},因此它们也满足上述三个 \bref{eq:Intdeltasum=0}。上述所有结论在被验证是否适用于束缚电源 ${Q}^{\;\!\mathcolor{gray}{t}}_{\;\!\textcolor{Maroon}{\text{b}}\mathcolor{gray}{z}}$ 时,都可以用 \bref{eq:multipole-modify} 朝着 \bref{eq:JKL} 的进一步展开来证明。

\vspace*{-4.5em}

\marginLeft[-1.3em]{ssec:EB-boundary}\subsection{$\bar{E},\bar{B}$ 表面场、$\bar{E},\bar{B}$ 边界条件}\label{ssec:EB-boundary}

当 \bref{ssec:PMQN} 的束缚源 \bref{eq:p-b,eq:j-b}、自由电源 \bref{eq:div-e-f},被上面 \bref{ssec:step-delta} 升级为 \bref{eq:e-b-01',eq:e-f-01'} 时,\bref{ssec:EBpJ} 中的 \textcolor{Maroon}{Maxwell-Lorentz-Heaviside} \bref{eq:curl-E,eq:div-B,eq:div-E,eq:curl-B} 升级为
\begin{subequations} \label{eq:curl-EB}
%	\abovedisplayskip=0pt
%	\belowdisplayskip=0pt
	\small
\begin{align}
	\epsilon^{\hphantom{\symup{\iota}\hat{1}}\hat{2}}_{\symup{\iota}\mathcolor{gray}{\hat{1}}} &\mathcolor{gray}{\nabla^{\hat{1}}} E^{\;\!\mathcolor{gray}{t}}_{\;\! \hat{2}\mathcolor{gray}{z}} + \mathcolor{gray}{\nabla^t} B^{\;\!\mathcolor{gray}{t}}_{\;\! \symup{\iota}\mathcolor{gray}{z}} = 0~, \label{eq:curl-E'} \\
	{\symup{c}}^2 \epsilon^{\hphantom{\symup{\iota}\hat{1}}\hat{2}}_{\symup{\iota}\mathcolor{gray}{\hat{1}}} &\mathcolor{gray}{\nabla^{\hat{1}}} B^{\;\!\mathcolor{gray}{t}}_{\;\! \hat{2}\mathcolor{gray}{z}} - \mathcolor{gray}{\nabla^t} E^{\;\!\mathcolor{gray}{t}}_{\;\! \symup{\iota}\mathcolor{gray}{z}} = \left[ \leftindex_{\textcolor{Maroon}{\mathsfit{z}}} {\mathbb{1}}_{\mathcolor{gray}{z}} \left( \leftindex^{\textcolor{Maroon}{\mathsfit{z}}} \;\! J^{\;\!\mathcolor{gray}{t}}_{\;\!\textcolor{Maroon}{\text{b}} \symup{\iota}\mathcolor{gray}{z}} + \leftindex^{\textcolor{Maroon}{\mathsfit{z}}} \;\! J^{\;\!\mathcolor{gray}{t}}_{\;\!\textcolor{Maroon}{\text{f}} \symup{\iota}\mathcolor{gray}{z}} \right) - \leftindex_{\textcolor{Maroon}{\mathsfit{z}}} \;\! \delta_{\mathcolor{gray}{z}} \left( \leftindex^{\textcolor{Maroon}{\mathsfit{z}}}
	{\mathcal{K}}^{\;\!\mathcolor{gray}{t}}_{\;\! \symup{\iota}\symup{z}\mathcolor{gray}{z}} - \leftindex^{\textcolor{Maroon}{\mathsfit{z}}} \;\!
	{\alpha}^{\;\!\mathcolor{gray}{t}}_{\;\! \symup{\iota}\mathcolor{gray}{z}} \right) - \leftindex_{\textcolor{Maroon}{\mathsfit{z}}} \;\! \delta'_{\mathcolor{gray}{z}} \leftindex^{\textcolor{Maroon}{\mathsfit{z}}} \;\! {\mathcal{L}}^{\;\!\mathcolor{gray}{t}}_{\;\! \symup{\iota}\symup{z} \symup{z} \mathcolor{gray}{z}} \right] \big/ {\symup{\varepsilon}}_0 ~, \label{eq:curl-B'} \\
	{\symup{\varepsilon}}_0 &\mathcolor{gray}{\nabla^\iota} E^{\;\!\mathcolor{gray}{t}}_{\;\! \mathcolor{gray}{\symup{\iota}} \mathcolor{gray}{z}} =  \leftindex_{\textcolor{Maroon}{\mathsfit{z}}} {\mathbb{1}}_{\mathcolor{gray}{z}} \left( \leftindex^{\textcolor{Maroon}{\mathsfit{z}}}  {\rho}^{\;\!\mathcolor{gray}{t}}_{\;\!\textcolor{Maroon}{\text{b}}\mathcolor{gray}{z}} + \leftindex^{\textcolor{Maroon}{\mathsfit{z}}} {\rho}^{\;\!\mathcolor{gray}{t}}_{\;\!\textcolor{Maroon}{\text{f}}\mathcolor{gray}{z}} \right) - \leftindex_{\textcolor{Maroon}{\mathsfit{z}}} \;\! \delta_{\mathcolor{gray}{z}} \left( \leftindex^{\textcolor{Maroon}{\mathsfit{z}}} \;\! {\mathcal{P}}^{\;\!\mathcolor{gray}{t}}_{\;\! \symup{z} \mathcolor{gray}{z}} - \leftindex^{\textcolor{Maroon}{\mathsfit{z}}} \;\! {\sigma}^{\;\!\mathcolor{gray}{t}}_{\;\! \mathcolor{gray}{z}} \right) - \leftindex_{\textcolor{Maroon}{\mathsfit{z}}} \;\! \delta'_{\mathcolor{gray}{z}} \leftindex^{\textcolor{Maroon}{\mathsfit{z}}} \;\! {\mathcal{Q}}^{\;\!\mathcolor{gray}{t}}_{\;\! \symup{z} \symup{z} \mathcolor{gray}{z}} - \leftindex_{\textcolor{Maroon}{\mathsfit{z}}} \;\! \delta''_{\mathcolor{gray}{z}} \leftindex^{\textcolor{Maroon}{\mathsfit{z}}} \;\! {\mathcal{O}}^{\;\!\mathcolor{gray}{t}}_{\;\! \symup{z} \symup{z} \symup{z} \mathcolor{gray}{z}}~, \label{eq:div-E'} \\
	&\mathcolor{gray}{\nabla^\iota} B^{\;\!\mathcolor{gray}{t}}_{\;\! \mathcolor{gray}{\symup{\iota}} \mathcolor{gray}{z}} = 0~. \label{eq:div-B'}
\end{align}
\end{subequations}
上述 \bref{eq:curl-B',eq:div-E'} 要想成立,意味着\textcolor{NavyBlue}{基本场} $\bar{E}^{\;\!\mathcolor{gray}{t}}_{\;\!\mathcolor{gray}{z}}, \bar{B}^{\;\!\mathcolor{gray}{t}}_{\;\!\mathcolor{gray}{z}}$ 也需要被展开为与(束缚)\textcolor{Plum}{奇异}源 \bref{eq:e-b-01'} 类似的\Footnote{不展开为\textcolor{Plum}{奇异}自由源 \bref{eq:e-f-01} 的类似物,是因为由表面束缚源产生的表面场,也应在同介质内抵消  ---  即因果同形。此外,正如 \bref{eq:e-f-01'} 一样,总可以重定义 \bref{eq:EB-01} 为任何形式,它们都是等价的。},到至少低一层次的\textcolor{Plum}{奇异}场:
\begin{subequations} \label{eq:EB-01}
%	\abovedisplayskip=6pt
\begin{align}
	\hphantom{xxxxxxxxx} E^{\;\!\mathcolor{gray}{t}}_{\;\! \symup{\iota}\mathcolor{gray}{z}} &= &&\hspace{-2.5em}\leftindex_{\textcolor{Maroon}{\mathsfit{z}}} {\mathbb{1}}_{\mathcolor{gray}{z}} \leftindex^{\textcolor{Maroon}{\mathsfit{z}}} \;\! E^{\;\!\mathcolor{gray}{t}}_{\;\! \symup{\iota}\mathcolor{gray}{z}} &&\hspace{-2.5em}- \leftindex_{\textcolor{Maroon}{\mathsfit{z}}} \;\! \delta_{\mathcolor{gray}{z}} \leftindex^{\textcolor{Maroon}{\mathsfit{z}}} \;\!
	{\mathcal{E}}^{\;\!\textcolor{Maroon}{(1)}\mathcolor{gray}{t}}_{\;\! \symup{\iota}\mathcolor{gray}{z}} &&\hspace{-2.5em}- \leftindex_{\textcolor{Maroon}{\mathsfit{z}}} \;\! \delta'_{\mathcolor{gray}{z}} \leftindex^{\textcolor{Maroon}{\mathsfit{z}}} \;\! {\mathcal{E}}^{\;\!\textcolor{Maroon}{(2)}\mathcolor{gray}{t}}_{\;\! \symup{\iota}\mathcolor{gray}{z}} ~,&& \label{eq:E-01} \\
	\hphantom{xxxxxxxxx} B^{\;\!\mathcolor{gray}{t}}_{\;\! \symup{\iota}\mathcolor{gray}{z}} &= &&\hspace{-2.5em}\leftindex_{\textcolor{Maroon}{\mathsfit{z}}} {\mathbb{1}}_{\mathcolor{gray}{z}} \leftindex^{\textcolor{Maroon}{\mathsfit{z}}} \;\! B^{\;\!\mathcolor{gray}{t}}_{\;\! \symup{\iota}\mathcolor{gray}{z}} &&\hspace{-2.5em}- \leftindex_{\textcolor{Maroon}{\mathsfit{z}}} \;\! \delta_{\mathcolor{gray}{z}} \leftindex^{\textcolor{Maroon}{\mathsfit{z}}} \;\!
	{\mathcal{B}}^{\;\!\textcolor{Maroon}{(1)}\mathcolor{gray}{t}}_{\;\! \symup{\iota}\mathcolor{gray}{z}} &&\hspace{-2.5em}- \leftindex_{\textcolor{Maroon}{\mathsfit{z}}} \;\! \delta'_{\mathcolor{gray}{z}} \leftindex^{\textcolor{Maroon}{\mathsfit{z}}} \;\! {\mathcal{B}}^{\;\!\textcolor{Maroon}{(2)}\mathcolor{gray}{t}}_{\;\! \symup{\iota}\mathcolor{gray}{z}} ~,&& \label{eq:B-01}
\end{align}
\end{subequations}
将上述 \bref{eq:EB-01} 带入 \bref{eq:curl-EB} 的四个方程中,得到 \bref{eq:div-e-b-01-deltas} 的类似物
%\clearpage
%\vspace*{-3.5em}
\begin{subequations} \label{eq:curl-EB-01}
%	\abovedisplayskip=6pt
	\footnotesize
\begin{align}
	\epsilon^{\hphantom{\symup{\symup{\iota}z}}\hat{2}}_{\symup{\iota} \symup{z}} \left( \leftindex_{\textcolor{Maroon}{\mathsfit{z}}} \;\! \delta_{\mathcolor{gray}{z}} \leftindex^{\textcolor{Maroon}{\mathsfit{z}}} E^{\;\!\mathcolor{gray}{t}}_{\;\! \hat{2}\mathcolor{gray}{z}} - \leftindex_{\textcolor{Maroon}{\mathsfit{z}}} \;\! \delta'_{\mathcolor{gray}{z}} \leftindex^{\textcolor{Maroon}{\mathsfit{z}}} \;\!
	{\mathcal{E}}^{\;\!\textcolor{Maroon}{(1)}\mathcolor{gray}{t}}_{\;\! \hat{2}\mathcolor{gray}{z}} \right) &- \epsilon^{\hphantom{\symup{\iota}\hat{1}}\hat{2}}_{\symup{\iota}\mathcolor{gray}{\hat{1}}} \left( \leftindex_{\textcolor{Maroon}{\mathsfit{z}}} \;\! \delta_{\mathcolor{gray}{z}} \mathcolor{gray}{\nabla^{\hat{1}}} \leftindex^{\textcolor{Maroon}{\mathsfit{z}}} \;\!
	{\mathcal{E}}^{\;\!\textcolor{Maroon}{(1)}\mathcolor{gray}{t}}_{\;\! \hat{2}\mathcolor{gray}{z}} + \leftindex_{\textcolor{Maroon}{\mathsfit{z}}} \;\! \delta'_{\mathcolor{gray}{z}} \mathcolor{gray}{\nabla^{\hat{1}}} \leftindex^{\textcolor{Maroon}{\mathsfit{z}}} \;\!
	{\mathcal{E}}^{\;\!\textcolor{Maroon}{(2)}\mathcolor{gray}{t}}_{\;\! \hat{2}\mathcolor{gray}{z}} \right) - \mathcolor{gray}{\nabla^t} \left( \leftindex_{\textcolor{Maroon}{\mathsfit{z}}} \;\! \delta_{\mathcolor{gray}{z}} \leftindex^{\textcolor{Maroon}{\mathsfit{z}}}
	{\mathcal{B}}^{\;\!\textcolor{Maroon}{(1)}\mathcolor{gray}{t}}_{\;\! \symup{\iota}\mathcolor{gray}{z}} + \leftindex_{\textcolor{Maroon}{\mathsfit{z}}} \;\! \delta'_{\mathcolor{gray}{z}} \leftindex^{\textcolor{Maroon}{\mathsfit{z}}} \;\! {\mathcal{B}}^{\;\!\textcolor{Maroon}{(2)}\mathcolor{gray}{t}}_{\;\! \symup{\iota}\mathcolor{gray}{z}} \right) \label{eq:curl-E-01} \\ &= \epsilon^{\hphantom{\symup{\symup{\iota}z}}\hat{2}}_{\symup{\iota} \symup{z}} \leftindex_{\textcolor{Maroon}{\mathsfit{z}}} \;\! \delta''_{\mathcolor{gray}{z}} \leftindex^{\textcolor{Maroon}{\mathsfit{z}}} \;\!
	{\mathcal{E}}^{\;\!\textcolor{Maroon}{(2)}\mathcolor{gray}{t}}_{\;\! \hat{2}\mathcolor{gray}{z}}~, \\ \epsilon^{\hphantom{\symup{\symup{\iota}z}}\hat{2}}_{\symup{\iota} \symup{z}} \left( \leftindex_{\textcolor{Maroon}{\mathsfit{z}}} \;\! \delta_{\mathcolor{gray}{z}} \leftindex^{\textcolor{Maroon}{\mathsfit{z}}} B^{\;\!\mathcolor{gray}{t}}_{\;\! \hat{2}\mathcolor{gray}{z}} - \leftindex_{\textcolor{Maroon}{\mathsfit{z}}} \;\! \delta'_{\mathcolor{gray}{z}} \leftindex^{\textcolor{Maroon}{\mathsfit{z}}} \;\!
	{\mathcal{B}}^{\;\!\textcolor{Maroon}{(1)}\mathcolor{gray}{t}}_{\;\! \hat{2}\mathcolor{gray}{z}} \right) &- \epsilon^{\hphantom{\symup{\iota}\hat{1}}\hat{2}}_{\symup{\iota}\mathcolor{gray}{\hat{1}}} \left( \leftindex_{\textcolor{Maroon}{\mathsfit{z}}} \;\! \delta_{\mathcolor{gray}{z}} \mathcolor{gray}{\nabla^{\hat{1}}} \leftindex^{\textcolor{Maroon}{\mathsfit{z}}} \;\!
	{\mathcal{B}}^{\;\!\textcolor{Maroon}{(1)}\mathcolor{gray}{t}}_{\;\! \hat{2}\mathcolor{gray}{z}} + \leftindex_{\textcolor{Maroon}{\mathsfit{z}}} \;\! \delta'_{\mathcolor{gray}{z}} \mathcolor{gray}{\nabla^{\hat{1}}} \leftindex^{\textcolor{Maroon}{\mathsfit{z}}} \;\!
	{\mathcal{B}}^{\;\!\textcolor{Maroon}{(2)}\mathcolor{gray}{t}}_{\;\! \hat{2}\mathcolor{gray}{z}} \right) + \mathcolor{gray}{\nabla^t} \left( \leftindex_{\textcolor{Maroon}{\mathsfit{z}}} \;\! \delta_{\mathcolor{gray}{z}} \leftindex^{\textcolor{Maroon}{\mathsfit{z}}}
	{\mathcal{E}}^{\;\!\textcolor{Maroon}{(1)}\mathcolor{gray}{t}}_{\;\! \symup{\iota}\mathcolor{gray}{z}} + \leftindex_{\textcolor{Maroon}{\mathsfit{z}}} \;\! \delta'_{\mathcolor{gray}{z}} \leftindex^{\textcolor{Maroon}{\mathsfit{z}}} \;\! {\mathcal{E}}^{\;\!\textcolor{Maroon}{(2)}\mathcolor{gray}{t}}_{\;\! \symup{\iota}\mathcolor{gray}{z}} \right) \big/ {\symup{c}}^2 \label{eq:curl-B-01} \\ &= - {\symup{\varepsilon}}_0 \left[ \leftindex_{\textcolor{Maroon}{\mathsfit{z}}} \;\! \delta_{\mathcolor{gray}{z}} \left( \leftindex^{\textcolor{Maroon}{\mathsfit{z}}}
	{\mathcal{K}}^{\;\!\mathcolor{gray}{t}}_{\;\! \symup{\iota}\symup{z}\mathcolor{gray}{z}} - \leftindex^{\textcolor{Maroon}{\mathsfit{z}}} \;\!
	{\alpha}^{\;\!\mathcolor{gray}{t}}_{\;\! \symup{\iota}\mathcolor{gray}{z}} \right) + \leftindex_{\textcolor{Maroon}{\mathsfit{z}}} \;\! \delta'_{\mathcolor{gray}{z}} \leftindex^{\textcolor{Maroon}{\mathsfit{z}}} \;\! {\mathcal{L}}^{\;\!\mathcolor{gray}{t}}_{\;\! \symup{\iota}\symup{z} \symup{z} \mathcolor{gray}{z}} \right] + \epsilon^{\hphantom{\symup{\symup{\iota}z}}\hat{2}}_{\symup{\iota} \symup{z}} \leftindex_{\textcolor{Maroon}{\mathsfit{z}}} \;\! \delta''_{\mathcolor{gray}{z}} \leftindex^{\textcolor{Maroon}{\mathsfit{z}}} \;\!
	{\mathcal{B}}^{\;\!\textcolor{Maroon}{(2)}\mathcolor{gray}{t}}_{\;\! \hat{2}\mathcolor{gray}{z}} ~, \\
	\left( \leftindex_{\textcolor{Maroon}{\mathsfit{z}}} \;\! \delta_{\mathcolor{gray}{z}} \leftindex^{\textcolor{Maroon}{\mathsfit{z}}} \;\! E^{\;\!\mathcolor{gray}{t}}_{\;\! \symup{z} \mathcolor{gray}{z}} - \leftindex_{\textcolor{Maroon}{\mathsfit{z}}} \;\! \delta'_{\mathcolor{gray}{z}} \leftindex^{\textcolor{Maroon}{\mathsfit{z}}} \;\!
	{\mathcal{E}}^{\;\!\textcolor{Maroon}{(1)}\mathcolor{gray}{t}}_{\;\! \symup{z} \mathcolor{gray}{z}} \right. &- \left. \leftindex_{\textcolor{Maroon}{\mathsfit{z}}} \;\! \delta''_{\mathcolor{gray}{z}} \leftindex^{\textcolor{Maroon}{\mathsfit{z}}} \;\! {\mathcal{E}}^{\;\!\textcolor{Maroon}{(2)}\mathcolor{gray}{t}}_{\;\! \symup{z} \mathcolor{gray}{z}} \right) - \left( \leftindex_{\textcolor{Maroon}{\mathsfit{z}}} \;\! \delta_{\mathcolor{gray}{z}} \mathcolor{gray}{\nabla^\iota} \leftindex^{\textcolor{Maroon}{\mathsfit{z}}} \;\!
	{\mathcal{E}}^{\;\!\textcolor{Maroon}{(1)}\mathcolor{gray}{t}}_{\;\! \mathcolor{gray}{\symup{\iota}} \mathcolor{gray}{z}} + \leftindex_{\textcolor{Maroon}{\mathsfit{z}}} \;\! \delta'_{\mathcolor{gray}{z}} \mathcolor{gray}{\nabla^\iota} \leftindex^{\textcolor{Maroon}{\mathsfit{z}}} \;\!
	{\mathcal{E}}^{\;\!\textcolor{Maroon}{(2)}\mathcolor{gray}{t}}_{\;\! \mathcolor{gray}{\symup{\iota}} \mathcolor{gray}{z}} \right) \label{eq:div-E-01} \\ &= - {\symup{\varepsilon}}_0^{-1} \left[ \leftindex_{\textcolor{Maroon}{\mathsfit{z}}} \;\! \delta_{\mathcolor{gray}{z}} \left( \leftindex^{\textcolor{Maroon}{\mathsfit{z}}} \;\! {\mathcal{P}}^{\;\!\mathcolor{gray}{t}}_{\;\! \symup{z} \mathcolor{gray}{z}} - \leftindex^{\textcolor{Maroon}{\mathsfit{z}}} \;\! {\sigma}^{\;\!\mathcolor{gray}{t}}_{\;\! \mathcolor{gray}{z}} \right) + \leftindex_{\textcolor{Maroon}{\mathsfit{z}}} \;\! \delta'_{\mathcolor{gray}{z}} \leftindex^{\textcolor{Maroon}{\mathsfit{z}}} \;\! {\mathcal{Q}}^{\;\!\mathcolor{gray}{t}}_{\;\! \symup{z} \symup{z} \mathcolor{gray}{z}} + \leftindex_{\textcolor{Maroon}{\mathsfit{z}}} \;\! \delta''_{\mathcolor{gray}{z}} \leftindex^{\textcolor{Maroon}{\mathsfit{z}}} \;\! {\mathcal{O}}^{\;\!\mathcolor{gray}{t}}_{\;\! \symup{z} \symup{z} \symup{z} \mathcolor{gray}{z}} \right]~, \\
	\left( \leftindex_{\textcolor{Maroon}{\mathsfit{z}}} \;\! \delta_{\mathcolor{gray}{z}} \leftindex^{\textcolor{Maroon}{\mathsfit{z}}} \;\! B^{\;\!\mathcolor{gray}{t}}_{\;\! \symup{z} \mathcolor{gray}{z}} - \leftindex_{\textcolor{Maroon}{\mathsfit{z}}} \;\! \delta'_{\mathcolor{gray}{z}} \leftindex^{\textcolor{Maroon}{\mathsfit{z}}} \;\!
	{\mathcal{B}}^{\;\!\textcolor{Maroon}{(1)}\mathcolor{gray}{t}}_{\;\! \symup{z} \mathcolor{gray}{z}} \right. &- \left. \leftindex_{\textcolor{Maroon}{\mathsfit{z}}} \;\! \delta''_{\mathcolor{gray}{z}} \leftindex^{\textcolor{Maroon}{\mathsfit{z}}} \;\! {\mathcal{B}}^{\;\!\textcolor{Maroon}{(2)}\mathcolor{gray}{t}}_{\;\! \symup{z} \mathcolor{gray}{z}} \right) - \left( \leftindex_{\textcolor{Maroon}{\mathsfit{z}}} \;\! \delta_{\mathcolor{gray}{z}} \mathcolor{gray}{\nabla^\iota} \leftindex^{\textcolor{Maroon}{\mathsfit{z}}} \;\!
	{\mathcal{B}}^{\;\!\textcolor{Maroon}{(1)}\mathcolor{gray}{t}}_{\;\! \mathcolor{gray}{\symup{\iota}} \mathcolor{gray}{z}} + \leftindex_{\textcolor{Maroon}{\mathsfit{z}}} \;\! \delta'_{\mathcolor{gray}{z}} \mathcolor{gray}{\nabla^\iota} \leftindex^{\textcolor{Maroon}{\mathsfit{z}}} \;\!
	{\mathcal{B}}^{\;\!\textcolor{Maroon}{(2)}\mathcolor{gray}{t}}_{\;\! \mathcolor{gray}{\symup{\iota}} \mathcolor{gray}{z}} \right) \label{eq:div-B-01} \\ &= 0~. 
\end{align}
\end{subequations}
合并同层次\textcolor{Plum}{奇异}项(通过 \bref{eq:Intdeltasum=0}),得到 \bref{eq:div-e-b-01-delta-conclusions} 的对应版本
\begin{subequations} \label{eq:curl-EB-01-deltas}
\begin{align}
	{\delta}_{\mathcolor{gray}{z}} ~\textcolor{Maroon}{\text{项}}:&\hspace{1.0em}  \epsilon^{\hphantom{\symup{\symup{\iota}z}}\hat{2}}_{\symup{\iota} \symup{z}} E^{\;\!\mathcolor{gray}{t}}_{\;\! \hat{2}\mathcolor{gray}{0}} - \epsilon^{\hphantom{\symup{\iota}\hat{1}}\hat{2}}_{\symup{\iota}\mathcolor{gray}{\hat{1}}} \mathcolor{gray}{\nabla^{\hat{1}}} 
	{\mathcal{E}}^{\;\!\textcolor{Maroon}{(1)}\mathcolor{gray}{t}}_{\;\! \hat{2}\mathcolor{gray}{0}} - \mathcolor{gray}{\nabla^t} 
	{\mathcal{B}}^{\;\!\textcolor{Maroon}{(1)}\mathcolor{gray}{t}}_{\;\! \symup{\iota}\mathcolor{gray}{0}} \hspace{-1.8em}&&=\hspace{0.2em} 0~, \hspace{-2.5em} &&\hspace{-1.3em} \label{eq:curl-EB-01-delta} \\
	&\hspace{1.0em} \epsilon^{\hphantom{\symup{\symup{\iota}z}}\hat{2}}_{\symup{\iota} \symup{z}} B^{\;\!\mathcolor{gray}{t}}_{\;\! \hat{2}\mathcolor{gray}{0}} - \epsilon^{\hphantom{\symup{\iota}\hat{1}}\hat{2}}_{\symup{\iota}\mathcolor{gray}{\hat{1}}} \mathcolor{gray}{\nabla^{\hat{1}}} 
	{\mathcal{B}}^{\;\!\textcolor{Maroon}{(1)}\mathcolor{gray}{t}}_{\;\! \hat{2}\mathcolor{gray}{0}} + \mathcolor{gray}{\nabla^t} 
	{\mathcal{E}}^{\;\!\textcolor{Maroon}{(1)}\mathcolor{gray}{t}}_{\;\! \symup{\iota}\mathcolor{gray}{0}} \big/ {\symup{c}}^2 \hspace{-1.8em}&&=\hspace{0.2em} -\hspace{0.2em} {\symup{\varepsilon}}_0 \left( 
	{\mathcal{K}}^{\;\!\mathcolor{gray}{t}}_{\;\! \symup{\iota}\symup{z}\mathcolor{gray}{0}} \hspace{-2.5em}\right. &&\hspace{-1.5em}- \left. 
	{\alpha}^{\;\!\mathcolor{gray}{t}}_{\;\! \symup{\iota}\mathcolor{gray}{0}} \right)~, \label{eq:curl-EB-01-delta2} \\
	&\hspace{1.0em} \hphantom{\epsilon^{\hphantom{\symup{\symup{\iota}z}}\hat{2}}_{\symup{\iota} \symup{z}}} E^{\;\!\mathcolor{gray}{t}}_{\;\! \symup{z} \mathcolor{gray}{0}} - \hphantom{\epsilon^{\hphantom{\symup{\iota}\hat{1}}\hat{2}}_{\symup{\iota}\mathcolor{gray}{\hat{1}}}} \mathcolor{gray}{\nabla^\iota} 
	{\mathcal{E}}^{\;\!\textcolor{Maroon}{(1)}\mathcolor{gray}{t}}_{\;\! \mathcolor{gray}{\symup{\iota}} \mathcolor{gray}{0}} \hspace{-1.8em}&&=\hspace{0.2em} -\hspace{0.2em} {\symup{\varepsilon}}_0^{-1} \left( {\mathcal{P}}^{\;\!\mathcolor{gray}{t}}_{\;\! \symup{z} \mathcolor{gray}{0}} \hspace{-2.5em}\right. &&\hspace{-1.5em}- \left. {\sigma}^{\;\!\mathcolor{gray}{t}}_{\;\! \mathcolor{gray}{0}} \right)~, \label{eq:curl-EB-01-delta3} \\ 
	&\hspace{1.0em} \hphantom{\epsilon^{\hphantom{\symup{\symup{\iota}z}}\hat{2}}_{\symup{\iota} \symup{z}}} B^{\;\!\mathcolor{gray}{t}}_{\;\! \symup{z} \mathcolor{gray}{0}} - \hphantom{\epsilon^{\hphantom{\symup{\iota}\hat{1}}\hat{2}}_{\symup{\iota}\mathcolor{gray}{\hat{1}}}} \mathcolor{gray}{\nabla^\iota} 
	{\mathcal{B}}^{\;\!\textcolor{Maroon}{(1)}\mathcolor{gray}{t}}_{\;\! \mathcolor{gray}{\symup{\iota}} \mathcolor{gray}{0}} \hspace{-1.8em}&&=\hspace{0.2em} 0~, \hspace{-2.5em} &&\hspace{-1.3em} \label{eq:curl-EB-01-delta4} \\[0.7em]
	{\delta}'_{\mathcolor{gray}{z}} ~\textcolor{Maroon}{\text{项}}:&\hspace{1.0em}  \epsilon^{\hphantom{\symup{\symup{\iota}z}}\hat{2}}_{\symup{\iota} \symup{z}} {\mathcal{E}}^{\;\!\textcolor{Maroon}{(1)}\mathcolor{gray}{t}}_{\;\! \hat{2}\mathcolor{gray}{0}} + \epsilon^{\hphantom{\symup{\iota}\hat{1}}\hat{2}}_{\symup{\iota}\mathcolor{gray}{\hat{1}}} \mathcolor{gray}{\nabla^{\hat{1}}} 
	{\mathcal{E}}^{\;\!\textcolor{Maroon}{(2)}\mathcolor{gray}{t}}_{\;\! \hat{2}\mathcolor{gray}{0}} + \mathcolor{gray}{\nabla^t} 
	{\mathcal{B}}^{\;\!\textcolor{Maroon}{(2)}\mathcolor{gray}{t}}_{\;\! \symup{\iota}\mathcolor{gray}{0}} \hspace{-1.8em}&&=\hspace{0.2em} 0~, \hspace{-2.5em} &&\hspace{-1.3em} \label{eq:curl-EB-01-delta'} \\
	&\hspace{1.0em} \epsilon^{\hphantom{\symup{\symup{\iota}z}}\hat{2}}_{\symup{\iota} \symup{z}} {\mathcal{B}}^{\;\!\textcolor{Maroon}{(1)}\mathcolor{gray}{t}}_{\;\! \hat{2}\mathcolor{gray}{0}} + \epsilon^{\hphantom{\symup{\iota}\hat{1}}\hat{2}}_{\symup{\iota}\mathcolor{gray}{\hat{1}}} \mathcolor{gray}{\nabla^{\hat{1}}} 
	{\mathcal{B}}^{\;\!\textcolor{Maroon}{(2)}\mathcolor{gray}{t}}_{\;\! \hat{2}\mathcolor{gray}{0}} - \mathcolor{gray}{\nabla^t} 
	{\mathcal{E}}^{\;\!\textcolor{Maroon}{(2)}\mathcolor{gray}{t}}_{\;\! \symup{\iota}\mathcolor{gray}{0}} \big/ {\symup{c}}^2 \hspace{-1.8em}&&=\hspace{0.2em} {\symup{\varepsilon}}_0 {\mathcal{L}}^{\;\!\mathcolor{gray}{t}}_{\;\! \symup{\iota}\symup{z} \symup{z} \mathcolor{gray}{0}}~, \hspace{-2.5em} &&\hspace{-1.3em} \label{eq:curl-EB-01-delta'2} \\
	&\hspace{1.0em} \hphantom{\epsilon^{\hphantom{\symup{\symup{\iota}z}}\hat{2}}_{\symup{\iota} \symup{z}}} {\mathcal{E}}^{\;\!\textcolor{Maroon}{(1)}\mathcolor{gray}{t}}_{\;\! \symup{z} \mathcolor{gray}{0}} + \hphantom{\epsilon^{\hphantom{\symup{\iota}\hat{1}}\hat{2}}_{\symup{\iota}\mathcolor{gray}{\hat{1}}}} \mathcolor{gray}{\nabla^\iota} 
	{\mathcal{E}}^{\;\!\textcolor{Maroon}{(2)}\mathcolor{gray}{t}}_{\;\! \mathcolor{gray}{\symup{\iota}} \mathcolor{gray}{0}} \hspace{-1.8em}&&=\hspace{0.2em} {\symup{\varepsilon}}_0^{-1} {\mathcal{Q}}^{\;\!\mathcolor{gray}{t}}_{\;\! \symup{z} \symup{z} \mathcolor{gray}{0}}~, \hspace{-2.5em} &&\hspace{-1.3em} \label{eq:curl-EB-01-delta'3} \\ 
	&\hspace{1.0em} \hphantom{\epsilon^{\hphantom{\symup{\symup{\iota}z}}\hat{2}}_{\symup{\iota} \symup{z}}} {\mathcal{B}}^{\;\!\textcolor{Maroon}{(1)}\mathcolor{gray}{t}}_{\;\! \symup{z} \mathcolor{gray}{0}} + \hphantom{\epsilon^{\hphantom{\symup{\iota}\hat{1}}\hat{2}}_{\symup{\iota}\mathcolor{gray}{\hat{1}}}} \mathcolor{gray}{\nabla^\iota} 
	{\mathcal{B}}^{\;\!\textcolor{Maroon}{(2)}\mathcolor{gray}{t}}_{\;\! \mathcolor{gray}{\symup{\iota}} \mathcolor{gray}{0}} \hspace{-1.8em}&&=\hspace{0.2em} 0~, \hspace{-2.5em} &&\hspace{-1.3em} \label{eq:curl-EB-01-delta'4} \\[0.7em]
	{\delta}''_{\mathcolor{gray}{z}} ~\textcolor{Maroon}{\text{项}}:&\hspace{1.0em} \epsilon^{\hphantom{\symup{\symup{\iota}z}}\hat{2}}_{\symup{\iota} \symup{z}} 
	{\mathcal{E}}^{\;\!\textcolor{Maroon}{(2)}\mathcolor{gray}{t}}_{\;\! \hat{2}\mathcolor{gray}{0}} \hspace{-1.8em}&&=\hspace{0.2em} 0~, &&\hspace{-1.3em} \label{eq:curl-EB-01-delta''} \\
	&\hspace{1.0em} \epsilon^{\hphantom{\symup{\symup{\iota}z}}\hat{2}}_{\symup{\iota} \symup{z}} 
	{\mathcal{B}}^{\;\!\textcolor{Maroon}{(2)}\mathcolor{gray}{t}}_{\;\! \hat{2}\mathcolor{gray}{0}} \hspace{-1.8em}&&=\hspace{0.2em} 0~, &&\hspace{-1.3em} \label{eq:curl-EB-01-delta''2} \\
	&\hspace{1.0em} \hphantom{\epsilon^{\hphantom{\symup{\symup{\iota}z}}\hat{2}}_{\symup{\iota} \symup{z}}} 
	{\mathcal{E}}^{\;\!\textcolor{Maroon}{(2)}\mathcolor{gray}{t}}_{\;\! \symup{z} \mathcolor{gray}{0}} \hspace{-1.8em}&&=\hspace{0.2em} {\symup{\varepsilon}}_0^{-1} {\mathcal{O}}^{\;\!\mathcolor{gray}{t}}_{\;\! \symup{z} \symup{z} \symup{z} \mathcolor{gray}{0}}~, &&\hspace{-1.3em} \label{eq:curl-EB-01-delta''3} \\
	&\hspace{1.0em} \hphantom{\epsilon^{\hphantom{\symup{\symup{\iota}z}}\hat{2}}_{\symup{\iota} \symup{z}}} 
	{\mathcal{B}}^{\;\!\textcolor{Maroon}{(2)}\mathcolor{gray}{t}}_{\;\! \symup{z} \mathcolor{gray}{0}} \hspace{-1.8em}&&=\hspace{0.2em} 0~. &&\hspace{-1.3em} \label{eq:curl-EB-01-delta''4}
\end{align}
\end{subequations}
能满足 \bref{eq:curl-EB-01-delta''2,eq:curl-EB-01-delta''4} 的
$\bar{\mathcal{B}}^{\;\!\textcolor{Maroon}{(2)}\mathcolor{gray}{t}}_{\;\!  \mathcolor{gray}{0}} = \left( 0,~ 0,~ 0 \right)^{\mathsf{\textcolor{Plum}{T}}}$,能满足 \bref{eq:curl-EB-01-delta'',eq:curl-EB-01-delta''3} 的 $\bar{\mathcal{E}}^{\;\!\textcolor{Maroon}{(2)}\mathcolor{gray}{t}}_{\;\!  \mathcolor{gray}{0}} = {\symup{\varepsilon}}_0^{-1} \left( 0,~ 0,~ {\mathcal{O}}^{\;\!\mathcolor{gray}{t}}_{\;\! \symup{z} \symup{z} \symup{z} \mathcolor{gray}{0}} \right)^{\mathsf{\textcolor{Plum}{T}}}$;将 $\bar{\mathcal{B}}^{\;\!\textcolor{Maroon}{(2)}\mathcolor{gray}{t}}_{\;\!  \mathcolor{gray}{0}},\bar{\mathcal{E}}^{\;\!\textcolor{Maroon}{(2)}\mathcolor{gray}{t}}_{\;\! \mathcolor{gray}{0}}$ 代入 \bref{eq:curl-EB-01-delta'2,eq:curl-EB-01-delta'4} 可得 $\bar{\mathcal{B}}^{\;\!\textcolor{Maroon}{(1)}\mathcolor{gray}{t}}_{\;\!  \mathcolor{gray}{0}} = {\symup{\varepsilon}}_0 \left( {\mathcal{L}}^{\;\!\mathcolor{gray}{t}}_{\;\! \symup{y} \symup{z} \symup{z} \mathcolor{gray}{0}},~ - {\mathcal{L}}^{\;\!\mathcolor{gray}{t}}_{\;\! \symup{x} \symup{z} \symup{z} \mathcolor{gray}{0}},~ 0 \right)^{\mathsf{\textcolor{Plum}{T}}}$,然后再代入 \bref{eq:curl-EB-01-delta',eq:curl-EB-01-delta'3},得到 $\bar{\mathcal{E}}^{\;\!\textcolor{Maroon}{(1)}\mathcolor{gray}{t}}_{\;\!  \mathcolor{gray}{0}} = {\symup{\varepsilon}}_0^{-1} \left( \mathcolor{gray}{\nabla_x} \mathcal{E}^{\;\!\textcolor{Maroon}{(2)}\mathcolor{gray}{t}}_{\;\! \symup{z} \mathcolor{gray}{0}}, \right.$ $\left. \mathcolor{gray}{\nabla_y} \mathcal{E}^{\;\!\textcolor{Maroon}{(2)}\mathcolor{gray}{t}}_{\;\! \symup{z} \mathcolor{gray}{0}},~~ {\mathcal{Q}}^{\;\!\mathcolor{gray}{t}}_{\;\! \symup{z} \symup{z} \mathcolor{gray}{0}} - \mathcolor{gray}{\nabla_z} \mathcal{E}^{\;\!\textcolor{Maroon}{(2)}\mathcolor{gray}{t}}_{\;\! \mathcolor{gray}{\symup{z}} \mathcolor{gray}{0}} \right)^{\mathsf{\textcolor{Plum}{T}}}$,即 $\bar{\mathcal{E}}^{\;\!\textcolor{Maroon}{(1)}\mathcolor{gray}{t}}_{\;\!  \mathcolor{gray}{0}} = {\symup{\varepsilon}}_0^{-1} \left( \mathcolor{gray}{\nabla_x} {\mathcal{O}}^{\;\!\mathcolor{gray}{t}}_{\;\! \symup{z} \symup{z} \symup{z} \mathcolor{gray}{0}},~ \mathcolor{gray}{\nabla_y} {\mathcal{O}}^{\;\!\mathcolor{gray}{t}}_{\;\! \symup{z} \symup{z} \symup{z} \mathcolor{gray}{0}},~ {\mathcal{Q}}^{\;\!\mathcolor{gray}{t}}_{\;\! \symup{z} \symup{z} \mathcolor{gray}{0}} - \mathcolor{gray}{\nabla_z} {\mathcal{O}}^{\;\!\mathcolor{gray}{t}}_{\;\! \symup{z} \symup{z} \mathcolor{gray}{\symup{z}} \mathcolor{gray}{0}} \right)^{\mathsf{\textcolor{Plum}{T}}}$。

将上一段各非零量展开至 \bref{eq:multipole} 层次,即有
\begin{subequations} \label{eq:EB^(2-1)_0}
\begin{align}
	{\mathcal{E}}^{\;\!\textcolor{Maroon}{(2)}\mathcolor{gray}{t}}_{\;\! \symup{z} \mathcolor{gray}{0}} &= {\symup{\varepsilon}}_0^{-1} {\mathcal{O}}^{\;\!\mathcolor{gray}{t}}_{\;\!\textcolor{Maroon}{\text{b}}\symup{z} \symup{z} \symup{z} \mathcolor{gray}{0}}~; \label{eq:E^(2)_z0} \\[0.7em]
	{\mathcal{B}}^{\;\!\textcolor{Maroon}{(1)}\mathcolor{gray}{t}}_{\;\! \symup{x} \mathcolor{gray}{0}} &= {\symup{\varepsilon}}_0 \left( -\hspace{0.2em} \mathcolor{gray}{\nabla^t}
	{\mathcal{O}}^{\;\!\mathcolor{gray}{t}}_{\;\!\textcolor{Maroon}{\text{b}}\symup{y}\symup{z}\symup{z}\mathcolor{gray}{0}} + {\mathcal{L}}^{\;\!\mathcolor{gray}{t}}_{\;\!\textcolor{Maroon}{\text{m}}\symup{y}\symup{z}\symup{z}\mathcolor{gray}{0}} \right)~, \label{eq:B^(1)_x0} \\
	{\mathcal{B}}^{\;\!\textcolor{Maroon}{(1)}\mathcolor{gray}{t}}_{\;\! \symup{y} \mathcolor{gray}{0}} &= {\symup{\varepsilon}}_0 \left( +\hspace{0.2em} \mathcolor{gray}{\nabla^t}
	{\mathcal{O}}^{\;\!\mathcolor{gray}{t}}_{\;\!\textcolor{Maroon}{\text{b}}\symup{x}\symup{z}\symup{z}\mathcolor{gray}{0}} - {\mathcal{L}}^{\;\!\mathcolor{gray}{t}}_{\;\!\textcolor{Maroon}{\text{m}}\symup{x}\symup{z}\symup{z}\mathcolor{gray}{0}} \right)~; \label{eq:B^(1)_y0} \\[0.7em]
	{\mathcal{E}}^{\;\!\textcolor{Maroon}{(1)}\mathcolor{gray}{t}}_{\;\! \symup{x} \mathcolor{gray}{0}} &= {\symup{\varepsilon}}_0^{-1} \mathcolor{gray}{\nabla_x} {\mathcal{O}}^{\;\!\mathcolor{gray}{t}}_{\;\!\textcolor{Maroon}{\text{b}}\symup{z} \symup{z} \symup{z} \mathcolor{gray}{0}}~, \label{eq:E^(1)_x0} \\
	{\mathcal{E}}^{\;\!\textcolor{Maroon}{(1)}\mathcolor{gray}{t}}_{\;\! \symup{y} \mathcolor{gray}{0}} &= {\symup{\varepsilon}}_0^{-1} \mathcolor{gray}{\nabla_y} {\mathcal{O}}^{\;\!\mathcolor{gray}{t}}_{\;\!\textcolor{Maroon}{\text{b}}\symup{z} \symup{z} \symup{z} \mathcolor{gray}{0}}~, \label{eq:E^(1)_y0} \\
	{\mathcal{E}}^{\;\!\textcolor{Maroon}{(1)}\mathcolor{gray}{t}}_{\;\! \symup{z} \mathcolor{gray}{0}} &= {\symup{\varepsilon}}_0^{-1} \left( {\mathcal{Q}}^{\;\!\mathcolor{gray}{t}}_{\;\!\textcolor{Maroon}{\text{b}}\symup{z} \symup{z} \mathcolor{gray}{0}} + 2 \mathcolor{gray}{\nabla^{\hat{2}}} {\mathcal{O}}^{\;\!\mathcolor{gray}{t}}_{\;\!\textcolor{Maroon}{\text{b}}\symup{z} \symup{z} \mathcolor{gray}{\hat{2}} \mathcolor{gray}{0}} - \mathcolor{gray}{\nabla_z} {\mathcal{O}}^{\;\!\mathcolor{gray}{t}}_{\;\!\textcolor{Maroon}{\text{b}}\symup{z} \symup{z} \mathcolor{gray}{\symup{z}} \mathcolor{gray}{0}} \right)~, \label{eq:E^(1)_z0}
\end{align}
\end{subequations}

接着,展开 \bref{eq:curl-EB-01-delta2,eq:curl-EB-01-delta4},得到
\begin{subequations} \label{eq:B_0}
\begin{align}
	B^{\;\!\mathcolor{gray}{t}}_{\;\! \symup{x}\mathcolor{gray}{0}} &= \mathcolor{gray}{\nabla_z} 
	{\mathcal{B}}^{\;\!\textcolor{Maroon}{(1)}\mathcolor{gray}{t}}_{\;\! \symup{x} \mathcolor{gray}{0}} - \mathcolor{gray}{\nabla^t} 
	{\mathcal{E}}^{\;\!\textcolor{Maroon}{(1)}\mathcolor{gray}{t}}_{\;\! \symup{y} \mathcolor{gray}{0}} \big/ {\symup{c}}^2 - {\symup{\varepsilon}}_0 \left( 
	{\mathcal{K}}^{\;\!\mathcolor{gray}{t}}_{\;\! \symup{y}\symup{z}\mathcolor{gray}{0}} - 
	{\alpha}^{\;\!\mathcolor{gray}{t}}_{\;\! \symup{y}\mathcolor{gray}{0}} \right)~, \label{eq:B_x0} \\
	B^{\;\!\mathcolor{gray}{t}}_{\;\! \symup{y}\mathcolor{gray}{0}} &= \mathcolor{gray}{\nabla_z} 
	{\mathcal{B}}^{\;\!\textcolor{Maroon}{(1)}\mathcolor{gray}{t}}_{\;\! \symup{y} \mathcolor{gray}{0}} + \mathcolor{gray}{\nabla^t} 
	{\mathcal{E}}^{\;\!\textcolor{Maroon}{(1)}\mathcolor{gray}{t}}_{\;\! \symup{x} \mathcolor{gray}{0}} \big/ {\symup{c}}^2 + {\symup{\varepsilon}}_0 \left( 
	{\mathcal{K}}^{\;\!\mathcolor{gray}{t}}_{\;\! \symup{x}\symup{z}\mathcolor{gray}{0}} - 
	{\alpha}^{\;\!\mathcolor{gray}{t}}_{\;\! \symup{x}\mathcolor{gray}{0}} \right)~, \label{eq:B_y0} \\
	B^{\;\!\mathcolor{gray}{t}}_{\;\! \symup{z} \mathcolor{gray}{0}} &= \mathcolor{gray}{\nabla^\iota} 
	{\mathcal{B}}^{\;\!\textcolor{Maroon}{(1)}\mathcolor{gray}{t}}_{\;\! \mathcolor{gray}{\symup{\iota}} \mathcolor{gray}{0}} = {\symup{\varepsilon}}_0 \left( \mathcolor{gray}{\nabla_x} {\mathcal{L}}^{\;\!\mathcolor{gray}{t}}_{\;\! \symup{y}\symup{z}\symup{z}\mathcolor{gray}{0}} - \mathcolor{gray}{\nabla_y}
	{\mathcal{L}}^{\;\!\mathcolor{gray}{t}}_{\;\! \symup{x}\symup{z}\symup{z}\mathcolor{gray}{0}} \right)~, \label{eq:B_z0}
\end{align}
\end{subequations}
同样,展开 \bref{eq:curl-EB-01-delta,eq:curl-EB-01-delta3},得到
\begin{subequations} \label{eq:E_0}
\begin{align}
	E^{\;\!\mathcolor{gray}{t}}_{\;\! \symup{x}\mathcolor{gray}{0}} &= \mathcolor{gray}{\nabla_z}
	{\mathcal{E}}^{\;\!\textcolor{Maroon}{(1)}\mathcolor{gray}{t}}_{\;\! \symup{x} \mathcolor{gray}{0}} - \mathcolor{gray}{\nabla_x}
	{\mathcal{E}}^{\;\!\textcolor{Maroon}{(1)}\mathcolor{gray}{t}}_{\;\! \symup{z} \mathcolor{gray}{0}} + \mathcolor{gray}{\nabla^t}
	{\mathcal{B}}^{\;\!\textcolor{Maroon}{(1)}\mathcolor{gray}{t}}_{\;\! \symup{y} \mathcolor{gray}{0}}~, \label{eq:E_x0} \\
	E^{\;\!\mathcolor{gray}{t}}_{\;\! \symup{y}\mathcolor{gray}{0}} &= \mathcolor{gray}{\nabla_z}
	{\mathcal{E}}^{\;\!\textcolor{Maroon}{(1)}\mathcolor{gray}{t}}_{\;\! \symup{y} \mathcolor{gray}{0}} - \mathcolor{gray}{\nabla_y}
	{\mathcal{E}}^{\;\!\textcolor{Maroon}{(1)}\mathcolor{gray}{t}}_{\;\! \symup{z} \mathcolor{gray}{0}} - \mathcolor{gray}{\nabla^t}
	{\mathcal{B}}^{\;\!\textcolor{Maroon}{(1)}\mathcolor{gray}{t}}_{\;\! \symup{x} \mathcolor{gray}{0}}~, \label{eq:E_y0} \\
	E^{\;\!\mathcolor{gray}{t}}_{\;\! \symup{z} \mathcolor{gray}{0}} &= \mathcolor{gray}{\nabla^\iota} 
	{\mathcal{E}}^{\;\!\textcolor{Maroon}{(1)}\mathcolor{gray}{t}}_{\;\! \mathcolor{gray}{\symup{\iota}} \mathcolor{gray}{0}} - {\symup{\varepsilon}}_0^{-1} \left( {\mathcal{P}}^{\;\!\mathcolor{gray}{t}}_{\;\! \symup{z} \mathcolor{gray}{0}} - {\sigma}^{\;\!\mathcolor{gray}{t}}_{\;\! \mathcolor{gray}{0}} \right)~. \label{eq:E_z0}
\end{align}
\end{subequations}
%注意,如果表面自由电荷/流 $\leftindex^{\textcolor{Maroon}{\mathsfit{z}}} \;\! {\sigma},\leftindex^{\textcolor{Maroon}{\mathsfit{z}}} \;\! {\alpha}_{\;\! \symup{\iota}}$ 都可忽略,则所有体/表面场的三分量均与所处介质 $\textcolor{Maroon}{\mathsfit{z}}$(的种类/位置)无关;否则如果自由电荷不可忽略,则只有 $\leftindex^{\textcolor{Maroon}{\mathsfit{z}}} \;\! B^{\;\!\mathcolor{gray}{t}}_{\;\! \symup{x}\mathcolor{gray}{0}}, \leftindex^{\textcolor{Maroon}{\mathsfit{z}}} \;\! B^{\;\!\mathcolor{gray}{t}}_{\;\! \symup{y}\mathcolor{gray}{0}}; \leftindex^{\textcolor{Maroon}{\mathsfit{z}}} \;\! E^{\;\!\mathcolor{gray}{t}}_{\;\! \symup{z}\mathcolor{gray}{0}}$ 与所处介质 $\textcolor{Maroon}{\mathsfit{z}}$(的种类/位置)有关。

接着,将前述所得的 $\bar{\mathcal{B}}^{\;\!\textcolor{Maroon}{(1)}\mathcolor{gray}{t}}_{\;\!  \mathcolor{gray}{0}},\bar{\mathcal{E}}^{\;\!\textcolor{Maroon}{(1)}\mathcolor{gray}{t}}_{\;\! \mathcolor{gray}{0}}$ 三分量表达式代入 \bref{eq:B_0,eq:E_0},并像 \bref{eq:EB^(2-1)_0} 一样,展开至 \bref{eq:multipole} 层次,得 $\bar{B}^{\;\!\mathcolor{gray}{t}}_{\;\! \mathcolor{gray}{0}}$ 的三分量
\begin{subequations} \label{eq:B_0'}
\begin{align}
	B^{\;\!\mathcolor{gray}{t}}_{\;\! \symup{x}\mathcolor{gray}{0}} = &\hphantom{+} {\symup{\varepsilon}}_0 \left[ \mathcolor{gray}{\nabla^t} \left(  {\mathcal{Q}}^{\;\!\mathcolor{gray}{t}}_{\;\!\textcolor{Maroon}{\text{b}} \symup{y} \symup{z} \mathcolor{gray}{0}} + \mathcolor{gray}{\nabla_x}  {\mathcal{O}}^{\;\!\mathcolor{gray}{t}}_{\;\!\textcolor{Maroon}{\text{b}} \symup{y} \symup{z} \mathcolor{gray}{\symup{x}} \mathcolor{gray}{0}} + \mathcolor{gray}{\nabla_y}  {\mathcal{O}}^{\;\!\mathcolor{gray}{t}}_{\;\!\textcolor{Maroon}{\text{b}} \symup{y} \symup{z} \mathcolor{gray}{\symup{y}} \mathcolor{gray}{0}} - \mathcolor{gray}{\nabla_y}  {\mathcal{O}}^{\;\!\mathcolor{gray}{t}}_{\;\!\textcolor{Maroon}{\text{b}} \symup{z} \symup{z} \symup{z} \mathcolor{gray}{0}} \right) + 
	{\alpha}^{\;\!\mathcolor{gray}{t}}_{\;\! \symup{y}\mathcolor{gray}{0}} \right. \label{eq:B_x0'} \\ & \left. + M^{\;\!\mathcolor{gray}{t}}_{\;\! \symup{x}\mathcolor{gray}{0}} +
	\mathcolor{gray}{\nabla^{\hat{3}}} \left(  {\mathcal{L}}^{\;\!\mathcolor{gray}{t}}_{\;\!\textcolor{Maroon}{\text{m}} \mathcolor{gray}{\hat{3}} \symup{y} \symup{z} \mathcolor{gray}{0}} -  {\mathcal{L}}^{\;\!\mathcolor{gray}{t}}_{\;\!\textcolor{Maroon}{\text{m}} \symup{y} \symup{z} \mathcolor{gray}{\hat{3}} \mathcolor{gray}{0}} \right) + \mathcolor{gray}{\nabla_z}  {\mathcal{L}}^{\;\!\mathcolor{gray}{t}}_{\;\!\textcolor{Maroon}{\text{m}}\symup{y} \symup{z} \mathcolor{gray}{\symup{z}} \mathcolor{gray}{0}} \right]~, \\
	B^{\;\!\mathcolor{gray}{t}}_{\;\! \symup{y}\mathcolor{gray}{0}} = &- {\symup{\varepsilon}}_0 \left[ \mathcolor{gray}{\nabla^t} \left(  {\mathcal{Q}}^{\;\!\mathcolor{gray}{t}}_{\;\!\textcolor{Maroon}{\text{b}} \symup{x} \symup{z} \mathcolor{gray}{0}} + \mathcolor{gray}{\nabla_x}  {\mathcal{O}}^{\;\!\mathcolor{gray}{t}}_{\;\!\textcolor{Maroon}{\text{b}} \symup{x} \symup{z} \mathcolor{gray}{\symup{x}} \mathcolor{gray}{0}} + \mathcolor{gray}{\nabla_y}  {\mathcal{O}}^{\;\!\mathcolor{gray}{t}}_{\;\!\textcolor{Maroon}{\text{b}} \symup{x} \symup{z} \mathcolor{gray}{\symup{y}} \mathcolor{gray}{0}} - \mathcolor{gray}{\nabla_x}  {\mathcal{O}}^{\;\!\mathcolor{gray}{t}}_{\;\!\textcolor{Maroon}{\text{b}} \symup{z} \symup{z} \symup{z} \mathcolor{gray}{0}} \right) + 
	{\alpha}^{\;\!\mathcolor{gray}{t}}_{\;\! \symup{x}\mathcolor{gray}{0}} \right. \label{eq:B_y0'} \\ & \left. - M^{\;\!\mathcolor{gray}{t}}_{\;\! \symup{y}\mathcolor{gray}{0}} +
	\mathcolor{gray}{\nabla^{\hat{3}}} \left(  {\mathcal{L}}^{\;\!\mathcolor{gray}{t}}_{\;\!\textcolor{Maroon}{\text{m}} \mathcolor{gray}{\hat{3}} \symup{x} \symup{z} \mathcolor{gray}{0}} -  {\mathcal{L}}^{\;\!\mathcolor{gray}{t}}_{\;\!\textcolor{Maroon}{\text{m}} \symup{x} \symup{z} \mathcolor{gray}{\hat{3}} \mathcolor{gray}{0}} \right) + \mathcolor{gray}{\nabla_z}  {\mathcal{L}}^{\;\!\mathcolor{gray}{t}}_{\;\!\textcolor{Maroon}{\text{m}} \symup{x} \symup{z} \mathcolor{gray}{\symup{z}} \mathcolor{gray}{0}} \right]~, \\
	B^{\;\!\mathcolor{gray}{t}}_{\;\! \symup{z} \mathcolor{gray}{0}} = &\hphantom{+} {\symup{\varepsilon}}_0 \left[ \mathcolor{gray}{\nabla^t} \left( \mathcolor{gray}{\nabla_y}
	{\mathcal{O}}^{\;\!\mathcolor{gray}{t}}_{\;\!\textcolor{Maroon}{\text{b}}\symup{x}\symup{z}\symup{z}\mathcolor{gray}{0}} - \mathcolor{gray}{\nabla_x}
	{\mathcal{O}}^{\;\!\mathcolor{gray}{t}}_{\;\!\textcolor{Maroon}{\text{b}}\symup{y}\symup{z}\symup{z}\mathcolor{gray}{0}} \right) + \left( \mathcolor{gray}{\nabla_x} {\mathcal{L}}^{\;\!\mathcolor{gray}{t}}_{\;\!\textcolor{Maroon}{\text{m}}\symup{y}\symup{z}\symup{z}\mathcolor{gray}{0}} - \mathcolor{gray}{\nabla_y}
	{\mathcal{L}}^{\;\!\mathcolor{gray}{t}}_{\;\!\textcolor{Maroon}{\text{m}}\symup{x}\symup{z}\symup{z}\mathcolor{gray}{0}} \right) \right]~, \label{eq:B_z0'}
\end{align}
\end{subequations}
以及 $\bar{E}^{\;\!\mathcolor{gray}{t}}_{\;\! \mathcolor{gray}{0}}$ 的三分量
\begin{subequations} \label{eq:E_0'}
\begin{align}
	E^{\;\!\mathcolor{gray}{t}}_{\;\! \symup{x}\mathcolor{gray}{0}} = &- {\symup{\varepsilon}}_0^{-1} \mathcolor{gray}{\nabla_x} \left[ {\mathcal{Q}}^{\;\!\mathcolor{gray}{t}}_{\;\!\textcolor{Maroon}{\text{b}}\symup{z} \symup{z} \mathcolor{gray}{0}} + 2 \left( \mathcolor{gray}{\nabla_x} {\mathcal{O}}^{\;\!\mathcolor{gray}{t}}_{\;\!\textcolor{Maroon}{\text{b}}\symup{z} \symup{z} \mathcolor{gray}{\symup{x}} \mathcolor{gray}{0}} + \mathcolor{gray}{\nabla_y}  {\mathcal{O}}^{\;\!\mathcolor{gray}{t}}_{\;\!\textcolor{Maroon}{\text{b}}\symup{z} \symup{z} \mathcolor{gray}{\symup{y}} \mathcolor{gray}{0}} \right) \right] \label{eq:E_x0'} \\ &+ {\symup{\varepsilon}}_0 \mathcolor{gray}{\nabla^t} \left( \mathcolor{gray}{\nabla^t} {\mathcal{O}}^{\;\!\mathcolor{gray}{t}}_{\;\!\textcolor{Maroon}{\text{b}}\symup{x} \symup{z} \symup{z} \mathcolor{gray}{0}} - {\mathcal{L}}^{\;\!\mathcolor{gray}{t}}_{\;\!\textcolor{Maroon}{\text{m}}\symup{x} \symup{z} \symup{z} \mathcolor{gray}{0}} \right) ~, \\ E^{\;\!\mathcolor{gray}{t}}_{\;\! \symup{y}\mathcolor{gray}{0}} = &- {\symup{\varepsilon}}_0^{-1} \mathcolor{gray}{\nabla_y} \left[ {\mathcal{Q}}^{\;\!\mathcolor{gray}{t}}_{\;\!\textcolor{Maroon}{\text{b}}\symup{z} \symup{z} \mathcolor{gray}{0}} + 2 \left( \mathcolor{gray}{\nabla_x} {\mathcal{O}}^{\;\!\mathcolor{gray}{t}}_{\;\!\textcolor{Maroon}{\text{b}}\symup{z} \symup{z} \mathcolor{gray}{\symup{x}} \mathcolor{gray}{0}} + \mathcolor{gray}{\nabla_y}  {\mathcal{O}}^{\;\!\mathcolor{gray}{t}}_{\;\!\textcolor{Maroon}{\text{b}}\symup{z} \symup{z} \mathcolor{gray}{\symup{y}} \mathcolor{gray}{0}} \right) \right] \label{eq:E_y0'} \\ &+ {\symup{\varepsilon}}_0 \mathcolor{gray}{\nabla^t} \left( \mathcolor{gray}{\nabla^t} {\mathcal{O}}^{\;\!\mathcolor{gray}{t}}_{\;\!\textcolor{Maroon}{\text{b}}\symup{y} \symup{z} \symup{z} \mathcolor{gray}{0}} - {\mathcal{L}}^{\;\!\mathcolor{gray}{t}}_{\;\!\textcolor{Maroon}{\text{m}}\symup{y} \symup{z} \symup{z} \mathcolor{gray}{0}} \right) ~, \\
	E^{\;\!\mathcolor{gray}{t}}_{\;\! \symup{z}\mathcolor{gray}{0}} = &- {\symup{\varepsilon}}_0^{-1} \left[ {\mathcal{P}}^{\;\!\mathcolor{gray}{t}}_{\;\!\textcolor{Maroon}{\text{b}} \symup{z} \mathcolor{gray}{0}} + \left( \mathcolor{gray}{\nabla_x} {\mathcal{Q}}^{\;\!\mathcolor{gray}{t}}_{\;\!\textcolor{Maroon}{\text{b}} \symup{z} \mathcolor{gray}{\symup{x}} \mathcolor{gray}{0}} + \mathcolor{gray}{\nabla_y} {\mathcal{Q}}^{\;\!\mathcolor{gray}{t}}_{\;\!\textcolor{Maroon}{\text{b}} \symup{z} \mathcolor{gray}{\symup{y}} \mathcolor{gray}{0}} \right) - {\sigma}^{\;\!\mathcolor{gray}{t}}_{\;\! \mathcolor{gray}{0}} \right. \label{eq:E_z0'} \\ & \left. +\hspace{0.2em} 2 \mathcolor{gray}{\nabla_x} \mathcolor{gray}{\nabla_y} {\mathcal{O}}^{\;\!\mathcolor{gray}{t}}_{\;\!\textcolor{Maroon}{\text{b}} \mathcolor{gray}{\symup{x}} \mathcolor{gray}{\symup{y}} \symup{z} \mathcolor{gray}{0}} + \left( \mathcolor{gray}{\nabla_x^2} {\mathcal{O}}^{\;\!\mathcolor{gray}{t}}_{\;\!\textcolor{Maroon}{\text{b}} \symup{z} \mathcolor{gray}{\symup{x}} \mathcolor{gray}{\symup{x}} \mathcolor{gray}{0}} + \mathcolor{gray}{\nabla_y^2} {\mathcal{O}}^{\;\!\mathcolor{gray}{t}}_{\;\!\textcolor{Maroon}{\text{b}} \symup{z} \mathcolor{gray}{\symup{y}} \mathcolor{gray}{\symup{y}} \mathcolor{gray}{0}} \right) - \left( \mathcolor{gray}{\nabla_x^2} + \mathcolor{gray}{\nabla_y^2} \right) {\mathcal{O}}^{\;\!\mathcolor{gray}{t}}_{\;\!\textcolor{Maroon}{\text{b}} \symup{z} \symup{z} \symup{z} \mathcolor{gray}{0}} \right] ~.
\end{align}
\end{subequations}

现像 \bref{ssec:PMQN,ssec:DH-boundary} 一样,将 \bref{eq:EB^(2-1)_0,eq:B_0',eq:E_0'} 再展开至 \bref{eq:multipole} 的下一层次,直至各项全是裸\textcolor{Plum}{多极}矩 $\bar{P}^{\;\!\mathcolor{gray}{t}}_{\;\!\mathcolor{gray}{z}},\bar{\bar{Q}}^{\;\!\mathcolor{gray}{t}}_{\;\!\mathcolor{gray}{z}},\bar{\bar{\bar{O}}}^{\;\!\mathcolor{gray}{t}}_{\;\!\mathcolor{gray}{z}} ; \bar{M}^{\;\!\mathcolor{gray}{t}}_{\;\!\mathcolor{gray}{z}}, \bar{\bar{N}}^{\;\!\mathcolor{gray}{t}}_{\;\!\mathcolor{gray}{z}}$ 形式\Footnote{$B^{\;\!\mathcolor{gray}{t}}_{\;\! \symup{x}\mathcolor{gray}{0}},B^{\;\!\mathcolor{gray}{t}}_{\;\! \symup{y}\mathcolor{gray}{0}}$ 中的 $\mathcolor{gray}{\nabla_z}  {\mathcal{L}}^{\;\!\mathcolor{gray}{t}}_{\;\!\textcolor{Maroon}{\text{m}}\symup{y} \symup{z} \mathcolor{gray}{\symup{z}} \mathcolor{gray}{0}},\mathcolor{gray}{\nabla_z}  {\mathcal{L}}^{\;\!\mathcolor{gray}{t}}_{\;\!\textcolor{Maroon}{\text{m}}\symup{x} \symup{z} \mathcolor{gray}{\symup{z}} \mathcolor{gray}{0}}$ 总会被 $\mathcolor{gray}{\nabla^{\hat{3}}}  {\mathcal{L}}^{\;\!\mathcolor{gray}{t}}_{\;\!\textcolor{Maroon}{\text{m}} \mathcolor{gray}{\hat{3}} \symup{y} \symup{z} \mathcolor{gray}{0}},\mathcolor{gray}{\nabla^{\hat{3}}}  {\mathcal{L}}^{\;\!\mathcolor{gray}{t}}_{\;\!\textcolor{Maroon}{\text{m}} \mathcolor{gray}{\hat{3}} \symup{x} \symup{z} \mathcolor{gray}{0}}$ 中的某项抵消;$E^{\;\!\mathcolor{gray}{t}}_{\;\! \symup{z}\mathcolor{gray}{0}}$ 中的 ${\mathcal{P}}^{\;\!\mathcolor{gray}{t}}_{\;\!\textcolor{Maroon}{\text{b}} \symup{z} \mathcolor{gray}{0}}$ 会分裂出很多项,并分别与其后续的所有子项合并\textcolor{Plum}{同类项},且产生新项。在得到 \bref{eq:EB^(2-0)_0=} 的过程中,已经使用到了\textcolor{Plum}{多极}矩的各分量(角标)的\textcolor{Plum}{置换对称性}。}
\begin{subequations} \label{eq:EB^(2-0)_0=}
%	\belowdisplayskip=15pt
\begin{align}
	{\mathcal{E}}^{\;\!\textcolor{Maroon}{(2)}\mathcolor{gray}{t}}_{\;\! \symup{z} \mathcolor{gray}{0}} = &\hphantom{+} {\symup{\varepsilon}}_0^{-1} O^{\;\!\mathcolor{gray}{t}}_{\;\! \symup{z}\symup{z}\symup{z}\mathcolor{gray}{0}}~; \label{eq:E^(2)_z0=} \\[1.5em]
	{\mathcal{B}}^{\;\!\textcolor{Maroon}{(1)}\mathcolor{gray}{t}}_{\;\! \symup{x} \mathcolor{gray}{0}} = &- {\symup{\varepsilon}}_0 \left( \mathcolor{gray}{\nabla^t}
	O^{\;\!\mathcolor{gray}{t}}_{\;\!\symup{y}\symup{z}\symup{z}\mathcolor{gray}{0}} - N^{\;\!\mathcolor{gray}{t}}_{\;\!\symup{x}\symup{z}\mathcolor{gray}{0}} \right)~, \label{eq:B^(1)_x0=} \\
	{\mathcal{B}}^{\;\!\textcolor{Maroon}{(1)}\mathcolor{gray}{t}}_{\;\! \symup{y} \mathcolor{gray}{0}} = &\hphantom{+} {\symup{\varepsilon}}_0 \left( \mathcolor{gray}{\nabla^t}
	O^{\;\!\mathcolor{gray}{t}}_{\;\!\symup{x}\symup{z}\symup{z}\mathcolor{gray}{0}} + N^{\;\!\mathcolor{gray}{t}}_{\;\!\symup{y}\symup{z}\mathcolor{gray}{0}} \right)~; \label{eq:B^(1)_y0=} \\[1.5em]
	{\mathcal{E}}^{\;\!\textcolor{Maroon}{(1)}\mathcolor{gray}{t}}_{\;\! \symup{x} \mathcolor{gray}{0}} = &\hphantom{+} {\symup{\varepsilon}}_0^{-1} \mathcolor{gray}{\nabla_x} O^{\;\!\mathcolor{gray}{t}}_{\;\! \symup{z}\symup{z}\symup{z}\mathcolor{gray}{0}}~, \label{eq:E^(1)_x0=} \\
	{\mathcal{E}}^{\;\!\textcolor{Maroon}{(1)}\mathcolor{gray}{t}}_{\;\! \symup{y} \mathcolor{gray}{0}} = &\hphantom{+} {\symup{\varepsilon}}_0^{-1} \mathcolor{gray}{\nabla_y} O^{\;\!\mathcolor{gray}{t}}_{\;\! \symup{z}\symup{z}\symup{z}\mathcolor{gray}{0}}~, \label{eq:E^(1)_y0=} \\
	{\mathcal{E}}^{\;\!\textcolor{Maroon}{(1)}\mathcolor{gray}{t}}_{\;\! \symup{z} \mathcolor{gray}{0}} = &- {\symup{\varepsilon}}_0^{-1} \left( Q^{\;\!\mathcolor{gray}{t}}_{\;\! \symup{z} \symup{z} \mathcolor{gray}{0}} - 3~ \mathcolor{gray}{\nabla^{\hat{2}}} O^{\;\!\mathcolor{gray}{t}}_{\;\! \symup{z} \symup{z} \mathcolor{gray}{\hat{2}} \mathcolor{gray}{0}} + \mathcolor{gray}{\nabla_z} O^{\;\!\mathcolor{gray}{t}}_{\;\! \symup{z} \symup{z} \mathcolor{gray}{\symup{z}} \mathcolor{gray}{0}} \right)~; \label{eq:E^(1)_z0=} \\[1.5em]
	B^{\;\!\mathcolor{gray}{t}}_{\;\! \symup{x}\mathcolor{gray}{0}} = &\hphantom{+} {\symup{\varepsilon}}_0 \left[ \mathcolor{gray}{\nabla^t} \left( 2~ \mathcolor{gray}{\nabla_x} O^{\;\!\mathcolor{gray}{t}}_{\;\! \symup{y} \symup{z} \mathcolor{gray}{\symup{x}} \mathcolor{gray}{0}} + 2~ \mathcolor{gray}{\nabla_y} O^{\;\!\mathcolor{gray}{t}}_{\;\! \symup{y} \symup{z} \mathcolor{gray}{\symup{y}} \mathcolor{gray}{0}} + \mathcolor{gray}{\nabla_z} O^{\;\!\mathcolor{gray}{t}}_{\;\! \symup{y} \symup{z} \mathcolor{gray}{\symup{z}} \mathcolor{gray}{0}} - \mathcolor{gray}{\nabla_y} O^{\;\!\mathcolor{gray}{t}}_{\;\! \symup{z} \symup{z} \symup{z} \mathcolor{gray}{0}} \right) \right. \label{eq:B_x0=} \\ & \left. + \left( M^{\;\!\mathcolor{gray}{t}}_{\;\! \symup{x}\mathcolor{gray}{0}} - \mathcolor{gray}{\nabla^t} Q^{\;\!\mathcolor{gray}{t}}_{\;\! \symup{y} \symup{z} \mathcolor{gray}{0}} \right) +
	\left( \mathcolor{gray}{\nabla_x} N^{\;\!\mathcolor{gray}{t}}_{\;\!\symup{z} \symup{z} \mathcolor{gray}{0}} - \mathcolor{gray}{\nabla^{\hat{3}}} N^{\;\!\mathcolor{gray}{t}}_{\;\!\symup{x} \mathcolor{gray}{\hat{3}} \mathcolor{gray}{0}} \right) + 
	{\alpha}^{\;\!\mathcolor{gray}{t}}_{\;\! \symup{y}\mathcolor{gray}{0}} \right]~, \\
	B^{\;\!\mathcolor{gray}{t}}_{\;\! \symup{y}\mathcolor{gray}{0}} = &\hphantom{+} {\symup{\varepsilon}}_0 \left[ - \mathcolor{gray}{\nabla^t} \left( 2~ \mathcolor{gray}{\nabla_x} O^{\;\!\mathcolor{gray}{t}}_{\;\! \symup{x} \symup{z} \mathcolor{gray}{\symup{x}} \mathcolor{gray}{0}} + 2~ \mathcolor{gray}{\nabla_y}  O^{\;\!\mathcolor{gray}{t}}_{\;\! \symup{x} \symup{z} \mathcolor{gray}{\symup{y}} \mathcolor{gray}{0}} + \mathcolor{gray}{\nabla_z}  O^{\;\!\mathcolor{gray}{t}}_{\;\! \symup{x} \symup{z} \mathcolor{gray}{\symup{z}} \mathcolor{gray}{0}} - \mathcolor{gray}{\nabla_x}  O^{\;\!\mathcolor{gray}{t}}_{\;\! \symup{z} \symup{z} \symup{z} \mathcolor{gray}{0}} \right) \right. \label{eq:B_y0=} \\ & \left. + \left( M^{\;\!\mathcolor{gray}{t}}_{\;\! \symup{y}\mathcolor{gray}{0}} + \mathcolor{gray}{\nabla^t} Q^{\;\!\mathcolor{gray}{t}}_{\;\! \symup{x} \symup{z} \mathcolor{gray}{0}} \right) +
	\left( \mathcolor{gray}{\nabla_y} N^{\;\!\mathcolor{gray}{t}}_{\;\! \symup{z} \symup{z} \mathcolor{gray}{0}} - \mathcolor{gray}{\nabla^{\hat{3}}} N^{\;\!\mathcolor{gray}{t}}_{\;\! \symup{y} \mathcolor{gray}{\hat{3}} \mathcolor{gray}{0}} \right) - 
	{\alpha}^{\;\!\mathcolor{gray}{t}}_{\;\! \symup{x}\mathcolor{gray}{0}} \right]~, \\
	B^{\;\!\mathcolor{gray}{t}}_{\;\! \symup{z} \mathcolor{gray}{0}} = &\hphantom{+} {\symup{\varepsilon}}_0 \left[ \mathcolor{gray}{\nabla^t} \left( \mathcolor{gray}{\nabla_y}
	O^{\;\!\mathcolor{gray}{t}}_{\;\! \symup{x}\symup{z}\symup{z} \mathcolor{gray}{0}} - \mathcolor{gray}{\nabla_x}
	O^{\;\!\mathcolor{gray}{t}}_{\;\! \symup{y}\symup{z}\symup{z}\mathcolor{gray}{0}} \right) + \left( 
	N^{\;\!\mathcolor{gray}{t}}_{\;\! \mathcolor{gray}{\symup{x}} \symup{z} \mathcolor{gray}{0}} + \mathcolor{gray}{\nabla_y} N^{\;\!\mathcolor{gray}{t}}_{\;\! \mathcolor{gray}{\symup{y}} \symup{z} \mathcolor{gray}{0}} \right) \right]~; \label{eq:B_z0=} \\[1.7em]
	E^{\;\!\mathcolor{gray}{t}}_{\;\! \symup{x}\mathcolor{gray}{0}} = &\hphantom{+} {\symup{\varepsilon}}_0^{-1} \mathcolor{gray}{\nabla_x} \left[ Q^{\;\!\mathcolor{gray}{t}}_{\;\! \symup{z} \symup{z} \mathcolor{gray}{0}} - 3~ \mathcolor{gray}{\nabla_x} O^{\;\!\mathcolor{gray}{t}}_{\;\! \symup{z} \symup{z} \mathcolor{gray}{\symup{x}} \mathcolor{gray}{0}} - 3~ \mathcolor{gray}{\nabla_y}  O^{\;\!\mathcolor{gray}{t}}_{\;\! \symup{z} \symup{z} \mathcolor{gray}{\symup{y}} \mathcolor{gray}{0}} - \mathcolor{gray}{\nabla_z}  O^{\;\!\mathcolor{gray}{t}}_{\;\! \symup{z} \symup{z} \mathcolor{gray}{\symup{z}} \mathcolor{gray}{0}} \right] \label{eq:E_x0=} \\ &+ {\symup{\varepsilon}}_0 \mathcolor{gray}{\nabla^t} \left( \mathcolor{gray}{\nabla^t} O^{\;\!\mathcolor{gray}{t}}_{\;\!\symup{x} \symup{z} \symup{z} \mathcolor{gray}{0}} + N^{\;\!\mathcolor{gray}{t}}_{\;\! \symup{y} \symup{z} \mathcolor{gray}{0}} \right) ~, \\ E^{\;\!\mathcolor{gray}{t}}_{\;\! \symup{y}\mathcolor{gray}{0}} = &\hphantom{+} {\symup{\varepsilon}}_0^{-1} \mathcolor{gray}{\nabla_y} \left[ Q^{\;\!\mathcolor{gray}{t}}_{\;\! \symup{z} \symup{z} \mathcolor{gray}{0}} - 3~ \mathcolor{gray}{\nabla_x} O^{\;\!\mathcolor{gray}{t}}_{\;\! \symup{z} \symup{z} \mathcolor{gray}{\symup{x}} \mathcolor{gray}{0}} - 3~ \mathcolor{gray}{\nabla_y}  O^{\;\!\mathcolor{gray}{t}}_{\;\! \symup{z} \symup{z} \mathcolor{gray}{\symup{y}} \mathcolor{gray}{0}} - \mathcolor{gray}{\nabla_z}  O^{\;\!\mathcolor{gray}{t}}_{\;\! \symup{z} \symup{z} \mathcolor{gray}{\symup{z}} \mathcolor{gray}{0}} \right] \label{eq:E_y0=} \\ &+ {\symup{\varepsilon}}_0 \mathcolor{gray}{\nabla^t} \left( \mathcolor{gray}{\nabla^t} O^{\;\!\mathcolor{gray}{t}}_{\;\! \symup{y} \symup{z} \symup{z} \mathcolor{gray}{0}} - N^{\;\!\mathcolor{gray}{t}}_{\;\! \symup{x} \symup{z} \mathcolor{gray}{0}} \right) ~, \\
	E^{\;\!\mathcolor{gray}{t}}_{\;\! \symup{z}\mathcolor{gray}{0}} = &- {\symup{\varepsilon}}_0^{-1} \left[ P^{\;\!\mathcolor{gray}{t}}_{\;\! \symup{z} \mathcolor{gray}{0}} - 2~ \mathcolor{gray}{\nabla_x} Q^{\;\!\mathcolor{gray}{t}}_{\;\! \symup{z} \mathcolor{gray}{\symup{x}} \mathcolor{gray}{0}} - 2~ \mathcolor{gray}{\nabla_y} Q^{\;\!\mathcolor{gray}{t}}_{\;\! \symup{z} \mathcolor{gray}{\symup{y}} \mathcolor{gray}{0}} - \mathcolor{gray}{\nabla_z} Q^{\;\!\mathcolor{gray}{t}}_{\;\! \symup{z} \mathcolor{gray}{\symup{z}} \mathcolor{gray}{0}} - {\sigma}^{\;\!\mathcolor{gray}{t}}_{\;\! \mathcolor{gray}{0}} \right. \label{eq:E_z0=} \\ & + 3~ \mathcolor{gray}{\nabla_z} \left( \mathcolor{gray}{\nabla_x} O^{\;\!\mathcolor{gray}{t}}_{\;\! \symup{z} \mathcolor{gray}{\symup{z}} \mathcolor{gray}{\symup{x}} \mathcolor{gray}{0}} + \mathcolor{gray}{\nabla_y} O^{\;\!\mathcolor{gray}{t}}_{\;\! \symup{z} \mathcolor{gray}{\symup{z}} \mathcolor{gray}{\symup{y}} \mathcolor{gray}{0}} \right) + 6~ \mathcolor{gray}{\nabla_x} \mathcolor{gray}{\nabla_y} O^{\;\!\mathcolor{gray}{t}}_{\;\! \mathcolor{gray}{\symup{x}} \mathcolor{gray}{\symup{y}} \symup{z} \mathcolor{gray}{0}} \\ & \left. + 3 \left( \mathcolor{gray}{\nabla_x^2} O^{\;\!\mathcolor{gray}{t}}_{\;\! \symup{z} \mathcolor{gray}{\symup{x}} \mathcolor{gray}{\symup{x}} \mathcolor{gray}{0}} + \mathcolor{gray}{\nabla_y^2} O^{\;\!\mathcolor{gray}{t}}_{\;\! \symup{z} \mathcolor{gray}{\symup{y}} \mathcolor{gray}{\symup{y}} \mathcolor{gray}{0}} \right) + \left( \mathcolor{gray}{\nabla_z^2} - \mathcolor{gray}{\nabla_x^2} - \mathcolor{gray}{\nabla_y^2} \right) O^{\;\!\mathcolor{gray}{t}}_{\;\! \symup{z} \mathcolor{gray}{\symup{z}} \mathcolor{gray}{\symup{z}} \mathcolor{gray}{0}} \right] ~.
\end{align}
\end{subequations}
上述 \bref{eq:EB^(2-0)_0=} 与文献\cite{delangeElectromagneticBoundaryConditions2013}的主要结果的绝大部分一致,除了 \bref{eq:E^(2)_z0=,eq:B^(1)_x0=,eq:B^(1)_y0=,eq:E^(1)_x0=,eq:E^(1)_y0=} 反号。这些分量异号的原因,主要应归结为:本文对 \bref{eq:e-b-01'} 的定义与文献\cite{delangeElectromagneticBoundaryConditions2013}相同,但本文对 \bref{eq:e-b-01',eq:e-f-01',eq:EB-01} 的定义却均与文献\cite{delangeElectromagneticBoundaryConditions2013}不同。此外,这里暂不涉及 2 种介质,而是单侧半无限介质,另一侧真空。

再利用\textcolor{Plum}{多极}矩 $\bar{P}^{\;\!\mathcolor{gray}{t}}_{\;\!\mathcolor{gray}{z}},\bar{\bar{Q}}^{\;\!\mathcolor{gray}{t}}_{\;\!\mathcolor{gray}{z}},\bar{\bar{\bar{O}}}^{\;\!\mathcolor{gray}{t}}_{\;\!\mathcolor{gray}{z}} ; \bar{M}^{\;\!\mathcolor{gray}{t}}_{\;\!\mathcolor{gray}{z}}, \bar{\bar{N}}^{\;\!\mathcolor{gray}{t}}_{\;\!\mathcolor{gray}{z}}$ 的(角标中)各分量的\textcolor{Plum}{置换对称性},将上述 \bref{eq:EB^(2-0)_0=} 各张量的分量以从左到右 xyz 排列,且重排各项至电磁场匹配(电/磁的首项总对应电/磁项),并将整体分组为电磁两组。

对于电场,其(在宏观阶跃边界 $\mathcolor{gray}{z \mathcolor{black}{=} 0}$ 附近的)非零分量包含
\begin{subequations} \label{eq:E^(2-0)_0==}
\begin{align}
	{\mathcal{E}}^{\;\!\textcolor{Maroon}{(2)}\mathcolor{gray}{t}}_{\;\! \symup{z} \mathcolor{gray}{0}} = &\hphantom{+} {\symup{\varepsilon}}_0^{-1} O^{\;\!\mathcolor{gray}{t}}_{\;\! \symup{z}\symup{z}\symup{z}\mathcolor{gray}{0}}~; \label{eq:E^(2)_z0==} \\[0.7em]
	{\mathcal{E}}^{\;\!\textcolor{Maroon}{(1)}\mathcolor{gray}{t}}_{\;\! \symup{x} \mathcolor{gray}{0}} = &\hphantom{+} {\symup{\varepsilon}}_0^{-1} \mathcolor{gray}{\nabla_x} O^{\;\!\mathcolor{gray}{t}}_{\;\! \symup{z}\symup{z}\symup{z}\mathcolor{gray}{0}}~, \label{eq:E^(1)_x0==} \\
	{\mathcal{E}}^{\;\!\textcolor{Maroon}{(1)}\mathcolor{gray}{t}}_{\;\! \symup{y} \mathcolor{gray}{0}} = &\hphantom{+} {\symup{\varepsilon}}_0^{-1} \mathcolor{gray}{\nabla_y} O^{\;\!\mathcolor{gray}{t}}_{\;\! \symup{z}\symup{z}\symup{z}\mathcolor{gray}{0}}~, \label{eq:E^(1)_y0==} \\
	{\mathcal{E}}^{\;\!\textcolor{Maroon}{(1)}\mathcolor{gray}{t}}_{\;\! \symup{z} \mathcolor{gray}{0}} = &- {\symup{\varepsilon}}_0^{-1} \left( Q^{\;\!\mathcolor{gray}{t}}_{\;\! \symup{z} \symup{z} \mathcolor{gray}{0}} - 3 \mathcolor{gray}{\nabla^\iota} O^{\;\!\mathcolor{gray}{t}}_{\;\! \mathcolor{gray}{\symup{\iota}} \symup{z} \symup{z} \mathcolor{gray}{0}} + \mathcolor{gray}{\nabla_z} O^{\;\!\mathcolor{gray}{t}}_{\;\! \mathcolor{gray}{\symup{z}} \symup{z} \symup{z} \mathcolor{gray}{0}} \right)~; \label{eq:E^(1)_z0==} \\[0.7em]
	E^{\;\!\mathcolor{gray}{t}}_{\;\! \symup{x}\mathcolor{gray}{0}} = &\hphantom{+} {\symup{\varepsilon}}_0^{-1} \mathcolor{gray}{\nabla_x} \left[ Q^{\;\!\mathcolor{gray}{t}}_{\;\! \symup{z} \symup{z} \mathcolor{gray}{0}} - 3~ \mathcolor{gray}{\nabla_x} O^{\;\!\mathcolor{gray}{t}}_{\;\! \mathcolor{gray}{\symup{x}} \symup{z} \symup{z} \mathcolor{gray}{0}} - 3~ \mathcolor{gray}{\nabla_y}  O^{\;\!\mathcolor{gray}{t}}_{\;\! \mathcolor{gray}{\symup{y}} \symup{z} \symup{z} \mathcolor{gray}{0}} - \mathcolor{gray}{\nabla_z}  O^{\;\!\mathcolor{gray}{t}}_{\;\! \mathcolor{gray}{\symup{z}} \symup{z} \symup{z} \mathcolor{gray}{0}} \right] \label{eq:E_x0==} \\ &+ {\symup{\varepsilon}}_0 \mathcolor{gray}{\nabla^t} \left( \mathcolor{gray}{\nabla^t} O^{\;\!\mathcolor{gray}{t}}_{\;\!\symup{x} \symup{z} \symup{z} \mathcolor{gray}{0}} + N^{\;\!\mathcolor{gray}{t}}_{\;\! \symup{y} \symup{z} \mathcolor{gray}{0}} \right) ~, \\ E^{\;\!\mathcolor{gray}{t}}_{\;\! \symup{y}\mathcolor{gray}{0}} = &\hphantom{+} {\symup{\varepsilon}}_0^{-1} \mathcolor{gray}{\nabla_y} \left[ Q^{\;\!\mathcolor{gray}{t}}_{\;\! \symup{z} \symup{z} \mathcolor{gray}{0}} - 3~ \mathcolor{gray}{\nabla_x} O^{\;\!\mathcolor{gray}{t}}_{\;\! \mathcolor{gray}{\symup{x}} \symup{z} \symup{z} \mathcolor{gray}{0}} - 3~ \mathcolor{gray}{\nabla_y}  O^{\;\!\mathcolor{gray}{t}}_{\;\! \mathcolor{gray}{\symup{y}} \symup{z} \symup{z} \mathcolor{gray}{0}} - \mathcolor{gray}{\nabla_z}  O^{\;\!\mathcolor{gray}{t}}_{\;\! \mathcolor{gray}{\symup{z}} \symup{z} \symup{z} \mathcolor{gray}{0}} \right] \label{eq:E_y0==} \\ &+ {\symup{\varepsilon}}_0 \mathcolor{gray}{\nabla^t} \left( \mathcolor{gray}{\nabla^t} O^{\;\!\mathcolor{gray}{t}}_{\;\! \symup{y} \symup{z} \symup{z} \mathcolor{gray}{0}} - N^{\;\!\mathcolor{gray}{t}}_{\;\! \symup{x} \symup{z} \mathcolor{gray}{0}} \right) ~, \\
	E^{\;\!\mathcolor{gray}{t}}_{\;\! \symup{z} \mathcolor{gray}{0}} = &- {\symup{\varepsilon}}_0^{-1} \left[ \left( P^{\;\!\mathcolor{gray}{t}}_{\;\! \symup{z} \mathcolor{gray}{0}} - {\sigma}^{\;\!\mathcolor{gray}{t}}_{\;\! \mathcolor{gray}{0}} \right) - 2~ \mathcolor{gray}{\nabla_x} Q^{\;\!\mathcolor{gray}{t}}_{\;\! \mathcolor{gray}{\symup{x}} \symup{z} \mathcolor{gray}{0}} - 2~ \mathcolor{gray}{\nabla_y} Q^{\;\!\mathcolor{gray}{t}}_{\;\! \mathcolor{gray}{\symup{y}} \symup{z} \mathcolor{gray}{0}} - \mathcolor{gray}{\nabla_z} Q^{\;\!\mathcolor{gray}{t}}_{\;\! \mathcolor{gray}{\symup{z}} \symup{z} \mathcolor{gray}{0}} \right. \label{eq:E_z0==} \\ & + 3~ \mathcolor{gray}{\nabla_z} \left( \mathcolor{gray}{\nabla_x} O^{\;\!\mathcolor{gray}{t}}_{\;\! \mathcolor{gray}{\symup{x}} \mathcolor{gray}{\symup{z}} \symup{z} \mathcolor{gray}{0}} + \mathcolor{gray}{\nabla_y} O^{\;\!\mathcolor{gray}{t}}_{\;\! \mathcolor{gray}{\symup{y}} \mathcolor{gray}{\symup{z}} \symup{z} \mathcolor{gray}{0}} \right) + 6~ \mathcolor{gray}{\nabla_x} \mathcolor{gray}{\nabla_y} O^{\;\!\mathcolor{gray}{t}}_{\;\! \mathcolor{gray}{\symup{x}} \mathcolor{gray}{\symup{y}} \symup{z} \mathcolor{gray}{0}} \\ & \left. + 3 \left( \mathcolor{gray}{\nabla_x^2} O^{\;\!\mathcolor{gray}{t}}_{\;\! \mathcolor{gray}{\symup{x}} \mathcolor{gray}{\symup{x}} \symup{z} \mathcolor{gray}{0}} + \mathcolor{gray}{\nabla_y^2} O^{\;\!\mathcolor{gray}{t}}_{\;\! \mathcolor{gray}{\symup{y}} \mathcolor{gray}{\symup{y}} \symup{z} \mathcolor{gray}{0}} \right) + \left( \mathcolor{gray}{\nabla_z^2} - \mathcolor{gray}{\nabla_x^2} - \mathcolor{gray}{\nabla_y^2} \right) O^{\;\!\mathcolor{gray}{t}}_{\;\! \mathcolor{gray}{\symup{z}} \mathcolor{gray}{\symup{z}} \symup{z} \mathcolor{gray}{0}} \right] ~.
\end{align}
\end{subequations}
对于磁感应场,其(在宏观阶跃边界 $\mathcolor{gray}{z \mathcolor{black}{=} 0}$ 附近的)非零分量为
\begin{subequations} \label{eq:B^(2-0)_0==}
\begin{align}
	{\mathcal{B}}^{\;\!\textcolor{Maroon}{(1)}\mathcolor{gray}{t}}_{\;\! \symup{x} \mathcolor{gray}{0}} = &\hphantom{+} {\symup{\varepsilon}}_0 \left( N^{\;\!\mathcolor{gray}{t}}_{\;\!\symup{x}\symup{z}\mathcolor{gray}{0}} - \mathcolor{gray}{\nabla^t}
	O^{\;\!\mathcolor{gray}{t}}_{\;\!\symup{y}\symup{z}\symup{z}\mathcolor{gray}{0}} \right)~, \label{eq:B^(1)_x0==} \\
	{\mathcal{B}}^{\;\!\textcolor{Maroon}{(1)}\mathcolor{gray}{t}}_{\;\! \symup{y} \mathcolor{gray}{0}} = &\hphantom{+} {\symup{\varepsilon}}_0 \left( N^{\;\!\mathcolor{gray}{t}}_{\;\!\symup{y}\symup{z}\mathcolor{gray}{0}} + \mathcolor{gray}{\nabla^t}
	O^{\;\!\mathcolor{gray}{t}}_{\;\!\symup{x}\symup{z}\symup{z}\mathcolor{gray}{0}} \right)~; \label{eq:B^(1)_y0==} \\[1.0em]
	B^{\;\!\mathcolor{gray}{t}}_{\;\! \symup{x}\mathcolor{gray}{0}} = &\hphantom{+} {\symup{\varepsilon}}_0 \left[ M^{\;\!\mathcolor{gray}{t}}_{\;\! \symup{x}\mathcolor{gray}{0}} - \left( \mathcolor{gray}{\nabla^t} Q^{\;\!\mathcolor{gray}{t}}_{\;\! \symup{y} \symup{z} \mathcolor{gray}{0}} - 
	{\alpha}^{\;\!\mathcolor{gray}{t}}_{\;\! \symup{y}\mathcolor{gray}{0}} \right) +
	\left( \mathcolor{gray}{\nabla_x} N^{\;\!\mathcolor{gray}{t}}_{\;\!\symup{z} \symup{z} \mathcolor{gray}{0}} - \mathcolor{gray}{\nabla^\iota} N^{\;\!\mathcolor{gray}{t}}_{\;\! \mathcolor{gray}{\symup{\iota}} \symup{x} \mathcolor{gray}{0}} \right) \right. \label{eq:B_x0==} \\ & \left. + \mathcolor{gray}{\nabla^t} \left( 2~ \mathcolor{gray}{\nabla_x} O^{\;\!\mathcolor{gray}{t}}_{\;\! \mathcolor{gray}{\symup{x}} \symup{y} \symup{z} \mathcolor{gray}{0}} + 2~ \mathcolor{gray}{\nabla_y} O^{\;\!\mathcolor{gray}{t}}_{\;\! \mathcolor{gray}{\symup{y}} \symup{y} \symup{z} \mathcolor{gray}{0}} + \mathcolor{gray}{\nabla_z} O^{\;\!\mathcolor{gray}{t}}_{\;\! \symup{y} \symup{z} \mathcolor{gray}{\symup{z}} \mathcolor{gray}{0}} - \mathcolor{gray}{\nabla_y} O^{\;\!\mathcolor{gray}{t}}_{\;\! \symup{z} \symup{z} \symup{z} \mathcolor{gray}{0}} \right) \right]~, \\
	B^{\;\!\mathcolor{gray}{t}}_{\;\! \symup{y}\mathcolor{gray}{0}} = &\hphantom{+} {\symup{\varepsilon}}_0 \left[ M^{\;\!\mathcolor{gray}{t}}_{\;\! \symup{y}\mathcolor{gray}{0}} + \left( \mathcolor{gray}{\nabla^t} Q^{\;\!\mathcolor{gray}{t}}_{\;\! \symup{x} \symup{z} \mathcolor{gray}{0}} -
	{\alpha}^{\;\!\mathcolor{gray}{t}}_{\;\! \symup{x}\mathcolor{gray}{0}} \right) +
	\left( \mathcolor{gray}{\nabla_y} N^{\;\!\mathcolor{gray}{t}}_{\;\! \symup{z} \symup{z} \mathcolor{gray}{0}} - \mathcolor{gray}{\nabla^\iota} N^{\;\!\mathcolor{gray}{t}}_{\;\! \mathcolor{gray}{\symup{\iota}} \symup{y} \mathcolor{gray}{0}} \right) \right. \label{eq:B_y0==} \\ & \left. - \mathcolor{gray}{\nabla^t} \left( 2~ \mathcolor{gray}{\nabla_x} O^{\;\!\mathcolor{gray}{t}}_{\;\! \symup{x} \mathcolor{gray}{\symup{x}} \symup{z} \mathcolor{gray}{0}} + 2~ \mathcolor{gray}{\nabla_y}  O^{\;\!\mathcolor{gray}{t}}_{\;\! \symup{x} \mathcolor{gray}{\symup{y}} \symup{z} \mathcolor{gray}{0}} + \mathcolor{gray}{\nabla_z}  O^{\;\!\mathcolor{gray}{t}}_{\;\! \symup{x} \symup{z} \mathcolor{gray}{\symup{z}} \mathcolor{gray}{0}} - \mathcolor{gray}{\nabla_x}  O^{\;\!\mathcolor{gray}{t}}_{\;\! \symup{z} \symup{z} \symup{z} \mathcolor{gray}{0}} \right) \right]~, \\
	B^{\;\!\mathcolor{gray}{t}}_{\;\! \symup{z} \mathcolor{gray}{0}} = &\hphantom{+} {\symup{\varepsilon}}_0 \left[ \left( \mathcolor{gray}{\nabla_x}
	N^{\;\!\mathcolor{gray}{t}}_{\;\! \mathcolor{gray}{\symup{x}} \symup{z} \mathcolor{gray}{0}} + \mathcolor{gray}{\nabla_y} N^{\;\!\mathcolor{gray}{t}}_{\;\! \mathcolor{gray}{\symup{y}} \symup{z} \mathcolor{gray}{0}} \right) + \mathcolor{gray}{\nabla^t} \left( \mathcolor{gray}{\nabla_y}
	O^{\;\!\mathcolor{gray}{t}}_{\;\! \symup{x}\symup{z}\symup{z} \mathcolor{gray}{0}} - \mathcolor{gray}{\nabla_x}
	O^{\;\!\mathcolor{gray}{t}}_{\;\! \symup{y}\symup{z}\symup{z}\mathcolor{gray}{0}} \right) \right]~. \label{eq:B_z0==}
\end{align}
\end{subequations}
注意,上述 \bref{eq:E^(2-0)_0==} 与 \bref{eq:B^(2-0)_0==} 并不是对\textcolor{Maroon}{边界条件}的完整描述。二者是通过直接沿用 \bref{eq:e-b-01'} $\to$ \bref{eq:div-e-b-01-deltas} 的方法得到的  ---  然而原则上不能这样做:因为(电磁)源可以因其所在的一侧是半无限真空而为零,但(电磁)场不会因真空而消失:真空中也允许(电磁)场的存在。因此,原则上从 \bref{eq:curl-EB-01-deltas} 处开始,所有公式中的每一个源/场量 $X$,都需要继承并添补上 \bref{eq:curl-EB-01} 中的左上角介质标记 $\textcolor{Maroon}{\mathsfit{z}}$ 并对两侧半无限介质(在接触面 $\mathcolor{gray}{z \mathcolor{black}{=} 0}$ 处)乘以 $\leftindex_{\textcolor{Maroon}{\mathsfit{z}}} \;\! \delta_{\mathcolor{gray}{z}}$(或其他对应的 $\leftindex_{\textcolor{Maroon}{\mathsfit{z}}} \;\! \delta'_{\mathcolor{gray}{z}},\leftindex_{\textcolor{Maroon}{\mathsfit{z}}} \;\! \delta''_{\mathcolor{gray}{z}}$)并求和以成为 $\leftindex_{\textcolor{Maroon}{\mathsfit{z}}} \;\! \delta_{\mathcolor{gray}{z}} \leftindex^{\textcolor{Maroon}{\mathsfit{z}}} X$(或 $\leftindex_{\textcolor{Maroon}{\mathsfit{z}}} \;\! \delta'_{\mathcolor{gray}{z}} \leftindex^{\textcolor{Maroon}{\mathsfit{z}}} X,\leftindex_{\textcolor{Maroon}{\mathsfit{z}}} \;\! \delta''_{\mathcolor{gray}{z}} \leftindex^{\textcolor{Maroon}{\mathsfit{z}}} X$)。 ---  这一点从 \bref{eq:E_x0==,eq:E_y0==,eq:E_z0==} 和 \bref{eq:B_x0==,eq:B_y0==,eq:B_z0==} 如果不加上左上角介质标记 $\textcolor{Maroon}{\mathsfit{z}}$,则与 \bref{eq:EB-01} 冲突\Footnote{具有相同符号,但含义却不同。多对一的满射是不允许的。}的角度,也能看出。

总之,须对两侧介质 $\textcolor{Maroon}{\mathsfit{z}} = \textcolor{Maroon}{0,1}$ 中的 \bref{eq:E^(2-0)_0==} 与 \bref{eq:B^(2-0)_0==} 乘以 $\leftindex_{\textcolor{Maroon}{\mathsfit{z}}} \;\! \delta_{\mathcolor{gray}{z}}$(或 $\leftindex_{\textcolor{Maroon}{\mathsfit{z}}} \;\! \delta'_{\mathcolor{gray}{z}},\leftindex_{\textcolor{Maroon}{\mathsfit{z}}} \;\! \delta''_{\mathcolor{gray}{z}}$)并求和,即将其形式还原为 \bref{eq:curl-EB-01} 后,才是完整的\textcolor{Maroon}{边界条件}。以代表 \bref{eq:E^(2-0)_0==} 和 \bref{eq:B^(2-0)_0==} 的场=源多项式,即 $Y^{\;\!\mathcolor{gray}{t}}_{\mathcolor{gray}{0}} = \leftindex_{\textcolor{Maroon}{i}} \;\! X^{\;\!\textcolor{Maroon}{(i)}\mathcolor{gray}{t}}_{\mathcolor{gray}{0}}$ 为例,它须升级为
\begin{subequations} \label{eq:2BC}
\begin{align}
	\leftindex_{\textcolor{Maroon}{\mathsfit{z}}} \;\! n_{\mathcolor{gray}{z}} \leftindex^{\textcolor{Maroon}{\mathsfit{z}}} \;\! Y^{\;\!\mathcolor{gray}{t}}_{\mathcolor{gray}{0}} &= \leftindex_{\textcolor{Maroon}{\mathsfit{z}}} \;\! n_{\mathcolor{gray}{z}} \leftindex^{\textcolor{Maroon}{\mathsfit{z}}}_{\textcolor{Maroon}{i}} \;\! X^{\;\!\textcolor{Maroon}{(i)}\mathcolor{gray}{t}}_{\mathcolor{gray}{0}} \label{eq:^zY_0} \\ \leftindex^{\textcolor{Maroon}{\mathsfit{0}}} \;\! Y^{\;\!\mathcolor{gray}{t}}_{\mathcolor{gray}{0}} - \leftindex^{\textcolor{Maroon}{\mathsfit{1}}} \;\! Y^{\;\!\mathcolor{gray}{t}}_{\mathcolor{gray}{0}} &= \leftindex^{\textcolor{Maroon}{\mathsfit{0}}}_{\textcolor{Maroon}{i}} \;\! X^{\;\!\textcolor{Maroon}{(i)}\mathcolor{gray}{t}}_{\mathcolor{gray}{0}} - \leftindex^{\textcolor{Maroon}{\mathsfit{1}}}_{\textcolor{Maroon}{i}} \;\! X^{\;\!\textcolor{Maroon}{(i)}\mathcolor{gray}{t}}_{\mathcolor{gray}{0}} \label{eq:^0Y_0-^1Y_0} \\ \leftindex^{\textcolor{Maroon}{\mathsfit{1}}} \;\! Y^{\;\!\mathcolor{gray}{t}}_{\mathcolor{gray}{0}} &= \leftindex^{\textcolor{Maroon}{\mathsfit{0}}} \;\! Y^{\;\!\mathcolor{gray}{t}}_{\mathcolor{gray}{0}} + \left( \leftindex^{\textcolor{Maroon}{\mathsfit{1}}}_{\textcolor{Maroon}{i}} \;\! X^{\;\!\textcolor{Maroon}{(i)}\mathcolor{gray}{t}}_{\mathcolor{gray}{0}} - \leftindex^{\textcolor{Maroon}{\mathsfit{0}}}_{\textcolor{Maroon}{i}} \;\! X^{\;\!\textcolor{Maroon}{(i)}\mathcolor{gray}{t}}_{\mathcolor{gray}{0}} \right) ~, \label{eq:^1Y_0}
\end{align}
\end{subequations}
有了上述双侧半无限介质的\textcolor{Maroon}{边界条件} \bref{eq:^1Y_0} 之后,才能谈论单侧半无限介质的\textcolor{Maroon}{边界条件}。当入射侧介质 \textcolor{Maroon}{0} 为真空时,(电磁)源$\leftindex^{\textcolor{Maroon}{\mathsfit{0}}}_{\textcolor{Maroon}{i}} \;\! X^{\;\!\textcolor{Maroon}{(i)}\mathcolor{gray}{t}}_{\mathcolor{gray}{0}} = 0$ 但(电磁)场$\leftindex^{\textcolor{Maroon}{\mathsfit{0}}} \;\! Y^{\;\!\mathcolor{gray}{t}}_{\mathcolor{gray}{0}} \neq 0$,于是 \bref{eq:^1Y_0} 变为
\begin{align} \label{eq:1BC}
	\leftindex^{\textcolor{Maroon}{\mathsfit{1}}} \;\! Y^{\;\!\mathcolor{gray}{t}}_{\mathcolor{gray}{0}} = \leftindex^{\textcolor{Maroon}{\mathsfit{0}}} \;\! Y^{\;\!\mathcolor{gray}{t}}_{\mathcolor{gray}{0}} + \leftindex^{\textcolor{Maroon}{\mathsfit{1}}}_{\textcolor{Maroon}{i}} \;\! X^{\;\!\textcolor{Maroon}{(i)}\mathcolor{gray}{t}}_{\mathcolor{gray}{0}} ~.
\end{align}

上述 \bref{eq:1BC} 或 \bref{eq:2BC},结合 \bref{eq:E^(2-0)_0==} 和 \bref{eq:B^(2-0)_0==},给出了以下深刻结论:{\one} 由于 \bref{eq:E_x0==} 和 \bref{eq:E_y0==} 中的第一项,当展开至\textcolor{NavyBlue}{电四-磁偶}极矩 $\bar{\bar{Q}}^{\;\!\mathcolor{gray}{t}}_{\;\!\mathcolor{gray}{z}}, \bar{M}^{\;\!\mathcolor{gray}{t}}_{\;\!\mathcolor{gray}{z}}$ 层次及以下时,\textcolor{NavyBlue}{基本场}之一的电场 $\bar{E}^{\;\!\mathcolor{gray}{t}}_{\;\!\mathcolor{gray}{z}}$ 的切向分量 $\bar{E}^{\;\!\mathcolor{gray}{t}}_{\;\! \symup{\rho} \mathcolor{gray}{z}} := \begin{pmatrix} E^{\;\!\mathcolor{gray}{t}}_{\;\! \symup{x} \mathcolor{gray}{z}}, & E^{\;\!\mathcolor{gray}{t}}_{\;\! \symup{y} \mathcolor{gray}{z}} \end{pmatrix}^{\mathsf{\textcolor{Plum}{T}}}$(在接触面 $\mathcolor{gray}{z \mathcolor{black}{=} 0}$ 的左右两侧介质 $\textcolor{Maroon}{\mathsfit{z}} = \textcolor{Maroon}{0,1}$ 中)不\textcolor{Plum}{连续},即 $\leftindex^{\textcolor{Maroon}{\mathsfit{1}}} {\bar{E}}^{\;\!\mathcolor{gray}{t}}_{\;\! \symup{\rho} \mathcolor{gray}{0}} \neq \leftindex^{\textcolor{Maroon}{\mathsfit{0}}} {\bar{E}}^{\;\!\mathcolor{gray}{t}}_{\;\! \symup{\rho} \mathcolor{gray}{0}}$;{\two} 从 \bref{eq:B_z0==} 可见,在\textcolor{NavyBlue}{电八-磁四}极矩 $\bar{\bar{\bar{O}}}^{\;\!\mathcolor{gray}{t}}_{\;\!\mathcolor{gray}{z}}, \bar{\bar{N}}^{\;\!\mathcolor{gray}{t}}_{\;\!\mathcolor{gray}{z}}$ 层次及以下,\textcolor{NavyBlue}{基本场}之一的磁感应场 $\bar{B}^{\;\!\mathcolor{gray}{t}}_{\;\!\mathcolor{gray}{z}}$ 的法向分量 $B^{\;\!\mathcolor{gray}{t}}_{\;\! \symup{z} \mathcolor{gray}{z}}$(在 $\mathcolor{gray}{z \mathcolor{black}{=} 0^+}$ 和 $\mathcolor{gray}{z \mathcolor{black}{=} 0^-}$ 处的值)不\textcolor{Plum}{连续},即 $B^{\;\!\mathcolor{gray}{t}}_{\;\! \symup{z} \mathcolor{gray}{0^+}} \neq B^{\;\!\mathcolor{gray}{t}}_{\;\! \symup{z} \mathcolor{gray}{0^-}}$。 ---  注意,原则上这里的 $\bar{E}^{\;\!\mathcolor{gray}{t}}_{\;\! \symup{\rho} \mathcolor{gray}{z}},B^{\;\!\mathcolor{gray}{t}}_{\;\! \symup{z} \mathcolor{gray}{z}}$ 也指 \bref{eq:EB-01} 等号右侧多项式中的第一项,而不是等号左侧的总场,因此也需要对它们添加左上角介质标记 $\textcolor{Maroon}{\mathsfit{z}}$。然而由于这里已经处于两侧介质内部,而不是恰好在接触面上,那么一旦 $\mathcolor{gray}{z} \neq \mathcolor{gray}{0}$,则这里的 $\bar{E}^{\;\!\mathcolor{gray}{t}}_{\;\! \symup{\rho} \mathcolor{gray}{z}},B^{\;\!\mathcolor{gray}{t}}_{\;\! \symup{z} \mathcolor{gray}{z}}$ 直接指代总场 \bref{eq:EB-01} 也没问题,因为此时二者相等。

值得注意的是,该 \bref{ssec:EB-boundary} 从 \bref{eq:curl-EB-01-deltas} 开始直到此处,只给出了 \bref{eq:EB-01} 中等号右侧各多项式中的场量在接触面 $\mathcolor{gray}{z \mathcolor{black}{=} 0}$ 附近的关于源的\textcolor{Plum}{多极}展开式和\textcolor{Plum}{连续}性关系。在深入材料内部的其他 $\mathcolor{gray}{z}$ 两侧的\textcolor{NavyBlue}{基本场} $\bar{E}^{\;\!\mathcolor{gray}{t}}_{\;\!\mathcolor{gray}{z}}, \bar{B}^{\;\!\mathcolor{gray}{t}}_{\;\!\mathcolor{gray}{z}}$ 的\textcolor{Plum}{连续}性关系,也可以套用该理论:只需将那里的 $\mathcolor{gray}{z}$ 视为新的 $\mathcolor{gray}{0}$ 即可。一般而言,在源 $\bar{P}^{\;\!\mathcolor{gray}{t}}_{\;\!\mathcolor{gray}{z}},\bar{\bar{Q}}^{\;\!\mathcolor{gray}{t}}_{\;\!\mathcolor{gray}{z}},\bar{\bar{\bar{O}}}^{\;\!\mathcolor{gray}{t}}_{\;\!\mathcolor{gray}{z}} ; \bar{M}^{\;\!\mathcolor{gray}{t}}_{\;\!\mathcolor{gray}{z}}, \bar{\bar{N}}^{\;\!\mathcolor{gray}{t}}_{\;\!\mathcolor{gray}{z}};{\rho}^{\;\!\mathcolor{gray}{t}}_{\;\!\textcolor{Maroon}{\text{f}}\mathcolor{gray}{z}}, \bar{J}^{\;\!\mathcolor{gray}{t}}_{\;\!\textcolor{Maroon}{\text{f}}\mathcolor{gray}{z}}$ \textcolor{Plum}{连续}性分布的材料内部,即使材料/源(的 $\mathcolor{gray}{\bar{r}}$ 分布)不是\textcolor{Plum}{均匀}的,场 $\bar{E}^{\;\!\mathcolor{gray}{t}}_{\;\!\mathcolor{gray}{z}}, \bar{B}^{\;\!\mathcolor{gray}{t}}_{\;\!\mathcolor{gray}{z}}$(的分布)也是\textcolor{Plum}{连续}的。然而反过来,即使材料/源是\textcolor{Plum}{均匀}(分布)的,场 $\bar{E}^{\;\!\mathcolor{gray}{t}}_{\;\!\mathcolor{gray}{z}}, \bar{B}^{\;\!\mathcolor{gray}{t}}_{\;\!\mathcolor{gray}{z}}$ 也不是\textcolor{Plum}{均匀}(分布)的。

\bref{ssec:step-delta,ssec:EB-boundary,ssec:DH-boundary} 的理论用于解决宏观上突变的\textcolor{Maroon}{边界条件}问题。以至于该理论可以提供 $\mathcolor{gray}{z \mathcolor{black}{=} 0}$ 处\textcolor{NavyBlue}{基本场} $\bar{E}^{\;\!\mathcolor{gray}{t}}_{\;\!\mathcolor{gray}{z}}, \bar{B}^{\;\!\mathcolor{gray}{t}}_{\;\!\mathcolor{gray}{z}}$ 的\textcolor{Plum}{连续}性关系和(场关于源的)值,以及深入材料内部的其他 $\mathcolor{gray}{z}$ 处的\textcolor{Plum}{连续}性关系,但无法给出在其他 $\mathcolor{gray}{z}$ 处的值。材料内部\textcolor{NavyBlue}{基本场} $\bar{E}^{\;\!\mathcolor{gray}{t}}_{\;\!\mathcolor{gray}{z}}, \bar{B}^{\;\!\mathcolor{gray}{t}}_{\;\!\mathcolor{gray}{z}}$ 的值及其动力学过程,需要进一步知晓\textcolor{Maroon}{本构关系}的显示表达式(即材料/源关于场的响应函数)\cite{raabMultipoleTheoryElectromagnetism2004},并耦合之以联立求解有限体积内\textcolor{Plum}{连续}分布\cite{landauCHAPTERXIELECTROMAGNETIC1984}的 \textcolor{Maroon}{Maxwell-Lorentz-Heaviside} \bref{eq:curl-E,eq:div-B,eq:div-E,eq:curl-B}。

\clearpage
\vspace*{-8.0em}

\marginLeft[-2.4em]{ssec:DH-boundary}\subsection{$\bar{D},\bar{H}$ 辅助场、$\bar{D},\bar{H}$ 非唯一性}\label{ssec:DH-boundary}

当 \bref{ssec:PMQN} 的束缚源 \bref{eq:p-b,eq:j-b}、自由电源 \bref{eq:div-e-f},被 \bref{ssec:step-delta} 升级为 \bref{eq:e-b-01',eq:e-f-01'} 的同时,\bref{ssec:EBpJ} 的\textcolor{NavyBlue}{基本场} $\bar{E}^{\;\!\mathcolor{gray}{t}}_{\;\!\mathcolor{gray}{z}}, \bar{B}^{\;\!\mathcolor{gray}{t}}_{\;\!\mathcolor{gray}{z}}$ 也被上面 \bref{ssec:EB-boundary} 升级为 \bref{eq:EB-01}。此时,\bref{ssec:EHpJf} 中的 \textcolor{Maroon}{Maxwell-Lorentz-Heaviside} \bref{eq:curl-EK,eq:div-Bk,eq:div-D,eq:curl-H} 升级为 \bref{eq:curl-EB} 的类似物
\begin{subequations} \label{eq:curl-EH}
\begin{align}
	\epsilon^{\hphantom{\symup{\iota}\hat{1}}\hat{2}}_{\symup{\iota}\mathcolor{gray}{\hat{1}}} \mathcolor{gray}{\nabla^{\hat{1}}} E^{\;\!\mathcolor{gray}{t}}_{\;\! \hat{2}\mathcolor{gray}{z}} + \mathcolor{gray}{\nabla^t} B^{\;\!\mathcolor{gray}{t}}_{\;\! \symup{\iota}\mathcolor{gray}{z}} &= 0~, \label{eq:curl-E'duplicate} \\
	\epsilon^{\hphantom{\symup{\iota}\hat{1}}\hat{2}}_{\symup{\iota}\mathcolor{gray}{\hat{1}}} \mathcolor{gray}{\nabla^{\hat{1}}} H^{\;\!\mathcolor{gray}{t}}_{\;\! \hat{2}\mathcolor{gray}{z}} - \mathcolor{gray}{\nabla^t} D^{\;\!\mathcolor{gray}{t}}_{\;\! \symup{\iota}\mathcolor{gray}{z}} &= \leftindex_{\textcolor{Maroon}{\mathsfit{z}}} {\mathbb{1}}_{\mathcolor{gray}{z}} \leftindex^{\textcolor{Maroon}{\mathsfit{z}}} \;\! J^{\;\!\mathcolor{gray}{t}}_{\;\!\textcolor{Maroon}{\text{f}} \symup{\iota}\mathcolor{gray}{z}} + \leftindex_{\textcolor{Maroon}{\mathsfit{z}}} \;\! \delta_{\mathcolor{gray}{z}} \leftindex^{\textcolor{Maroon}{\mathsfit{z}}} \;\!
	{\alpha}^{\;\!\mathcolor{gray}{t}}_{\;\! \symup{\iota}\mathcolor{gray}{z}} ~, \label{eq:curl-H'} \\
	\mathcolor{gray}{\nabla^\iota} D^{\;\!\mathcolor{gray}{t}}_{\;\! \mathcolor{gray}{\symup{\iota}} \mathcolor{gray}{z}} &= \leftindex_{\textcolor{Maroon}{\mathsfit{z}}} {\mathbb{1}}_{\mathcolor{gray}{z}} \leftindex^{\textcolor{Maroon}{\mathsfit{z}}} {\rho}^{\;\!\mathcolor{gray}{t}}_{\;\!\textcolor{Maroon}{\text{f}}\mathcolor{gray}{z}} + \leftindex_{\textcolor{Maroon}{\mathsfit{z}}} \;\! \delta_{\mathcolor{gray}{z}} \leftindex^{\textcolor{Maroon}{\mathsfit{z}}} \;\! {\sigma}^{\;\!\mathcolor{gray}{t}}_{\;\! \mathcolor{gray}{z}}~, \label{eq:div-D'} \\
	\mathcolor{gray}{\nabla^\iota} B^{\;\!\mathcolor{gray}{t}}_{\;\! \mathcolor{gray}{\symup{\iota}} \mathcolor{gray}{z}} &= 0~. \label{eq:div-B'duplicate}
\end{align}
\end{subequations}
其中,$\bar{E}^{\;\!\mathcolor{gray}{t}}_{\;\!\mathcolor{gray}{z}}, \bar{B}^{\;\!\mathcolor{gray}{t}}_{\;\!\mathcolor{gray}{z}}$ 为 \bref{eq:EB-01} 等号左侧的总场,$\bar{D}^{\;\!\mathcolor{gray}{t}}_{\;\!\mathcolor{gray}{z}}, \bar{H}^{\;\!\mathcolor{gray}{t}}_{\;\!\mathcolor{gray}{z}}$ 也具有相同的\textcolor{Plum}{奇异}场层次结构,但尚未展开成 \bref{eq:EB-01} 的形式。将上述 \bref{eq:div-D'} 减去 \bref{eq:div-E'},\bref{eq:curl-H'} 减去 \bref{eq:curl-B'} 乘以 ${\symup{\varepsilon}}_0$,得到
\begin{subequations} \label{eq:curl-EH-EB}
	\small
\begin{align}
	&\mathcolor{gray}{\nabla^\iota} \left( D^{\;\!\mathcolor{gray}{t}}_{\;\! \mathcolor{gray}{\symup{\iota}} \mathcolor{gray}{z}} - {\symup{\varepsilon}}_0 E^{\;\!\mathcolor{gray}{t}}_{\;\! \mathcolor{gray}{\symup{\iota}} \mathcolor{gray}{z}} \right) = - \left( \leftindex_{\textcolor{Maroon}{\mathsfit{z}}} {\mathbb{1}}_{\mathcolor{gray}{z}} \leftindex^{\textcolor{Maroon}{\mathsfit{z}}}  {\rho}^{\;\!\mathcolor{gray}{t}}_{\;\!\textcolor{Maroon}{\text{b}}\mathcolor{gray}{z}} - \leftindex_{\textcolor{Maroon}{\mathsfit{z}}} \;\! \delta_{\mathcolor{gray}{z}} \leftindex^{\textcolor{Maroon}{\mathsfit{z}}} \;\! {\mathcal{P}}^{\;\!\mathcolor{gray}{t}}_{\;\! \symup{z} \mathcolor{gray}{z}} - \leftindex_{\textcolor{Maroon}{\mathsfit{z}}} \;\! \delta'_{\mathcolor{gray}{z}} \leftindex^{\textcolor{Maroon}{\mathsfit{z}}} \;\! {\mathcal{Q}}^{\;\!\mathcolor{gray}{t}}_{\;\! \symup{z} \symup{z} \mathcolor{gray}{z}} - \leftindex_{\textcolor{Maroon}{\mathsfit{z}}} \;\! \delta''_{\mathcolor{gray}{z}} \leftindex^{\textcolor{Maroon}{\mathsfit{z}}} \;\! {\mathcal{O}}^{\;\!\mathcolor{gray}{t}}_{\;\! \symup{z} \symup{z} \symup{z} \mathcolor{gray}{z}} \right) ~, \label{eq:div-D'-E'} \\
	\epsilon^{\hphantom{\symup{\iota}\hat{1}}\hat{2}}_{\symup{\iota}\mathcolor{gray}{\hat{1}}} &\mathcolor{gray}{\nabla^{\hat{1}}} \left( H^{\;\!\mathcolor{gray}{t}}_{\;\! \hat{2}\mathcolor{gray}{z}} - {\symup{\mu}}_0^{-1} B^{\;\!\mathcolor{gray}{t}}_{\;\! \hat{2}\mathcolor{gray}{z}} \right) - \mathcolor{gray}{\nabla^t} \left( D^{\;\!\mathcolor{gray}{t}}_{\;\! \symup{\iota}\mathcolor{gray}{z}} - {\symup{\varepsilon}}_0 E^{\;\!\mathcolor{gray}{t}}_{\;\! \symup{\iota}\mathcolor{gray}{z}} \right) = - \left( \leftindex_{\textcolor{Maroon}{\mathsfit{z}}} {\mathbb{1}}_{\mathcolor{gray}{z}} \leftindex^{\textcolor{Maroon}{\mathsfit{z}}} \;\! J^{\;\!\mathcolor{gray}{t}}_{\;\!\textcolor{Maroon}{\text{b}} \symup{\iota}\mathcolor{gray}{z}} - \leftindex_{\textcolor{Maroon}{\mathsfit{z}}} \;\! \delta_{\mathcolor{gray}{z}} \leftindex^{\textcolor{Maroon}{\mathsfit{z}}}
	{\mathcal{K}}^{\;\!\mathcolor{gray}{t}}_{\;\! \symup{\iota}\symup{z}\mathcolor{gray}{z}} - \leftindex_{\textcolor{Maroon}{\mathsfit{z}}} \;\! \delta'_{\mathcolor{gray}{z}} \leftindex^{\textcolor{Maroon}{\mathsfit{z}}} \;\! {\mathcal{L}}^{\;\!\mathcolor{gray}{t}}_{\;\! \symup{\iota}\symup{z} \symup{z} \mathcolor{gray}{z}} \right) ~, \label{eq:curl-H'-B'}
\end{align}
\end{subequations}
该 \bref{eq:curl-EH-EB} 暗示 $\bar{D}^{\;\!\mathcolor{gray}{t}}_{\;\!\mathcolor{gray}{z}}, \bar{H}^{\;\!\mathcolor{gray}{t}}_{\;\!\mathcolor{gray}{z}}$ 具有以下类似但不同于 \bref{eq:EB-01} 的\textcolor{Plum}{奇异}场层次\Footnote{\bref{eq:DH-01} 可视为教科书式定义下的\textcolor{NavyBlue}{辅助场} $\bar{D}^{\;\!\mathcolor{gray}{t}}_{\;\!\mathcolor{gray}{z}}, \bar{H}^{\;\!\mathcolor{gray}{t}}_{\;\!\mathcolor{gray}{z}}$ 同时朝\textcolor{Plum}{多极}形式和\textcolor{Plum}{奇异}结构的扩展。}
\begin{subequations} \label{eq:DH-01}
\begin{align}
	\hphantom{xxxxx} D^{\;\!\mathcolor{gray}{t}}_{\;\! \symup{\iota}\mathcolor{gray}{z}} &= \hspace{0.2em} {\symup{\varepsilon}}_0 &&\hspace{-2.7em} E^{\;\!\mathcolor{gray}{t}}_{\;\! \symup{\iota}\mathcolor{gray}{z}} \hspace{0.5em} + &&\hspace{-2.5em}\leftindex_{\textcolor{Maroon}{\mathsfit{z}}} {\mathbb{1}}_{\mathcolor{gray}{z}} \leftindex^{\textcolor{Maroon}{\mathsfit{z}}} \;\! D^{\;\!\mathcolor{gray}{t}}_{\;\! \symup{\iota}\mathcolor{gray}{z}} &&\hspace{-2.5em}- \leftindex_{\textcolor{Maroon}{\mathsfit{z}}} \;\! \delta_{\mathcolor{gray}{z}} \leftindex^{\textcolor{Maroon}{\mathsfit{z}}} \;\!
	{\mathcal{D}}^{\;\!\textcolor{Maroon}{(1)}\mathcolor{gray}{t}}_{\;\! \symup{\iota}\mathcolor{gray}{z}} &&\hspace{-2.5em}- \leftindex_{\textcolor{Maroon}{\mathsfit{z}}} \;\! \delta'_{\mathcolor{gray}{z}} \leftindex^{\textcolor{Maroon}{\mathsfit{z}}} \;\! {\mathcal{D}}^{\;\!\textcolor{Maroon}{(2)}\mathcolor{gray}{t}}_{\;\! \symup{\iota}\mathcolor{gray}{z}} ~, \label{eq:D-01} \\
	\hphantom{xxxxx} H^{\;\!\mathcolor{gray}{t}}_{\;\! \symup{\iota}\mathcolor{gray}{z}} &= \hspace{0.2em} {\symup{\mu}}_0^{-1} &&\hspace{-2.7em} B^{\;\!\mathcolor{gray}{t}}_{\;\! \symup{\iota}\mathcolor{gray}{z}} \hspace{0.5em} + &&\hspace{-2.5em}\leftindex_{\textcolor{Maroon}{\mathsfit{z}}} {\mathbb{1}}_{\mathcolor{gray}{z}} \leftindex^{\textcolor{Maroon}{\mathsfit{z}}} \;\! H^{\;\!\mathcolor{gray}{t}}_{\;\! \symup{\iota}\mathcolor{gray}{z}} &&\hspace{-2.5em}- \leftindex_{\textcolor{Maroon}{\mathsfit{z}}} \;\! \delta_{\mathcolor{gray}{z}} \leftindex^{\textcolor{Maroon}{\mathsfit{z}}} \;\!
	{\mathcal{H}}^{\;\!\textcolor{Maroon}{(1)}\mathcolor{gray}{t}}_{\;\! \symup{\iota}\mathcolor{gray}{z}} &&\hspace{-2.5em}~, \label{eq:H-01}
\end{align}
\end{subequations}
其中 $\bar{E}^{\;\!\mathcolor{gray}{t}}_{\;\!\mathcolor{gray}{z}}, \bar{B}^{\;\!\mathcolor{gray}{t}}_{\;\!\mathcolor{gray}{z}}$ 为 \bref{eq:EB-01} 中具有\textcolor{Plum}{奇异}多项式的总场。

将上述 \bref{eq:DH-01} 代入 \bref{eq:curl-EH-EB},得到类似但不同于 \bref{eq:curl-EB-01} 的形式
\begin{subequations} \label{eq:curl-DH-01}
	\footnotesize
\begin{align}
	\left( \leftindex_{\textcolor{Maroon}{\mathsfit{z}}} \;\! \delta_{\mathcolor{gray}{z}} \leftindex^{\textcolor{Maroon}{\mathsfit{z}}} \;\! D^{\;\!\mathcolor{gray}{t}}_{\;\! \symup{z} \mathcolor{gray}{z}} - \leftindex_{\textcolor{Maroon}{\mathsfit{z}}} \;\! \delta'_{\mathcolor{gray}{z}} \leftindex^{\textcolor{Maroon}{\mathsfit{z}}} \;\!
	{\mathcal{D}}^{\;\!\textcolor{Maroon}{(1)}\mathcolor{gray}{t}}_{\;\! \symup{z} \mathcolor{gray}{z}} \right. &- \left. \leftindex_{\textcolor{Maroon}{\mathsfit{z}}} \;\! \delta''_{\mathcolor{gray}{z}} \leftindex^{\textcolor{Maroon}{\mathsfit{z}}} \;\! {\mathcal{D}}^{\;\!\textcolor{Maroon}{(2)}\mathcolor{gray}{t}}_{\;\! \symup{z} \mathcolor{gray}{z}} \right) + \left( \leftindex_{\textcolor{Maroon}{\mathsfit{z}}} {\mathbb{1}}_{\mathcolor{gray}{z}} \mathcolor{gray}{\nabla^\iota} \leftindex^{\textcolor{Maroon}{\mathsfit{z}}} \;\! D^{\;\!\mathcolor{gray}{t}}_{\;\! \mathcolor{gray}{\symup{\iota}} \mathcolor{gray}{z}} - \leftindex_{\textcolor{Maroon}{\mathsfit{z}}} \;\! \delta_{\mathcolor{gray}{z}} \mathcolor{gray}{\nabla^\iota} \leftindex^{\textcolor{Maroon}{\mathsfit{z}}} \;\!
	{\mathcal{D}}^{\;\!\textcolor{Maroon}{(1)}\mathcolor{gray}{t}}_{\;\! \mathcolor{gray}{\symup{\iota}} \mathcolor{gray}{z}} - \leftindex_{\textcolor{Maroon}{\mathsfit{z}}} \;\! \delta'_{\mathcolor{gray}{z}} \mathcolor{gray}{\nabla^\iota} \leftindex^{\textcolor{Maroon}{\mathsfit{z}}} \;\!
	{\mathcal{D}}^{\;\!\textcolor{Maroon}{(2)}\mathcolor{gray}{t}}_{\;\! \mathcolor{gray}{\symup{\iota}} \mathcolor{gray}{z}} \right) \label{eq:div-D-01} \\ &= - \left( \leftindex_{\textcolor{Maroon}{\mathsfit{z}}} {\mathbb{1}}_{\mathcolor{gray}{z}} \leftindex^{\textcolor{Maroon}{\mathsfit{z}}}  {\rho}^{\;\!\mathcolor{gray}{t}}_{\;\!\textcolor{Maroon}{\text{b}}\mathcolor{gray}{z}} - \leftindex_{\textcolor{Maroon}{\mathsfit{z}}} \;\! \delta_{\mathcolor{gray}{z}} \leftindex^{\textcolor{Maroon}{\mathsfit{z}}} \;\! {\mathcal{P}}^{\;\!\mathcolor{gray}{t}}_{\;\! \symup{z} \mathcolor{gray}{z}} - \leftindex_{\textcolor{Maroon}{\mathsfit{z}}} \;\! \delta'_{\mathcolor{gray}{z}} \leftindex^{\textcolor{Maroon}{\mathsfit{z}}} \;\! {\mathcal{Q}}^{\;\!\mathcolor{gray}{t}}_{\;\! \symup{z} \symup{z} \mathcolor{gray}{z}} - \leftindex_{\textcolor{Maroon}{\mathsfit{z}}} \;\! \delta''_{\mathcolor{gray}{z}} \leftindex^{\textcolor{Maroon}{\mathsfit{z}}} \;\! {\mathcal{O}}^{\;\!\mathcolor{gray}{t}}_{\;\! \symup{z} \symup{z} \symup{z} \mathcolor{gray}{z}} \right)~, \\
	\epsilon^{\hphantom{\symup{\symup{\iota}z}}\hat{2}}_{\symup{\iota} \symup{z}} \left( \leftindex_{\textcolor{Maroon}{\mathsfit{z}}} \;\! \delta_{\mathcolor{gray}{z}} \leftindex^{\textcolor{Maroon}{\mathsfit{z}}} H^{\;\!\mathcolor{gray}{t}}_{\;\! \hat{2}\mathcolor{gray}{z}} - \leftindex_{\textcolor{Maroon}{\mathsfit{z}}} \;\! \delta'_{\mathcolor{gray}{z}} \leftindex^{\textcolor{Maroon}{\mathsfit{z}}} \;\!
	{\mathcal{H}}^{\;\!\textcolor{Maroon}{(1)}\mathcolor{gray}{t}}_{\;\! \hat{2}\mathcolor{gray}{z}} \right) &+ \epsilon^{\hphantom{\symup{\iota}\hat{1}}\hat{2}}_{\symup{\iota}\mathcolor{gray}{\hat{1}}} \left( \leftindex_{\textcolor{Maroon}{\mathsfit{z}}} {\mathbb{1}}_{\mathcolor{gray}{z}} \mathcolor{gray}{\nabla^{\hat{1}}} \leftindex^{\textcolor{Maroon}{\mathsfit{z}}} \;\! H^{\;\!\mathcolor{gray}{t}}_{\;\! \hat{2}\mathcolor{gray}{z}} - \leftindex_{\textcolor{Maroon}{\mathsfit{z}}} \;\! \delta_{\mathcolor{gray}{z}} \mathcolor{gray}{\nabla^{\hat{1}}} \leftindex^{\textcolor{Maroon}{\mathsfit{z}}} \;\!
	{\mathcal{H}}^{\;\!\textcolor{Maroon}{(1)}\mathcolor{gray}{t}}_{\;\! \hat{2}\mathcolor{gray}{z}} \right) + \mathcolor{gray}{\nabla^t} \left( \leftindex_{\textcolor{Maroon}{\mathsfit{z}}} \;\! \delta_{\mathcolor{gray}{z}} \leftindex^{\textcolor{Maroon}{\mathsfit{z}}}
	{\mathcal{D}}^{\;\!\textcolor{Maroon}{(1)}\mathcolor{gray}{t}}_{\;\! \symup{\iota}\mathcolor{gray}{z}} + \leftindex_{\textcolor{Maroon}{\mathsfit{z}}} \;\! \delta'_{\mathcolor{gray}{z}} \leftindex^{\textcolor{Maroon}{\mathsfit{z}}} \;\! {\mathcal{D}}^{\;\!\textcolor{Maroon}{(2)}\mathcolor{gray}{t}}_{\;\! \symup{\iota}\mathcolor{gray}{z}} \right) \label{eq:curl-H-01} \\ &= - \left( \leftindex_{\textcolor{Maroon}{\mathsfit{z}}} {\mathbb{1}}_{\mathcolor{gray}{z}} \leftindex^{\textcolor{Maroon}{\mathsfit{z}}} \;\! J^{\;\!\mathcolor{gray}{t}}_{\;\!\textcolor{Maroon}{\text{b}} \symup{\iota}\mathcolor{gray}{z}} - \leftindex_{\textcolor{Maroon}{\mathsfit{z}}} \;\! \delta_{\mathcolor{gray}{z}} \leftindex^{\textcolor{Maroon}{\mathsfit{z}}}
	{\mathcal{K}}^{\;\!\mathcolor{gray}{t}}_{\;\! \symup{\iota}\symup{z}\mathcolor{gray}{z}} - \leftindex_{\textcolor{Maroon}{\mathsfit{z}}} \;\! \delta'_{\mathcolor{gray}{z}} \leftindex^{\textcolor{Maroon}{\mathsfit{z}}} \;\! {\mathcal{L}}^{\;\!\mathcolor{gray}{t}}_{\;\! \symup{\iota}\symup{z} \symup{z} \mathcolor{gray}{z}} \right) + \mathcolor{gray}{\nabla^t} \leftindex_{\textcolor{Maroon}{\mathsfit{z}}} {\mathbb{1}}_{\mathcolor{gray}{z}} \leftindex^{\textcolor{Maroon}{\mathsfit{z}}} \;\! D^{\;\!\mathcolor{gray}{t}}_{\;\! \symup{\iota}\mathcolor{gray}{z}} ~,
\end{align}
\end{subequations}
合并同层次\textcolor{Plum}{奇异}项(通过 \bref{eq:Intdeltasum=0})和额外的 ${\mathbb{1}}_{\mathcolor{gray}{z}} ~\textcolor{Maroon}{\text{项}}$,得 \bref{eq:curl-EB-01-deltas} 的对应版本
\begin{subequations} \label{eq:curl-DH-01-deltas}
\begin{align}
	{\mathbb{1}}_{\mathcolor{gray}{z}} ~\textcolor{Maroon}{\text{项}}:&\hspace{1.0em}  \hphantom{\epsilon^{\hphantom{\symup{\symup{\iota}z}}\hat{2}}_{\symup{\iota} \symup{z}} H^{\;\!\mathcolor{gray}{t}}_{\;\! \hat{2}\mathcolor{gray}{0}} - \epsilon^{\hphantom{\symup{\iota}\hat{1}}\hat{2}}_{\symup{\iota}\mathcolor{gray}{\hat{1}}}} \mathcolor{gray}{\nabla^\iota} \;\! D^{\;\!\mathcolor{gray}{t}}_{\;\! \mathcolor{gray}{\symup{\iota}} \mathcolor{gray}{z}} \hspace{-3.8em}&&=\hspace{0.2em} -\hspace{0.2em} {\rho}^{\;\!\mathcolor{gray}{t}}_{\;\!\textcolor{Maroon}{\text{b}}\mathcolor{gray}{z}}~,  \label{eq:curl-DH-01-one} \\
	&\hspace{1.0em} \hphantom{\epsilon^{\hphantom{\symup{\symup{\iota}z}}\hat{2}}_{\symup{\iota} \symup{z}} H^{\;\!\mathcolor{gray}{t}}_{\;\! \hat{2}\mathcolor{gray}{0}} -}\;\!\;\!\;\!\;\! \epsilon^{\hphantom{\symup{\iota}\hat{1}}\hat{2}}_{\symup{\iota}\mathcolor{gray}{\hat{1}}} \mathcolor{gray}{\nabla^{\hat{1}}} H^{\;\!\mathcolor{gray}{t}}_{\;\! \hat{2}\mathcolor{gray}{z}} - \mathcolor{gray}{\nabla^t} D^{\;\!\mathcolor{gray}{t}}_{\;\! \symup{\iota}\mathcolor{gray}{z}} \hspace{-3.8em}&&=\hspace{0.2em} -\hspace{0.2em} J^{\;\!\mathcolor{gray}{t}}_{\;\!\textcolor{Maroon}{\text{b}} \symup{\iota}\mathcolor{gray}{z}}~, \label{eq:curl-DH-01-one2} \\[0.7em]
	{\delta}_{\mathcolor{gray}{z}} ~\textcolor{Maroon}{\text{项}}:&\hspace{1.0em}  \hphantom{\epsilon^{\hphantom{\symup{\symup{\iota}z}}\hat{2}}_{\symup{\iota} \symup{z}}}\;\! D^{\;\!\mathcolor{gray}{t}}_{\;\! \symup{z} \mathcolor{gray}{0}} - \hphantom{\epsilon^{\hphantom{\symup{\iota}\hat{1}}\hat{2}}_{\symup{\iota}\mathcolor{gray}{\hat{1}}}} \mathcolor{gray}{\nabla^\iota}
	{\mathcal{D}}^{\;\!\textcolor{Maroon}{(1)}\mathcolor{gray}{t}}_{\;\! \mathcolor{gray}{\symup{\iota}} \mathcolor{gray}{0}} \hspace{-3.8em}&&=\hspace{0.2em} {\mathcal{P}}^{\;\!\mathcolor{gray}{t}}_{\;\! \symup{z} \mathcolor{gray}{0}}~,  \label{eq:curl-DH-01-delta} \\
	&\hspace{1.0em} \epsilon^{\hphantom{\symup{\symup{\iota}z}}\hat{2}}_{\symup{\iota} \symup{z}} H^{\;\!\mathcolor{gray}{t}}_{\;\! \hat{2}\mathcolor{gray}{0}} - \epsilon^{\hphantom{\symup{\iota}\hat{1}}\hat{2}}_{\symup{\iota}\mathcolor{gray}{\hat{1}}} \mathcolor{gray}{\nabla^{\hat{1}}} 
	{\mathcal{H}}^{\;\!\textcolor{Maroon}{(1)}\mathcolor{gray}{t}}_{\;\! \hat{2}\mathcolor{gray}{0}} + \mathcolor{gray}{\nabla^t} 
	{\mathcal{D}}^{\;\!\textcolor{Maroon}{(1)}\mathcolor{gray}{t}}_{\;\! \symup{\iota}\mathcolor{gray}{0}} \hspace{-3.8em}&&=\hspace{0.2em} {\mathcal{K}}^{\;\!\mathcolor{gray}{t}}_{\;\! \symup{\iota}\symup{z}\mathcolor{gray}{0}}~, \label{eq:curl-DH-01-delta2} \\[0.7em]
	{\delta}'_{\mathcolor{gray}{z}} ~\textcolor{Maroon}{\text{项}}:&\hspace{1.0em}  \hphantom{\epsilon^{\hphantom{\symup{\symup{\iota}z}}\hat{2}}_{\symup{\iota} \symup{z}}} {\mathcal{D}}^{\;\!\textcolor{Maroon}{(1)}\mathcolor{gray}{t}}_{\;\! \symup{z} \mathcolor{gray}{0}} + \hphantom{\epsilon^{\hphantom{\symup{\iota}\hat{1}}\hat{2}}_{\symup{\iota}\mathcolor{gray}{\hat{1}}}}\!\!\;\! \mathcolor{gray}{\nabla^\iota} 
	{\mathcal{D}}^{\;\!\textcolor{Maroon}{(2)}\mathcolor{gray}{t}}_{\;\! \mathcolor{gray}{\symup{\iota}} \mathcolor{gray}{0}} \hspace{-3.8em}&&=\hspace{0.2em} -\hspace{0.2em} {\mathcal{Q}}^{\;\!\mathcolor{gray}{t}}_{\;\! \symup{z} \symup{z} \mathcolor{gray}{0}}~,  \label{eq:curl-DH-01-delta'} \\
	&\hspace{1.0em} \epsilon^{\hphantom{\symup{\symup{\iota}z}}\hat{2}}_{\symup{\iota} \symup{z}} {\mathcal{H}}^{\;\!\textcolor{Maroon}{(1)}\mathcolor{gray}{t}}_{\;\! \hat{2}\mathcolor{gray}{0}} \hphantom{\epsilon^{\hphantom{\symup{\symup{\iota}z}}\hat{2}}_{\symup{\iota} \symup{z}} H^{\;\!\mathcolor{gray}{t}}_{\;\! \hat{2}\mathcolor{gray}{0}} - \epsilon^{\hphantom{\symup{\iota}\hat{1}}\hat{2}}_{\symup{\iota}\mathcolor{gray}{\hat{1}}}}\!\! - \mathcolor{gray}{\nabla^t} 
	{\mathcal{D}}^{\;\!\textcolor{Maroon}{(2)}\mathcolor{gray}{t}}_{\;\! \symup{\iota}\mathcolor{gray}{0}} \hspace{-3.8em}&&=\hspace{0.2em} -\hspace{0.2em} {\mathcal{L}}^{\;\!\mathcolor{gray}{t}}_{\;\! \symup{\iota}\symup{z} \symup{z} \mathcolor{gray}{0}}~,  \label{eq:curl-DH-01-delta'2} \\[0.7em]
	{\delta}''_{\mathcolor{gray}{z}} ~\textcolor{Maroon}{\text{项}}:&\hspace{1.0em} \hphantom{\epsilon^{\hphantom{\symup{\symup{\iota}z}}\hat{2}}_{\symup{\iota} \symup{z}}} 
	{\mathcal{D}}^{\;\!\textcolor{Maroon}{(2)}\mathcolor{gray}{t}}_{\;\! \symup{z} \mathcolor{gray}{0}} \hspace{-3.8em}&&=\hspace{0.2em} -\hspace{0.2em} {\mathcal{O}}^{\;\!\mathcolor{gray}{t}}_{\;\! \symup{z} \symup{z} \symup{z} \mathcolor{gray}{0}}~, \label{eq:curl-DH-01-delta''}
\end{align}
\end{subequations}
其中,对比 \bref{eq:curl-DH-01-one,eq:pb} 得 $D^{\;\!\mathcolor{gray}{t}}_{\;\! \symup{\iota}\mathcolor{gray}{z}} = {\mathcal{P}}^{\;\!\mathcolor{gray}{t}}_{\;\!\textcolor{Maroon}{\text{b}} \symup{\iota} \mathcolor{gray}{z}}$;将其和 \bref{eq:Je-Jb} 代入 \bref{eq:curl-DH-01-one2} 得 $H^{\;\!\mathcolor{gray}{t}}_{\;\! \hat{2}\mathcolor{gray}{z}} = - M^{\;\!\mathcolor{gray}{t}}_{\;\! \hat{2} \mathcolor{gray}{z}} + \mathcolor{gray}{\nabla^{\hat{3}}} N^{\;\!\mathcolor{gray}{t}}_{\;\! \hat{2} \mathcolor{gray}{\hat{3}} \mathcolor{gray}{z}}$;继续将 $D^{\;\!\mathcolor{gray}{t}}_{\;\! \symup{\iota}\mathcolor{gray}{0}}$ 和 \bref{eq:Pb-QB} 代入 \bref{eq:curl-DH-01-delta},得到 ${\mathcal{D}}^{\;\!\textcolor{Maroon}{(1)}\mathcolor{gray}{t}}_{\;\! \symup{\iota}\mathcolor{gray}{0}} = - {\mathcal{Q}}^{\;\!\mathcolor{gray}{t}}_{\;\!\textcolor{Maroon}{\text{B}} \symup{z}\symup{\iota} \mathcolor{gray}{0}}$;将其和 $H^{\;\!\mathcolor{gray}{t}}_{\;\! \hat{2}\mathcolor{gray}{0}}$,以及 \bref{eq:KE-Kb} 代入 \bref{eq:curl-DH-01-delta2} 得 ${\mathcal{H}}^{\;\!\textcolor{Maroon}{(1)}\mathcolor{gray}{t}}_{\;\! \hat{2}\mathcolor{gray}{0}} = - N^{\;\!\mathcolor{gray}{t}}_{\;\! \hat{2}\symup{z}\mathcolor{gray}{0}}$;继续将 ${\mathcal{D}}^{\;\!\textcolor{Maroon}{(1)}\mathcolor{gray}{t}}_{\;\! \symup{\iota}\mathcolor{gray}{0}}$ 和 \bref{eq:Qb-KM} 代入 \bref{eq:curl-DH-01-delta'} 得 ${\mathcal{D}}^{\;\!\textcolor{Maroon}{(2)}\mathcolor{gray}{t}}_{\;\! \symup{\iota}\mathcolor{gray}{0}} = - {\mathcal{O}}^{\;\!\mathcolor{gray}{t}}_{\;\!\textcolor{Maroon}{\text{b}} \symup{z}\symup{z}\symup{\iota} \mathcolor{gray}{0}}$。可以验证上述得到的 ${\mathcal{H}}^{\;\!\textcolor{Maroon}{(1)}\mathcolor{gray}{t}}_{\;\! \hat{2}\mathcolor{gray}{0}}, {\mathcal{D}}^{\;\!\textcolor{Maroon}{(2)}\mathcolor{gray}{t}}_{\;\! \symup{\iota}\mathcolor{gray}{0}}$ 满足 \bref{eq:curl-DH-01-delta'2,eq:curl-DH-01-delta'',eq:LE-Lb}。

将上一段各量展开至 \bref{eq:multipole} 层次\Footnote{用到了\textcolor{Plum}{多极}矩角标分量的\textcolor{Plum}{置换对称性}。}
\begin{subequations} \label{eq:DH^(2-0)_0}
\begin{align}
	D^{\;\!\mathcolor{gray}{t}}_{\;\! \symup{\iota}\mathcolor{gray}{z}} &= {\mathcal{P}}^{\;\!\mathcolor{gray}{t}}_{\;\!\textcolor{Maroon}{\text{b}} \symup{\iota} \mathcolor{gray}{z}}~, \label{eq:D^(0)} \\
	{\mathcal{D}}^{\;\!\textcolor{Maroon}{(1)}\mathcolor{gray}{t}}_{\;\! \symup{\iota}\mathcolor{gray}{0}} &= -\hspace{0.2em} {\mathcal{Q}}^{\;\!\mathcolor{gray}{t}}_{\;\!\textcolor{Maroon}{\text{b}} \symup{\iota}\symup{z} \mathcolor{gray}{0}} - \mathcolor{gray}{\nabla^{\hat{2}}} {\mathcal{O}}^{\;\!\mathcolor{gray}{t}}_{\;\!\textcolor{Maroon}{\text{b}} \symup{\iota} \symup{z} \mathcolor{gray}{\hat{2}} \mathcolor{gray}{0}}~, \label{eq:D^(1)_0} \\
	{\mathcal{D}}^{\;\!\textcolor{Maroon}{(2)}\mathcolor{gray}{t}}_{\;\! \symup{\iota}\mathcolor{gray}{0}} &= -\hspace{0.2em} {\mathcal{O}}^{\;\!\mathcolor{gray}{t}}_{\;\!\textcolor{Maroon}{\text{b}} \symup{\iota} \symup{z}\symup{z} \mathcolor{gray}{0}}~, \label{eq:D^(2)_0}
\end{align}
\end{subequations}
及其下一(裸\textcolor{Plum}{多极}矩)层次\cite{OriginDependenceMaterial,langeCompletionMultipoleTheory2003,raabTransformedMultipoleTheory2005,grahamMultipoleSolutionMacroscopic2000}:
\begin{subequations} \label{eq:DH^(2-0)_0=}
\begin{align}
	D^{\;\!\mathcolor{gray}{t}}_{\;\! \symup{\iota}\mathcolor{gray}{z}} &= P^{\;\!\mathcolor{gray}{t}}_{\;\! \symup{\iota}\mathcolor{gray}{z}} - \mathcolor{gray}{\nabla^{\hat{1}}} Q^{\;\!\mathcolor{gray}{t}}_{\;\! \symup{\iota} \mathcolor{gray}{\hat{1}} \mathcolor{gray}{z}} + \mathcolor{gray}{\nabla^{\hat{1}}} \mathcolor{gray}{\nabla^{\hat{2}}} O^{\;\!\mathcolor{gray}{t}}_{\;\! \symup{\iota} \mathcolor{gray}{\hat{1}\hat{2}} \mathcolor{gray}{z}} - \cdots~, \label{eq:D^(0)=} \\
	{\mathcal{D}}^{\;\!\textcolor{Maroon}{(1)}\mathcolor{gray}{t}}_{\;\! \symup{\iota}\mathcolor{gray}{0}} &= Q^{\;\!\mathcolor{gray}{t}}_{\;\! \symup{\iota}\symup{z}\mathcolor{gray}{z}} - 2~ \mathcolor{gray}{\nabla^{\hat{1}}} O^{\;\!\mathcolor{gray}{t}}_{\;\! \symup{\iota} \mathcolor{gray}{\hat{1}} \symup{z}\mathcolor{gray}{z}} + \cdots~, \label{eq:D^(1)_0=} \\
	{\mathcal{D}}^{\;\!\textcolor{Maroon}{(2)}\mathcolor{gray}{t}}_{\;\! \symup{\iota}\mathcolor{gray}{0}} &= -\hspace{0.2em} O^{\;\!\mathcolor{gray}{t}}_{\;\! \symup{\iota}\symup{z}\symup{z}\mathcolor{gray}{z}} + \cdots~, \label{eq:D^(2)_0=} \\[0.7em]
	H^{\;\!\mathcolor{gray}{t}}_{\;\! \symup{\iota}\mathcolor{gray}{z}} &= -\hspace{0.2em} M^{\;\!\mathcolor{gray}{t}}_{\;\! \symup{\iota} \mathcolor{gray}{z}} + \mathcolor{gray}{\nabla^{\hat{1}}} N^{\;\!\mathcolor{gray}{t}}_{\;\! \symup{\iota} \mathcolor{gray}{\hat{1}} \mathcolor{gray}{z}} - \cdots~, \label{eq:H^(0)=} \\
	{\mathcal{H}}^{\;\!\textcolor{Maroon}{(1)}\mathcolor{gray}{t}}_{\;\! \symup{\iota}\mathcolor{gray}{0}} &= -\hspace{0.2em} N^{\;\!\mathcolor{gray}{t}}_{\;\! \symup{\iota}\symup{z}\mathcolor{gray}{0}} + \cdots~. \label{eq:H^(1)_0=}
\end{align}
\end{subequations}

若将 $\bar{D}^{\;\!\mathcolor{gray}{t}}_{\;\!\mathcolor{gray}{z}}, \bar{H}^{\;\!\mathcolor{gray}{t}}_{\;\!\mathcolor{gray}{z}}$ 展开成类似 \bref{eq:EB-01}(而不是 \bref{eq:DH-01})的\textcolor{Plum}{奇异}场层次,即
\begin{subequations} \label{eq:DH-01'}
\begin{align}
	\hphantom{xxxxx} D^{\;\!\mathcolor{gray}{t}}_{\;\! \symup{\iota}\mathcolor{gray}{z}} &= \hspace{0.2em} &&\hspace{-4.5em}\leftindex_{\textcolor{Maroon}{\mathsfit{z}}} {\mathbb{1}}_{\mathcolor{gray}{z}} \leftindex^{\textcolor{Maroon}{\mathsfit{z}}} \;\! D^{\;\!\mathcolor{gray}{t}}_{\;\! \symup{\iota}\mathcolor{gray}{z}} &&\hspace{-4.5em}- \leftindex_{\textcolor{Maroon}{\mathsfit{z}}} \;\! \delta_{\mathcolor{gray}{z}} \leftindex^{\textcolor{Maroon}{\mathsfit{z}}} \;\!
	{\mathcal{D}}^{\;\!\textcolor{Maroon}{(1)}\mathcolor{gray}{t}}_{\;\! \symup{\iota}\mathcolor{gray}{z}} &&\hspace{-4.5em}- \leftindex_{\textcolor{Maroon}{\mathsfit{z}}} \;\! \delta'_{\mathcolor{gray}{z}} \leftindex^{\textcolor{Maroon}{\mathsfit{z}}} \;\! {\mathcal{D}}^{\;\!\textcolor{Maroon}{(2)}\mathcolor{gray}{t}}_{\;\! \symup{\iota}\mathcolor{gray}{z}} ~, \label{eq:D-01'} \\
	\hphantom{xxxxx} H^{\;\!\mathcolor{gray}{t}}_{\;\! \symup{\iota}\mathcolor{gray}{z}} &= \hspace{0.2em} &&\hspace{-4.5em}\leftindex_{\textcolor{Maroon}{\mathsfit{z}}} {\mathbb{1}}_{\mathcolor{gray}{z}} \leftindex^{\textcolor{Maroon}{\mathsfit{z}}} \;\! H^{\;\!\mathcolor{gray}{t}}_{\;\! \symup{\iota}\mathcolor{gray}{z}} &&\hspace{-4.5em}- \leftindex_{\textcolor{Maroon}{\mathsfit{z}}} \;\! \delta_{\mathcolor{gray}{z}} \leftindex^{\textcolor{Maroon}{\mathsfit{z}}} \;\!
	{\mathcal{H}}^{\;\!\textcolor{Maroon}{(1)}\mathcolor{gray}{t}}_{\;\! \symup{\iota}\mathcolor{gray}{z}} &&\hspace{-4.5em}~, \label{eq:H-01'}
\end{align}
\end{subequations}
那么对于电位移场,其(在宏观阶跃边界 $\mathcolor{gray}{z \mathcolor{black}{=} 0}$ 附近的)非零分量有
\begin{subequations} \label{eq:D^(2-0)_0=}
\begin{align}
	{\mathcal{D}}^{\;\!\textcolor{Maroon}{(2)}\mathcolor{gray}{t}}_{\;\! \symup{x} \mathcolor{gray}{0}} = &- O^{\;\!\mathcolor{gray}{t}}_{\;\! \symup{x}\symup{z}\symup{z}\mathcolor{gray}{0}}~, \label{eq:D^(2)_x0=} \\
	{\mathcal{D}}^{\;\!\textcolor{Maroon}{(2)}\mathcolor{gray}{t}}_{\;\! \symup{y} \mathcolor{gray}{0}} = &- O^{\;\!\mathcolor{gray}{t}}_{\;\! \symup{y}\symup{z}\symup{z}\mathcolor{gray}{0}}~; \label{eq:D^(2)_y0=} \\[1.0em]
	{\mathcal{D}}^{\;\!\textcolor{Maroon}{(1)}\mathcolor{gray}{t}}_{\;\! \symup{x} \mathcolor{gray}{0}} = &\hphantom{+} Q^{\;\!\mathcolor{gray}{t}}_{\;\! \symup{x}\symup{z}\mathcolor{gray}{z}} - 2~ \mathcolor{gray}{\nabla^\iota} O^{\;\!\mathcolor{gray}{t}}_{\;\! \mathcolor{gray}{\symup{\iota}} \symup{x}\symup{z}\mathcolor{gray}{z}} + \mathcolor{gray}{\nabla_x} O^{\;\!\mathcolor{gray}{t}}_{\;\! \symup{z}\symup{z}\symup{z}\mathcolor{gray}{0}}~, \label{eq:D^(1)_x0=} \\
	{\mathcal{D}}^{\;\!\textcolor{Maroon}{(1)}\mathcolor{gray}{t}}_{\;\! \symup{y} \mathcolor{gray}{0}} = &\hphantom{+} Q^{\;\!\mathcolor{gray}{t}}_{\;\! \symup{y}\symup{z}\mathcolor{gray}{z}} - 2~ \mathcolor{gray}{\nabla^\iota} O^{\;\!\mathcolor{gray}{t}}_{\;\! \mathcolor{gray}{\symup{\iota}} \symup{y}\symup{z}\mathcolor{gray}{z}} + \mathcolor{gray}{\nabla_y} O^{\;\!\mathcolor{gray}{t}}_{\;\! \symup{z}\symup{z}\symup{z}\mathcolor{gray}{0}}~, \label{eq:D^(1)_y0=} \\
	{\mathcal{D}}^{\;\!\textcolor{Maroon}{(1)}\mathcolor{gray}{t}}_{\;\! \symup{z} \mathcolor{gray}{0}} = &\hphantom{+} \mathcolor{gray}{\nabla_x} O^{\;\!\mathcolor{gray}{t}}_{\;\! \mathcolor{gray}{\symup{x}} \symup{z} \symup{z} \mathcolor{gray}{0}} + \mathcolor{gray}{\nabla_y} O^{\;\!\mathcolor{gray}{t}}_{\;\! \mathcolor{gray}{\symup{y}} \symup{z} \symup{z} \mathcolor{gray}{0}}~; \label{eq:D^(1)_z0=} \\[1.0em]
	D^{\;\!\mathcolor{gray}{t}}_{\;\! \symup{x}\mathcolor{gray}{0}} = &\hphantom{+} P^{\;\!\mathcolor{gray}{t}}_{\;\! \symup{x}\mathcolor{gray}{z}} + \mathcolor{gray}{\nabla_x} \left[ Q^{\;\!\mathcolor{gray}{t}}_{\;\! \symup{z} \symup{z} \mathcolor{gray}{0}} - 3~ \mathcolor{gray}{\nabla_x} O^{\;\!\mathcolor{gray}{t}}_{\;\! \mathcolor{gray}{\symup{x}} \symup{z} \symup{z} \mathcolor{gray}{0}} - 3~ \mathcolor{gray}{\nabla_y}  O^{\;\!\mathcolor{gray}{t}}_{\;\! \mathcolor{gray}{\symup{y}} \symup{z} \symup{z}  \mathcolor{gray}{0}} - \mathcolor{gray}{\nabla_z}  O^{\;\!\mathcolor{gray}{t}}_{\;\! \mathcolor{gray}{\symup{z}} \symup{z} \symup{z} \mathcolor{gray}{0}} \right] \label{eq:D_x0=} \\ &- \mathcolor{gray}{\nabla^\iota} \left( Q^{\;\!\mathcolor{gray}{t}}_{\;\! \mathcolor{gray}{\symup{\iota}} \symup{x}\mathcolor{gray}{z}} - \mathcolor{gray}{\nabla^{\hat{1}}} O^{\;\!\mathcolor{gray}{t}}_{\;\! \mathcolor{gray}{\symup{\iota} \hat{1}} \symup{x}\mathcolor{gray}{z}} \right) + \mathcolor{gray}{\nabla^t} \left( \mathcolor{gray}{\nabla^t} O^{\;\!\mathcolor{gray}{t}}_{\;\!\symup{x} \symup{z} \symup{z} \mathcolor{gray}{0}} + N^{\;\!\mathcolor{gray}{t}}_{\;\! \symup{y} \symup{z} \mathcolor{gray}{0}} \right) \big/ \symup{c}^2 ~, \\ 
	D^{\;\!\mathcolor{gray}{t}}_{\;\! \symup{y}\mathcolor{gray}{0}} = &\hphantom{+} P^{\;\!\mathcolor{gray}{t}}_{\;\! \symup{y}\mathcolor{gray}{z}} + \mathcolor{gray}{\nabla_y} \left[ Q^{\;\!\mathcolor{gray}{t}}_{\;\! \symup{z} \symup{z} \mathcolor{gray}{0}} - 3~ \mathcolor{gray}{\nabla_x} O^{\;\!\mathcolor{gray}{t}}_{\;\! \mathcolor{gray}{\symup{x}} \symup{z} \symup{z} \mathcolor{gray}{0}} - 3~ \mathcolor{gray}{\nabla_y}  O^{\;\!\mathcolor{gray}{t}}_{\;\! \mathcolor{gray}{\symup{y}} \symup{z} \symup{z} \mathcolor{gray}{0}} - \mathcolor{gray}{\nabla_z}  O^{\;\!\mathcolor{gray}{t}}_{\;\! \mathcolor{gray}{\symup{z}} \symup{z} \symup{z} \mathcolor{gray}{0}} \right] \label{eq:D_y0=} \\ &- \mathcolor{gray}{\nabla^\iota} \left( Q^{\;\!\mathcolor{gray}{t}}_{\;\! \mathcolor{gray}{\symup{\iota}} \symup{y}\mathcolor{gray}{z}} - \mathcolor{gray}{\nabla^{\hat{1}}} O^{\;\!\mathcolor{gray}{t}}_{\;\! \mathcolor{gray}{\symup{\iota} \hat{1}} \symup{y}\mathcolor{gray}{z}} \right) + \mathcolor{gray}{\nabla^t} \left( \mathcolor{gray}{\nabla^t} O^{\;\!\mathcolor{gray}{t}}_{\;\! \symup{y} \symup{z} \symup{z} \mathcolor{gray}{0}} - N^{\;\!\mathcolor{gray}{t}}_{\;\! \symup{x} \symup{z} \mathcolor{gray}{0}} \right) \big/ \symup{c}^2 ~, \\
	D^{\;\!\mathcolor{gray}{t}}_{\;\! \symup{z} \mathcolor{gray}{0}} = &\hphantom{+} {\sigma}^{\;\!\mathcolor{gray}{t}}_{\;\! \mathcolor{gray}{0}} + \left( \mathcolor{gray}{\nabla_x} Q^{\;\!\mathcolor{gray}{t}}_{\;\! \mathcolor{gray}{\symup{x}} \symup{z} \mathcolor{gray}{0}} + \mathcolor{gray}{\nabla_y} Q^{\;\!\mathcolor{gray}{t}}_{\;\! \mathcolor{gray}{\symup{y}} \symup{z} \mathcolor{gray}{0}} \right) \label{eq:D_z0=} \\ & - \left[ \mathcolor{gray}{\nabla_z} \left( \mathcolor{gray}{\nabla_x} O^{\;\!\mathcolor{gray}{t}}_{\;\! \mathcolor{gray}{\symup{x}} \mathcolor{gray}{\symup{z}} \symup{z} \mathcolor{gray}{0}} + \mathcolor{gray}{\nabla_y} O^{\;\!\mathcolor{gray}{t}}_{\;\! \mathcolor{gray}{\symup{y}} \mathcolor{gray}{\symup{z}} \symup{z} \mathcolor{gray}{0}} \right) + 4~ \mathcolor{gray}{\nabla_x} \mathcolor{gray}{\nabla_y} O^{\;\!\mathcolor{gray}{t}}_{\;\! \mathcolor{gray}{\symup{x}} \mathcolor{gray}{\symup{y}} \symup{z} \mathcolor{gray}{0}} \right. \\ & \left. + 2 \left( \mathcolor{gray}{\nabla_x^2} O^{\;\!\mathcolor{gray}{t}}_{\;\! \mathcolor{gray}{\symup{x}} \mathcolor{gray}{\symup{x}} \symup{z} \mathcolor{gray}{0}} + \mathcolor{gray}{\nabla_y^2} O^{\;\!\mathcolor{gray}{t}}_{\;\! \mathcolor{gray}{\symup{y}} \mathcolor{gray}{\symup{y}} \symup{z} \mathcolor{gray}{0}} \right) - \left( \mathcolor{gray}{\nabla_x^2} + \mathcolor{gray}{\nabla_y^2} \right) O^{\;\!\mathcolor{gray}{t}}_{\;\! \symup{z} \symup{z} \symup{z} \mathcolor{gray}{0}} \right] ~.
\end{align}
\end{subequations}
对于磁场,其(在宏观阶跃边界 $\mathcolor{gray}{z \mathcolor{black}{=} 0}$ 附近的)非零分量有
\begin{subequations} \label{eq:H^(2-0)_0=}
\begin{align}
	{\mathcal{H}}^{\;\!\textcolor{Maroon}{(1)}\mathcolor{gray}{t}}_{\;\! \symup{x} \mathcolor{gray}{0}} = &- \mathcolor{gray}{\nabla^t}
	O^{\;\!\mathcolor{gray}{t}}_{\;\!\symup{y}\symup{z}\symup{z}\mathcolor{gray}{0}}~, \label{eq:H^(1)_x0=} \\
	{\mathcal{H}}^{\;\!\textcolor{Maroon}{(1)}\mathcolor{gray}{t}}_{\;\! \symup{y} \mathcolor{gray}{0}} = &\hphantom{+} \mathcolor{gray}{\nabla^t}
	O^{\;\!\mathcolor{gray}{t}}_{\;\!\symup{x}\symup{z}\symup{z}\mathcolor{gray}{0}}~; \label{eq:H^(1)_y0=} \\[1.0em]
	H^{\;\!\mathcolor{gray}{t}}_{\;\! \symup{x}\mathcolor{gray}{0}} = & - \left( \mathcolor{gray}{\nabla^t} Q^{\;\!\mathcolor{gray}{t}}_{\;\! \symup{y} \symup{z} \mathcolor{gray}{0}} - 
	{\alpha}^{\;\!\mathcolor{gray}{t}}_{\;\! \symup{y}\mathcolor{gray}{0}} \right) + \mathcolor{gray}{\nabla_x} N^{\;\!\mathcolor{gray}{t}}_{\;\!\symup{z} \symup{z} \mathcolor{gray}{0}} \label{eq:H_x0=} \\ & + \mathcolor{gray}{\nabla^t} \left( 2~ \mathcolor{gray}{\nabla_x} O^{\;\!\mathcolor{gray}{t}}_{\;\! \mathcolor{gray}{\symup{x}} \symup{y} \symup{z} \mathcolor{gray}{0}} + 2~ \mathcolor{gray}{\nabla_y}  O^{\;\!\mathcolor{gray}{t}}_{\;\! \symup{y} \mathcolor{gray}{\symup{y}} \symup{z} \mathcolor{gray}{0}} + \mathcolor{gray}{\nabla_z} O^{\;\!\mathcolor{gray}{t}}_{\;\! \symup{y} \mathcolor{gray}{\symup{z}} \symup{z} \mathcolor{gray}{0}} - \mathcolor{gray}{\nabla_y} O^{\;\!\mathcolor{gray}{t}}_{\;\! \symup{z} \symup{z} \symup{z} \mathcolor{gray}{0}} \right)~, \\
	H^{\;\!\mathcolor{gray}{t}}_{\;\! \symup{y}\mathcolor{gray}{0}} = &\hphantom{+} \left( \mathcolor{gray}{\nabla^t} Q^{\;\!\mathcolor{gray}{t}}_{\;\! \symup{x} \symup{z} \mathcolor{gray}{0}} -
	{\alpha}^{\;\!\mathcolor{gray}{t}}_{\;\! \symup{x}\mathcolor{gray}{0}} \right) + \mathcolor{gray}{\nabla_y} N^{\;\!\mathcolor{gray}{t}}_{\;\! \symup{z} \symup{z} \mathcolor{gray}{0}} \label{eq:H_y0=} \\ & - \mathcolor{gray}{\nabla^t} \left( 2~ \mathcolor{gray}{\nabla_x} O^{\;\!\mathcolor{gray}{t}}_{\;\! \symup{x} \mathcolor{gray}{\symup{x}} \symup{z} \mathcolor{gray}{0}} + 2~ \mathcolor{gray}{\nabla_y}  O^{\;\!\mathcolor{gray}{t}}_{\;\! \symup{x} \mathcolor{gray}{\symup{y}} \symup{z} \mathcolor{gray}{0}} + \mathcolor{gray}{\nabla_z}  O^{\;\!\mathcolor{gray}{t}}_{\;\! \symup{x} \mathcolor{gray}{\symup{z}} \symup{z} \mathcolor{gray}{0}} - \mathcolor{gray}{\nabla_x}  O^{\;\!\mathcolor{gray}{t}}_{\;\! \symup{z} \symup{z} \symup{z} \mathcolor{gray}{0}} \right)~, \\
	H^{\;\!\mathcolor{gray}{t}}_{\;\! \symup{z} \mathcolor{gray}{0}} = & - M^{\;\!\mathcolor{gray}{t}}_{\;\! \symup{\iota} \mathcolor{gray}{z}} + \left[ 2~ \mathcolor{gray}{\nabla_x}
	N^{\;\!\mathcolor{gray}{t}}_{\;\! \mathcolor{gray}{\symup{x}}\symup{z} \mathcolor{gray}{0}} + 2~ \mathcolor{gray}{\nabla_y} N^{\;\!\mathcolor{gray}{t}}_{\;\! \mathcolor{gray}{\symup{y}} \symup{z} \mathcolor{gray}{0}} + \mathcolor{gray}{\nabla_z} N^{\;\!\mathcolor{gray}{t}}_{\;\! \mathcolor{gray}{\symup{z}} \symup{z} \mathcolor{gray}{0}} \right. \label{eq:H_z0=} \\ & \left. + \mathcolor{gray}{\nabla^t} \left( \mathcolor{gray}{\nabla_y}
	O^{\;\!\mathcolor{gray}{t}}_{\;\! \symup{x}\symup{z}\symup{z} \mathcolor{gray}{0}} - \mathcolor{gray}{\nabla_x}
	O^{\;\!\mathcolor{gray}{t}}_{\;\! \symup{y}\symup{z}\symup{z}\mathcolor{gray}{0}} \right) \right]~.
\end{align}
\end{subequations}
同样,从 \bref{eq:curl-DH-01-delta} 开始到 \bref{eq:H^(2-0)_0=},所有公式\Footnote{除了 \bref{eq:D^(0),eq:D^(0)=,eq:H^(0)=} 这些体项,即 ${\mathbb{1}}_{\mathcolor{gray}{z}} ~\textcolor{Maroon}{\text{项}}$ 中远离接触面 $\mathcolor{gray}{z \mathcolor{black}{=} 0}$ 的地方的对应项。}中的每一个源/场量 $X$,都需要继承并添补上 \bref{eq:curl-DH-01} 中的左上角介质标记 $\textcolor{Maroon}{\mathsfit{z}}$ 并对两侧半无限介质(在接触面 $\mathcolor{gray}{z \mathcolor{black}{=} 0}$ 处)乘以 $\leftindex_{\textcolor{Maroon}{\mathsfit{z}}} \;\! \delta_{\mathcolor{gray}{z}}$(或其他对应的 $\leftindex_{\textcolor{Maroon}{\mathsfit{z}}} \;\! \delta'_{\mathcolor{gray}{z}},\leftindex_{\textcolor{Maroon}{\mathsfit{z}}} \;\! \delta''_{\mathcolor{gray}{z}}$)并求和以成为 $\leftindex_{\textcolor{Maroon}{\mathsfit{z}}} \;\! \delta_{\mathcolor{gray}{z}} \leftindex^{\textcolor{Maroon}{\mathsfit{z}}} X$(或 $\leftindex_{\textcolor{Maroon}{\mathsfit{z}}} \;\! \delta'_{\mathcolor{gray}{z}} \leftindex^{\textcolor{Maroon}{\mathsfit{z}}} X,\leftindex_{\textcolor{Maroon}{\mathsfit{z}}} \;\! \delta''_{\mathcolor{gray}{z}} \leftindex^{\textcolor{Maroon}{\mathsfit{z}}} X$),才是有效的约束,和完整的\textcolor{Maroon}{边界条件}。

上述 \bref{eq:D_z0=,eq:H_x0=,eq:H_y0=},结合 \bref{eq:1BC} 或 \bref{eq:2BC} 给出了以下深刻结论:在\textcolor{NavyBlue}{电四-磁偶}极矩 $\bar{\bar{Q}}^{\;\!\mathcolor{gray}{t}}_{\;\!\mathcolor{gray}{z}}, \bar{M}^{\;\!\mathcolor{gray}{t}}_{\;\!\mathcolor{gray}{z}}$ 层次及以下,由场$+$源构成的两个\textcolor{NavyBlue}{辅助场}:电位移场 $\bar{D}^{\;\!\mathcolor{gray}{t}}_{\;\!\mathcolor{gray}{z}}$ 的法向分量 $D^{\;\!\mathcolor{gray}{t}}_{\;\! \symup{z} \mathcolor{gray}{z}}$、磁场 $\bar{H}^{\;\!\mathcolor{gray}{t}}_{\;\!\mathcolor{gray}{z}}$ 的切向分量 $\bar{H}^{\;\!\mathcolor{gray}{t}}_{\;\! \symup{\rho} \mathcolor{gray}{z}}$(在 $\mathcolor{gray}{z \mathcolor{black}{=} 0^+}$ 和 $\mathcolor{gray}{z \mathcolor{black}{=} 0^-}$ 处的值)均不\textcolor{Plum}{连续},即有 $\leftindex^{\textcolor{Maroon}{1}} D^{\;\!\mathcolor{gray}{t}}_{\;\! \symup{z} \mathcolor{gray}{0^+}} \neq \leftindex^{\textcolor{Maroon}{0}} D^{\;\!\mathcolor{gray}{t}}_{\;\! \symup{z} \mathcolor{gray}{0^-}}$ 以及 $\leftindex^{\textcolor{Maroon}{1}} {\bar{H}}^{\;\!\mathcolor{gray}{t}}_{\;\! \symup{\rho} \mathcolor{gray}{0^+}} \neq \leftindex^{\textcolor{Maroon}{0}} {\bar{H}}^{\;\!\mathcolor{gray}{t}}_{\;\! \symup{\rho} \mathcolor{gray}{0^-}}$。

由于矢量场 $\bar{A}^{\;\!\mathcolor{gray}{t}}_{\;\!\mathcolor{gray}{z}}$ 的旋度\textcolor{NavyBlue}{无源}、标量场 $\phi^{\;\!\mathcolor{gray}{t}}_{\;\!\mathcolor{gray}{z}}$ 的梯度无旋,\bref{eq:DH-01} 中的 $\bar{D}^{\;\!\mathcolor{gray}{t}}_{\;\!\mathcolor{gray}{z}}, \bar{H}^{\;\!\mathcolor{gray}{t}}_{\;\!\mathcolor{gray}{z}}$ 可以分别加上 $\mathcolor{gray}{\bar{\nabla} \times} \bar{A}^{\;\!\mathcolor{gray}{t}}_{\;\!\mathcolor{gray}{z}}$、$\mathcolor{gray}{\bar{\nabla}} \phi^{\;\!\mathcolor{gray}{t}}_{\;\!\mathcolor{gray}{z}}$ 而不影响 \bref{eq:Div-D}、\bref{eq:Curl-H} 的成立,因此\textcolor{NavyBlue}{辅助场} $\bar{D}^{\;\!\mathcolor{gray}{t}}_{\;\!\mathcolor{gray}{z}}$ $~\left(~ + \mathcolor{gray}{\bar{\nabla} \times} \bar{A}^{\;\!\mathcolor{gray}{t}}_{\;\!\mathcolor{gray}{z}} ~\right)~$ 和 $\bar{H}^{\;\!\mathcolor{gray}{t}}_{\;\!\mathcolor{gray}{z}}$ $~\left(~ + \mathcolor{gray}{\bar{\nabla}} \phi^{\;\!\mathcolor{gray}{t}}_{\;\!\mathcolor{gray}{z}} ~\right)~$ 都不是唯一确定的\Footnote{类似标势与矢势也需要某种“规范”(如库伦/洛伦兹规范)才能确定一样。\textcolor{Maroon}{多极理论}中的这种“规范”为:源的宏观表达式在选择的坐标原点的平移下需保持不变,也称本构张量的原点独立性。}。 ---  这一额外的\textcolor{Plum}{自由度}可以被用于确定在微观上与坐标原点无关的平移不变源 $\bar{P}^{\;\!\mathcolor{gray}{t}}_{\;\!\mathcolor{gray}{z}},\bar{\bar{Q}}^{\;\!\mathcolor{gray}{t}}_{\;\!\mathcolor{gray}{z}},\bar{\bar{\bar{O}}}^{\;\!\mathcolor{gray}{t}}_{\;\!\mathcolor{gray}{z}} ; \bar{M}^{\;\!\mathcolor{gray}{t}}_{\;\!\mathcolor{gray}{z}}, \bar{\bar{N}}^{\;\!\mathcolor{gray}{t}}_{\;\!\mathcolor{gray}{z}}$ (关于\textcolor{NavyBlue}{基本场} $\bar{E}^{\;\!\mathcolor{gray}{t}}_{\;\!\mathcolor{gray}{z}}, \bar{B}^{\;\!\mathcolor{gray}{t}}_{\;\!\mathcolor{gray}{z}}$ 的\textcolor{Maroon}{本构关系})的唯一确定且正确的宏观表达式\cite{welterTranslationallyInvariantSemiclassical2013,delangeTranslationalInvariancePost2012,langeTransitionMicroscopicMacroscopic2012,langeMultipoleTheoryHehl2015,raabCommentOriginDependence2010a,OriginindependentCalculationQuadrupole}\Footnote{微观表达式却不是唯一确定的\cite{OriginDependenceMaterial}。}。

源(动)是因,(生)场为果。在 ${\mathbb{1}}_{\mathcolor{gray}{z}} ~\textcolor{Maroon}{\text{项}}$ 中,$\bar{D}^{\;\!\mathcolor{gray}{t}}_{\;\!\mathcolor{gray}{z}}, \bar{H}^{\;\!\mathcolor{gray}{t}}_{\;\!\mathcolor{gray}{z}}$ 由\textcolor{NavyBlue}{基本场} $\bar{E}^{\;\!\mathcolor{gray}{t}}_{\;\!\mathcolor{gray}{z}}, \bar{B}^{\;\!\mathcolor{gray}{t}}_{\;\!\mathcolor{gray}{z}}$ 和束缚源 $\bar{P}^{\;\!\mathcolor{gray}{t}}_{\;\!\mathcolor{gray}{z}},\bar{\bar{Q}}^{\;\!\mathcolor{gray}{t}}_{\;\!\mathcolor{gray}{z}},\bar{\bar{\bar{O}}}^{\;\!\mathcolor{gray}{t}}_{\;\!\mathcolor{gray}{z}} ; \bar{M}^{\;\!\mathcolor{gray}{t}}_{\;\!\mathcolor{gray}{z}}, \bar{\bar{N}}^{\;\!\mathcolor{gray}{t}}_{\;\!\mathcolor{gray}{z}}$ 构成,因此称其为“\textcolor{NavyBlue}{辅助场}”,其本质为\textcolor{NavyBlue}{基本场}和束缚源的混合。相比而言,\textcolor{NavyBlue}{基本场} $\bar{E}^{\;\!\mathcolor{gray}{t}}_{\;\!\mathcolor{gray}{z}}, \bar{B}^{\;\!\mathcolor{gray}{t}}_{\;\!\mathcolor{gray}{z}}$ 在其 ${\mathbb{1}}_{\mathcolor{gray}{z}} ~\textcolor{Maroon}{\text{项}}$ 中的值,在远离\Footnote{在接近接触面 $\mathcolor{gray}{z \mathcolor{black}{=} 0}$ 的薄层内,\textcolor{NavyBlue}{基本场} $\bar{E}^{\;\!\mathcolor{gray}{t}}_{\;\!\mathcolor{gray}{z}}, \bar{B}^{\;\!\mathcolor{gray}{t}}_{\;\!\mathcolor{gray}{z}}$“在两种介质中的差”仍是束缚源 $\bar{P}^{\;\!\mathcolor{gray}{t}}_{\;\!\mathcolor{gray}{z}},\bar{\bar{Q}}^{\;\!\mathcolor{gray}{t}}_{\;\!\mathcolor{gray}{z}},\bar{\bar{\bar{O}}}^{\;\!\mathcolor{gray}{t}}_{\;\!\mathcolor{gray}{z}} ; \bar{M}^{\;\!\mathcolor{gray}{t}}_{\;\!\mathcolor{gray}{z}}, \bar{\bar{N}}^{\;\!\mathcolor{gray}{t}}_{\;\!\mathcolor{gray}{z}}$ 的显示多项式函数,如 \bref{eq:E_x0==,eq:E_y0==,eq:E_z0==} 和 \bref{eq:B_x0==,eq:B_y0==,eq:B_z0==} 所示。}接触面 $\mathcolor{gray}{z \mathcolor{black}{=} 0}$ 处,不是束缚源 $\bar{P}^{\;\!\mathcolor{gray}{t}}_{\;\!\mathcolor{gray}{z}},\bar{\bar{Q}}^{\;\!\mathcolor{gray}{t}}_{\;\!\mathcolor{gray}{z}},\bar{\bar{\bar{O}}}^{\;\!\mathcolor{gray}{t}}_{\;\!\mathcolor{gray}{z}} ; \bar{M}^{\;\!\mathcolor{gray}{t}}_{\;\!\mathcolor{gray}{z}}, \bar{\bar{N}}^{\;\!\mathcolor{gray}{t}}_{\;\!\mathcolor{gray}{z}}$ 的显示多项式函数,但仍因受到体区域 \textcolor{Maroon}{Maxwell-Lorentz-Heaviside} \bref{eq:curl-E,eq:div-B,eq:div-E,eq:curl-B}的约束,而是这些束缚源的隐式函数。总之,$\bar{D}^{\;\!\mathcolor{gray}{t}}_{\;\!\mathcolor{gray}{z}}, \bar{H}^{\;\!\mathcolor{gray}{t}}_{\;\!\mathcolor{gray}{z}};\bar{E}^{\;\!\mathcolor{gray}{t}}_{\;\!\mathcolor{gray}{z}}, \bar{B}^{\;\!\mathcolor{gray}{t}}_{\;\!\mathcolor{gray}{z}}$ 均直/间接地是 $\bar{P}^{\;\!\mathcolor{gray}{t}}_{\;\!\mathcolor{gray}{z}},\bar{\bar{Q}}^{\;\!\mathcolor{gray}{t}}_{\;\!\mathcolor{gray}{z}},\bar{\bar{\bar{O}}}^{\;\!\mathcolor{gray}{t}}_{\;\!\mathcolor{gray}{z}} ; \bar{M}^{\;\!\mathcolor{gray}{t}}_{\;\!\mathcolor{gray}{z}}, \bar{\bar{N}}^{\;\!\mathcolor{gray}{t}}_{\;\!\mathcolor{gray}{z}}$ 的函数。

反过来,\bref{sec:maxwell} 整体(直到 \bref{ssec:DH-boundary} 末的这里),直接通过多项式组成\textcolor{NavyBlue}{辅助场} $\bar{D}^{\;\!\mathcolor{gray}{t}}_{\;\!\mathcolor{gray}{z}}, \bar{H}^{\;\!\mathcolor{gray}{t}}_{\;\!\mathcolor{gray}{z}}$,以及间接通过 \textcolor{Maroon}{Maxwell-Lorentz-Heaviside} \bref{eq:curl-E,eq:div-B,eq:div-E,eq:curl-B} 控制\textcolor{NavyBlue}{基本场} $\bar{E}^{\;\!\mathcolor{gray}{t}}_{\;\!\mathcolor{gray}{z}}, \bar{B}^{\;\!\mathcolor{gray}{t}}_{\;\!\mathcolor{gray}{z}}$ 的介质/束缚源部分 $\bar{P}^{\;\!\mathcolor{gray}{t}}_{\;\!\mathcolor{gray}{z}},\bar{\bar{Q}}^{\;\!\mathcolor{gray}{t}}_{\;\!\mathcolor{gray}{z}},\bar{\bar{\bar{O}}}^{\;\!\mathcolor{gray}{t}}_{\;\!\mathcolor{gray}{z}} ; \bar{M}^{\;\!\mathcolor{gray}{t}}_{\;\!\mathcolor{gray}{z}}, \bar{\bar{N}}^{\;\!\mathcolor{gray}{t}}_{\;\!\mathcolor{gray}{z}}$,暂时不写作 \textcolor{NavyBlue}{基本场} $\bar{E}^{\;\!\mathcolor{gray}{t}}_{\;\!\mathcolor{gray}{z}}, \bar{B}^{\;\!\mathcolor{gray}{t}}_{\;\!\mathcolor{gray}{z}}$ 或任何其他(引力、应力、温度)场的显示函数。

\clearpage

\vspace*{-9.0em}

\marginLeft[-2.4em]{sec:constitutive}\section{\textcolor{Maroon}{Constitutive relations} 本构关系:\textcolor{Maroon}{$\text{源} = g(\text{场})$}}\label{sec:constitutive}

该节从经典/现代的\textcolor{Maroon}{多极理论}视角,查看源关于场的(\textcolor{Plum}{非线性}) \textcolor{Maroon}{Constitutive Relations} \textcolor{Maroon}{本构关系}。 --- 对应“物质告诉时空怎么弯曲,时空告诉物质怎么运动”的后半句:即“场告诉源怎么运动”。

\vspace*{-4.0em}

\marginLeft[-2.4em]{ssec:PMQN-nonlinear}\subsection{Volterra 级数 $\longrightarrow$ 非线性多极理论}\label{ssec:PMQN-nonlinear}

材料的束缚源 ${\rho}^{\;\!\mathcolor{gray}{t}}_{\;\!\textcolor{Maroon}{\text{b}}\mathcolor{gray}{z}}, \bar{J}^{\;\!\mathcolor{gray}{t}}_{\;\!\textcolor{Maroon}{\text{b}}\mathcolor{gray}{z}};\bar{P}^{\;\!\mathcolor{gray}{t}}_{\;\!\mathcolor{gray}{z}},\bar{\bar{Q}}^{\;\!\mathcolor{gray}{t}}_{\;\!\mathcolor{gray}{z}},\bar{\bar{\bar{O}}}^{\;\!\mathcolor{gray}{t}}_{\;\!\mathcolor{gray}{z}} ; \bar{M}^{\;\!\mathcolor{gray}{t}}_{\;\!\mathcolor{gray}{z}}, \bar{\bar{N}}^{\;\!\mathcolor{gray}{t}}_{\;\!\mathcolor{gray}{z}}$ 部分,可能存在不含时 $\mathcolor{gray}{t}$ 的永久/自发/固有(宏观)极/磁化\textcolor{Plum}{多极}矩 $\bar{P}^{\;\!\textcolor{Maroon}{(0)}}_{\;\!\mathcolor{gray}{z}},\bar{\bar{Q}}^{\;\!\textcolor{Maroon}{(0)}}_{\;\!\mathcolor{gray}{z}},\bar{\bar{\bar{O}}}^{\;\!\textcolor{Maroon}{(0)}}_{\;\!\mathcolor{gray}{z}} ; \bar{M}^{\;\!\textcolor{Maroon}{(0)}}_{\;\!\mathcolor{gray}{z}}, \bar{\bar{N}}^{\;\!\textcolor{Maroon}{(0)}}_{\;\!\mathcolor{gray}{z}}$。即没有被施加外场时,其结构自身就具有的极/磁化\textcolor{Plum}{多极}矩,对应材料\Footnote{这里的“材料”具体指:原/离子=多电子体系,或晶格 or 分子 or 化学键\cite{boydNonlinearOptics2019} or 基团=多原/离子体系。}\textcolor{NavyBlue}{波函数}的\textcolor{NavyBlue}{哈密顿量} $\bar{H}_0$(的\textcolor{NavyBlue}{本征态}中的\textcolor{NavyBlue}{基态})。典型地如非中心对称晶体结构的铁电/铁磁材料,具有永久\textcolor{NavyBlue}{电}/\textcolor{NavyBlue}{磁偶极矩}$\bar{P}^{\;\!\textcolor{Maroon}{(0)}}_{\;\!\mathcolor{gray}{z}}, \bar{M}^{\;\!\textcolor{Maroon}{(0)}}_{\;\!\mathcolor{gray}{z}}$。中心对称的有机材料中的苯环结构、反铁电/反铁磁材料等,具有永久\textcolor{NavyBlue}{电}/\textcolor{NavyBlue}{磁四极矩}$\bar{\bar{Q}}^{\;\!\textcolor{Maroon}{(0)}}_{\;\!\mathcolor{gray}{z}};\bar{\bar{N}}^{\;\!\textcolor{Maroon}{(0)}}_{\;\!\mathcolor{gray}{z}}$。液晶、超材料、复合材料、多铁材料等可以同时具有永久\textcolor{NavyBlue}{电}、\textcolor{NavyBlue}{磁}\textcolor{Plum}{多极}矩$\bar{P}^{\;\!\textcolor{Maroon}{(0)}}_{\;\!\mathcolor{gray}{z}},\bar{\bar{Q}}^{\;\!\textcolor{Maroon}{(0)}}_{\;\!\mathcolor{gray}{z}},\bar{\bar{\bar{O}}}^{\;\!\textcolor{Maroon}{(0)}}_{\;\!\mathcolor{gray}{z}} ; \bar{M}^{\;\!\textcolor{Maroon}{(0)}}_{\;\!\mathcolor{gray}{z}}, \bar{\bar{N}}^{\;\!\textcolor{Maroon}{(0)}}_{\;\!\mathcolor{gray}{z}}; \cdots$。

材料的束缚源和自由源 ${\rho}^{\;\!\mathcolor{gray}{t}}_{\;\!\textcolor{Maroon}{\text{b}}\mathcolor{gray}{z}}, \bar{J}^{\;\!\mathcolor{gray}{t}}_{\;\!\textcolor{Maroon}{\text{b}}\mathcolor{gray}{z}};{\rho}^{\;\!\mathcolor{gray}{t}}_{\;\!\textcolor{Maroon}{\text{f}}\mathcolor{gray}{z}}, \bar{J}^{\;\!\mathcolor{gray}{t}}_{\;\!\textcolor{Maroon}{\text{f}}\mathcolor{gray}{z}}$ 中的每一个,均可能会产生(含时 $\mathcolor{gray}{t}$ 或不含时 $\mathcolor{gray}{t}$ 的)感生/诱导/响应电荷/流\cite{markelExternalInducedFree2018,raabMultipoleTheoryElectromagnetism2004,tsukermanPolarizationArbitraryCharge2021a}和极/磁化\textcolor{Plum}{多极}矩。即与外场相互作用时,材料\textcolor{NavyBlue}{哈密顿量} $\bar{H} = \bar{H}_0 + \bar{V}$ 中多出的(由外场控制的)势能项 $\Delta \bar{H} = \bar{V}$\cite{boydNonlinearOptics2019,raabMultipoleTheoryElectromagnetism2004},所对应的电荷体系(相对于无外场时的)额外重新分布。值得注意的是,受场影响的带电体系重新分布后的终态(可能含时 $\mathcolor{gray}{t}$ 且不是热力学平衡态),减去原始分布的初态之差 $\Delta \bar{H}$,相对于初态 $\bar{H}_0$ 不一定是\textcolor{NavyBlue}{微扰} $\bar{H}'$\cite{boydNonlinearOptics2019},即可能有 $\bar{H}_0 \ll \Delta \bar{H}$,此时体现为材料对场的\textcolor{Plum}{线性}/\textcolor{Plum}{非线性}\textcolor{NavyBlue}{共振}、(饱和)\textcolor{NavyBlue}{吸收},及其可能导致的电离/击穿\cite{boydNonlinearOptics2019}、结构损伤、极化反转、材料改性\cite{xuFemtosecondLaserWriting2022,weiExperimentalDemonstrationThreedimensional2018,xuThreedimensionalNonlinearPhotonic2018,keren-zurNewDimensionNonlinear2018}、电光声强耦合、电滞/磁滞现象等。

当外/主场、驱动场是直/交流(对应不含时$\mathcolor{gray}{t}$/含时$\mathcolor{gray}{t}$、静态/动态)时,材料束缚/自由源也会因外场的存在(or 变化)而变化,即因果/迟滞地被外/主场驱动而产生回/反/响应。它们作为次\textcolor{NavyBlue}{波源},辐射出新/次场,并与外/主场叠加(成\textcolor{NavyBlue}{辅助场} $\bar{D}^{\;\!\mathcolor{gray}{t}}_{\;\!\mathcolor{gray}{z}}, \bar{H}^{\;\!\mathcolor{gray}{t}}_{\;\!\mathcolor{gray}{z}}$,如果外/主场是\textcolor{NavyBlue}{基本场} $\bar{E}^{\;\!\mathcolor{gray}{t}}_{\;\!\mathcolor{gray}{z}}, \bar{B}^{\;\!\mathcolor{gray}{t}}_{\;\!\mathcolor{gray}{z}}$ 的话)。不论外场是静态的还是动态的,束缚/自由源对外场的响应,都可能是外场及其导数的\textcolor{Plum}{非线性}函数,甚至包含与外场(的函数)在时域和/或空域上的卷积,对应材料的因果/推迟和\textcolor{Plum}{非局域}响应。

外场可以是电磁/引力/应力/温度场(它们都与“力”有关)。因此,电磁源 ${\rho}^{\;\!\mathcolor{gray}{t}}_{\;\!\mathcolor{gray}{z}}, \bar{J}^{\;\!\mathcolor{gray}{t}}_{\;\!\mathcolor{gray}{z}}$ 的自变(场)量是复杂的,许多因素都可以触发并更新源的分布 --- 以至于$\text{源} = f(\text{场})$即\textcolor{Maroon}{本构关系}本身,就涵盖了理论和应用物理学中的大面积主题\Footnote{超导/量子/固体/凝聚态/半导体/天体/等离子体物理、应用磁学/超声探测、流体力学、激光/电/探针/化学加工、材料/统计/计算物理/化学等。},使得本文既无法从细节上面面俱到,也无法从大局上给出其统一形式。

在可以同时扮演\textcolor{NavyBlue}{驱动源}角色的众多\textcolor{NavyBlue}{主动}场中,本文把重心落脚在\textcolor{NavyBlue}{基本场} $\bar{E}^{\;\!\mathcolor{gray}{t}}_{\;\!\mathcolor{gray}{z}}, \bar{B}^{\;\!\mathcolor{gray}{t}}_{\;\!\mathcolor{gray}{z}}$ 上,并适当忽视引力/应力/温度场等带来的\textcolor{NavyBlue}{多物理场耦合}效应。即,在将场作为\textcolor{gray}{自变量}方面,本文(在\textcolor{Maroon}{本构关系}中)只将源 ${\rho}^{\;\!\mathcolor{gray}{t}}_{\;\!\mathcolor{gray}{z}}, \bar{J}^{\;\!\mathcolor{gray}{t}}_{\;\!\mathcolor{gray}{z}}$ 视为\textcolor{NavyBlue}{基本场} $\bar{E}^{\;\!\mathcolor{gray}{t}}_{\;\!\mathcolor{gray}{z}}, \bar{B}^{\;\!\mathcolor{gray}{t}}_{\;\!\mathcolor{gray}{z}}$ 的(\textcolor{Plum}{线性}、\textcolor{Plum}{非线性})函数,使得至少底层变量只包含 $\bar{E}^{\;\!\mathcolor{gray}{t}}_{\;\!\mathcolor{gray}{z}}, \bar{B}^{\;\!\mathcolor{gray}{t}}_{\;\!\mathcolor{gray}{z}}$:即,中间过程可以还同时是热/声/应力/温度(梯度)场的函数,但这些中间场变量,也仅/纯粹由 $\bar{E}^{\;\!\mathcolor{gray}{t}}_{\;\!\mathcolor{gray}{z}}, \bar{B}^{\;\!\mathcolor{gray}{t}}_{\;\!\mathcolor{gray}{z}}$ 引起或导致,即共享相同的最多 2 个底层场量 $\bar{E}^{\;\!\mathcolor{gray}{t}}_{\;\!\mathcolor{gray}{z}}, \bar{B}^{\;\!\mathcolor{gray}{t}}_{\;\!\mathcolor{gray}{z}}$,也即最底层的驱动力源与 $\bar{E}^{\;\!\mathcolor{gray}{t}}_{\;\!\mathcolor{gray}{z}}, \bar{B}^{\;\!\mathcolor{gray}{t}}_{\;\!\mathcolor{gray}{z}}$ 以外的其他场量无关。

电磁源 ${\rho}^{\;\!\mathcolor{gray}{t}}_{\;\!\mathcolor{gray}{z}}, \bar{J}^{\;\!\mathcolor{gray}{t}}_{\;\!\mathcolor{gray}{z}}$ 中,束缚源 ${\rho}^{\;\!\mathcolor{gray}{t}}_{\;\!\textcolor{Maroon}{\text{b}}\mathcolor{gray}{z}}, \bar{J}^{\;\!\mathcolor{gray}{t}}_{\;\!\textcolor{Maroon}{\text{b}}\mathcolor{gray}{z}}$ 的各阶极化/磁化\textcolor{Plum}{多极}矩 $\bar{P}^{\;\!\mathcolor{gray}{t}}_{\;\!\mathcolor{gray}{z}},\bar{\bar{Q}}^{\;\!\mathcolor{gray}{t}}_{\;\!\mathcolor{gray}{z}},\bar{\bar{\bar{O}}}^{\;\!\mathcolor{gray}{t}}_{\;\!\mathcolor{gray}{z}} ; \bar{M}^{\;\!\mathcolor{gray}{t}}_{\;\!\mathcolor{gray}{z}}, \bar{\bar{N}}^{\;\!\mathcolor{gray}{t}}_{\;\!\mathcolor{gray}{z}}$ 均分别有自己的(关于 $\bar{E}^{\;\!\mathcolor{gray}{t}}_{\;\!\mathcolor{gray}{z}}, \bar{B}^{\;\!\mathcolor{gray}{t}}_{\;\!\mathcolor{gray}{z}}$ 的)\textcolor{Plum}{线性}、\textcolor{Plum}{非线性}(函数),即“\textcolor{Plum}{多极}矩的阶”与其“\textcolor{Plum}{非线性}的阶”是无关的,正如其也与(源的)“\textcolor{Plum}{奇异}层次的阶”无关一样(见 \bref{ssec:step-delta}):这是三个独立的\textcolor{Plum}{自由度},三者两两无关。

在所有的束缚/自由源 ${\rho}^{\;\!\mathcolor{gray}{t}}_{\;\!\textcolor{Maroon}{\text{f}}\mathcolor{gray}{z}}, \bar{J}^{\;\!\mathcolor{gray}{t}}_{\;\!\textcolor{Maroon}{\text{f}}\mathcolor{gray}{z}};{\rho}^{\;\!\mathcolor{gray}{t}}_{\;\!\textcolor{Maroon}{\text{b}}\mathcolor{gray}{z}}, \bar{J}^{\;\!\mathcolor{gray}{t}}_{\;\!\textcolor{Maroon}{\text{b}}\mathcolor{gray}{z}};\bar{P}^{\;\!\mathcolor{gray}{t}}_{\;\!\mathcolor{gray}{z}},\bar{\bar{Q}}^{\;\!\mathcolor{gray}{t}}_{\;\!\mathcolor{gray}{z}},\bar{\bar{\bar{O}}}^{\;\!\mathcolor{gray}{t}}_{\;\!\mathcolor{gray}{z}} ; \bar{M}^{\;\!\mathcolor{gray}{t}}_{\;\!\mathcolor{gray}{z}}, \bar{\bar{N}}^{\;\!\mathcolor{gray}{t}}_{\;\!\mathcolor{gray}{z}}$ 中,该 \bref{ssec:PMQN-nonlinear} 适合从读者熟悉的感应\Footnote{永久 $\bar{P}^{\;\!\textcolor{Maroon}{(0)}}_{\;\!\mathcolor{gray}{z}}$ 对麦氏方程组的解的影响,与直流/静态的感应 $\bar{P}^{\;\!}_{\;\!\mathcolor{gray}{z}}$ 对外场的影响类似,因此无需特殊考虑 $\bar{P}^{\;\!\textcolor{Maroon}{(0)}}_{\;\!\mathcolor{gray}{z}}$。}电偶极矩体密度\Footnote{这里的体密度并不对应 ${\mathbb{1}}_{\mathcolor{gray}{z}} ~\textcolor{Maroon}{\text{项}}$ or 体项:只能针对源 ${\rho}^{\;\!\mathcolor{gray}{t}}_{\;\!\mathcolor{gray}{z}}, \bar{J}^{\;\!\mathcolor{gray}{t}}_{\;\!\mathcolor{gray}{z}}$ 或场 $\bar{E}^{\;\!\mathcolor{gray}{t}}_{\;\!\mathcolor{gray}{z}}, \bar{B}^{\;\!\mathcolor{gray}{t}}_{\;\!\mathcolor{gray}{z}};\bar{D}^{\;\!\mathcolor{gray}{t}}_{\;\!\mathcolor{gray}{z}}, \bar{H}^{\;\!\mathcolor{gray}{t}}_{\;\!\mathcolor{gray}{z}}$ 提\textcolor{Plum}{奇异}层次,\textcolor{Plum}{多极}矩本身没有“表面项”或“体项”,只是组成了源和场的“表面项”和“体项”,见 \bref{ssec:step-delta}。} $\bar{P}^{\;\!\mathcolor{gray}{t}}_{\;\!\mathcolor{gray}{z}}$(关于 $\bar{E}^{\;\!\mathcolor{gray}{t}}_{\;\!\mathcolor{gray}{z}}, \bar{B}^{\;\!\mathcolor{gray}{t}}_{\;\!\mathcolor{gray}{z}}$ 的) \textcolor{Plum}{线性}、\textcolor{Plum}{非线性}(函数)写起。为包含\textcolor{NavyBlue}{共振}、强场、等离子体、\textcolor{NavyBlue}{多物理场耦合}所带来的\textcolor{Plum}{非线性},$\bar{P}^{\;\!\mathcolor{gray}{t}}_{\;\!\mathcolor{gray}{z}} \left( \bar{E}^{\;\!\mathcolor{gray}{t}}_{\;\!\mathcolor{gray}{z}}, \bar{B}^{\;\!\mathcolor{gray}{t}}_{\;\!\mathcolor{gray}{z}} \right)$ 的表达式最多只能在\textcolor{Plum}{数学}形式上写作 $P^{\;\!\mathcolor{gray}{t}}_{\;\! \symup{\iota}\mathcolor{gray}{z}} \left( \mathcolor{gray}{\nabla^t}, \mathcolor{gray}{\nabla_{\hat{1}}} ; E^{\;\!\mathcolor{gray}{t}}_{\;\! \hat{2}\mathcolor{gray}{z}}, B^{\;\!\mathcolor{gray}{t}}_{\;\! \hat{3}\mathcolor{gray}{z}} \right)$ 而无法提供更多的微观或宏观层面的信息量;同时,$P^{\;\!\mathcolor{gray}{t}}_{\;\! \symup{\iota}\mathcolor{gray}{z}}$ 中的各\textcolor{Plum}{线性}、\textcolor{Plum}{非线性}项中的材料系数(各阶极化率),即“系统状态”如原子能级、分子振/转动能级、电子云晶格场离子声子格波能带,也将同样与 $\bar{E}^{\;\!\mathcolor{gray}{t}}_{\;\!\mathcolor{gray}{z}}, \bar{B}^{\;\!\mathcolor{gray}{t}}_{\;\!\mathcolor{gray}{z}}$ 的各分量的各阶时空导数有关,并有自己的$\text{极化率} = f(\text{场})$ \textcolor{Maroon}{本构关系}。然而由于其\textcolor{Plum}{线性}项的各阶极化率本身就与 $\bar{E}^{\;\!\mathcolor{gray}{t}}_{\;\!\mathcolor{gray}{z}}, \bar{B}^{\;\!\mathcolor{gray}{t}}_{\;\!\mathcolor{gray}{z}}$ 有关,而使得该\textcolor{Plum}{线性}项总体(=极化率$\cdot$场)自带\textcolor{Plum}{非线性},以至于抹去了剩余对应\textcolor{Plum}{非线性}项的存在意义,并到头来最终抹去了\textcolor{Plum}{线性}项和一阶极化率本身的存在意义\cite{boydNonlinearOptics2019},直至只剩下 $P^{\;\!\mathcolor{gray}{t}}_{\;\! \symup{\iota}\mathcolor{gray}{z}} \left( \mathcolor{gray}{\nabla^t}, \mathcolor{gray}{\nabla_{\hat{1}}} ; E^{\;\!\mathcolor{gray}{t}}_{\;\! \hat{2}\mathcolor{gray}{z}}, B^{\;\!\mathcolor{gray}{t}}_{\;\! \hat{3}\mathcolor{gray}{z}} \right)$ 表达式整体。

当然,以激光加工为例,各阶极化率(物质结构)是在激光加工过程中被$\text{极化率} = f(\text{场})$\textcolor{Maroon}{本构关系}改变的,但对 $P^{\;\!\mathcolor{gray}{t}}_{\;\! \symup{\iota}\mathcolor{gray}{z}}$ 的评估是在激光加工结束后,再对改变后的晶格结构终态,所对应的 $P'^{\;\!\mathcolor{gray}{t}}_{\;\! \symup{\iota}\mathcolor{gray}{z}}$ 进行计算的 --- 当这两个过程\Footnote{过程 1:晶体结构被强外场作用时/后发生永久改变,该过程晶格结构的热力学转变,关于强场的函数。过程 2:改变后的晶格结构所对应的 $P'^{\;\!\mathcolor{gray}{t}}_{\;\! \symup{\iota}\mathcolor{gray}{z}}$,关于弱场的函数。}分开时,可以独立对各阶极化率构建\textcolor{Maroon}{本构关系},否则不如直接对 $P^{\;\!\mathcolor{gray}{t}}_{\;\! \symup{\iota}\mathcolor{gray}{z}}$ 构建\textcolor{Maroon}{本构关系}。此外,$P^{\;\!\mathcolor{gray}{t}}_{\;\! \symup{\iota}\mathcolor{gray}{z}} \left( \mathcolor{gray}{\nabla^t}, \mathcolor{gray}{\nabla_{\hat{1}}} ; E^{\;\!\mathcolor{gray}{t}}_{\;\! \hat{2}\mathcolor{gray}{z}}, B^{\;\!\mathcolor{gray}{t}}_{\;\! \hat{3}\mathcolor{gray}{z}} \right)$ 表达式本身还有个问题:它的空间\textcolor{NavyBlue}{色散}/\textcolor{Plum}{非局域}项,只能通过空域偏导 $\mathcolor{gray}{\nabla_{\hat{1}}}$ 体现,无法通过\textcolor{Plum}{空域卷积}/对不同地点处的乘积求和体现。

上述材料在与外场作用后,即使撤去外场,材料的终态相对初态性质改变了的系统,属于\textcolor{Plum}{非线性}时变系统:对于不同时刻的相同驱动场\textcolor{Plum}{输入},材料可能有不同的响应\textcolor{Plum}{输出},对应的 $\text{源} = f(\text{场})$ \textcolor{Maroon}{本构关系} $P^{\;\!\mathcolor{gray}{t}}_{\;\! \symup{\iota}\mathcolor{gray}{z}} \left( \mathcolor{gray}{\nabla^t}, \mathcolor{gray}{\nabla_{\hat{1}}} ; E^{\;\!\mathcolor{gray}{t}}_{\;\! \hat{2}\mathcolor{gray}{z}}, B^{\;\!\mathcolor{gray}{t}}_{\;\! \hat{3}\mathcolor{gray}{z}} \right)$ 以及 $\text{材料系数} = f(\text{场})$ \textcolor{Maroon}{本构关系},均没有一般的表达式。因此,为得到可分析的对象,本文主要考虑\textcolor{Plum}{非线性}时不变系统\cite{zalevskyOpticalImplementationSecondorder2001}\Footnote{系统的响应\textcolor{Plum}{输出},与系统状态、\textcolor{Plum}{输入}时刻无关(时间平移不变),只与其他\textcolor{Plum}{输入}参数有关的\textcolor{Plum}{非线性}系统。},即对于不同时刻的相同驱动场\textcolor{Plum}{输入},材料的响应\textcolor{Plum}{输出}是相同的系统,也即材料电子云分布/晶格场状态在撤去外场后等于施加外场前的系统\Footnote{但允许在外场持续作用时,材料常数的弹性强烈变化,如\textcolor{NavyBlue}{拉比振荡}。然而一旦驱动场被撤走,材料常数/体系\textcolor{NavyBlue}{波函数}必须复原(以迎接下一次驱动)。复原的时间既可以很短,如\textcolor{NavyBlue}{电磁诱导透明}、\textcolor{NavyBlue}{受激拉曼散射};也可能很长甚至可调,如光折变效应;也可以根本不复原,如激光直写/雕刻或加工。}。然而即使只考虑\textcolor{Plum}{非线性}时不变\Footnote{变与不变,需要指定起终点/态。这里指的是材料受到驱动前(的平衡态)为始态,\textcolor{NavyBlue}{驱动源}被撤销、并且材料恢复平衡态后(的态)为终态。}系统,强场、\textcolor{NavyBlue}{非微扰}、强/级联/多对象(格波、声波、超声波、超晶格振动、自旋波、激子)耦合的情况下,在材料与场的强相互作用的过程中,$P^{\;\!\mathcolor{gray}{t}}_{\;\! \symup{\iota}\mathcolor{gray}{z}} \left( \mathcolor{gray}{\nabla^t}, \mathcolor{gray}{\nabla_{\hat{1}}} ; E^{\;\!\mathcolor{gray}{t}}_{\;\! \hat{2}\mathcolor{gray}{z}}, B^{\;\!\mathcolor{gray}{t}}_{\;\! \hat{3}\mathcolor{gray}{z}} \right)$ 以及 $\text{材料系数} = f(\text{场})$ 仍不一定能展成收敛的级数形式,典型地如\textcolor{Plum}{非线性}(如包含电/磁致伸缩项的)\textcolor{NavyBlue}{压电}/\textcolor{NavyBlue}{磁方程}、\textcolor{Plum}{非线性}(如包含太赫兹驱动场和阻尼/晶格振动耗散项的)\textcolor{NavyBlue}{黄昆方程}\cite{wuqiangShouJiShengZiJiHuaJiYuanYuTaiHeZiGuangWuLiTeYao2024}、\textcolor{NavyBlue}{光学双稳态}\cite{boydNonlinearOptics2019}等。同样,本文也不考虑上述无法用 \textcolor{Maroon}{Volterra 级数}\cite{pintoExactVolterraseriesComputation1982,shenNonlinearOpticalSusceptibilities2001}描述的、材料系数与外场(瞬时弹性)相关的\textcolor{Plum}{非线性}时不变系统。

本文将自身局限于所有可以用“材料系数与外场无关的 \textcolor{Maroon}{Volterra 级数}\cite{pintoExactVolterraseriesComputation1982,shenNonlinearOpticalSusceptibilities2001}”描述的\textcolor{Plum}{非线性}\textcolor{NavyBlue}{光学}现象。此外,由于 \textcolor{Maroon}{Volterra 级数}仍然过于广义了,本文还将继续把自身锚定于 \bref{eq:P_wk2} 中所有阶的卷积核 $\chi^{\;\! \mathcolor{gray}{\omega} \hat{1}}_{\;\! \symup{\iota} \mathcolor{gray}{\bar{k}} \textcolor{Maroon}{(1)}}, \chi^{\;\! \mathcolor{gray}{\omega} \hat{1} \hat{2}}_{\;\! \symup{\iota} \mathcolor{gray}{\bar{k}} \textcolor{Maroon}{(2)}}, \chi^{\;\! \mathcolor{gray}{\omega} \hat{1} \hat{2} \hat{3}}_{\;\! \symup{\iota} \mathcolor{gray}{\bar{k}} \textcolor{Maroon}{(3)}}$ 全都是只关于 $\mathcolor{gray}{\omega}, \mathcolor{gray}{\bar{k}}$ 的单\textcolor{Plum}{线性}卷积核\Footnote{卷积核可以是(更多正交/独立变量的)双\textcolor{Plum}{线性}、多\textcolor{Plum}{线性}的(函数),即它们除了是 $\mathcolor{gray}{\omega}, \mathcolor{gray}{\bar{k}}$ 的函数外,还可以是多重单\textcolor{Plum}{线性}卷积(如 $E^{\;\!\mathcolor{gray}{\omega}}_{\;\! \hat{1} \mathcolor{gray}{\bar{k}}} ~\mathcolor{gray}{\widetilde \circledast}~ E^{\;\!\mathcolor{gray}{\omega}}_{\;\! \hat{2} \mathcolor{gray}{\bar{k}}} ~\mathcolor{gray}{\widetilde \circledast}~ E^{\;\!\mathcolor{gray}{\omega}}_{\;\! \hat{3} \mathcolor{gray}{\bar{k}}}$)中,多重积分内所有被积变量的函数,以至于无法从积分中提取出来,也无法通过\textcolor{Plum}{傅立叶正}/\textcolor{Plum}{逆变换},将其转化为频/空域的乘积。},并因此可以从积分号内(从被积表达式中)提出来的框架内,以适应\textcolor{Maroon}{多极理论}的“短程”\Footnote{只是\textcolor{Maroon}{多极理论}中的微分,相对于 \textcolor{Maroon}{Volterra 级数}的积分,不是那么长程。但相对于原子/晶格量级,\textcolor{Maroon}{多极理论}所能描述的现象仍然已属于长程相互作用,以至于在绝大多数情况下是完全足够的;此外,时空域的超距作用在多大程度上适用,爱因斯坦应该会投否定票。 ---  尽管 \textcolor{Maroon}{Volterra 级数}仍能海纳百川地描述任何程度的非瞬时\textcolor{Plum}{非局域}现象。然而,其代价便是,计算量是巨大的,这也是本文想要规避的一点。}\textcolor{Plum}{非局域}近似(而不是 \textcolor{Maroon}{Volterra 级数}可以进一步描述的长程\textcolor{Plum}{非局域}现象)。此时 $P^{\;\!\mathcolor{gray}{t}}_{\;\! \symup{\iota}\mathcolor{gray}{z}} \left( \mathcolor{gray}{\nabla^t}, \mathcolor{gray}{\nabla_{\hat{1}}} ; E^{\;\!\mathcolor{gray}{t}}_{\;\! \hat{2}\mathcolor{gray}{z}}, B^{\;\!\mathcolor{gray}{t}}_{\;\! \hat{3}\mathcolor{gray}{z}} \right)$ 可以进一步写为\cite{teixeiraOpticalTransmissionModeling2013,andreasczylwikNonlinearSystemModeling1986,shenNonlinearOpticalSusceptibilities2001,zalevskyOpticalImplementationSecondorder2001,zhangNonlinearQuantumInputoutput2014}
\begin{subequations}
	\abovedisplayskip=10pt
	\belowdisplayskip=12pt
\begin{align}
	P^{\;\!\mathcolor{gray}{t}}_{\;\! \symup{\iota}\mathcolor{gray}{\bar{r}}} &= \mathcolor{gray}{\mathcal F_{\bar{x}}^{-1}} \left[ P^{\;\! \mathcolor{gray}{\omega}}_{\;\! \symup{\iota}\mathcolor{gray}{\bar{k}}} \right] = P^{\;\! \textcolor{Maroon}{(1)} \mathcolor{gray}{t}}_{\;\! \symup{\iota}\mathcolor{gray}{\bar{r}}} + P^{\;\! \textcolor{Maroon}{(2)} \mathcolor{gray}{t}}_{\;\! \symup{\iota}\mathcolor{gray}{\bar{r}}} + P^{\;\! \textcolor{Maroon}{(3)} \mathcolor{gray}{t}}_{\;\! \symup{\iota}\mathcolor{gray}{\bar{r}}} + \cdots \\ &= {\symup{\varepsilon_0}} \left\{ \chi^{\;\! \mathcolor{gray}{t} \hat{1}}_{\;\! \symup{\iota} \mathcolor{gray}{\bar{r}} \textcolor{Maroon}{(1)}} ~\mathcolor{gray}{\widetilde \circledast}~ E^{\;\!\mathcolor{gray}{t}}_{\;\! \hat{1} \mathcolor{gray}{\bar{r}}} + \chi^{\;\! \mathcolor{gray}{t} \hat{1} \hat{2}}_{\;\! \symup{\iota} \mathcolor{gray}{\bar{r}} \textcolor{Maroon}{(2)}}~\mathcolor{gray}{\widetilde \circledast}\left( E^{\;\!\mathcolor{gray}{t}}_{\;\! \hat{1} \mathcolor{gray}{\bar{r}}} E^{\;\!\mathcolor{gray}{t}}_{\;\! \hat{2} \mathcolor{gray}{\bar{r}}} \right) \right. \label{eq:P_tr} \\ &+ \left. \chi^{\;\! \mathcolor{gray}{t} \hat{1} \hat{2} \hat{3}}_{\;\! \symup{\iota} \mathcolor{gray}{\bar{r}} \textcolor{Maroon}{(3)}}~\mathcolor{gray}{\widetilde \circledast}\left( E^{\;\!\mathcolor{gray}{t}}_{\;\! \hat{1} \mathcolor{gray}{\bar{r}}} E^{\;\!\mathcolor{gray}{t}}_{\;\! \hat{2} \mathcolor{gray}{\bar{r}}} E^{\;\!\mathcolor{gray}{t}}_{\;\! \hat{3} \mathcolor{gray}{\bar{r}}} \right) + \cdots \right\}~,
\end{align}
\end{subequations}
其中,$\mathcolor{gray}{\widetilde \circledast}$ 在这里定义为 $\mathcolor{gray}{\bar{x}} := \left( \mathcolor{gray}{t}, \mathcolor{gray}{\bar{r}} \right)$ 域\Footnote{严格地说,$\mathcolor{gray}{\bar{x}} := \left( \mathcolor{gray}{t},~ ^*{\mathcolor{gray}{\bar{r}}}^{\mathsf{\textcolor{Plum}{T}}} \right)^{\mathsf{\textcolor{Plum}{T}}}$ 还包含 \textit{*args} 解包 $^*\left( \cdot \right)$、转置 $^{\mathsf{\textcolor{Plum}{T}}}$ 等操作(的配合),以将其定义为标准列向量。}或 $\mathcolor{gray}{\bar{\kappa}} := \left( \mathcolor{gray}{\omega}, \mathcolor{gray}{\bar{k}} \right)$ 域\Footnote{视算符两边所连接的场的\textcolor{gray}{自变量}而定,如 \bref{eq:P_wk2} 中的 $\mathcolor{gray}{\widetilde \circledast}$ 则运行在 $\mathcolor{gray}{\bar{\kappa}}$ 域。}的 4 维卷积算符;对应地,定义了 4 维时空域 $\mathcolor{gray}{\bar{x}} \in \mathcolor{gray}{\bar{\mathbb{R}}_{\textcolor{Maroon}{1+3}}}$ 中的\textcolor{Plum}{傅立叶正} $\mathcolor{gray}{\mathcal F_{\bar{\kappa}}}$、\textcolor{Plum}{逆} $\mathcolor{gray}{\mathcal F_{\bar{x}}^{-1}}$ \textcolor{Plum}{变换对}
\begin{subequations} \label{eq:FT-xkappa}
	\abovedisplayskip=10pt
	\belowdisplayskip=12pt
\begin{align}
	\mathcolor{gray}{\mathcal F_{\bar{\kappa}}} \left[ \cdot \right] &:= \mathcolor{gray}{\mathcal F_{\omega}^{-1}} \left[ \mathcolor{gray}{\mathcal F_{\bar{k}}} \left[ \cdot \right] \right] = \mathcolor{gray}{\mathcal F_{\bar{k}}} \left[ \mathcolor{gray}{\mathcal F_{\omega}^{-1}} \left[ \cdot \right] \right] ~, \label{eq:FT-kappa} \\
	\mathcolor{gray}{\mathcal F_{\bar{x}}^{-1}} \left[ \cdot \right] &:= \mathcolor{gray}{\mathcal F_{t}} \left[ \mathcolor{gray}{\mathcal F_{\bar{r}}^{-1}} \left[ \cdot \right] \right] = \mathcolor{gray}{\mathcal F_{\bar{r}}^{-1}} \left[ \mathcolor{gray}{\mathcal F_{t}} \left[ \cdot \right] \right] ~. \label{eq:IFT-x}
\end{align}
\end{subequations}
其中,定义了 1D(一维)时域 $\mathcolor{gray}{t} \in \mathcolor{gray}{\mathbb{R}}$ 中的\textcolor{Plum}{傅立叶正} $\mathcolor{gray}{\mathcal F_{t}}$、\textcolor{Plum}{逆} $\mathcolor{gray}{\mathcal F_{\omega}^{-1}}$ \textcolor{Plum}{变换对}\Footnote{时域\textcolor{Plum}{傅立叶变换}的核,需要与空域的共轭,这样正\textcolor{gray}{(角)频率}正\textcolor{PineGreen}{波矢}才对应前向行波\cite{mcleodVectorFourierOptics2014}。}
\begin{subequations} \label{eq:FT-tw}
	\abovedisplayskip=10pt
	\belowdisplayskip=12pt
\begin{align}
	\mathcolor{gray}{\mathcal F_{t}} \left[ \cdot \right] &:= \frac{ 1 }{ 2\symup{\pi} } \mathcolor{gray}{\int_{-\infty}^{+\infty}} \cdot~ \mathbb{e}^{-\mathbb{i}\mathcolor{gray}{\omega} \mathcolor{gray}{t}} \mathbb{d}\mathcolor{gray}{\omega} ~, \label{eq:FT-t} \\
	\mathcolor{gray}{\mathcal F_{\omega}^{-1}} \left[ \cdot \right] &:= \hphantom{\frac{ 1 }{ 2\symup{\pi} }} \mathcolor{gray}{\int_{-\infty}^{+\infty}} \cdot~ \mathbb{e}^{\mathbb{i}\mathcolor{gray}{\omega} \mathcolor{gray}{t}} \hphantom{^-} \mathbb{d}\mathcolor{gray}{t} ~. \label{eq:IFT-w}
\end{align}
\end{subequations}
以及 3 维空域 $\mathcolor{gray}{\bar{r}} \in \mathcolor{gray}{\bar{\mathbb{R}}_{\textcolor{Plum}{3}}}$ 中的\textcolor{Plum}{傅立叶正} $\mathcolor{gray}{\mathcal F_{\bar{k}}}$、\textcolor{Plum}{逆} $\mathcolor{gray}{\mathcal F_{\bar{r}}^{-1}}$ \textcolor{Plum}{变换对}\Footnote{其中,定义了:\textcolor{Plum}{数学}常量 $\mathbb{e},\mathbb{i}$、\textcolor{Plum}{数学}微分符号 $\mathbb{d}$,见 \bref{hook:mathbb}。}
\begin{subequations} \label{eq:FT-rk}
	\abovedisplayskip=10pt
	\belowdisplayskip=12pt
\begin{align}
	\mathcolor{gray}{\mathcal F_{\bar{k}}} \left[ \cdot \right] &:= \frac{ 1 }{ \left( 2\symup{\pi} \right)^3 } \mathcolor{gray}{\iiint_{-\infty}^{+\infty}} \cdot~ \mathbb{e}^{-\mathbb{i}\mathcolor{gray}{\bar{k}} \cdot \mathcolor{gray}{\bar{r}}} \mathbb{d}\mathcolor{gray}{\bar{r}} ~, \label{eq:FT-k} \\
	\mathcolor{gray}{\mathcal F_{\bar{r}}^{-1}} \left[ \cdot \right] &:= \hphantom{\frac{ 1 }{ \left( 2\symup{\pi} \right)^3 }} \mathcolor{gray}{\iiint_{-\infty}^{+\infty}} \cdot~ \mathbb{e}^{\mathbb{i}\mathcolor{gray}{\bar{k}} \cdot \mathcolor{gray}{\bar{r}}} \hphantom{^-} \mathbb{d}\mathcolor{gray}{\bar{k}} ~. \label{eq:IFT-r}
\end{align}
\end{subequations}
其中,定义了 $\mathcolor{gray}{\omega}$ 无关的 3 维\textcolor{gray}{倒空间}实向径 $=$ \textcolor{gray}{空间频率} $\mathcolor{gray}{\bar{k}} := \left( \mathcolor{gray}{k_{\symup{x}}},~ \mathcolor{gray}{k_{\symup{y}}},~ \mathcolor{gray}{k_{\symup{z}}} \right)^{\mathsf{\textcolor{Plum}{T}}} \asymp \left( \mathcolor{gray}{\bar{k}_{\symup{\rho}}},~ \mathcolor{gray}{k_{\symup{z}}} \right)$\Footnote{符号 $\asymp$ 表示:“含义相同,尽管形式(\textcolor{Plum}{数学}符号表达式)不同”,也即“指代同一对象”。},而非与\textcolor{gray}{波长}相关的\textcolor{PineGreen}{波矢} $\bar{k}^{\;\! \mathcolor{gray}{\omega}} := \left( \mathcolor{gray}{k_{\symup{x}}},~ \mathcolor{gray}{k_{\symup{y}}},~ k_{\symup{z}}^{\;\! \mathcolor{gray}{\omega}} \right)^{\mathsf{\textcolor{Plum}{T}}} \asymp \left( \mathcolor{gray}{\bar{k}_{\symup{\rho}}},~ k_{\symup{z}}^{\;\! \mathcolor{gray}{\omega}} \right)$。其中,定义了实的 2 维\textcolor{Plum}{横向}\textcolor{gray}{空间频率}(列向量) $\mathcolor{gray}{\bar{k}_{\symup{\rho}}} := \left( \mathcolor{gray}{k_{\symup{x}}},~ \mathcolor{gray}{k_{\symup{y}}} \right)^{\mathsf{\textcolor{Plum}{T}}}$。注意,$\mathcolor{gray}{\bar{k}}$ 与同样作为被积(和场量的)\textcolor{gray}{自变量}的 3 维\textcolor{gray}{正空间}实向径 $\mathcolor{gray}{\bar{r}}$ 等价,并作为\textcolor{Plum}{傅立叶变换}对互相共轭。

对 \bref{eq:P_tr} 执行 4 维\textcolor{Plum}{傅立叶(正)变换}\cite{xieAnalytic3DVector} $\mathcolor{gray}{\mathcal F_{\bar{\kappa}}}$ 即 \bref{eq:FT-kappa},得到 $\mathcolor{gray}{\bar{\kappa}}$ 域的
\begin{subequations}
\begin{align}
	P^{\;\!\mathcolor{gray}{\omega}}_{\;\! \symup{\iota}\mathcolor{gray}{\bar{k}}} &= \mathcolor{gray}{\mathcal F_{\bar{\kappa}}} \left[ P^{\;\! \mathcolor{gray}{t}}_{\;\! \symup{\iota}\mathcolor{gray}{\bar{r}}} \right] = P^{\;\! \textcolor{Maroon}{(1)} \mathcolor{gray}{\omega}}_{\;\! \symup{\iota}\mathcolor{gray}{\bar{k}}} + P^{\;\! \textcolor{Maroon}{(2)} \mathcolor{gray}{\omega}}_{\;\! \symup{\iota}\mathcolor{gray}{\bar{k}}} + P^{\;\! \textcolor{Maroon}{(3)} \mathcolor{gray}{\omega}}_{\;\! \symup{\iota}\mathcolor{gray}{\bar{k}}} + \cdots \\ &= {\symup{\varepsilon_0}} \left\{ \chi^{\;\! \mathcolor{gray}{\omega} \hat{1}}_{\;\! \symup{\iota} \mathcolor{gray}{\bar{k}} \textcolor{Maroon}{(1)}} \mathcolor{gray}{\mathcal F_{\bar{\kappa}}} \left[ E^{\;\!\mathcolor{gray}{t}}_{\;\! \hat{1} \mathcolor{gray}{\bar{r}}} \right] + \chi^{\;\! \mathcolor{gray}{\omega} \hat{1} \hat{2}}_{\;\! \symup{\iota} \mathcolor{gray}{\bar{k}} \textcolor{Maroon}{(2)}} \mathcolor{gray}{\mathcal F_{\bar{\kappa}}} \left[ E^{\;\!\mathcolor{gray}{t}}_{\;\! \hat{1} \mathcolor{gray}{\bar{r}}} E^{\;\!\mathcolor{gray}{t}}_{\;\! \hat{2} \mathcolor{gray}{\bar{r}}} \right] \right. \label{eq:P_wk} \\ &+ \left. \chi^{\;\! \mathcolor{gray}{\omega} \hat{1} \hat{2} \hat{3}}_{\;\! \symup{\iota} \mathcolor{gray}{\bar{k}} \textcolor{Maroon}{(3)}} \mathcolor{gray}{\mathcal F_{\bar{\kappa}}} \left[ E^{\;\!\mathcolor{gray}{t}}_{\;\! \hat{1} \mathcolor{gray}{\bar{r}}} E^{\;\!\mathcolor{gray}{t}}_{\;\! \hat{2} \mathcolor{gray}{\bar{r}}} E^{\;\!\mathcolor{gray}{t}}_{\;\! \hat{3} \mathcolor{gray}{\bar{r}}} \right] + \cdots \right\}
	\\ &= {\symup{\varepsilon_0}} \left\{ \chi^{\;\! \mathcolor{gray}{\omega} \hat{1}}_{\;\! \symup{\iota} \mathcolor{gray}{\bar{k}} \textcolor{Maroon}{(1)}} E^{\;\!\mathcolor{gray}{\omega}}_{\;\! \hat{1} \mathcolor{gray}{\bar{k}}} + \chi^{\;\! \mathcolor{gray}{\omega} \hat{1} \hat{2}}_{\;\! \symup{\iota} \mathcolor{gray}{\bar{k}} \textcolor{Maroon}{(2)}} \left( E^{\;\!\mathcolor{gray}{\omega}}_{\;\! \hat{1} \mathcolor{gray}{\bar{k}}} ~\mathcolor{gray}{\widetilde \circledast}~ E^{\;\!\mathcolor{gray}{\omega}}_{\;\! \hat{2} \mathcolor{gray}{\bar{k}}} \right) \right. \label{eq:P_wk2} \\ &+ \left. \chi^{\;\! \mathcolor{gray}{\omega} \hat{1} \hat{2} \hat{3}}_{\;\! \symup{\iota} \mathcolor{gray}{\bar{k}} \textcolor{Maroon}{(3)}} \left( E^{\;\!\mathcolor{gray}{\omega}}_{\;\! \hat{1} \mathcolor{gray}{\bar{k}}} ~\mathcolor{gray}{\widetilde \circledast}~ E^{\;\!\mathcolor{gray}{\omega}}_{\;\! \hat{2} \mathcolor{gray}{\bar{k}}} ~\mathcolor{gray}{\widetilde \circledast}~ E^{\;\!\mathcolor{gray}{\omega}}_{\;\! \hat{3} \mathcolor{gray}{\bar{k}}} \right) + \cdots \right\}~,
\end{align}
\end{subequations}
注意到 $\chi^{\;\! \mathcolor{gray}{\omega} \hat{1}}_{\;\! \symup{\iota} \mathcolor{gray}{\bar{k}} \textcolor{Maroon}{(1)}}$ 与 $\mathcolor{gray}{\bar{k}}$ 有关,即材料常数除了有\textcolor{gray}{时间}/\textcolor{gray}{角频率} $\mathcolor{gray}{\omega}$ \textcolor{NavyBlue}{色散}外,还有空间/\textcolor{PineGreen}{波矢} $\mathcolor{gray}{\bar{k}}$ \textcolor{NavyBlue}{色散},意味着介质的响应是\textcolor{Plum}{非局域}的:从\textcolor{gray}{正空间}上讲,即长程相互作用:$\mathcolor{gray}{\bar{r}'}$ 处的外场可以影响 $\mathcolor{gray}{\bar{r}}$ 处的材料常数,见 \bref{eq:P_tr};从\textcolor{gray}{倒空间}上讲,即外场的梯度也起作用:材料常数在 $\mathcolor{gray}{\bar{r}}$ 处的值,不仅是同一地点的外场的函数,还是其在该 $\mathcolor{gray}{\bar{r}}$ 处的空间各阶导数 $\mathcolor{gray}{\nabla^n_{\hat{1}}}$ 的函数,即 $\chi^{\;\! \mathcolor{gray}{t} \hat{1}}_{\;\! \symup{\iota} \mathcolor{gray}{\bar{r}} \textcolor{Maroon}{(1)}} \left( \mathcolor{gray}{\nabla^t}, \mathcolor{gray}{\nabla_{\hat{2}}} ; E^{\;\!\mathcolor{gray}{t}}_{\;\! \hat{3}\mathcolor{gray}{\bar{r}}}, B^{\;\!\mathcolor{gray}{t}}_{\;\! \hat{4}\mathcolor{gray}{\bar{r}}} \right)$。 ---  然而,在电偶极 $\bar{P}^{\;\!\mathcolor{gray}{t}}_{\;\!\mathcolor{gray}{z}}$ 近似下,一般认为材料常数与 $\mathcolor{gray}{\bar{k}}$ 无关,只是 $\mathcolor{gray}{\omega}$ 的函数,如大部分国内外教材中所定义的 $\chi^{\;\! \mathcolor{gray}{\omega} \hat{1}}_{\;\! \symup{\iota} \textcolor{Maroon}{(1)}} \asymp \chi^{ \textcolor{Maroon}{(1)} \mathcolor{gray}{\omega}}_{\;\! ij}$。这意味着 $\chi^{\;\! \mathcolor{gray}{\omega} \hat{1}}_{\;\! \symup{\iota} \mathcolor{gray}{\bar{k}} \textcolor{Maroon}{(1)}}$ 中的空间\textcolor{NavyBlue}{色散}项(即含 $\mathcolor{gray}{\bar{k}}$ 项),应由电偶极矩 $\bar{P}^{\;\!\mathcolor{gray}{t}}_{\;\!\mathcolor{gray}{z}}$ 之外的电\textcolor{Plum}{多极}矩 $\bar{\bar{Q}}^{\;\!\mathcolor{gray}{t}}_{\;\!\mathcolor{gray}{z}},\bar{\bar{\bar{O}}}^{\;\!\mathcolor{gray}{t}}_{\;\!\mathcolor{gray}{z}}$ 贡献\cite{shenNonlinearOpticalSusceptibilities2001},以至于不应将该部分包含在电偶极化强度 $P^{\;\!\mathcolor{gray}{\omega}}_{\;\! \symup{\iota}\mathcolor{gray}{\bar{k}}}$ 的 \bref{eq:P_wk} 之内(但后续的 \bref{eq:P(1)_wk} 会打破该认识)。此外,\bref{eq:P_wk} 无法给出 $\chi^{\;\! \mathcolor{gray}{\omega} \hat{1}}_{\;\! \symup{\iota} \mathcolor{gray}{\bar{k}} \textcolor{Maroon}{(1)}} \Longleftrightarrow \chi^{\;\! \mathcolor{gray}{\omega} \hat{1}}_{\;\! \symup{\iota} \textcolor{Maroon}{(1)}} \left( \mathcolor{gray}{\bar{k}} \right)$ 关于 $\mathcolor{gray}{\bar{k}}$ 的显式公式。

以上三点原因,迫使\textcolor{Plum}{线性}、\textcolor{Plum}{非线性}\textcolor{PineGreen}{晶体光学},转而继续向\textcolor{Maroon}{多极理论}寻求帮助。在半经典\textcolor{NavyBlue}{量子力学}的框架下,考虑场和电荷分布体系间的相互作用能远小于体系自能的情况,从一、二、三阶\textcolor{NavyBlue}{微扰}下的含时\textcolor{NavyBlue}{哈密顿量}的\textcolor{NavyBlue}{微观起源}的角度,\textcolor{Maroon}{多极理论}建议将 $\mathcolor{gray}{\bar{\kappa}}$ 域的材料常数 $\chi^{\;\! \mathcolor{gray}{\omega} \hat{1}}_{\;\! \symup{\iota} \mathcolor{gray}{\bar{k}} \textcolor{Maroon}{(1)}}, \chi^{\;\! \mathcolor{gray}{\omega} \hat{1} \hat{2}}_{\;\! \symup{\iota} \mathcolor{gray}{\bar{k}} \textcolor{Maroon}{(2)}}, \chi^{\;\! \mathcolor{gray}{\omega} \hat{1} \hat{2} \hat{3}}_{\;\! \symup{\iota} \mathcolor{gray}{\bar{k}} \textcolor{Maroon}{(3)}}$ 的空间\textcolor{NavyBlue}{色散}部分剥离,并全部转移到场 $E^{\;\!\mathcolor{gray}{\omega}}_{\;\! \hat{n} \mathcolor{gray}{\bar{k}}}$ 中。以 \bref{eq:P_wk} 的\textcolor{Plum}{线性}项 $P^{\;\! \textcolor{Maroon}{(1)} \mathcolor{gray}{\omega}}_{\;\! \symup{\iota}\mathcolor{gray}{\bar{k}}} = {\symup{\varepsilon_0}} \chi^{\;\! \mathcolor{gray}{\omega} \hat{1}}_{\;\! \symup{\iota} \mathcolor{gray}{\bar{k}} \textcolor{Maroon}{(1)}} E^{\;\!\mathcolor{gray}{\omega}}_{\;\! \hat{1} \mathcolor{gray}{\bar{k}}}$ 为例,材料常数 $\chi^{\;\! \mathcolor{gray}{\omega} \hat{1}}_{\;\! \symup{\iota} \mathcolor{gray}{\bar{k}} \textcolor{Maroon}{(1)}}$ 的空间/\textcolor{PineGreen}{波矢} $\mathcolor{gray}{\bar{k}}$ \textcolor{NavyBlue}{色散}部分,可以如下地转移到场 $E^{\;\!\mathcolor{gray}{\omega}}_{\;\! \hat{n} \mathcolor{gray}{\bar{k}}}$ 中
\begin{subequations}
\begin{align}
	P^{\;\! \textcolor{Maroon}{(1)} \mathcolor{gray}{\omega}}_{\;\! \symup{\iota}\mathcolor{gray}{\bar{k}}} =&~ {\symup{\varepsilon_0}} \chi^{\;\! \mathcolor{gray}{\omega} \hat{1}}_{\;\! \symup{\iota} \mathcolor{gray}{\bar{k}} \textcolor{Maroon}{(1)}} E^{\;\!\mathcolor{gray}{\omega}}_{\;\! \hat{1} \mathcolor{gray}{\bar{k}}} \\ =&~ {\symup{\varepsilon_0}} \left\{ \chi^{\;\! \mathcolor{gray}{\omega} \hat{1}}_{\;\! \symup{\iota} \textcolor{Maroon}{(1)}} E^{\;\!\mathcolor{gray}{\omega}}_{\;\! \hat{1} \mathcolor{gray}{\bar{k}}} + \chi^{\;\! \mathcolor{gray}{\omega} \hat{1} \mathcolor{gray}{\hat{2}}}_{\;\! \symup{\iota} \textcolor{Maroon}{(1)}} E^{\;\!\mathcolor{gray}{\omega}}_{\;\! \mathcolor{gray}{\hat{2}} \hat{1} \mathcolor{gray}{\bar{k}}} + \chi^{\;\! \mathcolor{gray}{\omega} \hat{1} \mathcolor{gray}{\hat{2} \hat{3}}}_{\;\! \symup{\iota} \textcolor{Maroon}{(1)}} E^{\;\!\mathcolor{gray}{\omega}}_{\;\! \mathcolor{gray}{\hat{3} \hat{2}} \hat{1} \mathcolor{gray}{\bar{k}}} + \cdots \right\}  \label{eq:P(1)_wk} \\ :=&~ {\symup{\varepsilon_0}} \left\{ \chi^{\;\! \mathcolor{gray}{\omega} \hat{1}}_{\;\! \symup{\iota} \textcolor{Maroon}{(1)}} E^{\;\!\mathcolor{gray}{\omega}}_{\;\! \hat{1} \mathcolor{gray}{\bar{k}}} + \chi^{\;\! \mathcolor{gray}{\omega} \hat{1} \mathcolor{gray}{\hat{2}}}_{\;\! \symup{\iota} \textcolor{Maroon}{(1)}} \left( \mathcolor{gray}{k_{\hat{2}}} E^{\;\!\mathcolor{gray}{\omega}}_{\;\hat{1} \mathcolor{gray}{\bar{k}}} \right) + \chi^{\;\! \mathcolor{gray}{\omega} \hat{1} \mathcolor{gray}{\hat{2} \hat{3}}}_{\;\! \symup{\iota} \textcolor{Maroon}{(1)}} \left( \mathcolor{gray}{k_{\hat{3}}} \mathcolor{gray}{k_{\hat{2}}} E^{\;\!\mathcolor{gray}{\omega}}_{\;\! \hat{1} \mathcolor{gray}{\bar{k}}} \right) + \cdots \right\}~,
\end{align}
\end{subequations}
从上述\textcolor{Maroon}{多极理论}给出的结果可见,即使是电偶极矩 $\bar{P}^{\;\!\mathcolor{gray}{\omega}}_{\;\!\mathcolor{gray}{\bar{k}}}$(的\textcolor{Plum}{线性}部分),也含空间/\textcolor{PineGreen}{波矢} $\mathcolor{gray}{\bar{k}}$ \textcolor{NavyBlue}{色散}(而不仅是电\textcolor{Plum}{多极}矩 $\bar{\bar{Q}}^{\;\!\mathcolor{gray}{\omega}}_{\;\!\mathcolor{gray}{\bar{k}}},\bar{\bar{\bar{O}}}^{\;\!\mathcolor{gray}{\omega}}_{\;\!\mathcolor{gray}{\bar{k}}}$ 的专属特性)。 ---  \textcolor{Maroon}{多极理论}对此的解释是,从\textcolor{NavyBlue}{微观起源}上来讲,$\chi^{\;\! \mathcolor{gray}{\omega} \hat{1}}_{\;\! \symup{\iota} \textcolor{Maroon}{(1)}}$ 项来自于\textcolor{NavyBlue}{电偶-电偶}极相互作用、$\chi^{\;\! \mathcolor{gray}{\omega} \hat{1} \mathcolor{gray}{\hat{2}}}_{\;\! \symup{\iota} \textcolor{Maroon}{(1)}}$ 项起源于\textcolor{NavyBlue}{电偶-电四/磁偶}极相互作用、$\chi^{\;\! \mathcolor{gray}{\omega} \hat{1} \mathcolor{gray}{\hat{2} \hat{3}}}_{\;\! \symup{\iota} \textcolor{Maroon}{(1)}}$ 项则对应\textcolor{NavyBlue}{电偶-电八/磁四}极相互作用。

也就是说,$\chi^{\;\! \mathcolor{gray}{\omega} \hat{1}}_{\;\! \symup{\iota} \textcolor{Maroon}{(1)}},\chi^{\;\! \mathcolor{gray}{\omega} \hat{1} \mathcolor{gray}{\hat{2}}}_{\;\! \symup{\iota} \textcolor{Maroon}{(1)}},\chi^{\;\! \mathcolor{gray}{\omega} \hat{1} \mathcolor{gray}{\hat{2} \hat{3}}}_{\;\! \symup{\iota} \textcolor{Maroon}{(1)}}$ 中的上指标$\hat{1},\hat{1} \hat{2},\hat{1} \hat{2} \hat{3}$分别对应(受电场$E^{\;\!\mathcolor{gray}{\omega}}_{\;\! \hat{n} \mathcolor{gray}{\bar{k}}}$驱动的)\textcolor{NavyBlue}{电二/四/八}极子\Footnote{准确地来说,应称小写的 $p,q,o;m,n$ 为\textcolor{Plum}{多极}矩(算符),而称大写的 $P,Q,O;M,N$ 为\textcolor{Plum}{多极}矩(体)密度 or 电磁\textcolor{Plum}{多极}化强度;但为了简称,本文暂称前/后者为\textcolor{Plum}{多极}子/矩。} $p_{\;\! \hat{1}}, q_{\;\! \hat{1} \hat{2}}, o_{\;\! \hat{1} \hat{2} \hat{3}}$ or \textcolor{NavyBlue}{磁零/二/四}极子 $0, m_{\;\! \hat{2}}, n_{\;\! \hat{2} \hat{3}}$ 的响应,而这些材料常数的下脚标$\symup{\iota}$皆对应(级联受 $p_{\;\! \hat{1}}, q_{\;\! \hat{1} \hat{2}}, o_{\;\! \hat{1} \hat{2} \hat{3}}; m_{\;\! \hat{2}}, n_{\;\! \hat{2} \hat{3}}$ 驱动的)电偶极子 $p_{\;\! \symup{\iota}}$ 的响应。 --- 这意味着当极化率 $\chi^{\;\! \mathcolor{gray}{\omega} \hat{1}}_{\;\! \symup{\iota} \textcolor{Maroon}{(1)}}$、$\chi^{\;\! \mathcolor{gray}{\omega} \hat{1} \mathcolor{gray}{\hat{2}}}_{\;\! \symup{\iota} \textcolor{Maroon}{(1)}}$、$\chi^{\;\! \mathcolor{gray}{\omega} \hat{1} \mathcolor{gray}{\hat{2} \hat{3}}}_{\;\! \symup{\iota} \textcolor{Maroon}{(1)}}$ 分别由 $p_{\;\! \symup{\iota}}, p^{\;\! \hat{1}}$、$p_{\;\! \symup{\iota}}, q^{\;\! \hat{1} \hat{2}}$、$p_{\;\! \symup{\iota}}, o^{\;\! \hat{1} \hat{2} \hat{3}}$ 构成时,电矩算子 $p_{\;\! \hat{1}}$、$ q_{\;\! \hat{1} \hat{2}}$、$ o_{\;\! \hat{1} \hat{2} \hat{3}}$ 的下标 $\hat{1}$、$\hat{1} \hat{2}$、$\hat{1} \hat{2} \hat{3}$ 的\textcolor{Plum}{置换对称性},将传递到包含这些算子的极化率的矩阵元素中去。即 $\chi^{\;\! \mathcolor{gray}{\omega} \hat{1}}_{\;\! \symup{\iota} \textcolor{Maroon}{(1)}}$、$\chi^{\;\! \mathcolor{gray}{\omega} \hat{1} \mathcolor{gray}{\hat{2}}}_{\;\! \symup{\iota} \textcolor{Maroon}{(1)}}$、$\chi^{\;\! \mathcolor{gray}{\omega} \hat{1} \mathcolor{gray}{\hat{2} \hat{3}}}_{\;\! \symup{\iota} \textcolor{Maroon}{(1)}}$ 中的指标 $\hat{1}$、$\hat{1} \hat{2}$、$\hat{1} \hat{2} \hat{3}$ 内部可以两两交换顺序,而不改变极化率本体/主体的值,即使场的角标没有跟随排列(当极化率的张量元 $\hat{1}$、$\hat{1} \hat{2}$、$\hat{1} \hat{2} \hat{3}$ 重新排列时)。如 $\chi^{\;\! \mathcolor{gray}{\omega} \hat{1} \mathcolor{gray}{\hat{2}}}_{\;\! \symup{\iota} \textcolor{Maroon}{(1)}} = \chi^{\;\! \mathcolor{gray}{\omega} \mathcolor{gray}{\hat{2}} \hat{1}}_{\;\! \symup{\iota} \textcolor{Maroon}{(1)}}$。

上述这条规律
\begin{align}
	\textbf{\text{材料系数继承了\textcolor{Plum}{多极}子算符的内禀角标\textcolor{Plum}{置换对称性}}}~, \label{eq:symmetry1}
\end{align}
对于包含磁\textcolor{Plum}{多极}子的磁化率、磁电/电磁系数也均成立,但注意单个磁\textcolor{Plum}{多极}子(如$m_{ij}$)内部没有角标的全\textcolor{Plum}{置换对称性}\Footnote{但可能仍会有一定的/部分的\textcolor{Plum}{置换对称性}:比如磁八极子$m_{ijk} = m_{ikj}$关于其后两个角标$j,k$对称。}。此外,该规律不仅对由 2 个\textcolor{Plum}{多极}子构成的\textcolor{Plum}{线性}极化率成立,对由 3 个及以上\textcolor{Plum}{多极}子构成的\textcolor{Plum}{非线性}/超极化率也成立\cite{raabMultipoleTheoryElectromagnetism2004}。

此外,极化率 $\chi^{\;\! \mathcolor{gray}{\omega} \hat{1}}_{\;\! \symup{\iota} \textcolor{Maroon}{(1)}},\chi^{\;\! \mathcolor{gray}{\omega} \hat{1} \mathcolor{gray}{\hat{2}}}_{\;\! \symup{\iota} \textcolor{Maroon}{(1)}},\chi^{\;\! \mathcolor{gray}{\omega} \hat{1} \mathcolor{gray}{\hat{2} \hat{3}}}_{\;\! \symup{\iota} \textcolor{Maroon}{(1)}}$ 中还包含\textcolor{Plum}{多极}子(算符)之内的对称性之外的、\textcolor{Plum}{多极}子(算符们,作为整体)之间(而不是其角标)的\textcolor{Plum}{置换对称性},即允许由\textcolor{Plum}{多极}子 $p_{\;\! \symup{\iota}} \otimes p_{\;\! \hat{1}}, q_{\;\! \hat{1} \hat{2}}, o_{\;\! \hat{1} \hat{2} \hat{3}}$ 通过\textcolor{Plum}{克罗内克积/并矢积/张量积}构成的 $\chi^{\;\! \mathcolor{gray}{\omega} \hat{1}}_{\;\! \symup{\iota} \textcolor{Maroon}{(1)}},\chi^{\;\! \mathcolor{gray}{\omega} \hat{1} \mathcolor{gray}{\hat{2}}}_{\;\! \symup{\iota} \textcolor{Maroon}{(1)}},\chi^{\;\! \mathcolor{gray}{\omega} \hat{1} \mathcolor{gray}{\hat{2} \hat{3}}}_{\;\! \symup{\iota} \textcolor{Maroon}{(1)}}$ 以\textcolor{Plum}{多极}子为最小单位任意排列角标后,材料常数本体保持不变或变为相反数 or 变为与其相等的另一个材料常数或其相反数。 ---  以\textcolor{NavyBlue}{空间(偶)对称} $=$ \textcolor{NavyBlue}{空偶}的 $\chi^{\;\! \mathcolor{gray}{\omega} \hat{1}}_{\;\! \symup{\iota} \textcolor{Maroon}{(1)}} \sim p_{\;\! \symup{\iota}} q^{\;\! \hat{1}}$ 为例,对于其\textcolor{NavyBlue}{时间(偶)对称} $=$ \textcolor{NavyBlue}{时偶}\Footnote{类似但不等价于\textcolor{Plum}{(矩阵)对称} $\frac{1}{2} \left[ (\cdot) + (\cdot)^\intercal \right]$:对于高阶张量,\textcolor{NavyBlue}{PT 对称}指定两组张量元,而不是\textcolor{Plum}{对称}所须指定的两个。具体来说,当指定$i$、$jk$两组指标时,$U^{\hphantom{i}jk}_{i}$的 \textcolor{NavyBlue}{PT 对称}部分$U^{\hphantom{i}jk}_{i\textcolor{Maroon}{(\textcolor{NavyBlue}{\text{S}})}} = \frac{1}{2!} ( U^{\hphantom{i}jk}_{i} + \dot{U}^{jk}_{\hphantom{jk}i} )$;当指定$j$、$k$两个指标时,$U^{\hphantom{i}jk}_{i}$的\textcolor{Plum}{对称}部分$U^{\hphantom{i}(jk)}_{i\textcolor{Plum}{(\textcolor{NavyBlue}{\text{S}})}} = \frac{1}{2!} ( U^{\hphantom{i}jk}_{i} + U^{\hphantom{i}kj}_{i} )$。\textcolor{NavyBlue}{PT 对称}和\textcolor{Plum}{对称}也可以拓展至多组、多个指标。前者对应多个相互作用的\textcolor{Plum}{多极}子的\textcolor{Plum}{克罗内克积/并矢积/张量积}的最终结果为 \textcolor{NavyBlue}{PT 对称};后者则是指,对初始角标取遍所有排列后,对应的这些张量元的平均值,若总等于初始角标排列下的张量元,则称该张量对这些角标是\textcolor{Plum}{(全)对称}的\cite{chen-zhuChenZhuxieUndergraduate_courses2024}。从这个角度,电磁\textcolor{Plum}{多极}子/矩属于后者:它们对于其所有角标\textcolor{Plum}{全对称}, ---  而极/磁化率张量却是包含了后者的前者:它们既对于部分角标(这部分继承了 \bref{eq:symmetry1})是\textcolor{Plum}{全对称}的;然而对于所有角标而言,却只能称 \textcolor{NavyBlue}{PT 对称},且只能以\textcolor{Plum}{多极}子们为(最小)单位置换角标。}部分 $\chi^{\;\! \mathcolor{gray}{\omega} \hat{1}}_{\;\! \symup{\iota} \textcolor{Maroon}{(1\textcolor{NavyBlue}{\text{S}})}}$,脚标 $\symup{\iota}$ 可以与指标 $\hat{1}$ 整体左右交换位置\Footnote{角标的左右顺序是关键,具有\textcolor{NavyBlue}{物理意义};上下位置只用于\textcolor{Plum}{数学}上提示\textcolor{Plum}{爱因斯坦求和},没有\textcolor{NavyBlue}{物理意义}。}而不改变极化率本体/主体的值,即 $\chi^{\;\! \mathcolor{gray}{\omega} \hat{1}}_{\;\! \symup{\iota} \textcolor{Maroon}{(1\textcolor{NavyBlue}{\text{S}})}} = \chi^{\;\! \hat{1} \mathcolor{gray}{\omega} }_{\;\! \textcolor{Maroon}{(1\textcolor{NavyBlue}{\text{S}})} \symup{\iota}}$;对于其\textcolor{NavyBlue}{时间奇/反(对)称} $=$ \textcolor{NavyBlue}{时奇}\Footnote{类似于但不是\textcolor{Plum}{(矩阵)反称} $\frac{1}{2} \left[ (\cdot) - (\cdot)^\intercal \right]$:因为高阶张量的\textcolor{Plum}{反称}须指定两个,而非 \textcolor{NavyBlue}{PT 反称}的两组张量元。\textcolor{NavyBlue}{PT 反称}和\textcolor{Plum}{反称}也可以指定多组、多个角标,此时每项会涉及奇偶排列反号的问题。此外,与\textcolor{Plum}{(反)对称}类似,对于高阶张量的\textcolor{Plum}{(非)厄米}部分的定义,也有类似的限制和拓展。}部分 $\chi^{\;\! \mathcolor{gray}{\omega} \hat{1}}_{\;\! \symup{\iota} \textcolor{Maroon}{(1\textcolor{NavyBlue}{\text{A}})}}$,当其脚标 $\symup{\iota}$ 与指标 $\hat{1}$ 整体左右换位后,值 $\chi^{\;\! \mathcolor{gray}{\omega} \hat{1}}_{\;\! \symup{\iota} \textcolor{Maroon}{(1\textcolor{NavyBlue}{\text{A}})}} = - \chi^{\;\! \hat{1} \mathcolor{gray}{\omega} }_{\;\! \textcolor{Maroon}{(1\textcolor{NavyBlue}{\text{A}})} \symup{\iota}}$ 取反\cite{raabMultipoleTheoryElectromagnetism2004}。

以高一阶的、同样\textcolor{NavyBlue}{空偶}的 $\chi^{\;\! \mathcolor{gray}{\omega} \hat{1} \mathcolor{gray}{\hat{2}}}_{\;\! \symup{\iota} \textcolor{Maroon}{(1)}} \sim p_{\;\! \symup{\iota}} q^{\;\! \hat{1} \mathcolor{gray}{\hat{2}}}$ 为例,对于其\textcolor{NavyBlue}{时偶}部分$\chi^{\;\! \mathcolor{gray}{\omega} \hat{1} \mathcolor{gray}{\hat{2}}}_{\;\! \symup{\iota} \textcolor{Maroon}{(1\textcolor{NavyBlue}{\text{S}})}}$,脚标 $\symup{\iota}$ 可以与指标 $\hat{1} \hat{2}$ 整体左右交换位置,变成另一个极化率本体/主体,即 $\chi^{\;\! \mathcolor{gray}{\omega} \hat{1} \mathcolor{gray}{\hat{2}}}_{\;\! \symup{\iota} \textcolor{Maroon}{(1\textcolor{NavyBlue}{\text{S}})}} = \dot{\chi}^{\;\! \hat{1} \mathcolor{gray}{\hat{2}} \mathcolor{gray}{\omega} }_{\;\! \textcolor{Maroon}{(1\textcolor{NavyBlue}{\text{S}})} \symup{\iota}}$;对于其\textcolor{NavyBlue}{时奇}部分 $\chi^{\;\! \mathcolor{gray}{\omega} \hat{1} \mathcolor{gray}{\hat{2}}}_{\;\! \symup{\iota} \textcolor{Maroon}{(1\textcolor{NavyBlue}{\text{A}})}}$,当其脚标 $\symup{\iota}$ 与指标 $\hat{1} \hat{2}$ 整体左右换位后,值 $\chi^{\;\! \mathcolor{gray}{\omega} \hat{1} \mathcolor{gray}{\hat{2}}}_{\;\! \symup{\iota} \textcolor{Maroon}{(1\textcolor{NavyBlue}{\text{A}})}} = - \dot{\chi}^{\;\! \hat{1} \mathcolor{gray}{\hat{2}} \mathcolor{gray}{\omega} }_{\;\! \textcolor{Maroon}{(1\textcolor{NavyBlue}{\text{A}})} \symup{\iota}}$ 反号于与脚标与之全同的另一个极化率本体/主体\cite{raabMultipoleTheoryElectromagnetism2004}。其中,同样\textcolor{NavyBlue}{空偶}的 $\dot{\chi}^{\;\! \hat{1} \mathcolor{gray}{\hat{2}} \mathcolor{gray}{\omega} }_{\;\! \textcolor{Maroon}{(1\textcolor{NavyBlue}{\text{A}})} \symup{\iota}} \sim q^{\;\! \hat{1} \mathcolor{gray}{\hat{2}}} p_{\;\! \symup{\iota}}$ 是稍后出现的 \bref{eq:Q(1)_wk} 中,$Q^{\;\! \textcolor{Maroon}{(1)} \mathcolor{gray}{\omega}}_{\;\! \symup{\iota} \hat{1} \mathcolor{gray}{\bar{k}}}$ 的\textcolor{Plum}{线性}响应部分的第一项 ${\symup{\varepsilon_0}} \dot{\chi}^{\;\! \mathcolor{gray}{\omega} \hat{2}}_{\;\! \symup{\iota} \hat{1} \textcolor{Maroon}{(1)}} E^{\;\!\mathcolor{gray}{\omega}}_{\;\! \hat{2} \mathcolor{gray}{\bar{k}}}$ 的极化率 $\dot{\chi}^{\;\! \mathcolor{gray}{\omega} \hat{2}}_{\;\! \symup{\iota} \hat{1} \textcolor{Maroon}{(1)}}$ \textcolor{NavyBlue}{时奇}部分 $\dot{\chi}^{\;\! \mathcolor{gray}{\omega} \hat{1} \hat{2}}_{\;\! \symup{\iota} \textcolor{Maroon}{(1\textcolor{NavyBlue}{\text{A}})}}$ 的角标以\textcolor{Plum}{多极}子 $p_{\;\! \symup{\iota}}, q^{\;\! \hat{1} \mathcolor{gray}{\hat{2}}}$ 为单位的置换版。

上述 2 个例子$\chi^{\;\! \mathcolor{gray}{\omega} \hat{1}}_{\;\! \symup{\iota} \textcolor{Maroon}{(1)}}, \chi^{\;\! \mathcolor{gray}{\omega} \hat{1} \mathcolor{gray}{\hat{2}}}_{\;\! \symup{\iota} \textcolor{Maroon}{(1)}}$的“以\textcolor{Plum}{多极}子为单位的角标\textcolor{Plum}{置换对称性}”的不同之处在于:组成$\chi^{\;\! \mathcolor{gray}{\omega} \hat{1}}_{\;\! \symup{\iota} \textcolor{Maroon}{(1)}}$的$p_{\;\! \symup{\iota}}, q^{\;\! \hat{1}}$是同种类的\textcolor{Plum}{多极}子,而组成$\chi^{\;\! \mathcolor{gray}{\omega} \hat{1} \mathcolor{gray}{\hat{2}}}_{\;\! \symup{\iota} \textcolor{Maroon}{(1)}}$的$p_{\;\! \symup{\iota}}, q^{\;\! \hat{1} \hat{2}}$是不同种类的\textcolor{Plum}{多极}子。前者对应角标置换则变成自身或自身的相反数,后者则对应角标置换则变成另一个/种(对应置换角标后的相应\textcolor{Plum}{多极}子顺序的)极化率的相反数。该规律普遍适用,即“(若)材料常数由相同种类\textcolor{Plum}{多极}子构成$\longleftrightarrow$(则)交换角标后为自身、(若)材料常数由不同种类\textcolor{Plum}{多极}子构成$\longleftrightarrow$(则)另一个材料常数”。

上述 2 个例子的相同之处在于,$\chi^{\;\! \mathcolor{gray}{\omega} \hat{1}}_{\;\! \symup{\iota} \textcolor{Maroon}{(1)}}, \chi^{\;\! \mathcolor{gray}{\omega} \hat{1} \mathcolor{gray}{\hat{2}}}_{\;\! \symup{\iota} \textcolor{Maroon}{(1)}}$都是\textcolor{NavyBlue}{空偶}的,以致于它们的\textcolor{NavyBlue}{时偶}部分在置换角标后,不用取负;而其\textcolor{NavyBlue}{时奇}部分在置换角标后,则须取负。反过来,如果材料常数是\textcolor{NavyBlue}{空间奇/反(对)称} $=$ \textcolor{NavyBlue}{空奇}的,则其\textcolor{NavyBlue}{时奇}部分在置换角标后不取负,\textcolor{NavyBlue}{时偶}部分取负。这也可以通过 \textcolor{NavyBlue}{PT 对称性}来理解:即 \textcolor{NavyBlue}{PT 对称}(对应\textcolor{NavyBlue}{时偶空偶}或\textcolor{NavyBlue}{时奇空奇})$\longleftrightarrow$ 正,即\textcolor{Plum}{置换对称};而 \textcolor{NavyBlue}{PT 反(对)称}(对应\textcolor{NavyBlue}{时偶空奇}或\textcolor{NavyBlue}{时奇空偶})$\longleftrightarrow$ 负,即\textcolor{Plum}{置换反称}。 ---  即,上述材料常数矩阵的 \textcolor{NavyBlue}{PT 对称}特性,与其以\textcolor{Plum}{多极}子为单位的角标置换\textcolor{Plum}{矩阵对称}特性是相同(方向)的。

上述 2 个例子所遵循的规律,即
\begin{align}
	\textbf{\text{材料系数以\textcolor{Plum}{多极}子为(最小)单位有角标\textcolor{Plum}{置换对称性}}}~, \label{eq:symmetry2}
\end{align}
对磁\textcolor{Plum}{多极}子参与构成的磁化率、磁电/电磁系数也均成立。但只针对由 2 个\textcolor{Plum}{多极}子构成的\textcolor{Plum}{线性}极化率成立\cite{raabMultipoleTheoryElectromagnetism2004},即不一定能简单延拓至由 3 个及以上\textcolor{Plum}{多极}子构成的\textcolor{Plum}{非线性}/超极化率,这需要进一步检查\textcolor{NavyBlue}{量子力学}复跃迁/密度矩阵\cite{boydNonlinearOptics2019,barronTimeReversalMolecular2001a,buckinghamQuadrupoleMomentsDipolar1968}。

总之,\bref{eq:P(1)_wk} 可看作二阶过程\Footnote{一阶\textcolor{Plum}{线性}\textcolor{NavyBlue}{电偶-电磁多}极光学过程,与二阶\textcolor{Plum}{非线性}\textcolor{NavyBlue}{电偶-电偶}极光学过程类似,暂可理解为二阶过程。}:电场$\bar{E}^{\;\!\mathcolor{gray}{\omega}}_{\;\!\mathcolor{gray}{\bar{k}}}$(的各阶时空导数)$\xrightarrow[]{\text{驱动}}$ \textcolor{NavyBlue}{电磁多}极矩 $\bar{P}^{\;\!\mathcolor{gray}{\omega}}_{\;\!\mathcolor{gray}{\bar{k}}},\bar{\bar{Q}}^{\;\!\mathcolor{gray}{\omega}}_{\;\!\mathcolor{gray}{\bar{k}}},\bar{\bar{\bar{O}}}^{\;\!\mathcolor{gray}{\omega}}_{\;\!\mathcolor{gray}{\bar{k}}}; \bar{M}^{\;\!\mathcolor{gray}{\omega}}_{\;\!\mathcolor{gray}{\bar{k}}},\bar{\bar{N}}^{\;\!\mathcolor{gray}{\omega}}_{\;\!\mathcolor{gray}{\bar{k}}}$(的各阶散度,见\bref{eq:div-Q(1)_wk})$\xrightarrow[]{\text{耦合至}}$ \textcolor{NavyBlue}{电偶}极矩 $\bar{P}^{\;\!\mathcolor{gray}{\omega}}_{\;\!\mathcolor{gray}{\bar{k}}}$。

需注意的是,\bref{eq:P(1)_wk} 包含了以下可能的外场:交/直流电场 $E^{\;\!\mathcolor{gray}{t}}_{\;\! \hat{n} \mathcolor{gray}{\bar{r}}}$、交流电场的各 $\geq 1$ 阶时间导数 $\mathcolor{gray}{\nabla^t} E^{\;\!\mathcolor{gray}{t}}_{\;\! \hat{n} \mathcolor{gray}{\bar{r}}}$、交/直流电场的各 $\geq 1$ 阶空间导数 $\mathcolor{gray}{\nabla_{\hat{m}}} E^{\;\!\mathcolor{gray}{t}}_{\;\! \hat{n} \mathcolor{gray}{\bar{r}}}$;交流磁感应场 $B^{\;\!\mathcolor{gray}{t}}_{\;\! \hat{n} \mathcolor{gray}{\bar{r}}}$、交流磁感应场的各 $\geq 1$ 阶时间导数 $\mathcolor{gray}{\nabla^t} B^{\;\!\mathcolor{gray}{t}}_{\;\! \hat{n} \mathcolor{gray}{\bar{r}}}$、交流磁感应场的各 $\geq 1$ 阶空间导数 $\mathcolor{gray}{\nabla_{\hat{m}}} B^{\;\!\mathcolor{gray}{t}}_{\;\! \hat{n} \mathcolor{gray}{\bar{r}}}$, ---  但唯独不包含直流静磁感应场 $B^{\;\!\mathcolor{gray}{\text{dc}}}_{\;\! \hat{n} \mathcolor{gray}{\bar{r}}}$,及其各 $\geq 1$ 阶空间导数 $\mathcolor{gray}{\nabla_{\hat{m}}} B^{\;\!\mathcolor{gray}{\text{dc}}}_{\;\! \hat{n} \mathcolor{gray}{\bar{r}}}$。这是因为:\textcolor{Plum}{叉积算子} $\epsilon^{\hphantom{\symup{\iota}\hat{m}}\hat{n}}_{\hat{l}\hat{m}}$ 可以包含在 $\chi^{\;\! \mathcolor{gray}{\omega} \hat{1} \mathcolor{gray}{\hat{2}}}_{\;\! \symup{\iota} \textcolor{Maroon}{(1)}}$ 及以上阶次的材料常数中,使得通过\textcolor{Maroon}{法拉第电磁感应定律} $\epsilon^{\hphantom{\symup{\iota}\hat{m}}\hat{n}}_{\hat{l}\mathcolor{gray}{\hat{m}}} \mathcolor{gray}{\nabla^{\hat{m}}} E^{\;\!\mathcolor{gray}{t}}_{\;\! \hat{n} \mathcolor{gray}{\bar{r}}} = - \mathcolor{gray}{\nabla^t} B^{\;\!\mathcolor{gray}{t}}_{\;\! \hat{l} \mathcolor{gray}{\bar{r}}}$,材料常数中的 $\epsilon^{\hphantom{\symup{\iota}\hat{m}}\hat{n}}_{\hat{l}\hat{m}}$ 与梯度算子 $\mathcolor{gray}{\nabla^{\hat{m}}}$、电场 $E^{\;\!\mathcolor{gray}{t}}_{\;\! \hat{n} \mathcolor{gray}{\bar{r}}}$ 三者的乘积能且只能转换为磁感应场的时间导数 $\mathcolor{gray}{\nabla^t} B^{\;\!\mathcolor{gray}{t}}_{\;\! \hat{l} \mathcolor{gray}{\bar{r}}}$。

此即为什么 $\chi^{\;\! \mathcolor{gray}{\omega} \hat{1} \mathcolor{gray}{\hat{2}}}_{\;\! \symup{\iota} \textcolor{Maroon}{(1)}}$ 也能描述\textcolor{NavyBlue}{电偶-磁偶}极相互作用:考察 $P^{\;\! \textcolor{Maroon}{(1)} \mathcolor{gray}{\omega}}_{\;\! \symup{\iota}\mathcolor{gray}{\bar{k}}\textcolor{NavyBlue}{\text{m}}} = \chi^{\;\! \mathcolor{gray}{\omega} \hat{1} \mathcolor{gray}{\hat{2}}}_{\;\! \symup{\iota} \textcolor{Maroon}{(1)}\textcolor{NavyBlue}{\text{m}}} E^{\;\!\mathcolor{gray}{\omega}}_{\;\! \mathcolor{gray}{\hat{2}} \hat{1} \mathcolor{gray}{\bar{k}}} := \chi^{\;\! \mathcolor{gray}{\omega}}_{\;\! \symup{\iota} \textcolor{Maroon}{(1)}\textcolor{NavyBlue}{\text{m}}} \epsilon^{\hphantom{\symup{\iota}} \mathcolor{gray}{\hat{2}} \hat{1}}_{\symup{\iota}} E^{\;\!\mathcolor{gray}{\omega}}_{\;\! \mathcolor{gray}{\hat{2}} \hat{1} \mathcolor{gray}{\bar{k}}} = \chi^{\;\! \mathcolor{gray}{\omega}}_{\;\! \symup{\iota} \textcolor{Maroon}{(1)}\textcolor{NavyBlue}{\text{m}}} \mathbb{i} \mathcolor{gray}{\omega} B^{\;\!\mathcolor{gray}{\omega}}_{\;\! \symup{\iota} \mathcolor{gray}{\bar{k}}}$,但这只包含\textcolor{NavyBlue}{电偶-磁偶}张量的主轴\textcolor{Plum}{各向异性};为了在\textcolor{NavyBlue}{电偶-电四}极(的相同)水平上,全面描述\textcolor{NavyBlue}{电偶-磁偶}极的\textcolor{Plum}{各向异性}耦合,需(在相同阶 $\chi^{\;\! \mathcolor{gray}{\omega} \hat{1} \mathcolor{gray}{\hat{2}}}_{\;\! \symup{\iota} \textcolor{Maroon}{(1)}\textcolor{NavyBlue}{\text{m}}}$ 的水平上,多用一个重复角标 $\hat{3}$ 并求和以)额外构造出 $P^{\;\! \textcolor{Maroon}{(1)} \mathcolor{gray}{\omega}}_{\;\! \symup{\iota}\mathcolor{gray}{\bar{k}}\textcolor{NavyBlue}{\text{m}}} = \chi^{\;\! \mathcolor{gray}{\omega} \hat{1} \mathcolor{gray}{\hat{2}}}_{\;\! \symup{\iota} \textcolor{Maroon}{(1)}\textcolor{NavyBlue}{\text{m}}} E^{\;\!\mathcolor{gray}{\omega}}_{\;\! \mathcolor{gray}{\hat{2}} \hat{1} \mathcolor{gray}{\bar{k}}} := \chi^{\;\! \mathcolor{gray}{\omega} \hat{3}}_{\;\! \symup{\iota} \textcolor{Maroon}{(1)}\textcolor{NavyBlue}{\text{m}}} \epsilon^{\hphantom{\hat{3}} \mathcolor{gray}{\hat{2}} \hat{1}}_{\hat{3}} E^{\;\!\mathcolor{gray}{\omega}}_{\;\! \mathcolor{gray}{\hat{2}} \hat{1} \mathcolor{gray}{\bar{k}}} = \chi^{\;\! \mathcolor{gray}{\omega} \hat{3}}_{\;\! \symup{\iota} \textcolor{Maroon}{(1)}\textcolor{NavyBlue}{\text{m}}} \mathbb{i} \mathcolor{gray}{\omega} B^{\;\!\mathcolor{gray}{\omega}}_{\;\! \hat{3} \mathcolor{gray}{\bar{k}}}$。其中,$\chi^{\;\! \mathcolor{gray}{\omega} \hat{1} \mathcolor{gray}{\hat{2}}}_{\;\! \symup{\iota} \textcolor{Maroon}{(1)}\textcolor{NavyBlue}{\text{m}}} = \chi^{\;\! \mathcolor{gray}{\omega} \hat{3}}_{\;\! \symup{\iota} \textcolor{Maroon}{(1)}\textcolor{NavyBlue}{\text{m}}} \epsilon^{\hphantom{\hat{3}} \mathcolor{gray}{\hat{2}} \hat{1}}_{\hat{3}}$ 所分解出的\textcolor{NavyBlue}{电偶-磁偶}极 $\chi^{\;\! \mathcolor{gray}{\omega} \hat{3}}_{\;\! \symup{\iota} \textcolor{Maroon}{(1)}\textcolor{NavyBlue}{\text{m}}}$ 类似\textcolor{NavyBlue}{电偶-电偶}极 $\chi^{\;\! \mathcolor{gray}{\omega} \hat{1}}_{\;\! \symup{\iota} \textcolor{Maroon}{(1)}\textcolor{NavyBlue}{\text{p}}}$。 --- 同理,为了在\textcolor{NavyBlue}{电偶-电八}极(的相同)水平上,描述\textcolor{NavyBlue}{电偶-磁四}极\textcolor{Plum}{各向异性}相互作用,$\chi^{\;\! \mathcolor{gray}{\omega} \hat{1} \mathcolor{gray}{\hat{2} \hat{3}}}_{\;\! \symup{\iota} \textcolor{Maroon}{(1)}\textcolor{NavyBlue}{\text{n}}} = \chi^{\;\! \mathcolor{gray}{\omega} \hat{4} \mathcolor{gray}{\hat{3}}}_{\;\! \symup{\iota} \textcolor{Maroon}{(1)}\textcolor{NavyBlue}{\text{n}}} \epsilon^{\hphantom{\hat{4}} \mathcolor{gray}{\hat{2}} \hat{1}}_{\hat{4}}$也需要反向利用指标缩并\cite{chen-zhuChenZhuxieUndergraduate_courses2024},以分解出类似\textcolor{NavyBlue}{电偶-电四}极$\chi^{\;\! \mathcolor{gray}{\omega} \hat{1} \mathcolor{gray}{\hat{2}}}_{\;\! \symup{\iota} \textcolor{Maroon}{(1)}\textcolor{NavyBlue}{\text{q}}}$ 的 $\chi^{\;\! \mathcolor{gray}{\omega} \hat{4} \mathcolor{gray}{\hat{3}}}_{\;\! \symup{\iota} \textcolor{Maroon}{(1)}\textcolor{NavyBlue}{\text{n}}}$ 以包含(\textcolor{Plum}{线性})磁致电响应 $P^{\;\! \textcolor{Maroon}{(1)} \mathcolor{gray}{\omega}}_{\;\! \symup{\iota}\mathcolor{gray}{\bar{k}}\textcolor{NavyBlue}{\text{n}}} = \chi^{\;\! \mathcolor{gray}{\omega} \hat{1} \mathcolor{gray}{\hat{2} \hat{3}}}_{\;\! \symup{\iota} \textcolor{Maroon}{(1)}\textcolor{NavyBlue}{\text{n}}} E^{\;\!\mathcolor{gray}{\omega}}_{\;\! \mathcolor{gray}{\hat{3} \hat{2}} \hat{1} \mathcolor{gray}{\bar{k}}} := \chi^{\;\! \mathcolor{gray}{\omega} \hat{4} \mathcolor{gray}{\hat{3}}}_{\;\! \symup{\iota} \textcolor{Maroon}{(1)}\textcolor{NavyBlue}{\text{n}}} \epsilon^{\hphantom{\hat{4}} \mathcolor{gray}{\hat{2}} \hat{1}}_{\hat{4}} E^{\;\!\mathcolor{gray}{\omega}}_{\;\! \mathcolor{gray}{\hat{3} \hat{2}} \hat{1} \mathcolor{gray}{\bar{k}}} = \chi^{\;\! \mathcolor{gray}{\omega} \hat{4} \mathcolor{gray}{\hat{3}}}_{\;\! \symup{\iota} \textcolor{Maroon}{(1)}\textcolor{NavyBlue}{\text{n}}} \mathbb{i} \mathcolor{gray}{\omega} B^{\;\!\mathcolor{gray}{\omega}}_{\;\! \mathcolor{gray}{\hat{3}} \hat{4} \mathcolor{gray}{\bar{k}}}$ 中的\textcolor{NavyBlue}{电偶-磁四}极\Footnote{由于使用到了定义(即 $:=$),这里并没有使用到前述电\textcolor{Plum}{多极}子角标的\textcolor{Plum}{置换对称性} \bref{eq:symmetry1}。对于构成极化率的\textcolor{Plum}{多极}矩算子,当磁偶极子共用电四极子的极化率(而不是低一阶的磁化率)时,并没有继承电四极子的\textcolor{Plum}{置换对称性},比如有 $\chi^{\;\! \mathcolor{gray}{\omega} \hat{1} \mathcolor{gray}{\hat{2} \hat{3}}}_{\;\! \symup{\iota} \textcolor{Maroon}{(1)}\textcolor{NavyBlue}{\text{o}}} = \chi^{\;\! \mathcolor{gray}{\omega} \hat{1} \mathcolor{gray}{\hat{3} \hat{2}}}_{\;\! \symup{\iota} \textcolor{Maroon}{(1)}\textcolor{NavyBlue}{\text{o}}}$,但 $\chi^{\;\! \mathcolor{gray}{\omega} \hat{1} \mathcolor{gray}{\hat{2} \hat{3}}}_{\;\! \symup{\iota} \textcolor{Maroon}{(1)}\textcolor{NavyBlue}{\text{n}}} := \chi^{\;\! \mathcolor{gray}{\omega} \hat{4} \mathcolor{gray}{\hat{3}}}_{\;\! \symup{\iota} \textcolor{Maroon}{(1)}\textcolor{NavyBlue}{\text{n}}} \epsilon^{\hphantom{\hat{4}} \mathcolor{gray}{\hat{2}} \hat{1}}_{\hat{4}} \neq \chi^{\;\! \mathcolor{gray}{\omega} \hat{4} \mathcolor{gray}{\hat{2}}}_{\;\! \symup{\iota} \textcolor{Maroon}{(1)}\textcolor{NavyBlue}{\text{n}}} \epsilon^{\hphantom{\hat{4}} \mathcolor{gray}{\hat{3}} \hat{1}}_{\hat{4}} =: \chi^{\;\! \mathcolor{gray}{\omega} \hat{1} \mathcolor{gray}{\hat{3} \hat{2}}}_{\;\! \symup{\iota} \textcolor{Maroon}{(1)}\textcolor{NavyBlue}{\text{n}}}$。此外,对于单个磁\textcolor{Plum}{多极}子内部,如$m_{ij}$,其角标间也没有\textcolor{Plum}{置换对称性},--- 这意味着 $\chi^{\;\! \mathcolor{gray}{\omega} \hat{4} \mathcolor{gray}{\hat{3}}}_{\;\! \symup{\iota} \textcolor{Maroon}{(1)}\textcolor{NavyBlue}{\text{n}}} \neq \chi^{\;\! \mathcolor{gray}{\omega} \mathcolor{gray}{\hat{3}} \hat{4}}_{\;\! \symup{\iota} \textcolor{Maroon}{(1)}\textcolor{NavyBlue}{\text{n}}}$。}。

此外,还需要提醒的是,空域偏导算子 $\mathcolor{gray}{\nabla_{\hat{m}}}$ 作用于(非矩阵指数 \bref{eq:vec-matrix_exp} 的\textcolor{PineGreen}{平面波基}\cite{xieAnalytic3DVector}下)的\textcolor{gray}{单色}时变场后,严格来说得到的不是 \bref{eq:P(1)_wk} 中的
\begin{subequations}
\begin{align}
	\mathcolor{gray}{k_{\hat{m}}} = \mathcolor{gray}{k_{\symup{x}}} ~\textcolor{Maroon}{\text{或}}~ \mathcolor{gray}{k_{\symup{y}}} ~\textcolor{Maroon}{\text{或}}~ \mathcolor{gray}{k_{\symup{z}}}~, \label{eq:k_hat_m} \\
	\text{而是}~~~~ k^{\;\! \mathcolor{gray}{\omega}}_{\hat{m}} = \mathcolor{gray}{k_{\symup{x}}} ~\textcolor{Maroon}{\text{或}}~ \mathcolor{gray}{k_{\symup{y}}} ~\textcolor{Maroon}{\text{或}}~ k_{\symup{z}}^{\;\! \mathcolor{gray}{\omega}}~, \label{eq:k_what_m}
\end{align}
\end{subequations}
因此,严格来说\textcolor{Maroon}{多极理论}与 \textcolor{Maroon}{Volterra 级数}各自给出的\textcolor{Plum}{非局域}表达式不同,下文会再次更新 \bref{eq:P_tr,eq:P_wk,eq:P(1)_wk} 所涉及的所有公式的定义,以将其彻底纳入\textcolor{Maroon}{多极理论}框架内(将 \textcolor{Maroon}{Volterra 级数}视为辅助解释,但不用之计算)。

接着,同样以 \bref{eq:P_wk} 的二阶\textcolor{Plum}{非线性}项 $P^{\;\! \textcolor{Maroon}{(2)} \mathcolor{gray}{\omega}}_{\;\! \symup{\iota}\mathcolor{gray}{\bar{k}}} = \chi^{\;\! \mathcolor{gray}{\omega} \hat{1} \hat{2}}_{\;\! \symup{\iota} \mathcolor{gray}{\bar{k}} \textcolor{Maroon}{(2)}} \left( E^{\;\!\mathcolor{gray}{\omega}}_{\;\! \hat{1} \mathcolor{gray}{\bar{k}}} ~\mathcolor{gray}{\widetilde \circledast}~ E^{\;\!\mathcolor{gray}{\omega}}_{\;\! \hat{2} \mathcolor{gray}{\bar{k}}} \right)$ 为例,材料常数 $\chi^{\;\! \mathcolor{gray}{\omega} \hat{1} \hat{2}}_{\;\! \symup{\iota} \mathcolor{gray}{\bar{k}} \textcolor{Maroon}{(2)}}$ 的空间/\textcolor{PineGreen}{波矢} $\mathcolor{gray}{\bar{k}}$ \textcolor{NavyBlue}{色散}部分,可以如下地转移到场 $E^{\;\!\mathcolor{gray}{\omega}}_{\;\! \hat{n} \mathcolor{gray}{\bar{k}}}$ 中
\begin{subequations}
\begin{align}
	P^{\;\! \textcolor{Maroon}{(2)} \mathcolor{gray}{\omega}}_{\;\! \symup{\iota}\mathcolor{gray}{\bar{k}}} =&~ {\symup{\varepsilon_0}} \chi^{\;\! \mathcolor{gray}{\omega} \hat{1} \hat{2}}_{\;\! \symup{\iota} \mathcolor{gray}{\bar{k}} \textcolor{Maroon}{(2)}} \left( E^{\;\!\mathcolor{gray}{\omega}}_{\;\! \hat{1} \mathcolor{gray}{\bar{k}}} ~\mathcolor{gray}{\widetilde \circledast}~ E^{\;\!\mathcolor{gray}{\omega}}_{\;\! \hat{2} \mathcolor{gray}{\bar{k}}} \right) \\ =&~ {\symup{\varepsilon_0}} \left\{ \chi^{\;\! \mathcolor{gray}{\omega} \hat{1} \hat{2}}_{\;\! \symup{\iota} \textcolor{Maroon}{(2)}} \left( E^{\;\!\mathcolor{gray}{\omega}}_{\;\! \hat{1} \mathcolor{gray}{\bar{k}}} ~\mathcolor{gray}{\widetilde \circledast}~ E^{\;\!\mathcolor{gray}{\omega}}_{\;\! \hat{2} \mathcolor{gray}{\bar{k}}} \right) + \chi^{\;\! \mathcolor{gray}{\omega} \hat{1} \hat{2} \mathcolor{gray}{\hat{3}}}_{\;\! \symup{\iota} \textcolor{Maroon}{(2)}} \left( E^{\;\!\mathcolor{gray}{\omega}}_{\;\! \mathcolor{gray}{\hat{3}} \hat{1} \mathcolor{gray}{\bar{k}}} ~\mathcolor{gray}{\widetilde \circledast}~ E^{\;\!\mathcolor{gray}{\omega}}_{\;\! \hat{2} \mathcolor{gray}{\bar{k}}} \right) + \cdots \right\} \label{eq:P(2)_wk}
	\\ :=&~ {\symup{\varepsilon_0}} \left\{ \chi^{\;\! \mathcolor{gray}{\omega} \hat{1} \hat{2}}_{\;\! \symup{\iota} \textcolor{Maroon}{(2)}} \left( E^{\;\!\mathcolor{gray}{\omega}}_{\;\! \hat{1} \mathcolor{gray}{\bar{k}}} ~\mathcolor{gray}{\widetilde \circledast}~ E^{\;\!\mathcolor{gray}{\omega}}_{\;\! \hat{2} \mathcolor{gray}{\bar{k}}} \right) + \chi^{\;\! \mathcolor{gray}{\omega} \hat{1} \hat{2} \mathcolor{gray}{\hat{3}}}_{\;\! \symup{\iota} \textcolor{Maroon}{(2)}} \left[ \left( \mathcolor{gray}{k_{\hat{3}}} E^{\;\!\mathcolor{gray}{\omega}}_{\;\! \hat{1} \mathcolor{gray}{\bar{k}}} \right) ~\mathcolor{gray}{\widetilde \circledast}~ E^{\;\!\mathcolor{gray}{\omega}}_{\;\! \hat{2} \mathcolor{gray}{\bar{k}}} \right] + \cdots \right\}~,
\end{align}
\end{subequations}
其中,利用卷积交换律、重复指标对跨对\textcolor{Plum}{置换对称性}\cite{boydNonlinearOptics2019}得 $\chi^{\;\! \mathcolor{gray}{\omega} \hat{1} \hat{2} \mathcolor{gray}{\hat{3}}}_{\;\! \symup{\iota} \textcolor{Maroon}{(2)}} \left( E^{\;\!\mathcolor{gray}{\omega}}_{\;\! \mathcolor{gray}{\hat{3}} \hat{1} \mathcolor{gray}{\bar{k}}} ~\mathcolor{gray}{\widetilde \circledast}~ E^{\;\!\mathcolor{gray}{\omega}}_{\;\! \hat{2} \mathcolor{gray}{\bar{k}}} \right) = \chi^{\;\! \mathcolor{gray}{\omega} \hat{1} \hat{2} \mathcolor{gray}{\hat{3}}}_{\;\! \symup{\iota} \textcolor{Maroon}{(2)}} \left( E^{\;\!\mathcolor{gray}{\omega}}_{\;\! \hat{2} \mathcolor{gray}{\bar{k}}} ~\mathcolor{gray}{\widetilde \circledast}~ E^{\;\!\mathcolor{gray}{\omega}}_{\;\! \mathcolor{gray}{\hat{3}} \hat{1} \mathcolor{gray}{\bar{k}}} \right) = \chi^{\;\! \mathcolor{gray}{\omega} \hat{2} \hat{1} \mathcolor{gray}{\hat{3}}}_{\;\! \symup{\iota} \textcolor{Maroon}{(2)}} \left( E^{\;\!\mathcolor{gray}{\omega}}_{\;\! \hat{1} \mathcolor{gray}{\bar{k}}} ~\mathcolor{gray}{\widetilde \circledast}~ E^{\;\!\mathcolor{gray}{\omega}}_{\;\! \mathcolor{gray}{\hat{3}} \hat{2} \mathcolor{gray}{\bar{k}}} \right)$;另一方面,利用卷积的导数定理有 $\mathcolor{gray}{\nabla_{\hat{3}}} \left( E^{\;\!\mathcolor{gray}{\omega}}_{\;\! \hat{1} \mathcolor{gray}{\bar{k}}} ~\mathcolor{gray}{\widetilde \circledast}~ E^{\;\!\mathcolor{gray}{\omega}}_{\;\! \hat{2} \mathcolor{gray}{\bar{k}}} \right) = E^{\;\!\mathcolor{gray}{\omega}}_{\;\! \mathcolor{gray}{\hat{3}} \hat{1} \mathcolor{gray}{\bar{k}}} ~\mathcolor{gray}{\widetilde \circledast}~ E^{\;\!\mathcolor{gray}{\omega}}_{\;\! \hat{2} \mathcolor{gray}{\bar{k}}} = E^{\;\!\mathcolor{gray}{\omega}}_{\;\! \hat{1} \mathcolor{gray}{\bar{k}}} ~\mathcolor{gray}{\widetilde \circledast}~ E^{\;\!\mathcolor{gray}{\omega}}_{\;\! \mathcolor{gray}{\hat{3}} \hat{2} \mathcolor{gray}{\bar{k}}}$\Footnote{同样出于 $\mathcolor{gray}{k_{\hat{m}}} \neq k^{\;\! \mathcolor{gray}{\omega}}_{\hat{m}}$ 的原因,这里并不严谨:须完全在\textcolor{Maroon}{多极理论}框架内才有该结论;但不影响其成立。 ---  仔细观察会发现,该条连等式是错误的。因为此处的卷积算符,以及场的\textcolor{gray}{自变量},都完全在 $\mathcolor{gray}{\bar{\kappa}}$ 域,而 $\mathcolor{gray}{\nabla_{\hat{3}}}$ 却在 $\mathcolor{gray}{\bar{x}}$ 域(中的 $\mathcolor{gray}{\bar{r}}$ 域),因此原则上应有 $\mathcolor{gray}{\nabla_{\hat{3}}} \left( E^{\;\!\mathcolor{gray}{\omega}}_{\;\! \hat{1} \mathcolor{gray}{\bar{k}}} ~\mathcolor{gray}{\widetilde \circledast}~ E^{\;\!\mathcolor{gray}{\omega}}_{\;\! \hat{2} \mathcolor{gray}{\bar{k}}} \right) = 0$,而不是所得出的错误结论 $E^{\;\!\mathcolor{gray}{\omega}}_{\;\! \mathcolor{gray}{\hat{3}} \hat{1} \mathcolor{gray}{\bar{k}}} ~\mathcolor{gray}{\widetilde \circledast}~ E^{\;\!\mathcolor{gray}{\omega}}_{\;\! \hat{2} \mathcolor{gray}{\bar{k}}} = E^{\;\!\mathcolor{gray}{\omega}}_{\;\! \hat{1} \mathcolor{gray}{\bar{k}}} ~\mathcolor{gray}{\widetilde \circledast}~ E^{\;\!\mathcolor{gray}{\omega}}_{\;\! \mathcolor{gray}{\hat{3}} \hat{2} \mathcolor{gray}{\bar{k}}}$,尽管这不影响黑色指标集内部的\textcolor{Plum}{置换对称性}成立。然而,由于此处的“卷积的导数定理”以及将场排列成匹配\textcolor{Plum}{多极}子的\textcolor{Maroon}{模式},既是在\textcolor{Maroon}{多极理论}框架下的惯性选择,也具有一定的启发意义,本文选择在此处保留该错误过程。正确的结论应由“导数的\textcolor{Plum}{傅立叶变换}” --- 即\textcolor{Plum}{傅立叶变换}的导数定理(少了“定理”两字则天壤之别)给出,即 $\mathcolor{gray}{\mathcal F_{\bar{\kappa}}} \left[ \mathcolor{gray}{\nabla_{\hat{3}}} \left( E^{\;\!\mathcolor{gray}{t}}_{\;\! \hat{1} \mathcolor{gray}{\bar{r}}} E^{\;\!\mathcolor{gray}{t}}_{\;\! \hat{2} \mathcolor{gray}{\bar{r}}} \right) \right] = \mathcolor{gray}{k_{\hat{3}}} \left( E^{\;\!\mathcolor{gray}{\omega}}_{\;\! \hat{1} \mathcolor{gray}{\bar{k}}} ~\mathcolor{gray}{\widetilde \circledast}~ E^{\;\!\mathcolor{gray}{\omega}}_{\;\! \hat{2} \mathcolor{gray}{\bar{k}}} \right) = E^{\;\!\mathcolor{gray}{\omega}}_{\;\! \mathcolor{gray}{\hat{3}} \hat{1} \mathcolor{gray}{\bar{k}}} ~\mathcolor{gray}{\widetilde \circledast}~ E^{\;\!\mathcolor{gray}{\omega}}_{\;\! \hat{2} \mathcolor{gray}{\bar{k}}} + E^{\;\!\mathcolor{gray}{\omega}}_{\;\! \hat{1} \mathcolor{gray}{\bar{k}}} ~\mathcolor{gray}{\widetilde \circledast}~ E^{\;\!\mathcolor{gray}{\omega}}_{\;\! \mathcolor{gray}{\hat{3}} \hat{2} \mathcolor{gray}{\bar{k}}}$。可类推 $\mathcolor{gray}{k_{\hat{4}}} \mathcolor{gray}{k_{\hat{3}}} \left( E^{\;\!\mathcolor{gray}{\omega}}_{\;\! \hat{1} \mathcolor{gray}{\bar{k}}} ~\mathcolor{gray}{\widetilde \circledast}~ E^{\;\!\mathcolor{gray}{\omega}}_{\;\! \hat{2} \mathcolor{gray}{\bar{k}}} \right)$。},可得 $\chi^{\;\! \mathcolor{gray}{\omega} \hat{1} \hat{2} \mathcolor{gray}{\hat{3}}}_{\;\! \symup{\iota} \textcolor{Maroon}{(2)}} \left( E^{\;\!\mathcolor{gray}{\omega}}_{\;\! \mathcolor{gray}{\hat{3}} \hat{1} \mathcolor{gray}{\bar{k}}} ~\mathcolor{gray}{\widetilde \circledast}~ E^{\;\!\mathcolor{gray}{\omega}}_{\;\! \hat{2} \mathcolor{gray}{\bar{k}}} \right) = \chi^{\;\! \mathcolor{gray}{\omega} \hat{1} \hat{2} \mathcolor{gray}{\hat{3}}}_{\;\! \symup{\iota} \textcolor{Maroon}{(2)}} \left( E^{\;\!\mathcolor{gray}{\omega}}_{\;\! \hat{1} \mathcolor{gray}{\bar{k}}} ~\mathcolor{gray}{\widetilde \circledast}~ E^{\;\!\mathcolor{gray}{\omega}}_{\;\! \mathcolor{gray}{\hat{3}} \hat{2} \mathcolor{gray}{\bar{k}}} \right)$;两相对比,即有 $\chi^{\;\! \mathcolor{gray}{\omega} \hat{1} \hat{2} \mathcolor{gray}{\hat{3}}}_{\;\! \symup{\iota} \textcolor{Maroon}{(2)}} = \chi^{\;\! \mathcolor{gray}{\omega} \hat{2} \hat{1} \mathcolor{gray}{\hat{3}}}_{\;\! \symup{\iota} \textcolor{Maroon}{(2)}}$。同理,有 $\chi^{\;\! \mathcolor{gray}{\omega} \hat{1} \hat{2}}_{\;\! \symup{\iota} \textcolor{Maroon}{(2)}} = \chi^{\;\! \mathcolor{gray}{\omega} \hat{2} \hat{1}}_{\;\! \symup{\iota} \textcolor{Maroon}{(2)}}$。

推而广之,作为上标(连接待转换的“\textcolor{Plum}{原因张量}”)的所有数字指标 $\hat{1} \hat{2} \mathcolor{gray}{\hat{3}}$ 中,黑色的指标之间,可以互相置换;同理,\textcolor{gray}{灰色}的指标集内,也可以互相置换\Footnote{后者是因为,空域偏导算符作用于哪个场分量,以及作用的先后顺序,均是互相独立的 --- 它也有交换律,重复指标对也可同时替换为另一对。这与前者是类似的:从 $\chi^{\;\! \hat{1} \hat{2}}_{\;\! \symup{\iota} \textcolor{Maroon}{(2)}} E_{\;\! \hat{1}} E_{\;\! \hat{2}}$ 相似于 $\chi^{\;\! \mathcolor{gray}{\hat{3}} \mathcolor{gray}{\hat{4}}}_{\;\! \symup{\iota} \textcolor{Maroon}{(2)}} \mathcolor{gray}{k_{\hat{3}}} \mathcolor{gray}{k_{\hat{4}}}$ 的角度。 --- 然而,如果从重复指标求和的角度,而不是(甚至即使)单独看求和的每一项\cite{boydNonlinearOptics2019},则“重复指标对跨对\textcolor{Plum}{置换对称性}”将不成立,以至于黑色、\textcolor{gray}{灰色}指标集各自内部,不一定有交换律/\textcolor{Plum}{置换对称性},例如 \bref{eq:p<->n}。};如果还有\textcolor{gray}{灰}-黑色指标间跨集相互置换\Footnote{暂未见文献报道该条规则成立。但仍在 \bref{eq:symmetry4} 前实现了对该规则的初步证明。},则除第一个(连接转换至的“\textcolor{Plum}{结果张量}”)的希腊脚标 $\symup{\iota}$ 外,\textcolor{Plum}{非线性}极化率剩余角标具有全排列\textcolor{Plum}{置换对称性}。 --- 这比超极化率(黑色指标间)的内禀\textcolor{Plum}{置换对称性}更高级,除了继承了场的角标(以及\textcolor{gray}{频率})不用同时跟随交换,还包含\textcolor{PineGreen}{波矢}(\textcolor{gray}{灰色}指标集),类似前述 \bref{eq:symmetry1}。此即
\begin{align}
	\textbf{\text{\textcolor{Maroon}{Volterra 级数}版的\textcolor{Plum}{非线性}极化率的内禀\textcolor{Plum}{置换对称性}}}~, \label{eq:symmetry3}
\end{align}
它既不强调场的角标和\textcolor{gray}{频率}(同时跟随交换),也不强调极化率的\textcolor{gray}{频率(组分)}。该\textcolor{Plum}{置换对称性}有点像,但不是 \bref{eq:symmetry2};也暂时不等价于 \bref{eq:symmetry1}。

此外,可以看出,\textcolor{Plum}{非线性}项仍然可以是\textcolor{Plum}{非局域}的,因此\textcolor{Plum}{(非)线性}响应与\textcolor{Plum}{(非)局域}响应又成为一对(新的)相互独立的概念和\textcolor{Plum}{自由度}。这意味着 \bref{eq:P(2)_wk} 中也包含\textcolor{NavyBlue}{电偶-电磁多}极相互作用:比如 $\chi^{\;\! \mathcolor{gray}{\omega} \hat{1} \hat{2} \mathcolor{gray}{\hat{3}}}_{\;\! \symup{\iota} \textcolor{Maroon}{(2)}} \sim p_{\;\! \symup{\iota}} p^{\;\! \hat{1}} q^{\;\! \hat{2} \mathcolor{gray}{\hat{3}}}$ 就代表\textcolor{NavyBlue}{电偶-$(\text{电偶}\otimes\text{电四/磁偶})$}极相互作用,且同样有单个\textcolor{Plum}{多极}子(如 $q^{\;\! \hat{2} \mathcolor{gray}{\hat{3}}}$)内部、可能有多个{\textcolor{Plum}{多极}子}(如$\{p_{\;\! \symup{\iota}}, p^{\;\! \hat{1}}, q^{\;\! \hat{2} \mathcolor{gray}{\hat{3}}}\}$)之间的角标\textcolor{Plum}{置换对称性}。 ---  当后者(即 \bref{eq:symmetry2})成立时,\textcolor{Plum}{非线性}\textcolor{NavyBlue}{光学}中无\textcolor{NavyBlue}{损}/无\textcolor{NavyBlue}{耗散}/无\textcolor{NavyBlue}{吸收}即远离\textcolor{NavyBlue}{共振}情况下的\textcolor{Plum}{全排列对称性}\cite{boydNonlinearOptics2019}\Footnote{在\textcolor{Plum}{非线性}\textcolor{NavyBlue}{光学}中,角标(笛卡尔指标)的\textcolor{Plum}{全排列对称性},仍要求\textcolor{gray}{频率}同时跟随交换。},只是\textcolor{Maroon}{多极理论}中 \bref{eq:symmetry2} 的特例,并且不像后者,在\textcolor{NavyBlue}{吸收}和\textcolor{NavyBlue}{共振}时仍然成立,举例如 $\chi^{\;\! \mathcolor{gray}{\omega} \hat{1} \hat{2}}_{\;\! \symup{\iota} \textcolor{Maroon}{(2)}} \sim p_{\;\! \symup{\iota}} p^{\;\! \hat{1}} p^{\;\! \hat{2}} = p^{\;\! \hat{1}} p_{\;\! \symup{\iota}} p^{\;\! \hat{2}} \sim \chi^{\;\! \mathcolor{gray}{\omega} \symup{\iota} \hat{2}}_{\;\! \hat{1} \textcolor{Maroon}{(2)}}$\cite{raabMultipoleTheoryElectromagnetism2004}。

既然 $\bar{P}^{\;\!\mathcolor{gray}{\omega}}_{\;\!\mathcolor{gray}{\bar{k}}}$ 的\textcolor{Plum}{非线性}部分 \bref{eq:P(2)_wk} 也包含了\textcolor{Plum}{非局域}的电场 $E^{\;\!\mathcolor{gray}{\omega}}_{\;\! \hat{n} \mathcolor{gray}{\bar{k}}}$ 空间导数相关的\textcolor{NavyBlue}{电四/电八}极响应,那么其也包含了磁场 $B^{\;\!\mathcolor{gray}{\omega}}_{\;\! \hat{n} \mathcolor{gray}{\bar{k}}}$ 相关的\textcolor{NavyBlue}{磁偶/磁四}极响应。即也就是说,\bref{eq:P(2)_wk} 中,包含\textcolor{PineGreen}{波矢}/空间\textcolor{NavyBlue}{色散}的极化率,都可以拆分为\textcolor{NavyBlue}{电}$\longleftarrow$\textcolor{NavyBlue}{电$\otimes$ $\nabla \text{电}$}、\textcolor{NavyBlue}{电}$\longleftarrow$\textcolor{NavyBlue}{电$\otimes$磁}等成分,比如 $\chi^{\;\! \mathcolor{gray}{\omega} \hat{1} \hat{2} \mathcolor{gray}{\hat{3}}}_{\;\! \symup{\iota} \textcolor{Maroon}{(2)}} = \chi^{\;\! \mathcolor{gray}{\omega} \hat{1} \hat{2} \mathcolor{gray}{\hat{3}}}_{\;\! \symup{\iota} \textcolor{Maroon}{(2)}\textcolor{NavyBlue}{\text{pq}}} + \chi^{\;\! \mathcolor{gray}{\omega} \hat{1} \hat{2} \mathcolor{gray}{\hat{3}}}_{\;\! \symup{\iota} \textcolor{Maroon}{(2)}\textcolor{NavyBlue}{\text{pm}}}$\Footnote{利用卷积的导数定理,可以发现 $\chi^{\;\! \mathcolor{gray}{\omega} \hat{1} \hat{2} \mathcolor{gray}{\hat{3}}}_{\;\! \symup{\iota} \textcolor{Maroon}{(2)}\textcolor{NavyBlue}{\text{pq}}} = \chi^{\;\! \mathcolor{gray}{\omega} \hat{1} \hat{2} \mathcolor{gray}{\hat{3}}}_{\;\! \symup{\iota} \textcolor{Maroon}{(2)}\textcolor{NavyBlue}{\text{qp}}}$、$\chi^{\;\! \mathcolor{gray}{\omega} \hat{1} \hat{2} \mathcolor{gray}{\hat{3}}}_{\;\! \symup{\iota} \textcolor{Maroon}{(2)}\textcolor{NavyBlue}{\text{pm}}} = \chi^{\;\! \mathcolor{gray}{\omega} \hat{1} \hat{2} \mathcolor{gray}{\hat{3}}}_{\;\! \symup{\iota} \textcolor{Maroon}{(2)}\textcolor{NavyBlue}{\text{mp}}}$。}。包含更高阶空间\textcolor{NavyBlue}{色散}的\textcolor{NavyBlue}{电偶}极\textcolor{Plum}{非线性}极化率,还将包含更复杂的电磁组合\textcolor{Plum}{非线性}成分,如 $\chi^{\;\! \mathcolor{gray}{\omega} \hat{1} \hat{2} \mathcolor{gray}{\hat{3} \hat{4}}}_{\;\! \symup{\iota} \textcolor{Maroon}{(2)}} = \chi^{\;\! \mathcolor{gray}{\omega} \hat{1} \hat{2} \mathcolor{gray}{\hat{3} \hat{4}}}_{\;\! \symup{\iota} \textcolor{Maroon}{(2)}\textcolor{NavyBlue}{\text{po}}} + \chi^{\;\! \mathcolor{gray}{\omega} \hat{1} \hat{2} \mathcolor{gray}{\hat{3} \hat{4}}}_{\;\! \symup{\iota} \textcolor{Maroon}{(2)}\textcolor{NavyBlue}{\text{pn}}} + \chi^{\;\! \mathcolor{gray}{\omega} \hat{1} \hat{2} \mathcolor{gray}{\hat{3} \hat{4}}}_{\;\! \symup{\iota} \textcolor{Maroon}{(2)}\textcolor{NavyBlue}{\text{qq}}} + \chi^{\;\! \mathcolor{gray}{\omega} \hat{1} \hat{2} \mathcolor{gray}{\hat{3} \hat{4}}}_{\;\! \symup{\iota} \textcolor{Maroon}{(2)}\textcolor{NavyBlue}{\text{qm}}} + \chi^{\;\! \mathcolor{gray}{\omega} \hat{1} \hat{2} \mathcolor{gray}{\hat{3} \hat{4}}}_{\;\! \symup{\iota} \textcolor{Maroon}{(2)}\textcolor{NavyBlue}{\text{mm}}}$\Footnote{其中,应用卷积的导数定理,原则上也有 $\chi^{\;\! \mathcolor{gray}{\omega} \hat{1} \hat{2} \mathcolor{gray}{\hat{3} \hat{4}}}_{\;\! \symup{\iota} \textcolor{Maroon}{(2)}\textcolor{NavyBlue}{\text{po}}} = \chi^{\;\! \mathcolor{gray}{\omega} \hat{1} \hat{2} \mathcolor{gray}{\hat{3} \hat{4}}}_{\;\! \symup{\iota} \textcolor{Maroon}{(2)}\textcolor{NavyBlue}{\text{qq}}}$。} 中还包含\textcolor{NavyBlue}{电}$\longleftarrow$\textcolor{NavyBlue}{电$\otimes$ $\nabla\nabla \text{电}$}、\textcolor{NavyBlue}{电}$\longleftarrow$\textcolor{NavyBlue}{电$\otimes$ $\nabla \text{磁}$}、\textcolor{NavyBlue}{电}$\longleftarrow$\textcolor{NavyBlue}{$\nabla \text{电}$ $\otimes$ $\nabla \text{电}$}、\textcolor{NavyBlue}{电}$\longleftarrow$\textcolor{NavyBlue}{$\nabla \text{电}$ $\otimes$ 磁}、\textcolor{NavyBlue}{电}$\longleftarrow$\textcolor{NavyBlue}{磁 $\otimes$ 磁}。

仿照 \bref{eq:symmetry1} 所暗含的 \bref{eq:P(1)_wk} 中“$E^{\;\!\mathcolor{gray}{\omega}}_{\;\! \hat{1} \mathcolor{gray}{\bar{k}}}$ 对应 $\chi^{\;\! \mathcolor{gray}{\omega} \hat{1}}_{\;\! \symup{\iota} \textcolor{Maroon}{(1)}} \sim p_{\;\! \symup{\iota}} p^{\;\! \hat{1}}$ 中的 $p^{\;\! \hat{1}}$,$E^{\;\!\mathcolor{gray}{\omega}}_{\;\! \mathcolor{gray}{\hat{2}} \hat{1} \mathcolor{gray}{\bar{k}}}$ 对应 $\chi^{\;\! \mathcolor{gray}{\omega} \hat{1} \mathcolor{gray}{\hat{2}}}_{\;\! \symup{\iota} \textcolor{Maroon}{(1)}} \sim p_{\;\! \symup{\iota}} q^{\;\! \hat{1} \hat{2}}$ 中的 $q^{\;\! \hat{1} \hat{2}}$,$E^{\;\!\mathcolor{gray}{\omega}}_{\;\! \mathcolor{gray}{\hat{3} \hat{2}} \hat{1} \mathcolor{gray}{\bar{k}}}$ 对应 $\chi^{\;\! \mathcolor{gray}{\omega} \hat{1} \mathcolor{gray}{\hat{2} \hat{3}}}_{\;\! \symup{\iota} \textcolor{Maroon}{(1)}} \sim p_{\;\! \symup{\iota}} o^{\;\! \hat{1} \hat{2} \hat{3}}$ 中的 $o^{\;\! \hat{1} \hat{2} \hat{3}}$”,可以发现 \bref{eq:Q(1)_wk} 中的 $\chi^{\;\! \mathcolor{gray}{\omega} \hat{1} \hat{2} \mathcolor{gray}{\hat{3}}}_{\;\! \symup{\iota} \textcolor{Maroon}{(2)}} = \chi^{\;\! \mathcolor{gray}{\omega} \hat{2} \hat{1} \mathcolor{gray}{\hat{3}}}_{\;\! \symup{\iota} \textcolor{Maroon}{(2)}}$(利用了 \bref{eq:symmetry3})并没有指定 $\chi^{\;\! \mathcolor{gray}{\omega} \hat{1} \hat{2} \mathcolor{gray}{\hat{3}}}_{\;\! \symup{\iota} \textcolor{Maroon}{(2)}} \sim p_{\;\! \symup{\iota}} q^{\;\! \hat{1}} q^{\;\! \hat{2} \mathcolor{gray}{\hat{3}}}$ 还是 $\chi^{\;\! \mathcolor{gray}{\omega} \hat{2} \hat{1} \mathcolor{gray}{\hat{3}}}_{\;\! \symup{\iota} \textcolor{Maroon}{(2)}} \sim p_{\;\! \symup{\iota}} q^{\;\! \hat{2}} q^{\;\! \hat{1} \mathcolor{gray}{\hat{3}}}$,因此原则上两种分解都是对的。然而根据 \bref{eq:symmetry1},上述两种分解中的 $q^{\;\! \hat{2} \mathcolor{gray}{\hat{3}}} = q^{\;\! \mathcolor{gray}{\hat{3}} \hat{2}}$、$q^{\;\! \mathcolor{gray}{\hat{3}} \hat{1}} = q^{\;\! \hat{1} \mathcolor{gray}{\hat{3}}}$ 均允许黑$\longleftrightarrow$ \textcolor{gray}{灰}指标交换。也就是说,证明了允许任一数字\textcolor{gray}{灰}指标与任一数字黑指标交换。该规律可拓展至上一段中的 $\chi^{\;\! \mathcolor{gray}{\omega} \hat{1} \hat{2} \mathcolor{gray}{\hat{3} \hat{4}}}_{\;\! \symup{\iota} \textcolor{Maroon}{(2)}\textcolor{NavyBlue}{\text{po}}}, \chi^{\;\! \mathcolor{gray}{\omega} \hat{1} \hat{2} \mathcolor{gray}{\hat{3} \hat{4}}}_{\;\! \symup{\iota} \textcolor{Maroon}{(2)}\textcolor{NavyBlue}{\text{pn}}}, \chi^{\;\! \mathcolor{gray}{\omega} \hat{1} \hat{2} \mathcolor{gray}{\hat{3} \hat{4}}}_{\;\! \symup{\iota} \textcolor{Maroon}{(2)}\textcolor{NavyBlue}{\text{qq}}}, \chi^{\;\! \mathcolor{gray}{\omega} \hat{1} \hat{2} \mathcolor{gray}{\hat{3} \hat{4}}}_{\;\! \symup{\iota} \textcolor{Maroon}{(2)}\textcolor{NavyBlue}{\text{qm}}}, \chi^{\;\! \mathcolor{gray}{\omega} \hat{1} \hat{2} \mathcolor{gray}{\hat{3} \hat{4}}}_{\;\! \symup{\iota} \textcolor{Maroon}{(2)}\textcolor{NavyBlue}{\text{mm}}}$ 也分别适用。再将该条规律结合 \bref{eq:symmetry3},这样就在一定程度上证明了
\begin{align}
	\textbf{\text{连接“\textcolor{Plum}{原因张量}”的(数字)上指标的\textcolor{Plum}{全排列对称性}}}~. \label{eq:symmetry4}
\end{align}
此外,一旦材料常数中的任何角标因 \bref{eq:symmetry1,eq:symmetry2,eq:symmetry3,eq:symmetry4} 而变得可交换,那么对应的场中的相应角标,也可按相同规则交换\Footnote{这是因为,材料常数和其所转换的(作为原因/驱动的)场中,能发生\textcolor{Plum}{爱因斯坦求和}的相同角标,可以替换为其他任何角标,或同步交换(即材料常数的角标交换,且场的对应角标也同时交换时,结果不变)。--- 然而,该规则成立的理由在\textcolor{Plum}{数学}上不充分。只是它在\textcolor{NavyBlue}{物理}上“没有不成立的必要”、“如果成立会更简单”。} --- 的同时保持材料常数的所有角标排列顺序不变。

类似 \bref{eq:P(1)_wk},电四极矩 $Q^{\;\!\mathcolor{gray}{t}}_{\;\! \symup{\iota} \hat{1} \mathcolor{gray}{z}} \left( \mathcolor{gray}{\nabla^t}, \mathcolor{gray}{\nabla_{\hat{2}}} ; E^{\;\!\mathcolor{gray}{t}}_{\;\! \hat{3}\mathcolor{gray}{z}}, B^{\;\!\mathcolor{gray}{t}}_{\;\! \hat{4}\mathcolor{gray}{z}} \right)$ 在 $\mathcolor{gray}{\bar{\kappa}}$ 域的\textcolor{Plum}{线性}响应部分
\begin{subequations}
\begin{align}
	Q^{\;\! \textcolor{Maroon}{(1)} \mathcolor{gray}{\omega}}_{\;\! \symup{\iota} \hat{1} \mathcolor{gray}{\bar{k}}} =&~ {\symup{\varepsilon_0}} \dot{\chi}^{\;\! \mathcolor{gray}{\omega} \hat{2}}_{\;\! \symup{\iota} \hat{1} \mathcolor{gray}{\bar{k}} \textcolor{Maroon}{(1)}} E^{\;\!\mathcolor{gray}{\omega}}_{\;\! \hat{2} \mathcolor{gray}{\bar{k}}} \\ =&~ {\symup{\varepsilon_0}} \left\{ \dot{\chi}^{\;\! \mathcolor{gray}{\omega} \hat{2}}_{\;\! \symup{\iota} \hat{1} \textcolor{Maroon}{(1)}} E^{\;\!\mathcolor{gray}{\omega}}_{\;\! \hat{2} \mathcolor{gray}{\bar{k}}} + \dot{\chi}^{\;\! \mathcolor{gray}{\omega} \hat{2} \mathcolor{gray}{\hat{3}}}_{\;\! \symup{\iota} \hat{1} \textcolor{Maroon}{(1)}} E^{\;\!\mathcolor{gray}{\omega}}_{\;\! \mathcolor{gray}{\hat{3}} \hat{2} \mathcolor{gray}{\bar{k}}} + \dot{\chi}^{\;\! \mathcolor{gray}{\omega} \hat{2} \mathcolor{gray}{\hat{3} \hat{4}}}_{\;\! \symup{\iota} \hat{1} \textcolor{Maroon}{(1)}} E^{\;\!\mathcolor{gray}{\omega}}_{\;\! \mathcolor{gray}{\hat{4} \hat{3}} \hat{2} \mathcolor{gray}{\bar{k}}} + \cdots \right\}  \label{eq:Q(1)_wk} \\ :=&~ {\symup{\varepsilon_0}} \left\{ \dot{\chi}^{\;\! \mathcolor{gray}{\omega} \hat{2}}_{\;\! \symup{\iota} \hat{1} \textcolor{Maroon}{(1)}} E^{\;\!\mathcolor{gray}{\omega}}_{\;\! \hat{2} \mathcolor{gray}{\bar{k}}} + \dot{\chi}^{\;\! \mathcolor{gray}{\omega} \hat{2} \mathcolor{gray}{\hat{3}}}_{\;\! \symup{\iota} \hat{1} \textcolor{Maroon}{(1)}} \left( \mathcolor{gray}{k_{\hat{3}}} E^{\;\!\mathcolor{gray}{\omega}}_{\;\! \hat{2} \mathcolor{gray}{\bar{k}}} \right) + \dot{\chi}^{\;\! \mathcolor{gray}{\omega} \hat{2} \mathcolor{gray}{\hat{3} \hat{4}}}_{\;\! \symup{\iota} \hat{1} \textcolor{Maroon}{(1)}} \left( \mathcolor{gray}{k_{\hat{4}} k_{\hat{3}}} E^{\;\!\mathcolor{gray}{\omega}}_{\;\! \hat{2} \mathcolor{gray}{\bar{k}}} \right) + \cdots \right\}~,
\end{align}
\end{subequations}
其中,\bref{hook:dot} 定义了 $\dot{\chi}$ 为 $Q$ 的极化率,以与 $P$ 的极化率 $\chi$ 区分开来。此时查看一下 $Q^{\;\! \textcolor{Maroon}{(1)}}_{\;\! \symup{\iota} \hat{1}}$ 在 $\mathcolor{gray}{\bar{\kappa}}$ 域的散度 $\mathcolor{gray}{k^{\hat{1}}} Q^{\;\! \textcolor{Maroon}{(1)}}_{\;\! \symup{\iota} \mathcolor{gray}{\hat{1}}}$,有
\begin{subequations}
\begin{align}
	\mathcolor{gray}{k^{\hat{1}}} Q^{\;\! \textcolor{Maroon}{(1)} \mathcolor{gray}{\omega}}_{\;\! \symup{\iota} \mathcolor{gray}{\hat{1}} \mathcolor{gray}{\bar{k}}} = {\symup{\varepsilon_0}} \mathcolor{gray}{k^{\hat{1}}} \dot{\chi}^{\;\! \mathcolor{gray}{\omega} \hat{2}}_{\;\! \symup{\iota} \mathcolor{gray}{\hat{1}} \mathcolor{gray}{\bar{k}} \textcolor{Maroon}{(1)}} E^{\;\!\mathcolor{gray}{\omega}}_{\;\! \hat{2} \mathcolor{gray}{\bar{k}}} &= {\symup{\varepsilon_0}} \dot{\chi}^{\;\! \mathcolor{gray}{\omega} \mathcolor{gray}{\hat{1}} \hat{2} }_{\;\! \symup{\iota} \mathcolor{gray}{\bar{k}} \textcolor{Maroon}{(1)}} \left( \mathcolor{gray}{k_{\hat{1}}} E^{\;\!\mathcolor{gray}{\omega}}_{\;\! \hat{2} \mathcolor{gray}{\bar{k}}} \right) \label{eq:div-Q(1)_wk} \\ &= {\symup{\varepsilon_0}} \dot{\chi}^{\;\! \mathcolor{gray}{\omega} \mathcolor{gray}{\hat{1}} \hat{2} }_{\;\! \symup{\iota} \mathcolor{gray}{\bar{k}} \textcolor{Maroon}{(1)}} E^{\;\!\mathcolor{gray}{\omega}}_{\;\! \mathcolor{gray}{\hat{1}} \hat{2} \mathcolor{gray}{\bar{k}}} = {\symup{\varepsilon_0}} \dot{\chi}^{\;\! \mathcolor{gray}{\omega} \mathcolor{gray}{\hat{2}} \hat{1} }_{\;\! \symup{\iota} \mathcolor{gray}{\bar{k}} \textcolor{Maroon}{(1)}} E^{\;\!\mathcolor{gray}{\omega}}_{\;\! \mathcolor{gray}{\hat{2}} \hat{1} \mathcolor{gray}{\bar{k}}} \label{eq:div-Q(1)_wk-b}~,
\end{align}
\end{subequations}
接着,将 \bref{eq:P(1)_wk,eq:div-Q(1)_wk-b} 按照 \bref{eq:D^(0)=} 的方式,进行 $P^{\;\! \textcolor{Maroon}{(1)} \mathcolor{gray}{\omega}}_{\;\! \symup{\iota}\mathcolor{gray}{\bar{k}}} - \mathcolor{gray}{k^{\hat{1}}} Q^{\;\! \textcolor{Maroon}{(1)} \mathcolor{gray}{\omega}}_{\;\! \symup{\iota} \mathcolor{gray}{\hat{1}} \mathcolor{gray}{\bar{k}}}$ 操作,并丢弃第一项,只考察其\textcolor{Plum}{非局域}部分${\symup{\varepsilon_0}} \left\{ \chi^{\;\! \mathcolor{gray}{\omega} \hat{1} \mathcolor{gray}{\hat{2}}}_{\;\! \symup{\iota} \textcolor{Maroon}{(1)}} E^{\;\!\mathcolor{gray}{\omega}}_{\;\! \mathcolor{gray}{\hat{2}} \hat{1} \mathcolor{gray}{\bar{k}}} + \chi^{\;\! \mathcolor{gray}{\omega} \hat{1} \mathcolor{gray}{\hat{2} \hat{3}}}_{\;\! \symup{\iota} \textcolor{Maroon}{(1)}} E^{\;\!\mathcolor{gray}{\omega}}_{\;\! \mathcolor{gray}{\hat{3} \hat{2}} \hat{1} \mathcolor{gray}{\bar{k}}} + \cdots - \dot{\chi}^{\;\! \mathcolor{gray}{\omega} \mathcolor{gray}{\hat{2}} \hat{1} }_{\;\! \symup{\iota} \mathcolor{gray}{\bar{k}} \textcolor{Maroon}{(1)}} E^{\;\!\mathcolor{gray}{\omega}}_{\;\! \mathcolor{gray}{\hat{2}} \hat{1} \mathcolor{gray}{\bar{k}}} \right\}$,利用 $\chi^{\;\! \mathcolor{gray}{\omega} \hat{1} \mathcolor{gray}{\hat{2}}}_{\;\! \symup{\iota} \textcolor{Maroon}{(1)}}, \chi^{\;\! \mathcolor{gray}{\omega} \hat{1} \mathcolor{gray}{\hat{2} \hat{3}}}_{\;\! \symup{\iota} \textcolor{Maroon}{(1)}} = \chi^{\;\! \mathcolor{gray}{\omega} \mathcolor{gray}{\hat{2}} \hat{1}}_{\;\! \symup{\iota} \textcolor{Maroon}{(1)}}, \chi^{\;\! \mathcolor{gray}{\omega} \mathcolor{gray}{\hat{2}} \hat{1} \mathcolor{gray}{\hat{3}}}_{\;\! \symup{\iota} \textcolor{Maroon}{(1)}}$ 所继承的 \bref{eq:symmetry1},并将 $\dot{\chi}^{\;\! \mathcolor{gray}{\omega} \mathcolor{gray}{\hat{2}} \hat{1} }_{\;\! \symup{\iota} \mathcolor{gray}{\bar{k}} \textcolor{Maroon}{(1)}} E^{\;\!\mathcolor{gray}{\omega}}_{\;\! \mathcolor{gray}{\hat{2}} \hat{1} \mathcolor{gray}{\bar{k}}}$ 展开、合并\textcolor{Plum}{同类项},得 ${\symup{\varepsilon_0}} \left\{ \left( \chi^{\;\! \mathcolor{gray}{\omega} \mathcolor{gray}{\hat{2}} \hat{1}}_{\;\! \symup{\iota} \textcolor{Maroon}{(1)}} - \dot{\chi}^{\;\! \mathcolor{gray}{\omega} \mathcolor{gray}{\hat{2}} \hat{1}}_{\;\! \symup{\iota} \textcolor{Maroon}{(1)}} \right) E^{\;\!\mathcolor{gray}{\omega}}_{\;\! \mathcolor{gray}{\hat{2}} \hat{1} \mathcolor{gray}{\bar{k}}} + \left( \chi^{\;\! \mathcolor{gray}{\omega} \mathcolor{gray}{\hat{2}} \hat{1} \mathcolor{gray}{\hat{3}}}_{\;\! \symup{\iota} \textcolor{Maroon}{(1)}} - \dot{\chi}^{\;\! \mathcolor{gray}{\omega} \mathcolor{gray}{\hat{2}} \hat{1} \mathcolor{gray}{\hat{3}}}_{\;\! \symup{\iota} \textcolor{Maroon}{(1)}} \right) E^{\;\!\mathcolor{gray}{\omega}}_{\;\! \mathcolor{gray}{\hat{3} \hat{2}} \hat{1} \mathcolor{gray}{\bar{k}}} + \cdots \right\}$。从该表达式可见,$P^{\;\! \textcolor{Maroon}{(1)} \mathcolor{gray}{\omega}}_{\;\! \symup{\iota}\mathcolor{gray}{\bar{k}}}$ 的\textcolor{Plum}{非局域}部分,包含了 $Q^{\;\! \textcolor{Maroon}{(1)} \mathcolor{gray}{\omega}}_{\;\! \symup{\iota} \mathcolor{gray}{\hat{1}} \mathcolor{gray}{\bar{k}}}$ 的散度:因二者的每一项,都一一对应地互为\textcolor{Plum}{同类项};并且不仅可提取出公共系数 $E^{\;\!\mathcolor{gray}{\omega}}_{\;\! \mathcolor{gray}{\hat{2}} \hat{1} \mathcolor{gray}{\bar{k}}}, E^{\;\!\mathcolor{gray}{\omega}}_{\;\! \mathcolor{gray}{\hat{3} \hat{2}} \hat{1} \mathcolor{gray}{\bar{k}}}$,剩下的同类因子(如 $\chi^{\;\! \mathcolor{gray}{\omega} \mathcolor{gray}{\hat{2}} \hat{1}}_{\;\! \symup{\iota} \textcolor{Maroon}{(1)}}, \dot{\chi}^{\;\! \mathcolor{gray}{\omega} \mathcolor{gray}{\hat{2}} \hat{1}}_{\;\! \symup{\iota} \textcolor{Maroon}{(1)}}$、$\chi^{\;\! \mathcolor{gray}{\omega} \mathcolor{gray}{\hat{2}} \hat{1} \mathcolor{gray}{\hat{3}}}_{\;\! \symup{\iota} \textcolor{Maroon}{(1)}}, \dot{\chi}^{\;\! \mathcolor{gray}{\omega} \mathcolor{gray}{\hat{2}} \hat{1} \mathcolor{gray}{\hat{3}}}_{\;\! \symup{\iota} \textcolor{Maroon}{(1)}}$)的脚标顺序也是出奇地一致。因此有理由怀疑,同阶\textcolor{Plum}{非局域}的极化率 $\chi^{\;\! \mathcolor{gray}{\omega}}_{\;\! \textcolor{Maroon}{(1)}}, \dot{\chi}^{\;\! \mathcolor{gray}{\omega}}_{\;\! \textcolor{Maroon}{(1)}}$ 之间,还有一些其他的联系\Footnote{但这里既不认为 $\chi^{\;\! \mathcolor{gray}{\omega}}_{\;\! \textcolor{Maroon}{(1)}}, \dot{\chi}^{\;\! \mathcolor{gray}{\omega}}_{\;\! \textcolor{Maroon}{(1)}}$ 中的某一个可完全替代为为另一个,也不认为二者相等。 --- 因为,一方面 $P^{\;\! \textcolor{Maroon}{(1)} \mathcolor{gray}{\omega}}_{\;\! \symup{\iota}\mathcolor{gray}{\bar{k}}}$ 的高阶空间\textcolor{NavyBlue}{色散}项 还会与 \bref{eq:D^(0)=} 中的 $O^{\;\! \textcolor{Maroon}{(1)} \mathcolor{gray}{\omega}}_{\;\! \symup{\iota} \hat{1} \hat{2} \mathcolor{gray}{\bar{k}}}$ 合并\textcolor{Plum}{同类项},另一方面,$Q^{\;\! \textcolor{Maroon}{(1)} \mathcolor{gray}{\omega}}_{\;\! \symup{\iota} \hat{1} \mathcolor{gray}{\bar{k}}}, O^{\;\! \textcolor{Maroon}{(1)} \mathcolor{gray}{\omega}}_{\;\! \symup{\iota} \hat{1} \hat{2} \mathcolor{gray}{\bar{k}}}$ 本身关于场的空间\textcolor{NavyBlue}{色散}项也需要得到独立体现,比如在\textcolor{Maroon}{边界条件} \bref{ssec:EB-boundary,ssec:DH-boundary} 处。},除 \bref{eq:symmetry1} 以外。

类似 \bref{eq:P(2)_wk},电四极矩 $Q^{\;\!\mathcolor{gray}{t}}_{\;\! \symup{\iota} \hat{1} \mathcolor{gray}{z}} \left( \mathcolor{gray}{\nabla^t}, \mathcolor{gray}{\nabla_{\hat{2}}} ; E^{\;\!\mathcolor{gray}{t}}_{\;\! \hat{3}\mathcolor{gray}{z}}, B^{\;\!\mathcolor{gray}{t}}_{\;\! \hat{4}\mathcolor{gray}{z}} \right)$ 在 $\mathcolor{gray}{\bar{\kappa}}$ 域的二阶\textcolor{Plum}{非线性}为
\begin{subequations}
\begin{align}
	Q^{\;\! \textcolor{Maroon}{(2)} \mathcolor{gray}{\omega}}_{\;\! \symup{\iota} \hat{1} \mathcolor{gray}{\bar{k}}} =&~ {\symup{\varepsilon_0}} \dot{\chi}^{\;\! \mathcolor{gray}{\omega} \hat{2} \hat{3}}_{\;\! \symup{\iota} \hat{1} \mathcolor{gray}{\bar{k}} \textcolor{Maroon}{(2)}} \left( E^{\;\!\mathcolor{gray}{\omega}}_{\;\! \hat{2} \mathcolor{gray}{\bar{k}}} ~\mathcolor{gray}{\widetilde \circledast}~ E^{\;\!\mathcolor{gray}{\omega}}_{\;\! \hat{3} \mathcolor{gray}{\bar{k}}} \right) \\ =&~ {\symup{\varepsilon_0}} \left\{ \dot{\chi}^{\;\! \mathcolor{gray}{\omega} \hat{2} \hat{3}}_{\;\! \symup{\iota} \hat{1} \textcolor{Maroon}{(2)}} \left( E^{\;\!\mathcolor{gray}{\omega}}_{\;\! \hat{2} \mathcolor{gray}{\bar{k}}} ~\mathcolor{gray}{\widetilde \circledast}~ E^{\;\!\mathcolor{gray}{\omega}}_{\;\! \hat{3} \mathcolor{gray}{\bar{k}}} \right) + \dot{\chi}^{\;\! \mathcolor{gray}{\omega} \hat{2} \hat{3} \mathcolor{gray}{\hat{4}}}_{\;\! \symup{\iota} \hat{1} \textcolor{Maroon}{(2)}} \left( E^{\;\!\mathcolor{gray}{\omega}}_{\;\! \mathcolor{gray}{\hat{4}} \hat{2} \mathcolor{gray}{\bar{k}}} ~\mathcolor{gray}{\widetilde \circledast}~ E^{\;\!\mathcolor{gray}{\omega}}_{\;\! \hat{3} \mathcolor{gray}{\bar{k}}} \right) + \cdots \right\} \label{eq:Q(2)_wk}
	\\ :=&~ {\symup{\varepsilon_0}} \left\{ \dot{\chi}^{\;\! \mathcolor{gray}{\omega} \hat{2} \hat{3}}_{\;\! \symup{\iota} \hat{1} \textcolor{Maroon}{(2)}} \left( E^{\;\!\mathcolor{gray}{\omega}}_{\;\! \hat{2} \mathcolor{gray}{\bar{k}}} ~\mathcolor{gray}{\widetilde \circledast}~ E^{\;\!\mathcolor{gray}{\omega}}_{\;\! \hat{3} \mathcolor{gray}{\bar{k}}} \right) + \dot{\chi}^{\;\! \mathcolor{gray}{\omega} \hat{2} \hat{3} \mathcolor{gray}{\hat{4}}}_{\;\! \symup{\iota} \hat{1} \textcolor{Maroon}{(2)}} \left[ \left( \mathcolor{gray}{k_{\hat{4}}} E^{\;\!\mathcolor{gray}{\omega}}_{\;\! \hat{2} \mathcolor{gray}{\bar{k}}} \right) ~\mathcolor{gray}{\widetilde \circledast}~ E^{\;\!\mathcolor{gray}{\omega}}_{\;\! \hat{3} \mathcolor{gray}{\bar{k}}} \right] + \cdots \right\}~,
\end{align}
\end{subequations}
它也会贡献到\textcolor{Maroon}{边界条件} \bref{ssec:EB-boundary,ssec:DH-boundary} 处的表面/体源、表面场,以影响体场的\textcolor{Plum}{连续}性。如果继续查看 $Q^{\;\! \textcolor{Maroon}{(2)}}_{\;\! \symup{\iota} \hat{1}}$ 在 $\mathcolor{gray}{\bar{\kappa}}$ 域的散度 $\mathcolor{gray}{k^{\hat{1}}} Q^{\;\! \textcolor{Maroon}{(2)}}_{\;\! \symup{\iota} \mathcolor{gray}{\hat{1}}}$,那么遵循与 \bref{eq:div-Q(1)_wk-b} 相同的方法,却暂无法将 $\mathcolor{gray}{k^{\hat{1}}} Q^{\;\! \textcolor{Maroon}{(2)}}_{\;\! \symup{\iota} \mathcolor{gray}{\hat{1}}}$ 合并进 $P^{\;\! \textcolor{Maroon}{(2)}}_{\;\! \symup{\iota}}$ 的空间\textcolor{NavyBlue}{色散}项,除非\textcolor{Plum}{非线性}极化率对于其连接“\textcolor{Plum}{原因张量}”的对应角标,具有全排列\textcolor{Plum}{置换对称性} \bref{eq:symmetry4}。

在如上展示了 $P^{\;\! \textcolor{Maroon}{(1)} \mathcolor{gray}{\omega}}_{\;\! \symup{\iota}\mathcolor{gray}{\bar{k}}}, P^{\;\! \textcolor{Maroon}{(2)} \mathcolor{gray}{\omega}}_{\;\! \symup{\iota}\mathcolor{gray}{\bar{k}}}$ 以及 $Q^{\;\! \textcolor{Maroon}{(1)} \mathcolor{gray}{\omega}}_{\;\! \symup{\iota} \hat{1} \mathcolor{gray}{\bar{k}}}, Q^{\;\! \textcolor{Maroon}{(2)} \mathcolor{gray}{\omega}}_{\;\! \symup{\iota} \hat{1} \mathcolor{gray}{\bar{k}}}$ 后,展开剩余电磁\textcolor{Plum}{多极}子 $\bar{\bar{\bar{O}}}^{\;\!\mathcolor{gray}{\omega}}_{\;\!\mathcolor{gray}{\bar{k}}};$ $ \bar{M}^{\;\!\mathcolor{gray}{\omega}}_{\;\!\mathcolor{gray}{\bar{k}}},\bar{\bar{N}}^{\;\!\mathcolor{gray}{\omega}}_{\;\!\mathcolor{gray}{\bar{k}}}$ 的对应项的方法是类似的。特别是对于 $M^{\;\! \textcolor{Maroon}{(1)} \mathcolor{gray}{\omega}}_{\;\! \symup{\iota}\mathcolor{gray}{\bar{k}}}, M^{\;\! \textcolor{Maroon}{(2)} \mathcolor{gray}{\omega}}_{\;\! \symup{\iota}\mathcolor{gray}{\bar{k}}}$ 以及 $N^{\;\! \textcolor{Maroon}{(1)} \mathcolor{gray}{\omega}}_{\;\! \symup{\iota} \hat{1} \mathcolor{gray}{\bar{k}}}, N^{\;\! \textcolor{Maroon}{(2)} \mathcolor{gray}{\omega}}_{\;\! \symup{\iota} \hat{1} \mathcolor{gray}{\bar{k}}}$,对二者的展开几乎完全等价于 \bref{eq:P(1)_wk,eq:P(2)_wk,eq:Q(1)_wk,eq:Q(2)_wk},只需将极化率(如 $\chi^{\;\! \textcolor{NavyBlue}{\text{p}} \mathcolor{gray}{\omega} \hat{1} \mathcolor{gray}{\hat{2} \hat{3}}}_{\;\! \symup{\iota} \textcolor{Maroon}{(1)}} := \chi^{\;\! \mathcolor{gray}{\omega} \hat{1} \mathcolor{gray}{\hat{2} \hat{3}}}_{\;\! \symup{\iota} \textcolor{Maroon}{(1)}}, \chi^{\;\! \textcolor{NavyBlue}{\text{q}} \mathcolor{gray}{\omega} \hat{2} \hat{3} \mathcolor{gray}{\hat{4}}}_{\;\! \symup{\iota} \hat{1} \textcolor{Maroon}{(2)}} := \dot{\chi}^{\;\! \mathcolor{gray}{\omega} \hat{2} \hat{3} \mathcolor{gray}{\hat{4}}}_{\;\! \symup{\iota} \hat{1} \textcolor{Maroon}{(2)}}$ 等)改为磁化率(如 $\eta^{\;\! \textcolor{NavyBlue}{\text{m}} \mathcolor{gray}{\omega} \hat{1} \mathcolor{gray}{\hat{2} \hat{3}}}_{\;\! \symup{\iota} \textcolor{Maroon}{(1)}}, \eta^{\;\! \textcolor{NavyBlue}{\text{n}} \mathcolor{gray}{\omega} \hat{2} \hat{3} \mathcolor{gray}{\hat{4}}}_{\;\! \symup{\iota} \hat{1} \textcolor{Maroon}{(2)}}$ 等)即可,作为\textcolor{Plum}{(非)线性}、\textcolor{Plum}{(非)局域}\textcolor{NavyBlue}{驱动源}的\textcolor{NavyBlue}{场}项,仍然全以电场的空间导数展开,因为这样也包含了动态磁场及其时空导数\cite{vandendriesscheInfluenceMagneticFields2014}。

从这开始,将弃用带正上标小实点的 $\dot{\chi}^{\;\! \mathcolor{gray}{\omega} \hat{2} \hat{3} \mathcolor{gray}{\hat{4}}}_{\;\! \symup{\iota} \hat{1} \textcolor{Maroon}{(2)}}, \dot{\eta}^{\;\! \textcolor{NavyBlue}{\text{n}} \mathcolor{gray}{\omega} \hat{2} \mathcolor{gray}{\hat{3} \hat{4}}}_{\;\! \symup{\iota} \hat{1} \textcolor{Maroon}{(1)}}$ 并以更明确的 $\chi^{\;\! \textcolor{NavyBlue}{\text{q}} \mathcolor{gray}{\omega} \hat{2} \hat{3} \mathcolor{gray}{\hat{4}}}_{\;\! \symup{\iota} \hat{1} \textcolor{Maroon}{(2)}}$ 或 $\eta^{\;\! \textcolor{NavyBlue}{\text{n}} \mathcolor{gray}{\omega} \hat{2} \mathcolor{gray}{\hat{3} \hat{4}}}_{\;\! \symup{\iota} \hat{1} \textcolor{Maroon}{(1)}}$ 代替;相应地,\bref{hook:ddot} 中的 $\ddot{\chi}$ 也由以 $\textcolor{NavyBlue}{\text{o}}$ 为指标的 $\chi$ 代替,如 $\ddot{\chi}^{\;\! \mathcolor{gray}{\omega} \hat{3} \mathcolor{gray}{\hat{4}}}_{\;\! \symup{\iota} \hat{1} \hat{2} \textcolor{Maroon}{(1)}}$ $\longrightarrow$ $\chi^{\;\! \textcolor{NavyBlue}{\text{o}} \mathcolor{gray}{\omega} \hat{3} \mathcolor{gray}{\hat{4}}}_{\;\! \symup{\iota} \hat{1} \hat{2} \textcolor{Maroon}{(1)}}$。此外,$\chi^{\;\! \mathcolor{gray}{\omega} \hat{1} \mathcolor{gray}{\hat{2} \hat{3}}}_{\;\! \symup{\iota} \textcolor{Maroon}{(1)}}$ 也更新了定义,不再代表电偶极化率 $\chi^{\;\! \textcolor{NavyBlue}{\text{p}} \mathcolor{gray}{\omega} \hat{1} \mathcolor{gray}{\hat{2} \hat{3}}}_{\;\! \symup{\iota} \textcolor{Maroon}{(1)}}$,而可能是电\textcolor{Plum}{多极}化率(即电电/电磁-\textcolor{Plum}{多极}耦合系数)以及它们的\textcolor{PineGreen}{线性叠加},视指标中的 \textcolor{NavyBlue}{\text{p}},\textcolor{NavyBlue}{\text{q}},\textcolor{NavyBlue}{\text{o}} 而定。同理,$\eta^{\;\! \mathcolor{gray}{\omega} \hat{1} \hat{2} \mathcolor{gray}{\hat{3}}}_{\;\! \symup{\iota} \textcolor{Maroon}{(2)}}$ 也不代表磁偶极化率 $\eta^{\;\! \textcolor{NavyBlue}{\text{m}} \mathcolor{gray}{\omega} \hat{1} \hat{2} \mathcolor{gray}{\hat{3}}}_{\;\! \symup{\iota} \textcolor{Maroon}{(2)}}$,而可能是磁\textcolor{Plum}{多极}化率(即磁磁/磁电-\textcolor{Plum}{多极}耦合系数)以及它们的\textcolor{PineGreen}{线性叠加},视指标中的 \textcolor{NavyBlue}{\text{m}},\textcolor{NavyBlue}{\text{n}} 而定。

\vspace*{-4.0em}

\marginLeft[-2.4em]{ssec:DH-nonlinear}\subsection{$\bar{D},\bar{H}$ 的非局域、非线性电磁响应}\label{ssec:DH-nonlinear}

在 \bref{eq:symmetry4} 条件下,\bref{eq:D-01} 中的 ${\mathbb{1}}_{\mathcolor{gray}{z}} ~\textcolor{Maroon}{\text{项}}$ 所对应的 \bref{eq:D^(0)=} 即 $D^{\;\!\mathcolor{gray}{t}}_{\;\! \symup{\iota}\mathcolor{gray}{z}} = P^{\;\!\mathcolor{gray}{t}}_{\;\! \symup{\iota}\mathcolor{gray}{z}} - \mathcolor{gray}{\nabla^{\hat{1}}} Q^{\;\!\mathcolor{gray}{t}}_{\;\! \symup{\iota} \mathcolor{gray}{\hat{1}} \mathcolor{gray}{z}} + \mathcolor{gray}{\nabla^{\hat{1}}} \mathcolor{gray}{\nabla^{\hat{2}}} O^{\;\!\mathcolor{gray}{t}}_{\;\! \symup{\iota} \mathcolor{gray}{\hat{1}\hat{2}} \mathcolor{gray}{z}} - \cdots$,它的各项可以合并\textcolor{Plum}{同类项},以至于在这里可以仿照 \bref{eq:P_tr} 在 $\mathcolor{gray}{\bar{x}} = \left( \mathcolor{gray}{t}, \mathcolor{gray}{\bar{r}} \right) \asymp \left( \mathcolor{gray}{t}, \mathcolor{gray}{\bar{\rho}}, \mathcolor{gray}{z} \right)$ 域定义 $D^{\;\!\mathcolor{gray}{t}}_{\;\! \symup{\iota}\mathcolor{gray}{z}}$ 关于电磁场的\textcolor{Plum}{非局域}\textcolor{Plum}{非线性}响应
\begin{subequations}
\begin{align}
	D^{\;\!\mathcolor{gray}{t}}_{\;\! \symup{\iota}\mathcolor{gray}{z}} &:= \mathcolor{gray}{\mathcal F^{-\bar{\rho}}_{t}} \left[ D^{\;\! \mathcolor{gray}{\omega}}_{\;\! \symup{\iota}\mathcolor{gray}{z}} \right] = D^{\;\! \textcolor{Maroon}{(1)} \mathcolor{gray}{t}}_{\;\! \symup{\iota}\mathcolor{gray}{z}} + D^{\;\! \textcolor{Maroon}{(2)} \mathcolor{gray}{t}}_{\;\! \symup{\iota}\mathcolor{gray}{z}} + D^{\;\! \textcolor{Maroon}{(3)} \mathcolor{gray}{t}}_{\;\! \symup{\iota}\mathcolor{gray}{z}} + \cdots \\
	&\hphantom{:}= {\symup{\varepsilon_0}} \left\{ \mathcolor{gray}{\underbrace{\nabla_{\check{\jmath}}}} \left[ \chi^{\;\! \mathcolor{gray}{t} \hat{1} \;\! \mathcolor{gray}{\overbrace{\check{\symup{\jmath}}}}}_{\;\! \symup{\iota} \mathcolor{gray}{z} \textcolor{Maroon}{(1)}} ~\mathcolor{gray}{\widetilde *}~ E^{\;\!\mathcolor{gray}{t}}_{\;\! \hat{1} \mathcolor{gray}{z}} \right] + \mathcolor{gray}{\underbrace{\nabla_{\check{\jmath}}}} \left[ \chi^{\;\! \mathcolor{gray}{t} \hat{1} \hat{2} \;\! \mathcolor{gray}{\overbrace{\check{\symup{\jmath}}}}}_{\;\! \symup{\iota} \mathcolor{gray}{z} \textcolor{Maroon}{(2)}} ~\mathcolor{gray}{\widetilde *} \left( E^{\;\!\mathcolor{gray}{t}}_{\;\! \hat{1} \mathcolor{gray}{z}} E^{\;\!\mathcolor{gray}{t}}_{\;\! \hat{2} \mathcolor{gray}{z}} \right) \right] \right. \label{eq:D_trho} \\ &\hphantom{:}+ \left. \mathcolor{gray}{\underbrace{\nabla_{\check{\jmath}}}} \left[ \chi^{\;\! \mathcolor{gray}{t} \hat{1} \hat{2} \hat{3} \;\! \mathcolor{gray}{\overbrace{\check{\symup{\jmath}}}}}_{\;\! \symup{\iota} \mathcolor{gray}{z} \textcolor{Maroon}{(3)}} ~\mathcolor{gray}{\widetilde *} \left( E^{\;\!\mathcolor{gray}{t}}_{\;\! \hat{1} \mathcolor{gray}{z}} E^{\;\!\mathcolor{gray}{t}}_{\;\! \hat{2} \mathcolor{gray}{z}} E^{\;\!\mathcolor{gray}{t}}_{\;\! \hat{3} \mathcolor{gray}{z}} \right) \right] + \cdots \right\}  \\
	&\hphantom{:}= {\symup{\varepsilon_0}} \mathcolor{gray}{\underbrace{\nabla_{\check{\jmath}}}} \left[ \chi^{\;\! \mathcolor{gray}{t} \hat{1} \;\! \mathcolor{gray}{\overbrace{\check{\symup{\jmath}}}}}_{\;\! \symup{\iota} \mathcolor{gray}{z} \textcolor{Maroon}{(1)}} ~\mathcolor{gray}{\widetilde *}~ E^{\;\!\mathcolor{gray}{t}}_{\;\! \hat{1} \mathcolor{gray}{z}} + \chi^{\;\! \mathcolor{gray}{t} \hat{1} \hat{2} \;\! \mathcolor{gray}{\overbrace{\check{\symup{\jmath}}}}}_{\;\! \symup{\iota} \mathcolor{gray}{z} \textcolor{Maroon}{(2)}} ~\mathcolor{gray}{\widetilde *} \left( E^{\;\!\mathcolor{gray}{t}}_{\;\! \hat{1} \mathcolor{gray}{z}} E^{\;\!\mathcolor{gray}{t}}_{\;\! \hat{2} \mathcolor{gray}{z}} \right) \right. \label{eq:D_trho2} \\ &\hphantom{:}+ \left. \chi^{\;\! \mathcolor{gray}{t} \hat{1} \hat{2} \hat{3} \;\! \mathcolor{gray}{\overbrace{\check{\symup{\jmath}}}}}_{\;\! \symup{\iota} \mathcolor{gray}{z} \textcolor{Maroon}{(3)}} ~\mathcolor{gray}{\widetilde *} \left( E^{\;\!\mathcolor{gray}{t}}_{\;\! \hat{1} \mathcolor{gray}{z}} E^{\;\!\mathcolor{gray}{t}}_{\;\! \hat{2} \mathcolor{gray}{z}} E^{\;\!\mathcolor{gray}{t}}_{\;\! \hat{3} \mathcolor{gray}{z}} \right) + \cdots \right]~,
\end{align}
\end{subequations}
该表达式兼顾地吸收了 \textcolor{Maroon}{Volterra 级数}(\textcolor{Plum}{非均匀}时变磁/极化率的 $\mathcolor{gray}{t}$ 域卷积)和\textcolor{Maroon}{多极理论}(场的 $\mathcolor{gray}{\bar{\rho}}$ 域各阶梯度)二者的特点,既是 \bref{eq:P_tr} 与\textcolor{Maroon}{多极理论}的中和,也是 \bref{eq:P(1)_wk,eq:P(2)_wk,eq:Q(1)_wk,eq:Q(2)_wk}(在时域上)与 \textcolor{Maroon}{Volterra 级数}的平均。

对于\textcolor{Plum}{非线性}\textcolor{Maroon}{多极理论} \bref{eq:D_trho} 有很多种定义。以 \bref{eq:D_trho} 中的二阶\textcolor{Plum}{非线性}项 $D^{\;\! \textcolor{Maroon}{(2)} \mathcolor{gray}{t}}_{\;\! \symup{\iota}\mathcolor{gray}{z}}$ 为例,按照\textcolor{NavyBlue}{理论框架}的类别,分出了如下互相独立(暂不分优劣)的 4 种大类;也对每一大类下的 3 种子类,依据所涵盖的广度顺序,进行了排序:
\begin{subequations}
%	\abovedisplayskip=4pt
%	\belowdisplayskip=4pt
	\small
\begin{align}
	\mathcolor{gray}{\underbrace{\nabla_{\check{\jmath}}}} \left[ \chi^{\;\! \mathcolor{gray}{t} \hat{1} \hat{2} \;\! \mathcolor{gray}{\overbrace{\check{\symup{\jmath}}}}}_{\;\! \symup{\iota} \mathcolor{gray}{z} \textcolor{Maroon}{(2)}} ~\mathcolor{gray}{\widetilde \circledast} \left( E^{\;\!\mathcolor{gray}{t}}_{\;\! \hat{1} \mathcolor{gray}{z}} E^{\;\!\mathcolor{gray}{t}}_{\;\! \hat{2} \mathcolor{gray}{z}} \right) \right] &> \chi^{\;\! \mathcolor{gray}{t} \hat{1} \hat{2} \;\! \mathcolor{gray}{\overbrace{\check{\symup{\jmath}}}}}_{\;\! \symup{\iota} \mathcolor{gray}{z} \textcolor{Maroon}{(2)}} ~\mathcolor{gray}{\widetilde \circledast} \left[ \mathcolor{gray}{\underbrace{\nabla_{\check{\jmath}}}} \left( E^{\;\!\mathcolor{gray}{t}}_{\;\! \hat{1} \mathcolor{gray}{z}} E^{\;\!\mathcolor{gray}{t}}_{\;\! \hat{2} \mathcolor{gray}{z}} \right) \right] \hspace{-2em}&&> \chi^{\;\! \mathcolor{gray}{t} \hat{1} \hat{2} \;\! \mathcolor{gray}{\overbrace{\check{\symup{\jmath}}}}}_{\;\! \symup{\iota} \mathcolor{gray}{z} \textcolor{Maroon}{(2)}} ~\mathcolor{gray}{\widetilde \circledast} \left[ \left( \mathcolor{gray}{\underbrace{\nabla_{\check{\jmath}}}} E^{\;\!\mathcolor{gray}{t}}_{\;\! \hat{1} \mathcolor{gray}{z}} \right) E^{\;\!\mathcolor{gray}{t}}_{\;\! \hat{2} \mathcolor{gray}{z}} \right]~, \label{eq:Volterra} \\
	\mathcolor{gray}{\underbrace{\nabla_{\check{\jmath}}}} \left[ \chi^{\;\! \mathcolor{gray}{t} \hat{1} \hat{2} \;\! \mathcolor{gray}{\overbrace{\check{\symup{\jmath}}}}}_{\;\! \symup{\iota} \mathcolor{gray}{z} \textcolor{Maroon}{(2)}} ~\mathcolor{gray}{\widetilde *} \left( E^{\;\!\mathcolor{gray}{t}}_{\;\! \hat{1} \mathcolor{gray}{z}} E^{\;\!\mathcolor{gray}{t}}_{\;\! \hat{2} \mathcolor{gray}{z}} \right) \right] &> \chi^{\;\! \mathcolor{gray}{t} \hat{1} \hat{2} \;\! \mathcolor{gray}{\overbrace{\check{\symup{\jmath}}}}}_{\;\! \symup{\iota} \mathcolor{gray}{z} \textcolor{Maroon}{(2)}} ~\mathcolor{gray}{\widetilde *} \left[ \mathcolor{gray}{\underbrace{\nabla_{\check{\jmath}}}} \left( E^{\;\!\mathcolor{gray}{t}}_{\;\! \hat{1} \mathcolor{gray}{z}} E^{\;\!\mathcolor{gray}{t}}_{\;\! \hat{2} \mathcolor{gray}{z}} \right) \right] \hspace{-2em}&&> \chi^{\;\! \mathcolor{gray}{t} \hat{1} \hat{2} \;\! \mathcolor{gray}{\overbrace{\check{\symup{\jmath}}}}}_{\;\! \symup{\iota} \mathcolor{gray}{z} \textcolor{Maroon}{(2)}} ~\mathcolor{gray}{\widetilde *} \left[ \left( \mathcolor{gray}{\underbrace{\nabla_{\check{\jmath}}}} E^{\;\!\mathcolor{gray}{t}}_{\;\! \hat{1} \mathcolor{gray}{z}} \right) E^{\;\!\mathcolor{gray}{t}}_{\;\! \hat{2} \mathcolor{gray}{z}} \right]~, \label{eq:Volterra-time} \\
	\mathcolor{gray}{\underbrace{\nabla_{\check{\jmath}}}} \left[ \chi^{\;\! \mathcolor{gray}{t} \hat{1} \hat{2} \;\! \mathcolor{gray}{\overbrace{\check{\symup{\jmath}}}}}_{\;\! \symup{\iota} \mathcolor{gray}{z} \textcolor{Maroon}{(2)}} \mathcolor{gray}{*}\! \left( E^{\;\!\mathcolor{gray}{t}}_{\;\! \hat{1} \mathcolor{gray}{z}} E^{\;\!\mathcolor{gray}{t}}_{\;\! \hat{2} \mathcolor{gray}{z}} \right) \right] &> \chi^{\;\! \mathcolor{gray}{t} \hat{1} \hat{2} \;\! \mathcolor{gray}{\overbrace{\check{\symup{\jmath}}}}}_{\;\! \symup{\iota} \mathcolor{gray}{z} \textcolor{Maroon}{(2)}} \mathcolor{gray}{*}\! \left[ \mathcolor{gray}{\underbrace{\nabla_{\check{\jmath}}}} \left( E^{\;\!\mathcolor{gray}{t}}_{\;\! \hat{1} \mathcolor{gray}{z}} E^{\;\!\mathcolor{gray}{t}}_{\;\! \hat{2} \mathcolor{gray}{z}} \right) \right] \hspace{-2em}&&> \chi^{\;\! \mathcolor{gray}{t} \hat{1} \hat{2} \;\! \mathcolor{gray}{\overbrace{\check{\symup{\jmath}}}}}_{\;\! \symup{\iota} \mathcolor{gray}{z} \textcolor{Maroon}{(2)}} \mathcolor{gray}{*}\! \left[ \left( \mathcolor{gray}{\underbrace{\nabla_{\check{\jmath}}}} E^{\;\!\mathcolor{gray}{t}}_{\;\! \hat{1} \mathcolor{gray}{z}} \right) E^{\;\!\mathcolor{gray}{t}}_{\;\! \hat{2} \mathcolor{gray}{z}} \right]~, \label{eq:Volterra-space} \\
	\mathcolor{gray}{\underbrace{\nabla_{\check{\jmath}}}} \left[ \chi^{\;\! \mathcolor{gray}{t} \hat{1} \hat{2} \;\! \mathcolor{gray}{\overbrace{\check{\symup{\jmath}}}}}_{\;\! \symup{\iota} \mathcolor{gray}{z} \textcolor{Maroon}{(2)}} ~\cdot \left( E^{\;\!\mathcolor{gray}{t}}_{\;\! \hat{1} \mathcolor{gray}{z}} E^{\;\!\mathcolor{gray}{t}}_{\;\! \hat{2} \mathcolor{gray}{z}} \right) \right] &> \chi^{\;\! \mathcolor{gray}{t} \hat{1} \hat{2} \;\! \mathcolor{gray}{\overbrace{\check{\symup{\jmath}}}}}_{\;\! \symup{\iota} \mathcolor{gray}{z} \textcolor{Maroon}{(2)}} ~\cdot \left[ \mathcolor{gray}{\underbrace{\nabla_{\check{\jmath}}}} \left( E^{\;\!\mathcolor{gray}{t}}_{\;\! \hat{1} \mathcolor{gray}{z}} E^{\;\!\mathcolor{gray}{t}}_{\;\! \hat{2} \mathcolor{gray}{z}} \right) \right] \hspace{-2em}&&> \chi^{\;\! \mathcolor{gray}{t} \hat{1} \hat{2} \;\! \mathcolor{gray}{\overbrace{\check{\symup{\jmath}}}}}_{\;\! \symup{\iota} \mathcolor{gray}{z} \textcolor{Maroon}{(2)}} ~\cdot \left[ \left( \mathcolor{gray}{\underbrace{\nabla_{\check{\jmath}}}} E^{\;\!\mathcolor{gray}{t}}_{\;\! \hat{1} \mathcolor{gray}{z}} \right) E^{\;\!\mathcolor{gray}{t}}_{\;\! \hat{2} \mathcolor{gray}{z}} \right]~, \label{eq:Volterra-space+time}
\end{align}
\end{subequations}
其中,$\mathcolor{gray}{\widetilde \circledast}, \mathcolor{gray}{\widetilde *}, \mathcolor{gray}{*}, \cdot$ 分别表示\textcolor{Plum}{时空域卷积}、仅\textcolor{Plum}{时域卷积}、仅\textcolor{Plum}{空域卷积}、\textcolor{Plum}{时空域逐场点乘积}。根据\textcolor{NavyBlue}{实证主义}的原则,本文选择了既 {\one} 贴近\textcolor{NavyBlue}{实验现象}、参考其他\textcolor{Plum}{非线性}领域(如\textcolor{Plum}{连续}介质\textcolor{NavyBlue}{力学})的理论,但不偏离\textcolor{Plum}{非线性}\textcolor{NavyBlue}{光学}($\in$ \textcolor{Plum}{连续}介质\textcolor{NavyBlue}{电动力学})的传统定义太多,又 {\two} 在初始有较高的理论高度以在后续可以在相同框架内无缝迁移应用到不同的子领域,同时 {\three} 在理论的极限上还预留有一定的拓展空间的,各方面博弈后的最终妥协结果,即 \bref{eq:Volterra-time} 中最广义的左侧第一种定义。

\bref{eq:D_trho} 继承了 \textcolor{Maroon}{Volterra 级数}在时域上的卷积的定义,使得该理论既 {\one} 可以描述超快光学中材料的非瞬时(\textcolor{Plum}{非线性})响应,又 {\two} 在 $\mathcolor{gray}{\omega}$ 域使 \bref{eq:D_wkrho} 符合\textcolor{Plum}{非线性}\textcolor{NavyBlue}{光学}对极化率系数、\textcolor{gray}{混频}光\textcolor{gray}{能量守恒}等的传统定义,并自动符合内禀\textcolor{Plum}{置换对称性};但 \bref{eq:D_trho} 却又 {\three} 摈弃了\textcolor{Plum}{空域卷积} \bref{eq:Volterra},因为这与\textcolor{Maroon}{光子晶体主方程}\cite{sakodaOpticalPropertiesPhotonic2005,joannopoulosPhotonicCrystalsMolding2008}、\textcolor{Maroon}{散射势理论}\cite{PrinciplesOptics7th,gerkeAperiodicVolumeOptics2010}、\textcolor{Plum}{线性}衍射中的\textcolor{Maroon}{严格耦合模理论}\cite{moharamRigorousCoupledwaveAnalysis1981,ZhuBangTaoGeXiangYiXingZhouQiXingJieGourcwaSuanFaJiBingXingJiSuanJiaSu2016}、\textcolor{Maroon}{准相位匹配理论}\cite{arieQuasiPhaseMatching2007a,zhangUniversalModelingSecondorder2018,chenQuasiphasematchingdivisionMultiplexingHolography2021b,chenLaserNanoprinting3D2023}、\textcolor{Plum}{(非)线性}全息实验现象(及其背后的理论)\cite{zhangNonlinearPhotonicCrystals2021,chenQuasiphasematchingdivisionMultiplexingHolography2021b,chenLaserNanoprinting3D2023,gerkeAperiodicVolumeOptics2010}中的任何一个都不一致:即实验表明,材料系数与场,在\textcolor{gray}{正空间}中必须是\textcolor{Plum}{乘积}、在\textcolor{gray}{倒空间}中以\textcolor{Plum}{卷积}形式存在。 ---  尽管如此,{\four} \bref{eq:D_trho} 中对\textcolor{Maroon}{多极理论}的各阶梯度算子的引入,在\textcolor{Maroon}{多极理论}框架内纳入了光与物质的\textcolor{Plum}{非局域}相互作用,一定程度上弥补了丢掉 \textcolor{Maroon}{Volterra 级数}的\textcolor{Plum}{空域卷积}的损失。此外,{\five} 选择 \bref{eq:Volterra-time} 中的左一,而非左二或左三,出于以下三个原因:它最广义(甚至包含极化率的梯度)、它兼容 $\bar{\bar{Q}}$ 的空域散度 \bref{eq:div-Q(1)_wk}、它自带/不破坏内禀\textcolor{Plum}{置换对称性}(相对于左三)。

其中,在 \bref{hook:hat,hook:check} 中给同为正上标的倒尖括号赋予了与正尖括号相同的含义:将数字\Footnote{如果尖括号下是非数字的变量,则需先将取为某数字,再将其提升为某一随机(离散/\textcolor{Plum}{连续})变量(可跳过),然后再取其值(比如塌缩为某一变量的分量),举例即 $\check{\symup{\jmath}}, \mathcolor{gray}{\check{\jmath}} \xrightarrow[]{\symup{\jmath}, \mathcolor{gray}{\jmath}~ \text{取值}} \check{3}, \mathcolor{gray}{\check{2}} \xrightarrow[]{\text{实例化(可跳过)}} \symup{i}, \mathcolor{gray}{l} \xrightarrow[]{\text{取值}} \symup{z}, \mathcolor{gray}{y}$。}提升为变量,但二者各为不同的实例化后的对象,如 $\check{1} \xrightarrow[]{\text{实例化}} \symup{i} \xrightarrow[]{\text{取值}} \symup{y} \neq \symup{x} \xleftarrow[]{\text{取值}} \symup{j} \xleftarrow[]{\text{实例化}} \hat{1}$。此外,作为$\overbrace{\text{上}}$或$\underbrace{\text{下}}$标的、横躺着的大括号,表示所括对象的同类元素的乘积或组合,如乘积 $\mathcolor{gray}{\underbrace{\nabla_{\check{\jmath}}}} =$ \textcolor{gray}{空} or $\mathcolor{gray}{\nabla_{\check{1}}}$ or $\mathcolor{gray}{\nabla_{\check{1}}} \mathcolor{gray}{\nabla_{\check{2}}}$ or $\mathcolor{gray}{\nabla_{\check{1}}} \mathcolor{gray}{\nabla_{\check{2}}} \mathcolor{gray}{\nabla_{\check{3}}}$ or $\cdots$、组合 $\mathcolor{gray}{\overbrace{\check{\symup{\jmath}}}} =$ \textcolor{gray}{空} or $\mathcolor{gray}{\check{1}}$ or $\mathcolor{gray}{\check{1}} \mathcolor{gray}{\check{2}}$ or $\mathcolor{gray}{\check{1}} \mathcolor{gray}{\check{2}} \mathcolor{gray}{\check{3}}$ or $\cdots$,见 \bref{hook:brace}。配合(双重)\textcolor{Plum}{爱因斯坦求和},这隐含了:遍历所有可能的乘积/组合(然后再乘积/组合内每一个因子/元素)对应的表达式整体,并求和。显而易见,\bref{eq:D_trho2} 具有前述\textcolor{Plum}{置换对称性}即 \bref{eq:symmetry3}。

从这里及以后,$\mathcolor{gray}{\widetilde \circledast}$ 的定义被降维为 $\left( \mathcolor{gray}{t}, \mathcolor{gray}{\bar{\rho}} \right)$ 或 $\left( \mathcolor{gray}{\omega}, \mathcolor{gray}{\bar{k}_{\symup{\rho}}} \right)$ 域的 $1+2$ 维卷积算符;相应成套定义了 $\mathcolor{gray}{\widetilde *}$ 为 $\mathcolor{gray}{t}$ 或 $\mathcolor{gray}{\omega}$ 域的 1 维卷积算子、$\mathcolor{gray}{*}$ 为 $\mathcolor{gray}{\rho}$ 或 $\mathcolor{gray}{\bar{k}_{\symup{\rho}}}$ 域的 2 维卷积算子;仿照 \bref{eq:FT-xkappa},定义了 $1+2$ 维时空域 $\mathcolor{gray}{\bar{\mathbb{R}}_{1+2}}$ 中的\textcolor{Plum}{傅立叶正} $\mathcolor{gray}{\mathcal F_{\bar{k}_{\symup{\rho}}}^{-\omega}}$、\textcolor{Plum}{逆} $\mathcolor{gray}{\mathcal F^{-\bar{\rho}}_{t}}$ \textcolor{Plum}{变换对}
\begin{subequations} \label{eq:FT-trho_wkrho}
	\abovedisplayskip=4pt
	\belowdisplayskip=6pt
\begin{align}
	\mathcolor{gray}{\mathcal F_{\bar{k}_{\symup{\rho}}}^{-\omega}} \left[ \cdot \right] &:= \mathcolor{gray}{\mathcal F_{\omega}^{-1}} \left[ \mathcolor{gray}{\mathcal F} \left[ \cdot \right] \right] = \mathcolor{gray}{\mathcal F} \left[ \mathcolor{gray}{\mathcal F_{\omega}^{-1}} \left[ \cdot \right] \right] ~, \label{eq:FT-wkrho} \\
	\mathcolor{gray}{\mathcal F^{-\bar{\rho}}_{t}} \left[ \cdot \right] &:= \mathcolor{gray}{\mathcal F_{t}} \left[ \mathcolor{gray}{\mathcal F^{-1}} \left[ \cdot \right] \right] = \mathcolor{gray}{\mathcal F^{-1}} \left[ \mathcolor{gray}{\mathcal F_{t}} \left[ \cdot \right] \right] ~. \label{eq:IFT-trho}
\end{align}
\end{subequations}
其中,仿照 \bref{eq:FT-rk},定义了 2 维空域 $\mathcolor{gray}{\bar{\rho}} \in \mathcolor{gray}{\bar{\mathbb{R}}_{\textcolor{Plum}{2}}}$ 中的\textcolor{Plum}{傅立叶正} $\mathcolor{gray}{\mathcal F}$、\textcolor{Plum}{逆} $\mathcolor{gray}{\mathcal F^{-1}}$ \textcolor{Plum}{变换对}
\begin{subequations} \label{eq:FT-rho_krho}
	\belowdisplayskip=8pt
\begin{align}
	\mathcolor{gray}{\mathcal F} \left[ \cdot \right] &:= \frac{ 1 }{ \left( 2\symup{\pi} \right)^2 } \mathcolor{gray}{\iint_{-\infty}^{+\infty}} \cdot~ \mathbb{e}^{-\mathbb{i}\mathcolor{gray}{\bar{k}_{\symup{\rho}}} \cdot \mathcolor{gray}{\bar{\rho}}} \mathbb{d}\mathcolor{gray}{\bar{\rho}} ~, \label{eq:FT-krho} \\
	\mathcolor{gray}{\mathcal F^{-1}} \left[ \cdot \right] &:= \hphantom{\frac{ 1 }{ \left( 2\symup{\pi} \right)^3 }} \mathcolor{gray}{\iint_{-\infty}^{+\infty}} \cdot~ \mathbb{e}^{\mathbb{i}\mathcolor{gray}{\bar{k}_{\symup{\rho}}} \cdot \mathcolor{gray}{\bar{\rho}}} \hphantom{^-} \mathbb{d}\mathcolor{gray}{\bar{k}_{\symup{\rho}}} ~. \label{eq:IFT-rho}
\end{align}
\end{subequations}

对 \bref{eq:D_trho2} 执行 $1+2$ 维\textcolor{Plum}{傅立叶变换} $\mathcolor{gray}{\mathcal F_{\bar{k}_{\symup{\rho}}}^{-\omega}}$ 即 \bref{eq:FT-wkrho},回到 $\left( \mathcolor{gray}{\omega}, \mathcolor{gray}{\bar{k}_{\symup{\rho}}}, \mathcolor{gray}{z} \right)$ 域,得 \bref{eq:P_wk} 对应的 $\Xint{\mathcolor{gray}{-}}{255}{D}^{\;\!\mathcolor{gray}{\omega}}_{\;\! \symup{\iota}\mathcolor{gray}{z}}$\Footnote{对表示\textcolor{Plum}{连续}变量的任何斜体字符,覆盖其一半身位地,从左侧嵌入的一条灰色短线“\textcolor{gray}{--}”,表示该\textcolor{Plum}{因变量}的空间\textcolor{gray}{自变量},含二维实\textcolor{Plum}{横向}\textcolor{gray}{空间频率} $\mathcolor{gray}{\bar{k}_{\symup{\rho}}} \in \mathcolor{gray}{\bar{\mathbb{R}}_{\textcolor{Plum}{2}}}$,见 \bref{hook:Xint}。}关于\textcolor{gray}{单色}电磁场\textcolor{gray}{混频}集的\textcolor{Plum}{非局域}\textcolor{Plum}{非线性}响应级数
\begin{subequations}
	\abovedisplayskip=6pt
	\belowdisplayskip=6pt
\begin{align}
	\Xint{\mathcolor{gray}{-}}{30}{D}^{\;\!\mathcolor{gray}{\omega}}_{\;\! \symup{\iota}\mathcolor{gray}{z}} &:= \mathcolor{gray}{\mathcal F_{\bar{k}_{\symup{\rho}}}^{-\omega}} \left[ D^{\;\! \mathcolor{gray}{t}}_{\;\! \symup{\iota}\mathcolor{gray}{z}} \right] = \Xint{\mathcolor{gray}{-}}{30}{D}^{\;\! \textcolor{Maroon}{(1)} \mathcolor{gray}{\omega}}_{\;\! \symup{\iota}\mathcolor{gray}{z}} + \Xint{\mathcolor{gray}{-}}{30}{D}^{\;\! \textcolor{Maroon}{(2)} \mathcolor{gray}{\omega}}_{\;\! \symup{\iota}\mathcolor{gray}{z}} + \Xint{\mathcolor{gray}{-}}{30}{D}^{\;\! \textcolor{Maroon}{(3)} \mathcolor{gray}{\omega}}_{\;\! \symup{\iota}\mathcolor{gray}{z}} + \cdots \\ &\hphantom{:}= {\symup{\varepsilon_0}} \mathcolor{gray}{\underbrace{\nabla_{\check{\jmath}}}} \left[ \Xint{{}^{}_{\mathcolor{gray}{-}}}{23}{\chi}^{\;\! \mathcolor{gray}{\omega} \hat{1} \;\! \mathcolor{gray}{\overbrace{\check{\symup{\jmath}}}}}_{\;\! \symup{\iota} \mathcolor{gray}{z} \textcolor{Maroon}{(1)}} \mathcolor{gray}{*} \mathcolor{gray}{\mathcal F_{\bar{k}_{\symup{\rho}}}^{-\omega}} \left[ E^{\;\!\mathcolor{gray}{t}}_{\;\! \hat{1} \mathcolor{gray}{z}} \right] + \Xint{{}^{}_{\mathcolor{gray}{-}}}{23}{\chi}^{\;\! \mathcolor{gray}{\omega} \hat{1} \hat{2} \;\! \mathcolor{gray}{\overbrace{\check{\symup{\jmath}}}}}_{\;\! \symup{\iota} \mathcolor{gray}{z} \textcolor{Maroon}{(2)}} \mathcolor{gray}{*} \mathcolor{gray}{\mathcal F_{\bar{k}_{\symup{\rho}}}^{-\omega}} \left[ E^{\;\!\mathcolor{gray}{t}}_{\;\! \hat{1} \mathcolor{gray}{z}} E^{\;\!\mathcolor{gray}{t}}_{\;\! \hat{2} \mathcolor{gray}{z}} \right] \right. \label{eq:D_wkrho} \\ &\hphantom{:}+ \left. \Xint{{}^{}_{\mathcolor{gray}{-}}}{23}{\chi}^{\;\! \mathcolor{gray}{\omega} \hat{1} \hat{2} \hat{3} \;\! \mathcolor{gray}{\overbrace{\check{\symup{\jmath}}}}}_{\;\! \symup{\iota} \mathcolor{gray}{z} \textcolor{Maroon}{(3)}} \mathcolor{gray}{*} \mathcolor{gray}{\mathcal F_{\bar{k}_{\symup{\rho}}}^{-\omega}} \left[ E^{\;\!\mathcolor{gray}{t}}_{\;\! \hat{1} \mathcolor{gray}{z}} E^{\;\!\mathcolor{gray}{t}}_{\;\! \hat{2} \mathcolor{gray}{z}} E^{\;\!\mathcolor{gray}{t}}_{\;\! \hat{3} \mathcolor{gray}{z}} \right] + \cdots \right]
	\\ &\hphantom{:}= {\symup{\varepsilon_0}} \mathcolor{gray}{\underbrace{k^\nabla_{\;\! \check{\symup{\jmath}}}}} \left[ \Xint{{}^{}_{\mathcolor{gray}{-}}}{23}{\chi}^{\;\! \mathcolor{gray}{\omega} \hat{1} \;\! \mathcolor{gray}{\overbrace{\check{\symup{\jmath}}}}}_{\;\! \symup{\iota} \mathcolor{gray}{z} \textcolor{Maroon}{(1)}} \mathcolor{gray}{*} \Xint{\mathcolor{gray}{-}}{295}{E}^{\;\!\mathcolor{gray}{\omega}}_{\;\! \hat{1} \mathcolor{gray}{z}} + \Xint{{}^{}_{\mathcolor{gray}{-}}}{23}{\chi}^{\;\! \mathcolor{gray}{\omega} \hat{1} \hat{2} \;\! \mathcolor{gray}{\overbrace{\check{\symup{\jmath}}}}}_{\;\! \symup{\iota} \mathcolor{gray}{z} \textcolor{Maroon}{(2)}} \mathcolor{gray}{*} \left( \Xint{\mathcolor{gray}{-}}{295}{E}^{\;\!\mathcolor{gray}{\omega}}_{\;\! \hat{1} \mathcolor{gray}{z}} ~\mathcolor{gray}{\widetilde \circledast}~ \Xint{\mathcolor{gray}{-}}{295}{E}^{\;\!\mathcolor{gray}{\omega}}_{\;\! \hat{2} \mathcolor{gray}{z}} \right) \right. \label{eq:D_wknabla} \\ &\hphantom{:}+ \left. \Xint{{}^{}_{\mathcolor{gray}{-}}}{23}{\chi}^{\;\! \mathcolor{gray}{\omega} \hat{1} \hat{2} \hat{3} \;\! \mathcolor{gray}{\overbrace{\check{\symup{\jmath}}}}}_{\;\! \symup{\iota} \mathcolor{gray}{z} \textcolor{Maroon}{(3)}} \mathcolor{gray}{*} \left( \Xint{\mathcolor{gray}{-}}{295}{E}^{\;\!\mathcolor{gray}{\omega}}_{\;\! \hat{1} \mathcolor{gray}{z}} ~\mathcolor{gray}{\widetilde \circledast}~ \Xint{\mathcolor{gray}{-}}{295}{E}^{\;\!\mathcolor{gray}{\omega}}_{\;\! \hat{2} \mathcolor{gray}{z}} ~\mathcolor{gray}{\widetilde \circledast}~ \Xint{\mathcolor{gray}{-}}{295}{E}^{\;\!\mathcolor{gray}{\omega}}_{\;\! \hat{3} \mathcolor{gray}{z}} \right) + \cdots \right]~,
\end{align}
\end{subequations}
在 \bref{eq:D_wknabla} 中,类似 \bref{eq:k_hat_m,eq:k_what_m} 地定义了
\abovedisplayskip=4pt
\belowdisplayskip=6pt
\begin{align} \label{eq:k_check_j^nabla}
	\mathcolor{gray}{k^\nabla_{\;\! \check{\symup{\jmath}}}} = \mathbb{i} \mathcolor{gray}{k_{\symup{x}}} ~\textcolor{Maroon}{\text{或}}~ \mathbb{i} \mathcolor{gray}{k_{\symup{y}}} ~\textcolor{Maroon}{\text{或}}~ \mathcolor{gray}{\nabla_z}~,
\end{align}
值得注意的是,\bref{eq:D_wkrho} 包含了 \bref{eq:P(1)_wk,eq:P(2)_wk,eq:Q(1)_wk,eq:Q(2)_wk} 所涉及的一切,因此其内涵的广度甚至超过了 \bref{eq:P_wk}:所有的\textcolor{NavyBlue}{电(偶)$\longleftarrow$电(和/或)磁(场)}的\textcolor{Plum}{(非)线性}\textcolor{Plum}{(非)局域}作用,均包含在了 \bref{eq:D_wkrho} 中。

此外,尽管 \bref{eq:D_wknabla} 中的场 $\Xint{\mathcolor{gray}{-}}{25}{E}^{\;\! \mathcolor{gray}{\omega}}_{\;\! \mathcolor{gray}{z}}$,以及包含极化率、磁电系数 $\Xint{{}^{}_{\mathcolor{gray}{-}}}{23}{\chi}^{\;\! \textcolor{NavyBlue}{\text{p}} \mathcolor{gray}{\omega}}_{\;\! \mathcolor{gray}{z}}, \Xint{{}^{}_{\mathcolor{gray}{-}}}{23}{\chi}^{\;\! \textcolor{NavyBlue}{\text{q}} \mathcolor{gray}{\omega}}_{\;\! \mathcolor{gray}{z}}, \Xint{{}^{}_{\mathcolor{gray}{-}}}{23}{\chi}^{\;\! \textcolor{NavyBlue}{\text{o}} \mathcolor{gray}{\omega}}_{\;\! \mathcolor{gray}{z}}$的材料常数 $\Xint{{}^{}_{\mathcolor{gray}{-}}}{23}{\chi}^{\;\! \mathcolor{gray}{\omega}}_{\;\! \mathcolor{gray}{z}}$\Footnote{其中 $\Xint{{}^{}_{\mathcolor{gray}{-}}}{23}{\chi}^{\;\! \mathcolor{gray}{\omega}}_{\;\! \mathcolor{gray}{z}}$ 又包含诸如 $\Xint{{}^{}_{\mathcolor{gray}{-}}}{23}{\chi}^{\;\! \mathcolor{gray}{\omega}}_{\;\! \mathcolor{gray}{z} \textcolor{NavyBlue}{\text{p}}}, \Xint{{}^{}_{\mathcolor{gray}{-}}}{23}{\chi}^{\;\! \mathcolor{gray}{\omega}}_{\;\! \mathcolor{gray}{z} \textcolor{NavyBlue}{\text{q}}}, \Xint{{}^{}_{\mathcolor{gray}{-}}}{23}{\chi}^{\;\! \mathcolor{gray}{\omega}}_{\;\! \mathcolor{gray}{z} \textcolor{NavyBlue}{\text{m}}}; \Xint{{}^{}_{\mathcolor{gray}{-}}}{23}{\chi}^{\;\! \mathcolor{gray}{\omega}}_{\;\! \mathcolor{gray}{z} \textcolor{NavyBlue}{\text{pp}}}, \Xint{{}^{}_{\mathcolor{gray}{-}}}{23}{\chi}^{\;\! \mathcolor{gray}{\omega}}_{\;\! \mathcolor{gray}{z} \textcolor{NavyBlue}{\text{pq}}}, \Xint{{}^{}_{\mathcolor{gray}{-}}}{23}{\chi}^{\;\! \mathcolor{gray}{\omega}}_{\;\! \mathcolor{gray}{z} \textcolor{NavyBlue}{\text{pm}}}$ 等。}(已经不再是常数了)因包含一条从左侧嵌入主体中部的灰色短线(见 \bref{hook:Xint}),而全都是 $\mathcolor{gray}{\bar{k}_{\symup{\rho}}}$ 的函数,且运行在 $\left( \mathcolor{gray}{\omega}, \mathcolor{gray}{\bar{k}_{\symup{\rho}}}, \mathcolor{gray}{z} \right)$ 域。在这一点上,\bref{eq:D_wknabla} 相对于 \bref{eq:P(1)_wk,eq:P(2)_wk,eq:Q(1)_wk,eq:Q(2)_wk} 的升级是巨大的:尽管已经将“所有可以分离的”\textcolor{PineGreen}{波矢}分量 $\mathcolor{gray}{k^\nabla_{\;\! \check{\symup{\jmath}}}}$ 全都分离到场中,前者的材料系数却仍然是\textcolor{Plum}{横向}\textcolor{gray}{空间频率} $\mathcolor{gray}{\bar{k}_{\symup{\rho}}}$ 的函数。 --- 对这一点,可以作如下解释:能够从材料系数中剥离的空间\textcolor{NavyBlue}{色散},属于\textcolor{Plum}{多极}子的\textcolor{Plum}{非局域}响应\cite{raabMultipoleTheoryElectromagnetism2004}。而材料系数中剩下的仍与 $\mathcolor{gray}{\bar{k}_{\symup{\rho}}}$ 相关的部分,源自 \bref{eq:D_trho2} 中的各阶 $\chi^{\;\! \mathcolor{gray}{t}}_{\;\! \mathcolor{gray}{z}}$ 在\textcolor{gray}{正空间} $\mathcolor{gray}{\bar{r}} \in \mathcolor{gray}{\bar{\mathbb{R}}_{\textcolor{Plum}{3}}}$ 中的\textcolor{Plum}{非均匀}分布\Footnote{比如被飞秒激光直写\cite{xuFemtosecondLaserWriting2022,weiExperimentalDemonstrationThreedimensional2018,xuThreedimensionalNonlinearPhotonic2018,keren-zurNewDimensionNonlinear2018}、探针极化\cite{maTipinducedNanoscaleDomain2024,maFabrication100nmperiodDomain2023}、电极化、纳米压印、化学刻蚀、机械加工、聚焦离子束切割等技术手段所调制的区域、液晶,以及利用光折变效应在晶体中储存的 3D(三维)全息图\cite{ZhengDaHuaiNiSuanLiJingTiCongQuanXiCunChuDaoSanWeiXianShi2023}等等。}。

类似 \bref{eq:D_trho2},\bref{eq:H-01} 中的 ${\mathbb{1}}_{\mathcolor{gray}{z}} ~\textcolor{Maroon}{\text{项}}$ 所对应的 \bref{eq:H^(0)=} 即 $H^{\;\!\mathcolor{gray}{t}}_{\;\! \symup{\iota}\mathcolor{gray}{z}} = -\hspace{0.2em} M^{\;\!\mathcolor{gray}{t}}_{\;\! \symup{\iota} \mathcolor{gray}{z}} + \mathcolor{gray}{\nabla^{\hat{1}}} N^{\;\!\mathcolor{gray}{t}}_{\;\! \symup{\iota} \mathcolor{gray}{\hat{1}} \mathcolor{gray}{z}} - \cdots$,它的各项在 \bref{eq:symmetry4} 的条件下也可以合并\textcolor{Plum}{同类项},如此便可在 $\mathcolor{gray}{\bar{x}} = \left( \mathcolor{gray}{t}, \mathcolor{gray}{\bar{r}} \right) \asymp \left( \mathcolor{gray}{t}, \mathcolor{gray}{\bar{\rho}}, \mathcolor{gray}{z} \right)$ 域定义 $H^{\;\!\mathcolor{gray}{t}}_{\;\! \symup{\iota}\mathcolor{gray}{z}}$ 关于时变电磁场的\textcolor{Plum}{非局域}\textcolor{Plum}{非线性}响应级数展开式
\begin{subequations}
	\abovedisplayskip=6pt
	\belowdisplayskip=8pt
\begin{align}
	H^{\;\!\mathcolor{gray}{t}}_{\;\! \symup{\iota}\mathcolor{gray}{z}} &:= \mathcolor{gray}{\mathcal F^{-\bar{\rho}}_{t}} \left[ H^{\;\! \mathcolor{gray}{\omega}}_{\;\! \symup{\iota}\mathcolor{gray}{z}} \right] = H^{\;\! \textcolor{Maroon}{(1)} \mathcolor{gray}{t}}_{\;\! \symup{\iota}\mathcolor{gray}{z}} + H^{\;\! \textcolor{Maroon}{(2)} \mathcolor{gray}{t}}_{\;\! \symup{\iota}\mathcolor{gray}{z}} + H^{\;\! \textcolor{Maroon}{(3)} \mathcolor{gray}{t}}_{\;\! \symup{\iota}\mathcolor{gray}{z}} + \cdots \\
	&\hphantom{:}= \sqrt{\frac{\symup{\varepsilon_0}}{\symup{\mu_0}}} \mathcolor{gray}{\underbrace{\nabla_{\check{\jmath}}}} \left[ ~\eta^{\;\! \mathcolor{gray}{t} \hat{1} \;\! \mathcolor{gray}{\overbrace{\check{\symup{\jmath}}}}}_{\;\! \symup{\iota} \mathcolor{gray}{z} \textcolor{Maroon}{(1)}} ~\mathcolor{gray}{\widetilde *}~ E^{\;\!\mathcolor{gray}{t}}_{\;\! \hat{1} \mathcolor{gray}{z}} + ~\eta^{\;\! \mathcolor{gray}{t} \hat{1} \hat{2} \;\! \mathcolor{gray}{\overbrace{\check{\symup{\jmath}}}}}_{\;\! \symup{\iota} \mathcolor{gray}{z} \textcolor{Maroon}{(2)}} ~\mathcolor{gray}{\widetilde *} \left( E^{\;\!\mathcolor{gray}{t}}_{\;\! \hat{1} \mathcolor{gray}{z}} E^{\;\!\mathcolor{gray}{t}}_{\;\! \hat{2} \mathcolor{gray}{z}} \right) \right. \label{eq:H_trho} \\ &\hphantom{:}+ \left. \eta^{\;\! \mathcolor{gray}{t} \hat{1} \hat{2} \hat{3} \;\! \mathcolor{gray}{\overbrace{\check{\symup{\jmath}}}}}_{\;\! \symup{\iota} \mathcolor{gray}{z} \textcolor{Maroon}{(3)}} ~\mathcolor{gray}{\widetilde *} \left( E^{\;\!\mathcolor{gray}{t}}_{\;\! \hat{1} \mathcolor{gray}{z}} E^{\;\!\mathcolor{gray}{t}}_{\;\! \hat{2} \mathcolor{gray}{z}} E^{\;\!\mathcolor{gray}{t}}_{\;\! \hat{3} \mathcolor{gray}{z}} \right) + \cdots \right]~,
\end{align}
\end{subequations}
以及 定义了 \bref{eq:D_wkrho} 所对应的 $\left( \mathcolor{gray}{\omega}, \mathcolor{gray}{\bar{k}_{\symup{\rho}}}, \mathcolor{gray}{z} \right)$ 域中的 $\Xint{\mathcolor{gray}{-}}{23}{H}^{\;\!\mathcolor{gray}{\omega}}_{\;\! \symup{\iota}\mathcolor{gray}{z}}$ 关于\textcolor{gray}{单色}电磁场 $\Xint{\mathcolor{gray}{-}}{25}{E}^{\;\!\mathcolor{gray}{\omega}}_{\;\! \hat{n} \mathcolor{gray}{z}}, \Xint{\mathcolor{gray}{-}}{27}{B}^{\;\!\mathcolor{gray}{\omega}}_{\;\! \hat{m} \mathcolor{gray}{z}}$ \textcolor{gray}{混频}集的\textcolor{Plum}{非局域}\textcolor{Plum}{非线性}响应级数
\begin{subequations}
	\abovedisplayskip=6pt
	\belowdisplayskip=6pt
\begin{align}
	\Xint{\mathcolor{gray}{-}}{275}{H}^{\;\!\mathcolor{gray}{\omega}}_{\;\! \symup{\iota}\mathcolor{gray}{z}} &:= \mathcolor{gray}{\mathcal F_{\bar{k}_{\symup{\rho}}}^{-\omega}} \left[ H^{\;\! \mathcolor{gray}{t}}_{\;\! \symup{\iota}\mathcolor{gray}{z}} \right] = \Xint{\mathcolor{gray}{-}}{275}{H}^{\;\! \textcolor{Maroon}{(1)} \mathcolor{gray}{\omega}}_{\;\! \symup{\iota}\mathcolor{gray}{z}} + \Xint{\mathcolor{gray}{-}}{275}{H}^{\;\! \textcolor{Maroon}{(2)} \mathcolor{gray}{\omega}}_{\;\! \symup{\iota}\mathcolor{gray}{z}} + \Xint{\mathcolor{gray}{-}}{275}{H}^{\;\! \textcolor{Maroon}{(3)} \mathcolor{gray}{\omega}}_{\;\! \symup{\iota}\mathcolor{gray}{z}} + \cdots \\ &\hphantom{:}= \sqrt{\frac{\symup{\varepsilon_0}}{\symup{\mu_0}}} \mathcolor{gray}{\underbrace{\nabla_{\check{\jmath}}}} \left\{ \Xint{\begin{smallmatrix} ~ \\ {}^{}_{\mathcolor{gray}{-}} \\ ~ \end{smallmatrix}}{17}{\eta}^{\;\! \mathcolor{gray}{\omega} \hat{1} \;\! \mathcolor{gray}{\overbrace{\check{\symup{\jmath}}}}}_{\;\! \symup{\iota} \mathcolor{gray}{z} \textcolor{Maroon}{(1)}} \mathcolor{gray}{*} \mathcolor{gray}{\mathcal F_{\bar{k}_{\symup{\rho}}}^{-\omega}} \left[ E^{\;\!\mathcolor{gray}{t}}_{\;\! \hat{1} \mathcolor{gray}{z}} \right] + \Xint{\begin{smallmatrix} ~ \\ {}^{}_{\mathcolor{gray}{-}} \\ ~ \end{smallmatrix}}{17}{\eta}^{\;\! \mathcolor{gray}{\omega} \hat{1} \hat{2} \;\! \mathcolor{gray}{\overbrace{\check{\symup{\jmath}}}}}_{\;\! \symup{\iota} \mathcolor{gray}{z} \textcolor{Maroon}{(2)}} \mathcolor{gray}{*} \mathcolor{gray}{\mathcal F_{\bar{k}_{\symup{\rho}}}^{-\omega}} \left[ E^{\;\!\mathcolor{gray}{t}}_{\;\! \hat{1} \mathcolor{gray}{z}} E^{\;\!\mathcolor{gray}{t}}_{\;\! \hat{2} \mathcolor{gray}{z}} \right] \right. \label{eq:H_wkrho} \\ &\hphantom{:}+ \left. \Xint{\begin{smallmatrix} ~ \\ {}^{}_{\mathcolor{gray}{-}} \\ ~ \end{smallmatrix}}{17}{\eta}^{\;\! \mathcolor{gray}{\omega} \hat{1} \hat{2} \hat{3} \;\! \mathcolor{gray}{\overbrace{\check{\symup{\jmath}}}}}_{\;\! \symup{\iota} \mathcolor{gray}{z} \textcolor{Maroon}{(3)}} \mathcolor{gray}{*} \mathcolor{gray}{\mathcal F_{\bar{k}_{\symup{\rho}}}^{-\omega}} \left[ E^{\;\!\mathcolor{gray}{t}}_{\;\! \hat{1} \mathcolor{gray}{z}} E^{\;\!\mathcolor{gray}{t}}_{\;\! \hat{2} \mathcolor{gray}{z}} E^{\;\!\mathcolor{gray}{t}}_{\;\! \hat{3} \mathcolor{gray}{z}} \right] + \cdots \right\}
	\\ &\hphantom{:}= \sqrt{\frac{\symup{\varepsilon_0}}{\symup{\mu_0}}} \mathcolor{gray}{\underbrace{k^\nabla_{\;\! \check{\symup{\jmath}}}}} \left[ \Xint{\begin{smallmatrix} ~ \\ {}^{}_{\mathcolor{gray}{-}} \\ ~ \end{smallmatrix}}{17}{\eta}^{\;\! \mathcolor{gray}{\omega} \hat{1} \;\! \mathcolor{gray}{\overbrace{\check{\symup{\jmath}}}}}_{\;\! \symup{\iota} \mathcolor{gray}{z} \textcolor{Maroon}{(1)}} \mathcolor{gray}{*} \Xint{\mathcolor{gray}{-}}{295}{E}^{\;\!\mathcolor{gray}{\omega}}_{\;\! \hat{1} \mathcolor{gray}{z}} + \Xint{\begin{smallmatrix} ~ \\ {}^{}_{\mathcolor{gray}{-}} \\ ~ \end{smallmatrix}}{17}{\eta}^{\;\! \mathcolor{gray}{\omega} \hat{1} \hat{2} \;\! \mathcolor{gray}{\overbrace{\check{\symup{\jmath}}}}}_{\;\! \symup{\iota} \mathcolor{gray}{z} \textcolor{Maroon}{(2)}} \mathcolor{gray}{*} \left( \Xint{\mathcolor{gray}{-}}{295}{E}^{\;\!\mathcolor{gray}{\omega}}_{\;\! \hat{1} \mathcolor{gray}{z}} ~\mathcolor{gray}{\widetilde \circledast}~ \Xint{\mathcolor{gray}{-}}{295}{E}^{\;\!\mathcolor{gray}{\omega}}_{\;\! \hat{2} \mathcolor{gray}{z}} \right) \right. \label{eq:H_wknabla} \\ &\hphantom{:}+ \left. \Xint{\begin{smallmatrix} ~ \\ {}^{}_{\mathcolor{gray}{-}} \\ ~ \end{smallmatrix}}{17}{\eta}^{\;\! \mathcolor{gray}{\omega} \hat{1} \hat{2} \hat{3} \;\! \mathcolor{gray}{\overbrace{\check{\symup{\jmath}}}}}_{\;\! \symup{\iota} \mathcolor{gray}{z} \textcolor{Maroon}{(3)}} \mathcolor{gray}{*} \left( \Xint{\mathcolor{gray}{-}}{295}{E}^{\;\!\mathcolor{gray}{\omega}}_{\;\! \hat{1} \mathcolor{gray}{z}} ~\mathcolor{gray}{\widetilde \circledast}~ \Xint{\mathcolor{gray}{-}}{295}{E}^{\;\!\mathcolor{gray}{\omega}}_{\;\! \hat{2} \mathcolor{gray}{z}} ~\mathcolor{gray}{\widetilde \circledast}~ \Xint{\mathcolor{gray}{-}}{295}{E}^{\;\!\mathcolor{gray}{\omega}}_{\;\! \hat{3} \mathcolor{gray}{z}} \right) + \cdots \right]~,
\end{align}
\end{subequations}
注意,尽管 $\bar{H}$ 在 \bref{eq:H_trho,eq:H_wkrho} 中,看上去只和电场 $\bar{E}$ 有关,但通过 \textcolor{Maroon}{法拉第电磁感应定律} \bref{eq:Curl-E},$\bar{H}^{\;\!\mathcolor{gray}{t}}_{\;\! \mathcolor{gray}{z}}$ 也是时变磁感应强度 $\bar{B}^{\;\!\mathcolor{gray}{t}}_{\;\! \mathcolor{gray}{z}}$ 的\textcolor{Plum}{线性}、\textcolor{Plum}{非线性}函数(静磁场 $\bar{B}^{\;\!\mathcolor{gray}{\text{dc}}}_{\;\! \mathcolor{gray}{z}}$ 及其梯度所导致的现象,暂未包含在内)。

值得对比的是,在该 \bref{ssec:DH-nonlinear} 中,尽管 \bref{eq:D_trho2,eq:D_wkrho,eq:H_trho,eq:H_wkrho} 相对于 \bref{eq:P_tr,eq:P_wk} 在 \textcolor{Maroon}{Volterra 级数}的基础上往\textcolor{Maroon}{多极理论}方向进行了拓展,但原来 \bref{eq:P_tr,eq:P_wk} 中的 $4=3+1$ 维\textcolor{Plum}{傅立叶变换}和卷积,到这里却全降为了($2$ 维和)$3=2+1$ 维,以纳入传统空域 $2$($+1$) 维\textcolor{NavyBlue}{傅立叶光学}框架内的同时,降低采样点数目和计算量。 ---  随之而来的妥协是,在 x, y, z 方向上,忽略了材料与场的\textcolor{Plum}{空域卷积},只有场和材料自身(点对点)的\textcolor{Plum}{多极}矩\textcolor{Plum}{非局域}响应。尽管这在很大程度上,对很多场景,都已经适用了。

而本文对\textcolor{Maroon}{多极理论}的拓展方面,采用了 \bref{eq:Volterra-time} 中的左一形式的\textcolor{NavyBlue}{理论框架},这借鉴了\textcolor{Plum}{非局域}\textcolor{Plum}{连续}介质\textcolor{NavyBlue}{力学}理论:有些高阶\textcolor{Plum}{连续}介质\textcolor{NavyBlue}{力学}理论会引入\textcolor{Maroon}{应变梯度}\cite{LiuYingHuaGaoJieLianXuJieZhiLiLun2024},那么类似地在\textcolor{Plum}{非局域}\textcolor{Plum}{连续}介质\textcolor{NavyBlue}{电动力学}中,\textcolor{Plum}{多极}矩中可能会有\textcolor{Maroon}{材料系数的梯度}带来的高阶响应部分,而不仅仅是传统\textcolor{Maroon}{多极理论}所认为的,只有\textcolor{Maroon}{场的梯度}会贡献到\textcolor{Plum}{多极}响应中。因此本文的理论在超材料、超表面、超高频光栅的严格耦合波分析、纳米畴、带衬底的纳米点/线/面阵列等(涉及空域材料系数起伏的)领域的奇特效应的解释上,也许即将扮演不可或缺的角色。

\vspace*{-0.5em}

\marginLeft[-2.4em]{sec:waveq}\section{\textcolor{Maroon}{Wave equation} 波动方程:\textcolor{Maroon}{$f^{-1}(\text{场}) = g(\text{场})$}}\label{sec:waveq}

\vspace*{-6.5em}

\marginLeft[-2.4em]{ssec:Exp-waveq}\subsection{非局域、非均匀、非线性电场 $\bar{E}$ 波动方程}\label{ssec:Exp-waveq}

由于 \bref{eq:D_trho2,eq:H_trho} 中已隐含了 \textcolor{Maroon}{\text{法拉第电磁感应定律}} 即一条旋度方程 \bref{eq:curl-EK},那么只需要第二条旋度方程 \bref{eq:curl-H},即可在 ${\mathbb{1}}_{\mathcolor{gray}{z}} ~\textcolor{Maroon}{\text{项}}$ 所对应的体区域内,求解 \textcolor{Maroon}{Maxwell-Lorentz-Heaviside} 方程组 与 \textcolor{Maroon}{本构关系} \bref{eq:D_trho2,eq:H_trho} 耦合在一起的整体\Footnote{因为两个散度方程是冗余的,通过源\textcolor{Plum}{连续}性 \bref{eq:Div-em-f} 可以证明。}。如果将 \bref{eq:curl-H} 右侧的自由电流源体密度 $J^{\;\!\mathcolor{gray}{t}}_{\;\! \textcolor{Maroon}{\text{f}} \symup{\iota}\mathcolor{gray}{z}}$ 通过各阶电导率张量连接电场的\textcolor{Plum}{(非)线性}函数,合并至 $D^{\;\!\mathcolor{gray}{t}}_{\;\! \symup{\iota}\mathcolor{gray}{z}}$ 的\textcolor{Plum}{非线性}级数展开 \bref{eq:D_trho2} 中,则在分别代入包含了 \bref{eq:D_trho2,eq:H_trho} 的 \bref{eq:D-01,eq:H-01} 之后,\bref{eq:curl-H} 即 $\epsilon^{\hphantom{\symup{\iota}\dot{1}}\dot{2}}_{\symup{\iota}\mathcolor{gray}{\dot{1}}} \mathcolor{gray}{\nabla^{\dot{1}}} H^{\;\!\mathcolor{gray}{t}}_{\;\! \dot{2}\mathcolor{gray}{z}} = \mathcolor{gray}{\nabla^t} D^{\;\!\mathcolor{gray}{t}}_{\;\! \symup{\iota}\mathcolor{gray}{z}}$ 在 $\left( \mathcolor{gray}{t}, \mathcolor{gray}{\bar{\rho}} \right)$ 域上写作
\clearpage
\vspace*{-4.5em}
\begin{subequations}
\begin{align}
	\sqrt{\frac{\symup{\varepsilon_0}}{\symup{\mu_0}}} \epsilon^{\hphantom{\symup{\iota}\dot{1}}\dot{2}}_{\symup{\iota}\mathcolor{gray}{\dot{1}}} \mathcolor{gray}{\nabla^{\dot{1}}} \mathcolor{gray}{\underbrace{\nabla_{\check{\jmath}}}} &\left[~ \varsigma^{\;\! \mathcolor{gray}{t} \hat{1} \;\! \mathcolor{gray}{\overbrace{\check{\symup{\jmath}}}}}_{\;\! \dot{2} \mathcolor{gray}{z} \textcolor{Maroon}{(1)}} ~\mathcolor{gray}{\widetilde *}~ E^{\;\!\mathcolor{gray}{t}}_{\;\! \hat{1} \mathcolor{gray}{z}} + ~\eta^{\;\! \mathcolor{gray}{t} \hat{1} \hat{2} \;\! \mathcolor{gray}{\overbrace{\check{\symup{\jmath}}}}}_{\;\! \dot{2} \mathcolor{gray}{z} \textcolor{Maroon}{(2)}} ~\mathcolor{gray}{\widetilde *} \left( E^{\;\!\mathcolor{gray}{t}}_{\;\! \hat{1} \mathcolor{gray}{z}} E^{\;\!\mathcolor{gray}{t}}_{\;\! \hat{2} \mathcolor{gray}{z}} \right) + \cdots \right] \label{eq:wave_trho} = \\
	{\symup{\varepsilon_0}} \mathcolor{gray}{\nabla^t} \mathcolor{gray}{\underbrace{\nabla_{\check{\jmath}}}} &\left[~ \varepsilon^{\;\! \mathcolor{gray}{t} \hat{1} \;\! \mathcolor{gray}{\overbrace{\check{\symup{\jmath}}}}}_{\;\! \symup{\iota} \mathcolor{gray}{z} \textcolor{Maroon}{(1)}} ~\mathcolor{gray}{\widetilde *}~ E^{\;\!\mathcolor{gray}{t}}_{\;\! \hat{1} \mathcolor{gray}{z}} + \chi^{\;\! \mathcolor{gray}{t} \hat{1} \hat{2} \;\! \mathcolor{gray}{\overbrace{\check{\symup{\jmath}}}}}_{\;\! \symup{\iota} \mathcolor{gray}{z} \textcolor{Maroon}{(2)}} ~\mathcolor{gray}{\widetilde *} \left( E^{\;\!\mathcolor{gray}{t}}_{\;\! \hat{1} \mathcolor{gray}{z}} E^{\;\!\mathcolor{gray}{t}}_{\;\! \hat{2} \mathcolor{gray}{z}} \right) + \cdots \right]~,
\end{align}
\end{subequations}
对 \bref{eq:wave_trho} 两边 $1+2$ 维\textcolor{Plum}{傅立叶变换} $\mathcolor{gray}{\mathcal F_{\bar{k}_{\symup{\rho}}}^{-\omega}}$ 即 \bref{eq:FT-wkrho} 得其 $\left( \mathcolor{gray}{\omega}, \mathcolor{gray}{\bar{k}_{\symup{\rho}}} \right)$ 域版本
\begin{subequations}
\begin{align}
	{\symup{c}} \epsilon^{\hphantom{\symup{\iota}}\mathcolor{gray}{\dot{1}}\dot{2}}_{\symup{\iota}} \mathcolor{gray}{k^\nabla_{\dot{1}}} \mathcolor{gray}{\underbrace{k^\nabla_{\;\! \check{\symup{\jmath}}}}} &\left[ \Xint{\begin{smallmatrix} ~ \\ {}^{}_{\mathcolor{gray}{-}} \\ ~ \end{smallmatrix}}{13}{\varsigma}^{\;\! \mathcolor{gray}{\omega} \hat{1} \;\! \mathcolor{gray}{\overbrace{\check{\symup{\jmath}}}}}_{\;\! \dot{2} \mathcolor{gray}{z} \textcolor{Maroon}{(1)}} \mathcolor{gray}{*} \Xint{\mathcolor{gray}{-}}{295}{E}^{\;\!\mathcolor{gray}{\omega}}_{\;\! \hat{1} \mathcolor{gray}{z}} + \Xint{\begin{smallmatrix} ~ \\ {}^{}_{\mathcolor{gray}{-}} \\ ~ \end{smallmatrix}}{17}{\eta}^{\;\! \mathcolor{gray}{\omega} \hat{1} \hat{2} \;\! \mathcolor{gray}{\overbrace{\check{\symup{\jmath}}}}}_{\;\! \dot{2} \mathcolor{gray}{z} \textcolor{Maroon}{(2)}} \mathcolor{gray}{*} \left( \Xint{\mathcolor{gray}{-}}{295}{E}^{\;\!\mathcolor{gray}{\omega}}_{\;\! \hat{1} \mathcolor{gray}{z}} ~\mathcolor{gray}{\widetilde \circledast}~ \Xint{\mathcolor{gray}{-}}{295}{E}^{\;\!\mathcolor{gray}{\omega}}_{\;\! \hat{2} \mathcolor{gray}{z}} \right) + \cdots \right] \label{eq:wave_wkrho} = \\
	- \mathbb{i} \mathcolor{gray}{\omega} \mathcolor{gray}{\underbrace{k^\nabla_{\;\! \check{\symup{\jmath}}}}} &\left[ \Xint{\begin{smallmatrix} ~ \\ {}^{}_{\mathcolor{gray}{-}} \\ ~ \end{smallmatrix}}{16}{\varepsilon}^{\;\! \mathcolor{gray}{\omega} \hat{1} \;\! \mathcolor{gray}{\overbrace{\check{\symup{\jmath}}}}}_{\;\! \symup{\iota} \mathcolor{gray}{z} \textcolor{Maroon}{(1)}} \mathcolor{gray}{*} \Xint{\mathcolor{gray}{-}}{295}{E}^{\;\!\mathcolor{gray}{\omega}}_{\;\! \hat{1} \mathcolor{gray}{z}} + ~\Xint{{}^{}_{\mathcolor{gray}{-}}}{23}{\chi}^{\;\! \mathcolor{gray}{\omega} \hat{1} \hat{2} \;\! \mathcolor{gray}{\overbrace{\check{\symup{\jmath}}}}}_{\;\! \symup{\iota} \mathcolor{gray}{z} \textcolor{Maroon}{(2)}} \mathcolor{gray}{*} \left( \Xint{\mathcolor{gray}{-}}{295}{E}^{\;\!\mathcolor{gray}{\omega}}_{\;\! \hat{1} \mathcolor{gray}{z}} ~\mathcolor{gray}{\widetilde \circledast}~ \Xint{\mathcolor{gray}{-}}{295}{E}^{\;\!\mathcolor{gray}{\omega}}_{\;\! \hat{2} \mathcolor{gray}{z}} \right) + \cdots \right]~,
\end{align}
\end{subequations}
其中,定义了 $\dot{2}$ 为与 $\hat{2},\check{2}$ 同类 CLASS 的不同实例化对象 OBJECT,见 \bref{hook:Dot}。此外,分别定义了 $\left( \mathcolor{gray}{t}, \mathcolor{gray}{\bar{\rho}} \right)$ 域(左列)和 $\left( \mathcolor{gray}{\omega}, \mathcolor{gray}{\bar{k}_{\symup{\rho}}} \right)$ 域(右列)上的\textcolor{NavyBlue}{电 $\to$ 磁}系数(上行)和\textcolor{NavyBlue}{电 $\to$ 电}(介电)系数(下行)
\begin{subequations}
	\belowdisplayskip=8pt
\begin{align}
	&\varsigma^{\;\! \mathcolor{gray}{t} \hat{1} \mathcolor{gray}{\check{1}}}_{\;\! \dot{2} \mathcolor{gray}{z} \textcolor{Maroon}{(1)}} := \sqrt{\frac{\symup{\mu_0}}{\symup{\varepsilon_0}}} \frac{1}{\symup{\mu_0}} \mathbb{1}^{\;\! \mathcolor{gray}{t}} \epsilon^{\;\! \hphantom{\dot{2}} \mathcolor{gray}{\check{1}} \hat{1}}_{\;\! \dot{2}} + ~\eta^{\;\! \mathcolor{gray}{t} \hat{1} \mathcolor{gray}{\check{1}}}_{\;\! \dot{2} \mathcolor{gray}{z} \textcolor{Maroon}{(1)}}
	&&\Xint{\begin{smallmatrix} ~ \\ {}^{}_{\mathcolor{gray}{-}} \\ ~ \end{smallmatrix}}{13}{\varsigma}^{\;\! \mathcolor{gray}{\omega} \hat{1} \mathcolor{gray}{\check{1}}}_{\;\! \dot{2} \mathcolor{gray}{z} \textcolor{Maroon}{(1)}} := \frac{\symup{c}}{\mathbb{i} \mathcolor{gray}{\omega}} \epsilon^{\;\! \hphantom{\dot{2}} \mathcolor{gray}{\check{1}} \hat{1}}_{\;\! \dot{2}} + \Xint{\begin{smallmatrix} ~ \\ {}^{}_{\mathcolor{gray}{-}} \\ ~ \end{smallmatrix}}{17}{\eta}^{\;\! \mathcolor{gray}{\omega} \hat{1} \mathcolor{gray}{\check{1}}}_{\;\! \dot{2} \mathcolor{gray}{z} \textcolor{Maroon}{(1)}}~, \label{eq:varsigma_rho} \\
	&\varepsilon^{\;\! \mathcolor{gray}{t} \hat{1}}_{\;\! \symup{\iota} \mathcolor{gray}{z} \textcolor{Maroon}{(1)}} := \delta^{\;\! \mathcolor{gray}{t}} \delta^{\;\! \hphantom{\symup{\iota}} \hat{1}}_{\;\! \symup{\iota} \hphantom{z}} + \chi^{\;\! \mathcolor{gray}{t} \hat{1}}_{\;\! \symup{\iota} \mathcolor{gray}{z} \textcolor{Maroon}{(1)}}
	&&\Xint{\begin{smallmatrix} ~ \\ {}^{}_{\mathcolor{gray}{-}} \\ ~ \end{smallmatrix}}{16}{\varepsilon}^{\;\! \mathcolor{gray}{\omega} \hat{1}}_{\;\! \symup{\iota} \mathcolor{gray}{z} \textcolor{Maroon}{(1)}} := \delta^{\;\! \hphantom{\symup{\iota}} \hat{1}}_{\;\! \symup{\iota} \hphantom{z}} + \Xint{{}^{}_{\mathcolor{gray}{-}}}{23}{\chi}^{\;\! \mathcolor{gray}{\omega} \hat{1}}_{\;\! \symup{\iota} \mathcolor{gray}{z} \textcolor{Maroon}{(1)}}~, \label{eq:varepsilon_rho}
\end{align}
\end{subequations}
其中,定义了单位张量 $\bar{\bar{\symup{I}}}$ 的指标形式,即 \textcolor{Plum}{Kronecker} $\delta^{\;\! \hphantom{\symup{\iota}} \hat{1}}_{\;\! \symup{\iota} \hphantom{z}} = 1$ \textcolor{Maroon}{if} $\symup{\iota} = \hat{1}$ \textcolor{Maroon}{else} $= 0$。注意,它应与 \textcolor{Plum}{Dirac} $\delta^{\;\! \mathcolor{gray}{t}} = \infty$ \textcolor{Maroon}{if} $\mathcolor{gray}{t} = 0$ \textcolor{Maroon}{else} $= 0$ 区分开。此外,在\textcolor{gray}{时间}-\textcolor{gray}{角频率} $\mathcolor{gray}{t} - \mathcolor{gray}{\omega}$ 域,\bref{eq:varsigma_rho,eq:varepsilon_rho} 隐含了两对\textcolor{Plum}{傅立叶变换}对\Footnote{有趣的是,从微积分(而不是\textcolor{Plum}{傅立叶变换})的角度,这两对 FT pair 之间(而不是每对 FT pair 内部)还有冥冥的联系:即 \textcolor{Plum}{Dirac delta} 函数与 \textcolor{Plum}{Heaviside step} 函数,构成了像-原函数关系对 $\delta^{\;\! \mathcolor{gray}{t}} \longleftrightarrow \mathbb{1}^{\;\! \mathcolor{gray}{t}}$。}:$\delta^{\;\! \mathcolor{gray}{t}} \longleftrightarrow 1^{\;\! \mathcolor{gray}{\omega}}$ 和 $\mathbb{1}^{\;\! \mathcolor{gray}{t}} \longleftrightarrow \frac{1}{\mathbb{i} \mathcolor{gray}{\omega}}$。

\bref{eq:wave_wkrho} 是关于 $\mathcolor{gray}{\nabla_z}$ 的微分、关于 $\mathcolor{gray}{\omega}, \mathcolor{gray}{\bar{k}_{\symup{\rho}}}$ 的卷积型积分\textcolor{Plum}{非线性}电场波动方程,也是本文关注的焦点,它统治了矢量电磁场/光场在\textcolor{Plum}{连续}介质内部\textcolor{Plum}{连续}演化/分布的动力学行为,是\textcolor{Plum}{线性}和\textcolor{Plum}{非线性}\textcolor{PineGreen}{晶体光学}的“\textcolor{NavyBlue}{第一性原理}”,本文后续内容都追根溯源地来自于之。本小节至节末都在进一步化简该方程。值得注意的是,\textbf{绝大多数专著、文章、教材}中的 $\textcolor{Plum}{2 \times 2}$、$\textcolor{Plum}{3 \times 3}$、$\textcolor{Plum}{4 \times 4}$ 的\textcolor{Plum}{非均匀}\textcolor{Plum}{非线性}\textcolor{Plum}{非局域}电磁场一、二阶偏微分方程\cite{zhangRigorousModelingLaser2015,zhangFullyVectorialSimulation2016,stallingaBerreman4x4Matrix1999,borzdovWavesLinearQuadratic1996,changWavePropagationBianisotropic2014,sturmElectromagneticWavesCrystals2024,sturmElectromagneticWavesCrystals2024,mcleodVectorFourierOptics2014,berryOpticalSingularitiesBianisotropic2005,raabMultipoleTheoryElectromagnetism2004},\textbf{都是 \bref{eq:wave_wkrho} 在特定条件下的简化版}。%,和后续尝试在各种条件下求解的主要对象

对 \bref{eq:wave_wkrho} 合并\textcolor{Plum}{同类项},将\textcolor{Plum}{线性}场源项归至方程左侧,并将所有阶次的\textcolor{Plum}{非线性}\textcolor{NavyBlue}{波源}项移至等号右侧,得 $\left( \mathcolor{gray}{\omega}, \mathcolor{gray}{\bar{k}_{\symup{\rho}}} \right)$ 域\textcolor{NavyBlue}{有源}/\textcolor{Plum}{非齐次}/\textcolor{Plum}{非线性}电场 $\Xint{\mathcolor{gray}{-}}{295}{E}^{\;\!\mathcolor{gray}{\omega}}_{\;\! \hat{1} \mathcolor{gray}{z}}$ 波动方程:
\begin{subequations}
%	\abovedisplayskip=10pt
	\belowdisplayskip=3pt
\begin{align}
	&\mathcolor{gray}{\underbrace{k^\nabla_{\;\! \check{\symup{\jmath}}}}} \left[~ {\symup{c}} \epsilon^{\hphantom{\symup{\iota}}\mathcolor{gray}{\dot{1}}\dot{2}}_{\symup{\iota}} \mathcolor{gray}{k^\nabla_{\dot{1}}} \left( \Xint{\begin{smallmatrix} ~ \\ {}^{}_{\mathcolor{gray}{-}} \\ ~ \end{smallmatrix}}{13}{\varsigma}^{\;\! \mathcolor{gray}{\omega} \hat{1} \;\! \mathcolor{gray}{\overbrace{\check{\symup{\jmath}}}}}_{\;\! \dot{2} \mathcolor{gray}{z} \textcolor{Maroon}{(1)}} \mathcolor{gray}{*} \Xint{\mathcolor{gray}{-}}{295}{E}^{\;\!\mathcolor{gray}{\omega}}_{\;\! \hat{1} \mathcolor{gray}{z}} \right) + \mathbb{i} \mathcolor{gray}{\omega} \Xint{\begin{smallmatrix} ~ \\ {}^{}_{\mathcolor{gray}{-}} \\ ~ \end{smallmatrix}}{16}{\varepsilon}^{\;\! \mathcolor{gray}{\omega} \hat{1} \;\! \mathcolor{gray}{\overbrace{\check{\symup{\jmath}}}}}_{\;\! \symup{\iota} \mathcolor{gray}{z} \textcolor{Maroon}{(1)}} \mathcolor{gray}{*} \Xint{\mathcolor{gray}{-}}{295}{E}^{\;\!\mathcolor{gray}{\omega}}_{\;\! \hat{1} \mathcolor{gray}{z}} \right]  \label{eq:nonlinear-wave_wkrho} \\ = - &\mathcolor{gray}{\underbrace{k^\nabla_{\;\! \check{\symup{\jmath}}}}} \left\{ {\symup{c}} \epsilon^{\hphantom{\symup{\iota}}\mathcolor{gray}{\dot{1}}\dot{2}}_{\symup{\iota}} \mathcolor{gray}{k^\nabla_{\dot{1}}} \left[ \Xint{\begin{smallmatrix} ~ \\ {}^{}_{\mathcolor{gray}{-}} \\ ~ \end{smallmatrix}}{17}{\eta}^{\;\! \mathcolor{gray}{\omega} \hat{1} \hat{2} \;\! \mathcolor{gray}{\overbrace{\check{\symup{\jmath}}}}}_{\;\! \dot{2} \mathcolor{gray}{z} \textcolor{Maroon}{(2)}} \mathcolor{gray}{*} \left( \Xint{\mathcolor{gray}{-}}{295}{E}^{\;\!\mathcolor{gray}{\omega}}_{\;\! \hat{1} \mathcolor{gray}{z}} ~\mathcolor{gray}{\widetilde \circledast}~ \Xint{\mathcolor{gray}{-}}{295}{E}^{\;\!\mathcolor{gray}{\omega}}_{\;\! \hat{2} \mathcolor{gray}{z}} \right) \right] \right. \\ 
	&~~~~~~~~~~~~~~~ +\left. \mathbb{i} \mathcolor{gray}{\omega} ~\Xint{{}^{}_{\mathcolor{gray}{-}}}{23}{\chi}^{\;\! \mathcolor{gray}{\omega} \hat{1} \hat{2} \;\! \mathcolor{gray}{\overbrace{\check{\symup{\jmath}}}}}_{\;\! \symup{\iota} \mathcolor{gray}{z} \textcolor{Maroon}{(2)}} \mathcolor{gray}{*} \left( \Xint{\mathcolor{gray}{-}}{295}{E}^{\;\!\mathcolor{gray}{\omega}}_{\;\! \hat{1} \mathcolor{gray}{z}} ~\mathcolor{gray}{\widetilde \circledast}~ \Xint{\mathcolor{gray}{-}}{295}{E}^{\;\!\mathcolor{gray}{\omega}}_{\;\! \hat{2} \mathcolor{gray}{z}} \right) + \cdots \right\}~,
\end{align}
\end{subequations}
\clearpage
从 \bref{eq:varsigma_rho} 可以看出,在方程左侧的场项中,由于\textcolor{NavyBlue}{磁$\longleftarrow$电}系数 $\Xint{\begin{smallmatrix} ~ \\ {}^{}_{\mathcolor{gray}{-}} \\ ~ \end{smallmatrix}}{13}{\varsigma}^{\;\! \mathcolor{gray}{\omega} \hat{1} \;\! \mathcolor{gray}{\overbrace{\check{\symup{\jmath}}}}}_{\;\! \dot{2} \mathcolor{gray}{z} \textcolor{Maroon}{(1)}}$ 在 $\Xint{\begin{smallmatrix} ~ \\ {}^{}_{\mathcolor{gray}{-}} \\ ~ \end{smallmatrix}}{13}{\varsigma}^{\;\! \mathcolor{gray}{\omega} \hat{1} \mathcolor{gray}{\check{1}}}_{\;\! \dot{2} \mathcolor{gray}{z} \textcolor{Maroon}{(1)}}$ 层次中隐含了 $\epsilon^{\;\! \hphantom{\dot{2}} \hat{1} \mathcolor{gray}{\check{1}}}_{\;\! \dot{2}}$ 算符,而它又是必须计算的\Footnote{对应各向同性\textcolor{NavyBlue}{磁偶$\longleftarrow$磁}响应 ${\symup{\mu}}_0^{-1} B^{\;\!\mathcolor{gray}{t}}_{\;\! \symup{\iota}\mathcolor{gray}{z}}$,见 \bref{eq:H-01}。},因此组合 $\mathcolor{gray}{\overbrace{\check{\symup{\jmath}}}}$ 除了取最基础的 \textcolor{gray}{空} 值之外,还必须取高一阶的 $\mathcolor{gray}{\check{1}}$,并求和。然而,在方程右侧的\textcolor{Plum}{非线性}源项中,无论是 $\Xint{\begin{smallmatrix} ~ \\ {}^{}_{\mathcolor{gray}{-}} \\ ~ \end{smallmatrix}}{17}{\eta}^{\;\! \mathcolor{gray}{\omega} \hat{1} \hat{2} \;\! \mathcolor{gray}{\overbrace{\check{\symup{\jmath}}}}}_{\;\! \dot{2} \mathcolor{gray}{z} \textcolor{Maroon}{(2)}}$ 还是 $\Xint{{}^{}_{\mathcolor{gray}{-}}}{23}{\chi}^{\;\! \mathcolor{gray}{\omega} \hat{1} \hat{2} \;\! \mathcolor{gray}{\overbrace{\check{\symup{\jmath}}}}}_{\;\! \symup{\iota} \mathcolor{gray}{z} \textcolor{Maroon}{(2)}}$,均对组合 $\mathcolor{gray}{\overbrace{\check{\symup{\jmath}}}}$ 没有类似的要求,因此在这里暂时忽略了 $H^{\;\!\mathcolor{gray}{t}}_{\;\! \symup{\iota}\mathcolor{gray}{z}}, D^{\;\!\mathcolor{gray}{t}}_{\;\! \symup{\iota}\mathcolor{gray}{z}}$ 关于场的\textcolor{Plum}{非局域}\textcolor{Plum}{非线性}项\Footnote{包含了\textcolor{NavyBlue}{电、磁偶}极子的梯/旋/散度、\textcolor{NavyBlue}{电、磁四}极子的散度、\textcolor{NavyBlue}{电、磁八}极子的二阶散度等的二阶及以上的\textcolor{Plum}{非线性}项,参考 \bref{eq:Q(2)_wk} 的散度是如何与 \bref{eq:P(2)_wk} 的高一阶\textcolor{PineGreen}{波矢}\textcolor{NavyBlue}{色散}项融合,并最终合并入 \bref{eq:D^(0)=}。}。进一步地,丢掉方程右侧三阶\textcolor{Plum}{非线性}及以上的项,只保留二阶\textcolor{Plum}{局域}\textcolor{Plum}{非线性} \textcolor{NavyBlue}{磁偶-$(\text{电偶}\otimes\text{电偶})$} $\Xint{\begin{smallmatrix} ~ \\ {}^{}_{\mathcolor{gray}{-}} \\ ~ \end{smallmatrix}}{17}{\eta}^{\;\! \mathcolor{gray}{\omega} \hat{1} \hat{2}}_{\;\! \dot{2} \mathcolor{gray}{z} \textcolor{Maroon}{(2)}}$、\textcolor{NavyBlue}{电偶-$(\text{电偶}\otimes\text{电偶})$} $\Xint{{}^{}_{\mathcolor{gray}{-}}}{23}{\chi}^{\;\! \mathcolor{gray}{\omega} \hat{1} \hat{2}}_{\;\! \symup{\iota} \mathcolor{gray}{z} \textcolor{Maroon}{(2)}}$ 极响应,得到较简单的二阶\textcolor{Plum}{非线性}波动方程
\begin{subequations}
	\belowdisplayskip=8pt
\begin{align}
	{\symup{c}} \epsilon^{\hphantom{\symup{\iota}}\mathcolor{gray}{\dot{1}}\dot{2}}_{\symup{\iota}} \mathcolor{gray}{k^\nabla_{\dot{1}}} \left( \Xint{\begin{smallmatrix} ~ \\ {}^{}_{\mathcolor{gray}{-}} \\ ~ \end{smallmatrix}}{13}{\varsigma}^{\;\! \mathcolor{gray}{\omega} \hat{1}}_{\;\! \dot{2} \mathcolor{gray}{z} \textcolor{Maroon}{(1)}} \mathcolor{gray}{*} \Xint{\mathcolor{gray}{-}}{295}{E}^{\;\!\mathcolor{gray}{\omega}}_{\;\! \hat{1} \mathcolor{gray}{z}} \right) &+ \mathbb{i} \mathcolor{gray}{\omega} \Xint{\begin{smallmatrix} ~ \\ {}^{}_{\mathcolor{gray}{-}} \\ ~ \end{smallmatrix}}{16}{\varepsilon}^{\;\! \mathcolor{gray}{\omega} \hat{1}}_{\;\! \symup{\iota} \mathcolor{gray}{z} \textcolor{Maroon}{(1)}} \mathcolor{gray}{*} \Xint{\mathcolor{gray}{-}}{295}{E}^{\;\!\mathcolor{gray}{\omega}}_{\;\! \hat{1} \mathcolor{gray}{z}}  \label{eq:nonlinear(2)-wave_wkrho} \\ 
	+~ \mathcolor{gray}{k^\nabla_{\check{1}}} \left[~ {\symup{c}} \epsilon^{\hphantom{\symup{\iota}}\mathcolor{gray}{\dot{1}}\dot{2}}_{\symup{\iota}} \mathcolor{gray}{k^\nabla_{\dot{1}}} \left( \Xint{\begin{smallmatrix} ~ \\ {}^{}_{\mathcolor{gray}{-}} \\ ~ \end{smallmatrix}}{13}{\varsigma}^{\;\! \mathcolor{gray}{\omega} \hat{1} \mathcolor{gray}{\check{1}}}_{\;\! \dot{2} \mathcolor{gray}{z} \textcolor{Maroon}{(1)}} \mathcolor{gray}{*} \Xint{\mathcolor{gray}{-}}{295}{E}^{\;\!\mathcolor{gray}{\omega}}_{\;\! \hat{1} \mathcolor{gray}{z}} \right) \right.&+ \left.\mathbb{i} \mathcolor{gray}{\omega} \Xint{\begin{smallmatrix} ~ \\ {}^{}_{\mathcolor{gray}{-}} \\ ~ \end{smallmatrix}}{16}{\varepsilon}^{\;\! \mathcolor{gray}{\omega} \hat{1} \mathcolor{gray}{\check{1}}}_{\;\! \symup{\iota} \mathcolor{gray}{z} \textcolor{Maroon}{(1)}} \mathcolor{gray}{*} \Xint{\mathcolor{gray}{-}}{295}{E}^{\;\!\mathcolor{gray}{\omega}}_{\;\! \hat{1} \mathcolor{gray}{z}} ~\right] \\ 
	= -~ {\symup{c}} \epsilon^{\hphantom{\symup{\iota}}\mathcolor{gray}{\dot{1}}\dot{2}}_{\symup{\iota}} \mathcolor{gray}{k^\nabla_{\dot{1}}} \left[ \Xint{\begin{smallmatrix} ~ \\ {}^{}_{\mathcolor{gray}{-}} \\ ~ \end{smallmatrix}}{17}{\eta}^{\;\! \mathcolor{gray}{\omega} \hat{1} \hat{2}}_{\;\! \dot{2} \mathcolor{gray}{z} \textcolor{Maroon}{(2)}} \mathcolor{gray}{*} \left( \Xint{\mathcolor{gray}{-}}{295}{E}^{\;\!\mathcolor{gray}{\omega}}_{\;\! \hat{1} \mathcolor{gray}{z}} ~\mathcolor{gray}{\widetilde \circledast}~ \Xint{\mathcolor{gray}{-}}{295}{E}^{\;\!\mathcolor{gray}{\omega}}_{\;\! \hat{2} \mathcolor{gray}{z}} \right) \right] &- \mathbb{i} \mathcolor{gray}{\omega} ~\Xint{{}^{}_{\mathcolor{gray}{-}}}{23}{\chi}^{\;\! \mathcolor{gray}{\omega} \hat{1} \hat{2}}_{\;\! \symup{\iota} \mathcolor{gray}{z} \textcolor{Maroon}{(2)}} \mathcolor{gray}{*} \left( \Xint{\mathcolor{gray}{-}}{295}{E}^{\;\!\mathcolor{gray}{\omega}}_{\;\! \hat{1} \mathcolor{gray}{z}} ~\mathcolor{gray}{\widetilde \circledast}~ \Xint{\mathcolor{gray}{-}}{295}{E}^{\;\!\mathcolor{gray}{\omega}}_{\;\! \hat{2} \mathcolor{gray}{z}} \right)~,
\end{align}
\end{subequations}
该方程左侧的\textcolor{Plum}{线性}项的\textcolor{NavyBlue}{驱动源}/\textcolor{NavyBlue}{场},处在\textcolor{NavyBlue}{电四-磁偶}极水平,涵盖了所有最低阶的\textcolor{Plum}{非局域}\textcolor{Plum}{线性}响应,包括 \textcolor{NavyBlue}{磁偶-电偶} $\Xint{\begin{smallmatrix} ~ \\ {}^{}_{\mathcolor{gray}{-}} \\ ~ \end{smallmatrix}}{13}{\varsigma}^{\;\! \mathcolor{gray}{\omega} \hat{1}}_{\;\! \dot{2} \mathcolor{gray}{z} \textcolor{Maroon}{(1)}}$、\textcolor{NavyBlue}{磁偶-电四/磁偶} $\Xint{\begin{smallmatrix} ~ \\ {}^{}_{\mathcolor{gray}{-}} \\ ~ \end{smallmatrix}}{13}{\varsigma}^{\;\! \mathcolor{gray}{\omega} \hat{1} \mathcolor{gray}{\check{1}}}_{\;\! \dot{2} \mathcolor{gray}{z} \textcolor{Maroon}{(1)}}$;\textcolor{NavyBlue}{电偶-电偶} $\Xint{\begin{smallmatrix} ~ \\ {}^{}_{\mathcolor{gray}{-}} \\ ~ \end{smallmatrix}}{16}{\varepsilon}^{\;\! \mathcolor{gray}{\omega} \hat{1}}_{\;\! \symup{\iota} \mathcolor{gray}{z} \textcolor{Maroon}{(1)}}$、\textcolor{NavyBlue}{电偶-电四/磁偶} $\Xint{\begin{smallmatrix} ~ \\ {}^{}_{\mathcolor{gray}{-}} \\ ~ \end{smallmatrix}}{16}{\varepsilon}^{\;\! \mathcolor{gray}{\omega} \hat{1} \mathcolor{gray}{\check{1}}}_{\;\! \symup{\iota} \mathcolor{gray}{z} \textcolor{Maroon}{(1)}}$ 极响应。

\bref{eq:nonlinear(2)-wave_wkrho} 左侧的\textcolor{Plum}{线性}部分包括了材料的最低阶但完整的\textcolor{PineGreen}{双各向异性}\cite{langeMultipoleTheoryHehl2015},但右侧的\textcolor{Plum}{非线性}部分却没有(至少需要对等的 4 个材料系数,甚至更多,才能描述相同阶的\textcolor{Plum}{非局域})。为此,不如舍去二阶 \textcolor{NavyBlue}{磁偶-$(\text{电偶}\otimes\text{电偶})$} $\Xint{\begin{smallmatrix} ~ \\ {}^{}_{\mathcolor{gray}{-}} \\ ~ \end{smallmatrix}}{17}{\eta}^{\;\! \mathcolor{gray}{\omega} \hat{1} \hat{2}}_{\;\! \dot{2} \mathcolor{gray}{z} \textcolor{Maroon}{(2)}}$ 极\textcolor{Plum}{局域}\textcolor{Plum}{非线性},只保留最重要的\textcolor{NavyBlue}{电偶-$(\text{电偶}\otimes\text{电偶})$} $\Xint{{}^{}_{\mathcolor{gray}{-}}}{23}{\chi}^{\;\! \mathcolor{gray}{\omega} \hat{1} \hat{2}}_{\;\! \symup{\iota} \mathcolor{gray}{z} \textcolor{Maroon}{(2)}}$ 极响应。此外,\textcolor{NavyBlue}{磁偶-电偶}极系数 $\Xint{\begin{smallmatrix} ~ \\ {}^{}_{\mathcolor{gray}{-}} \\ ~ \end{smallmatrix}}{13}{\varsigma}^{\;\! \mathcolor{gray}{\omega} \hat{1}}_{\;\! \dot{2} \mathcolor{gray}{z} \textcolor{Maroon}{(1)}}$,在与 \textcolor{Plum}{Levi-Civita 张量} $\epsilon^{\hphantom{\symup{\iota}}\mathcolor{gray}{\dot{1}}\dot{2}}_{\symup{\iota}}$ 索引收缩后,对应的\textcolor{NavyBlue}{磁$\longleftarrow$电}系数(注:对称性 \bref{eq:symmetry1} 不适用于该张量;此外,定义了\textcolor{PineGreen}{真空波数} $k_{\textcolor{Maroon}{\mathsf{o}}}^{\;\! \mathcolor{gray}{\omega}} := \frac{\mathcolor{gray}{\omega}}{\symup{c}} = \frac{2 \symup{\pi}}{\mathcolor{gray}{\lambda}}$)
\begin{align} \label{eq:m<-p}
	\Xint{\begin{smallmatrix} ~ \\ {}^{}_{\mathcolor{gray}{-}} \\ ~ \end{smallmatrix}}{12}{\zeta}^{\;\! \textcolor{Plum}{\text{b}} \mathcolor{gray}{\omega} \hat{1} \mathcolor{gray}{\dot{1}}}_{\;\! \symup{\iota} \mathcolor{gray}{z} \textcolor{Maroon}{(1)}} := \frac{\symup{c}}{\mathbb{i} \mathcolor{gray}{\omega}} \Xint{\begin{smallmatrix} ~ \\ {}^{}_{\mathcolor{gray}{-}} \\ ~ \end{smallmatrix}}{13}{\varsigma}^{\;\! \mathcolor{gray}{\omega} \hat{1}}_{\;\! \dot{2} \mathcolor{gray}{z} \textcolor{Maroon}{(1)}} \epsilon^{\hphantom{\symup{\iota}}\mathcolor{gray}{\dot{1}}\dot{2}}_{\symup{\iota}} =: \frac{1}{\mathbb{i} k_{\textcolor{Maroon}{\mathsf{o}}}^{\;\! \mathcolor{gray}{\omega}}} \epsilon^{\hphantom{\symup{\iota}}\mathcolor{gray}{\dot{1}}\dot{2}}_{\symup{\iota}} \Xint{\begin{smallmatrix} ~ \\ {}^{}_{\mathcolor{gray}{-}} \\ ~ \end{smallmatrix}}{13}{\varsigma}^{\;\! \mathcolor{gray}{\omega} \hat{1}}_{\;\! \dot{2} \mathcolor{gray}{z} \textcolor{Maroon}{(1)}} ~,
\end{align}
与\textcolor{NavyBlue}{电$\longleftarrow$磁}系数,即 \textcolor{NavyBlue}{电偶-电四/磁偶}极系数(对称性 \bref{eq:symmetry1} 适用,但未用到)
\begin{align} \label{eq:p<-q/m}
	\Xint{\begin{smallmatrix} ~ \\ {}^{}_{\mathcolor{gray}{-}} \\ ~ \end{smallmatrix}}{12}{\zeta}^{\;\! \textcolor{Plum}{\text{e}} \mathcolor{gray}{\omega} \hat{1} \mathcolor{gray}{\check{1}}}_{\;\! \symup{\iota} \mathcolor{gray}{z} \textcolor{Maroon}{(1)}} := \Xint{\begin{smallmatrix} ~ \\ {}^{}_{\mathcolor{gray}{-}} \\ ~ \end{smallmatrix}}{16}{\varepsilon}^{\;\! \mathcolor{gray}{\omega} \hat{1} \mathcolor{gray}{\check{1}}}_{\;\! \symup{\iota} \mathcolor{gray}{z} \textcolor{Maroon}{(1)}} ~,
\end{align}
一起,可以合并为拥有统一\textcolor{NavyBlue}{磁$\longleftrightarrow$电}(交叉耦合)系数\cite{welterTranslationallyInvariantSemiclassical2013}(\bref{eq:symmetry1} 不适用)
\begin{align} \label{eq:p<->m} %\label 还可以是 m<->p
	\Xint{\begin{smallmatrix} ~ \\ {}^{}_{\mathcolor{gray}{-}} \\ ~ \end{smallmatrix}}{12}{\zeta}^{\;\! \mathcolor{gray}{\omega} \hat{1} \mathcolor{gray}{\check{1}}}_{\;\! \symup{\iota} \mathcolor{gray}{z} \textcolor{Maroon}{(1)}} := \Xint{\begin{smallmatrix} ~ \\ {}^{}_{\mathcolor{gray}{-}} \\ ~ \end{smallmatrix}}{12}{\zeta}^{\;\! \textcolor{Plum}{\text{b}} \mathcolor{gray}{\omega} \hat{1} \mathcolor{gray}{\check{1}}}_{\;\! \symup{\iota} \mathcolor{gray}{z} \textcolor{Maroon}{(1)}} + \Xint{\begin{smallmatrix} ~ \\ {}^{}_{\mathcolor{gray}{-}} \\ ~ \end{smallmatrix}}{12}{\zeta}^{\;\! \textcolor{Plum}{\text{e}} \mathcolor{gray}{\omega} \hat{1} \mathcolor{gray}{\check{1}}}_{\;\! \symup{\iota} \mathcolor{gray}{z} \textcolor{Maroon}{(1)}} ~,
\end{align}
的\textcolor{Plum}{同类项}。在如上修改了\textcolor{Plum}{线性}和\textcolor{Plum}{非线性}部分之后,\bref{eq:nonlinear(2)-wave_wkrho} 除以 $\mathbb{i} \mathcolor{gray}{\omega}$ 后变为
\begin{subequations}
\begin{align}
	\mathcolor{gray}{k^\nabla_{\check{1}}} \left( \Xint{\begin{smallmatrix} ~ \\ {}^{}_{\mathcolor{gray}{-}} \\ ~ \end{smallmatrix}}{12}{\zeta}^{\;\! \mathcolor{gray}{\omega} \hat{1} \mathcolor{gray}{\check{1}}}_{\;\! \symup{\iota} \mathcolor{gray}{z} \textcolor{Maroon}{(1)}} \mathcolor{gray}{*} \Xint{\mathcolor{gray}{-}}{295}{E}^{\;\!\mathcolor{gray}{\omega}}_{\;\! \hat{1} \mathcolor{gray}{z}} \right) &+ \Xint{\begin{smallmatrix} ~ \\ {}^{}_{\mathcolor{gray}{-}} \\ ~ \end{smallmatrix}}{16}{\varepsilon}^{\;\! \mathcolor{gray}{\omega} \hat{1}}_{\;\! \symup{\iota} \mathcolor{gray}{z} \textcolor{Maroon}{(1)}} \mathcolor{gray}{*} \Xint{\mathcolor{gray}{-}}{295}{E}^{\;\!\mathcolor{gray}{\omega}}_{\;\! \hat{1} \mathcolor{gray}{z}}  \label{eq:nonlinear(2)-wave_wkrho-simplify1} \\ 
	+~ \frac{1}{\mathbb{i} k_{\textcolor{Maroon}{\mathsf{o}}}^{\;\! \mathcolor{gray}{\omega}}} \epsilon^{\hphantom{\symup{\iota}}\mathcolor{gray}{\dot{1}}\dot{2}}_{\symup{\iota}} \mathcolor{gray}{k^\nabla_{\dot{1}}} \mathcolor{gray}{k^\nabla_{\check{1}}} \left( \Xint{\begin{smallmatrix} ~ \\ {}^{}_{\mathcolor{gray}{-}} \\ ~ \end{smallmatrix}}{13}{\varsigma}^{\;\! \mathcolor{gray}{\omega} \hat{1} \mathcolor{gray}{\check{1}}}_{\;\! \dot{2} \mathcolor{gray}{z} \textcolor{Maroon}{(1)}} \mathcolor{gray}{*} \Xint{\mathcolor{gray}{-}}{295}{E}^{\;\!\mathcolor{gray}{\omega}}_{\;\! \hat{1} \mathcolor{gray}{z}} \right) &= - ~\Xint{{}^{}_{\mathcolor{gray}{-}}}{23}{\chi}^{\;\! \mathcolor{gray}{\omega} \hat{1} \hat{2}}_{\;\! \symup{\iota} \mathcolor{gray}{z} \textcolor{Maroon}{(2)}} \mathcolor{gray}{*} \left( \Xint{\mathcolor{gray}{-}}{295}{E}^{\;\!\mathcolor{gray}{\omega}}_{\;\! \hat{1} \mathcolor{gray}{z}} ~\mathcolor{gray}{\widetilde \circledast}~ \Xint{\mathcolor{gray}{-}}{295}{E}^{\;\!\mathcolor{gray}{\omega}}_{\;\! \hat{2} \mathcolor{gray}{z}} \right)~, \label{eq:nonlinear(2)-wave_wkrho-simplify1-2}
\end{align}
\end{subequations}
其中,仿照 \bref{eq:p<->m},继续定义 \textcolor{NavyBlue}{磁偶-电四/磁偶} $\epsilon^{\hphantom{\symup{\iota}}\mathcolor{gray}{\dot{1}}\dot{2}}_{\symup{\iota}} \Xint{\begin{smallmatrix} ~ \\ {}^{}_{\mathcolor{gray}{-}} \\ ~ \end{smallmatrix}}{13}{\varsigma}^{\;\! \mathcolor{gray}{\omega} \hat{1} \mathcolor{gray}{\check{1}}}_{\;\! \dot{2} \mathcolor{gray}{z} \textcolor{Maroon}{(1)}}$ 与下一阶 \textcolor{NavyBlue}{电偶-电八/磁四} $\Xint{\begin{smallmatrix} ~ \\ {}^{}_{\mathcolor{gray}{-}} \\ ~ \end{smallmatrix}}{16}{\varepsilon}^{\;\! \mathcolor{gray}{\omega} \hat{1} \mathcolor{gray}{\check{1} \check{2}}}_{\;\! \symup{\iota} \mathcolor{gray}{z} \textcolor{Maroon}{(1)}}$ 合并\textcolor{Plum}{同类项}后的系数(对称性 \bref{eq:symmetry1} 不适用,甚至 $\Xint{\begin{smallmatrix} ~ \\ {}^{}_{\mathcolor{gray}{-}} \\ ~ \end{smallmatrix}}{12}{\zeta}^{\;\! \mathcolor{gray}{\omega} \hat{1} \mathcolor{gray}{\check{1} \check{2}}}_{\;\! \symup{\iota} \mathcolor{gray}{z} \textcolor{Maroon}{(1)}} \neq \Xint{\begin{smallmatrix} ~ \\ {}^{}_{\mathcolor{gray}{-}} \\ ~ \end{smallmatrix}}{12}{\zeta}^{\;\! \mathcolor{gray}{\omega} \hat{1} \mathcolor{gray}{\check{2} \check{1}}}_{\;\! \symup{\iota} \mathcolor{gray}{z} \textcolor{Maroon}{(1)}}$)
\begin{subequations}
	\abovedisplayskip=-8pt
\begin{align} %除了 p<->n,\label 还可以是 m<->m/q
	\Xint{\begin{smallmatrix} ~ \\ {}^{}_{\mathcolor{gray}{-}} \\ ~ \end{smallmatrix}}{12}{\zeta}^{\;\! \mathcolor{gray}{\omega} \hat{1} \mathcolor{gray}{\check{1} \check{2}}}_{\;\! \symup{\iota} \mathcolor{gray}{z} \textcolor{Maroon}{(1)}} &:= \Xint{\begin{smallmatrix} ~ \\ {}^{}_{\mathcolor{gray}{-}} \\ ~ \end{smallmatrix}}{12}{\zeta}^{\;\! \textcolor{Plum}{\text{b}} \mathcolor{gray}{\omega} \hat{1} \mathcolor{gray}{\check{1} \check{2}}}_{\;\! \symup{\iota} \mathcolor{gray}{z} \textcolor{Maroon}{(1)}} + \Xint{\begin{smallmatrix} ~ \\ {}^{}_{\mathcolor{gray}{-}} \\ ~ \end{smallmatrix}}{12}{\zeta}^{\;\! \textcolor{Plum}{\text{e}} \mathcolor{gray}{\omega} \hat{1} \mathcolor{gray}{\check{1} \check{2}}}_{\;\! \symup{\iota} \mathcolor{gray}{z} \textcolor{Maroon}{(1)}} \label{eq:p<->n} \\ &:= \frac{1}{\mathbb{i} k_{\textcolor{Maroon}{\mathsf{o}}}^{\;\! \mathcolor{gray}{\omega}}} \Xint{\begin{smallmatrix} ~ \\ {}^{}_{\mathcolor{gray}{-}} \\ ~ \end{smallmatrix}}{13}{\varsigma}^{\;\! \mathcolor{gray}{\omega} \hat{1} \mathcolor{gray}{\check{1}}}_{\;\! \dot{2} \mathcolor{gray}{z} \textcolor{Maroon}{(1)}} \epsilon^{\hphantom{\symup{\iota}}\mathcolor{gray}{\check{2}}\dot{2}}_{\symup{\iota}} + \Xint{\begin{smallmatrix} ~ \\ {}^{}_{\mathcolor{gray}{-}} \\ ~ \end{smallmatrix}}{16}{\varepsilon}^{\;\! \mathcolor{gray}{\omega} \hat{1} \mathcolor{gray}{\check{1} \check{2}}}_{\;\! \symup{\iota} \mathcolor{gray}{z} \textcolor{Maroon}{(1)}}~,
\end{align}
\end{subequations}
以进一步简化 \bref{eq:nonlinear(2)-wave_wkrho-simplify1-2} 得
\begin{subequations}
\begin{align}
	\mathcolor{gray}{k^\nabla_{\check{1}}} \left( \Xint{\begin{smallmatrix} ~ \\ {}^{}_{\mathcolor{gray}{-}} \\ ~ \end{smallmatrix}}{12}{\zeta}^{\;\! \mathcolor{gray}{\omega} \hat{1} \mathcolor{gray}{\check{1}}}_{\;\! \symup{\iota} \mathcolor{gray}{z} \textcolor{Maroon}{(1)}} \mathcolor{gray}{*} \Xint{\mathcolor{gray}{-}}{295}{E}^{\;\!\mathcolor{gray}{\omega}}_{\;\! \hat{1} \mathcolor{gray}{z}} \right) &+ \Xint{\begin{smallmatrix} ~ \\ {}^{}_{\mathcolor{gray}{-}} \\ ~ \end{smallmatrix}}{16}{\varepsilon}^{\;\! \mathcolor{gray}{\omega} \hat{1}}_{\;\! \symup{\iota} \mathcolor{gray}{z} \textcolor{Maroon}{(1)}} \mathcolor{gray}{*} \Xint{\mathcolor{gray}{-}}{295}{E}^{\;\!\mathcolor{gray}{\omega}}_{\;\! \hat{1} \mathcolor{gray}{z}}  \label{eq:nonlinear(2)-wave_wkrho-simplify2} \\ 
	+~ \mathcolor{gray}{k^\nabla_{\check{2}}} \mathcolor{gray}{k^\nabla_{\check{1}}} \left( \Xint{\begin{smallmatrix} ~ \\ {}^{}_{\mathcolor{gray}{-}} \\ ~ \end{smallmatrix}}{12}{\zeta}^{\;\! \mathcolor{gray}{\omega} \hat{1} \mathcolor{gray}{\check{1} \check{2}}}_{\;\! \symup{\iota} \mathcolor{gray}{z} \textcolor{Maroon}{(1)}} \mathcolor{gray}{*} \Xint{\mathcolor{gray}{-}}{295}{E}^{\;\!\mathcolor{gray}{\omega}}_{\;\! \hat{1} \mathcolor{gray}{z}} \right) &= - ~\Xint{{}^{}_{\mathcolor{gray}{-}}}{23}{\chi}^{\;\! \mathcolor{gray}{\omega} \hat{1} \hat{2}}_{\;\! \symup{\iota} \mathcolor{gray}{z} \textcolor{Maroon}{(2)}} \mathcolor{gray}{*} \left( \Xint{\mathcolor{gray}{-}}{295}{E}^{\;\!\mathcolor{gray}{\omega}}_{\;\! \hat{1} \mathcolor{gray}{z}} ~\mathcolor{gray}{\widetilde \circledast}~ \Xint{\mathcolor{gray}{-}}{295}{E}^{\;\!\mathcolor{gray}{\omega}}_{\;\! \hat{2} \mathcolor{gray}{z}} \right)~, \label{eq:nonlinear(2)-wave_wkrho-simplify2-2}
\end{align}
\end{subequations}
{\one} 该方程左侧在形式上是\textcolor{PineGreen}{双各向异性}材料\cite{berryOpticalSingularitiesBianisotropic2005,changWavePropagationBianisotropic2014}中的光子晶体主方程\cite{sakodaOpticalPropertiesPhotonic2005,joannopoulosPhotonicCrystalsMolding2008},\textcolor{Maroon}{FMM} \textcolor{Maroon}{傅立叶模态法}或光子晶体领域的“\textcolor{Maroon}{Bloch 模式}”暗示方程的解,应为含空\textcolor{NavyBlue}{振幅}和\textcolor{NavyBlue}{相位}因子之积。{\two} 该方程右侧的\textcolor{Plum}{非线性}\textcolor{NavyBlue}{驱动源},从\textcolor{Plum}{非线性}\textcolor{NavyBlue}{光学}的角度,也暗示该方程的解,应为空变(但不一定缓变)\textcolor{NavyBlue}{振幅}包络和(相对)高频振荡\textcolor{NavyBlue}{相位}两部分组成\cite{boydNonlinearOptics2019}。{\three} 从空域中无突变边界的、源\textcolor{Plum}{连续}分布的,\textcolor{Plum}{连续}媒介的角度,与\textcolor{Plum}{连续}源耦合的场解 $\Xint{\mathcolor{gray}{-}}{25}{E}^{\;\!\mathcolor{gray}{\omega}}_{\;\! \hat{1} \mathcolor{gray}{z}}$,在空域上也应是\textcolor{Plum}{连续}函数 --- 那么将$\left( \mathcolor{gray}{\omega}, \mathcolor{gray}{\bar{k}_{\symup{\rho}}}, \mathcolor{gray}{z} \right)$ 域上具有最大\textcolor{Plum}{自由度}(最少限制/最广义)的\textcolor{Plum}{连续}复矢量场函数\Footnote{在未写成分量式之前,矩阵指数 $\mathbb{e}^{\mathbb{i} \Xint{\begin{smallmatrix} ~ \\ {}^{}_{\mathcolor{gray}{-}} \\ ~ \end{smallmatrix}}{15}{\bar{\bar{k}}}_{\textcolor{Maroon}{\symup{z}}}^{\;\! \mathcolor{gray}{\omega}} \mathcolor{gray}{z}}$ 只能左乘列向量 $\Xint{{}^{}_{\mathcolor{gray}{-}}}{10}{\bar{g}}^{\;\!\mathcolor{gray}{\omega}}_{\;\! \mathcolor{gray}{z}}$;写成分量式后,乘积的左右顺序无关紧要。其中,还类似 $\hat{1},\check{1},\dot{1}$ 地,定义了 $\breve{1}$,见 \bref{hook:breve}。},即电场\textcolor{Maroon}{傅立叶基}/\textcolor{Maroon}{谱}/\textcolor{Maroon}{分量}
\begin{subequations} \label{eq:matrix_exp}
\begin{align}
	\Xint{\mathcolor{gray}{-}}{30}{\bar{E}}^{\;\!\mathcolor{gray}{\omega}}_{\;\! \mathcolor{gray}{z}} &:= \mathbb{e}^{\mathbb{i} \Xint{\begin{smallmatrix} ~ \\ {}^{}_{\mathcolor{gray}{-}} \\ ~ \end{smallmatrix}}{15}{\bar{\bar{k}}}_{\textcolor{Maroon}{\symup{z}}}^{\;\! \mathcolor{gray}{\omega}} \mathcolor{gray}{z}} \cdot \Xint{{}^{}_{\mathcolor{gray}{-}}}{10}{\bar{g}}^{\;\!\mathcolor{gray}{\omega}}_{\;\! \mathcolor{gray}{z}} ~, \label{eq:vec-matrix_exp} \\
	\Xint{\mathcolor{gray}{-}}{30}{E}^{\;\!\mathcolor{gray}{\omega}}_{\;\! \hat{1} \mathcolor{gray}{z}} &:= \mathbb{e}^{\mathbb{i} \Xint{\begin{smallmatrix} ~ \\ {}^{}_{\mathcolor{gray}{-}} \\ ~ \end{smallmatrix}}{15}{k}_{\hat{1} \textcolor{Maroon}{\symup{z}}}^{\;\! \mathcolor{gray}{\omega} \breve{1}} \mathcolor{gray}{z}} \Xint{{}^{}_{\mathcolor{gray}{-}}}{10}{g}^{\;\!\mathcolor{gray}{\omega}}_{\;\! \breve{1} \mathcolor{gray}{z}} = \Xint{{}^{}_{\mathcolor{gray}{-}}}{10}{g}^{\;\!\mathcolor{gray}{\omega}}_{\;\! \breve{1} \mathcolor{gray}{z}} \mathbb{e}^{\mathbb{i} \Xint{\begin{smallmatrix} ~ \\ {}^{}_{\mathcolor{gray}{-}} \\ ~ \end{smallmatrix}}{15}{k}_{\hat{1} \textcolor{Maroon}{\symup{z}}}^{\;\! \mathcolor{gray}{\omega} \breve{1}} \mathcolor{gray}{z}} ~, \label{eq:components-matrix_exp}
\end{align}
\end{subequations}
设为 \bref{eq:nonlinear(2)-wave_wkrho} 的解的形式,是合适的\cite{xieAnalytic3DVector}。注意,(试探)解中包含了 2 阶 3 维方阵 $\Xint{\begin{smallmatrix} ~ \\ {}^{}_{\mathcolor{gray}{-}} \\ ~ \end{smallmatrix}}{15}{\bar{\bar{k}}}_{\textcolor{Maroon}{\symup{z}}\textcolor{Plum}{\left[3 \times 3\right]}}^{\;\! \mathcolor{gray}{\omega}}$ 的\textcolor{Maroon}{矩阵指数} $\mathbb{e}^{\mathbb{i} \Xint{\begin{smallmatrix} ~ \\ {}^{}_{\mathcolor{gray}{-}} \\ ~ \end{smallmatrix}}{15}{\bar{\bar{k}}}_{\textcolor{Maroon}{\symup{z}}}^{\;\! \mathcolor{gray}{\omega}} \mathcolor{gray}{z}}$\cite{pessoaAvoidingMatrixExponentials2024,molerNineteenDubiousWays2003},它(在功能上)也(被)称作\textcolor{Maroon}{转移矩阵}。

将试探解写为 \bref{eq:components-matrix_exp} 的形式后,便可将其代入 \bref{eq:nonlinear(2)-wave_wkrho-simplify2}。注意到材料系数与场之间的卷积 $\mathcolor{gray}{*}$ 只在 $\mathcolor{gray}{\bar{k}_{\symup{\rho}}}$ 域发生,因此对于 \bref{eq:k_check_j^nabla} 即算子 $\mathcolor{gray}{k^\nabla_{\check{1}}}$ 中的 $\mathcolor{gray}{\nabla_z}$ 而言,材料系数和场是乘积关系。然而,即便将 $\Xint{\mathcolor{gray}{-}}{25}{E}^{\;\!\mathcolor{gray}{\omega}}_{\;\! \hat{1} \mathcolor{gray}{z}} = \Xint{{}^{}_{\mathcolor{gray}{-}}}{10}{g}^{\;\!\mathcolor{gray}{\omega}}_{\;\! \breve{1} \mathcolor{gray}{z}} \mathbb{e}^{\mathbb{i} \Xint{\begin{smallmatrix} ~ \\ {}^{}_{\mathcolor{gray}{-}} \\ ~ \end{smallmatrix}}{15}{k}_{\hat{1} \textcolor{Maroon}{\symup{z}}}^{\;\! \mathcolor{gray}{\omega} \breve{1}} \mathcolor{gray}{z}}$ 代入方程后,其结构仍然是复杂的。为此,需要作进一步的简化。

观察 \bref{eq:nonlinear(2)-wave_wkrho-simplify2} 可以发现,所有被算符 $\mathcolor{gray}{k^\nabla_{\check{1}}}$ 作用的项,其材料系数最好不是 $\mathcolor{gray}{z}$ 的函数,否则会多出至少一倍数目的项。为此,丢弃 \bref{eq:nonlinear(2)-wave_wkrho-simplify2} 中因材料系数的空间变化所引起的\textcolor{Plum}{非局域}效应,只保留场的空间变化对应的\textcolor{Plum}{非局域}效应,即有
\begin{subequations}
	\abovedisplayskip=-8pt
\begin{align}
	\Xint{\begin{smallmatrix} ~ \\ {}^{}_{\mathcolor{gray}{-}} \\ ~ \end{smallmatrix}}{12}{\zeta}^{\;\! \mathcolor{gray}{\omega} \hat{1} \mathcolor{gray}{\check{1}}}_{\;\! \symup{\iota} \mathcolor{gray}{z} \textcolor{Maroon}{(1)}} \mathcolor{gray}{*} \left( \mathcolor{gray}{k^\nabla_{\check{1}}} \Xint{\mathcolor{gray}{-}}{295}{E}^{\;\!\mathcolor{gray}{\omega}}_{\;\! \hat{1} \mathcolor{gray}{z}} \right) &+ \Xint{\begin{smallmatrix} ~ \\ {}^{}_{\mathcolor{gray}{-}} \\ ~ \end{smallmatrix}}{16}{\varepsilon}^{\;\! \mathcolor{gray}{\omega} \hat{1}}_{\;\! \symup{\iota} \mathcolor{gray}{z} \textcolor{Maroon}{(1)}} \mathcolor{gray}{*} \Xint{\mathcolor{gray}{-}}{295}{E}^{\;\!\mathcolor{gray}{\omega}}_{\;\! \hat{1} \mathcolor{gray}{z}}  \label{eq:nonlinear(2)-wave_wkrho-simplify3} \\ 
	+~ \Xint{\begin{smallmatrix} ~ \\ {}^{}_{\mathcolor{gray}{-}} \\ ~ \end{smallmatrix}}{12}{\zeta}^{\;\! \mathcolor{gray}{\omega} \hat{1} \mathcolor{gray}{\check{1} \check{2}}}_{\;\! \symup{\iota} \mathcolor{gray}{z} \textcolor{Maroon}{(1)}} \mathcolor{gray}{*} \left( \mathcolor{gray}{k^\nabla_{\check{2}}} \mathcolor{gray}{k^\nabla_{\check{1}}} \Xint{\mathcolor{gray}{-}}{295}{E}^{\;\!\mathcolor{gray}{\omega}}_{\;\! \hat{1} \mathcolor{gray}{z}} \right) &= - ~\Xint{{}^{}_{\mathcolor{gray}{-}}}{23}{\chi}^{\;\! \mathcolor{gray}{\omega} \hat{1} \hat{2}}_{\;\! \symup{\iota} \mathcolor{gray}{z} \textcolor{Maroon}{(2)}} \mathcolor{gray}{*} \left( \Xint{\mathcolor{gray}{-}}{295}{E}^{\;\!\mathcolor{gray}{\omega}}_{\;\! \hat{1} \mathcolor{gray}{z}} ~\mathcolor{gray}{\widetilde \circledast}~ \Xint{\mathcolor{gray}{-}}{295}{E}^{\;\!\mathcolor{gray}{\omega}}_{\;\! \hat{2} \mathcolor{gray}{z}} \right)~, \label{eq:nonlinear(2)-wave_wkrho-simplify3-2}
\end{align}
\end{subequations}
当 \bref{eq:k_check_j^nabla} 即算子 $\mathcolor{gray}{k^\nabla_{\check{1}}}$ 中的 $\mathcolor{gray}{\nabla_z}$ 作用于 \bref{eq:nonlinear(2)-wave_wkrho-simplify3} 的试探解 \bref{eq:components-matrix_exp} 即 $
\Xint{\mathcolor{gray}{-}}{25}{E}^{\;\!\mathcolor{gray}{\omega}}_{\;\! \hat{1} \mathcolor{gray}{z}} = \mathbb{e}^{\mathbb{i} \Xint{\begin{smallmatrix} ~ \\ {}^{}_{\mathcolor{gray}{-}} \\ ~ \end{smallmatrix}}{15}{k}_{\hat{1} \textcolor{Maroon}{\symup{z}}}^{\;\! \mathcolor{gray}{\omega} \breve{1}} \mathcolor{gray}{z}} \Xint{{}^{}_{\mathcolor{gray}{-}}}{10}{g}^{\;\!\mathcolor{gray}{\omega}}_{\;\! \breve{1} \mathcolor{gray}{z}}$ 之后,结果为
\begin{align} \label{eq:matrix_exp-dz}
	\mathcolor{gray}{\nabla_z} \Xint{\mathcolor{gray}{-}}{30}{E}^{\;\!\mathcolor{gray}{\omega}}_{\;\! \hat{1} \mathcolor{gray}{z}} = \mathcolor{gray}{\nabla_z} \left( \mathbb{e}^{\mathbb{i} \Xint{\begin{smallmatrix} ~ \\ {}^{}_{\mathcolor{gray}{-}} \\ ~ \end{smallmatrix}}{15}{k}_{\hat{1} \textcolor{Maroon}{\symup{z}}}^{\;\! \mathcolor{gray}{\omega} \breve{1}} \mathcolor{gray}{z}} \Xint{{}^{}_{\mathcolor{gray}{-}}}{10}{g}^{\;\!\mathcolor{gray}{\omega}}_{\;\! \breve{1} \mathcolor{gray}{z}} \right) = \left( \mathbb{i} \Xint{\begin{smallmatrix} ~ \\ {}^{}_{\mathcolor{gray}{-}} \\ ~ \end{smallmatrix}}{15}{k}_{\hat{1} \textcolor{Maroon}{\symup{z}}}^{\;\! \mathcolor{gray}{\omega} \breve{2}} \mathbb{e}^{\mathbb{i} \Xint{\begin{smallmatrix} ~ \\ {}^{}_{\mathcolor{gray}{-}} \\ ~ \end{smallmatrix}}{15}{k}_{\breve{2} \textcolor{Maroon}{\symup{z}}}^{\;\! \mathcolor{gray}{\omega} \breve{1}} \mathcolor{gray}{z}} + \mathbb{e}^{\mathbb{i} \Xint{\begin{smallmatrix} ~ \\ {}^{}_{\mathcolor{gray}{-}} \\ ~ \end{smallmatrix}}{15}{k}_{\hat{1} \textcolor{Maroon}{\symup{z}}}^{\;\! \mathcolor{gray}{\omega} \breve{1}} \mathcolor{gray}{z}} \mathcolor{gray}{\nabla_z} \right) \Xint{{}^{}_{\mathcolor{gray}{-}}}{10}{g}^{\;\!\mathcolor{gray}{\omega}}_{\;\! \breve{1} \mathcolor{gray}{z}} ~,
\end{align}
注意,没法从结果中提取公因式 $\mathbb{e}^{\mathbb{i} \Xint{\begin{smallmatrix} ~ \\ {}^{}_{\mathcolor{gray}{-}} \\ ~ \end{smallmatrix}}{15}{k}_{\breve{2} \textcolor{Maroon}{\symup{z}}}^{\;\! \mathcolor{gray}{\omega} \breve{1}} \mathcolor{gray}{z}} \neq \mathbb{e}^{\mathbb{i} \Xint{\begin{smallmatrix} ~ \\ {}^{}_{\mathcolor{gray}{-}} \\ ~ \end{smallmatrix}}{15}{k}_{\hat{1} \textcolor{Maroon}{\symup{z}}}^{\;\! \mathcolor{gray}{\omega} \breve{1}} \mathcolor{gray}{z}}$。同样,当 \bref{eq:nonlinear(2)-wave_wkrho-simplify3} 中的二阶算子 $\mathcolor{gray}{k^\nabla_{\check{2}}} \mathcolor{gray}{k^\nabla_{\check{1}}}$ 中的 $\mathcolor{gray}{\nabla_z} \mathcolor{gray}{\nabla_z}$ 作用于试探解 \bref{eq:components-matrix_exp} 即 $
\Xint{\mathcolor{gray}{-}}{25}{E}^{\;\!\mathcolor{gray}{\omega}}_{\;\! \hat{1} \mathcolor{gray}{z}} = \mathbb{e}^{\mathbb{i} \Xint{\begin{smallmatrix} ~ \\ {}^{}_{\mathcolor{gray}{-}} \\ ~ \end{smallmatrix}}{15}{k}_{\hat{1} \textcolor{Maroon}{\symup{z}}}^{\;\! \mathcolor{gray}{\omega} \breve{1}} \mathcolor{gray}{z}} \Xint{{}^{}_{\mathcolor{gray}{-}}}{10}{g}^{\;\!\mathcolor{gray}{\omega}}_{\;\! \breve{1} \mathcolor{gray}{z}}$ 时,结果为
\begin{align} \label{eq:matrix_exp-ddz}
	\mathcolor{gray}{\nabla_z^2} \Xint{\mathcolor{gray}{-}}{30}{E}^{\;\!\mathcolor{gray}{\omega}}_{\;\! \hat{1} \mathcolor{gray}{z}} = \left( - \Xint{\begin{smallmatrix} ~ \\ {}^{}_{\mathcolor{gray}{-}} \\ ~ \end{smallmatrix}}{15}{k}_{\hat{1} \textcolor{Maroon}{\symup{z}}}^{\;\! \mathcolor{gray}{\omega} \breve{3}} \Xint{\begin{smallmatrix} ~ \\ {}^{}_{\mathcolor{gray}{-}} \\ ~ \end{smallmatrix}}{15}{k}_{\breve{3} \textcolor{Maroon}{\symup{z}}}^{\;\! \mathcolor{gray}{\omega} \breve{2}} \mathbb{e}^{\mathbb{i} \Xint{\begin{smallmatrix} ~ \\ {}^{}_{\mathcolor{gray}{-}} \\ ~ \end{smallmatrix}}{15}{k}_{\breve{2} \textcolor{Maroon}{\symup{z}}}^{\;\! \mathcolor{gray}{\omega} \breve{1}} \mathcolor{gray}{z}} + 2 \mathbb{i} \Xint{\begin{smallmatrix} ~ \\ {}^{}_{\mathcolor{gray}{-}} \\ ~ \end{smallmatrix}}{15}{k}_{\hat{1} \textcolor{Maroon}{\symup{z}}}^{\;\! \mathcolor{gray}{\omega} \breve{2}} \mathbb{e}^{\mathbb{i} \Xint{\begin{smallmatrix} ~ \\ {}^{}_{\mathcolor{gray}{-}} \\ ~ \end{smallmatrix}}{15}{k}_{\breve{2} \textcolor{Maroon}{\symup{z}}}^{\;\! \mathcolor{gray}{\omega} \breve{1}} \mathcolor{gray}{z}} \mathcolor{gray}{\nabla_z} + \mathbb{e}^{\mathbb{i} \Xint{\begin{smallmatrix} ~ \\ {}^{}_{\mathcolor{gray}{-}} \\ ~ \end{smallmatrix}}{15}{k}_{\hat{1} \textcolor{Maroon}{\symup{z}}}^{\;\! \mathcolor{gray}{\omega} \breve{1}} \mathcolor{gray}{z}} \mathcolor{gray}{\nabla_z^2} \right) \Xint{{}^{}_{\mathcolor{gray}{-}}}{10}{g}^{\;\!\mathcolor{gray}{\omega}}_{\;\! \breve{1} \mathcolor{gray}{z}} ~,
\end{align}
回忆\textcolor{NavyBlue}{有源}/\textcolor{Plum}{非线性}/\textcolor{NavyBlue}{主动}/\textcolor{Plum}{非齐次}波动方程的含空\textcolor{NavyBlue}{变振幅}解,以及\textcolor{NavyBlue}{无源}/\textcolor{Plum}{线性}/\textcolor{NavyBlue}{被动}/\textcolor{Plum}{齐次}波动方程的定\textcolor{NavyBlue}{常振幅}解,将其分别对比上述 \bref{eq:matrix_exp-dz,eq:matrix_exp-ddz} 中含 $\mathcolor{gray}{\nabla_z}$ 的部分 $\mathbb{e}^{\mathbb{i} \Xint{\begin{smallmatrix} ~ \\ {}^{}_{\mathcolor{gray}{-}} \\ ~ \end{smallmatrix}}{15}{k}_{\hat{1} \textcolor{Maroon}{\symup{z}}}^{\;\! \mathcolor{gray}{\omega} \breve{1}} \mathcolor{gray}{z}} \mathcolor{gray}{\nabla_z} \Xint{{}^{}_{\mathcolor{gray}{-}}}{10}{g}^{\;\!\mathcolor{gray}{\omega}}_{\;\! \breve{1} \mathcolor{gray}{z}}$、$\left( 2 \mathbb{i} \Xint{\begin{smallmatrix} ~ \\ {}^{}_{\mathcolor{gray}{-}} \\ ~ \end{smallmatrix}}{15}{k}_{\hat{1} \textcolor{Maroon}{\symup{z}}}^{\;\! \mathcolor{gray}{\omega} \breve{2}} \mathbb{e}^{\mathbb{i} \Xint{\begin{smallmatrix} ~ \\ {}^{}_{\mathcolor{gray}{-}} \\ ~ \end{smallmatrix}}{15}{k}_{\breve{2} \textcolor{Maroon}{\symup{z}}}^{\;\! \mathcolor{gray}{\omega} \breve{1}} \mathcolor{gray}{z}} \mathcolor{gray}{\nabla_z} + \mathbb{e}^{\mathbb{i} \Xint{\begin{smallmatrix} ~ \\ {}^{}_{\mathcolor{gray}{-}} \\ ~ \end{smallmatrix}}{15}{k}_{\hat{1} \textcolor{Maroon}{\symup{z}}}^{\;\! \mathcolor{gray}{\omega} \breve{1}} \mathcolor{gray}{z}} \mathcolor{gray}{\nabla_z^2} \right) \Xint{{}^{}_{\mathcolor{gray}{-}}}{10}{g}^{\;\!\mathcolor{gray}{\omega}}_{\;\! \breve{1} \mathcolor{gray}{z}}$,以及不含 $\mathcolor{gray}{\nabla_z}$ 的部分 $\mathbb{i} \Xint{\begin{smallmatrix} ~ \\ {}^{}_{\mathcolor{gray}{-}} \\ ~ \end{smallmatrix}}{15}{k}_{\hat{1} \textcolor{Maroon}{\symup{z}}}^{\;\! \mathcolor{gray}{\omega} \breve{2}} \Xint{\mathcolor{gray}{-}}{25}{E}^{\;\!\mathcolor{gray}{\omega}}_{\;\! \breve{2} \mathcolor{gray}{z}}$、$ - \Xint{\begin{smallmatrix} ~ \\ {}^{}_{\mathcolor{gray}{-}} \\ ~ \end{smallmatrix}}{15}{k}_{\hat{1} \textcolor{Maroon}{\symup{z}}}^{\;\! \mathcolor{gray}{\omega} \breve{3}} \Xint{\begin{smallmatrix} ~ \\ {}^{}_{\mathcolor{gray}{-}} \\ ~ \end{smallmatrix}}{15}{k}_{\breve{3} \textcolor{Maroon}{\symup{z}}}^{\;\! \mathcolor{gray}{\omega} \breve{2}} \Xint{\mathcolor{gray}{-}}{25}{E}^{\;\!\mathcolor{gray}{\omega}}_{\;\! \breve{2} \mathcolor{gray}{z}}$,这启发我们将 \bref{eq:nonlinear(2)-wave_wkrho-simplify3} 等号左侧中,作用于场的所有算子,分解为\textcolor{Plum}{非线性}算子 $\Xint{\mathcolor{gray}{-}}{21}{\bar{\bar{V}}}^{\;\! \mathcolor{gray}{\omega}}$ 和\textcolor{Plum}{线性}算子 $\Xint{\mathcolor{gray}{-}}{30}{\bar{\bar{L}}}^{\;\! \mathcolor{gray}{\omega}}$ 之和。

然而,\textcolor{Plum}{线性}算子 $\Xint{\mathcolor{gray}{-}}{30}{\bar{\bar{L}}}^{\;\! \mathcolor{gray}{\omega}}$ 要想干净地存在(即可分离),则进一步要求 \bref{eq:nonlinear(2)-wave_wkrho-simplify3} 在等号左侧的所有材料系数 $\Xint{\begin{smallmatrix} ~ \\ {}^{}_{\mathcolor{gray}{-}} \\ ~ \end{smallmatrix}}{12}{\zeta}^{\;\! \mathcolor{gray}{\omega} \hat{1} \mathcolor{gray}{\check{1}}}_{\;\! \symup{\iota} \mathcolor{gray}{z} \textcolor{Maroon}{(1)}}, \Xint{\begin{smallmatrix} ~ \\ {}^{}_{\mathcolor{gray}{-}} \\ ~ \end{smallmatrix}}{16}{\varepsilon}^{\;\! \mathcolor{gray}{\omega} \hat{1}}_{\;\! \symup{\iota} \mathcolor{gray}{z} \textcolor{Maroon}{(1)}}, \Xint{\begin{smallmatrix} ~ \\ {}^{}_{\mathcolor{gray}{-}} \\ ~ \end{smallmatrix}}{12}{\zeta}^{\;\! \mathcolor{gray}{\omega} \hat{1} \mathcolor{gray}{\check{1} \check{2}}}_{\;\! \symup{\iota} \mathcolor{gray}{z} \textcolor{Maroon}{(1)}}$ 的 $\mathcolor{gray}{\bar{r}}$ 域对应物 $\zeta^{\;\! \mathcolor{gray}{t} \hat{1} \mathcolor{gray}{\check{1}}}_{\;\! \symup{\iota} \mathcolor{gray}{z} \textcolor{Maroon}{(1)}}, \varepsilon^{\;\! \mathcolor{gray}{t} \hat{1}}_{\;\! \symup{\iota} \mathcolor{gray}{z} \textcolor{Maroon}{(1)}}, \zeta^{\;\! \mathcolor{gray}{t} \hat{1} \mathcolor{gray}{\check{1} \check{2}}}_{\;\! \symup{\iota} \mathcolor{gray}{z} \textcolor{Maroon}{(1)}}$ 不含空 $\mathcolor{gray}{\bar{r}}$、为常数;为此,这里选择采取($\mathcolor{gray}{\bar{r}}$ 域上的)\textcolor{Plum}{线性}材料系数弱调制条件
\begin{subequations} \label{eq:weak_modulated_linear_sus}
\begin{align}
	\varepsilon^{\;\! \mathcolor{gray}{t} \hat{1}}_{\;\! \symup{\iota} \mathcolor{gray}{z} \textcolor{Maroon}{(1)}} = \varepsilon^{\;\! \mathcolor{gray}{t} \hat{1}}_{\;\! \symup{\iota} \textcolor{Maroon}{(1)}} + \tilde{\varepsilon}^{\;\! \mathcolor{gray}{t} \hat{1}}_{\;\! \symup{\iota} \mathcolor{gray}{z} \textcolor{Maroon}{(1)}}~, \label{eq:weak_modulated_varepsilon} \\
	\zeta^{\;\! \mathcolor{gray}{t} \hat{1} \mathcolor{gray}{\check{1}}}_{\;\! \symup{\iota} \mathcolor{gray}{z} \textcolor{Maroon}{(1)}} = \zeta^{\;\! \mathcolor{gray}{t} \hat{1} \mathcolor{gray}{\check{1}}}_{\;\! \symup{\iota} \textcolor{Maroon}{(1)}} + \tilde{\zeta}^{\;\! \mathcolor{gray}{t} \hat{1} \mathcolor{gray}{\check{1}}}_{\;\! \symup{\iota} \mathcolor{gray}{z} \textcolor{Maroon}{(1)}}~, \\
	\zeta^{\;\! \mathcolor{gray}{t} \hat{1} \mathcolor{gray}{\check{1} \check{2}}}_{\;\! \symup{\iota} \mathcolor{gray}{z} \textcolor{Maroon}{(1)}} = \zeta^{\;\! \mathcolor{gray}{t} \hat{1} \mathcolor{gray}{\check{1} \check{2}}}_{\;\! \symup{\iota} \textcolor{Maroon}{(1)}} + \tilde{\zeta}^{\;\! \mathcolor{gray}{t} \hat{1} \mathcolor{gray}{\check{1} \check{2}}}_{\;\! \symup{\iota} \mathcolor{gray}{z} \textcolor{Maroon}{(1)}}~,
\end{align}
\end{subequations}
其中,$\tilde{\varepsilon}^{\;\! \mathcolor{gray}{t} \hat{1}}_{\;\! \symup{\iota} \mathcolor{gray}{z} \textcolor{Maroon}{(1)}}, \tilde{\zeta}^{\;\! \mathcolor{gray}{t} \hat{1} \mathcolor{gray}{\check{1}}}_{\;\! \symup{\iota} \mathcolor{gray}{z} \textcolor{Maroon}{(1)}}, \tilde{\zeta}^{\;\! \mathcolor{gray}{t} \hat{1} \mathcolor{gray}{\check{1} \check{2}}}_{\;\! \symup{\iota} \mathcolor{gray}{z} \textcolor{Maroon}{(1)}} := \delta \varepsilon^{\;\! \mathcolor{gray}{t} \hat{1}}_{\;\! \symup{\iota} \mathcolor{gray}{z} \textcolor{Maroon}{(1)}}, \delta \zeta^{\;\! \mathcolor{gray}{t} \hat{1} \mathcolor{gray}{\check{1}}}_{\;\! \symup{\iota} \mathcolor{gray}{z} \textcolor{Maroon}{(1)}}, \delta \zeta^{\;\! \mathcolor{gray}{t} \hat{1} \mathcolor{gray}{\check{1} \check{2}}}_{\;\! \symup{\iota} \mathcolor{gray}{z} \textcolor{Maroon}{(1)}}$ 代表略微偏离恒定/无调制材料常数背景(但仍然是时间\textcolor{NavyBlue}{色散}的)$\varepsilon^{\;\! \mathcolor{gray}{t} \hat{1}}_{\;\! \symup{\iota} \textcolor{Maroon}{(1)}}, \zeta^{\;\! \mathcolor{gray}{t} \hat{1} \mathcolor{gray}{\check{1}}}_{\;\! \symup{\iota} \textcolor{Maroon}{(1)}}, \zeta^{\;\! \mathcolor{gray}{t} \hat{1} \mathcolor{gray}{\check{1} \check{2}}}_{\;\! \symup{\iota} \textcolor{Maroon}{(1)}}$ 的\textcolor{NavyBlue}{微扰}(增量),见 \bref{hook:tilde}。

\clearpage

将 \bref{eq:weak_modulated_linear_sus} 代入 \bref{eq:wave_trho} 并一步步地回到 $\left( \mathcolor{gray}{\omega}, \mathcolor{gray}{\bar{k}_{\symup{\rho}}} \right)$ 域下的 \bref{eq:nonlinear(2)-wave_wkrho-simplify3};这等价于将 \bref{eq:nonlinear(2)-wave_wkrho-simplify3} 中,\textcolor{Plum}{线性}材料系数\textcolor{Plum}{均匀}分布的背景 $\varepsilon^{\;\! \mathcolor{gray}{\omega} \hat{1}}_{\;\! \symup{\iota} \textcolor{Maroon}{(1)}}, \zeta^{\;\! \mathcolor{gray}{\omega} \hat{1} \mathcolor{gray}{\check{1}}}_{\;\! \symup{\iota} \textcolor{Maroon}{(1)}}, \zeta^{\;\! \mathcolor{gray}{\omega} \hat{1} \mathcolor{gray}{\check{1} \check{2}}}_{\;\! \symup{\iota} \textcolor{Maroon}{(1)}}$\Footnote{由于其时域版本 $\varepsilon^{\;\! \mathcolor{gray}{t} \hat{1}}_{\;\! \symup{\iota} \textcolor{Maroon}{(1)}}, \zeta^{\;\! \mathcolor{gray}{t} \hat{1} \mathcolor{gray}{\check{1}}}_{\;\! \symup{\iota} \textcolor{Maroon}{(1)}}, \zeta^{\;\! \mathcolor{gray}{t} \hat{1} \mathcolor{gray}{\check{1} \check{2}}}_{\;\! \symup{\iota} \textcolor{Maroon}{(1)}}$ 在\textcolor{gray}{正空间}上不含空 $\mathcolor{gray}{\bar{r}}$,因此相应地,\textcolor{gray}{波长}域的\textcolor{gray}{单色} $\varepsilon^{\;\! \mathcolor{gray}{\omega} \hat{1}}_{\;\! \symup{\iota} \textcolor{Maroon}{(1)}}, \zeta^{\;\! \mathcolor{gray}{\omega} \hat{1} \mathcolor{gray}{\check{1}}}_{\;\! \symup{\iota} \textcolor{Maroon}{(1)}}, \zeta^{\;\! \mathcolor{gray}{\omega} \hat{1} \mathcolor{gray}{\check{1} \check{2}}}_{\;\! \symup{\iota} \textcolor{Maroon}{(1)}}$ 在\textcolor{gray}{倒空间}中不含\textcolor{gray}{空间频率} $\mathcolor{gray}{\bar{k}_{\symup{\rho}}}$,因此后者在主体上没有从左侧嵌入的一条灰色短线“\textcolor{gray}{--}”。} 与场(梯度)的乘积项,保留在左侧;同时将\textcolor{Plum}{线性}系数的\textcolor{NavyBlue}{微扰} $\Xint{\begin{smallmatrix} ~ \\ {}^{}_{\mathcolor{gray}{-}} \\ ~ \end{smallmatrix}}{16}{\tilde{\varepsilon}}^{\;\! \mathcolor{gray}{\omega} \hat{1}}_{\;\! \symup{\iota} \mathcolor{gray}{z} \textcolor{Maroon}{(1)}}, \Xint{\begin{smallmatrix} ~ \\ {}^{}_{\mathcolor{gray}{-}} \\ ~ \end{smallmatrix}}{12}{\tilde{\zeta}}^{\;\! \mathcolor{gray}{\omega} \hat{1} \mathcolor{gray}{\check{1}}}_{\;\! \symup{\iota} \mathcolor{gray}{z} \textcolor{Maroon}{(1)}}, \Xint{\begin{smallmatrix} ~ \\ {}^{}_{\mathcolor{gray}{-}} \\ ~ \end{smallmatrix}}{12}{\tilde{\zeta}}^{\;\! \mathcolor{gray}{\omega} \hat{1} \mathcolor{gray}{\check{1} \check{2}}}_{\;\! \symup{\iota} \mathcolor{gray}{z} \textcolor{Maroon}{(1)}}$ 与场(梯度)的卷积项,放在波动方程右侧,作为\textcolor{Plum}{线性}(势)散射源\cite{bornPrinciplesOptics60th2019,gerkeAperiodicVolumeOptics2010},同类于\textcolor{Plum}{非线性}\textcolor{NavyBlue}{波源}。于是 \bref{eq:nonlinear(2)-wave_wkrho-simplify3} 进一步等价于
\begin{subequations}
\begin{align}
	\varepsilon^{\;\! \mathcolor{gray}{\omega} \hat{1}}_{\;\! \symup{\iota} \textcolor{Maroon}{(1)}} \Xint{\mathcolor{gray}{-}}{295}{E}^{\;\!\mathcolor{gray}{\omega}}_{\;\! \hat{1} \mathcolor{gray}{z}} + \zeta^{\;\! \mathcolor{gray}{\omega} \hat{1} \mathcolor{gray}{\check{1}}}_{\;\! \symup{\iota} \textcolor{Maroon}{(1)}} \Xint{\mathcolor{gray}{-}}{295}{E}^{\;\!\mathcolor{gray}{\omega}}_{\;\! \mathcolor{gray}{\check{1}} \hat{1} \mathcolor{gray}{z}} + \zeta^{\;\! \mathcolor{gray}{\omega} \hat{1} \mathcolor{gray}{\check{1} \check{2}}}_{\;\! \symup{\iota} \textcolor{Maroon}{(1)}} \Xint{\mathcolor{gray}{-}}{295}{E}^{\;\!\mathcolor{gray}{\omega}}_{\;\! \mathcolor{gray}{\check{2} \check{1}} \hat{1}}
	= &- \Xint{{}^{}_{\mathcolor{gray}{-}}}{23}{\chi}^{\;\! \mathcolor{gray}{\omega} \hat{1} \hat{2}}_{\;\! \symup{\iota} \mathcolor{gray}{z} \textcolor{Maroon}{(2)}} \mathcolor{gray}{*} \left( \Xint{\mathcolor{gray}{-}}{295}{E}^{\;\!\mathcolor{gray}{\omega}}_{\;\! \hat{1} \mathcolor{gray}{z}} ~\mathcolor{gray}{\widetilde \circledast}~ \Xint{\mathcolor{gray}{-}}{295}{E}^{\;\!\mathcolor{gray}{\omega}}_{\;\! \hat{2} \mathcolor{gray}{z}} \right) \label{eq:nonlinear(2)-wave_wkrho-simplify4} \\
	-~ \Xint{\begin{smallmatrix} ~ \\ {}^{}_{\mathcolor{gray}{-}} \\ ~ \end{smallmatrix}}{16}{\tilde{\varepsilon}}^{\;\! \mathcolor{gray}{\omega} \hat{1}}_{\;\! \symup{\iota} \mathcolor{gray}{z} \textcolor{Maroon}{(1)}} \mathcolor{gray}{*} \Xint{\mathcolor{gray}{-}}{295}{E}^{\;\!\mathcolor{gray}{\omega}}_{\;\! \hat{1} \mathcolor{gray}{z}}
	- \Xint{\begin{smallmatrix} ~ \\ {}^{}_{\mathcolor{gray}{-}} \\ ~ \end{smallmatrix}}{12}{\tilde{\zeta}}^{\;\! \mathcolor{gray}{\omega} \hat{1} \mathcolor{gray}{\check{1}}}_{\;\! \symup{\iota} \mathcolor{gray}{z} \textcolor{Maroon}{(1)}} \mathcolor{gray}{*} \Xint{\mathcolor{gray}{-}}{295}{E}^{\;\!\mathcolor{gray}{\omega}}_{\;\! \mathcolor{gray}{\check{1}} \hat{1} \mathcolor{gray}{z}} &- \Xint{\begin{smallmatrix} ~ \\ {}^{}_{\mathcolor{gray}{-}} \\ ~ \end{smallmatrix}}{12}{\tilde{\zeta}}^{\;\! \mathcolor{gray}{\omega} \hat{1} \mathcolor{gray}{\check{1} \check{2}}}_{\;\! \symup{\iota} \mathcolor{gray}{z} \textcolor{Maroon}{(1)}} \mathcolor{gray}{*} \Xint{\mathcolor{gray}{-}}{295}{E}^{\;\!\mathcolor{gray}{\omega}}_{\;\! \mathcolor{gray}{\check{2} \check{1}} \hat{1} \mathcolor{gray}{z}} ~, \label{eq:nonlinear(2)-wave_wkrho-simplify4-2}
\end{align}
\end{subequations}
其中,类似 \bref{eq:P(1)_wk} 地缩写了 $\Xint{\mathcolor{gray}{-}}{25}{E}^{\;\!\mathcolor{gray}{\omega}}_{\;\! \mathcolor{gray}{\check{1}} \hat{1} \mathcolor{gray}{z}} := \mathcolor{gray}{k^\nabla_{\check{1}}} \Xint{\mathcolor{gray}{-}}{25}{E}^{\;\!\mathcolor{gray}{\omega}}_{\;\! \hat{1} \mathcolor{gray}{z}}$、$\Xint{\mathcolor{gray}{-}}{25}{E}^{\;\!\mathcolor{gray}{\omega}}_{\;\! \mathcolor{gray}{\check{2} \check{1}} \hat{1} \mathcolor{gray}{z}} := \mathcolor{gray}{k^\nabla_{\check{2}}} \mathcolor{gray}{k^\nabla_{\check{1}}} \Xint{\mathcolor{gray}{-}}{25}{E}^{\;\!\mathcolor{gray}{\omega}}_{\;\! \hat{1} \mathcolor{gray}{z}}$。一般而言,材料系数的\textcolor{Plum}{线性}调制项所带来的\textcolor{Plum}{线性}散射,与\textcolor{Plum}{非线性}\textcolor{NavyBlue}{波源}产生\textcolor{gray}{新频率组分}的相干辐射的动力学过程,在\textcolor{gray}{正空间}\textcolor{NavyBlue}{驱动源}、\textcolor{gray}{倒空间}\textcolor{PineGreen}{波矢}匹配方面等许多方面有一定的相似性\cite{gerkeAperiodicVolumeOptics2010,chenQuasiphasematchingdivisionMultiplexingHolography2021b},但仍需要分开考虑(尽管原则上因同时发生而无法解耦)。那么这里不妨先丢弃光子晶体项/材料系数的\textcolor{Plum}{线性}调制项(后续会朝花夕拾它),但保留\textcolor{Plum}{非线性}项。这样一来,\bref{eq:nonlinear(2)-wave_wkrho-simplify4} 只剩下
\begin{align} \label{eq:nonlinear(2)-wave_wkrho-simplify5}
	\varepsilon^{\;\! \mathcolor{gray}{\omega} \hat{1}}_{\;\! \symup{\iota} \textcolor{Maroon}{(1)}} \Xint{\mathcolor{gray}{-}}{295}{E}^{\;\!\mathcolor{gray}{\omega}}_{\;\! \hat{1} \mathcolor{gray}{z}} + \zeta^{\;\! \mathcolor{gray}{\omega} \hat{1} \mathcolor{gray}{\check{1}}}_{\;\! \symup{\iota} \textcolor{Maroon}{(1)}} \Xint{\mathcolor{gray}{-}}{295}{E}^{\;\!\mathcolor{gray}{\omega}}_{\;\! \mathcolor{gray}{\check{1}} \hat{1} \mathcolor{gray}{z}} + \zeta^{\;\! \mathcolor{gray}{\omega} \hat{1} \mathcolor{gray}{\check{1} \check{2}}}_{\;\! \symup{\iota} \textcolor{Maroon}{(1)}} \Xint{\mathcolor{gray}{-}}{295}{E}^{\;\!\mathcolor{gray}{\omega}}_{\;\! \mathcolor{gray}{\check{2} \check{1}} \hat{1} \mathcolor{gray}{z}}
	= &- \Xint{{}^{}_{\mathcolor{gray}{-}}}{23}{\chi}^{\;\! \mathcolor{gray}{\omega} \hat{1} \hat{2}}_{\;\! \symup{\iota} \mathcolor{gray}{z} \textcolor{Maroon}{(2)}} \mathcolor{gray}{*} \left( \Xint{\mathcolor{gray}{-}}{295}{E}^{\;\!\mathcolor{gray}{\omega}}_{\;\! \hat{1} \mathcolor{gray}{z}} ~\mathcolor{gray}{\widetilde \circledast}~ \Xint{\mathcolor{gray}{-}}{295}{E}^{\;\!\mathcolor{gray}{\omega}}_{\;\! \hat{2} \mathcolor{gray}{z}} \right) ~,
\end{align}
至此,方程彻底分离了出了\textcolor{Plum}{非局域}\textcolor{PineGreen}{双各向异性}耦合的\textcolor{Plum}{线性}\textcolor{Plum}{均匀}部分,并全部收归至方程左侧;而且右侧只剩下一项具有代表性的\textcolor{Plum}{局域}二阶\textcolor{Plum}{非线性}\textcolor{Plum}{非均匀}\textcolor{NavyBlue}{波源}。为了将左侧\textcolor{Plum}{线性}\textcolor{Plum}{均匀}部分,分解为泾渭分明的\textcolor{Plum}{非线性}算子 $\Xint{\mathcolor{gray}{-}}{21}{\bar{\bar{V}}}^{\;\! \mathcolor{gray}{\omega}}$ 控制部分和纯净的\textcolor{Plum}{线性}算子 $\Xint{\mathcolor{gray}{-}}{30}{\bar{\bar{L}}}^{\;\! \mathcolor{gray}{\omega}}$ 作用部分(的叠加),\bref{eq:nonlinear(2)-wave_wkrho-simplify5} 最好写成矢量(非分量)形式
\begin{align} \label{eq:nonlinear(2)-wave_wkrho-simplify5-vector}
	\bar{\bar{\varepsilon}}^{\;\! \mathcolor{gray}{\omega}}_{\textcolor{Maroon}{(1)}} \Xint{\mathcolor{gray}{-}}{295}{\bar{E}}^{\;\!\mathcolor{gray}{\omega}}_{\;\! \mathcolor{gray}{z}} + \bar{\bar{\zeta}}^{\;\! \mathcolor{gray}{\omega} \mathcolor{gray}{\check{1}}}_{\textcolor{Maroon}{(1)}} \Xint{\mathcolor{gray}{-}}{295}{\bar{E}}^{\;\!\mathcolor{gray}{\omega}}_{\;\! \mathcolor{gray}{\check{1}} \mathcolor{gray}{z}} + \bar{\bar{\zeta}}^{\;\! \mathcolor{gray}{\omega} \mathcolor{gray}{\check{1} \check{2}}}_{\textcolor{Maroon}{(1)}} \Xint{\mathcolor{gray}{-}}{295}{\bar{E}}^{\;\!\mathcolor{gray}{\omega}}_{\;\! \mathcolor{gray}{\check{2} \check{1}} \mathcolor{gray}{z}}
	= &- \Xint{{}^{}_{\mathcolor{gray}{-}}}{23}{\bar{\bar{\bar{\chi}}}}^{\;\! \mathcolor{gray}{\omega}}_{\mathcolor{gray}{z} \textcolor{Maroon}{(2)}} ~{}^{\mathcolor{gray}{*}}_{\mathcolor{gray}{*}} \left( \Xint{\mathcolor{gray}{-}}{295}{\bar{E}}^{\;\!\mathcolor{gray}{\omega}}_{\;\! \mathcolor{gray}{z}} ~\mathcolor{gray}{\widetilde \circledast}~ \Xint{\mathcolor{gray}{-}}{295}{\bar{E}}^{\;\!\mathcolor{gray}{\omega}}_{\;\! \mathcolor{gray}{z}} \right) ~,
\end{align}
利用 \bref{eq:matrix_exp-dz,eq:matrix_exp-ddz},可以将 \bref{eq:nonlinear(2)-wave_wkrho-simplify5-vector} 的左侧,分离出\textcolor{Plum}{线性}算子 $\Xint{\mathcolor{gray}{-}}{30}{\bar{\bar{L}}}^{\;\! \mathcolor{gray}{\omega}}$ 对应的
\begin{align} \label{eq:matrix_exp_eq-left_L}
	\Xint{\mathcolor{gray}{-}}{32}{\bar{\bar{L}}}^{\;\! \mathcolor{gray}{\omega}} \Xint{\mathcolor{gray}{-}}{295}{\bar{E}}^{\;\!\mathcolor{gray}{\omega}}_{\;\! \mathcolor{gray}{z}} &:= \left( \bar{\bar{\varepsilon}}^{\;\! \mathcolor{gray}{\omega}}_{\textcolor{Maroon}{(1)}} + \bar{\bar{\zeta}}^{\;\! \mathcolor{gray}{\omega} \mathcolor{gray}{\check{1}}}_{\textcolor{Maroon}{(1)}} \mathbb{i} \Xint{\begin{smallmatrix} ~ \\ {}^{}_{\mathcolor{gray}{-}} \\ ~ \end{smallmatrix}}{15}{\bar{\bar{k}}}_{\;\! \mathcolor{gray}{\check{1}}}^{\;\! \mathcolor{gray}{\omega}} - \bar{\bar{\zeta}}^{\;\! \mathcolor{gray}{\omega} \mathcolor{gray}{\check{1} \check{2}}}_{\textcolor{Maroon}{(1)}} \Xint{\begin{smallmatrix} ~ \\ {}^{}_{\mathcolor{gray}{-}} \\ ~ \end{smallmatrix}}{15}{\bar{\bar{k}}}_{\;\! \mathcolor{gray}{\check{2}}}^{\;\! \mathcolor{gray}{\omega}} \Xint{\begin{smallmatrix} ~ \\ {}^{}_{\mathcolor{gray}{-}} \\ ~ \end{smallmatrix}}{15}{\bar{\bar{k}}}_{\;\! \mathcolor{gray}{\check{1}}}^{\;\! \mathcolor{gray}{\omega}} \right) \Xint{\mathcolor{gray}{-}}{295}{\bar{E}}^{\;\!\mathcolor{gray}{\omega}}_{\;\! \mathcolor{gray}{z}} ~,
\end{align}
其中,类似 \bref{eq:k_check_j^nabla} 地定义了
%\clearpage
%\vspace*{-4.5em}
\begin{align} \label{eq:barbar_k_check_j}
	\Xint{\begin{smallmatrix} ~ \\ {}^{}_{\mathcolor{gray}{-}} \\ ~ \end{smallmatrix}}{15}{\bar{\bar{k}}}_{\;\! \mathcolor{gray}{\check{\symup{\jmath}}}}^{\;\! \mathcolor{gray}{\omega}} := \mathcolor{gray}{k_{\symup{x}}} ~\textcolor{Maroon}{\text{或}}~ \mathcolor{gray}{k_{\symup{y}}} ~\textcolor{Maroon}{\text{或}}~ \Xint{\begin{smallmatrix} ~ \\ {}^{}_{\mathcolor{gray}{-}} \\ ~ \end{smallmatrix}}{15}{\bar{\bar{k}}}_{\textcolor{Maroon}{\symup{z}}}^{\;\! \mathcolor{gray}{\omega}}~,
\end{align}
同样利用 \bref{eq:matrix_exp-dz,eq:matrix_exp-ddz},可以将 \bref{eq:nonlinear(2)-wave_wkrho-simplify5-vector} 的左侧,分离出\textcolor{Plum}{非线性}算子 $\Xint{\mathcolor{gray}{-}}{21}{\bar{\bar{V}}}^{\;\! \mathcolor{gray}{\omega}}$
\begin{subequations}
\begin{align}
	\!\!\! \Xint{\mathcolor{gray}{-}}{25}{\bar{\bar{V}}}^{\;\! \mathcolor{gray}{\omega}} \Xint{{}^{}_{\mathcolor{gray}{-}}}{10}{\bar{g}}^{\;\!\mathcolor{gray}{\omega}}_{\;\! \mathcolor{gray}{z}} &:= \left[ \left( \bar{\bar{\zeta}}^{\;\! \mathcolor{gray}{\omega} \mathcolor{gray}{\symup{z}}}_{\textcolor{Maroon}{(1)}} + \bar{\bar{\zeta}}^{\;\! \mathcolor{gray}{\omega} \mathcolor{gray}{\check{1} \symup{z}}}_{\textcolor{Maroon}{(1)}} \mathbb{i} \Xint{\begin{smallmatrix} ~ \\ {}^{}_{\mathcolor{gray}{-}} \\ ~ \end{smallmatrix}}{15}{\bar{\bar{k}}}_{\;\! \mathcolor{gray}{\check{1}}}^{\;\! \mathcolor{gray}{\omega}} + \bar{\bar{\zeta}}^{\;\! \mathcolor{gray}{\omega} \mathcolor{gray}{\symup{z} \check{2}}}_{\textcolor{Maroon}{(1)}} \mathbb{i} \Xint{\begin{smallmatrix} ~ \\ {}^{}_{\mathcolor{gray}{-}} \\ ~ \end{smallmatrix}}{15}{\bar{\bar{k}}}_{\;\! \mathcolor{gray}{\check{2}}}^{\;\! \mathcolor{gray}{\omega}} \right) \mathbb{e}^{\mathbb{i} \Xint{\begin{smallmatrix} ~ \\ {}^{}_{\mathcolor{gray}{-}} \\ ~ \end{smallmatrix}}{15}{\bar{\bar{k}}}_{\textcolor{Maroon}{\symup{z}}}^{\;\! \mathcolor{gray}{\omega}} \mathcolor{gray}{z}} \mathcolor{gray}{\nabla_z} + \bar{\bar{\zeta}}^{\;\! \mathcolor{gray}{\omega} \mathcolor{gray}{\symup{z} \symup{z}}}_{\textcolor{Maroon}{(1)}} \mathbb{e}^{\mathbb{i} \Xint{\begin{smallmatrix} ~ \\ {}^{}_{\mathcolor{gray}{-}} \\ ~ \end{smallmatrix}}{15}{\bar{\bar{k}}}_{\textcolor{Maroon}{\symup{z}}}^{\;\! \mathcolor{gray}{\omega}} \mathcolor{gray}{z}} \mathcolor{gray}{\nabla_z^2} \right] \Xint{{}^{}_{\mathcolor{gray}{-}}}{10}{\bar{g}}^{\;\!\mathcolor{gray}{\omega}}_{\;\! \mathcolor{gray}{z}} \label{eq:matrix_exp_eq-left_V} \\
	\!\!\! &=: \left[ \Xint{\mathcolor{gray}{-}}{25}{\bar{\bar{V}}}^{\;\! \mathcolor{gray}{\omega}}_{\textcolor{Maroon}{\mathbb{1}}} + \Xint{\mathcolor{gray}{-}}{25}{\bar{\bar{V}}}^{\;\! \mathcolor{gray}{\omega}}_{\textcolor{Maroon}{\mathbb{2}}} \right] \Xint{{}^{}_{\mathcolor{gray}{-}}}{10}{\bar{g}}^{\;\!\mathcolor{gray}{\omega}}_{\;\! \mathcolor{gray}{z}} =: \left[ \Xint{\mathcolor{gray}{-}}{25}{\bar{\bar{\mathsfit{V}}}}^{\;\! \mathcolor{gray}{\omega}}_{\textcolor{Maroon}{\mathbb{1}}} \mathbb{e}^{\mathbb{i} \Xint{\begin{smallmatrix} ~ \\ {}^{}_{\mathcolor{gray}{-}} \\ ~ \end{smallmatrix}}{15}{\bar{\bar{k}}}_{\textcolor{Maroon}{\symup{z}}}^{\;\! \mathcolor{gray}{\omega}} \mathcolor{gray}{z}} \mathcolor{gray}{\nabla_z} + \Xint{\mathcolor{gray}{-}}{25}{\bar{\bar{\mathsfit{V}}}}^{\;\! \mathcolor{gray}{\omega}}_{\textcolor{Maroon}{\mathbb{2}}} \mathbb{e}^{\mathbb{i} \Xint{\begin{smallmatrix} ~ \\ {}^{}_{\mathcolor{gray}{-}} \\ ~ \end{smallmatrix}}{15}{\bar{\bar{k}}}_{\textcolor{Maroon}{\symup{z}}}^{\;\! \mathcolor{gray}{\omega}} \mathcolor{gray}{z}} \mathcolor{gray}{\nabla_z^2} \right] \Xint{{}^{}_{\mathcolor{gray}{-}}}{10}{\bar{g}}^{\;\!\mathcolor{gray}{\omega}}_{\;\! \mathcolor{gray}{z}} ~, \label{eq:matrix_exp_eq-left_V-1st2nd-decompose}
\end{align}
\end{subequations}
其中,定义了\textcolor{Plum}{一阶导} $=$ \textcolor{NavyBlue}{缓变振幅}\textcolor{Plum}{非线性}算子 $\Xint{\mathcolor{gray}{-}}{21}{\bar{\bar{V}}}^{\;\! \mathcolor{gray}{\omega}}_{\textcolor{Maroon}{\mathbb{1}}}$ 中的\textcolor{Maroon}{矩阵(算子)部分}\Footnote{新增了字体 \textbackslash mathsf,以区分 $\mathsfit{V} \longleftrightarrow V$,见 \bref{hook:mathsf}。}
\begin{align} \label{eq:matrix_exp-V1}
	\Xint{\mathcolor{gray}{-}}{25}{\bar{\bar{\mathsfit{V}}}}^{\;\! \mathcolor{gray}{\omega}}_{\textcolor{Maroon}{\mathbb{1}}} := \bar{\bar{\zeta}}^{\;\! \mathcolor{gray}{\omega} \mathcolor{gray}{\symup{z}}}_{\textcolor{Maroon}{(1)}} + \bar{\bar{\zeta}}^{\;\! \mathcolor{gray}{\omega} \mathcolor{gray}{\check{1} \symup{z}}}_{\textcolor{Maroon}{(1)}} \mathbb{i} \Xint{\begin{smallmatrix} ~ \\ {}^{}_{\mathcolor{gray}{-}} \\ ~ \end{smallmatrix}}{15}{k}_{\;\! \mathcolor{gray}{\check{1}}}^{\;\! \mathcolor{gray}{\omega}} + \bar{\bar{\zeta}}^{\;\! \mathcolor{gray}{\omega} \mathcolor{gray}{\symup{z} \check{2}}}_{\textcolor{Maroon}{(1)}} \mathbb{i} \Xint{\begin{smallmatrix} ~ \\ {}^{}_{\mathcolor{gray}{-}} \\ ~ \end{smallmatrix}}{15}{\bar{\bar{k}}}_{\;\! \mathcolor{gray}{\check{2}}}^{\;\! \mathcolor{gray}{\omega}} ~, 
\end{align}
以及\textcolor{Plum}{非线性}算子中的\textcolor{Maroon}{二阶导部分} $\Xint{\mathcolor{gray}{-}}{21}{\bar{\bar{V}}}^{\;\! \mathcolor{gray}{\omega}}_{\textcolor{Maroon}{\mathbb{2}}}$ 中的(\textcolor{Plum}{线性}/\textcolor{Plum}{无偏导}的)\textcolor{Maroon}{矩阵(算子)部分}
\begin{align} \label{eq:matrix_exp-V2}
	\Xint{\mathcolor{gray}{-}}{25}{\bar{\bar{\mathsfit{V}}}}^{\;\! \mathcolor{gray}{\omega}}_{\textcolor{Maroon}{\mathbb{2}}} := \bar{\bar{\zeta}}^{\;\! \mathcolor{gray}{\omega} \mathcolor{gray}{\symup{z} \symup{z}}}_{\textcolor{Maroon}{(1)}} ~, 
\end{align}
注意,\bref{eq:symmetry1} 中的\textcolor{Plum}{置换对称性}一般不成立,即普遍有 $\bar{\bar{\zeta}}^{\;\! \mathcolor{gray}{\omega} \mathcolor{gray}{\symup{x} \symup{z}}}_{\textcolor{Maroon}{(1)}} \neq \bar{\bar{\zeta}}^{\;\! \mathcolor{gray}{\omega} \mathcolor{gray}{\symup{z} \symup{x}}}_{\textcolor{Maroon}{(1)}}$、 $\bar{\bar{\zeta}}^{\;\! \mathcolor{gray}{\omega} \mathcolor{gray}{\symup{y} \symup{z}}}_{\textcolor{Maroon}{(1)}} \neq \bar{\bar{\zeta}}^{\;\! \mathcolor{gray}{\omega} \mathcolor{gray}{\symup{z} \symup{y}}}_{\textcolor{Maroon}{(1)}}$。此外,对于\textcolor{Maroon}{矩阵指数}形式的\textcolor{PineGreen}{平面波}解 \bref{eq:vec-matrix_exp},\textcolor{Plum}{非线性}算子 $\Xint{\mathcolor{gray}{-}}{21}{\bar{\bar{V}}}^{\;\! \mathcolor{gray}{\omega}}$ 不能写成直接作用于 $\Xint{\mathcolor{gray}{-}}{25}{\bar{E}}^{\;\!\mathcolor{gray}{\omega}}_{\;\! \mathcolor{gray}{z}} = \mathbb{e}^{\mathbb{i} \Xint{\begin{smallmatrix} ~ \\ {}^{}_{\mathcolor{gray}{-}} \\ ~ \end{smallmatrix}}{15}{\bar{\bar{k}}}_{\textcolor{Maroon}{\symup{z}}}^{\;\! \mathcolor{gray}{\omega}} \mathcolor{gray}{z}} \Xint{{}^{}_{\mathcolor{gray}{-}}}{10}{\bar{g}}^{\;\!\mathcolor{gray}{\omega}}_{\;\! \mathcolor{gray}{z}}$ 整体,即 \bref{eq:matrix_exp_eq-left_L} 的形式;只能退而求其次地,写成作用于 $\Xint{{}^{}_{\mathcolor{gray}{-}}}{10}{\bar{g}}^{\;\!\mathcolor{gray}{\omega}}_{\;\! \mathcolor{gray}{z}}$ 的形式,如 \bref{eq:matrix_exp_eq-left_V} 所示。

有了 $\left( \mathcolor{gray}{\omega}, \mathcolor{gray}{\bar{k}_{\symup{\rho}}} \right)$ 域中, \bref{eq:matrix_exp_eq-left_L} 定义的\textcolor{Plum}{线性}算子 $\Xint{\mathcolor{gray}{-}}{30}{\bar{\bar{L}}}^{\;\! \mathcolor{gray}{\omega}}$ 和 \bref{eq:matrix_exp_eq-left_V-1st2nd-decompose} 定义的\textcolor{Plum}{非线性}算子 $\Xint{\mathcolor{gray}{-}}{21}{\bar{\bar{V}}}^{\;\! \mathcolor{gray}{\omega}}$ 后,\bref{eq:nonlinear(2)-wave_wkrho-simplify5-vector} 可进一步简写作
\begin{subequations}
\begin{align}
	\left( \Xint{\mathcolor{gray}{-}}{32}{\bar{\bar{L}}}^{\;\! \mathcolor{gray}{\omega}} \mathbb{e}^{\mathbb{i} \Xint{\begin{smallmatrix} ~ \\ {}^{}_{\mathcolor{gray}{-}} \\ ~ \end{smallmatrix}}{15}{\bar{\bar{k}}}_{\textcolor{Maroon}{\symup{z}}}^{\;\! \mathcolor{gray}{\omega}} \mathcolor{gray}{z}} + \Xint{\mathcolor{gray}{-}}{25}{\bar{\bar{V}}}^{\;\! \mathcolor{gray}{\omega}} \right) \Xint{{}^{}_{\mathcolor{gray}{-}}}{10}{\bar{g}}^{\;\!\mathcolor{gray}{\omega}}_{\;\! \mathcolor{gray}{z}}
	&= - \Xint{\mathcolor{gray}{-}}{30}{\bar{P}}^{\;\! \mathcolor{gray}{\omega}}_{\;\! \mathcolor{gray}{z} \textcolor{Maroon}{(2)}} ~, \label{eq:nonlinear(2)-wave_wkrho-simplify6} \\
	\left( \Xint{\mathcolor{gray}{-}}{32}{\bar{\bar{L}}}^{\;\! \mathcolor{gray}{\omega}} \mathbb{e}^{\mathbb{i} \Xint{\begin{smallmatrix} ~ \\ {}^{}_{\mathcolor{gray}{-}} \\ ~ \end{smallmatrix}}{15}{\bar{\bar{k}}}_{\textcolor{Maroon}{\symup{z}}}^{\;\! \mathcolor{gray}{\omega}} \mathcolor{gray}{z}} + \Xint{\mathcolor{gray}{-}}{25}{\bar{\bar{\mathsfit{V}}}}^{\;\! \mathcolor{gray}{\omega}}_{\textcolor{Maroon}{\mathbb{1}}} \mathbb{e}^{\mathbb{i} \Xint{\begin{smallmatrix} ~ \\ {}^{}_{\mathcolor{gray}{-}} \\ ~ \end{smallmatrix}}{15}{\bar{\bar{k}}}_{\textcolor{Maroon}{\symup{z}}}^{\;\! \mathcolor{gray}{\omega}} \mathcolor{gray}{z}} \mathcolor{gray}{\nabla_z} + \Xint{\mathcolor{gray}{-}}{25}{\bar{\bar{\mathsfit{V}}}}^{\;\! \mathcolor{gray}{\omega}}_{\textcolor{Maroon}{\mathbb{2}}} \mathbb{e}^{\mathbb{i} \Xint{\begin{smallmatrix} ~ \\ {}^{}_{\mathcolor{gray}{-}} \\ ~ \end{smallmatrix}}{15}{\bar{\bar{k}}}_{\textcolor{Maroon}{\symup{z}}}^{\;\! \mathcolor{gray}{\omega}} \mathcolor{gray}{z}} \mathcolor{gray}{\nabla_z^2} \right) \Xint{{}^{}_{\mathcolor{gray}{-}}}{10}{\bar{g}}^{\;\!\mathcolor{gray}{\omega}}_{\;\! \mathcolor{gray}{z}}
	&= - \Xint{\mathcolor{gray}{-}}{30}{\bar{P}}^{\;\! \mathcolor{gray}{\omega}}_{\;\! \mathcolor{gray}{z} \textcolor{Maroon}{(2)}} ~, \label{eq:nonlinear(2)-wave_wkrho-simplify6-1st2nd-decompose}
\end{align}
\end{subequations}
其中,定义了 $\left( \mathcolor{gray}{\omega}, \mathcolor{gray}{\bar{k}_{\symup{\rho}}} \right)$ 域二阶\textcolor{Plum}{局域}\textcolor{Plum}{非线性}\textcolor{NavyBlue}{电偶-$(\text{电偶}\otimes\text{电偶})$}极矩场\Footnote{相对于最具体/完整的矢量形式 $\Xint{\mathcolor{gray}{-}}{24}{\bar{P}}^{\;\! \textcolor{Maroon}{(2)} \mathcolor{gray}{\omega}}_{\;\! \mathcolor{gray}{z} \textcolor{NavyBlue}{\text{pp}}} = \Xint{{}^{}_{\mathcolor{gray}{-}}}{23}{\bar{\bar{\bar{\chi}}}}^{\;\! \textcolor{NavyBlue}{\text{p}} \mathcolor{gray}{\omega}}_{\mathcolor{gray}{z} \textcolor{Maroon}{(2)} \textcolor{NavyBlue}{\text{pp}}} ~{}^{\mathcolor{gray}{*}}_{\mathcolor{gray}{*}} \cdots$ 和分量形式 $\Xint{\mathcolor{gray}{-}}{24}{P}^{\;\! \textcolor{Maroon}{(2)} \mathcolor{gray}{\omega}}_{\;\! \symup{\iota}\mathcolor{gray}{z} \textcolor{NavyBlue}{\text{pp}}} = \Xint{{}^{}_{\mathcolor{gray}{-}}}{23}{\chi}^{\;\! \textcolor{NavyBlue}{\text{p}} \mathcolor{gray}{\omega} \hat{1} \hat{2}}_{\;\! \symup{\iota} \mathcolor{gray}{z} \textcolor{Maroon}{(2)} \textcolor{NavyBlue}{\text{pp}}} \mathcolor{gray}{*} \cdots$,省略了一些角标。}
\begin{align} \label{eq:vec-DP^(2)-p_pp}
	\Xint{\mathcolor{gray}{-}}{30}{\bar{D}}^{\;\! \mathcolor{gray}{\omega}}_{\;\! \mathcolor{gray}{z} \textcolor{Maroon}{(2)}} &= \Xint{\mathcolor{gray}{-}}{30}{\bar{P}}^{\;\! \mathcolor{gray}{\omega}}_{\;\! \mathcolor{gray}{z} \textcolor{Maroon}{(2)}} = \Xint{{}^{}_{\mathcolor{gray}{-}}}{23}{\bar{\bar{\bar{\chi}}}}^{\;\! \mathcolor{gray}{\omega}}_{\mathcolor{gray}{z} \textcolor{Maroon}{(2)}} ~{}^{\mathcolor{gray}{*}}_{\mathcolor{gray}{*}} \left( \Xint{\mathcolor{gray}{-}}{295}{\bar{E}}^{\;\!\mathcolor{gray}{\omega}}_{\;\! \mathcolor{gray}{z}} ~\mathcolor{gray}{\widetilde \circledast}~ \Xint{\mathcolor{gray}{-}}{295}{\bar{E}}^{\;\!\mathcolor{gray}{\omega}}_{\;\! \mathcolor{gray}{z}} \right) ~,
\end{align}

\bref{eq:nonlinear(2)-wave_wkrho-simplify6-1st2nd-decompose} 是一个\textcolor{Plum}{非齐次} 2 阶 3 维偏微分矩阵方程 or 常微分方程组。一种通用的方法是将其降 1 阶、升 3 维\Footnote{“阶”指的是“最高导数的阶数”,而不是关于待解变量的\textcolor{Plum}{非线性}特性所对应的“次”数。},转换为 1 阶 6 维\textcolor{Plum}{非齐次}\textcolor{Plum}{线性}系统。但这要求 $\Xint{\mathcolor{gray}{-}}{25}{\bar{\bar{\mathsfit{V}}}}^{\;\! \mathcolor{gray}{\omega}}_{\textcolor{Maroon}{\mathbb{2}}} \mathbb{e}^{\mathbb{i} \Xint{\begin{smallmatrix} ~ \\ {}^{}_{\mathcolor{gray}{-}} \\ ~ \end{smallmatrix}}{15}{\bar{\bar{k}}}_{\textcolor{Maroon}{\symup{z}}}^{\;\! \mathcolor{gray}{\omega}} \mathcolor{gray}{z}}$ 是\textcolor{Plum}{可逆}的,即 $\Xint{\mathcolor{gray}{-}}{25}{\bar{\bar{\mathsfit{V}}}}^{\;\! \mathcolor{gray}{\omega}}_{\textcolor{Maroon}{\mathbb{2}}}$ 是\textcolor{Plum}{非奇异}的。然而,下文会发现,有些时候 $\Xint{\mathcolor{gray}{-}}{25}{\bar{\bar{\mathsfit{V}}}}^{\;\! \mathcolor{gray}{\omega}}_{\textcolor{Maroon}{\mathbb{2}}} = \bar{\bar{\zeta}}^{\;\! \mathcolor{gray}{\omega} \mathcolor{gray}{\symup{z} \symup{z}}}_{\textcolor{Maroon}{(1)}}$ 的逆不存在。从另一个角度,可以借鉴\textcolor{Plum}{非线性}\textcolor{NavyBlue}{光学}中的“\textcolor{NavyBlue}{缓变振幅近似}”思想,直接舍弃 \bref{eq:nonlinear(2)-wave_wkrho-simplify6-1st2nd-decompose} 中的 $\Xint{\mathcolor{gray}{-}}{25}{\bar{\bar{\mathsfit{V}}}}^{\;\! \mathcolor{gray}{\omega}}_{\textcolor{Maroon}{\mathbb{2}}}$ 项,得到 \textcolor{NavyBlue}{缓变振幅近似} 条件下的
\begin{align} \label{eq:nonlinear(2)-wave_wkrho-simplify6-SVA}
	\left( \Xint{\mathcolor{gray}{-}}{32}{\bar{\bar{L}}}^{\;\! \mathcolor{gray}{\omega}} \mathbb{e}^{\mathbb{i} \Xint{\begin{smallmatrix} ~ \\ {}^{}_{\mathcolor{gray}{-}} \\ ~ \end{smallmatrix}}{15}{\bar{\bar{k}}}_{\textcolor{Maroon}{\symup{z}}}^{\;\! \mathcolor{gray}{\omega}} \mathcolor{gray}{z}} + \Xint{\mathcolor{gray}{-}}{25}{\bar{\bar{\mathsfit{V}}}}^{\;\! \mathcolor{gray}{\omega}}_{\textcolor{Maroon}{\mathbb{1}}} \mathbb{e}^{\mathbb{i} \Xint{\begin{smallmatrix} ~ \\ {}^{}_{\mathcolor{gray}{-}} \\ ~ \end{smallmatrix}}{15}{\bar{\bar{k}}}_{\textcolor{Maroon}{\symup{z}}}^{\;\! \mathcolor{gray}{\omega}} \mathcolor{gray}{z}} \mathcolor{gray}{\nabla_z} \right) \Xint{{}^{}_{\mathcolor{gray}{-}}}{10}{\bar{g}}^{\;\!\mathcolor{gray}{\omega}}_{\;\! \mathcolor{gray}{z}}
	&= - \Xint{\mathcolor{gray}{-}}{30}{\bar{P}}^{\;\! \mathcolor{gray}{\omega}}_{\;\! \mathcolor{gray}{z} \textcolor{Maroon}{(2)}} ~, 
\end{align}
这便把 \bref{eq:nonlinear(2)-wave_wkrho-simplify6-1st2nd-decompose} 降阶至了 1 阶 3 维\textcolor{Plum}{非齐次}\textcolor{Plum}{线性}系统。可以使用 \bref{eq:vec-matrix_exp},进一步将上面的 \bref{eq:nonlinear(2)-wave_wkrho-simplify6-SVA} 中的待解矢量场 $\Xint{{}^{}_{\mathcolor{gray}{-}}}{10}{\bar{g}}^{\;\!\mathcolor{gray}{\omega}}_{\;\! \mathcolor{gray}{z}}$ \textcolor{Maroon}{变量替换}为 $\Xint{\mathcolor{gray}{-}}{25}{\bar{E}}^{\;\!\mathcolor{gray}{\omega}}_{\;\! \mathcolor{gray}{z}} = \mathbb{e}^{\mathbb{i} \Xint{\begin{smallmatrix} ~ \\ {}^{}_{\mathcolor{gray}{-}} \\ ~ \end{smallmatrix}}{15}{\bar{\bar{k}}}_{\textcolor{Maroon}{\symup{z}}}^{\;\! \mathcolor{gray}{\omega}} \mathcolor{gray}{z}} \Xint{{}^{}_{\mathcolor{gray}{-}}}{10}{\bar{g}}^{\;\!\mathcolor{gray}{\omega}}_{\;\! \mathcolor{gray}{z}}$,以消除方程中的\textcolor{Maroon}{矩阵指数},得
\begin{subequations}
\begin{align}
	\left[ \Xint{\mathcolor{gray}{-}}{32}{\bar{\bar{L}}}^{\;\! \mathcolor{gray}{\omega}} + \Xint{\mathcolor{gray}{-}}{25}{\bar{\bar{\mathsfit{V}}}}^{\;\! \mathcolor{gray}{\omega}}_{\textcolor{Maroon}{\mathbb{1}}} \left( \mathcolor{gray}{\nabla_z} - \mathbb{i} \Xint{\begin{smallmatrix} ~ \\ {}^{}_{\mathcolor{gray}{-}} \\ ~ \end{smallmatrix}}{15}{\bar{\bar{k}}}_{\textcolor{Maroon}{\symup{z}}}^{\;\! \mathcolor{gray}{\omega}} \right) \right] \Xint{\mathcolor{gray}{-}}{295}{\bar{E}}^{\;\!\mathcolor{gray}{\omega}}_{\;\! \mathcolor{gray}{z}}
	&= - \Xint{\mathcolor{gray}{-}}{30}{\bar{P}}^{\;\! \mathcolor{gray}{\omega}}_{\;\! \mathcolor{gray}{z} \textcolor{Maroon}{(2)}} \label{eq:nonlinear(2)-wave_wkrho-simplify6-SVA-E} \\
	\Xint{\mathcolor{gray}{-}}{25}{\bar{\bar{\mathsfit{V}}}}^{\;\! \mathcolor{gray}{\omega}}_{\textcolor{Maroon}{\mathbb{1}}} \mathcolor{gray}{\nabla_z} \Xint{\mathcolor{gray}{-}}{295}{\bar{E}}^{\;\!\mathcolor{gray}{\omega}}_{\;\! \mathcolor{gray}{z}}
	&= \left( \Xint{\mathcolor{gray}{-}}{25}{\bar{\bar{\mathsfit{V}}}}^{\;\! \mathcolor{gray}{\omega}}_{\textcolor{Maroon}{\mathbb{1}}} \mathbb{i} \Xint{\begin{smallmatrix} ~ \\ {}^{}_{\mathcolor{gray}{-}} \\ ~ \end{smallmatrix}}{15}{\bar{\bar{k}}}_{\textcolor{Maroon}{\symup{z}}}^{\;\! \mathcolor{gray}{\omega}} - \Xint{\mathcolor{gray}{-}}{32}{\bar{\bar{L}}}^{\;\! \mathcolor{gray}{\omega}} \right) \Xint{\mathcolor{gray}{-}}{295}{\bar{E}}^{\;\!\mathcolor{gray}{\omega}}_{\;\! \mathcolor{gray}{z}} - \Xint{\mathcolor{gray}{-}}{30}{\bar{P}}^{\;\! \mathcolor{gray}{\omega}}_{\;\! \mathcolor{gray}{z} \textcolor{Maroon}{(2)}} ~,  \label{eq:nonlinear(2)-wave_wkrho-simplify6-SVA-E'}
\end{align}
\end{subequations}
若 $\Xint{\mathcolor{gray}{-}}{25}{\bar{\bar{\mathsfit{V}}}}^{\;\! \mathcolor{gray}{\omega}}_{\textcolor{Maroon}{\mathbb{1}}}$ \textcolor{Plum}{可逆}( $\Xint{\mathcolor{gray}{-}}{25}{\bar{\bar{\mathsfit{V}}}}^{\;\! - \mathcolor{gray}{\omega}}_{\textcolor{Maroon}{\mathbb{1}}}$ 存在),\bref{eq:nonlinear(2)-wave_wkrho-simplify6-SVA-E'} 化为标准一阶\textcolor{Plum}{线性}常微分方程组及其解\Footnote{注意区分,被积变量 $\mathcolor{gray}{\mathtt{z}}$ 不同于积分上限 $\mathcolor{gray}{z}$。}
\begin{align}
	\mathcolor{gray}{\nabla_z} \Xint{\mathcolor{gray}{-}}{295}{\bar{E}}^{\;\!\mathcolor{gray}{\omega}}_{\;\! \mathcolor{gray}{z}}
	&= \left( \mathbb{i} \Xint{\begin{smallmatrix} ~ \\ {}^{}_{\mathcolor{gray}{-}} \\ ~ \end{smallmatrix}}{15}{\bar{\bar{k}}}_{\textcolor{Maroon}{\symup{z}}}^{\;\! \mathcolor{gray}{\omega}} - \Xint{\mathcolor{gray}{-}}{25}{\bar{\bar{\mathsfit{V}}}}^{\;\! - \mathcolor{gray}{\omega}}_{\textcolor{Maroon}{\mathbb{1}}} \Xint{\mathcolor{gray}{-}}{32}{\bar{\bar{L}}}^{\;\! \mathcolor{gray}{\omega}} \right) \Xint{\mathcolor{gray}{-}}{295}{\bar{E}}^{\;\!\mathcolor{gray}{\omega}}_{\;\! \mathcolor{gray}{z}} - \Xint{\mathcolor{gray}{-}}{25}{\bar{\bar{\mathsfit{V}}}}^{\;\! - \mathcolor{gray}{\omega}}_{\textcolor{Maroon}{\mathbb{1}}} \Xint{\mathcolor{gray}{-}}{30}{\bar{P}}^{\;\! \mathcolor{gray}{\omega}}_{\;\! \mathcolor{gray}{z} \textcolor{Maroon}{(2)}} \label{eq:simplify6-SVA-E'-V_1nonsingular} \\
	\Xint{\mathcolor{gray}{-}}{295}{\bar{E}}^{\;\!\mathcolor{gray}{\omega}}_{\;\! \mathcolor{gray}{z}}
	&= \mathbb{e}^{\left( \mathbb{i} \Xint{\begin{smallmatrix} ~ \\ {}^{}_{\mathcolor{gray}{-}} \\ ~ \end{smallmatrix}}{15}{\bar{\bar{k}}}_{\textcolor{Maroon}{\symup{z}}}^{\;\! \mathcolor{gray}{\omega}} - \Xint{\mathcolor{gray}{-}}{25}{\bar{\bar{\mathsfit{V}}}}^{\;\! - \mathcolor{gray}{\omega}}_{\textcolor{Maroon}{\mathbb{1}}} \Xint{\mathcolor{gray}{-}}{32}{\bar{\bar{L}}}^{\;\! \mathcolor{gray}{\omega}} \right) \mathcolor{gray}{z}} \left[ \Xint{\mathcolor{gray}{-}}{295}{\bar{E}}^{\;\!\mathcolor{gray}{\omega}}_{\;\! \mathcolor{gray}{0}} - \int_{\mathcolor{gray}{0}}^{\mathcolor{gray}{z}} \mathbb{e}^{- \left( \mathbb{i} \Xint{\begin{smallmatrix} ~ \\ {}^{}_{\mathcolor{gray}{-}} \\ ~ \end{smallmatrix}}{15}{\bar{\bar{k}}}_{\textcolor{Maroon}{\symup{z}}}^{\;\! \mathcolor{gray}{\omega}} - \Xint{\mathcolor{gray}{-}}{25}{\bar{\bar{\mathsfit{V}}}}^{\;\! - \mathcolor{gray}{\omega}}_{\textcolor{Maroon}{\mathbb{1}}} \Xint{\mathcolor{gray}{-}}{32}{\bar{\bar{L}}}^{\;\! \mathcolor{gray}{\omega}} \right) \mathcolor{gray}{\mathtt{z}}} \cdot \Xint{\mathcolor{gray}{-}}{25}{\bar{\bar{\mathsfit{V}}}}^{\;\! - \mathcolor{gray}{\omega}}_{\textcolor{Maroon}{\mathbb{1}}} \Xint{\mathcolor{gray}{-}}{30}{\bar{P}}^{\;\! \mathcolor{gray}{\omega}}_{\;\! \mathcolor{gray}{\mathtt{z}} \textcolor{Maroon}{(2)}} ~\mathbb{d} \mathcolor{gray}{\mathtt{z}} \right] ~, \label{eq:simplify6-SVA-E'-V_1nonsingular-solution}
\end{align}
其中,$\Xint{\mathcolor{gray}{-}}{25}{\bar{E}}^{\;\!\mathcolor{gray}{\omega}}_{\;\! \mathcolor{gray}{0}}$ 由入射面 $\mathcolor{gray}{z \mathcolor{black}{=} 0}$ 处的\textcolor{Maroon}{边界条件} \bref{eq:1BC} 确定,只剩下 $\Xint{\begin{smallmatrix} ~ \\ {}^{}_{\mathcolor{gray}{-}} \\ ~ \end{smallmatrix}}{15}{\bar{\bar{k}}}_{\textcolor{Maroon}{\symup{z}}}^{\;\! \mathcolor{gray}{\omega}}$ 待定。

进一步地,如果 \bref{eq:nonlinear(2)-wave_wkrho-simplify6-L} 成立,\bref{eq:simplify6-SVA-E'-V_1nonsingular} 及其解 \bref{eq:simplify6-SVA-E'-V_1nonsingular-solution} 可简化为
\begin{subequations}
\begin{align}
	\mathcolor{gray}{\nabla_z} \Xint{\mathcolor{gray}{-}}{295}{\bar{E}}^{\;\!\mathcolor{gray}{\omega}}_{\;\! \mathcolor{gray}{z}}
	&= \mathbb{i} \Xint{\begin{smallmatrix} ~ \\ {}^{}_{\mathcolor{gray}{-}} \\ ~ \end{smallmatrix}}{15}{\bar{\bar{k}}}_{\textcolor{Maroon}{\symup{z}}}^{\;\! \mathcolor{gray}{\omega}} \Xint{\mathcolor{gray}{-}}{295}{\bar{E}}^{\;\!\mathcolor{gray}{\omega}}_{\;\! \mathcolor{gray}{z}} - \Xint{\mathcolor{gray}{-}}{25}{\bar{\bar{\mathsfit{V}}}}^{\;\! - \mathcolor{gray}{\omega}}_{\textcolor{Maroon}{\mathbb{1}}} \Xint{\mathcolor{gray}{-}}{30}{\bar{P}}^{\;\! \mathcolor{gray}{\omega}}_{\;\! \mathcolor{gray}{z} \textcolor{Maroon}{(2)}} \label{eq:simplify6-LE0-SVA-E'-V_1nonsingular} \\
	\Xint{\mathcolor{gray}{-}}{295}{\bar{E}}^{\;\!\mathcolor{gray}{\omega}}_{\;\! \mathcolor{gray}{z}}
	&= \mathbb{e}^{\mathbb{i} \Xint{\begin{smallmatrix} ~ \\ {}^{}_{\mathcolor{gray}{-}} \\ ~ \end{smallmatrix}}{15}{\bar{\bar{k}}}_{\textcolor{Maroon}{\symup{z}}}^{\;\! \mathcolor{gray}{\omega}} \mathcolor{gray}{z}} \left[ \Xint{\mathcolor{gray}{-}}{295}{\bar{E}}^{\;\!\mathcolor{gray}{\omega}}_{\;\! \mathcolor{gray}{0}} - \int_{\mathcolor{gray}{0}}^{\mathcolor{gray}{z}} \mathbb{e}^{- \mathbb{i} \Xint{\begin{smallmatrix} ~ \\ {}^{}_{\mathcolor{gray}{-}} \\ ~ \end{smallmatrix}}{15}{\bar{\bar{k}}}_{\textcolor{Maroon}{\symup{z}}}^{\;\! \mathcolor{gray}{\omega}} \mathcolor{gray}{\mathtt{z}}} \cdot \Xint{\mathcolor{gray}{-}}{25}{\bar{\bar{\mathsfit{V}}}}^{\;\! - \mathcolor{gray}{\omega}}_{\textcolor{Maroon}{\mathbb{1}}} \Xint{\mathcolor{gray}{-}}{30}{\bar{P}}^{\;\! \mathcolor{gray}{\omega}}_{\;\! \mathcolor{gray}{\mathtt{z}} \textcolor{Maroon}{(2)}} ~\mathbb{d} \mathcolor{gray}{\mathtt{z}} \right] ~. \label{eq:simplify6-LE0-SVA-E'-V_1nonsingular-solution}
\end{align}
\end{subequations}
同时,\bref{eq:nonlinear(2)-wave_wkrho-simplify6-SVA} 及其解,也简化为(若 $\Xint{\mathcolor{gray}{-}}{25}{\bar{\bar{\mathsfit{V}}}}^{\;\! - \mathcolor{gray}{\omega}}_{\textcolor{Maroon}{\mathbb{1}}}$ 存在):
\begin{subequations} \label{eq:simplify6-LE0-SVA-E'-V_1nonsingular-gg}
\begin{align}
	\mathcolor{gray}{\nabla_z} \Xint{{}^{}_{\mathcolor{gray}{-}}}{10}{\bar{g}}^{\;\!\mathcolor{gray}{\omega}}_{\;\! \mathcolor{gray}{z}}
	&= - \mathbb{e}^{-\mathbb{i} \Xint{\begin{smallmatrix} ~ \\ {}^{}_{\mathcolor{gray}{-}} \\ ~ \end{smallmatrix}}{15}{\bar{\bar{k}}}_{\textcolor{Maroon}{\symup{z}}}^{\;\! \mathcolor{gray}{\omega}} \mathcolor{gray}{z}} \Xint{\mathcolor{gray}{-}}{25}{\bar{\bar{\mathsfit{V}}}}^{\;\! - \mathcolor{gray}{\omega}}_{\textcolor{Maroon}{\mathbb{1}}} \Xint{\mathcolor{gray}{-}}{30}{\bar{P}}^{\;\! \mathcolor{gray}{\omega}}_{\;\! \mathcolor{gray}{z} \textcolor{Maroon}{(2)}} \label{eq:simplify6-LE0-SVA-E'-V_1nonsingular-g} \\
	\Xint{{}^{}_{\mathcolor{gray}{-}}}{10}{\bar{g}}^{\;\!\mathcolor{gray}{\omega}}_{\;\! \mathcolor{gray}{z}}
	&= \Xint{{}^{}_{\mathcolor{gray}{-}}}{10}{\bar{g}}^{\;\!\mathcolor{gray}{\omega}}_{\;\! \mathcolor{gray}{0}} - \int_{\mathcolor{gray}{0}}^{\mathcolor{gray}{z}} \mathbb{e}^{- \mathbb{i} \Xint{\begin{smallmatrix} ~ \\ {}^{}_{\mathcolor{gray}{-}} \\ ~ \end{smallmatrix}}{15}{\bar{\bar{k}}}_{\textcolor{Maroon}{\symup{z}}}^{\;\! \mathcolor{gray}{\omega}} \mathcolor{gray}{\mathtt{z}}} \cdot \Xint{\mathcolor{gray}{-}}{25}{\bar{\bar{\mathsfit{V}}}}^{\;\! - \mathcolor{gray}{\omega}}_{\textcolor{Maroon}{\mathbb{1}}} \Xint{\mathcolor{gray}{-}}{30}{\bar{P}}^{\;\! \mathcolor{gray}{\omega}}_{\;\! \mathcolor{gray}{\mathtt{z}} \textcolor{Maroon}{(2)}} ~\mathbb{d} \mathcolor{gray}{\mathtt{z}} ~,  \label{eq:simplify6-LE0-SVA-E'-V_1nonsingular-solution-g}
\end{align}
\end{subequations}
对比 \bref{eq:simplify6-LE0-SVA-E'-V_1nonsingular-solution-g} 和 \bref{eq:simplify6-LE0-SVA-E'-V_1nonsingular-solution},可以发现 $\Xint{{}^{}_{\mathcolor{gray}{-}}}{10}{\bar{g}}^{\;\!\mathcolor{gray}{\omega}}_{\;\! \mathcolor{gray}{0}} = \Xint{\mathcolor{gray}{-}}{25}{\bar{E}}^{\;\!\mathcolor{gray}{\omega}}_{\;\! \mathcolor{gray}{0}}$,以及重新发现 \bref{eq:vec-matrix_exp}。

如果 \bref{eq:nonlinear(2)-wave_wkrho-simplify6-L} 不成立,作为 \bref{eq:simplify6-LE0-SVA-E'-V_1nonsingular-solution} 的扩展, \bref{eq:simplify6-SVA-E'-V_1nonsingular-solution} 也应给出类似 \bref{eq:vec-matrix_exp} 的\textcolor{Plum}{高级抽象}表达式,以及 \bref{eq:simplify6-LE0-SVA-E'-V_1nonsingular-solution-g} 的对应物。那么有
\begin{subequations} \label{eq:simplify6-SVA-E'-V_1nonsingular-Abstract_Exp}
\begin{align}
	\Xint{\mathcolor{gray}{-}}{295}{\bar{E}}^{\;\!\mathcolor{gray}{\omega}}_{\;\! \mathcolor{gray}{z}}
	&= \mathbb{e}^{\left( \mathbb{i} \Xint{\begin{smallmatrix} ~ \\ {}^{}_{\mathcolor{gray}{-}} \\ ~ \end{smallmatrix}}{15}{\bar{\bar{k}}}_{\textcolor{Maroon}{\symup{z}}}^{\;\! \mathcolor{gray}{\omega}} - \Xint{\mathcolor{gray}{-}}{25}{\bar{\bar{\mathsfit{V}}}}^{\;\! - \mathcolor{gray}{\omega}}_{\textcolor{Maroon}{\mathbb{1}}} \Xint{\mathcolor{gray}{-}}{32}{\bar{\bar{L}}}^{\;\! \mathcolor{gray}{\omega}} \right) \mathcolor{gray}{z}} \Xint{{}^{}_{\mathcolor{gray}{-}}}{10}{\bar{g}}^{\;\!\mathcolor{gray}{\omega}}_{\;\! \mathcolor{gray}{z}} ~, \label{eq:simplify6-SVA-E'-V_1nonsingular-abstract_exp} \\
	\Xint{{}^{}_{\mathcolor{gray}{-}}}{10}{\bar{g}}^{\;\!\mathcolor{gray}{\omega}}_{\;\! \mathcolor{gray}{z}}
	&= \Xint{{}^{}_{\mathcolor{gray}{-}}}{10}{\bar{g}}^{\;\!\mathcolor{gray}{\omega}}_{\;\! \mathcolor{gray}{0}} - \int_{\mathcolor{gray}{0}}^{\mathcolor{gray}{z}} \mathbb{e}^{- \left( \mathbb{i} \Xint{\begin{smallmatrix} ~ \\ {}^{}_{\mathcolor{gray}{-}} \\ ~ \end{smallmatrix}}{15}{\bar{\bar{k}}}_{\textcolor{Maroon}{\symup{z}}}^{\;\! \mathcolor{gray}{\omega}} - \Xint{\mathcolor{gray}{-}}{25}{\bar{\bar{\mathsfit{V}}}}^{\;\! - \mathcolor{gray}{\omega}}_{\textcolor{Maroon}{\mathbb{1}}} \Xint{\mathcolor{gray}{-}}{32}{\bar{\bar{L}}}^{\;\! \mathcolor{gray}{\omega}} \right) \mathcolor{gray}{\mathtt{z}}} \cdot \Xint{\mathcolor{gray}{-}}{25}{\bar{\bar{\mathsfit{V}}}}^{\;\! - \mathcolor{gray}{\omega}}_{\textcolor{Maroon}{\mathbb{1}}} \Xint{\mathcolor{gray}{-}}{30}{\bar{P}}^{\;\! \mathcolor{gray}{\omega}}_{\;\! \mathcolor{gray}{\mathtt{z}} \textcolor{Maroon}{(2)}} ~\mathbb{d} \mathcolor{gray}{\mathtt{z}} ~, \label{eq:simplify6-SVA-E'-V_1nonsingular-abstract_exp-gz}
\end{align}
\end{subequations}
其中,\bref{eq:simplify6-SVA-E'-V_1nonsingular-abstract_exp} 与 \bref{eq:vec-matrix_exp} 对应,\bref{eq:simplify6-SVA-E'-V_1nonsingular-abstract_exp-gz} 与 \bref{eq:simplify6-LE0-SVA-E'-V_1nonsingular-solution-g} 对应。

若所考虑的电场组分,不参与晶体内的\textcolor{Plum}{非线性}\textcolor{NavyBlue}{光学}/\textcolor{gray}{频率转换}过程,只在晶体中\textcolor{NavyBlue}{无源}且\textcolor{NavyBlue}{被动}地\textcolor{Plum}{线性}衍射、被晶体\textcolor{Plum}{线性}\textcolor{NavyBlue}{吸收}/\textcolor{NavyBlue}{放大}等,则 \bref{eq:nonlinear(2)-wave_wkrho-simplify6} 退化为无\textcolor{Plum}{非线性}\textcolor{NavyBlue}{驱动源}的形式,并且容易观察到\textcolor{Plum}{非线性}算子 $\Xint{\mathcolor{gray}{-}}{21}{\bar{\bar{V}}}^{\;\! \mathcolor{gray}{\omega}}$ 理应随着\textcolor{Plum}{非线性}\textcolor{NavyBlue}{驱动源}一起消失,因为此时 $\Xint{{}^{}_{\mathcolor{gray}{-}}}{10}{\bar{g}}^{\;\!\mathcolor{gray}{\omega}}_{\;\! \mathcolor{gray}{z}} \equiv \Xint{{}^{}_{\mathcolor{gray}{-}}}{10}{\bar{g}}^{\;\!\mathcolor{gray}{\omega}}$ 不再含 $\mathcolor{gray}{z}$,以至于 $\Xint{\mathcolor{gray}{-}}{21}{\bar{\bar{V}}}^{\;\! \mathcolor{gray}{\omega}} \Xint{{}^{}_{\mathcolor{gray}{-}}}{10}{\bar{g}}^{\;\!\mathcolor{gray}{\omega}}_{\;\! \mathcolor{gray}{z}} \equiv \Xint{\mathcolor{gray}{-}}{21}{\bar{\bar{V}}}^{\;\! \mathcolor{gray}{\omega}} \Xint{{}^{}_{\mathcolor{gray}{-}}}{10}{\bar{g}}^{\;\!\mathcolor{gray}{\omega}} = \bar{0}$,则有
\begin{subequations}
\begin{align}
	\Xint{\mathcolor{gray}{-}}{32}{\bar{\bar{L}}}^{\;\! \mathcolor{gray}{\omega}} \Xint{\mathcolor{gray}{-}}{295}{\bar{E}}^{\;\!\mathcolor{gray}{\omega}}_{\;\! \mathcolor{gray}{z}}
	&= \bar{0} \label{eq:nonlinear(2)-wave_wkrho-simplify6-L} \\
	\left( \bar{\bar{\varepsilon}}^{\;\! \mathcolor{gray}{\omega}}_{\textcolor{Maroon}{(1)}} + \bar{\bar{\zeta}}^{\;\! \mathcolor{gray}{\omega} \mathcolor{gray}{\check{1}}}_{\textcolor{Maroon}{(1)}} \mathbb{i} \Xint{\begin{smallmatrix} ~ \\ {}^{}_{\mathcolor{gray}{-}} \\ ~ \end{smallmatrix}}{15}{\bar{\bar{k}}}_{\;\! \mathcolor{gray}{\check{1}}}^{\;\! \mathcolor{gray}{\omega}} - \bar{\bar{\zeta}}^{\;\! \mathcolor{gray}{\omega} \mathcolor{gray}{\check{1} \check{2}}}_{\textcolor{Maroon}{(1)}} \Xint{\begin{smallmatrix} ~ \\ {}^{}_{\mathcolor{gray}{-}} \\ ~ \end{smallmatrix}}{15}{\bar{\bar{k}}}_{\;\! \mathcolor{gray}{\check{2}}}^{\;\! \mathcolor{gray}{\omega}} \Xint{\begin{smallmatrix} ~ \\ {}^{}_{\mathcolor{gray}{-}} \\ ~ \end{smallmatrix}}{15}{\bar{\bar{k}}}_{\;\! \mathcolor{gray}{\check{1}}}^{\;\! \mathcolor{gray}{\omega}} \right) \mathbb{e}^{\mathbb{i} \Xint{\begin{smallmatrix} ~ \\ {}^{}_{\mathcolor{gray}{-}} \\ ~ \end{smallmatrix}}{15}{\bar{\bar{k}}}_{\textcolor{Maroon}{\symup{z}}}^{\;\! \mathcolor{gray}{\omega}} \mathcolor{gray}{z}} \Xint{{}^{}_{\mathcolor{gray}{-}}}{10}{\bar{g}}^{\;\!\mathcolor{gray}{\omega}}_{\;\! \mathcolor{gray}{z}} &= \bar{0}  \label{eq:nonlinear(2)-wave_wkrho-simplify6-L2} \\
	\textcolor{Plum}{\det} \left[ \Xint{\mathcolor{gray}{-}}{32}{\bar{\bar{L}}}^{\;\! \mathcolor{gray}{\omega}} \right] = \textcolor{Plum}{\det} \left[ \bar{\bar{\varepsilon}}^{\;\! \mathcolor{gray}{\omega}}_{\textcolor{Maroon}{(1)}} + \bar{\bar{\zeta}}^{\;\! \mathcolor{gray}{\omega} \mathcolor{gray}{\check{1}}}_{\textcolor{Maroon}{(1)}} \mathbb{i} \Xint{\begin{smallmatrix} ~ \\ {}^{}_{\mathcolor{gray}{-}} \\ ~ \end{smallmatrix}}{15}{\bar{\bar{k}}}_{\;\! \mathcolor{gray}{\check{1}}}^{\;\! \mathcolor{gray}{\omega}} - \bar{\bar{\zeta}}^{\;\! \mathcolor{gray}{\omega} \mathcolor{gray}{\check{1} \check{2}}}_{\textcolor{Maroon}{(1)}} \Xint{\begin{smallmatrix} ~ \\ {}^{}_{\mathcolor{gray}{-}} \\ ~ \end{smallmatrix}}{15}{\bar{\bar{k}}}_{\;\! \mathcolor{gray}{\check{2}}}^{\;\! \mathcolor{gray}{\omega}} \Xint{\begin{smallmatrix} ~ \\ {}^{}_{\mathcolor{gray}{-}} \\ ~ \end{smallmatrix}}{15}{\bar{\bar{k}}}_{\;\! \mathcolor{gray}{\check{1}}}^{\;\! \mathcolor{gray}{\omega}} \right] &= 0 ~, \label{eq:nonlinear(2)-wave_wkrho-simplify6-L3}
\end{align}
\end{subequations}
其中,要想 \bref{eq:nonlinear(2)-wave_wkrho-simplify6-L2} 有非零解 $\Xint{{}^{}_{\mathcolor{gray}{-}}}{10}{\bar{g}}^{\;\!\mathcolor{gray}{\omega}}$,根据 \textcolor{Maroon}{Cramer's rule},其前面的所有矩阵,组成的系数行列式需为零。然而,由于 \textcolor{Maroon}{矩阵指数} 总是存在且\textcolor{Plum}{可逆},其行列式 $\textcolor{Plum}{\det} \left[ \mathbb{e}^{\mathbb{i} \Xint{\begin{smallmatrix} ~ \\ {}^{}_{\mathcolor{gray}{-}} \\ ~ \end{smallmatrix}}{15}{\bar{\bar{k}}}_{\textcolor{Maroon}{\symup{z}}}^{\;\! \mathcolor{gray}{\omega}} \mathcolor{gray}{z}} \right] \neq 0$ 必不可能为零,所以只可能有 \bref{eq:nonlinear(2)-wave_wkrho-simplify6-L3} 中的 $\textcolor{Plum}{\det} \left[ \Xint{\mathcolor{gray}{-}}{30}{\bar{\bar{L}}}^{\;\! \mathcolor{gray}{\omega}} \right]$ 为零。--- 然而,\bref{eq:nonlinear(2)-wave_wkrho-simplify6-L3} 提供的约束仍然太少(只有一个),属于欠定系统,不足以唯一确定地求解 $\Xint{\begin{smallmatrix} ~ \\ {}^{}_{\mathcolor{gray}{-}} \\ ~ \end{smallmatrix}}{15}{\bar{\bar{k}}}_{\textcolor{Maroon}{\symup{z}}}^{\;\! \mathcolor{gray}{\omega}}$ 的全部 9 个元素。因此,从另一个角度,即对于任意的(初始)矢量场 $\Xint{{}^{}_{\mathcolor{gray}{-}}}{10}{\bar{g}}^{\;\!\mathcolor{gray}{\omega}}$或 $\Xint{\mathcolor{gray}{-}}{25}{\bar{E}}^{\;\!\mathcolor{gray}{\omega}}_{\;\! \mathcolor{gray}{z}}$,\bref{eq:nonlinear(2)-wave_wkrho-simplify6-L2} 都应该成立\cite{sturmElectromagneticWavesCrystals2024},这使得应恒有
\begin{align} \label{eq:nonlinear(2)-wave_wkrho-simplify6-L4}
	\Xint{\mathcolor{gray}{-}}{32}{\bar{\bar{L}}}^{\;\! \mathcolor{gray}{\omega}}
	= \bar{\bar{\varepsilon}}^{\;\! \mathcolor{gray}{\omega}}_{\textcolor{Maroon}{(1)}} + \bar{\bar{\zeta}}^{\;\! \mathcolor{gray}{\omega} \mathcolor{gray}{\check{1}}}_{\textcolor{Maroon}{(1)}} \mathbb{i} \Xint{\begin{smallmatrix} ~ \\ {}^{}_{\mathcolor{gray}{-}} \\ ~ \end{smallmatrix}}{15}{\bar{\bar{k}}}_{\;\! \mathcolor{gray}{\check{1}}}^{\;\! \mathcolor{gray}{\omega}} - \bar{\bar{\zeta}}^{\;\! \mathcolor{gray}{\omega} \mathcolor{gray}{\check{1} \check{2}}}_{\textcolor{Maroon}{(1)}} \Xint{\begin{smallmatrix} ~ \\ {}^{}_{\mathcolor{gray}{-}} \\ ~ \end{smallmatrix}}{15}{\bar{\bar{k}}}_{\;\! \mathcolor{gray}{\check{2}}}^{\;\! \mathcolor{gray}{\omega}} \Xint{\begin{smallmatrix} ~ \\ {}^{}_{\mathcolor{gray}{-}} \\ ~ \end{smallmatrix}}{15}{\bar{\bar{k}}}_{\;\! \mathcolor{gray}{\check{1}}}^{\;\! \mathcolor{gray}{\omega}} &= \bar{\bar{0}}~,
\end{align}
这是一组由 9 个方程耦合而成的 9 元 2 次方程组。对它进行求解仍然是复杂的\cite{sturmElectromagneticWavesCrystals2024}。将满足 \bref{eq:nonlinear(2)-wave_wkrho-simplify6-L3,eq:nonlinear(2)-wave_wkrho-simplify6-L4} 的 $\Xint{\begin{smallmatrix} ~ \\ {}^{}_{\mathcolor{gray}{-}} \\ ~ \end{smallmatrix}}{15}{\bar{\bar{k}}}_{\textcolor{Maroon}{\symup{z}}}^{\;\! \mathcolor{gray}{\omega}}$ 代入 \bref{eq:vec-matrix_exp},即得到解 $\Xint{\mathcolor{gray}{-}}{25}{\bar{E}}^{\;\!\mathcolor{gray}{\omega}}_{\;\! \mathcolor{gray}{z}} = \mathbb{e}^{\mathbb{i} \Xint{\begin{smallmatrix} ~ \\ {}^{}_{\mathcolor{gray}{-}} \\ ~ \end{smallmatrix}}{15}{\bar{\bar{k}}}_{\textcolor{Maroon}{\symup{z}}}^{\;\! \mathcolor{gray}{\omega}} \mathcolor{gray}{z}} \Xint{{}^{}_{\mathcolor{gray}{-}}}{10}{\bar{g}}^{\;\!\mathcolor{gray}{\omega}}$。其中,$\Xint{\begin{smallmatrix} ~ \\ {}^{}_{\mathcolor{gray}{-}} \\ ~ \end{smallmatrix}}{15}{\bar{\bar{k}}}_{\textcolor{Maroon}{\symup{z}}}^{\;\! \mathcolor{gray}{\omega}}$ 包含了整套\textcolor{PineGreen}{特征系统},即所有的广义(子空间)\textcolor{PineGreen}{特征值}-\textcolor{PineGreen}{特征向量}对\cite{xieAnalytic3DVector}。

若所考虑的电场组分,参与材料内的\textcolor{Plum}{非线性}\textcolor{NavyBlue}{光学}/\textcolor{gray}{频率转换}过程,则所有的这些\textcolor{gray}{频率组分} $\{ \mathcolor{gray}{\omega} \}$\Footnote{既可能是\textcolor{Plum}{连续}谱 $\{ \mathcolor{gray}{\omega} \}$,也可能是离散谱 $\{ \mathcolor{gray}{\omega}_{\textcolor{Maroon}{i}} \}$。而且原则上应不缩写,即该扩写作 $\{ \Xint{\mathcolor{gray}{-}}{25}{\bar{E}}^{\;\!\mathcolor{gray}{\omega}}_{\;\! \mathcolor{gray}{z}} \}$ 或 $\{ \Xint{\mathcolor{gray}{-}}{25}{\bar{E}}_{\;\! \textcolor{Maroon}{i} \mathcolor{gray}{z}} := \Xint{\mathcolor{gray}{-}}{25}{\bar{E}}^{\;\!\mathcolor{gray}{\omega}_{\textcolor{Maroon}{i}}}_{\;\! \mathcolor{gray}{z}} \}$。},都满足 \bref{eq:nonlinear(2)-wave_wkrho-simplify6},不论是方程右侧的驱动场,还是等号左侧的产生场。更进一步地,这些 $\{ \mathcolor{gray}{\omega} \}$ 还应在此基础上,继续满足 \bref{eq:nonlinear(2)-wave_wkrho-simplify6-L},否则 \bref{eq:simplify6-SVA-E'-V_1nonsingular-Abstract_Exp} 中的 $\Xint{\begin{smallmatrix} ~ \\ {}^{}_{\mathcolor{gray}{-}} \\ ~ \end{smallmatrix}}{15}{\bar{\bar{k}}}_{\textcolor{Maroon}{\symup{z}}}^{\;\! \mathcolor{gray}{\omega}}$ 始终是个未知量。这意味着需要做出以下假设:
\begin{subequations}
\begin{align}
	\textbf{\text{晶体内所有场的\textcolor{PineGreen}{\textcolor{gray}{单色}平面波}分量(不论是在\textcolor{NavyBlue}{无源}\textcolor{Plum}{齐次}\textcolor{NavyBlue}{被动}\textcolor{Plum}{线性}衍射}} \label{eq:L+V-decompose} \\ 
	\textbf{\text{还是通过\textcolor{NavyBlue}{有源}\textcolor{Plum}{非齐次}\textcolor{NavyBlue}{主动}\textcolor{Plum}{非线性}产生)均由本征模或其\textcolor{PineGreen}{叠加}构成}}~,
\end{align}
\end{subequations}
强场瞬态情况下可能短暂偏离,但最终仍应趋向于\textcolor{PineGreen}{本征模}主导的稳态解。--- 然而,值得怀疑的是,该假设在弱场瞬态的情况下仍然成立,因为 \bref{eq:nonlinear(2)-wave_wkrho-simplify6-L} 在\textcolor{gray}{角频率}/\textcolor{gray}{波长}域中运行,与时域无关,也就与稳态/瞬态无关\Footnote{该结论的成立条件,需要进一步调研和研究,可能与\textcolor{NavyBlue}{基态}/\textcolor{NavyBlue}{激发态}、\textcolor{NavyBlue}{共振}/\textcolor{NavyBlue}{非共振}有关。}。\bref{eq:L+V-decompose} 即预设了 \bref{eq:nonlinear(2)-wave_wkrho-simplify6-L} 的普遍成立。这也意味着 \bref{eq:L+V-decompose,eq:nonlinear(2)-wave_wkrho-simplify6-L} 在\textcolor{Plum}{非线性}\textcolor{NavyBlue}{光学}层面的另一种等价表述:
\begin{subequations}
\begin{align}
	\textbf{\text{晶体内所有已经、正在、即将生成的场,无论是否已经、正在}} \label{eq:L+V-decompose2} \\ 
	\textbf{\text{、即将开始\textcolor{Plum}{线性}传播/衍射,均受到\textcolor{Plum}{线性}\textcolor{PineGreen}{晶体光学}的本征模约束}}~,
\end{align}
\end{subequations}
而一旦上述结论成立,\bref{eq:nonlinear(2)-wave_wkrho-simplify6} 减去 \bref{eq:nonlinear(2)-wave_wkrho-simplify6-L} 便得到\textcolor{Maroon}{矢量场} $\Xint{{}^{}_{\mathcolor{gray}{-}}}{10}{\bar{g}}^{\;\!\mathcolor{gray}{\omega}}_{\;\! \mathcolor{gray}{z}}$ 所满足的,二阶\textcolor{Plum}{纵向}\textcolor{Plum}{非线性}偏微分、二维\textcolor{Plum}{横向}场\textcolor{Plum}{积分方程}\cite{ponomarevAsymptoticSolutionConvolution2021}
\begin{subequations}
\begin{align}
	\Xint{\mathcolor{gray}{-}}{25}{\bar{\bar{V}}}^{\;\! \mathcolor{gray}{\omega}} \Xint{{}^{}_{\mathcolor{gray}{-}}}{10}{\bar{g}}^{\;\!\mathcolor{gray}{\omega}}_{\;\! \mathcolor{gray}{z}}
	&= - \Xint{{}^{}_{\mathcolor{gray}{-}}}{23}{\bar{\bar{\bar{\chi}}}}^{\;\! \mathcolor{gray}{\omega}}_{\mathcolor{gray}{z} \textcolor{Maroon}{(2)}} ~{}^{\mathcolor{gray}{*}}_{\mathcolor{gray}{*}} \left( \Xint{\mathcolor{gray}{-}}{295}{\bar{E}}^{\;\!\mathcolor{gray}{\omega}}_{\;\! \mathcolor{gray}{z}} ~\mathcolor{gray}{\widetilde \circledast}~ \Xint{\mathcolor{gray}{-}}{295}{\bar{E}}^{\;\!\mathcolor{gray}{\omega}}_{\;\! \mathcolor{gray}{z}} \right)  \label{eq:nonlinear(2)-wave_wkrho-simplify6-V} \\
	\left[ \left( \bar{\bar{\zeta}}^{\;\! \mathcolor{gray}{\omega} \mathcolor{gray}{\symup{z}}}_{\textcolor{Maroon}{(1)}} + \bar{\bar{\zeta}}^{\;\! \mathcolor{gray}{\omega} \mathcolor{gray}{\check{1} \symup{z}}}_{\textcolor{Maroon}{(1)}} \mathbb{i} \Xint{\begin{smallmatrix} ~ \\ {}^{}_{\mathcolor{gray}{-}} \\ ~ \end{smallmatrix}}{15}{\bar{\bar{k}}}_{\;\! \mathcolor{gray}{\check{1}}}^{\;\! \mathcolor{gray}{\omega}} \right.\right.&+\left.\left. \bar{\bar{\zeta}}^{\;\! \mathcolor{gray}{\omega} \mathcolor{gray}{\symup{z} \check{2}}}_{\textcolor{Maroon}{(1)}} \mathbb{i} \Xint{\begin{smallmatrix} ~ \\ {}^{}_{\mathcolor{gray}{-}} \\ ~ \end{smallmatrix}}{15}{\bar{\bar{k}}}_{\;\! \mathcolor{gray}{\check{2}}}^{\;\! \mathcolor{gray}{\omega}} \right) \mathbb{e}^{\mathbb{i} \Xint{\begin{smallmatrix} ~ \\ {}^{}_{\mathcolor{gray}{-}} \\ ~ \end{smallmatrix}}{15}{\bar{\bar{k}}}_{\textcolor{Maroon}{\symup{z}}}^{\;\! \mathcolor{gray}{\omega}} \mathcolor{gray}{z}} \mathcolor{gray}{\nabla_z} + \bar{\bar{\zeta}}^{\;\! \mathcolor{gray}{\omega} \mathcolor{gray}{\symup{z} \symup{z}}}_{\textcolor{Maroon}{(1)}} \mathbb{e}^{\mathbb{i} \Xint{\begin{smallmatrix} ~ \\ {}^{}_{\mathcolor{gray}{-}} \\ ~ \end{smallmatrix}}{15}{\bar{\bar{k}}}_{\textcolor{Maroon}{\symup{z}}}^{\;\! \mathcolor{gray}{\omega}} \mathcolor{gray}{z}} \mathcolor{gray}{\nabla_z^2} \right] \Xint{{}^{}_{\mathcolor{gray}{-}}}{10}{\bar{g}}^{\;\!\mathcolor{gray}{\omega}}_{\;\! \mathcolor{gray}{z}} \label{eq:nonlinear(2)-wave_wkrho-simplify6-V2} \\
	&= - \Xint{{}^{}_{\mathcolor{gray}{-}}}{23}{\bar{\bar{\bar{\chi}}}}^{\;\! \mathcolor{gray}{\omega}}_{\mathcolor{gray}{z} \textcolor{Maroon}{(2)}} ~{}^{\mathcolor{gray}{*}}_{\mathcolor{gray}{*}} \left[ \left( \mathbb{e}^{\mathbb{i} \Xint{\begin{smallmatrix} ~ \\ {}^{}_{\mathcolor{gray}{-}} \\ ~ \end{smallmatrix}}{15}{\bar{\bar{k}}}_{\textcolor{Maroon}{\symup{z}}}^{\;\! \mathcolor{gray}{\omega}} \mathcolor{gray}{z}} \Xint{{}^{}_{\mathcolor{gray}{-}}}{10}{\bar{g}}^{\;\!\mathcolor{gray}{\omega}}_{\;\! \mathcolor{gray}{z}} \right) ~\mathcolor{gray}{\widetilde \circledast}~ \left( \mathbb{e}^{\mathbb{i} \Xint{\begin{smallmatrix} ~ \\ {}^{}_{\mathcolor{gray}{-}} \\ ~ \end{smallmatrix}}{15}{\bar{\bar{k}}}_{\textcolor{Maroon}{\symup{z}}}^{\;\! \mathcolor{gray}{\omega}} \mathcolor{gray}{z}} \Xint{{}^{}_{\mathcolor{gray}{-}}}{10}{\bar{g}}^{\;\!\mathcolor{gray}{\omega}}_{\;\! \mathcolor{gray}{z}} \right) \right] \label{eq:nonlinear(2)-wave_wkrho-simplify6-V3} ~,
\end{align}
\end{subequations}
这是一个\textcolor{Plum}{卷积型微-积分}方程组。综合来看,晶体内任意\textcolor{Plum}{局域}二阶\textcolor{Plum}{非线性}\textcolor{NavyBlue}{光学}过程中,每个\textcolor{gray}{单色}行波均须同时满足以下三个方程中的任意两者:\bref{eq:nonlinear(2)-wave_wkrho-simplify6,eq:nonlinear(2)-wave_wkrho-simplify6-L,eq:nonlinear(2)-wave_wkrho-simplify6-V},且一般使用后两者。这三种“\textcolor{PineGreen}{晶体光学}”条件的另一种更“\textcolor{Plum}{数学}\textcolor{NavyBlue}{物理}”的表述是
\begin{align}
	\text{\textbf{\textcolor{Plum}{非线性}\textcolor{NavyBlue}{光学}\textcolor{PineGreen}{本征系统},视为\textcolor{Plum}{线性}\textcolor{NavyBlue}{光学}\textcolor{PineGreen}{本征系统}的 \textcolor{NavyBlue}{0 阶微扰}}} \label{eq:L+V-decompose3}~.
\end{align}

事实上,在一些(简单但基本的:如后文重点关注的\textcolor{PineGreen}{纯电各向异性}介质的)情况下,\bref{eq:matrix_exp-V1,eq:matrix_exp-V2} 的秩全都降至 2,甚至最末行的行向量双双降为 $\bar{0}^{\mathsf{\textcolor{Plum}{T}}}$,这使得 \bref{eq:nonlinear(2)-wave_wkrho-simplify6-V2} 不成立(左侧 $\symup{z}$ 分量为零,右侧却非零),并因此宣告了 \bref{eq:L+V-decompose,eq:L+V-decompose2,eq:L+V-decompose3,eq:nonlinear(2)-wave_wkrho-simplify6-V} 这四个假设的失效。这时,最近邻成立的公式,即待求解的对象,回滚至 \bref{eq:nonlinear(2)-wave_wkrho-simplify6-SVA-E,eq:nonlinear(2)-wave_wkrho-simplify6-SVA},其中有 2 个未知量 $\Xint{\begin{smallmatrix} ~ \\ {}^{}_{\mathcolor{gray}{-}} \\ ~ \end{smallmatrix}}{15}{\bar{\bar{k}}}_{\textcolor{Maroon}{\symup{z}}}^{\;\! \mathcolor{gray}{\omega}}$ 和场 $\Xint{{}^{}_{\mathcolor{gray}{-}}}{10}{\bar{g}}^{\;\!\mathcolor{gray}{\omega}}_{\;\! \mathcolor{gray}{z}}$ 或 $\Xint{\mathcolor{gray}{-}}{25}{\bar{E}}^{\;\!\mathcolor{gray}{\omega}}_{\;\! \mathcolor{gray}{z}}$。

此刻,唯一可行的解决方案,只剩下采用\textcolor{NavyBlue}{量子力学}的传统方法:对所有算子、算子的\textcolor{PineGreen}{本征系统},进行\textcolor{NavyBlue}{非简并微扰展开},才有希望使 {\one} \bref{eq:nonlinear(2)-wave_wkrho-simplify6-V} 的 \textcolor{NavyBlue}{1 阶微扰}修订版,在 $\textcolor{Plum}{\det} \left[ \Xint{\mathcolor{gray}{-}}{25}{\bar{\bar{\mathsfit{V}}}}^{\;\! \mathcolor{gray}{\omega}}_{\textcolor{Maroon}{\mathbb{1}}} \right] = 0$ 和/或 $\textcolor{Plum}{\det} \left[ \Xint{\mathcolor{gray}{-}}{25}{\bar{\bar{\mathsfit{V}}}}^{\;\! \mathcolor{gray}{\omega}}_{\textcolor{Maroon}{\mathbb{2}}} \right] = 0$ 的情况下有解、可解(而不是无解);{\two} \textcolor{Plum}{非线性}\textcolor{gray}{混频}生成的场的\textcolor{PineGreen}{本征系统},不再沿用 \bref{eq:nonlinear(2)-wave_wkrho-simplify6-L} 的解,而采取以之为基础的 \textcolor{NavyBlue}{1 阶微扰}修订版(但同时又保留 \textcolor{NavyBlue}{0 阶} $=$ \textcolor{NavyBlue}{无微扰}时的 \bref{eq:nonlinear(2)-wave_wkrho-simplify6-L} 成立),此时不再需要 \bref{eq:L+V-decompose,eq:L+V-decompose2,eq:nonlinear(2)-wave_wkrho-simplify6-V} 这四个相互等价的假设成立。

\clearpage
\vspace*{-7.5em}

%\vspace*{-3.5em}

\marginLeft[-2.4em]{ssec:Exp-solution-linear}\subsection{线性均匀非局域 $\bar{E}$ 波动方程的矩阵指数解}\label{ssec:Exp-solution-linear}

现假设已得到了上述 \bref{ssec:Exp-waveq} \textcolor{NavyBlue}{无源}波动方程 \bref{eq:nonlinear(2)-wave_wkrho-simplify6-L} 的\textcolor{PineGreen}{本征解} $\Xint{\begin{smallmatrix} ~ \\ {}^{}_{\mathcolor{gray}{-}} \\ ~ \end{smallmatrix}}{15}{\bar{\bar{k}}}_{\textcolor{Maroon}{\symup{z}}}^{\;\! \mathcolor{gray}{\omega}}$。一般来说,满足 \bref{eq:nonlinear(2)-wave_wkrho-simplify6-L3,eq:nonlinear(2)-wave_wkrho-simplify6-L4} 的 $\Xint{\begin{smallmatrix} ~ \\ {}^{}_{\mathcolor{gray}{-}} \\ ~ \end{smallmatrix}}{15}{\bar{\bar{k}}}_{\textcolor{Maroon}{\symup{z}}}^{\;\! \mathcolor{gray}{\omega}} \in \bar{\bar{\mathbb{C}}}_{\textcolor{Plum}{\left[3 \times 3\right]}}$ 可能是个缺陷矩阵\Footnote{非标量近零\textcolor{PineGreen}{折射率}材料还可能涉及\textcolor{Plum}{病态}/\textcolor{Plum}{奇异}矩阵,此时 $\Xint{\begin{smallmatrix} ~ \\ {}^{}_{\mathcolor{gray}{-}} \\ ~ \end{smallmatrix}}{15}{\bar{\bar{k}}}_{\textcolor{Maroon}{\symup{z}}}^{\;\! \mathcolor{gray}{\omega}}$ 的行列式等于或接近零、矩阵(接近)\textcolor{Plum}{不可逆}。此外,缺陷矩阵的特征(向量所张成的)空间,所对应的 \bref{eq:kz-non_diagonalization} 中的 $\overline{\Xint{\begin{smallmatrix} ~ \\ {}^{}_{\mathcolor{gray}{-}} \\ ~ \end{smallmatrix}}{15}{\bar{v}}^{\;\!\mathcolor{gray}{\omega}}_{\textcolor{PineGreen}{\jmath}}}^{\mathsf{\textcolor{Plum}{T}}}$,也是\textcolor{Plum}{病态}/\textcolor{Plum}{奇异}矩阵。}\cite{xieAnalytic3DVector},此时,$\Xint{\begin{smallmatrix} ~ \\ {}^{}_{\mathcolor{gray}{-}} \\ ~ \end{smallmatrix}}{15}{\bar{\bar{k}}}_{\textcolor{Maroon}{\symup{z}}}^{\;\! \mathcolor{gray}{\omega}}$ 含有(接近)简并的\textcolor{PineGreen}{特征向量}。也即这个同时是材料系数 $\bar{\bar{\varepsilon}}^{\;\! \mathcolor{gray}{\omega}}_{\textcolor{Maroon}{(1)}}, \bar{\bar{\zeta}}^{\;\! \mathcolor{gray}{\omega} \mathcolor{gray}{\check{1}}}_{\textcolor{Maroon}{(1)}}, \bar{\bar{\zeta}}^{\;\! \mathcolor{gray}{\omega} \mathcolor{gray}{\check{1} \check{2}}}_{\textcolor{Maroon}{(1)}}$ 和\textcolor{PineGreen}{波矢}分量 $\mathcolor{gray}{k_{\symup{x}}}, \mathcolor{gray}{k_{\symup{y}}}$ 的函数的\textcolor{Plum}{各向异性}(复)矩阵,可能无法(被)\textcolor{Plum}{对角化}\Footnote{其中,通过给列向量整体的头上加一根长线,定义了\textcolor{Plum}{广义列向量} $\overline{\Xint{\begin{smallmatrix} ~ \\ {}^{}_{\mathcolor{gray}{-}} \\ ~ \end{smallmatrix}}{22}{\bar{\lambda}}^{\;\!\mathcolor{gray}{\omega}}_{\textcolor{PineGreen}{\jmath}}}^{\mathsf{\textcolor{Plum}{T}}} := \begin{pmatrix} \Xint{\begin{smallmatrix} ~ \\ {}^{}_{\mathcolor{gray}{-}} \\ ~ \end{smallmatrix}}{22}{\bar{\lambda}}^{\;\!\mathcolor{gray}{\omega}}_{\textcolor{PineGreen}{1}}, \Xint{\begin{smallmatrix} ~ \\ {}^{}_{\mathcolor{gray}{-}} \\ ~ \end{smallmatrix}}{22}{\bar{\lambda}}^{\;\!\mathcolor{gray}{\omega}}_{\textcolor{PineGreen}{2}}, \Xint{\begin{smallmatrix} ~ \\ {}^{}_{\mathcolor{gray}{-}} \\ ~ \end{smallmatrix}}{22}{\bar{\lambda}}^{\;\!\mathcolor{gray}{\omega}}_{\textcolor{PineGreen}{3}} \end{pmatrix} := \textcolor{PineGreen}{\text{diag}} \left[ \Xint{\begin{smallmatrix} ~ \\ {}^{}_{\mathcolor{gray}{-}} \\ ~ \end{smallmatrix}}{22}{\lambda}^{\;\!\mathcolor{gray}{\omega}}_{\textcolor{PineGreen}{1}}, \Xint{\begin{smallmatrix} ~ \\ {}^{}_{\mathcolor{gray}{-}} \\ ~ \end{smallmatrix}}{22}{\lambda}^{\;\!\mathcolor{gray}{\omega}}_{\textcolor{PineGreen}{2}}, \Xint{\begin{smallmatrix} ~ \\ {}^{}_{\mathcolor{gray}{-}} \\ ~ \end{smallmatrix}}{22}{\lambda}^{\;\!\mathcolor{gray}{\omega}}_{\textcolor{PineGreen}{3}} \right]$(其中,$\textcolor{PineGreen}{\text{diag}} \left[ \cdot \right]$ 表示生成对角矩阵),见 \bref{hook:overline}。此外,通过给广义列向量解包后的整体头上加两根长线,定义了广义张量 $\overline{\overline{\Xint{\begin{smallmatrix} ~ \\ {}^{}_{\mathcolor{gray}{-}} \\ ~ \end{smallmatrix}}{22}{\lambda}^{\;\!\mathcolor{gray}{\omega}}_{\textcolor{PineGreen}{\jmath}}}} := \overline{\Xint{\begin{smallmatrix} ~ \\ {}^{}_{\mathcolor{gray}{-}} \\ ~ \end{smallmatrix}}{22}{\bar{\lambda}}^{\;\!\mathcolor{gray}{\omega}}_{\textcolor{PineGreen}{\jmath}}}^{{\mathsf{\textcolor{Plum}{T}}}}$,见 \bref{hook:ooverline}。更多地,还可定义 $\Xint{\begin{smallmatrix} ~ \\ {}^{}_{\mathcolor{gray}{-}} \\ ~ \end{smallmatrix}}{22}{\bar{\bar{\lambda}}}^{\;\!\mathcolor{gray}{\omega}} := \overline{\overline{\Xint{\begin{smallmatrix} ~ \\ {}^{}_{\mathcolor{gray}{-}} \\ ~ \end{smallmatrix}}{22}{\lambda}^{\;\!\mathcolor{gray}{\omega}}_{\textcolor{PineGreen}{\jmath}}}}, \Xint{\begin{smallmatrix} ~ \\ {}^{}_{\mathcolor{gray}{-}} \\ ~ \end{smallmatrix}}{15}{\bar{\bar{v}}}^{\;\!\mathcolor{gray}{\omega}} := \overline{\overline{\Xint{\begin{smallmatrix} ~ \\ {}^{}_{\mathcolor{gray}{-}} \\ ~ \end{smallmatrix}}{15}{v}^{\;\!\mathcolor{gray}{\omega}}_{\textcolor{PineGreen}{\jmath}}}}$。}
\begin{align} \label{eq:kz-non_diagonalization}
	\Xint{\begin{smallmatrix} ~ \\ {}^{}_{\mathcolor{gray}{-}} \\ ~ \end{smallmatrix}}{15}{\bar{\bar{k}}}_{\textcolor{Maroon}{\symup{z}}}^{\;\! \mathcolor{gray}{\omega}} \nRightarrow \overline{\Xint{\begin{smallmatrix} ~ \\ {}^{}_{\mathcolor{gray}{-}} \\ ~ \end{smallmatrix}}{15}{\bar{v}}^{\;\!\mathcolor{gray}{\omega}}_{\textcolor{PineGreen}{\jmath}}}^{\mathsf{\textcolor{Plum}{T}}} \cdot \overline{\Xint{\begin{smallmatrix} ~ \\ {}^{}_{\mathcolor{gray}{-}} \\ ~ \end{smallmatrix}}{22}{\bar{\lambda}}^{\;\!\mathcolor{gray}{\omega}}_{\textcolor{PineGreen}{\jmath}}}^{\mathsf{\textcolor{Plum}{T}}} \cdot \overline{\Xint{\begin{smallmatrix} ~ \\ {}^{}_{\mathcolor{gray}{-}} \\ ~ \end{smallmatrix}}{15}{\bar{v}}^{\;\!\mathcolor{gray}{\omega}}_{\textcolor{PineGreen}{\jmath}}}^{-{\mathsf{\textcolor{Plum}{T}}}} = \overline{\Xint{\begin{smallmatrix} ~ \\ {}^{}_{\mathcolor{gray}{-}} \\ ~ \end{smallmatrix}}{15}{\bar{v}}^{\;\!\mathcolor{gray}{\omega}}_{\textcolor{PineGreen}{\jmath}}}^{\mathsf{\textcolor{Plum}{T}}} \cdot \overline{\overline{\Xint{\begin{smallmatrix} ~ \\ {}^{}_{\mathcolor{gray}{-}} \\ ~ \end{smallmatrix}}{22}{\lambda}^{\;\!\mathcolor{gray}{\omega}}_{\textcolor{PineGreen}{\jmath}}}} \cdot \overline{\Xint{\begin{smallmatrix} ~ \\ {}^{}_{\mathcolor{gray}{-}} \\ ~ \end{smallmatrix}}{15}{\bar{v}}^{\;\!\mathcolor{gray}{\omega}}_{\textcolor{PineGreen}{\jmath}}}^{-{\mathsf{\textcolor{Plum}{T}}}} ~,
\end{align}
对应地,\bref{eq:vec-matrix_exp} 中的 \textcolor{Maroon}{矩阵指数} 也没法化简为(定义了\textcolor{PineGreen}{松树绿},见 \bref{hook:PineGreen})
\begin{subequations}
\begin{align}
	\mathbb{e}^{\mathbb{i} \Xint{\begin{smallmatrix} ~ \\ {}^{}_{\mathcolor{gray}{-}} \\ ~ \end{smallmatrix}}{15}{\bar{\bar{k}}}_{\textcolor{Maroon}{\symup{z}}}^{\;\! \mathcolor{gray}{\omega}} \mathcolor{gray}{z}} \nRightarrow \mathbb{e}^{\mathbb{i} \overline{\Xint{\begin{smallmatrix} ~ \\ {}^{}_{\mathcolor{gray}{-}} \\ ~ \end{smallmatrix}}{15}{\bar{v}}^{\;\!\mathcolor{gray}{\omega}}_{\textcolor{PineGreen}{\jmath}}}^{{\mathsf{\textcolor{Plum}{T}}}} \cdot \overline{\Xint{\begin{smallmatrix} ~ \\ {}^{}_{\mathcolor{gray}{-}} \\ ~ \end{smallmatrix}}{22}{\bar{\lambda}}^{\;\!\mathcolor{gray}{\omega}}_{\textcolor{PineGreen}{\jmath}}}^{{\mathsf{\textcolor{Plum}{T}}}} \cdot \overline{\Xint{\begin{smallmatrix} ~ \\ {}^{}_{\mathcolor{gray}{-}} \\ ~ \end{smallmatrix}}{15}{\bar{v}}^{\;\!\mathcolor{gray}{\omega}}_{\textcolor{PineGreen}{\jmath}}}^{\textcolor{Plum}{-\mathsf{T}}} \mathcolor{gray}{z}} = &\begin{pmatrix} \Xint{\begin{smallmatrix} ~ \\ {}^{}_{\mathcolor{gray}{-}} \\ ~ \end{smallmatrix}}{15}{\bar{v}}^{\;\!\mathcolor{gray}{\omega}}_{\textcolor{PineGreen}{1}}, \Xint{\begin{smallmatrix} ~ \\ {}^{}_{\mathcolor{gray}{-}} \\ ~ \end{smallmatrix}}{15}{\bar{v}}^{\;\!\mathcolor{gray}{\omega}}_{\textcolor{PineGreen}{2}}, \Xint{\begin{smallmatrix} ~ \\ {}^{}_{\mathcolor{gray}{-}} \\ ~ \end{smallmatrix}}{15}{\bar{v}}^{\;\!\mathcolor{gray}{\omega}}_{\textcolor{PineGreen}{3}} \end{pmatrix} \cdot \mathbb{e}^{\mathbb{i} \overline{\overline{\Xint{\begin{smallmatrix} ~ \\ {}^{}_{\mathcolor{gray}{-}} \\ ~ \end{smallmatrix}}{22}{\lambda}^{\;\!\mathcolor{gray}{\omega}}_{\textcolor{PineGreen}{\jmath}}}} \mathcolor{gray}{z}} \label{eq:e^ikzz-non_diagonalization} \\
	\cdot &\begin{pmatrix} \Xint{\begin{smallmatrix} ~ \\ {}^{}_{\mathcolor{gray}{-}} \\ ~ \end{smallmatrix}}{15}{\bar{v}}^{\;\!\mathcolor{gray}{\omega}}_{\textcolor{PineGreen}{1}}, \Xint{\begin{smallmatrix} ~ \\ {}^{}_{\mathcolor{gray}{-}} \\ ~ \end{smallmatrix}}{15}{\bar{v}}^{\;\!\mathcolor{gray}{\omega}}_{\textcolor{PineGreen}{2}}, \Xint{\begin{smallmatrix} ~ \\ {}^{}_{\mathcolor{gray}{-}} \\ ~ \end{smallmatrix}}{15}{\bar{v}}^{\;\!\mathcolor{gray}{\omega}}_{\textcolor{PineGreen}{3}} \end{pmatrix}^{-1} = \overline{\Xint{\begin{smallmatrix} ~ \\ {}^{}_{\mathcolor{gray}{-}} \\ ~ \end{smallmatrix}}{15}{\bar{v}}^{\;\!\mathcolor{gray}{\omega}}_{\textcolor{PineGreen}{\jmath}}}^{{\mathsf{\textcolor{Plum}{T}}}} \cdot \overline{\overline{\mathbb{e}^{\mathbb{i} \Xint{\begin{smallmatrix} ~ \\ {}^{}_{\mathcolor{gray}{-}} \\ ~ \end{smallmatrix}}{22}{\lambda}^{\;\!\mathcolor{gray}{\omega}}_{\textcolor{PineGreen}{\jmath}} \mathcolor{gray}{z}}}} \cdot \overline{\Xint{\begin{smallmatrix} ~ \\ {}^{}_{\mathcolor{gray}{-}} \\ ~ \end{smallmatrix}}{15}{\bar{v}}^{\;\!\mathcolor{gray}{\omega}}_{\textcolor{PineGreen}{\jmath}}}^{\textcolor{Plum}{-\mathsf{T}}} \label{eq:e^ikzz-non_diagonalization2} ~.
\end{align}
\end{subequations}

\bref{eq:kz-non_diagonalization} 在无法\textcolor{Plum}{对角化}的情况下, 最多只能(被)\textcolor{PineGreen}{Jordan-Chevalley 分解}为
\begin{align} \label{eq:kz-Jordan_decompose}
	\Xint{\begin{smallmatrix} ~ \\ {}^{}_{\mathcolor{gray}{-}} \\ ~ \end{smallmatrix}}{15}{\bar{\bar{k}}}_{\textcolor{Maroon}{\symup{z}}}^{\;\! \mathcolor{gray}{\omega}} = \overline{\Xint{\begin{smallmatrix} ~ \\ {}^{}_{\mathcolor{gray}{-}} \\ ~ \end{smallmatrix}}{15}{\bar{\nu}}^{\;\!\mathcolor{gray}{\omega}}_{\textcolor{PineGreen}{\imath}}}^{\mathsf{\textcolor{Plum}{T}}} \cdot \Xint{\mathcolor{gray}{-}}{25}{\bar{\bar{J}}}^{\;\!\mathcolor{gray}{\omega}} \cdot \overline{\Xint{\begin{smallmatrix} ~ \\ {}^{}_{\mathcolor{gray}{-}} \\ ~ \end{smallmatrix}}{15}{\bar{\nu}}^{\;\!\mathcolor{gray}{\omega}}_{\textcolor{PineGreen}{l}}}^{-{\mathsf{\textcolor{Plum}{T}}}} ~,
\end{align}
定义了非传统(父)$\Xint{\begin{smallmatrix} ~ \\ {}^{}_{\mathcolor{gray}{-}} \\ ~ \end{smallmatrix}}{15}{\bar{v}}^{\;\!\mathcolor{gray}{\omega}}_{\textcolor{PineGreen}{\imath}}$ 的 广义(子)\textcolor{PineGreen}{本征向量} $\Xint{\begin{smallmatrix} ~ \\ {}^{}_{\mathcolor{gray}{-}} \\ ~ \end{smallmatrix}}{15}{\bar{\nu}}^{\;\!\mathcolor{gray}{\omega}}_{\textcolor{PineGreen}{\imath}}$,以及 $\Xint{\begin{smallmatrix} ~ \\ {}^{}_{\mathcolor{gray}{-}} \\ ~ \end{smallmatrix}}{15}{\bar{\bar{k}}}_{\textcolor{Maroon}{\symup{z}}}^{\;\! \mathcolor{gray}{\omega}}$ 的 \textcolor{PineGreen}{Jordan 标准型}
\begin{align} \label{eq:kz-Jordan_normal_form}
	\Xint{\mathcolor{gray}{-}}{25}{\bar{\bar{J}}}^{\;\!\mathcolor{gray}{\omega}} = \overline{\overline{\Xint{\begin{smallmatrix} ~ \\ {}^{}_{\mathcolor{gray}{-}} \\ ~ \end{smallmatrix}}{22}{\lambda}^{\;\!\mathcolor{gray}{\omega}}_{\textcolor{PineGreen}{\jmath}}}} + \Xint{\mathcolor{gray}{-}}{18}{\bar{\bar{M}}}^{\;\!\mathcolor{gray}{\omega}} ~,
\end{align}
其中,$\Xint{\mathcolor{gray}{-}}{18}{\bar{\bar{M}}}^{\;\!\mathcolor{gray}{\omega}}$ 是个三阶\textcolor{PineGreen}{幂零矩阵},它满足 $\Xint{\mathcolor{gray}{-}}{18}{\bar{\bar{M}}}_{\mathcolor{gray}{\omega}}^{\;\! n \geq 3} = \bar{\bar{0}}$。对应地,\textcolor{Maroon}{转移矩阵} 展开为
\begin{subequations}
\begin{align}
	\mathbb{e}^{\mathbb{i} \Xint{\begin{smallmatrix} ~ \\ {}^{}_{\mathcolor{gray}{-}} \\ ~ \end{smallmatrix}}{15}{\bar{\bar{k}}}_{\textcolor{Maroon}{\symup{z}}}^{\;\! \mathcolor{gray}{\omega}} \mathcolor{gray}{z}} &= \mathbb{e}^{\mathbb{i} \overline{\Xint{\begin{smallmatrix} ~ \\ {}^{}_{\mathcolor{gray}{-}} \\ ~ \end{smallmatrix}}{15}{\bar{\nu}}^{\;\!\mathcolor{gray}{\omega}}_{\textcolor{PineGreen}{\imath}}}^{\mathsf{\textcolor{Plum}{T}}} \cdot \Xint{\mathcolor{gray}{-}}{25}{\bar{\bar{J}}}^{\;\!\mathcolor{gray}{\omega}} \cdot \overline{\Xint{\begin{smallmatrix} ~ \\ {}^{}_{\mathcolor{gray}{-}} \\ ~ \end{smallmatrix}}{15}{\bar{\nu}}^{\;\!\mathcolor{gray}{\omega}}_{\textcolor{PineGreen}{l}}}^{-{\mathsf{\textcolor{Plum}{T}}}} \mathcolor{gray}{z}} \label{eq:e^ikzz-Jordan_decompose} \\ &\xrightarrow[]{\Xint{\mathcolor{gray}{-}}{25}{\bar{\bar{J}}}^{\;\!\mathcolor{gray}{\omega}} = \overline{\overline{\Xint{\begin{smallmatrix} ~ \\ {}^{}_{\mathcolor{gray}{-}} \\ ~ \end{smallmatrix}}{22}{\lambda}^{\;\!\mathcolor{gray}{\omega}}_{\textcolor{PineGreen}{\jmath}}}} + \Xint{\mathcolor{gray}{-}}{18}{\bar{\bar{M}}}^{\;\!\mathcolor{gray}{\omega}}} \overline{\Xint{\begin{smallmatrix} ~ \\ {}^{}_{\mathcolor{gray}{-}} \\ ~ \end{smallmatrix}}{15}{\bar{\nu}}^{\;\!\mathcolor{gray}{\omega}}_{\textcolor{PineGreen}{\imath}}}^{{\mathsf{\textcolor{Plum}{T}}}} \cdot \mathbb{e}^{\mathbb{i} \overline{\Xint{\begin{smallmatrix} ~ \\ {}^{}_{\mathcolor{gray}{-}} \\ ~ \end{smallmatrix}}{22}{\bar{\lambda}}^{\;\!\mathcolor{gray}{\omega}}_{\textcolor{PineGreen}{\jmath}}}^{{\mathsf{\textcolor{Plum}{T}}}} \mathcolor{gray}{z}} \cdot \mathbb{e}^{\mathbb{i} \Xint{\mathcolor{gray}{-}}{18}{\bar{\bar{M}}}^{\;\!\mathcolor{gray}{\omega}} \mathcolor{gray}{z}} \cdot \overline{\Xint{\begin{smallmatrix} ~ \\ {}^{}_{\mathcolor{gray}{-}} \\ ~ \end{smallmatrix}}{15}{\bar{\nu}}^{\;\!\mathcolor{gray}{\omega}}_{\textcolor{PineGreen}{l}}}^{\textcolor{Plum}{-\mathsf{T}}} \label{eq:e^ikzz-Jordan_decompose2} \\ 
	&\xrightarrow[]{\hphantom{\Xint{\mathcolor{gray}{-}}{25}{\bar{\bar{J}}}^{\;\!\mathcolor{gray}{\omega}} = \overline{\overline{\Xint{\begin{smallmatrix} ~ \\ {}^{}_{\mathcolor{gray}{-}} \\ ~ \end{smallmatrix}}{22}{\lambda}^{\;\!\mathcolor{gray}{\omega}}_{\textcolor{PineGreen}{\jmath}}}} + \Xint{\mathcolor{gray}{-}}{18}{\bar{\bar{M}}}^{\;\!\mathcolor{gray}{\omega}}}} \overline{\Xint{\begin{smallmatrix} ~ \\ {}^{}_{\mathcolor{gray}{-}} \\ ~ \end{smallmatrix}}{15}{\bar{\nu}}^{\;\!\mathcolor{gray}{\omega}}_{\textcolor{PineGreen}{\imath}}}^{{\mathsf{\textcolor{Plum}{T}}}} \cdot \overline{\overline{\mathbb{e}^{\mathbb{i} \Xint{\begin{smallmatrix} ~ \\ {}^{}_{\mathcolor{gray}{-}} \\ ~ \end{smallmatrix}}{22}{\lambda}^{\;\!\mathcolor{gray}{\omega}}_{\textcolor{PineGreen}{\jmath}} \mathcolor{gray}{z}}}} \cdot \sum^{2}_{n=0} \frac{\left( \mathbb{i} \Xint{\mathcolor{gray}{-}}{18}{\bar{\bar{M}}}^{\;\!\mathcolor{gray}{\omega}} \mathcolor{gray}{z} \right)^n}{n!} \cdot \overline{\Xint{\begin{smallmatrix} ~ \\ {}^{}_{\mathcolor{gray}{-}} \\ ~ \end{smallmatrix}}{15}{\bar{\nu}}^{\;\!\mathcolor{gray}{\omega}}_{\textcolor{PineGreen}{l}}}^{\textcolor{Plum}{-\mathsf{T}}} \label{eq:e^ikzz-Jordan_decompose3} \\ 
	&\xrightarrow[]{\Xint{\mathcolor{gray}{-}}{25}{\bar{\bar{J}}}^{\;\!\mathcolor{gray}{\omega}} = \Xint{\mathcolor{gray}{-}}{18}{\bar{\bar{M}}}^{\;\!\mathcolor{gray}{\omega}} + \overline{\overline{\Xint{\begin{smallmatrix} ~ \\ {}^{}_{\mathcolor{gray}{-}} \\ ~ \end{smallmatrix}}{22}{\lambda}^{\;\!\mathcolor{gray}{\omega}}_{\textcolor{PineGreen}{\jmath}}}}} \overline{\Xint{\begin{smallmatrix} ~ \\ {}^{}_{\mathcolor{gray}{-}} \\ ~ \end{smallmatrix}}{15}{\bar{\nu}}^{\;\!\mathcolor{gray}{\omega}}_{\textcolor{PineGreen}{\imath}}}^{{\mathsf{\textcolor{Plum}{T}}}} \cdot \sum^{2}_{n=0} \frac{\left( \mathbb{i} \Xint{\mathcolor{gray}{-}}{18}{\bar{\bar{M}}}^{\;\!\mathcolor{gray}{\omega}} \mathcolor{gray}{z} \right)^n}{n!} \cdot \overline{\overline{\mathbb{e}^{\mathbb{i} \Xint{\begin{smallmatrix} ~ \\ {}^{}_{\mathcolor{gray}{-}} \\ ~ \end{smallmatrix}}{22}{\lambda}^{\;\!\mathcolor{gray}{\omega}}_{\textcolor{PineGreen}{\jmath}} \mathcolor{gray}{z}}}} \cdot \overline{\Xint{\begin{smallmatrix} ~ \\ {}^{}_{\mathcolor{gray}{-}} \\ ~ \end{smallmatrix}}{15}{\bar{\nu}}^{\;\!\mathcolor{gray}{\omega}}_{\textcolor{PineGreen}{l}}}^{\textcolor{Plum}{-\mathsf{T}}} \label{eq:e^ikzz-Jordan_decompose4} ~,
\end{align}
\end{subequations}
其中,非零 $\Xint{\mathcolor{gray}{-}}{18}{\bar{\bar{M}}}_{\mathcolor{gray}{\omega}}^{\;\! 0}$, $\Xint{\mathcolor{gray}{-}}{18}{\bar{\bar{M}}}_{\mathcolor{gray}{\omega}}^{\;\! 1}$, $\Xint{\mathcolor{gray}{-}}{18}{\bar{\bar{M}}}_{\mathcolor{gray}{\omega}}^{\;\! 2}$ 分别产生 $\mathcolor{gray}{z^0}, \mathcolor{gray}{z^1}, \mathcolor{gray}{z^2}$ 依赖的\textcolor{PineGreen}{基系数集} $\Xint{\mathcolor{gray}{-}}{18}{\bar{\bar{M}}}_{\mathcolor{gray}{\omega}}^{\;\! n} \cdot \overline{\Xint{\begin{smallmatrix} ~ \\ {}^{}_{\mathcolor{gray}{-}} \\ ~ \end{smallmatrix}}{15}{\bar{\nu}}^{\;\!\mathcolor{gray}{\omega}}_{\textcolor{PineGreen}{l}}}^{\textcolor{Plum}{-\mathsf{T}}} \cdot \Xint{{}^{}_{\mathcolor{gray}{-}}}{10}{\bar{g}}^{\;\!\mathcolor{gray}{\omega}}_0 \mathcolor{gray}{z^n}$ 或 \textcolor{PineGreen}{本征偏振态集} $\overline{\Xint{\begin{smallmatrix} ~ \\ {}^{}_{\mathcolor{gray}{-}} \\ ~ \end{smallmatrix}}{15}{\bar{\nu}}^{\;\!\mathcolor{gray}{\omega}}_{\textcolor{PineGreen}{\imath}}}^{\mathsf{\textcolor{Plum}{T}}} \cdot \Xint{\mathcolor{gray}{-}}{18}{\bar{\bar{M}}}_{\mathcolor{gray}{\omega}}^{\;\! n} \mathcolor{gray}{z^n}$。对 $\mathbb{e}^{\mathbb{i} \Xint{\begin{smallmatrix} ~ \\ {}^{}_{\mathcolor{gray}{-}} \\ ~ \end{smallmatrix}}{15}{\bar{\bar{k}}}_{\textcolor{Maroon}{\symup{z}}}^{\;\! \mathcolor{gray}{\omega}} \mathcolor{gray}{z}}$ 的这两种分解\bref{eq:e^ikzz-non_diagonalization,eq:e^ikzz-Jordan_decompose},甚至不分解直接计算\textcolor{Maroon}{矩阵指数}\cite{zarifiPlaneWaveReflection2014},在某些情况下会不可避免地失效:比如,对于\textcolor{Plum}{非厄米}的 $\Xint{\begin{smallmatrix} ~ \\ {}^{}_{\mathcolor{gray}{-}} \\ ~ \end{smallmatrix}}{15}{\bar{\bar{k}}}_{\textcolor{Maroon}{\symup{z}}}^{\;\! \mathcolor{gray}{\omega}}$ 以及长的\textcolor{gray}{传播距离} $\mathcolor{gray}{z}$,由于 $\Xint{\begin{smallmatrix} ~ \\ {}^{}_{\mathcolor{gray}{-}} \\ ~ \end{smallmatrix}}{15}{\bar{\bar{k}}}_{\textcolor{Maroon}{\symup{z}}}^{\;\! \mathcolor{gray}{\omega}}$ 是单个包含所有\textcolor{PineGreen}{本征模}/\textcolor{Maroon}{系统}的整体,它的独立传播迫使该 $\textcolor{Plum}{3 \times 3}$ 矩阵 $\Xint{\begin{smallmatrix} ~ \\ {}^{}_{\mathcolor{gray}{-}} \\ ~ \end{smallmatrix}}{15}{\bar{\bar{k}}}_{\textcolor{Maroon}{\symup{z}}}^{\;\! \mathcolor{gray}{\omega}}$ 的 $1\sim3$ 个\textcolor{PineGreen}{本征值}-\textcolor{PineGreen}{向量}对(们),在同一个共享 $\mathcolor{gray}{z}$ 值的横截面上,一起跟随着\textcolor{Plum}{同步}(但不一定\textcolor{Plum}{同向})传播。然而,如果这些\textcolor{Plum}{同步}传播的\textcolor{PineGreen}{本征模}之间,存在(相对的)\textcolor{Plum}{反向}传播,则某些\textcolor{PineGreen}{本征(行)波}的\textcolor{PineGreen}{本征值}即使在\textcolor{NavyBlue}{吸收}/\textcolor{NavyBlue}{耗散}介质中也会出现\textcolor{Plum}{负虚部},从而导致增益随着\textcolor{gray}{传播距离} $\mathcolor{gray}{z}$ 的增加而增加,并最终使\textcolor{Maroon}{矩阵指数}发散。

至此,\textcolor{Maroon}{矩阵指数解}所存在的问题,便一览无余:{\one} 满足 \bref{eq:nonlinear(2)-wave_wkrho-simplify6-L4} 的 $\Xint{\begin{smallmatrix} ~ \\ {}^{}_{\mathcolor{gray}{-}} \\ ~ \end{smallmatrix}}{15}{\bar{\bar{k}}}_{\textcolor{Maroon}{\symup{z}}}^{\;\! \mathcolor{gray}{\omega}} \in \bar{\bar{\mathbb{C}}}_{\textcolor{Plum}{\left[3 \times 3\right]}}$ 难以计算\cite{sturmElectromagneticWavesCrystals2024}。{\two} 即使得到了 $\Xint{\begin{smallmatrix} ~ \\ {}^{}_{\mathcolor{gray}{-}} \\ ~ \end{smallmatrix}}{15}{\bar{\bar{k}}}_{\textcolor{Maroon}{\symup{z}}}^{\;\! \mathcolor{gray}{\omega}}$,也需要计算它的\textcolor{Maroon}{指数} $\mathbb{e}^{\mathbb{i} \Xint{\begin{smallmatrix} ~ \\ {}^{}_{\mathcolor{gray}{-}} \\ ~ \end{smallmatrix}}{15}{\bar{\bar{k}}}_{\textcolor{Maroon}{\symup{z}}}^{\;\! \mathcolor{gray}{\omega}} \mathcolor{gray}{z}}$,而对后者的计算也会遇到一些棘手的问题:如果 $\Xint{\begin{smallmatrix} ~ \\ {}^{}_{\mathcolor{gray}{-}} \\ ~ \end{smallmatrix}}{15}{\bar{\bar{k}}}_{\textcolor{Maroon}{\symup{z}}}^{\;\! \mathcolor{gray}{\omega}}$ 中的\textcolor{PineGreen}{本征(行)波}们的传播方向,不全是同向的,则 \textcircled{1} \textcolor{Maroon}{矩阵指数解}无法区分\textcolor{Plum}{正}/\textcolor{Plum}{反向}\textcolor{PineGreen}{本征波},\textcircled{2} 并因此无法计算\textcolor{Plum}{非厄米}材料内的长程传播,因为对这些行波所组成的整体的统一计算最终势必会指数发散\cite{stallingaBerreman4x4Matrix1999}。{\three} 对“可能病态的” $\Xint{\begin{smallmatrix} ~ \\ {}^{}_{\mathcolor{gray}{-}} \\ ~ \end{smallmatrix}}{15}{\bar{\bar{k}}}_{\textcolor{Maroon}{\symup{z}}}^{\;\! \mathcolor{gray}{\omega}}$ 进行特征分解,以期 \textcircled{1} 利用 \bref{eq:kz-Jordan_decompose} 分离出\textcolor{PineGreen}{特征值}或\textcolor{PineGreen}{特征向量}并单独研究之,或者 \textcircled{2} 将上述获得的(广义非简并)\textcolor{PineGreen}{特征系统}中的其中一组同向传播\textcolor{PineGreen}{本征波}\Footnote{共有 2 组\textcolor{PineGreen}{本征(行)波},组内方向相同,组间方向相反。},排列成 \bref{eq:kz-Jordan_decompose}\cite{borzdovWavesLinearQuadratic1996},然后利用 \bref{eq:e^ikzz-Jordan_decompose2} 辅助计算其\textcolor{Maroon}{矩阵指数} $\mathbb{e}^{\mathbb{i} \Xint{\begin{smallmatrix} ~ \\ {}^{}_{\mathcolor{gray}{-}} \\ ~ \end{smallmatrix}}{15}{\bar{\bar{k}}}_{\textcolor{Maroon}{\symup{z}}}^{\;\! \mathcolor{gray}{\omega}} \mathcolor{gray}{z}}$,最“\textcolor{Plum}{分}/\textcolor{Plum}{解析}”的特征分解方法是 \textcolor{PineGreen}{Jordan-Chevalley 分解}及其等价方法。--- 然而,由于 $\Xint{\begin{smallmatrix} ~ \\ {}^{}_{\mathcolor{gray}{-}} \\ ~ \end{smallmatrix}}{15}{\bar{\bar{k}}}_{\textcolor{Maroon}{\symup{z}}}^{\;\! \mathcolor{gray}{\omega}}$ 是个 2D(二维)分布的矩阵场,只能对其进行批量 \textcolor{PineGreen}{Jordan-Chevalley 分解},而这只能是数值的,并且在\textcolor{PineGreen}{例外点}(\textcolor{PineGreen}{EP})附近在理论和数值上均是不稳定的\cite{molerNineteenDubiousWays2003}。{\four} 所有病态的 $\Xint{\begin{smallmatrix} ~ \\ {}^{}_{\mathcolor{gray}{-}} \\ ~ \end{smallmatrix}}{15}{\bar{\bar{k}}}_{\textcolor{Maroon}{\symup{z}}}^{\;\! \mathcolor{gray}{\omega}}$,如果没有通过 \textcolor{PineGreen}{Jordan-Chevalley 分解}或其等价的方法,获得其广义(子)\textcolor{PineGreen}{本征系统},那么在其非子基矢空间中,总存在接近简并的\textcolor{PineGreen}{特征向量},且无法再被进一步区分,不论是解析地还是数值地。此时,\textcolor{Maroon}{矩阵指数解}在形式和内容上不存在任何优势\cite{berremanOpticsStratifiedAnisotropic1972,stallingaBerreman4x4Matrix1999},相比接下来提出的方法。

作为通过特征分解,计算并区分近简并\textcolor{PineGreen}{特征向量}、\textcolor{Plum}{正反向}\textcolor{PineGreen}{本征波},方便计算\textcolor{Maroon}{矩阵指数},并有可能避免\textcolor{Maroon}{矩阵指数}发散的强大理论工具,对矩阵场 $\Xint{\begin{smallmatrix} ~ \\ {}^{}_{\mathcolor{gray}{-}} \\ ~ \end{smallmatrix}}{15}{\bar{\bar{k}}}_{\textcolor{Maroon}{\symup{z}}}^{\;\! \mathcolor{gray}{\omega}}$ 执行批量 \textcolor{PineGreen}{Jordan-Chevalley 分解}在数值上的不稳定性\cite{molerNineteenDubiousWays2003},迫使我们寻找矩阵形式\textcolor{PineGreen}{本征解} \bref{eq:vec-matrix_exp} 和与之适配的 \textcolor{PineGreen}{Jordan-Chevalley 分解}技术的组合的下位替代。

\clearpage
\vspace*{-8.0em}

%\vspace*{-5.0em}

\marginLeft[-2.4em]{ssec:E-waveq-linear}\subsection{线性均匀非局域、非矩阵指数 $\bar{E}$ 波动方程}\label{ssec:E-waveq-linear}

既然\textcolor{Maroon}{矩阵指数}形式的试探解 \bref{eq:vec-matrix_exp} 除了能区分(\textcolor{Plum}{有限个数}的)\textcolor{PineGreen}{光学奇点} or \textcolor{PineGreen}{例外点}处,接近简并的\textcolor{PineGreen}{特征向量}外,无法带来更多的好处,该节则尝试放弃该形式,转而采取更常见的\textcolor{PineGreen}{线性叠加}的\textcolor{gray}{单色}\textcolor{PineGreen}{平面波基}形式的试探解
\begin{subequations} \label{eq:plane_wave_basis}
\begin{align}
	\Xint{\mathcolor{gray}{-}}{30}{\bar{E}}^{\;\!\mathcolor{gray}{\omega}}_{\;\! \mathcolor{gray}{z}} &:= \leftindex_{\textcolor{PineGreen}{\jmath}} \;\! \Xint{\mathcolor{gray}{-}}{30}{\bar{E}}^{\;\!\mathcolor{gray}{\omega} \textcolor{PineGreen}{\jmath}}_{\;\! \mathcolor{gray}{z}} := \Xint{{}^{}_{\mathcolor{gray}{-}}}{10}{\bar{g}}^{\;\!\mathcolor{gray}{\omega} \textcolor{PineGreen}{\jmath}}_{\;\! \mathcolor{gray}{z}} \mathbb{e}^{\mathbb{i} \Xint{\begin{smallmatrix} ~ \\ {}^{}_{\mathcolor{gray}{-}} \\ ~ \end{smallmatrix}}{15}{k}_{\symup{z} \textcolor{PineGreen}{\jmath}}^{\;\! \mathcolor{gray}{\omega}} \mathcolor{gray}{z}} ~, \label{eq:vec-plane_wave_basis} \\
	\Xint{\mathcolor{gray}{-}}{30}{E}^{\;\!\mathcolor{gray}{\omega}}_{\;\! \hat{1} \mathcolor{gray}{z}} &:= \leftindex_{\textcolor{PineGreen}{\jmath}} \;\! \Xint{\mathcolor{gray}{-}}{30}{E}^{\;\!\mathcolor{gray}{\omega} \textcolor{PineGreen}{\jmath}}_{\;\! \hat{1} \mathcolor{gray}{z}} := \Xint{{}^{}_{\mathcolor{gray}{-}}}{10}{g}^{\;\!\mathcolor{gray}{\omega} \textcolor{PineGreen}{\jmath}}_{\;\! \hat{1} \mathcolor{gray}{z}} \mathbb{e}^{\mathbb{i} \Xint{\begin{smallmatrix} ~ \\ {}^{}_{\mathcolor{gray}{-}} \\ ~ \end{smallmatrix}}{15}{k}_{\symup{z} \textcolor{PineGreen}{\jmath}}^{\;\! \mathcolor{gray}{\omega}} \mathcolor{gray}{z}} ~, \label{eq:components-plane_wave_basis}
\end{align}
\end{subequations}
当 \bref{eq:k_check_j^nabla} 即算子 $\mathcolor{gray}{k^\nabla_{\check{1}}}$ 中的 $\mathcolor{gray}{\nabla_z}$ 作用于 \bref{eq:nonlinear(2)-wave_wkrho-simplify5} 的试探解 \bref{eq:components-plane_wave_basis} 的\textcolor{PineGreen}{每个(第$\textcolor{PineGreen}{\jmath}$个)模式}即 $
\Xint{\mathcolor{gray}{-}}{25}{E}^{\;\!\mathcolor{gray}{\omega} \textcolor{PineGreen}{\jmath}}_{\;\! \hat{1} \mathcolor{gray}{z}} = \Xint{{}^{}_{\mathcolor{gray}{-}}}{10}{g}^{\;\!\mathcolor{gray}{\omega} \textcolor{PineGreen}{\jmath}}_{\;\! \hat{1} \mathcolor{gray}{z}} \mathbb{e}^{\mathbb{i} \Xint{\begin{smallmatrix} ~ \\ {}^{}_{\mathcolor{gray}{-}} \\ ~ \end{smallmatrix}}{15}{k}_{\symup{z}}^{\;\! \mathcolor{gray}{\omega} \textcolor{PineGreen}{\jmath}} \mathcolor{gray}{z}}$ 之后,结果从 \bref{eq:matrix_exp-dz,eq:matrix_exp-ddz} 降阶为
\begin{subequations}
\begin{align}
	\mathcolor{gray}{\nabla_z} \Xint{\mathcolor{gray}{-}}{30}{E}^{\;\!\mathcolor{gray}{\omega} \textcolor{PineGreen}{\jmath}}_{\;\! \hat{1} \mathcolor{gray}{z}} = \mathcolor{gray}{\nabla_z} \left( \Xint{{}^{}_{\mathcolor{gray}{-}}}{10}{g}^{\;\!\mathcolor{gray}{\omega} \textcolor{PineGreen}{\jmath}}_{\;\! \hat{1} \mathcolor{gray}{z}} \mathbb{e}^{\mathbb{i} \Xint{\begin{smallmatrix} ~ \\ {}^{}_{\mathcolor{gray}{-}} \\ ~ \end{smallmatrix}}{15}{k}_{\symup{z}}^{\;\! \mathcolor{gray}{\omega} \textcolor{PineGreen}{\jmath}} \mathcolor{gray}{z}} \right) = \left( \mathbb{i} \Xint{\begin{smallmatrix} ~ \\ {}^{}_{\mathcolor{gray}{-}} \\ ~ \end{smallmatrix}}{15}{k}_{\symup{z}}^{\;\! \mathcolor{gray}{\omega} \textcolor{PineGreen}{\jmath}} + \mathcolor{gray}{\nabla_z} \right) \Xint{{}^{}_{\mathcolor{gray}{-}}}{10}{g}^{\;\!\mathcolor{gray}{\omega} \textcolor{PineGreen}{\jmath}}_{\;\! \hat{1} \mathcolor{gray}{z}} \mathbb{e}^{\mathbb{i} \Xint{\begin{smallmatrix} ~ \\ {}^{}_{\mathcolor{gray}{-}} \\ ~ \end{smallmatrix}}{15}{k}_{\symup{z}}^{\;\! \mathcolor{gray}{\omega} \textcolor{PineGreen}{\jmath}} \mathcolor{gray}{z}} ~, \label{eq:plane_wave_basis-dz} \\
	\mathcolor{gray}{\nabla_z^2} \Xint{\mathcolor{gray}{-}}{30}{E}^{\;\!\mathcolor{gray}{\omega} \textcolor{PineGreen}{\jmath}}_{\;\! \hat{1} \mathcolor{gray}{z}} = \left( - \Xint{\begin{smallmatrix} ~ \\ {}^{}_{\mathcolor{gray}{-}} \\ ~ \end{smallmatrix}}{15}{k}_{\symup{z} \mathcolor{gray}{\omega}}^{\;\! \textcolor{PineGreen}{\jmath} 2} + 2 \mathbb{i} \Xint{\begin{smallmatrix} ~ \\ {}^{}_{\mathcolor{gray}{-}} \\ ~ \end{smallmatrix}}{15}{k}_{\symup{z}}^{\;\! \mathcolor{gray}{\omega} \textcolor{PineGreen}{\jmath}} \mathcolor{gray}{\nabla_z} + \mathcolor{gray}{\nabla_z^2} \right) \Xint{{}^{}_{\mathcolor{gray}{-}}}{10}{g}^{\;\!\mathcolor{gray}{\omega} \textcolor{PineGreen}{\jmath}}_{\;\! \hat{1} \mathcolor{gray}{z}} \mathbb{e}^{\mathbb{i} \Xint{\begin{smallmatrix} ~ \\ {}^{}_{\mathcolor{gray}{-}} \\ ~ \end{smallmatrix}}{15}{k}_{\symup{z}}^{\;\! \mathcolor{gray}{\omega} \textcolor{PineGreen}{\jmath}} \mathcolor{gray}{z}} ~, \label{eq:plane_wave_basis-ddz}
\end{align}
\end{subequations}
使用 \bref{eq:plane_wave_basis-dz,eq:plane_wave_basis-ddz},将 \bref{eq:nonlinear(2)-wave_wkrho-simplify5-vector} 左侧分离出\textcolor{Plum}{线性}、\textcolor{Plum}{非线性}算子 $\Xint{\mathcolor{gray}{-}}{30}{\bar{\bar{L}}}^{\;\! \mathcolor{gray}{\omega} \textcolor{PineGreen}{\imath}}, \Xint{\mathcolor{gray}{-}}{21}{\bar{\bar{V}}}^{\;\! \mathcolor{gray}{\omega} \textcolor{PineGreen}{\imath}}$
\begin{subequations}
	\abovedisplayskip=-5pt
\begin{align}
	\Xint{\mathcolor{gray}{-}}{32}{\bar{\bar{L}}}^{\;\! \mathcolor{gray}{\omega} \textcolor{PineGreen}{\imath}} \Xint{\mathcolor{gray}{-}}{295}{\bar{E}}^{\;\!\mathcolor{gray}{\omega} \textcolor{PineGreen}{\jmath}}_{\;\! \mathcolor{gray}{z}} &:= \left( \bar{\bar{\varepsilon}}^{\;\! \mathcolor{gray}{\omega}}_{\textcolor{Maroon}{(1)}} + \bar{\bar{\zeta}}^{\;\! \mathcolor{gray}{\omega} \mathcolor{gray}{\check{1}}}_{\textcolor{Maroon}{(1)}} \mathbb{i} \Xint{\begin{smallmatrix} ~ \\ {}^{}_{\mathcolor{gray}{-}} \\ ~ \end{smallmatrix}}{15}{k}_{\;\! \mathcolor{gray}{\check{1}}}^{\;\! \mathcolor{gray}{\omega} \textcolor{PineGreen}{\jmath}} - \bar{\bar{\zeta}}^{\;\! \mathcolor{gray}{\omega} \mathcolor{gray}{\check{1} \check{2}}}_{\textcolor{Maroon}{(1)}} \Xint{\begin{smallmatrix} ~ \\ {}^{}_{\mathcolor{gray}{-}} \\ ~ \end{smallmatrix}}{15}{k}_{\;\! \mathcolor{gray}{\check{2}}}^{\;\! \mathcolor{gray}{\omega} \textcolor{PineGreen}{\jmath}} \Xint{\begin{smallmatrix} ~ \\ {}^{}_{\mathcolor{gray}{-}} \\ ~ \end{smallmatrix}}{15}{k}_{\;\! \mathcolor{gray}{\check{1}}}^{\;\! \mathcolor{gray}{\omega} \textcolor{PineGreen}{\jmath}} \right) \Xint{\mathcolor{gray}{-}}{295}{\bar{E}}^{\;\!\mathcolor{gray}{\omega} \textcolor{PineGreen}{\jmath}}_{\;\! \mathcolor{gray}{z}} ~, \label{eq:plane_wave_basis_eq-left_L} \\
	\Xint{\mathcolor{gray}{-}}{25}{\bar{\bar{V}}}^{\;\! \mathcolor{gray}{\omega} \textcolor{PineGreen}{\imath}} \Xint{{}^{}_{\mathcolor{gray}{-}}}{10}{\bar{g}}^{\;\!\mathcolor{gray}{\omega} \textcolor{PineGreen}{\jmath}}_{\;\! \mathcolor{gray}{z}} &:= \left[ \left( \bar{\bar{\zeta}}^{\;\! \mathcolor{gray}{\omega} \mathcolor{gray}{\symup{z}}}_{\textcolor{Maroon}{(1)}} + \bar{\bar{\zeta}}^{\;\! \mathcolor{gray}{\omega} \mathcolor{gray}{\check{1} \symup{z}}}_{\textcolor{Maroon}{(1)}} \mathbb{i} \Xint{\begin{smallmatrix} ~ \\ {}^{}_{\mathcolor{gray}{-}} \\ ~ \end{smallmatrix}}{15}{k}_{\;\! \mathcolor{gray}{\check{1}}}^{\;\! \mathcolor{gray}{\omega} \textcolor{PineGreen}{\jmath}} + \bar{\bar{\zeta}}^{\;\! \mathcolor{gray}{\omega} \mathcolor{gray}{\symup{z} \check{2}}}_{\textcolor{Maroon}{(1)}} \mathbb{i} \Xint{\begin{smallmatrix} ~ \\ {}^{}_{\mathcolor{gray}{-}} \\ ~ \end{smallmatrix}}{15}{k}_{\;\! \mathcolor{gray}{\check{2}}}^{\;\! \mathcolor{gray}{\omega} \textcolor{PineGreen}{\jmath}} \right) \mathcolor{gray}{\nabla_z} + \bar{\bar{\zeta}}^{\;\! \mathcolor{gray}{\omega} \mathcolor{gray}{\symup{z} \symup{z}}}_{\textcolor{Maroon}{(1)}} \mathcolor{gray}{\nabla_z^2} \right] \Xint{{}^{}_{\mathcolor{gray}{-}}}{10}{\bar{g}}^{\;\!\mathcolor{gray}{\omega} \textcolor{PineGreen}{\jmath}}_{\;\! \mathcolor{gray}{z}} \mathbb{e}^{\mathbb{i} \Xint{\begin{smallmatrix} ~ \\ {}^{}_{\mathcolor{gray}{-}} \\ ~ \end{smallmatrix}}{15}{k}_{\symup{z}}^{\;\! \mathcolor{gray}{\omega} \textcolor{PineGreen}{\jmath}} \mathcolor{gray}{z}} ~, \label{eq:plane_wave_basis_eq-left_V}
\end{align}
\end{subequations}
其中,类似 \bref{eq:barbar_k_check_j} 地定义了材料内\textcolor{Plum}{非均匀}\cite{wangComplexRayTracing2008a,changRayTracingAbsorbing2005,sturmElectromagneticWavesCrystals2024}复\textcolor{PineGreen}{波矢}量 $\Xint{\begin{smallmatrix} ~ \\ {}^{}_{\mathcolor{gray}{-}} \\ ~ \end{smallmatrix}}{15}{\bar{k}}^{\;\! \mathcolor{gray}{\omega} \textcolor{PineGreen}{\jmath}}$ 三分量\cite{xieAnalytic3DVector}
\begin{align} \label{eq:k_check_j}
	\Xint{\begin{smallmatrix} ~ \\ {}^{}_{\mathcolor{gray}{-}} \\ ~ \end{smallmatrix}}{15}{k}_{\;\! \mathcolor{gray}{\check{\symup{\jmath}}}}^{\;\! \mathcolor{gray}{\omega} \textcolor{PineGreen}{\jmath}} := \mathcolor{gray}{k_{\symup{x}}} ~\textcolor{Maroon}{\text{或}}~ \mathcolor{gray}{k_{\symup{y}}} ~\textcolor{Maroon}{\text{或}}~ \Xint{\begin{smallmatrix} ~ \\ {}^{}_{\mathcolor{gray}{-}} \\ ~ \end{smallmatrix}}{15}{k}_{\symup{z}}^{\;\! \mathcolor{gray}{\omega} \textcolor{PineGreen}{\jmath}}~,
\end{align}
有了 $\left( \mathcolor{gray}{\omega}, \mathcolor{gray}{\bar{k}_{\symup{\rho}}} \right)$ 域中,通过 \bref{eq:plane_wave_basis_eq-left_L} 定义的\textcolor{Plum}{线性}算子 $\Xint{\mathcolor{gray}{-}}{30}{\bar{\bar{L}}}^{\;\! \mathcolor{gray}{\omega} \textcolor{PineGreen}{\imath}}$ 和 \bref{eq:plane_wave_basis_eq-left_V} 定义的\textcolor{Plum}{非线性}算子 $\Xint{\mathcolor{gray}{-}}{21}{\bar{\bar{V}}}^{\;\! \mathcolor{gray}{\omega} \textcolor{PineGreen}{\imath}}$ 后,\bref{eq:nonlinear(2)-wave_wkrho-simplify5-vector} 可进一步简写作 \bref{eq:nonlinear(2)-wave_wkrho-simplify6}。

如 \bref{ssec:Exp-waveq} 所言,假设 \bref{eq:L+V-decompose,eq:L+V-decompose2,eq:nonlinear(2)-wave_wkrho-simplify6-V} 中的任意一个,即“\textbf{\textcolor{Plum}{非线性}\textcolor{NavyBlue}{光学}\textcolor{PineGreen}{本征系统},是\textcolor{Plum}{线性}\textcolor{NavyBlue}{光学}\textcolor{PineGreen}{本征系统}的 \textcolor{NavyBlue}{0 阶微扰}}”成立,则 \bref{eq:nonlinear(2)-wave_wkrho-simplify6} 可以进一步拆分为 \bref{eq:nonlinear(2)-wave_wkrho-simplify6-L,eq:nonlinear(2)-wave_wkrho-simplify6-V};再将 \bref{eq:plane_wave_basis_eq-left_L,eq:plane_wave_basis_eq-left_V} 代入上述二者中,最终得 \bref{eq:nonlinear(2)-wave_wkrho-simplify6-L3,eq:nonlinear(2)-wave_wkrho-simplify6-V2} 的降阶版:
\begin{subequations} \label{eq:nonlinear(2)-wave_wkrho-simplify6'}
\begin{align}
	\textcolor{Plum}{\det} \left[ \Xint{\mathcolor{gray}{-}}{32}{\bar{\bar{L}}}^{\;\! \mathcolor{gray}{\omega} \textcolor{PineGreen}{\imath}} \right] &= \textcolor{Plum}{\det} \left[ \bar{\bar{\varepsilon}}^{\;\! \mathcolor{gray}{\omega}}_{\textcolor{Maroon}{(1)}} + \bar{\bar{\zeta}}^{\;\! \mathcolor{gray}{\omega} \mathcolor{gray}{\check{1}}}_{\textcolor{Maroon}{(1)}} \mathbb{i} \Xint{\begin{smallmatrix} ~ \\ {}^{}_{\mathcolor{gray}{-}} \\ ~ \end{smallmatrix}}{15}{k}_{\;\! \mathcolor{gray}{\check{1}}}^{\;\! \mathcolor{gray}{\omega} \textcolor{PineGreen}{\imath}} - \bar{\bar{\zeta}}^{\;\! \mathcolor{gray}{\omega} \mathcolor{gray}{\check{1} \check{2}}}_{\textcolor{Maroon}{(1)}} \Xint{\begin{smallmatrix} ~ \\ {}^{}_{\mathcolor{gray}{-}} \\ ~ \end{smallmatrix}}{15}{k}_{\;\! \mathcolor{gray}{\check{2}}}^{\;\! \mathcolor{gray}{\omega} \textcolor{PineGreen}{\imath}} \Xint{\begin{smallmatrix} ~ \\ {}^{}_{\mathcolor{gray}{-}} \\ ~ \end{smallmatrix}}{15}{k}_{\;\! \mathcolor{gray}{\check{1}}}^{\;\! \mathcolor{gray}{\omega} \textcolor{PineGreen}{\imath}} \right] = 0 ~, \label{eq:nonlinear(2)-wave_wkrho-simplify6-L3'} \\
	\Xint{\mathcolor{gray}{-}}{25}{\bar{\bar{V}}}^{\;\! \mathcolor{gray}{\omega} \textcolor{PineGreen}{\imath}} \Xint{{}^{}_{\mathcolor{gray}{-}}}{10}{\bar{g}}^{\;\!\mathcolor{gray}{\omega} \textcolor{PineGreen}{\imath}}_{\;\! \mathcolor{gray}{z}}
	&= \left[ \left( \bar{\bar{\zeta}}^{\;\! \mathcolor{gray}{\omega} \mathcolor{gray}{\symup{z}}}_{\textcolor{Maroon}{(1)}} + \bar{\bar{\zeta}}^{\;\! \mathcolor{gray}{\omega} \mathcolor{gray}{\check{1} \symup{z}}}_{\textcolor{Maroon}{(1)}} \mathbb{i} \Xint{\begin{smallmatrix} ~ \\ {}^{}_{\mathcolor{gray}{-}} \\ ~ \end{smallmatrix}}{15}{k}_{\;\! \mathcolor{gray}{\check{1}}}^{\;\! \mathcolor{gray}{\omega} \textcolor{PineGreen}{\imath}} + \bar{\bar{\zeta}}^{\;\! \mathcolor{gray}{\omega} \mathcolor{gray}{\symup{z} \check{2}}}_{\textcolor{Maroon}{(1)}} \mathbb{i} \Xint{\begin{smallmatrix} ~ \\ {}^{}_{\mathcolor{gray}{-}} \\ ~ \end{smallmatrix}}{15}{k}_{\;\! \mathcolor{gray}{\check{2}}}^{\;\! \mathcolor{gray}{\omega} \textcolor{PineGreen}{\imath}} \right) \mathcolor{gray}{\nabla_z} + \bar{\bar{\zeta}}^{\;\! \mathcolor{gray}{\omega} \mathcolor{gray}{\symup{z} \symup{z}}}_{\textcolor{Maroon}{(1)}} \mathcolor{gray}{\nabla_z^2} \right] \Xint{{}^{}_{\mathcolor{gray}{-}}}{10}{\bar{g}}^{\;\!\mathcolor{gray}{\omega} \textcolor{PineGreen}{\imath}}_{\;\! \mathcolor{gray}{z}} \mathbb{e}^{\mathbb{i} \Xint{\begin{smallmatrix} ~ \\ {}^{}_{\mathcolor{gray}{-}} \\ ~ \end{smallmatrix}}{15}{k}_{\symup{z}}^{\;\! \mathcolor{gray}{\omega} \textcolor{PineGreen}{\imath}} \mathcolor{gray}{z}} \label{eq:nonlinear(2)-wave_wkrho-simplify6-V2'} \\
	&= - \Xint{{}^{}_{\mathcolor{gray}{-}}}{23}{\bar{\bar{\bar{\chi}}}}^{\;\! \mathcolor{gray}{\omega} \textcolor{PineGreen}{\imath \jmath l}}_{\mathcolor{gray}{z} \textcolor{Maroon}{(2)}} ~{}^{\mathcolor{gray}{*}}_{\mathcolor{gray}{*}} \left[ \left( \Xint{{}^{}_{\mathcolor{gray}{-}}}{10}{\bar{g}}^{\;\!\mathcolor{gray}{\omega}}_{\;\! \mathcolor{gray}{z} \textcolor{PineGreen}{\jmath}} \mathbb{e}^{\mathbb{i} \Xint{\begin{smallmatrix} ~ \\ {}^{}_{\mathcolor{gray}{-}} \\ ~ \end{smallmatrix}}{15}{k}_{\symup{z} \textcolor{PineGreen}{\jmath}}^{\;\! \mathcolor{gray}{\omega}} \mathcolor{gray}{z}} \right) ~\mathcolor{gray}{\widetilde \circledast}~ \left( \Xint{{}^{}_{\mathcolor{gray}{-}}}{10}{\bar{g}}^{\;\!\mathcolor{gray}{\omega}}_{\;\! \mathcolor{gray}{z} \textcolor{PineGreen}{l}} \mathbb{e}^{\mathbb{i} \Xint{\begin{smallmatrix} ~ \\ {}^{}_{\mathcolor{gray}{-}} \\ ~ \end{smallmatrix}}{15}{k}_{\symup{z} \textcolor{PineGreen}{l}}^{\;\! \mathcolor{gray}{\omega}} \mathcolor{gray}{z}} \right) \right] \label{eq:nonlinear(2)-wave_wkrho-simplify6-V3'} ~.
\end{align}
\end{subequations}
尽管从 \bref{eq:barbar_k_check_j} 中的矩阵 $\Xint{\begin{smallmatrix} ~ \\ {}^{}_{\mathcolor{gray}{-}} \\ ~ \end{smallmatrix}}{15}{\bar{\bar{k}}}_{\textcolor{Maroon}{\symup{z}}}^{\;\! \mathcolor{gray}{\omega}}$ 到 \bref{eq:k_check_j} 中的标量 $\Xint{\begin{smallmatrix} ~ \\ {}^{}_{\mathcolor{gray}{-}} \\ ~ \end{smallmatrix}}{15}{k}_{\symup{z}}^{\;\! \mathcolor{gray}{\omega} \textcolor{PineGreen}{\jmath}}$,波动/\textcolor{PineGreen}{特征方程}的“\textcolor{gray}{自变量}”阶数下降了,但代价是它包含了每个\textcolor{Maroon}{模式} $\textcolor{PineGreen}{\imath}$(\textcolor{Plum}{线性}\textcolor{PineGreen}{晶体光学}),甚至\textcolor{Maroon}{模式}的组合 $\textcolor{PineGreen}{\imath},\textcolor{PineGreen}{\jmath},\textcolor{PineGreen}{l}$(\textcolor{Plum}{非线性}\textcolor{PineGreen}{晶体光学})。

\bref{ssec:E-waveq-linear} 表明,当试探解的形式,从\textcolor{Maroon}{矩阵指数} \bref{eq:vec-matrix_exp} 简化为 \textcolor{PineGreen}{线性叠加}的\textcolor{gray}{单色}\textcolor{PineGreen}{平面波基} 后,\textcolor{Plum}{线性}\textcolor{NavyBlue}{无源}\textcolor{Plum}{齐次}和\textcolor{Plum}{非线性}\textcolor{NavyBlue}{有源}\textcolor{Plum}{非齐次}波动方程,分别从 \bref{eq:nonlinear(2)-wave_wkrho-simplify6-L3,eq:nonlinear(2)-wave_wkrho-simplify6-V2} 过渡到 \bref{eq:nonlinear(2)-wave_wkrho-simplify6-L3',eq:nonlinear(2)-wave_wkrho-simplify6-V2'},使两个方程的解更容易获得。

以\textcolor{Plum}{线性}\textcolor{PineGreen}{晶体光学}为例,\bref{eq:nonlinear(2)-wave_wkrho-simplify6-L3'} 只是一个关于标量场 $\Xint{\begin{smallmatrix} ~ \\ {}^{}_{\mathcolor{gray}{-}} \\ ~ \end{smallmatrix}}{15}{k}_{\symup{z}}^{\;\! \mathcolor{gray}{\omega} \textcolor{PineGreen}{\jmath}}$ 的\textcolor{PineGreen}{六次方程},可以使用 \textcolor{PineGreen}{QR 分解}、\textcolor{PineGreen}{Schur 分解}、求根公式\Footnote{原则上,5 次方程及以上没有求根公式,但特定系数和结构下可能存在因式分解的办法和根式解。}、牛顿迭代法,获得全部 6 个\textcolor{PineGreen}{本征值} $\Xint{\begin{smallmatrix} ~ \\ {}^{}_{\mathcolor{gray}{-}} \\ ~ \end{smallmatrix}}{15}{k}_{\symup{z} \textcolor{PineGreen}{\jmath}}^{\;\! \mathcolor{gray}{\omega}}$(即 \bref{eq:kz-non_diagonalization} 中的 $\Xint{\begin{smallmatrix} ~ \\ {}^{}_{\mathcolor{gray}{-}} \\ ~ \end{smallmatrix}}{22}{\lambda}^{\;\!\mathcolor{gray}{\omega}}_{\textcolor{PineGreen}{\jmath}}$)\cite{raabMultipoleTheoryElectromagnetism2004}。将这 6 个\textcolor{PineGreen}{本征值}分别代入 $\Xint{\mathcolor{gray}{-}}{30}{\bar{\bar{L}}}^{\;\! \mathcolor{gray}{\omega} \textcolor{PineGreen}{\imath}}$ 的\textcolor{Maroon}{零空间}\cite{changWavePropagationBianisotropic2014,chernChiralSurfaceWaves2017},可以分别获得与之对应的 6 个\textcolor{PineGreen}{本征向量}/\textcolor{PineGreen}{偏振态} $\Xint{{}^{}_{\mathcolor{gray}{-}}}{10}{\bar{g}}^{\;\!\mathcolor{gray}{\omega}}_{\textcolor{PineGreen}{\jmath}}$(即 \bref{eq:kz-non_diagonalization} 中的 $\Xint{\begin{smallmatrix} ~ \\ {}^{}_{\mathcolor{gray}{-}} \\ ~ \end{smallmatrix}}{15}{\bar{v}}^{\;\!\mathcolor{gray}{\omega}}_{\textcolor{PineGreen}{\jmath}}$),并通过对比 $\textcolor{Plum}{\text{Re}} \left[ \Xint{\mathcolor{gray}{-}}{25}{S}^{\;\! \mathcolor{gray}{\omega}}_{\mathrm{z} \textcolor{PineGreen}{\jmath}} \right]$\Footnote{$\textcolor{Plum}{\text{Re}} \left[ \cdot \right]$ 代表对 $\cdot$ 取\textcolor{Plum}{实部}。\textcolor{PineGreen}{Poynting 矢量}一般采取 \textcolor{PineGreen}{Abraham 形式},即 $\Xint{\mathcolor{gray}{-}}{25}{\bar{S}}^{\;\! \mathcolor{gray}{\omega}} := \Xint{\mathcolor{gray}{-}}{25}{\bar{E}}^{\;\! \mathcolor{gray}{\omega}} \times \Xint{\mathcolor{gray}{-}}{23}{\bar{H}}^{\;\! \mathcolor{gray}{\omega}}$,这使得对\textcolor{PineGreen}{能流}密度矢量场 $\Xint{\mathcolor{gray}{-}}{25}{\bar{S}}^{\;\! \mathcolor{gray}{\omega}}$、能量密度标量场 $\Xint{\mathcolor{gray}{-}}{05}{W}^{\;\! \mathcolor{gray}{\omega}}$\cite{chen-zhuChenZhuxieUndergraduate_courses2024} 均需要评估原点平移不变性,正如\textcolor{PineGreen}{反射}场景一样\cite{raabMultipoleTheoryElectromagnetism2004}(\textcolor{PineGreen}{p 206});尽管波动方程本身是原点无关的(\textcolor{PineGreen}{透射}场景),但除此以外所有独立涉及\textcolor{Maroon}{本构关系}的部分都需要评估。} 与 $0$ 的大小,将其分类为 3 个\textcolor{Plum}{正向}、3 个\textcolor{Plum}{反向}传播的\textcolor{PineGreen}{本征波}\cite{asoubarSimulationBirefringenceEffects2015,zhangFullyVectorialSimulation2016,liReformulationFourierModal1998}。其中,总有 2 个\textcolor{Plum}{正向}、2 个\textcolor{Plum}{反向}传播的横模,剩下 1 个\textcolor{Plum}{正向}、1 个\textcolor{Plum}{反向}传播的纵模,这种电场偏振方向接近平行于\textcolor{PineGreen}{波矢} $\Xint{\begin{smallmatrix} ~ \\ {}^{}_{\mathcolor{gray}{-}} \\ ~ \end{smallmatrix}}{15}{\bar{k}}^{\;\! \mathcolor{gray}{\omega} \textcolor{PineGreen}{\jmath}}$ 传播方向的纵波,是“附加波”的一个例子,是高阶\textcolor{Plum}{多极}子带来的空间\textcolor{NavyBlue}{色散}的结果\cite{raabMultipoleTheoryElectromagnetism2004}(\textcolor{PineGreen}{p 116})。

由于试探解的核心从 \bref{ssec:Exp-solution-linear} 中的 \bref{eq:barbar_k_check_j} 过渡到了 \bref{ssec:E-waveq-linear} 中的 \bref{eq:k_check_j},这使得“\textcolor{PineGreen}{波矢} $\symup{z}$ 分量”从 2 阶 3 维张量(场)$\Xint{\begin{smallmatrix} ~ \\ {}^{}_{\mathcolor{gray}{-}} \\ ~ \end{smallmatrix}}{15}{\bar{\bar{k}}}_{\textcolor{Maroon}{\symup{z}}}^{\;\! \mathcolor{gray}{\omega}}$ 降阶至标量(场)$\Xint{\begin{smallmatrix} ~ \\ {}^{}_{\mathcolor{gray}{-}} \\ ~ \end{smallmatrix}}{15}{k}_{\symup{z}}^{\;\! \mathcolor{gray}{\omega} \textcolor{PineGreen}{\jmath}}$,收缩的解空间将所有子\textcolor{PineGreen}{特征向量} $\Xint{\begin{smallmatrix} ~ \\ {}^{}_{\mathcolor{gray}{-}} \\ ~ \end{smallmatrix}}{15}{\bar{\nu}}^{\;\!\mathcolor{gray}{\omega}}_{\textcolor{PineGreen}{\imath}}$ 排挤出去,作为“父\textcolor{PineGreen}{特征向量}” $\Xint{\begin{smallmatrix} ~ \\ {}^{}_{\mathcolor{gray}{-}} \\ ~ \end{smallmatrix}}{15}{\bar{v}}^{\;\!\mathcolor{gray}{\omega}}_{\textcolor{PineGreen}{\jmath}} = \Xint{{}^{}_{\mathcolor{gray}{-}}}{10}{\bar{g}}^{\;\!\mathcolor{gray}{\omega}}_{\textcolor{PineGreen}{\jmath}}$ 放进 $\Xint{{}^{}_{\mathcolor{gray}{-}}}{10}{\bar{g}}^{\;\!\mathcolor{gray}{\omega}}_{\;\! \mathcolor{gray}{z} \textcolor{PineGreen}{\jmath}}$ 中,并因此一旦简并则无法再被分辨;剩下的\textcolor{PineGreen}{特征值}仍保留在 $\Xint{\begin{smallmatrix} ~ \\ {}^{}_{\mathcolor{gray}{-}} \\ ~ \end{smallmatrix}}{15}{\bar{\bar{k}}}_{\textcolor{Maroon}{\symup{z}}}^{\;\! \mathcolor{gray}{\omega}}$ 的 \textcolor{PineGreen}{标准型} $\overline{\overline{\Xint{\begin{smallmatrix} ~ \\ {}^{}_{\mathcolor{gray}{-}} \\ ~ \end{smallmatrix}}{15}{k}_{\symup{z} \textcolor{PineGreen}{\jmath}}^{\;\! \mathcolor{gray}{\omega}}}}$ 的对角线元素 $\Xint{\begin{smallmatrix} ~ \\ {}^{}_{\mathcolor{gray}{-}} \\ ~ \end{smallmatrix}}{15}{k}_{\symup{z} \textcolor{PineGreen}{\jmath}}^{\;\! \mathcolor{gray}{\omega}}$ 中。然而,若 $\Xint{\begin{smallmatrix} ~ \\ {}^{}_{\mathcolor{gray}{-}} \\ ~ \end{smallmatrix}}{15}{\bar{\bar{k}}}_{\textcolor{Maroon}{\symup{z}}}^{\;\! \mathcolor{gray}{\omega}}$ 是\textcolor{Plum}{非厄米}且\textcolor{Plum}{非正规}的\cite{wiersigDistanceExceptionalPoints2022},标量\textcolor{PineGreen}{特征值} $\Xint{\begin{smallmatrix} ~ \\ {}^{}_{\mathcolor{gray}{-}} \\ ~ \end{smallmatrix}}{15}{k}_{\symup{z} \textcolor{PineGreen}{\jmath}}^{\;\! \mathcolor{gray}{\omega}}$ 与父\textcolor{PineGreen}{特征向量} $\Xint{\begin{smallmatrix} ~ \\ {}^{}_{\mathcolor{gray}{-}} \\ ~ \end{smallmatrix}}{15}{\bar{v}}^{\;\!\mathcolor{gray}{\omega}}_{\textcolor{PineGreen}{\jmath}} = \Xint{{}^{}_{\mathcolor{gray}{-}}}{10}{\bar{g}}^{\;\!\mathcolor{gray}{\omega}}_{\textcolor{PineGreen}{\jmath}}$ 的组合,便再也无法比肩于 \bref{eq:kz-Jordan_decompose} 中 $\Xint{\begin{smallmatrix} ~ \\ {}^{}_{\mathcolor{gray}{-}} \\ ~ \end{smallmatrix}}{15}{\bar{\bar{k}}}_{\textcolor{Maroon}{\symup{z}}}^{\;\! \mathcolor{gray}{\omega}}$ 的 \textcolor{PineGreen}{Jordan 标准型} $\Xint{\mathcolor{gray}{-}}{25}{\bar{\bar{J}}}^{\;\!\mathcolor{gray}{\omega}} = \overline{\overline{\Xint{\begin{smallmatrix} ~ \\ {}^{}_{\mathcolor{gray}{-}} \\ ~ \end{smallmatrix}}{15}{k}_{\symup{z} \textcolor{PineGreen}{\jmath}}^{\;\! \mathcolor{gray}{\omega}}}} + \Xint{\mathcolor{gray}{-}}{18}{\bar{\bar{M}}}^{\;\!\mathcolor{gray}{\omega}}$ 和相应子\textcolor{PineGreen}{特征向量} $\Xint{\begin{smallmatrix} ~ \\ {}^{}_{\mathcolor{gray}{-}} \\ ~ \end{smallmatrix}}{15}{\bar{\nu}}^{\;\!\mathcolor{gray}{\omega}}_{\textcolor{PineGreen}{\imath}}$ 的组合。

在 \bref{eq:weak_modulated_linear_sus} 和 \bref{eq:L+V-decompose}(或\bref{eq:L+V-decompose2})的条件下,进一步假设:
\begin{align} \label{eq:L+V-polar}
	\textbf{\text{在 \textcolor{gray}{2D 傅立叶空间}中,\textcolor{Plum}{线性}、\textcolor{Plum}{非线性}\textcolor{PineGreen}{晶体光学}\textcolor{PineGreen}{本征模}均不含}} \mathcolor{gray}{z}~,
\end{align}
即文献\cite{xieAnalytic3DVector}中的 Eq. (S55b)。该假设在 \bref{eq:weak_modulated_linear_sus} 条件下的\textcolor{Plum}{线性}\textcolor{PineGreen}{晶体光学}中基本成立,但它在\textcolor{Plum}{非线性}\textcolor{PineGreen}{晶体光学}中是否成立,还须进一步探讨。在此条件下,\textcolor{PineGreen}{线性叠加}的\textcolor{gray}{单色}\textcolor{PineGreen}{平面波基} \bref{eq:vec-plane_wave_basis} 的\textcolor{NavyBlue}{非相位部分}(后简称为电场的“\textcolor{Maroon}{时空谱}”) $\Xint{{}^{}_{\mathcolor{gray}{-}}}{10}{\bar{g}}^{\;\!\mathcolor{gray}{\omega} \textcolor{PineGreen}{\jmath}}_{\;\! \mathcolor{gray}{z}}$ 总可以因式分解为“\textcolor{PineGreen}{本征复振幅} $\Xint{\begin{smallmatrix} ~ \\ {}^{}_{\mathcolor{gray}{-}} \\ ~ \end{smallmatrix}}{09}{\mathtt{g}}^{\;\!\mathcolor{gray}{\omega} \textcolor{PineGreen}{\jmath}}_{\;\! \mathcolor{gray}{z}}$· \textcolor{PineGreen}{本征偏振态} $\Xint{{}^{}_{\mathcolor{gray}{-}}}{10}{\bar{g}}^{\;\!\mathcolor{gray}{\omega} \textcolor{PineGreen}{\jmath}}$”之积(\textcolor{Maroon}{矩阵指数}形式的 \bref{eq:vec-matrix_exp} 也可以,但没必要分解):
\begin{subequations} \label{eq:amp_polar}
\begin{align}
	\Xint{{}^{}_{\mathcolor{gray}{-}}}{10}{\bar{g}}^{\;\!\mathcolor{gray}{\omega} \textcolor{PineGreen}{\jmath}}_{\;\! \mathcolor{gray}{z}} &:= \Xint{\begin{smallmatrix} ~ \\ {}^{}_{\mathcolor{gray}{-}} \\ ~ \end{smallmatrix}}{09}{\mathtt{g}}^{\;\!\mathcolor{gray}{\omega} \textcolor{PineGreen}{\jmath}}_{\;\! \mathcolor{gray}{z}} \Xint{{}^{}_{\mathcolor{gray}{-}}}{10}{\bar{g}}^{\;\!\mathcolor{gray}{\omega} \textcolor{PineGreen}{\jmath}} ~, \label{eq:vec-amp_polar} \\
	\Xint{{}^{}_{\mathcolor{gray}{-}}}{08}{g}^{\;\!\mathcolor{gray}{\omega} \textcolor{PineGreen}{\jmath}}_{\;\! \hat{1} \mathcolor{gray}{z}} &:= \Xint{\begin{smallmatrix} ~ \\ {}^{}_{\mathcolor{gray}{-}} \\ ~ \end{smallmatrix}}{09}{\mathtt{g}}^{\;\!\mathcolor{gray}{\omega} \textcolor{PineGreen}{\jmath}}_{\;\! \mathcolor{gray}{z}} \Xint{{}^{}_{\mathcolor{gray}{-}}}{10}{g}^{\;\!\mathcolor{gray}{\omega} \textcolor{PineGreen}{\jmath}}_{\;\! \hat{1}} ~, \label{eq:components-amp_polar}
\end{align}
\end{subequations}
即三分量之间的比例,完全由不含 $\mathcolor{gray}{z}$ 的\textcolor{PineGreen}{本征偏振态} $\Xint{{}^{}_{\mathcolor{gray}{-}}}{10}{\bar{g}}^{\;\!\mathcolor{gray}{\omega} \textcolor{PineGreen}{\jmath}}$ 独立控制(与\textcolor{PineGreen}{本征复振幅} $\Xint{\begin{smallmatrix} ~ \\ {}^{}_{\mathcolor{gray}{-}} \\ ~ \end{smallmatrix}}{09}{\mathtt{g}}^{\;\!\mathcolor{gray}{\omega} \textcolor{PineGreen}{\jmath}}_{\;\! \mathcolor{gray}{z}}$ 无关),而三分量的\textcolor{Maroon}{幅值}却共享同一个含 $\mathcolor{gray}{z}$ 的增幅因子,即按\textcolor{PineGreen}{本征复振幅} $\Xint{\begin{smallmatrix} ~ \\ {}^{}_{\mathcolor{gray}{-}} \\ ~ \end{smallmatrix}}{09}{\mathtt{g}}^{\;\!\mathcolor{gray}{\omega} \textcolor{PineGreen}{\jmath}}_{\;\! \mathcolor{gray}{z}}$ 的比例同步缩放。注意,这里只给 \textcolor{PineGreen}{本征复振幅} $\Xint{\begin{smallmatrix} ~ \\ {}^{}_{\mathcolor{gray}{-}} \\ ~ \end{smallmatrix}}{09}{\mathtt{g}}^{\;\!\mathcolor{gray}{\omega} \textcolor{PineGreen}{\jmath}}_{\;\! \mathcolor{gray}{z}}$ 单独赋予了字体 \textbackslash mathtt,见 \bref{hook:mathtt}。

从\textcolor{Plum}{自由度}上看,\bref{eq:amp_polar} 左侧,在 $\mathcolor{gray}{z}$ 上的\textcolor{Plum}{自由度}等于矢量的分量个数,即\textcolor{Plum}{维度} 3;而该方程的右侧,在 $\mathcolor{gray}{z}$ 上的\textcolor{Plum}{自由度},为 \textcolor{PineGreen}{本征偏振态} $\Xint{{}^{}_{\mathcolor{gray}{-}}}{10}{\bar{g}}^{\;\!\mathcolor{gray}{\omega} \textcolor{PineGreen}{\jmath}}$ 的\textcolor{Plum}{自由度} $+$ \textcolor{PineGreen}{本征复振幅} $\Xint{\begin{smallmatrix} ~ \\ {}^{}_{\mathcolor{gray}{-}} \\ ~ \end{smallmatrix}}{09}{\mathtt{g}}^{\;\!\mathcolor{gray}{\omega} \textcolor{PineGreen}{\jmath}}_{\;\! \mathcolor{gray}{z}}$ 的\textcolor{Plum}{自由度} $= 2 + 1 = 3$。因此,从这个角度,该分解并没有降低表示\textcolor{Plum}{自由度},所以是可接受的。在 \bref{eq:amp_polar} 的分解表示下,\bref{eq:vec-plane_wave_basis} 进一步细化为
\begin{subequations} \label{eq:amp_polar_phase}
\begin{align}
	\Xint{\mathcolor{gray}{-}}{30}{\bar{E}}^{\;\!\mathcolor{gray}{\omega}}_{\;\! \mathcolor{gray}{z}} &:= \leftindex_{\textcolor{PineGreen}{\jmath}} \;\! \Xint{\mathcolor{gray}{-}}{30}{\bar{E}}^{\;\!\mathcolor{gray}{\omega} \textcolor{PineGreen}{\jmath}}_{\;\! \mathcolor{gray}{z}} := \Xint{\begin{smallmatrix} ~ \\ {}^{}_{\mathcolor{gray}{-}} \\ ~ \end{smallmatrix}}{09}{\mathtt{g}}^{\;\!\mathcolor{gray}{\omega} \textcolor{PineGreen}{\jmath}}_{\;\! \mathcolor{gray}{z}} \Xint{{}^{}_{\mathcolor{gray}{-}}}{10}{\bar{g}}^{\;\!\mathcolor{gray}{\omega} \textcolor{PineGreen}{\jmath}} \mathbb{e}^{\mathbb{i} \Xint{\begin{smallmatrix} ~ \\ {}^{}_{\mathcolor{gray}{-}} \\ ~ \end{smallmatrix}}{15}{k}_{\symup{z} \textcolor{PineGreen}{\jmath}}^{\;\! \mathcolor{gray}{\omega}} \mathcolor{gray}{z}} ~, \label{eq:vec-amp_polar_phase} \\
	\Xint{\mathcolor{gray}{-}}{30}{E}^{\;\!\mathcolor{gray}{\omega}}_{\;\! \hat{1} \mathcolor{gray}{z}} &:= \leftindex_{\textcolor{PineGreen}{\jmath}} \;\! \Xint{\mathcolor{gray}{-}}{30}{E}^{\;\!\mathcolor{gray}{\omega} \textcolor{PineGreen}{\jmath}}_{\;\! \hat{1} \mathcolor{gray}{z}} := \Xint{\begin{smallmatrix} ~ \\ {}^{}_{\mathcolor{gray}{-}} \\ ~ \end{smallmatrix}}{09}{\mathtt{g}}^{\;\!\mathcolor{gray}{\omega} \textcolor{PineGreen}{\jmath}}_{\;\! \mathcolor{gray}{z}} \Xint{{}^{}_{\mathcolor{gray}{-}}}{10}{g}^{\;\!\mathcolor{gray}{\omega} \textcolor{PineGreen}{\jmath}}_{\;\! \hat{1}} \mathbb{e}^{\mathbb{i} \Xint{\begin{smallmatrix} ~ \\ {}^{}_{\mathcolor{gray}{-}} \\ ~ \end{smallmatrix}}{15}{k}_{\symup{z} \textcolor{PineGreen}{\jmath}}^{\;\! \mathcolor{gray}{\omega}} \mathcolor{gray}{z}} ~, \label{eq:components-amp_polar_phase}
\end{align}
\end{subequations}
注,在\textcolor{Maroon}{傅立叶频谱域},电场\textcolor{PineGreen}{本征波} $\Xint{\mathcolor{gray}{-}}{25}{\bar{E}}^{\;\!\mathcolor{gray}{\omega} \textcolor{PineGreen}{\jmath}}_{\;\! \mathcolor{gray}{z}}$ 以外的电磁场量的\textcolor{PineGreen}{本征波} $\Xint{\mathcolor{gray}{-}}{25}{\bar{D}}^{\;\!\mathcolor{gray}{\omega} \textcolor{PineGreen}{\jmath}}_{\;\! \mathcolor{gray}{z}}, \Xint{\mathcolor{gray}{-}}{25}{\bar{B}}^{\;\!\mathcolor{gray}{\omega} \textcolor{PineGreen}{\jmath}}_{\;\! \mathcolor{gray}{z}}, \Xint{\mathcolor{gray}{-}}{22}{\bar{H}}^{\;\!\mathcolor{gray}{\omega} \textcolor{PineGreen}{\jmath}}_{\;\! \mathcolor{gray}{z}}$,(形式上)都可以按 \bref{eq:matrix_exp,eq:plane_wave_basis} 进行基构建/分解为对应的\textcolor{NavyBlue}{非相位}·\textcolor{NavyBlue}{相位}两部分之积(之和),并按 \bref{eq:amp_polar_phase} 分解为\textcolor{PineGreen}{本征复振幅}·\textcolor{PineGreen}{本征偏振态}·\textcolor{NavyBlue}{相位}三部分之积之和, --- 通过将\textcolor{PineGreen}{本征波}的\textcolor{NavyBlue}{非相位}部分 $\Xint{\begin{smallmatrix} ~ \\ {}^{}_{\mathcolor{gray}{-}} \\ ~ \end{smallmatrix}}{09}{\bar{d}}^{\;\!\mathcolor{gray}{\omega} \textcolor{PineGreen}{\jmath}}_{\;\! \mathcolor{gray}{z}}, \Xint{\begin{smallmatrix} ~ \\ {}^{}_{\mathcolor{gray}{-}} \\ ~ \end{smallmatrix}}{10}{\bar{b}}^{\;\!\mathcolor{gray}{\omega} \textcolor{PineGreen}{\jmath}}_{\;\! \mathcolor{gray}{z}}, \Xint{\begin{smallmatrix} ~ \\ {}^{}_{\mathcolor{gray}{-}} \\ ~ \end{smallmatrix}}{11}{\bar{h}}^{\;\!\mathcolor{gray}{\omega} \textcolor{PineGreen}{\jmath}}_{\;\! \mathcolor{gray}{z}}$ 分解为\textcolor{PineGreen}{本征复振幅}·\textcolor{PineGreen}{本征偏振态}两部分之积 $\Xint{\begin{smallmatrix} ~ \\ {}^{}_{\mathcolor{gray}{-}} \\ ~ \end{smallmatrix}}{09}{\bar{d}}^{\;\!\mathcolor{gray}{\omega} \textcolor{PineGreen}{\jmath}}_{\;\! \mathcolor{gray}{z}} = \Xint{\mathcolor{gray}{-}}{02}{\mathtt{d}}^{\;\!\mathcolor{gray}{\omega} \textcolor{PineGreen}{\jmath}}_{\;\! \mathcolor{gray}{z}} \Xint{\begin{smallmatrix} ~ \\ {}^{}_{\mathcolor{gray}{-}} \\ ~ \end{smallmatrix}}{09}{\bar{d}}^{\;\!\mathcolor{gray}{\omega} \textcolor{PineGreen}{\jmath}}, \Xint{\begin{smallmatrix} ~ \\ {}^{}_{\mathcolor{gray}{-}} \\ ~ \end{smallmatrix}}{10}{\bar{b}}^{\;\!\mathcolor{gray}{\omega} \textcolor{PineGreen}{\jmath}}_{\;\! \mathcolor{gray}{z}} = \Xint{\mathcolor{gray}{-}}{02}{\mathtt{b}}^{\;\!\mathcolor{gray}{\omega} \textcolor{PineGreen}{\jmath}}_{\;\! \mathcolor{gray}{z}} \Xint{\begin{smallmatrix} ~ \\ {}^{}_{\mathcolor{gray}{-}} \\ ~ \end{smallmatrix}}{10}{\bar{b}}^{\;\!\mathcolor{gray}{\omega} \textcolor{PineGreen}{\jmath}}, \Xint{\begin{smallmatrix} ~ \\ {}^{}_{\mathcolor{gray}{-}} \\ ~ \end{smallmatrix}}{11}{\bar{h}}^{\;\!\mathcolor{gray}{\omega} \textcolor{PineGreen}{\jmath}}_{\;\! \mathcolor{gray}{z}} = \Xint{\mathcolor{gray}{-}}{02}{\mathtt{h}}^{\;\!\mathcolor{gray}{\omega} \textcolor{PineGreen}{\jmath}}_{\;\! \mathcolor{gray}{z}} \Xint{\begin{smallmatrix} ~ \\ {}^{}_{\mathcolor{gray}{-}} \\ ~ \end{smallmatrix}}{11}{\bar{h}}^{\;\!\mathcolor{gray}{\omega} \textcolor{PineGreen}{\jmath}}$,像 \bref{eq:amp_polar} 一样。

当 \bref{eq:vec-plane_wave_basis} 进一步细分为 \bref{eq:amp_polar_phase} 后,相应地 \bref{eq:nonlinear(2)-wave_wkrho-simplify6'} 也具体为:
\begin{subequations} \label{eq:nonlinear(2)-wave_wkrho-simplify6''}
\begin{align}
	\!\!\!\! \Xint{\mathcolor{gray}{-}}{32}{\bar{\bar{L}}}^{\;\! \mathcolor{gray}{\omega} \textcolor{PineGreen}{\imath}} \Xint{{}^{}_{\mathcolor{gray}{-}}}{10}{\bar{g}}^{\;\!\mathcolor{gray}{\omega} \textcolor{PineGreen}{\imath}} &= \left( \bar{\bar{\varepsilon}}^{\;\! \mathcolor{gray}{\omega}}_{\textcolor{Maroon}{(1)}} + \bar{\bar{\zeta}}^{\;\! \mathcolor{gray}{\omega} \mathcolor{gray}{\check{1}}}_{\textcolor{Maroon}{(1)}} \mathbb{i} \Xint{\begin{smallmatrix} ~ \\ {}^{}_{\mathcolor{gray}{-}} \\ ~ \end{smallmatrix}}{15}{k}_{\;\! \mathcolor{gray}{\check{1}}}^{\;\! \mathcolor{gray}{\omega} \textcolor{PineGreen}{\imath}} - \bar{\bar{\zeta}}^{\;\! \mathcolor{gray}{\omega} \mathcolor{gray}{\check{1} \check{2}}}_{\textcolor{Maroon}{(1)}} \Xint{\begin{smallmatrix} ~ \\ {}^{}_{\mathcolor{gray}{-}} \\ ~ \end{smallmatrix}}{15}{k}_{\;\! \mathcolor{gray}{\check{2}}}^{\;\! \mathcolor{gray}{\omega} \textcolor{PineGreen}{\imath}} \Xint{\begin{smallmatrix} ~ \\ {}^{}_{\mathcolor{gray}{-}} \\ ~ \end{smallmatrix}}{15}{k}_{\;\! \mathcolor{gray}{\check{1}}}^{\;\! \mathcolor{gray}{\omega} \textcolor{PineGreen}{\imath}} \right) \Xint{{}^{}_{\mathcolor{gray}{-}}}{10}{\bar{g}}^{\;\!\mathcolor{gray}{\omega} \textcolor{PineGreen}{\imath}} = \bar{0} ~, \label{eq:nonlinear(2)-wave_wkrho-simplify6-L3''} \\
	\!\!\!\! \Xint{\mathcolor{gray}{-}}{25}{\bar{\bar{V}}}^{\;\! \mathcolor{gray}{\omega} \textcolor{PineGreen}{\imath}} \Xint{{}^{}_{\mathcolor{gray}{-}}}{10}{\bar{g}}^{\;\!\mathcolor{gray}{\omega} \textcolor{PineGreen}{\imath}}_{\;\! \mathcolor{gray}{z}}
	&= \left[ \left( \bar{\bar{\zeta}}^{\;\! \mathcolor{gray}{\omega} \mathcolor{gray}{\symup{z}}}_{\textcolor{Maroon}{(1)}} + \bar{\bar{\zeta}}^{\;\! \mathcolor{gray}{\omega} \mathcolor{gray}{\check{1} \symup{z}}}_{\textcolor{Maroon}{(1)}} \mathbb{i} \Xint{\begin{smallmatrix} ~ \\ {}^{}_{\mathcolor{gray}{-}} \\ ~ \end{smallmatrix}}{15}{k}_{\;\! \mathcolor{gray}{\check{1}}}^{\;\! \mathcolor{gray}{\omega} \textcolor{PineGreen}{\imath}} + \bar{\bar{\zeta}}^{\;\! \mathcolor{gray}{\omega} \mathcolor{gray}{\symup{z} \check{2}}}_{\textcolor{Maroon}{(1)}} \mathbb{i} \Xint{\begin{smallmatrix} ~ \\ {}^{}_{\mathcolor{gray}{-}} \\ ~ \end{smallmatrix}}{15}{k}_{\;\! \mathcolor{gray}{\check{2}}}^{\;\! \mathcolor{gray}{\omega} \textcolor{PineGreen}{\imath}} \right) \mathcolor{gray}{\nabla_z} + \bar{\bar{\zeta}}^{\;\! \mathcolor{gray}{\omega} \mathcolor{gray}{\symup{z} \symup{z}}}_{\textcolor{Maroon}{(1)}} \mathcolor{gray}{\nabla_z^2} \right] \Xint{\begin{smallmatrix} ~ \\ {}^{}_{\mathcolor{gray}{-}} \\ ~ \end{smallmatrix}}{09}{\mathtt{g}}^{\;\!\mathcolor{gray}{\omega} \textcolor{PineGreen}{\imath}}_{\;\! \mathcolor{gray}{z}} \Xint{{}^{}_{\mathcolor{gray}{-}}}{10}{\bar{g}}^{\;\!\mathcolor{gray}{\omega} \textcolor{PineGreen}{\imath}} \mathbb{e}^{\mathbb{i} \Xint{\begin{smallmatrix} ~ \\ {}^{}_{\mathcolor{gray}{-}} \\ ~ \end{smallmatrix}}{15}{k}_{\symup{z}}^{\;\! \mathcolor{gray}{\omega} \textcolor{PineGreen}{\imath}} \mathcolor{gray}{z}} \label{eq:nonlinear(2)-wave_wkrho-simplify6-V2''} \\
	&= - \Xint{{}^{}_{\mathcolor{gray}{-}}}{23}{\bar{\bar{\bar{\chi}}}}^{\;\! \mathcolor{gray}{\omega} \textcolor{PineGreen}{\imath \jmath l}}_{\mathcolor{gray}{z} \textcolor{Maroon}{(2)}} ~{}^{\mathcolor{gray}{*}}_{\mathcolor{gray}{*}} \left[ \left( \Xint{\begin{smallmatrix} ~ \\ {}^{}_{\mathcolor{gray}{-}} \\ ~ \end{smallmatrix}}{09}{\mathtt{g}}^{\;\!\mathcolor{gray}{\omega}}_{\;\! \mathcolor{gray}{z} \textcolor{PineGreen}{\jmath}} \Xint{{}^{}_{\mathcolor{gray}{-}}}{10}{\bar{g}}^{\;\!\mathcolor{gray}{\omega}}_{\;\! \textcolor{PineGreen}{\jmath}} \mathbb{e}^{\mathbb{i} \Xint{\begin{smallmatrix} ~ \\ {}^{}_{\mathcolor{gray}{-}} \\ ~ \end{smallmatrix}}{15}{k}_{\symup{z} \textcolor{PineGreen}{\jmath}}^{\;\! \mathcolor{gray}{\omega}} \mathcolor{gray}{z}} \right) ~\mathcolor{gray}{\widetilde \circledast}~ \left( \Xint{\begin{smallmatrix} ~ \\ {}^{}_{\mathcolor{gray}{-}} \\ ~ \end{smallmatrix}}{09}{\mathtt{g}}^{\;\!\mathcolor{gray}{\omega}}_{\;\! \mathcolor{gray}{z} \textcolor{PineGreen}{l}} \Xint{{}^{}_{\mathcolor{gray}{-}}}{10}{\bar{g}}^{\;\!\mathcolor{gray}{\omega}}_{\;\! \textcolor{PineGreen}{l}} \mathbb{e}^{\mathbb{i} \Xint{\begin{smallmatrix} ~ \\ {}^{}_{\mathcolor{gray}{-}} \\ ~ \end{smallmatrix}}{15}{k}_{\symup{z} \textcolor{PineGreen}{l}}^{\;\! \mathcolor{gray}{\omega}} \mathcolor{gray}{z}} \right) \right] \label{eq:nonlinear(2)-wave_wkrho-simplify6-V3''} ~.
\end{align}
\end{subequations}
可以看到,相比 \bref{eq:nonlinear(2)-wave_wkrho-simplify6-L4,eq:nonlinear(2)-wave_wkrho-simplify6-V2}, \bref{eq:nonlinear(2)-wave_wkrho-simplify6-L3'',eq:nonlinear(2)-wave_wkrho-simplify6-V2''} 更容易求解。然而,随之而来的代价便是(仍以\textcolor{Plum}{线性}\textcolor{PineGreen}{晶体光学}所涉及的 $\Xint{\mathcolor{gray}{-}}{30}{\bar{\bar{L}}}^{\;\! \mathcolor{gray}{\omega} \textcolor{PineGreen}{\imath}} \Xint{{}^{}_{\mathcolor{gray}{-}}}{10}{\bar{g}}^{\;\!\mathcolor{gray}{\omega} \textcolor{PineGreen}{\imath}} = \bar{0}$ 为例),由于\textcolor{PineGreen}{本征值}-\textcolor{PineGreen}{本征向量}对 $\Xint{\begin{smallmatrix} ~ \\ {}^{}_{\mathcolor{gray}{-}} \\ ~ \end{smallmatrix}}{15}{k}_{\symup{z}}^{\;\! \mathcolor{gray}{\omega} \textcolor{PineGreen}{\imath}}, \Xint{{}^{}_{\mathcolor{gray}{-}}}{10}{\bar{g}}^{\;\!\mathcolor{gray}{\omega} \textcolor{PineGreen}{\imath}}$ 的一一对应关系,当\textcolor{PineGreen}{本征值} $\Xint{\begin{smallmatrix} ~ \\ {}^{}_{\mathcolor{gray}{-}} \\ ~ \end{smallmatrix}}{15}{k}_{\symup{z}}^{\;\! \mathcolor{gray}{\omega} \textcolor{PineGreen}{\imath}}, \Xint{\begin{smallmatrix} ~ \\ {}^{}_{\mathcolor{gray}{-}} \\ ~ \end{smallmatrix}}{15}{k}_{\symup{z}}^{\;\! \mathcolor{gray}{\omega} \textcolor{PineGreen}{\jmath}}$ 简并/相等时,\textcolor{PineGreen}{本征向量} $\Xint{{}^{}_{\mathcolor{gray}{-}}}{10}{\bar{g}}^{\;\!\mathcolor{gray}{\omega} \textcolor{PineGreen}{\imath}}, \Xint{{}^{}_{\mathcolor{gray}{-}}}{10}{\bar{g}}^{\;\!\mathcolor{gray}{\omega} \textcolor{PineGreen}{\jmath}}$ 要么可以是任意的非平行基(对应\textcolor{PineGreen}{恶魔点}),要么也是简并/平行的(对应\textcolor{PineGreen}{例外点}/\textcolor{PineGreen}{光学奇点}),--- 以至于在\textcolor{PineGreen}{恶魔点}(\textcolor{PineGreen}{DPs})处和\textcolor{PineGreen}{例外点}(\textcolor{PineGreen}{EPs})附近,需要对\textcolor{PineGreen}{特征向量}手动添加额外的特殊处理,甚至即使施行了“外科手术”,也无济于事:特别是在\textcolor{PineGreen}{例外点}附近。此时,在\textcolor{PineGreen}{例外点}附近的电磁场传播的动力学演化过程,将无法通过\textcolor{PineGreen}{线性叠加}的\textcolor{gray}{单色}\textcolor{PineGreen}{平面波基} \bref{eq:vec-plane_wave_basis} 及其配套的方法计算,而被彻底“禁止察看”。这便是 \bref{eq:k_check_j} 相对于 \bref{eq:barbar_k_check_j} 的主要和几乎唯一的劣势(另一个劣势是必须求解\textcolor{PineGreen}{本征系统},而\textcolor{Maroon}{矩阵指数}允许不关注\textcolor{PineGreen}{本征系统}\cite{zarifiPlaneWaveReflection2014})。

然而,对于\textcolor{Plum}{线性}\textcolor{PineGreen}{晶体光学}而言,在\textcolor{gray}{空间频率}域 $\mathcolor{gray}{\bar{k}_{\symup{\rho}}} \in \mathcolor{gray}{\bar{\mathbb{R}}_{\textcolor{Plum}{2}}}$ 内,由于\textcolor{PineGreen}{例外点}本身通常只以有限个点(一般最多 8 个\cite{berryOpticalSingularitiesBirefringent2003,berryOpticalSingularitiesBianisotropic2005,sturmPropagationElectromagneticWaves,sturmSingularOpticalAxes2016a,grundmannOpticallyAnisotropicMedia2017})存在,少数情况下才以\textcolor{Plum}{线}/\textcolor{Plum}{面}的形式覆盖不可计算区域\cite{xieAnalytic3DVector}。因此,大多数情况下,\textcolor{gray}{空间频率}域 $\mathcolor{gray}{\bar{k}_{\symup{\rho}}} \in \mathcolor{gray}{\bar{\mathbb{R}}_{\textcolor{Plum}{2}}}$ 内剩余的大面积无\textcolor{PineGreen}{光学奇点}的广袤空旷区域是可计算的,以至于减轻了采取\textcolor{PineGreen}{线性叠加}的\textcolor{gray}{单色}\textcolor{PineGreen}{平面波基}形式的试探解 \bref{eq:vec-plane_wave_basis} 的问题的严重性。除此之外,该基没有其他劣势,且除此之外几乎全是优势(如方便计算、遵循传统、比较\textcolor{NavyBlue}{物理}、易于理解)。在对比并评估完成 \bref{ssec:Exp-solution-linear,ssec:E-waveq-linear} 中\textcolor{PineGreen}{两种基}各自的利弊后,本文后续选择 \bref{eq:plane_wave_basis,eq:amp_polar_phase} 而不是 \bref{eq:matrix_exp} 作为\textcolor{PineGreen}{本征模}/波/基/态,并因此保持\textcolor{PineGreen}{光学奇点}内部的不可探测性(像黑洞),即简并\textcolor{PineGreen}{本征向量}的不可区分性。

%\clearpage
%\vspace*{-8.5em}

\vspace*{-4.5em}

\marginLeft[-2.4em]{ssec:E-waveq-nonlinear}\subsection{非线性非均匀局域单色平面波 $\bar{E}$ 波动方程}\label{ssec:E-waveq-nonlinear}

上面两小节,即\bref{ssec:Exp-solution-linear,ssec:E-waveq-linear},从\textcolor{Plum}{线性}\textcolor{PineGreen}{晶体光学}的角度,对比并确定/选用了\textcolor{Maroon}{傅立叶基}/\textcolor{Maroon}{分量} $\Xint{\mathcolor{gray}{-}}{25}{\bar{E}}^{\;\!\mathcolor{gray}{\omega}}_{\;\! \mathcolor{gray}{z}}$ 的形式为 \bref{eq:plane_wave_basis,eq:amp_polar_phase}。现在这一节将转而预处理\textcolor{Plum}{非线性}\textcolor{PineGreen}{晶体光学}的\textcolor{NavyBlue}{第一性原理} \bref{eq:nonlinear(2)-wave_wkrho-simplify6-V2',eq:nonlinear(2)-wave_wkrho-simplify6-V2''}。

对 \bref{eq:nonlinear(2)-wave_wkrho-simplify6-V2'} 应用\textcolor{NavyBlue}{缓变振幅近似},即忽略关于 $\mathcolor{gray}{z}$ 坐标的二阶偏导项,得到
\begin{align} \label{eq:nonlinear(2)-wave_wkrho-simplify6-V2'-SVA}
	\Xint{\mathcolor{gray}{-}}{25}{\bar{\bar{\mathsfit{V}}}}^{\;\! \mathcolor{gray}{\omega} \textcolor{PineGreen}{\imath}}_{\textcolor{Maroon}{\mathbb{1}}} \mathcolor{gray}{\nabla_z} \Xint{{}^{}_{\mathcolor{gray}{-}}}{10}{\bar{g}}^{\;\!\mathcolor{gray}{\omega} \textcolor{PineGreen}{\imath}}_{\;\! \mathcolor{gray}{z}} = - \Xint{\mathcolor{gray}{-}}{30}{\bar{P}}^{\;\! \mathcolor{gray}{\omega} \textcolor{PineGreen}{\imath}}_{\;\! \mathcolor{gray}{z} \textcolor{Maroon}{(2)}} \big/ \mathbb{e}^{\mathbb{i} \Xint{\begin{smallmatrix} ~ \\ {}^{}_{\mathcolor{gray}{-}} \\ ~ \end{smallmatrix}}{15}{k}_{\symup{z}}^{\;\! \mathcolor{gray}{\omega} \textcolor{PineGreen}{\imath}} \mathcolor{gray}{z}} ~,
\end{align}
其中,类似 \bref{eq:vec-DP^(2)-p_pp} 地,定义了 $\left( \mathcolor{gray}{\omega}, \mathcolor{gray}{\bar{k}_{\symup{\rho}}} \right)$ 域\textcolor{PineGreen}{平面波基}下的二阶\textcolor{Plum}{局域}\textcolor{Plum}{非线性}\textcolor{NavyBlue}{电偶-$(\text{电偶}\otimes\text{电偶})$}极矩场\Footnote{相对于最具体/完整的矢量形式 $\Xint{\mathcolor{gray}{-}}{24}{\bar{P}}^{\;\! \textcolor{Maroon}{(2)} \mathcolor{gray}{\omega} \textcolor{PineGreen}{\imath}}_{\;\! \mathcolor{gray}{z} \textcolor{NavyBlue}{\text{pp}}} = \Xint{{}^{}_{\mathcolor{gray}{-}}}{23}{\bar{\bar{\bar{\chi}}}}^{\;\! \textcolor{NavyBlue}{\text{p}} \mathcolor{gray}{\omega} \textcolor{PineGreen}{\imath \jmath l}}_{\mathcolor{gray}{z} \textcolor{Maroon}{(2)} \textcolor{NavyBlue}{\text{pp}}} ~{}^{\mathcolor{gray}{*}}_{\mathcolor{gray}{*}} \cdots$ 和分量形式 $\Xint{\mathcolor{gray}{-}}{24}{P}^{\;\! \textcolor{Maroon}{(2)} \mathcolor{gray}{\omega} \textcolor{PineGreen}{\imath}}_{\;\! \symup{\iota}\mathcolor{gray}{z} \textcolor{NavyBlue}{\text{pp}}} = \Xint{{}^{}_{\mathcolor{gray}{-}}}{23}{\chi}^{\;\! \textcolor{NavyBlue}{\text{p}} \mathcolor{gray}{\omega} \hat{1} \hat{2} \textcolor{PineGreen}{\imath}}_{\;\! \symup{\iota} \mathcolor{gray}{z} \textcolor{Maroon}{(2)} \textcolor{NavyBlue}{\text{pp}} \textcolor{PineGreen}{\jmath l}} \mathcolor{gray}{*} \cdots$,省略了一些角标。}
\begin{subequations} \label{eq:DP^(2)-plane_wave_basis-p_pp}
\begin{align}
	\Xint{\mathcolor{gray}{-}}{30}{\bar{D}}^{\;\! \mathcolor{gray}{\omega} \textcolor{PineGreen}{\imath}}_{\;\! \mathcolor{gray}{z} \textcolor{Maroon}{(2)}} &= \Xint{\mathcolor{gray}{-}}{30}{\bar{P}}^{\;\! \mathcolor{gray}{\omega} \textcolor{PineGreen}{\imath}}_{\;\! \mathcolor{gray}{z} \textcolor{Maroon}{(2)}} = \Xint{{}^{}_{\mathcolor{gray}{-}}}{23}{\bar{\bar{\bar{\chi}}}}^{\;\! \mathcolor{gray}{\omega} \textcolor{PineGreen}{\imath \jmath l}}_{\mathcolor{gray}{z} \textcolor{Maroon}{(2)}} ~{}^{\mathcolor{gray}{*}}_{\mathcolor{gray}{*}} \left( \Xint{\mathcolor{gray}{-}}{295}{\bar{E}}^{\;\!\mathcolor{gray}{\omega}}_{\;\! \mathcolor{gray}{z} \textcolor{PineGreen}{\jmath} } ~\mathcolor{gray}{\widetilde \circledast}~ \Xint{\mathcolor{gray}{-}}{295}{\bar{E}}^{\;\!\mathcolor{gray}{\omega}}_{\;\! \mathcolor{gray}{z} \textcolor{PineGreen}{l} } \right) ~, \label{eq:vec-DP^(2)-plane_wave_basis-p_pp} \\
	\Xint{\mathcolor{gray}{-}}{30}{D}^{\;\! \mathcolor{gray}{\omega} \textcolor{PineGreen}{\imath}}_{\;\! \symup{\iota}\mathcolor{gray}{z} \textcolor{Maroon}{(2)}} &= \Xint{\mathcolor{gray}{-}}{30}{P}^{\;\! \mathcolor{gray}{\omega} \textcolor{PineGreen}{\imath}}_{\;\! \symup{\iota}\mathcolor{gray}{z} \textcolor{Maroon}{(2)}} = \Xint{{}^{}_{\mathcolor{gray}{-}}}{23}{\chi}^{\;\! \mathcolor{gray}{\omega} \hat{1} \hat{2} \textcolor{PineGreen}{\imath}}_{\;\! \symup{\iota} \mathcolor{gray}{z} \textcolor{Maroon}{(2)} \textcolor{PineGreen}{\jmath l}} \mathcolor{gray}{*} \left( \Xint{\mathcolor{gray}{-}}{295}{E}^{\;\!\mathcolor{gray}{\omega} \textcolor{PineGreen}{\jmath}}_{\;\! \hat{1} \mathcolor{gray}{z}} ~\mathcolor{gray}{\widetilde \circledast}~ \Xint{\mathcolor{gray}{-}}{295}{E}^{\;\!\mathcolor{gray}{\omega} \textcolor{PineGreen}{l}}_{\;\! \hat{2} \mathcolor{gray}{z}} \right) ~. \label{eq:components-DP^(2)-plane_wave_basis-p_pp}
\end{align}
\end{subequations}
注意到,\bref{eq:nonlinear(2)-wave_wkrho-simplify6-V2'-SVA} 正是 \textcolor{NavyBlue}{缓变振幅近似} 和 \textcolor{PineGreen}{线性光学本征模约束} \bref{eq:nonlinear(2)-wave_wkrho-simplify6-L} 条件下的 \bref{eq:nonlinear(2)-wave_wkrho-simplify6-SVA} 的\textcolor{PineGreen}{平面波基}版本。进一步地,若 $\Xint{\mathcolor{gray}{-}}{25}{\bar{\bar{\mathsfit{V}}}}^{\;\! \mathcolor{gray}{\omega} \textcolor{PineGreen}{\imath}}_{\textcolor{Maroon}{\mathbb{1}}}$ 
即 \bref{eq:matrix_exp-V1} 的对应物
\begin{align} \label{eq:plane_wave_basis-V1}
	\Xint{\mathcolor{gray}{-}}{25}{\bar{\bar{\mathsfit{V}}}}^{\;\! \mathcolor{gray}{\omega} \textcolor{PineGreen}{\imath}}_{\textcolor{Maroon}{\mathbb{1}}} := \bar{\bar{\zeta}}^{\;\! \mathcolor{gray}{\omega} \mathcolor{gray}{\symup{z}}}_{\textcolor{Maroon}{(1)}} + \bar{\bar{\zeta}}^{\;\! \mathcolor{gray}{\omega} \mathcolor{gray}{\check{1} \symup{z}}}_{\textcolor{Maroon}{(1)}} \mathbb{i} \Xint{\begin{smallmatrix} ~ \\ {}^{}_{\mathcolor{gray}{-}} \\ ~ \end{smallmatrix}}{15}{k}_{\;\! \mathcolor{gray}{\check{1}}}^{\;\! \mathcolor{gray}{\omega} \textcolor{PineGreen}{\imath}} + \bar{\bar{\zeta}}^{\;\! \mathcolor{gray}{\omega} \mathcolor{gray}{\symup{z} \check{2}}}_{\textcolor{Maroon}{(1)}} \mathbb{i} \Xint{\begin{smallmatrix} ~ \\ {}^{}_{\mathcolor{gray}{-}} \\ ~ \end{smallmatrix}}{15}{k}_{\;\! \mathcolor{gray}{\check{2}}}^{\;\! \mathcolor{gray}{\omega} \textcolor{PineGreen}{\imath}}
\end{align}
\textcolor{Plum}{可逆},那么 \bref{eq:nonlinear(2)-wave_wkrho-simplify6-V2'-SVA} 及其解,将变为 \bref{eq:simplify6-LE0-SVA-E'-V_1nonsingular-gg} 的对应物
\begin{subequations} \label{eq:simplify6-LE0-SVA-V_1nonsingular}
\begin{align}
	\mathcolor{gray}{\nabla_z} \Xint{{}^{}_{\mathcolor{gray}{-}}}{10}{\bar{g}}^{\;\!\mathcolor{gray}{\omega} \textcolor{PineGreen}{\imath}}_{\;\! \mathcolor{gray}{z}}
	&= - \Xint{\mathcolor{gray}{-}}{25}{\bar{\bar{\mathsfit{V}}}}^{\;\! - \mathcolor{gray}{\omega} \textcolor{PineGreen}{\imath}}_{\textcolor{Maroon}{\mathbb{1}}} \Xint{\mathcolor{gray}{-}}{30}{\bar{P}}^{\;\! \mathcolor{gray}{\omega} \textcolor{PineGreen}{\imath}}_{\;\! \mathcolor{gray}{z} \textcolor{Maroon}{(2)}} \big/ \mathbb{e}^{\mathbb{i} \Xint{\begin{smallmatrix} ~ \\ {}^{}_{\mathcolor{gray}{-}} \\ ~ \end{smallmatrix}}{15}{k}_{\symup{z}}^{\;\! \mathcolor{gray}{\omega} \textcolor{PineGreen}{\imath}} \mathcolor{gray}{z}} \label{eq:simplify6-LE0-SVA-V_1nonsingular-g} \\
	\Xint{{}^{}_{\mathcolor{gray}{-}}}{10}{\bar{g}}^{\;\!\mathcolor{gray}{\omega} \textcolor{PineGreen}{\imath}}_{\;\! \mathcolor{gray}{z}}
	&= \Xint{{}^{}_{\mathcolor{gray}{-}}}{10}{\bar{g}}^{\;\!\mathcolor{gray}{\omega} \textcolor{PineGreen}{\imath}}_{\;\! \mathcolor{gray}{0}} - \int_{\mathcolor{gray}{0}}^{\mathcolor{gray}{z}} \Xint{\mathcolor{gray}{-}}{25}{\bar{\bar{\mathsfit{V}}}}^{\;\! - \mathcolor{gray}{\omega} \textcolor{PineGreen}{\imath}}_{\textcolor{Maroon}{\mathbb{1}}} \Xint{\mathcolor{gray}{-}}{30}{\bar{P}}^{\;\! \mathcolor{gray}{\omega} \textcolor{PineGreen}{\imath}}_{\;\! \mathcolor{gray}{\mathtt{z}} \textcolor{Maroon}{(2)}} \big/ \mathbb{e}^{\mathbb{i} \Xint{\begin{smallmatrix} ~ \\ {}^{}_{\mathcolor{gray}{-}} \\ ~ \end{smallmatrix}}{15}{k}_{\symup{z}}^{\;\! \mathcolor{gray}{\omega} \textcolor{PineGreen}{\imath}} \mathcolor{gray}{\mathtt{z}}} ~\mathbb{d} \mathcolor{gray}{\mathtt{z}} ~. \label{eq:simplify6-LE0-SVA-V_1nonsingular-solution-g}
\end{align}
\end{subequations}
同样,电场\textcolor{Maroon}{时空谱} $\Xint{{}^{}_{\mathcolor{gray}{-}}}{10}{\bar{g}}^{\;\!\mathcolor{gray}{\omega} \textcolor{PineGreen}{\jmath}}_{\;\! \mathcolor{gray}{z}}$ 初始值 $\Xint{{}^{}_{\mathcolor{gray}{-}}}{10}{\bar{g}}^{\;\!\mathcolor{gray}{\omega} \textcolor{PineGreen}{\imath}}_{\;\! \mathcolor{gray}{0}}$ 由入射面 $\mathcolor{gray}{z \mathcolor{black}{=} 0}$ 处的\textcolor{Maroon}{边界条件} \bref{eq:1BC} 确定。

等价(于 \bref{eq:simplify6-LE0-SVA-V_1nonsingular-g})地,当系数行列式 $\textcolor{Plum}{\det} \left[ \Xint{\mathcolor{gray}{-}}{25}{\bar{\bar{\mathsfit{V}}}}^{\;\! \mathcolor{gray}{\omega} \textcolor{PineGreen}{\imath}}_{\textcolor{Maroon}{\mathbb{1}}} \right]$ 不为零时\Footnote{后文会提到,对于\textcolor{gray}{自变量} $\mathcolor{gray}{\bar{k}_{\symup{\rho}}} = \left( \mathcolor{gray}{k_{\symup{x}}},~ \mathcolor{gray}{k_{\symup{y}}} \right)^{\mathsf{\textcolor{Plum}{T}}}$ 的大部分取值组合(特别是在斜出射晶体端面 $\mathcolor{gray}{\bar{k}_{\symup{\rho}}} \neq \mathcolor{gray}{\bar{0}}$ 的情况下),$\textcolor{Plum}{\det} \left[ \Xint{\mathcolor{gray}{-}}{25}{\bar{\bar{\mathsfit{V}}}}^{\;\! \mathcolor{gray}{\omega} \textcolor{PineGreen}{\imath}}_{\textcolor{Maroon}{\mathbb{1}}} \right] \neq 0$,\bref{eq:simplify6-LE0-SVA-V_1nonsingular-g,simplify6-V2'-SVA-Cramer} 成立。},对三元一次方程组 \bref{eq:nonlinear(2)-wave_wkrho-simplify6-V2'-SVA} 应用 \textcolor{Maroon}{Cramer's rule},此时方程组 \bref{eq:nonlinear(2)-wave_wkrho-simplify6-V2'-SVA} 具有唯一解
\begin{align} \label{simplify6-V2'-SVA-Cramer}
	\mathcolor{gray}{\nabla_z} \begin{pmatrix} \Xint{{}^{}_{\mathcolor{gray}{-}}}{10}{g}^{\;\!\mathcolor{gray}{\omega} \textcolor{PineGreen}{\imath}}_{\;\! \symup{x} \mathcolor{gray}{z}} \\ \Xint{{}^{}_{\mathcolor{gray}{-}}}{10}{g}^{\;\!\mathcolor{gray}{\omega} \textcolor{PineGreen}{\imath}}_{\;\! \symup{y} \mathcolor{gray}{z}} \\ \Xint{{}^{}_{\mathcolor{gray}{-}}}{10}{g}^{\;\!\mathcolor{gray}{\omega} \textcolor{PineGreen}{\imath}}_{\;\! \symup{z} \mathcolor{gray}{z}} \end{pmatrix} = - \frac{\begin{pmatrix} \textcolor{Plum}{\det} \left[ \Xint{\mathcolor{gray}{-}}{24}{\bar{P}}^{\;\! \mathcolor{gray}{\omega} \textcolor{PineGreen}{\imath}}_{\;\! \mathcolor{gray}{z} \textcolor{Maroon}{(2)}},~ \Xint{\mathcolor{gray}{-}}{25}{\bar{\mathsfit{v}}}^{\;\! \mathcolor{gray}{\omega} \textcolor{PineGreen}{\imath}}_{\textcolor{Maroon}{\mathbb{1}}\textcolor{Plum}{\symup{y}}},~ \Xint{\mathcolor{gray}{-}}{25}{\bar{\mathsfit{v}}}^{\;\! \mathcolor{gray}{\omega} \textcolor{PineGreen}{\imath}}_{\textcolor{Maroon}{\mathbb{1}}\textcolor{Plum}{\symup{z}}} \right] \\ \textcolor{Plum}{\det} \left[ \Xint{\mathcolor{gray}{-}}{25}{\bar{\mathsfit{v}}}^{\;\! \mathcolor{gray}{\omega} \textcolor{PineGreen}{\imath}}_{\textcolor{Maroon}{\mathbb{1}}\textcolor{Plum}{\symup{x}}},~ \Xint{\mathcolor{gray}{-}}{24}{\bar{P}}^{\;\! \mathcolor{gray}{\omega} \textcolor{PineGreen}{\imath}}_{\;\! \mathcolor{gray}{z} \textcolor{Maroon}{(2)}},~ \Xint{\mathcolor{gray}{-}}{25}{\bar{\mathsfit{v}}}^{\;\! \mathcolor{gray}{\omega} \textcolor{PineGreen}{\imath}}_{\textcolor{Maroon}{\mathbb{1}}\textcolor{Plum}{\symup{z}}} \right] \\ \textcolor{Plum}{\det} \left[ \Xint{\mathcolor{gray}{-}}{25}{\bar{\mathsfit{v}}}^{\;\! \mathcolor{gray}{\omega} \textcolor{PineGreen}{\imath}}_{\textcolor{Maroon}{\mathbb{1}}\textcolor{Plum}{\symup{x}}},~ \Xint{\mathcolor{gray}{-}}{25}{\bar{\mathsfit{v}}}^{\;\! \mathcolor{gray}{\omega} \textcolor{PineGreen}{\imath}}_{\textcolor{Maroon}{\mathbb{1}}\textcolor{Plum}{\symup{y}}},~ \Xint{\mathcolor{gray}{-}}{24}{\bar{P}}^{\;\! \mathcolor{gray}{\omega} \textcolor{PineGreen}{\imath}}_{\;\! \mathcolor{gray}{z} \textcolor{Maroon}{(2)}} \right] \end{pmatrix}}{\textcolor{Plum}{\det} \left[ \Xint{\mathcolor{gray}{-}}{25}{\bar{\bar{\mathsfit{V}}}}^{\;\! \mathcolor{gray}{\omega} \textcolor{PineGreen}{\imath}}_{\textcolor{Maroon}{\mathbb{1}}} \right] \mathbb{e}^{\mathbb{i} \Xint{\begin{smallmatrix} ~ \\ {}^{}_{\mathcolor{gray}{-}} \\ ~ \end{smallmatrix}}{15}{k}_{\symup{z}}^{\;\! \mathcolor{gray}{\omega} \textcolor{PineGreen}{\imath}} \mathcolor{gray}{z}}} ~,
\end{align}
其中,类似 \bref{eq:kz-non_diagonalization,eq:e^ikzz-non_diagonalization} 地,定义了 $\Xint{\mathcolor{gray}{-}}{25}{\bar{\bar{\mathsfit{V}}}}^{\;\! \mathcolor{gray}{\omega} \textcolor{PineGreen}{\imath}}_{\textcolor{Maroon}{\mathbb{1}}}$ 的列向量分解
\begin{align} \label{eq:plane_wave_basis-V1_decompose}
	\Xint{\mathcolor{gray}{-}}{25}{\bar{\bar{\mathsfit{V}}}}^{\;\! \mathcolor{gray}{\omega} \textcolor{PineGreen}{\imath}}_{\textcolor{Maroon}{\mathbb{1}}} := \overline{\Xint{\mathcolor{gray}{-}}{25}{\bar{\mathsfit{v}}}^{\;\! \mathcolor{gray}{\omega} \textcolor{PineGreen}{\imath}}_{\textcolor{Maroon}{\mathbb{1}} \textcolor{Plum}{\jmath}}}^{\mathsf{\textcolor{Plum}{T}}} = \left( \Xint{\mathcolor{gray}{-}}{25}{\bar{\mathsfit{v}}}^{\;\! \mathcolor{gray}{\omega} \textcolor{PineGreen}{\imath}}_{\textcolor{Maroon}{\mathbb{1}} \textcolor{Plum}{\symup{x}}},~ \Xint{\mathcolor{gray}{-}}{25}{\bar{\mathsfit{v}}}^{\;\! \mathcolor{gray}{\omega} \textcolor{PineGreen}{\imath}}_{\textcolor{Maroon}{\mathbb{1}} \textcolor{Plum}{\symup{y}}},~ \Xint{\mathcolor{gray}{-}}{25}{\bar{\mathsfit{v}}}^{\;\! \mathcolor{gray}{\omega} \textcolor{PineGreen}{\imath}}_{\textcolor{Maroon}{\mathbb{1}} \textcolor{Plum}{\symup{z}}} \right) = \begin{pmatrix} \Xint{\mathcolor{gray}{-}}{25}{\mathsfit{V}}^{\;\! \mathcolor{gray}{\omega} \textcolor{PineGreen}{\imath}}_{\textcolor{Maroon}{\mathbb{1}} \symup{x} \textcolor{Plum}{\symup{x}}} & \Xint{\mathcolor{gray}{-}}{25}{\mathsfit{V}}^{\;\! \mathcolor{gray}{\omega} \textcolor{PineGreen}{\imath}}_{\textcolor{Maroon}{\mathbb{1}} \symup{x} \textcolor{Plum}{\symup{y}}} & \Xint{\mathcolor{gray}{-}}{25}{\mathsfit{V}}^{\;\! \mathcolor{gray}{\omega} \textcolor{PineGreen}{\imath}}_{\textcolor{Maroon}{\mathbb{1}} \symup{x} \textcolor{Plum}{\symup{z}}} \\ \Xint{\mathcolor{gray}{-}}{25}{\mathsfit{V}}^{\;\! \mathcolor{gray}{\omega} \textcolor{PineGreen}{\imath}}_{\textcolor{Maroon}{\mathbb{1}} \symup{y} \textcolor{Plum}{\symup{x}}} & \Xint{\mathcolor{gray}{-}}{25}{\mathsfit{V}}^{\;\! \mathcolor{gray}{\omega} \textcolor{PineGreen}{\imath}}_{\textcolor{Maroon}{\mathbb{1}} \symup{y} \textcolor{Plum}{\symup{y}}} & \Xint{\mathcolor{gray}{-}}{25}{\mathsfit{V}}^{\;\! \mathcolor{gray}{\omega} \textcolor{PineGreen}{\imath}}_{\textcolor{Maroon}{\mathbb{1}} \symup{y} \textcolor{Plum}{\symup{z}}} \\ \Xint{\mathcolor{gray}{-}}{25}{\mathsfit{V}}^{\;\! \mathcolor{gray}{\omega} \textcolor{PineGreen}{\imath}}_{\textcolor{Maroon}{\mathbb{1}} \symup{z} \textcolor{Plum}{\symup{x}}} & \Xint{\mathcolor{gray}{-}}{25}{\mathsfit{V}}^{\;\! \mathcolor{gray}{\omega} \textcolor{PineGreen}{\imath}}_{\textcolor{Maroon}{\mathbb{1}} \symup{z} \textcolor{Plum}{\symup{y}}} & \Xint{\mathcolor{gray}{-}}{25}{\mathsfit{V}}^{\;\! \mathcolor{gray}{\omega} \textcolor{PineGreen}{\imath}}_{\textcolor{Maroon}{\mathbb{1}} \symup{z} \textcolor{Plum}{\symup{z}}} \end{pmatrix} ~,
\end{align}
这里使用了\textcolor{Plum}{李子紫}表示“另一种列向量”,见 \bref{hook:Plum}。尽管 \bref{simplify6-V2'-SVA-Cramer} 给出了\textcolor{gray}{傅立叶域}中矢量电场三分量\textcolor{NavyBlue}{非相位部分} $\Xint{{}^{}_{\mathcolor{gray}{-}}}{10}{\bar{g}}^{\;\!\mathcolor{gray}{\omega} \textcolor{PineGreen}{\jmath}}_{\;\! \mathcolor{gray}{z}}$ 的解析解,然而可以看出,该组\textcolor{gray}{混频}解(的三分量中的每一个分量,都)因含有 $\Xint{\mathcolor{gray}{-}}{24}{\bar{P}}^{\;\! \mathcolor{gray}{\omega} \textcolor{PineGreen}{\imath}}_{\;\! \mathcolor{gray}{z} \textcolor{Maroon}{(2)}}$ 的三分量的混合叠加,而计算较为麻烦\Footnote{下文会提到,作为高维矢量\textcolor{Maroon}{\textcolor{Plum}{非线性}卷积}积分,$\Xint{\mathcolor{gray}{-}}{24}{\bar{P}}^{\;\! \mathcolor{gray}{\omega} \textcolor{PineGreen}{\imath}}_{\;\! \mathcolor{gray}{z} \textcolor{Maroon}{(2)}}$ 的计算本身就已经很复杂。};且可以证明其对特定方向(即满足 $\textcolor{Plum}{\det} \left[ \Xint{\mathcolor{gray}{-}}{25}{\bar{\bar{\mathsfit{V}}}}^{\;\! \mathcolor{gray}{\omega} \textcolor{PineGreen}{\imath}}_{\textcolor{Maroon}{\mathbb{1}}} \right] = 0$,包括正出射晶体端面 $\mathcolor{gray}{\bar{k}_{\symup{\rho}}} = \mathcolor{gray}{\bar{0}}$ 时)的\textcolor{gray}{空间频率}组分的计算,会遇到\textcolor{Plum}{数值奇点}(分母除以 $0$)。

正如 \bref{ssec:Exp-waveq} 末所言,对于最简单的情况,即\textcolor{PineGreen}{纯电各向异性}材料内部,上述 \bref{eq:simplify6-LE0-SVA-V_1nonsingular-g,simplify6-V2'-SVA-Cramer} 竟不成立;对于复杂的情况,却反而有可能成立。这种“大概率成立、有小概率不成立”的事件,不是我们想看到的。此外,还想看到的是两种确定的情况,即要么 $\textcolor{Plum}{\det} \left[ \Xint{\mathcolor{gray}{-}}{25}{\bar{\bar{\mathsfit{V}}}}^{\;\! \mathcolor{gray}{\omega} \textcolor{PineGreen}{\imath}}_{\textcolor{Maroon}{\mathbb{1}}} \right] \not\equiv 0$,\bref{eq:simplify6-LE0-SVA-V_1nonsingular-g,simplify6-V2'-SVA-Cramer} 成立;要么 $\textcolor{Plum}{\det} \left[ \Xint{\mathcolor{gray}{-}}{25}{\bar{\bar{\mathsfit{V}}}}^{\;\! \mathcolor{gray}{\omega} \textcolor{PineGreen}{\imath}}_{\textcolor{Maroon}{\mathbb{1}}} \right] \equiv 0$,\bref{eq:simplify6-LE0-SVA-V_1nonsingular-g,simplify6-V2'-SVA-Cramer} 不成立;而不是模棱两可的中间/叠加态。

那有没有 $\Xint{\mathcolor{gray}{-}}{25}{\bar{\bar{\mathsfit{V}}}}^{\;\! \mathcolor{gray}{\omega} \textcolor{PineGreen}{\imath}}_{\textcolor{Maroon}{\mathbb{1}}}$ 确定\textcolor{Plum}{可逆},或确定\textcolor{Plum}{不可逆}的情况?有。但在当前的“\textbf{\textcolor{Plum}{非线性}\textcolor{NavyBlue}{光学}\textcolor{PineGreen}{本征系统},视为\textcolor{Plum}{线性}\textcolor{NavyBlue}{光学}\textcolor{PineGreen}{本征系统}的 \textcolor{NavyBlue}{0 阶微扰}}”,即 \bref{eq:nonlinear(2)-wave_wkrho-simplify6-L} 的条件下,只有一种比较通用的方法可以导向该种“确定的情况”:通过牺牲精度 --- 即,假设所考虑的\textcolor{gray}{时间频率} $\mathcolor{gray}{\omega}$ 组分,\textcolor{gray}{混频}场的(最大)\textcolor{Plum}{横向}\textcolor{gray}{空间频率}不太高,即
\begin{align} \label{eq:k_rho<<k_z}
	\mathcolor{gray}{k_{\symup{\rho}}} := \lvert \mathcolor{gray}{\bar{k}_{\symup{\rho}}} \rvert := \sqrt{ \mathcolor{gray}{k^2_{\symup{x}}} + \mathcolor{gray}{k^2_{\symup{y}}} } \ll \Xint{\begin{smallmatrix} ~ \\ {}^{}_{\mathcolor{gray}{-}} \\ ~ \end{smallmatrix}}{15}{k}_{\symup{z} \textcolor{Plum}{\text{R}}}^{\;\! \mathcolor{gray}{\omega} \textcolor{PineGreen}{\imath}} := \textcolor{Plum}{\text{Re}} \left[ \Xint{\begin{smallmatrix} ~ \\ {}^{}_{\mathcolor{gray}{-}} \\ ~ \end{smallmatrix}}{15}{k}_{\symup{z}}^{\;\! \mathcolor{gray}{\omega} \textcolor{PineGreen}{\imath}} \right] ~,
\end{align}
并且所\textcolor{gray}{混频}出的 $\Xint{{}^{}_{\mathcolor{gray}{-}}}{10}{\bar{g}}^{\;\!\mathcolor{gray}{\omega} \textcolor{PineGreen}{\imath}}_{\;\! \mathcolor{gray}{z}}$ 的 $\symup{z}$ 分量 $\Xint{{}^{}_{\mathcolor{gray}{-}}}{10}{g}^{\;\!\mathcolor{gray}{\omega} \textcolor{PineGreen}{\imath}}_{\;\! \symup{z} \mathcolor{gray}{z}}$,相比其\textcolor{Plum}{横向} $\symup{\bar{\rho}} := \left( \symup{x},~ \symup{y} \right)^{\mathsf{\textcolor{Plum}{T}}}$ 分量 $\Xint{{}^{}_{\mathcolor{gray}{-}}}{10}{\bar{g}}^{\;\!\mathcolor{gray}{\omega} \textcolor{PineGreen}{\imath}}_{\;\! \symup{\rho} \mathcolor{gray}{z}}$ 而言,\textcolor{Maroon}{幅值}不太高\Footnote{注意,这第二个条件非常宽松(相比第一个条件),由于电磁波的\textcolor{Plum}{横向}特性。},即 $\left| \Xint{{}^{}_{\mathcolor{gray}{-}}}{10}{g}^{\;\!\mathcolor{gray}{\omega} \textcolor{PineGreen}{\imath}}_{\;\! \symup{z} \mathcolor{gray}{z}} \right| := \sqrt{ \lvert \Xint{{}^{}_{\mathcolor{gray}{-}}}{10}{g}^{\;\!\mathcolor{gray}{\omega} \textcolor{PineGreen}{\imath}}_{\;\! \symup{z} \mathcolor{gray}{z}} \rvert^2 } := \sqrt{ \Xint{{}^{}_{\mathcolor{gray}{-}}}{10}{g}^{\;\!\mathcolor{gray}{\omega} \textcolor{PineGreen}{\imath}}_{\;\! \symup{z} \mathcolor{gray}{z}} \Xint{{}^{}_{\mathcolor{gray}{-}}}{10}{g}^{\;\!\mathcolor{gray}{\omega} \textcolor{PineGreen}{\imath} *}_{\;\! \symup{z} \mathcolor{gray}{z}} } \lesssim \Xint{{}^{}_{\mathcolor{gray}{-}}}{10}{g}^{\;\!\mathcolor{gray}{\omega} \textcolor{PineGreen}{\imath}}_{\;\! \symup{\rho} \mathcolor{gray}{z}} := \lvert \Xint{{}^{}_{\mathcolor{gray}{-}}}{10}{\bar{g}}^{\;\!\mathcolor{gray}{\omega} \textcolor{PineGreen}{\imath}}_{\;\! \symup{\rho} \mathcolor{gray}{z}} \rvert = \sqrt{ \lvert \Xint{{}^{}_{\mathcolor{gray}{-}}}{10}{g}^{\;\! \mathcolor{gray}{\omega} \textcolor{PineGreen}{\imath}}_{\;\! \symup{x} \mathcolor{gray}{z}} \rvert^2 + \lvert \Xint{{}^{}_{\mathcolor{gray}{-}}}{10}{g}^{\;\! \mathcolor{gray}{\omega} \textcolor{PineGreen}{\imath}}_{\;\! \symup{x} \mathcolor{gray}{z}} \rvert^2 }$。在这两个条件的共同作用下(以第一个条件 \bref{eq:k_rho<<k_z} 为主),\bref{eq:plane_wave_basis-V1} 即 $\Xint{\mathcolor{gray}{-}}{25}{\bar{\bar{\mathsfit{V}}}}^{\;\! \mathcolor{gray}{\omega} \textcolor{PineGreen}{\imath}}_{\textcolor{Maroon}{\mathbb{1}}}$ 中的 $\mathcolor{gray}{k_{\symup{x}}}$ 项、$\mathcolor{gray}{k_{\symup{y}}}$ 项,相对于 $\Xint{\begin{smallmatrix} ~ \\ {}^{}_{\mathcolor{gray}{-}} \\ ~ \end{smallmatrix}}{15}{k}_{\symup{z}}^{\;\! \mathcolor{gray}{\omega} \textcolor{PineGreen}{\imath}}$ 项,均可以省略,并变为
\begin{align} \label{eq:plane_wave_basis-V1-nokxky}
	\Xint{\mathcolor{gray}{-}}{25}{\bar{\bar{\mathsfit{V}}}}^{\;\! \mathcolor{gray}{\omega} \textcolor{PineGreen}{\imath}}_{\textcolor{Maroon}{\mathbb{1}}} := \bar{\bar{\zeta}}^{\;\! \mathcolor{gray}{\omega} \mathcolor{gray}{\symup{z}}}_{\textcolor{Maroon}{(1)}} + \bar{\bar{\zeta}}^{\;\! \mathcolor{gray}{\omega} \mathcolor{gray}{\symup{z} \symup{z}}}_{\textcolor{Maroon}{(1)}} 2\mathbb{i} \Xint{\begin{smallmatrix} ~ \\ {}^{}_{\mathcolor{gray}{-}} \\ ~ \end{smallmatrix}}{15}{k}_{\;\! \mathcolor{gray}{\symup{z}}}^{\;\! \mathcolor{gray}{\omega} \textcolor{PineGreen}{\imath}} ~.
\end{align}
尽管如此,\bref{eq:plane_wave_basis-V1-nokxky} 中的 $\Xint{\mathcolor{gray}{-}}{25}{\bar{\bar{\mathsfit{V}}}}^{\;\! \mathcolor{gray}{\omega} \textcolor{PineGreen}{\imath}}_{\textcolor{Maroon}{\mathbb{1}}}$ 是否\textcolor{Plum}{可逆},仍然是未知的。情况仍是不确定的。

对此,进一步简化场景:至少在\textcolor{Plum}{非线性}算符 $\Xint{\mathcolor{gray}{-}}{25}{\bar{\bar{\mathsfit{V}}}}^{\;\! \mathcolor{gray}{\omega} \textcolor{PineGreen}{\imath}}_{\textcolor{Maroon}{\mathbb{1}}}$ 层面\Footnote{\textcolor{Plum}{线性}算符 $\Xint{\mathcolor{gray}{-}}{30}{\bar{\bar{L}}}^{\;\! \mathcolor{gray}{\omega} \textcolor{PineGreen}{\imath}}$ 即 \bref{eq:nonlinear(2)-wave_wkrho-simplify6-L3''} 层面可以不受影响,但这又与\textcolor{Plum}{非线性}算符 $\Xint{\mathcolor{gray}{-}}{25}{\bar{\bar{\mathsfit{V}}}}^{\;\! \mathcolor{gray}{\omega} \textcolor{PineGreen}{\imath}}_{\textcolor{Maroon}{\mathbb{1}}}$ 的近似程度不一致。是否差异化对待 $\Xint{\mathcolor{gray}{-}}{30}{\bar{\bar{L}}}^{\;\! \mathcolor{gray}{\omega} \textcolor{PineGreen}{\imath}}, \Xint{\mathcolor{gray}{-}}{25}{\bar{\bar{\mathsfit{V}}}}^{\;\! \mathcolor{gray}{\omega} \textcolor{PineGreen}{\imath}}_{\textcolor{Maroon}{\mathbb{1}}}$ 各自的近似条件,见仁见智。},不考虑\textcolor{NavyBlue}{电偶-电四/磁偶}极、\textcolor{NavyBlue}{磁偶-电偶}极,所合并成的(\bref{eq:p<->m} 中的)交叉耦合系数 $\bar{\bar{\zeta}}^{\;\! \mathcolor{gray}{\omega} \mathcolor{gray}{\check{1}}}_{\textcolor{Maroon}{(1)}}$ 的“\textcolor{Plum}{$\symup{z}$ 分量}” $\bar{\bar{\zeta}}^{\;\! \mathcolor{gray}{\omega} \mathcolor{gray}{\symup{z}}}_{\textcolor{Maroon}{(1)}}$,以至于 $\bar{\bar{\zeta}}^{\;\! \mathcolor{gray}{\omega} \mathcolor{gray}{\symup{z}}}_{\textcolor{Maroon}{(1)}} \equiv \bar{\bar{0}}$;同时,对于\textcolor{NavyBlue}{磁偶-电四/磁偶}极、\textcolor{NavyBlue}{电偶-电八/磁四}极,所合并出的(\bref{eq:p<->n} 中的)交叉耦合系数 $\bar{\bar{\zeta}}^{\;\! \mathcolor{gray}{\omega} \mathcolor{gray}{\check{1} \check{2}}}_{\textcolor{Maroon}{(1)}}$,只考虑其“\textcolor{Plum}{$\symup{z}\symup{z}$ 分量}” $\bar{\bar{\zeta}}^{\;\! \mathcolor{gray}{\omega} \mathcolor{gray}{\symup{z} \symup{z}}}_{\textcolor{Maroon}{(1)}}$ 的各向同性部分 $\frac{1}{\mathbb{i} k_{\textcolor{Maroon}{\mathsf{o}}}^{\;\! \mathcolor{gray}{\omega}}} \Xint{\begin{smallmatrix} ~ \\ {}^{}_{\mathcolor{gray}{-}} \\ ~ \end{smallmatrix}}{13}{\varsigma}^{\;\! \mathcolor{gray}{\omega} \hat{1} \mathcolor{gray}{\symup{z}}}_{\;\! \dot{2} \mathcolor{gray}{z} \textcolor{Maroon}{(1)}} \epsilon^{\hphantom{\symup{\iota}}\mathcolor{gray}{\symup{z}}\dot{2}}_{\symup{\iota}} = \frac{1}{\mathbb{i} k_{\textcolor{Maroon}{\mathsf{o}}}^{\;\! \mathcolor{gray}{\omega}}} \left( \frac{\symup{c}}{\mathbb{i} \mathcolor{gray}{\omega}} \epsilon^{\;\! \hphantom{\dot{2}} \mathcolor{gray}{\symup{z}} \hat{1}}_{\;\! \dot{2}} \right) \epsilon^{\hphantom{\symup{\iota}}\mathcolor{gray}{\symup{z}}\dot{2}}_{\symup{\iota}} = \frac{1}{\mathbb{i} k_{\textcolor{Maroon}{\mathsf{o}}}^{\;\! \mathcolor{gray}{\omega}}}  \frac{\symup{c}}{\mathbb{i} \mathcolor{gray}{\omega}} \left( \delta^{\;\! \hphantom{\symup{\iota}} \mathcolor{gray}{\symup{z}}}_{\;\! \symup{\iota} \hphantom{z}} \delta^{\;\! \mathcolor{gray}{\symup{z}} \hat{1}} - \delta^{\;\! \hphantom{\symup{\iota}} \hat{1}}_{\;\! \symup{\iota} \hphantom{z}} \right)$,即有 $\bar{\bar{\zeta}}^{\;\! \mathcolor{gray}{\omega} \mathcolor{gray}{\symup{z} \symup{z}}}_{\textcolor{Maroon}{(1)}} \equiv \frac{1}{\mathbb{i} k_{\textcolor{Maroon}{\mathsf{o}}}^{\;\! \mathcolor{gray}{\omega}}}  \frac{\symup{c}}{\mathbb{i} \mathcolor{gray}{\omega}} \left( \hat{\symup{e}}_{\mathcolor{gray}{\symup{z}}} \otimes \hat{\symup{e}}_{\mathcolor{gray}{\symup{z}}} - \bar{\bar{\symup{I}}} \right) =: \frac{1}{k_{\textcolor{Maroon}{\mathsf{o}} \mathcolor{gray}{\omega}}^{\;\! 2}} \left( \bar{\bar{\symup{I}}} - \hat{\mathcolor{gray}{\symup{z}}} \hat{\mathcolor{gray}{\symup{z}}} \right)$。

应用上一段所描述的 2 个简化条件
\begin{align} \label{eq:zeta-restriction-V1}
	\bar{\bar{\zeta}}^{\;\! \mathcolor{gray}{\omega} \mathcolor{gray}{\symup{z}}}_{\textcolor{Maroon}{(1)}} \equiv \bar{\bar{0}} ~, ~~~~~~ \bar{\bar{\zeta}}^{\;\! \mathcolor{gray}{\omega} \mathcolor{gray}{\symup{z} \symup{z}}}_{\textcolor{Maroon}{(1)}} \equiv \frac{1}{k_{\textcolor{Maroon}{\mathsf{o}} \mathcolor{gray}{\omega}}^{\;\! 2}} \left( 1 - \hat{\mathcolor{gray}{\symup{z}}} \hat{\mathcolor{gray}{\symup{z}}}^{\mathsf{\textcolor{Plum}{T}}} \right) ~,
\end{align}
此时 \bref{eq:plane_wave_basis-V1-nokxky} 即 $\Xint{\mathcolor{gray}{-}}{25}{\bar{\bar{\mathsfit{V}}}}^{\;\! \mathcolor{gray}{\omega} \textcolor{PineGreen}{\imath}}_{\textcolor{Maroon}{\mathbb{1}}}$ 进一步简化为
\begin{align} \label{eq:plane_wave_basis-V1-nokxky-zeta}
	\Xint{\mathcolor{gray}{-}}{25}{\bar{\bar{\mathsfit{V}}}}^{\;\! \mathcolor{gray}{\omega} \textcolor{PineGreen}{\imath}}_{\textcolor{Maroon}{\mathbb{1}}} := 2\mathbb{i} \frac{\Xint{\begin{smallmatrix} ~ \\ {}^{}_{\mathcolor{gray}{-}} \\ ~ \end{smallmatrix}}{15}{k}_{\;\! \symup{z}}^{\;\! \mathcolor{gray}{\omega} \textcolor{PineGreen}{\imath}}}{k_{\textcolor{Maroon}{\mathsf{o}} \mathcolor{gray}{\omega}}^{\;\! 2}} \left( 1 - \hat{\mathcolor{gray}{\symup{z}}} \hat{\mathcolor{gray}{\symup{z}}}^{\mathsf{\textcolor{Plum}{T}}} \right) ~.
\end{align}
注意到,退化至 \bref{eq:plane_wave_basis-V1-nokxky-zeta} 的 $\Xint{\mathcolor{gray}{-}}{25}{\bar{\bar{\mathsfit{V}}}}^{\;\! \mathcolor{gray}{\omega} \textcolor{PineGreen}{\imath}}_{\textcolor{Maroon}{\mathbb{1}}}$ 是 3 维 2 秩、确定\textcolor{Plum}{不可逆}的,即 $\textcolor{Plum}{\det} \left[ \Xint{\mathcolor{gray}{-}}{25}{\bar{\bar{\mathsfit{V}}}}^{\;\! \mathcolor{gray}{\omega} \textcolor{PineGreen}{\imath}}_{\textcolor{Maroon}{\mathbb{1}}} \right] \equiv 0$。

对应地,\bref{eq:nonlinear(2)-wave_wkrho-simplify6-V2'-SVA} 也进一步退化为系数矩阵 $1 - \hat{\mathcolor{gray}{\symup{z}}} \hat{\mathcolor{gray}{\symup{z}}}^{\mathsf{\textcolor{Plum}{T}}}$ 最末行向量、最末列向量均为零向量的,3D(三维)\textcolor{Plum}{横向}形式
\begin{align}  \label{eq:simplify6-V2'-SVA-V_1singular-nokxky-zeta}
	\left( 1 - \hat{\mathcolor{gray}{\symup{z}}} \hat{\mathcolor{gray}{\symup{z}}}^{\mathsf{\textcolor{Plum}{T}}} \right) \mathcolor{gray}{\nabla_z} \Xint{{}^{}_{\mathcolor{gray}{-}}}{10}{\bar{g}}^{\;\!\mathcolor{gray}{\omega} \textcolor{PineGreen}{\imath}}_{\;\! \mathcolor{gray}{z}} = \mathbb{i} k_{\textcolor{Maroon}{\mathsf{o}} \mathcolor{gray}{\omega}}^{\;\! 2} \frac{\Xint{\mathcolor{gray}{-}}{25}{\bar{P}}^{\;\! \mathcolor{gray}{\omega} \textcolor{PineGreen}{\imath}}_{\;\! \mathcolor{gray}{z} \textcolor{Maroon}{(2)}}}{2 \Xint{\begin{smallmatrix} ~ \\ {}^{}_{\mathcolor{gray}{-}} \\ ~ \end{smallmatrix}}{15}{k}_{\;\! \symup{z}}^{\;\! \mathcolor{gray}{\omega} \textcolor{PineGreen}{\imath}} \mathbb{e}^{\mathbb{i} \Xint{\begin{smallmatrix} ~ \\ {}^{}_{\mathcolor{gray}{-}} \\ ~ \end{smallmatrix}}{15}{k}_{\symup{z}}^{\;\! \mathcolor{gray}{\omega} \textcolor{PineGreen}{\imath}} \mathcolor{gray}{z}}} ~,
\end{align}
注意,该式并不能导出右侧源项 $\Xint{\mathcolor{gray}{-}}{24}{P}^{\;\! \mathcolor{gray}{\omega} \textcolor{PineGreen}{\imath}}_{\;\! \symup{z} \mathcolor{gray}{z} \textcolor{Maroon}{(2)}} \equiv 0$,而只是起到了将 \bref{eq:simplify6-LE0-SVA-V_1nonsingular} 化简为:在 \bref{eq:zeta-restriction-V1} 条件下,一定成立的二维形式
\begin{subequations} \label{eq:simplify7-LE0-SVA-V_1singular-nokxky-zeta}
\begin{align}
	\mathcolor{gray}{\nabla_z} \Xint{{}^{}_{\mathcolor{gray}{-}}}{10}{\bar{g}}^{\;\!\mathcolor{gray}{\omega} \textcolor{PineGreen}{\imath}}_{\;\! \symup{\rho} \mathcolor{gray}{z}} &= \mathbb{i} k_{\textcolor{Maroon}{\mathsf{o}} \mathcolor{gray}{\omega}}^{\;\! 2} \frac{\Xint{\mathcolor{gray}{-}}{25}{\bar{P}}^{\;\! \mathcolor{gray}{\omega} \textcolor{PineGreen}{\imath}}_{\;\! \symup{\rho} \mathcolor{gray}{z} \textcolor{Maroon}{(2)}}}{2 \Xint{\begin{smallmatrix} ~ \\ {}^{}_{\mathcolor{gray}{-}} \\ ~ \end{smallmatrix}}{15}{k}_{\;\! \symup{z}}^{\;\! \mathcolor{gray}{\omega} \textcolor{PineGreen}{\imath}} \mathbb{e}^{\mathbb{i} \Xint{\begin{smallmatrix} ~ \\ {}^{}_{\mathcolor{gray}{-}} \\ ~ \end{smallmatrix}}{15}{k}_{\symup{z}}^{\;\! \mathcolor{gray}{\omega} \textcolor{PineGreen}{\imath}} \mathcolor{gray}{z}}} \label{eq:simplify7-LE0-SVA-V_1singular-nokxky-zeta-g} \\
	\Xint{{}^{}_{\mathcolor{gray}{-}}}{10}{\bar{g}}^{\;\!\mathcolor{gray}{\omega} \textcolor{PineGreen}{\imath}}_{\;\! \symup{\rho} \mathcolor{gray}{z}}
	&= \Xint{{}^{}_{\mathcolor{gray}{-}}}{10}{\bar{g}}^{\;\!\mathcolor{gray}{\omega} \textcolor{PineGreen}{\imath}}_{\;\! \symup{\rho} \mathcolor{gray}{0}} + \int_{\mathcolor{gray}{0}}^{\mathcolor{gray}{z}} \mathbb{i} k_{\textcolor{Maroon}{\mathsf{o}} \mathcolor{gray}{\omega}}^{\;\! 2} \frac{\Xint{\mathcolor{gray}{-}}{25}{\bar{P}}^{\;\! \mathcolor{gray}{\omega} \textcolor{PineGreen}{\imath}}_{\;\! \symup{\rho} \mathcolor{gray}{\mathtt{z}} \textcolor{Maroon}{(2)}}}{2 \Xint{\begin{smallmatrix} ~ \\ {}^{}_{\mathcolor{gray}{-}} \\ ~ \end{smallmatrix}}{15}{k}_{\;\! \symup{z}}^{\;\! \mathcolor{gray}{\omega} \textcolor{PineGreen}{\imath}} \mathbb{e}^{\mathbb{i} \Xint{\begin{smallmatrix} ~ \\ {}^{}_{\mathcolor{gray}{-}} \\ ~ \end{smallmatrix}}{15}{k}_{\symup{z}}^{\;\! \mathcolor{gray}{\omega} \textcolor{PineGreen}{\imath}} \mathcolor{gray}{\mathtt{z}}}} ~\mathbb{d} \mathcolor{gray}{\mathtt{z}} ~, \label{eq:simplify7-LE0-SVA-V_1singular-nokxky-zeta-solution-g}
\end{align}
\end{subequations}
即约束方程左侧的场分量的(演化的)作用。同样,电场\textcolor{Maroon}{时空谱} $\Xint{{}^{}_{\mathcolor{gray}{-}}}{10}{\bar{g}}^{\;\!\mathcolor{gray}{\omega} \textcolor{PineGreen}{\jmath}}_{\;\! \mathcolor{gray}{z}}$ 的\textcolor{Plum}{横向}场分量 $\Xint{{}^{}_{\mathcolor{gray}{-}}}{10}{\bar{g}}^{\;\!\mathcolor{gray}{\omega} \textcolor{PineGreen}{\imath}}_{\;\! \symup{\rho} \mathcolor{gray}{0}}$ 由入射面 $\mathcolor{gray}{z \mathcolor{black}{=} 0}$ 处的\textcolor{Maroon}{边界条件} \bref{eq:1BC} 确定。

该 \bref{ssec:E-waveq-nonlinear} 在 \textcolor{NavyBlue}{缓变振幅近似} 条件下,从\textcolor{PineGreen}{线性光学本征模} \bref{eq:nonlinear(2)-wave_wkrho-simplify6-L} 约束下的\textcolor{PineGreen}{平面波基}\textcolor{Plum}{非线性}矢量电场波动方程 \bref{eq:nonlinear(2)-wave_wkrho-simplify6-V2'-SVA} 开始,首先强调了该\textcolor{Plum}{非齐次} 1 阶 3 维偏微分矩阵方程的系数矩阵 $\Xint{\mathcolor{gray}{-}}{25}{\bar{\bar{\mathsfit{V}}}}^{\;\! \mathcolor{gray}{\omega} \textcolor{PineGreen}{\imath}}_{\textcolor{Maroon}{\mathbb{1}}}$ \textcolor{Plum}{满秩}与否,对该方程是否有解、求解的方法,以及解的形式和性质,影响很大\Footnote{无论 $\Xint{\mathcolor{gray}{-}}{25}{\bar{\bar{\mathsfit{V}}}}^{\;\! \mathcolor{gray}{\omega} \textcolor{PineGreen}{\imath}}_{\textcolor{Maroon}{\mathbb{1}}}$ \textcolor{Plum}{满秩}与否,统一的公式形式,或许可以由 \textcolor{Plum}{广义逆}/\textcolor{Plum}{Moore-Penrose 伪逆}(并在数值上)配合 \textcolor{Plum}{奇异值分解 Singular Value Decomposition, SVD} 提供。然而即便如此,\bref{ssec:Exp-waveq} 末的矛盾仍然无法避免。因此,终极的解决方法,仍然收敛至\textcolor{NavyBlue}{非简并微扰展开}。}。一般来说,对于绝大部分人工/天然电磁材料,以及绝大部分\textcolor{gray}{空间频率} $\mathcolor{gray}{\bar{k}_{\symup{\rho}}}$ 区域,在绝大部分\textcolor{gray}{波长} $\mathcolor{gray}{\lambda}$处,$\Xint{\mathcolor{gray}{-}}{25}{\bar{\bar{\mathsfit{V}}}}^{\;\! \mathcolor{gray}{\omega} \textcolor{PineGreen}{\imath}}_{\textcolor{Maroon}{\mathbb{1}}}$ 大概率是\textcolor{Plum}{满秩}的,但又无法保证;以至于方程组的解 \bref{eq:simplify6-LE0-SVA-V_1nonsingular,simplify6-V2'-SVA-Cramer} 不一定成立。

为了使情况变得确定,该 \bref{ssec:E-waveq-nonlinear} 对\textcolor{Plum}{非线性}算子 $\Xint{\mathcolor{gray}{-}}{25}{\bar{\bar{\mathsfit{V}}}}^{\;\! \mathcolor{gray}{\omega} \textcolor{PineGreen}{\imath}}_{\textcolor{Maroon}{\mathbb{1}}}$ 先后施加了 1 个近似条件 \bref{eq:k_rho<<k_z},以及 2 个简化条件 \bref{eq:zeta-restriction-V1}。满足这 3 个条件的 $\Xint{\mathcolor{gray}{-}}{25}{\bar{\bar{\mathsfit{V}}}}^{\;\! \mathcolor{gray}{\omega} \textcolor{PineGreen}{\imath}}_{\textcolor{Maroon}{\mathbb{1}}}$ 退化为 \bref{eq:plane_wave_basis-V1-nokxky-zeta},它是确定 3 维 2 秩地\textcolor{Plum}{不可逆}。如此一来,电场\textcolor{Maroon}{时空谱} $\Xint{{}^{}_{\mathcolor{gray}{-}}}{10}{\bar{g}}^{\;\!\mathcolor{gray}{\omega} \textcolor{PineGreen}{\jmath}}_{\;\! \mathcolor{gray}{z}}$ 的\textcolor{Plum}{横向} $\symup{\bar{\rho}} := \left( \symup{x},~ \symup{y} \right)^{\mathsf{\textcolor{Plum}{T}}}$ 分量 $\Xint{{}^{}_{\mathcolor{gray}{-}}}{10}{\bar{g}}^{\;\!\mathcolor{gray}{\omega} \textcolor{PineGreen}{\jmath}}_{\;\! \symup{\rho} \mathcolor{gray}{z}} := \left( \Xint{{}^{}_{\mathcolor{gray}{-}}}{10}{g}^{\;\!\mathcolor{gray}{\omega} \textcolor{PineGreen}{\jmath}}_{\;\! \symup{x} \mathcolor{gray}{z}},~ \Xint{{}^{}_{\mathcolor{gray}{-}}}{10}{g}^{\;\!\mathcolor{gray}{\omega} \textcolor{PineGreen}{\jmath}}_{\;\! \symup{y} \mathcolor{gray}{z}} \right)^{\mathsf{\textcolor{Plum}{T}}}$ 存在唯一确定的解析 \bref{eq:simplify7-LE0-SVA-V_1singular-nokxky-zeta-solution-g}。代价便是,\textcolor{Plum}{非线性}算子 $\Xint{\mathcolor{gray}{-}}{25}{\bar{\bar{\mathsfit{V}}}}^{\;\! \mathcolor{gray}{\omega} \textcolor{PineGreen}{\imath}}_{\textcolor{Maroon}{\mathbb{1}}}$ 被近似和简化至,仅适用于\textcolor{PineGreen}{纯电(非磁)各向异性}媒介。

\clearpage

\marginLeft[-2.4em]{sec:summary-chapter2}\section{\textcolor{Maroon}{Summary} 小结 \textcolor{Maroon}{of chapter 2}}\label{sec:summary-chapter2}

为了给\textcolor{Plum}{线性}、\textcolor{Plum}{非线性}\textcolor{PineGreen}{晶体光学}的\textcolor{Plum}{分析发展}奠定坚实的(\textcolor{NavyBlue}{半})\textcolor{NavyBlue}{经典}\textcolor{Plum}{数学}和\textcolor{NavyBlue}{物理}基础,本章先在 \bref{sec:maxwell} 引入了\textcolor{NavyBlue}{电动力学}框架内,\textcolor{NavyBlue}{物理上微观起源}\textcolor{Plum}{明确}、适合作为\textcolor{Maroon}{本构关系}参考的经典\textcolor{Maroon}{多极理论}(\bref{eq:Ob-Lm})及其现代版本(\bref{eq:Ob-LM}),以概述\textcolor{NavyBlue}{源}(\bref{eq:e-b-01',eq:e-f-01'})和\textcolor{NavyBlue}{场}(\bref{eq:EB-01})的体-表\textcolor{Plum}{奇异}层次\textcolor{Plum}{多极}展开,以及它们的\textcolor{Maroon}{边界条件}(\bref{eq:E^(2-0)_0==,eq:B^(2-0)_0==,eq:1BC})。

接着,在 \bref{sec:constitutive} 介绍了与\textcolor{Maroon}{多极理论}相容的、描述弱\textcolor{Plum}{非线性}时不变系统的\textcolor{Plum}{最广义数学}框架 \textcolor{Maroon}{Volterra 级数}(\bref{eq:P_tr}),将其与\textcolor{Maroon}{多极理论}有机融合(\bref{eq:P(1)_wk,eq:P(2)_wk,eq:Q(1)_wk,eq:Q(2)_wk}),获得了不论是从\textcolor{NavyBlue}{物理}/\textcolor{NavyBlue}{微观}的层面,还是从\textcolor{Plum}{数学}/\textcolor{Plum}{宏观}的角度,均\textcolor{Plum}{自洽}且\textcolor{Plum}{精确}地描述\textcolor{Plum}{线性}、\textcolor{Plum}{非线性}\textcolor{PineGreen}{晶体光学}的、在\textcolor{Plum}{具有可计算性}前提下最广义的,以至于最合适的\textcolor{Maroon}{本构关系}/\textcolor{NavyBlue}{理论框架}:\textcolor{Plum}{非线性}\textcolor{Maroon}{多极理论}(\bref{eq:D_trho2,eq:H_trho})。此外,一路上还推导了对\textcolor{Plum}{非线性}\textcolor{NavyBlue}{光学}社区/群体比较陌生,但\textcolor{Maroon}{多极理论}专家比较熟悉的关于材料系数张量的 4 个\textcolor{Plum}{置换对称性} \bref{eq:symmetry1,eq:symmetry2,eq:symmetry3,eq:symmetry4},以进一步细分包含(58 个)磁性点群的\textcolor{Plum}{时空对称性}以扩展传统(32 个)点群概念的同时,最大程度降低计算量。

在 \bref{ssec:Exp-waveq} 中,将\textcolor{Plum}{非线性}\textcolor{Maroon}{多极理论}\textcolor{NavyBlue}{框架}下的\textcolor{Maroon}{本构关系}的\textcolor{Maroon}{傅立叶}版本(\bref{eq:D_wknabla,eq:H_wknabla}),代入 $\bar{E}^{\;\!\mathcolor{gray}{t}}_{\;\!\mathcolor{gray}{z}}, \bar{B}^{\;\!\mathcolor{gray}{t}}_{\;\!\mathcolor{gray}{z}}$ 为\textcolor{NavyBlue}{基本场}(\bref{fig:EHDB})的 \textcolor{Maroon}{Maxwell-Lorentz-Heaviside} 方程组中的这条旋度方程 \bref{eq:curl-EK} 的\textcolor{Maroon}{傅立叶}版本,得到了最广义的\textcolor{Plum}{非线性}电场矢量波动微-积分方程 \bref{eq:wave_wkrho}。并将其进一步简化至方便计算的:只保留二阶\textcolor{Plum}{局域}\textcolor{Plum}{非线性}\textcolor{NavyBlue}{电偶-$(\text{电偶}\otimes\text{电偶})$} $\Xint{{}^{}_{\mathcolor{gray}{-}}}{23}{\chi}^{\;\! \mathcolor{gray}{\omega} \hat{1} \hat{2}}_{\;\! \symup{\iota} \mathcolor{gray}{z} \textcolor{Maroon}{(2)}}$ 极响应(\bref{eq:nonlinear(2)-wave_wkrho}),只考虑至\textcolor{NavyBlue}{磁偶-电四/磁偶} $\epsilon^{\hphantom{\symup{\iota}}\mathcolor{gray}{\dot{1}}\dot{2}}_{\symup{\iota}} \Xint{\begin{smallmatrix} ~ \\ {}^{}_{\mathcolor{gray}{-}} \\ ~ \end{smallmatrix}}{13}{\varsigma}^{\;\! \mathcolor{gray}{\omega} \hat{1} \mathcolor{gray}{\check{1}}}_{\;\! \dot{2} \mathcolor{gray}{z} \textcolor{Maroon}{(1)}}$ 与 \textcolor{NavyBlue}{电偶-电八/磁四} $\Xint{\begin{smallmatrix} ~ \\ {}^{}_{\mathcolor{gray}{-}} \\ ~ \end{smallmatrix}}{16}{\varepsilon}^{\;\! \mathcolor{gray}{\omega} \hat{1} \mathcolor{gray}{\check{1} \check{2}}}_{\;\! \symup{\iota} \mathcolor{gray}{z} \textcolor{Maroon}{(1)}}$ 极\textcolor{PineGreen}{双各向异性}(\bref{eq:p<->n})、以及\textcolor{PineGreen}{折射率}\textcolor{NavyBlue}{微扰}条件(\bref{eq:weak_modulated_varepsilon})下的势散射形式 \bref{eq:nonlinear(2)-wave_wkrho-simplify4}。

然后,在 \bref{eq:L+V-decompose,eq:L+V-decompose2} 这两个等价的“\textbf{\textcolor{Plum}{非线性}\textcolor{NavyBlue}{光学}\textcolor{PineGreen}{本征系统},视为\textcolor{Plum}{线性}\textcolor{NavyBlue}{光学}\textcolor{PineGreen}{本征系统}的 \textcolor{NavyBlue}{0 阶微扰}}”条件下,将 \bref{eq:nonlinear(2)-wave_wkrho-simplify4} 解耦为分别由\textcolor{Plum}{线性}和\textcolor{Plum}{非线性}算子 $\Xint{\mathcolor{gray}{-}}{30}{\bar{\bar{L}}}^{\;\! \mathcolor{gray}{\omega}},\Xint{\mathcolor{gray}{-}}{21}{\bar{\bar{V}}}^{\;\! \mathcolor{gray}{\omega}}$ 控制的 2 个(\textcolor{Plum}{卷积型微-积分})方程组(\bref{eq:nonlinear(2)-wave_wkrho-simplify6-L4,eq:nonlinear(2)-wave_wkrho-simplify6-V})。

得到了方程之后,就需要对其求解。在对比了\textcolor{Plum}{线性}\textcolor{PineGreen}{晶体光学}波动方程 \bref{eq:nonlinear(2)-wave_wkrho-simplify6-L4} 的\textcolor{Maroon}{矩阵指数解} \bref{eq:vec-matrix_exp} 的优势(帮助区分\textcolor{Plum}{有限个数}的\textcolor{PineGreen}{光学奇点})与劣势(详见 \bref{ssec:Exp-solution-linear} 末)之后,我们选择采用\textcolor{Maroon}{非矩阵指数}的、\textcolor{PineGreen}{线性叠加}的\textcolor{gray}{单色}\textcolor{PineGreen}{平面波解} \bref{eq:vec-plane_wave_basis} 来求解 \bref{eq:nonlinear(2)-wave_wkrho-simplify6-L4,eq:nonlinear(2)-wave_wkrho-simplify6-V} 的对应 \bref{eq:nonlinear(2)-wave_wkrho-simplify6'}。

在 \bref{eq:L+V-polar} 的条件下,\textcolor{PineGreen}{平面波解} \bref{eq:vec-plane_wave_basis} 及其所满足的\textcolor{Plum}{线性}、\textcolor{Plum}{非线性}方程 \bref{eq:nonlinear(2)-wave_wkrho-simplify6'},分离出不含 $\mathcolor{gray}{z}$ 的\textcolor{PineGreen}{本征偏振态} $\Xint{{}^{}_{\mathcolor{gray}{-}}}{10}{\bar{g}}^{\;\!\mathcolor{gray}{\omega} \textcolor{PineGreen}{\jmath}}$,以及含 $\mathcolor{gray}{z}$ 的\textcolor{PineGreen}{本征复振幅} $\Xint{\begin{smallmatrix} ~ \\ {}^{}_{\mathcolor{gray}{-}} \\ ~ \end{smallmatrix}}{09}{\mathtt{g}}^{\;\!\mathcolor{gray}{\omega} \textcolor{PineGreen}{\jmath}}_{\;\! \mathcolor{gray}{z}}$ 两部分,最终分别进化为 \bref{eq:amp_polar_phase,eq:nonlinear(2)-wave_wkrho-simplify6''},为 \bref{ssec:E-waveq-nonlinear,ssec:eigenmodes-compamp} 预求解\textcolor{PineGreen}{本征平面波基}下的这 2 个\textcolor{Plum}{线性}、\textcolor{Plum}{非线性}波动方程作好铺垫。从这以后,\textcolor{PineGreen}{例外点}/\textcolor{PineGreen}{光学奇点}的\textcolor{Plum}{内部结构}无法再被分辨清楚,但好处是允许快速准确地计算大部分\textcolor{gray}{空间频率}域 $\mathcolor{gray}{\bar{k}_{\symup{\rho}}} \in \mathcolor{gray}{\bar{\mathbb{R}}_{\textcolor{Plum}{2}}}$ 的\textcolor{PineGreen}{本征模} $\Xint{\mathcolor{gray}{-}}{25}{\bar{E}}^{\;\!\mathcolor{gray}{\omega} \textcolor{PineGreen}{\jmath}}_{\;\! \mathcolor{gray}{z}}$ 、电场\textcolor{Maroon}{时空谱} $\Xint{{}^{}_{\mathcolor{gray}{-}}}{10}{\bar{g}}^{\;\!\mathcolor{gray}{\omega} \textcolor{PineGreen}{\jmath}}_{\;\! \mathcolor{gray}{z}}$ 和\textcolor{Maroon}{傅立叶谱} $\Xint{\mathcolor{gray}{-}}{25}{\bar{E}}^{\;\!\mathcolor{gray}{\omega}}_{\;\! \mathcolor{gray}{z}}$。



