% based on 北理工博士毕业论文 的 chapter2 https://tex.nju.edu.cn/project/user/10d0ed00-6f46-42fa-8a3a-18270214cfa3/a0ed1339-764d-4cc7-9ac4-d5d05b33da53

\chapter{晶体中的线性、非线性光学过程}

%\chapter{\protect\hyperlink{chap:\thechapter}{晶体中的线性、非线性光学过程的解析解}}
%\addtocontents{toc}{\protect\linkdest{chap:\thechapter}}
%\label{晶体中的线性、非线性光学过程的解析解}
%
%%\section{晶体中的电场混频方程组}
%\section{\protect\hyperlink{chap:\thesection}{晶体中的电场混频方程组}}
%\addtocontents{toc}{\protect\linkdest{chap:\thesection}}
%\label{晶体中的电场混频方程组}

相对观察者静止\Footnote{否则 $\bar{D}, \bar{B}$ 将进一步相互耦合\cite{berryOpticalSingularitiesBianisotropic2005},以致材料的本构关系将固有双各向异性\cite{mackayElectromagneticAnisotropyBianisotropy2019,mackayModernAnalyticalElectromagnetic2020};$\bar{J}_{\;\!\textcolor{Maroon}{\symup{f}}}$ 也需要扩展到四维\cite{XieQuanMianHuiYiWas}。} 的三维空间 $\textcolor{gray}{\bar{r}}$ 坐标系下,在可能存在非零电荷源和电流源 ${\rho}_{\;\!\textcolor{Maroon}{\symup{f}}} \left( \textcolor{gray}{\bar{r}}, \textcolor{gray}{t} \right),$ $\bar{J}_{\;\!\textcolor{Maroon}{\symup{f}}} \left( \textcolor{gray}{\bar{r}}, \textcolor{gray}{t} \right)$\Footnote{对于符号约定,比如下标 $\textcolor{Maroon}{\symup{f}}$ 的\textcolor{Maroon}{褐红色}及其含义 `free',其定义见\bref{Maroon};此外,在 \bref{1bar} 中还约定:总使用 1 条上短横线 $\bar{~}$(而不是粗体)来表示矢量(如 $\bar{J}_{\;\!\textcolor{Maroon}{\symup{f}}}$),以区别于 \bref{0bar} 中定义的无上短横线的标量(如 ${\rho}_{\;\!\textcolor{Maroon}{\symup{f}}}$);粗体在本文中另有其含义,不用于表示矢量,见。} 的一般电磁介质内部,4 个空域时变\Footnote{指复矢量场 $\bar{E}, \bar{H}, \bar{D}, \bar{B}$ 均是四维时空 $\textcolor{gray}{\bar{r}}, \textcolor{gray}{t}$ 的函数,且因此均是复色 $\left\{ \omega \in \mathbbm{R} \right\}$ 的复场 $\in \mathbbm{C}^3 \left( \mathbbm{R}^3 \right)$,属于在视觉上占主导的因变量,并用黑色(见 \bref{black})的 \textit{斜体 oblique}(见 \bref{oblique})表示;相对地,自变量用视觉和含义上均更次要的\textcolor{gray}{灰色}来表示,见 \bref{gray}。}复色场 
$\bar{E}^{\;\!\textcolor{gray}{t}}_{\;\!\textcolor{gray}{z}}, \bar{H}^{\;\!\textcolor{gray}{t}}_{\;\!\textcolor{gray}{z}}, \bar{D}^{\;\!\textcolor{gray}{t}}_{\;\!\textcolor{gray}{z}}, \bar{B}^{\;\!\textcolor{gray}{t}}_{\;\!\textcolor{gray}{z}}$\Footnote{由于傅立叶光学一般运行在平行平面间,约定下述表示相互等价:$\bar{E} \left( \textcolor{gray}{\bar{r}}, \textcolor{gray}{t} \right) = \bar{E}^{\;\!\textcolor{gray}{t}}_{\;\!\textcolor{gray}{\bar{r}}} = \bar{E}^{\;\!\textcolor{gray}{t}}_{\;\!\textcolor{gray}{z}} \left( \textcolor{gray}{\bar{\rho}} \right)$,并因此经常省略面内自变量 $\textcolor{gray}{\bar{\rho}}$,以只写作朝 $\textcolor{Maroon}{+\symup{z}}$ 轴传播距离 $\textcolor{gray}{z}$ 的函数 $\bar{E}^{\;\!\textcolor{gray}{t}}_{\;\!\textcolor{gray}{z}} := \bar{E}^{\;\!\textcolor{gray}{t}}_{\;\!\textcolor{gray}{z}} \left( \textcolor{gray}{\bar{\rho}} \right)$。—— 同样的规则也适用于其他场量(如 ${\rho}_{\;\!\textcolor{Maroon}{\symup{f}}} \left( \textcolor{gray}{\bar{r}}, \textcolor{gray}{t} \right) \to {\rho}^{\;\!\textcolor{gray}{t}}_{\;\!\textcolor{Maroon}{\symup{f}}\textcolor{gray}{z}}$),且适用于空间频率域,见。},满足微分形式的麦氏方程组
\begin{subequations} \label{eq:maxwell}
\begin{align}
	\textcolor{Maroon}{\text{Coulomb's}} \to \textcolor{Maroon}{\text{Gauss's law for electric field}}\text{:}&\hspace{0.5em} \bar{\nabla} \cdot \bar{D}^{\;\!\textcolor{gray}{t}}_{\;\!\textcolor{gray}{z}} \hspace{-1.5em} &&= {\rho}^{\;\!\textcolor{gray}{t}}_{\;\!\textcolor{Maroon}{\symup{f}}\textcolor{gray}{z}}~, \label{eq:maxwell-d} \\ \textcolor{Maroon}{\text{Biot-Savart}} \to \textcolor{Maroon}{\text{Gauss's law for magnetic field}}\text{:}&\hspace{0.5em} \bar{\nabla} \cdot \bar{B}^{\;\!\textcolor{gray}{t}}_{\;\!\textcolor{gray}{z}} \hspace{-1.5em} &&= 0~, \label{eq:maxwell-b} \\ \textcolor{Maroon}{\text{Faraday's law of electromagnetic induction}}\text{:}&\hspace{0.5em} \bar{\nabla} \times \bar{E}^{\;\!\textcolor{gray}{t}}_{\;\!\textcolor{gray}{z}} \hspace{-1.5em} &&= - \frac{\partial \bar{B}^{\;\!\textcolor{gray}{t}}_{\;\!\textcolor{gray}{z}}}{\partial t}~, \label{eq:maxwell-e} \\ \textcolor{Maroon}{\text{Jefimenko's}} \to \textcolor{Maroon}{\text{Maxwell-Amp\`{e}re circuital law}}\text{:}&\hspace{0.5em} \bar{\nabla} \times \bar{H}^{\;\!\textcolor{gray}{t}}_{\;\!\textcolor{gray}{z}} \hspace{-1.5em} &&= \bar{J}^{\;\!\textcolor{gray}{t}}_{\;\!\textcolor{Maroon}{\symup{f}}\textcolor{gray}{z}} + \frac{\partial \bar{D}^{\;\!\textcolor{gray}{t}}_{\;\!\textcolor{gray}{z}}}{\partial t}~. \label{eq:maxwell-h}
\end{align}
\end{subequations}
其中,定义了如下 4 个本构关系:
\begin{subequations} \label{eq:cr}
\begin{align}
	\textcolor{Maroon}{\text{CR for magnetism}}\text{:}&\hspace{0.5em} \bar{B}^{\;\!\textcolor{gray}{t}}_{\;\!\textcolor{gray}{z}} \hspace{-1.5em} &&{\symup{\mu}}_0 \left( \bar{H}^{\;\!\textcolor{gray}{t}}_{\;\!\textcolor{gray}{z}} + \bar{M}^{\;\!\textcolor{gray}{t}}_{\;\!\textcolor{gray}{z}} \right) = {\symup{\mu}}_0 \left\{ \bar{\bar{\delta}}^{\;\!\textcolor{gray}{t}}~\widetilde *~\bar{H}^{\;\!\textcolor{gray}{t}}_{\;\!\textcolor{gray}{z}} + \bar{M}^{\;\!\textcolor{gray}{t}}_{\;\!\textcolor{gray}{z}} \right\} \\ & &&\xrightarrow[]{\bar{M}^{\;\!\textcolor{gray}{t}}_{\;\!\textcolor{gray}{z}} = \bar{M}^{\;\!\textcolor{Maroon}{\text{(1)}} \textcolor{gray}{t}}_{\;\!\textcolor{gray}{z}} + \bar{M}^{\;\!\textcolor{Maroon}{\text{NL}}, \textcolor{gray}{t}}_{\;\!\textcolor{gray}{z}}} {\symup{\mu}}_0 \left\{ \left[ \bar{\bar{\delta}}^{\;\!\textcolor{gray}{t}}~\widetilde *~\bar{H}^{\;\!\textcolor{gray}{t}}_{\;\!\textcolor{gray}{z}} + \bar{M}^{\;\!\textcolor{Maroon}{\text{(1)}} \textcolor{gray}{t}}_{\;\!\textcolor{gray}{z}} \right] + \bar{M}^{\;\!\textcolor{Maroon}{\text{NL}}, \textcolor{gray}{t}}_{\;\!\textcolor{gray}{z}} \right\} \\ & &&\xrightarrow[\displaystyle{ \bar{\bar{\mu}}^{\;\!\textcolor{Maroon}{\text{(1)}} \textcolor{gray}{t}}_{\;\!\textcolor{Maroon}{\text{r}}\textcolor{gray}{z}} := \bar{\bar{\delta}}^{\;\!\textcolor{gray}{t}} + \bar{\bar{\chi}}^{\;\!\textcolor{Maroon}{\text{(1)}}\textcolor{gray}{t}}_{\;\!\textcolor{Maroon}{\text{m}} \textcolor{gray}{z}}}]{\displaystyle{\bar{M}^{\;\!\textcolor{Maroon}{\text{(1)}} \textcolor{gray}{t}}_{\;\!\textcolor{gray}{z}} := \bar{\bar{\chi}}^{\;\!\textcolor{Maroon}{\text{(1)}}\textcolor{gray}{t}}_{\;\!\textcolor{Maroon}{\text{m}} \textcolor{gray}{z}} ~\widetilde *~\bar{H}^{\;\!\textcolor{gray}{t}}_{\;\!\textcolor{gray}{z}}}} {\symup{\mu}}_0 \left\{ \bar{\bar{\mu}}^{\;\!\textcolor{Maroon}{\text{(1)}} \textcolor{gray}{t}}_{\;\!\textcolor{Maroon}{\text{r}}\textcolor{gray}{z}}~\widetilde *~\bar{H}^{\;\!\textcolor{gray}{t}}_{\;\!\textcolor{gray}{z}} + \bar{M}^{\;\!\textcolor{Maroon}{\text{NL}}, \textcolor{gray}{t}}_{\;\!\textcolor{gray}{z}} \right\} \\ & &&= \bar{\bar{\mu}}^{\;\!\textcolor{Maroon}{\text{(1)}} \textcolor{gray}{t}}_{\;\!\textcolor{gray}{z}}~\widetilde *~\bar{H}^{\;\!\textcolor{gray}{t}}_{\;\!\textcolor{gray}{z}} + {\symup{\mu}}_0 \bar{M}^{\;\!\textcolor{Maroon}{\text{NL}}, \textcolor{gray}{t}}_{\;\!\textcolor{gray}{z}} =: \bar{B}^{\;\!\textcolor{Maroon}{\text{(1)}} \textcolor{gray}{t}}_{\;\!\textcolor{gray}{z}} + \bar{B}^{\;\!\textcolor{Maroon}{\text{NL}}, \textcolor{gray}{t}}_{\;\!\textcolor{gray}{z}}~, \label{cr-b} \\
	\textcolor{Maroon}{\text{Ohm's law}}\text{:}&\hspace{0.5em} \bar{J}^{\;\!\textcolor{gray}{t}}_{\;\!\textcolor{Maroon}{\symup{f}}\textcolor{gray}{z}} \hspace{-1.5em} &&= \bar{\bar{\sigma}}^{\;\!\textcolor{gray}{t}}_{\;\!\textcolor{gray}{z}}~\widetilde *~\bar{E}^{\;\!\textcolor{gray}{t}}_{\;\!\textcolor{gray}{z}} + \bar{J}^{\;\!\textcolor{Maroon}{\text{NL}}, \textcolor{gray}{t}}_{\;\!\textcolor{Maroon}{\symup{f}}\textcolor{gray}{z}} =: \bar{J}^{\;\!\textcolor{Maroon}{\text{(1)}} \textcolor{gray}{t}}_{\;\!\textcolor{Maroon}{\symup{f}}\textcolor{gray}{z}} + \bar{J}^{\;\!\textcolor{Maroon}{\text{NL}}, \textcolor{gray}{t}}_{\;\!\textcolor{Maroon}{\symup{f}}\textcolor{gray}{z}}~, \label{cr-ohm} \\
	\textcolor{Maroon}{\text{CR for electricity}}\text{:}&\hspace{0.5em} \bar{D}^{\;\!\textcolor{gray}{t}}_{\;\!\textcolor{gray}{z}} \hspace{-1.5em} &&= {\symup{\varepsilon}}_0 \bar{E}^{\;\!\textcolor{gray}{t}}_{\;\!\textcolor{gray}{z}} + \bar{P}^{\;\!\textcolor{gray}{t}}_{\;\!\textcolor{gray}{z}} = {\symup{\varepsilon}}_0 \bar{\bar{\delta}}^{\;\!\textcolor{gray}{t}}~\widetilde *~\bar{E}^{\;\!\textcolor{gray}{t}}_{\;\!\textcolor{gray}{z}} + \bar{P}^{\;\!\textcolor{gray}{t}}_{\;\!\textcolor{gray}{z}} \\ & &&\xrightarrow[]{\bar{P}^{\;\!\textcolor{gray}{t}}_{\;\!\textcolor{gray}{z}} = \bar{P}^{\;\!\textcolor{Maroon}{\text{(1)}} \textcolor{gray}{t}}_{\;\!\textcolor{gray}{z}} + \bar{P}^{\;\!\textcolor{Maroon}{\text{NL}}, \textcolor{gray}{t}}_{\;\!\textcolor{gray}{z}} + } \left[ {\symup{\varepsilon}}_0 \bar{\bar{\delta}}^{\;\!\textcolor{gray}{t}}~\widetilde *~\bar{E}^{\;\!\textcolor{gray}{t}}_{\;\!\textcolor{gray}{z}} + \bar{P}^{\;\!\textcolor{Maroon}{\text{(1)}} \textcolor{gray}{t}}_{\;\!\textcolor{gray}{z}} \right] + \bar{P}^{\;\!\textcolor{Maroon}{\text{NL}}, \textcolor{gray}{t}}_{\;\!\textcolor{gray}{z}} \\ & &&\xrightarrow[\displaystyle{ \bar{\bar{\varepsilon}}^{\;\!\textcolor{Maroon}{\text{(1)}} \textcolor{gray}{t}}_{\;\!\textcolor{Maroon}{\text{r}}\textcolor{gray}{z}} := \bar{\bar{\delta}}^{\;\!\textcolor{gray}{t}} + \bar{\bar{\chi}}^{\;\!\textcolor{Maroon}{\text{(1)}}\textcolor{gray}{t}}_{\;\!\textcolor{Maroon}{\text{e}} \textcolor{gray}{z}}}]{\displaystyle{\bar{P}^{\;\!\textcolor{Maroon}{\text{(1)}} \textcolor{gray}{t}}_{\;\!\textcolor{gray}{z}} := \bar{\bar{\chi}}^{\;\!\textcolor{Maroon}{\text{(1)}}\textcolor{gray}{t}}_{\;\!\textcolor{Maroon}{\text{e}} \textcolor{gray}{z}} ~\widetilde *~\bar{E}^{\;\!\textcolor{gray}{t}}_{\;\!\textcolor{gray}{z}}}} {\symup{\varepsilon}}_0 \bar{\bar{\varepsilon}}^{\;\!\textcolor{Maroon}{\text{(1)}} \textcolor{gray}{t}}_{\;\!\textcolor{Maroon}{\text{r}}\textcolor{gray}{z}}~\widetilde *~\bar{E}^{\;\!\textcolor{gray}{t}}_{\;\!\textcolor{gray}{z}} + \bar{P}^{\;\!\textcolor{Maroon}{\text{NL}}, \textcolor{gray}{t}}_{\;\!\textcolor{gray}{z}} \\ & &&= \bar{\bar{\varepsilon}}^{\;\!\textcolor{Maroon}{\text{(1)}} \textcolor{gray}{t}}_{\;\!\textcolor{gray}{z}}~\widetilde *~\bar{E}^{\;\!\textcolor{gray}{t}}_{\;\!\textcolor{gray}{z}} + \bar{P}^{\;\!\textcolor{Maroon}{\text{NL}}, \textcolor{gray}{t}}_{\;\!\textcolor{gray}{z}} =: \bar{D}^{\;\!\textcolor{Maroon}{\text{(1)}} \textcolor{gray}{t}}_{\;\!\textcolor{gray}{z}} + \bar{D}^{\;\!\textcolor{Maroon}{\text{NL}}, \textcolor{gray}{t}}_{\;\!\textcolor{gray}{z}}~, \label{cr-d}
\end{align}
\end{subequations}
%\bar{\bar{\bar{\sigma}}}^{\;\!\textcolor{gray}{t}}_{\;\!\textcolor{gray}{z}}~{}^{\widetilde *}_{\widetilde *} \left( \bar{E}^{\;\!\textcolor{gray}{t}}_{\;\!\textcolor{gray}{z}} \bar{E}^{\;\!\textcolor{gray}{t}}_{\;\!\textcolor{gray}{z}} \right) + \bar{\bar{\bar{\bar{\sigma}}}}^{\;\!\textcolor{gray}{t}}_{\;\!\textcolor{gray}{z}}~\begin{smallmatrix} \widetilde * \\ \widetilde * \\ \widetilde * \end{smallmatrix} \left( \bar{E}^{\;\!\textcolor{gray}{t}}_{\;\!\textcolor{gray}{z}} \bar{E}^{\;\!\textcolor{gray}{t}}_{\;\!\textcolor{gray}{z}} \bar{E}^{\;\!\textcolor{gray}{t}}_{\;\!\textcolor{gray}{z}} \right) + \cdots~
带入



