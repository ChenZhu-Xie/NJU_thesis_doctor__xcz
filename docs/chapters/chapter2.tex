% based on 北理工博士毕业论文 的 chapter2 https://tex.nju.edu.cn/project/user/10d0ed00-6f46-42fa-8a3a-18270214cfa3/a0ed1339-764d-4cc7-9ac4-d5d05b33da53

\chapter{晶体中的线性、非线性光学过程}

%\chapter{\protect\hyperlink{chap:\thechapter}{晶体中的线性、非线性光学过程的解析解}}
%\addtocontents{toc}{\protect\linkdest{chap:\thechapter}}
%\label{晶体中的线性、非线性光学过程的解析解}
%
%%\section{晶体中的电场混频方程组}
%\section{\protect\hyperlink{chap:\thesection}{晶体中的电场混频方程组}}
%\addtocontents{toc}{\protect\linkdest{chap:\thesection}}
%\label{晶体中的电场混频方程组}

相对静止\Footnote{否则 $\widetilde{\symbf D}, \widetilde{\symbf B}$ 间将相互耦合\cite{berryOpticalSingularitiesBianisotropic2005},以致材料的本构关系通常会呈现出双各向异性\cite{mackayElectromagneticAnisotropyBianisotropy2019,mackayModernAnalyticalElectromagnetic2020};$\widetilde{\symbf J}_{\symup{f}}$ 也需要扩展到四维\cite{XieQuanMianHuiYiWas}。} 的坐标系下,可能存在非零电荷源和电流源 ${\rho}_{\color{Maroon} \symup{f}} \left( {\color{gray} \bar{r}}, {\color{gray} t} \right), \bar{J}_{\color{Maroon} \symup{f}} \left( {\color{gray} \bar{r}}, {\color{gray} t} \right)$ 的一般电磁介质内部,4 个空域时变\Footnote{指复矢量场 $ \bar{E}, \bar{H}, \bar{D}, \bar{B} $ 均是四维时空 ${\color{gray} \bar{r}}, {\color{gray} t}$ 的函数,并因此通常不是单色的;对于约定,见\bref{Maroon}。}复色场 
$ \widetilde{\symbf E}, \widetilde{\symbf H},  \widetilde{\symbf D}, \widetilde{\symbf B} $,满足微分形式的麦氏方程组
\bref{eq:r-2}

\begin{align} \label{eq:r-2}
	\left( k^{2}_{\omega} - \bar{k}^{\color{gray}\omega}\bar{k}^{\intercal}_{\omega} - k^{2}_{0\omega} \bar{\bar{\varepsilon}}^{\;\!\prime\omega}_{\mathrm{r} z} \right) \cdot \bar{g}^{\;\!\omega} = \bar{0}.
\end{align}

testsetse\cite{ossikovskiConstitutiveRelationsOptically2021}

%\begin{subequations} \label{eq:2-1}
%	\begin{align} 
%		\left\{\ \begin{aligned}\nabla \cdot \widetilde{\symbf D} &= {\widetilde \rho}_{\symup{f}}  \\\nabla \cdot \widetilde{\symbf B} &= 0 \\\nabla \times \widetilde{\symbf E} &= - \frac{\partial \widetilde{\symbf B}}{\partial t} \\ \nabla \times \widetilde{\symbf H} &= \widetilde{\symbf J}_{\symup{f}} + \frac{\partial \widetilde{\symbf D}}{\partial t} \end{aligned}\right. \xrightarrow[{\symup{inside\ material}}]{\left\{\ \begin{aligned}{\widetilde \rho}_{\symup{f}} &\rightarrow 0  \\ \widetilde{\symbf J}_{\symup{f}} &= {\widetilde \rho}_{\symup{f}} \cdot {\symbf v}_{\symup{f}} \end{aligned}\right.} \left\{\ \begin{aligned}\nabla \cdot \widetilde{\symbf D} &= 0 \\\nabla \cdot \widetilde{\symbf B} &= 0 \\\nabla \times \widetilde{\symbf E} &= - \frac{\partial \widetilde{\symbf B}}{\partial t} \\ \nabla \times \widetilde{\symbf H} &= \widetilde{\symbf J}_{\symup{f}} + \frac{\partial \widetilde{\symbf D}}{\partial t} \end{aligned}\right. ~,
%	\end{align}
%\end{subequations}

