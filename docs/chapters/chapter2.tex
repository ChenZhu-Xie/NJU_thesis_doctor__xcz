\marginLeft[-2.5em]{chap:N/LCO}\chapter{晶体中的(非)线性光学过程}\label{chap:N/LCO}

\marginLeft[-2.5em]{sec:maxwell}\section{\textcolor{Maroon}{Maxwell-Lorentz-Heaviside} 方程组、电磁源}\label{sec:maxwell}

这一节从经典/现代的多极理论/张量分析视角,查看麦氏方程组、源和场的本构关系、守恒律、连续性、边界条件。

\marginLeft[-2.5em]{ssec:EHpJf}\subsection{$\bar{E},\bar{H}$ 双旋度、${\rho}_{\;\!\textcolor{Maroon}{\text{f}}}, \bar{J}_{\;\!\textcolor{Maroon}{\text{f}}}$ 裸源连续}\label{ssec:EHpJf}

相对观察者静止\Footnote{否则标/矢/张量场的2个实自变量:时间$\mathcolor{gray}{t} \in \mathcolor{gray}{\mathbb{R}_1}$和空间$\mathcolor{gray}{\bar{r}} \in \mathcolor{gray}{\bar{\mathbb{R}}^{\textcolor{Maroon}{(1)}}_3}$将在其他参考系(见\bref{uwav})的度量下相互混合,即$\mathcolor{gray}{\uwav{t}} \left( \mathcolor{gray}{t}, \mathcolor{gray}{\bar{r}} \right), \mathcolor{gray}{\uwav{\bar{r}}} \left( \mathcolor{gray}{t}, \mathcolor{gray}{\bar{r}} \right)$,并形成统一的$1+3$维黎曼时空/微分流形$\mathcolor{gray}{\uwav{\bar{x}}} \left( \mathcolor{gray}{\bar{x}} \right) \in \mathcolor{gray}{\bar{\mathbb{R}}_4}$;接着,定义四维位移矢量$\mathcolor{gray}{\bar{x}}$对四维标量固有时$\mathcolor{gray}{\uo{t}}$(见\bref{uo})的导数:四维速度矢量$\bar{u}_{\;\!\mathcolor{gray}{\bar{x}}} := \mathbb{d} \mathcolor{gray}{\bar{x}} \big/ \mathbb{d} \mathcolor{gray}{\uo{t}} \in \bar{\mathbb{C}}_4 ( \mathcolor{gray}{\bar{\mathbb{R}}_4} )$,并将总电流 $\bar{J}^{\;\!\mathcolor{gray}{t}}_{\;\!\mathcolor{gray}{z}} = {\rho}^{\;\!\mathcolor{gray}{t}}_{\;\!\mathcolor{gray}{z}} \bar{v}^{\;\!\mathcolor{gray}{t}}_{\;\!\mathcolor{gray}{z}}$ 与总电荷${\rho}^{\;\!\mathcolor{gray}{t}}_{\;\!\mathcolor{gray}{z}}$合并以扩展至四维$\bar{J}_{\;\! \mathcolor{gray}{\bar{x}}} = \uo{\rho}_{\;\! \mathcolor{gray}{\uo{\bar{x}}}} \bar{u}_{\;\! \mathcolor{gray}{\bar{x}}}  \in \bar{\mathbb{C}}_4 ( \mathcolor{gray}{\bar{\mathbb{R}}_4} )$\cite{lakhtakiaCovariancesInvariancesMaxwell1995,chen-zhuChenZhuxieUndergraduate_courses2024};此外,2个基本场 $\bar{E}_{\;\!\mathcolor{gray}{\bar{x}}}$ 与 $\bar{B}_{\;\!\mathcolor{gray}{\bar{x}}}$ 也将相互耦合,即$\uwav{\bar{E}}_{\;\!\mathcolor{gray}{\uwav{\bar{x}}}} \left( \bar{E}_{\;\!\mathcolor{gray}{\bar{x}}},\bar{B}_{\;\!\mathcolor{gray}{\bar{x}}} \right), \uwav{\bar{B}}_{\;\!\mathcolor{gray}{\uwav{\bar{x}}}} \left( \bar{E}_{\;\!\mathcolor{gray}{\bar{x}}},\bar{B}_{\;\!\mathcolor{gray}{\bar{x}}} \right)$,并成为一个整体:2阶4维电磁张量场 $\uwav{\bar{\bar{F}}}_{\;\!\mathcolor{gray}{\uwav{\bar{x}}}} ( \bar{\bar{F}}_{\;\!\mathcolor{gray}{\bar{x}}} ) \in \bar{\bar{\mathbb{C}}}_{\left[4 \times 4\right]} ( \bar{\mathbb{R}}_4 )$\cite{lakhtakiaCovariancesInvariancesMaxwell1995,berryOpticalSingularitiesBianisotropic2005,chen-zhuChenZhuxieUndergraduate_courses2024},以致材料的本构关系将呈现固有双各向异性$\uwav{\bar{D}}_{\;\!\mathcolor{gray}{\uwav{\bar{x}}}} [ \uwav{\bar{E}}_{\;\!\mathcolor{gray}{\uwav{\bar{x}}}} \left( \bar{E}_{\;\!\mathcolor{gray}{\bar{x}}},\bar{B}_{\;\!\mathcolor{gray}{\bar{x}}} \right) ], \uwav{\bar{H}}_{\;\!\mathcolor{gray}{\uwav{\bar{x}}}} [ \uwav{\bar{B}}_{\;\!\mathcolor{gray}{\uwav{\bar{x}}}} \left( \bar{E}_{\;\!\mathcolor{gray}{\bar{x}}},\bar{B}_{\;\!\mathcolor{gray}{\bar{x}}} \right) ]$或$\uwav{\bar{\bar{G}}}_{\;\!\mathcolor{gray}{\uwav{\bar{x}}}} [ \uwav{\bar{\bar{F}}}_{\;\!\mathcolor{gray}{\uwav{\bar{x}}}} ( \bar{\bar{F}}_{\;\!\mathcolor{gray}{\bar{x}}} ) ] = \uwav{\bar{\bar{G}}}_{\;\!\mathcolor{gray}{\uwav{\bar{x}}}} [ \bar{\bar{G}}_{\;\!\mathcolor{gray}{\bar{x}}} ( \bar{\bar{F}}_{\;\!\mathcolor{gray}{\bar{x}}} ) ]$\cite{langeMultipoleTheoryHehl2015,hehlLinearMediaClassical2005,mackayElectromagneticAnisotropyBianisotropy2019,mackayModernAnalyticalElectromagnetic2020,lakhtakiaCovariancesInvariancesMaxwell1995}。此外,${\rho}^{\;\!\mathcolor{gray}{t}}_{\;\!\mathcolor{gray}{z}}$对应的束缚/自由=价带/导带电子的(有效)运动质量(或相对论速度)也会变大,并因此同时对线性和非线性极化、磁化、自由电流$\bar{J}^{\;\!\mathcolor{gray}{t}}_{\;\!\textcolor{Maroon}{\text{f}}\mathcolor{gray}{z}} = {\rho}^{\;\!\mathcolor{gray}{t}}_{\;\!\textcolor{Maroon}{\text{f}}\mathcolor{gray}{z}} \bar{v}^{\;\!\mathcolor{gray}{t}}_{\;\!\textcolor{Maroon}{\text{f}}\mathcolor{gray}{z}}$产生额外影响。——上述场景可发生在超快和强场泵浦(及其通过多光子/隧穿电离产生的等离子体)中\cite{boydNonlinearOptics2019}、相对材料运动的坐标系下,甚至一直在发生在所有材料内(核库伦力导致的电子轨道运动:金的颜色\cite{boydNonlinearOptics2019}、汞常温液体\cite{pyykkoRelativisticEffectsChemistry2012}、铅酸电池的额外电压和稳定性\cite{ahujaRelativityLeadacidBattery2011};狄拉克方程描述的电子自旋运动、自旋-轨道、自旋-自旋耦合\cite{pyykkoRelativisticEffectsChemistry2012}:许多磁效应\cite{chen-zhuChenZhuxieUndergraduate_courses2024}),以至于可能(必)需要引入相对论(量子)电动力学——因为电子一直在材料内运动,尽管材料本身相对观察者的参考系静止。此外,所有的磁效应的本质和来源,似乎都可以通过纯电的方式解释,见 \bref{ssec:EBpJ,ssec:PMQN}。} 的 3 维空间$\mathcolor{gray}{\bar{r}}$ 坐标系下,在可能存在非零的时 $\mathcolor{gray}{t}$ 变自由电荷源和自由电流源${\rho}_{\;\!\textcolor{Maroon}{\text{f}}} \left( \mathcolor{gray}{\bar{r}}, \mathcolor{gray}{t} \right), \bar{J}_{\;\!\textcolor{Maroon}{\text{f}}} \left( \mathcolor{gray}{\bar{r}}, \mathcolor{gray}{t} \right)$\Footnote{对于符号约定,比如下标 $\textcolor{Maroon}{\text{f}}$ 的\textcolor{Maroon}{褐红色}及其含义 `\textcolor{Maroon}{free}',其定义见\bref{Maroon};此外,在 \bref{1bar} 中还约定:总使用 1 条上短横线 $\bar{~}$(而不是粗体)来表示矢量(如 $\bar{J}_{\;\!\textcolor{Maroon}{\text{f}}}$),以区别于 \bref{0bar} 中定义的无上短横线的标量(如 ${\rho}_{\;\!\textcolor{Maroon}{\text{f}}}$);粗体在本文中另有其含义,不用于表示矢量,见\bref{bold}。} 的,相对静止的一般电磁介质内部,4 个空域时变\Footnote{指复矢量场 $\bar{E}^{\;\!\mathcolor{gray}{t}}_{\;\!\mathcolor{gray}{z}}, \bar{H}^{\;\!\mathcolor{gray}{t}}_{\;\!\mathcolor{gray}{z}}, \bar{D}^{\;\!\mathcolor{gray}{t}}_{\;\!\mathcolor{gray}{z}}, \bar{B}^{\;\!\mathcolor{gray}{t}}_{\;\!\mathcolor{gray}{z}}$ 均是四维时空 $\mathcolor{gray}{\bar{r}}, \mathcolor{gray}{t}$ 的函数,且因此一般意义上是复色 $\left\{ \mathcolor{gray}{\omega} \in \mathcolor{gray}{\mathbb{R}} \right\}$ 的复场 $\in \bar{\mathbb{C}}_3 ( \mathcolor{gray}{\bar{\mathbb{R}}_{3+1}} )$(认为这四者必须为实场\cite{boydNonlinearOptics2019}也没关系:它们对$t$的傅立叶变换所得到的$\pm \omega$单色子波是复共轭的,以至于正负频率的对应复子波求和后会消掉虚部,只剩下实部的余弦$\cos$实子波,因此在正/倒空间中的总/子场均是有物理意义的实场),属于在视觉上占主导的因变量,并用黑色(见 \bref{black})的 \textit{斜体 oblique}(见 \bref{oblique})表示;相对地,自变量用视觉和含义上均更次要的\textcolor{gray}{灰色}来表示,见 \bref{gray}。}复色场 
$\bar{E}^{\;\!\mathcolor{gray}{t}}_{\;\!\mathcolor{gray}{z}}, \bar{H}^{\;\!\mathcolor{gray}{t}}_{\;\!\mathcolor{gray}{z}}, \bar{D}^{\;\!\mathcolor{gray}{t}}_{\;\!\mathcolor{gray}{z}}, \bar{B}^{\;\!\mathcolor{gray}{t}}_{\;\!\mathcolor{gray}{z}}$\Footnote{由于傅立叶光学一般运行在平行平面间,约定下述表示相互等价:$\bar{E} \left( \mathcolor{gray}{\bar{r}}, \mathcolor{gray}{t} \right) = \bar{E}^{\;\!\mathcolor{gray}{t}}_{\;\!\mathcolor{gray}{\bar{r}}} = \bar{E}^{\;\!\mathcolor{gray}{t}}_{\;\!\mathcolor{gray}{z}} \left( \mathcolor{gray}{\bar{\rho}} \right)$,并因此经常省略面内自变量 $\mathcolor{gray}{\bar{\rho}}$,以只写作朝 $\textcolor{Maroon}{+\symup{z}}$ 轴传播距离 $\mathcolor{gray}{z}$ 的函数 $\bar{E}^{\;\!\mathcolor{gray}{t}}_{\;\!\mathcolor{gray}{z}} := \bar{E}^{\;\!\mathcolor{gray}{t}}_{\;\!\mathcolor{gray}{z}} \left( \mathcolor{gray}{\bar{\rho}} \right)$。—— 同样的规则也适用于其他场量(如 ${\rho}_{\;\!\textcolor{Maroon}{\text{f}}} \left( \mathcolor{gray}{\bar{r}}, \mathcolor{gray}{t} \right) \to {\rho}^{\;\!\mathcolor{gray}{t}}_{\;\!\textcolor{Maroon}{\text{f}}\mathcolor{gray}{z}}$),且适用于空间频率域,见。},满足微分形式\Footnote{尽管是微分形式,仍然处于(相对的)宏观层面:典型的光波长 $1$um 是原子特征尺寸 $1\text{\r{A}} = 0.1$nm 的 $10^4$ 倍,因此\bref{eq:Curl-EH,eq:Div-DB,eq:Div-em-f}涉及的所有物理量均是空间平均后的结果\cite{mackayElectromagneticAnisotropyBianisotropy2019};如果要引入(非)线性极/磁化强度/率随考虑区域尺度的缩放,则需要 \textcolor{Maroon}{Clausius-Mossotti equation} 或 \textcolor{Maroon}{Lorentz-Lorenz law} 的局域场修正\cite{boydNonlinearOptics2019}。}的麦氏方程组的 2 个旋度假设\Footnote{使用爱因斯坦求和约定(2 个相同指标相遇则主体求和,但 2 个 同侧角标不适用\cite{frankelGeometryPhysicsIntroduction2011}),定义了空域 3 维向量微分算子 $\mathcolor{gray}{\bar{\nabla}} := \mathcolor{gray}{\bar{\nabla}_{\bar{r}}} := \hat{\symup{e}}_i \mathcolor{gray}{\nabla^i}$,其中 $i = x,y,z$,且 $\hat{\symup{e}}_x,\hat{\symup{e}}_y,\hat{\symup{e}}_z$ 分别为 $\symup{x,y,z}$ 方向的单位定矢(“定矢$\hat{\symup{e}}$”也“直体”:见 \bref{upright})。对于旋度\&叉乘运算,还可写成轴矢量构成的反对称二阶张量的矩阵乘积:$\mathcolor{gray}{\bar{\nabla} \times} \bar{v} = \mathcolor{gray}{\bar{\nabla}^\times \cdot} \bar{v}$\cite{ossikovskiConstitutiveRelationsOptically2021},并且矩阵乘积可以省略 `$\cdot$' 而写为 $\mathcolor{gray}{\bar{\nabla}^\times} \bar{v} = \mathcolor{gray}{\bar{\nabla}^\times \cdot} \bar{v}$。此外,定义了 \textcolor{Maroon}{\text{Levi-Civita symbol}} $\epsilon^{\hphantom{ij}k}_{i\mathcolor{gray}{j}}$、空域偏导算符 $\mathcolor{gray}{\nabla^j} := \mathcolor{gray}{ \partial \mathcolor{black}{\underline{~~}} \big/ \partial \mathcolor{gray}{j} }$(其中,黑色的 $i,j,k$ 取遍 $\symup{x},\symup{y},\symup{z}$、灰色的 $\mathcolor{gray}{i},\mathcolor{gray}{j},\mathcolor{gray}{k}$ 取遍 $\mathcolor{gray}{x},\mathcolor{gray}{y},\mathcolor{gray}{z}$)、时域偏导算符 $\mathcolor{gray}{\nabla^t} \underline{~~} := \mathcolor{gray}{\partial \mathcolor{black}{\underline{~~}} \big/ \partial t}$。}:
\begin{subequations} \label{eq:Curl-EH}
	\abovedisplayskip=0pt
	%\belowdisplayskip=0pt
\begin{align}
	\textcolor{Maroon}{\text{Faraday's law of electromagnetic induction}}\text{:}&\hspace{1.2em} \mathcolor{gray}{\bar{\nabla} \times} \bar{E}^{\;\!\mathcolor{gray}{t}}_{\;\!\mathcolor{gray}{z}} + \mathcolor{gray}{\nabla^t} \bar{B}^{\;\!\mathcolor{gray}{t}}_{\;\!\mathcolor{gray}{z}} \hspace{-0.9em} &&= - \bar{K}^{\;\!\mathcolor{gray}{t}}_{\;\!\textcolor{Maroon}{\text{f}}\mathcolor{gray}{z}} \label{eq:Curl-EK} \\ 
	&\epsilon^{\hphantom{ij}k}_{i\mathcolor{gray}{j}} \mathcolor{gray}{\nabla^j} E^{\;\!\mathcolor{gray}{t}}_{\;\! k\mathcolor{gray}{z}} + \mathcolor{gray}{\nabla^t} B^{\;\!\mathcolor{gray}{t}}_{\;\! i\mathcolor{gray}{z}} \hspace{-0.9em} &&= - K^{\;\!\mathcolor{gray}{t}}_{\;\! \textcolor{Maroon}{\text{f}} i\mathcolor{gray}{z}}~, \label{eq:curl-EK} \\ 
	\textcolor{Maroon}{\text{Jefimenko's}} \to \textcolor{Maroon}{\text{Amp\`{e}re-Maxwell circuital law}}\text{:}&\hspace{1.2em} \mathcolor{gray}{\bar{\nabla} \times} \bar{H}^{\;\!\mathcolor{gray}{t}}_{\;\!\mathcolor{gray}{z}} - \mathcolor{gray}{\nabla^t} \bar{D}^{\;\!\mathcolor{gray}{t}}_{\;\!\mathcolor{gray}{z}} \hspace{-0.9em} &&= \bar{J}^{\;\!\mathcolor{gray}{t}}_{\;\!\textcolor{Maroon}{\text{f}}\mathcolor{gray}{z}} \label{eq:Curl-H} \\ 
	&\epsilon^{\hphantom{ij}k}_{i\mathcolor{gray}{j}} \mathcolor{gray}{\nabla^j} H^{\;\!\mathcolor{gray}{t}}_{\;\! k\mathcolor{gray}{z}} - \mathcolor{gray}{\nabla^t} D^{\;\!\mathcolor{gray}{t}}_{\;\! i\mathcolor{gray}{z}} \hspace{-0.9em} &&= J^{\;\!\mathcolor{gray}{t}}_{\;\! \textcolor{Maroon}{\text{f}} i\mathcolor{gray}{z}}~. \label{eq:curl-H}
\end{align}
\end{subequations}
以及 2 个散度假设\Footnote{对于散度\&点积,也可以使用“行向量·列向量”的矩阵乘法,如 $\mathcolor{gray}{\bar{\nabla}^\intercal} \bar{v} = \mathcolor{gray}{\bar{\nabla} \cdot} \bar{v}$。此外,高阶散度/点积(如 \bref{eq:P-b} 中的 $\mathcolor{gray}{\bar{\nabla}} \mathcolor{gray}{\bar{\nabla} \colon}\! \bar{\bar{Q}}^{\;\!\mathcolor{gray}{t}}_{\;\!\mathcolor{gray}{z}}$)原则上也可用 $\mathcolor{gray}{\bar{\nabla} \cdot} ( \mathcolor{gray}{\bar{\nabla} \cdot} \bar{\bar{Q}}^{\;\!\mathcolor{gray}{t}}_{\;\!\mathcolor{gray}{z}} ) \neq \mathcolor{gray}{( \bar{\nabla} \cdot \bar{\nabla} )} \bar{\bar{Q}}^{\;\!\mathcolor{gray}{t}}_{\;\!\mathcolor{gray}{z}} = \mathcolor{gray}{\bar{\nabla}^2} \bar{\bar{Q}}^{\;\!\mathcolor{gray}{t}}_{\;\!\mathcolor{gray}{z}} = \mathcolor{gray}{\nabla^2} \bar{\bar{Q}}^{\;\!\mathcolor{gray}{t}}_{\;\!\mathcolor{gray}{z}}$ 或 $\mathcolor{gray}{\left( \bar{\nabla} \bar{\nabla} \right)^\intercal} \! \bar{\bar{Q}}^{\;\!\mathcolor{gray}{t}}_{\;\!\mathcolor{gray}{z}} = \mathcolor{gray}{\bar{\nabla}} ( \mathcolor{gray}{\bar{\nabla}^\intercal} \bar{\bar{Q}}^{\;\!\mathcolor{gray}{t}}_{\;\!\mathcolor{gray}{z}} )^\intercal$ 表示,但对于后者中高阶张量$\mathcolor{gray}{\bar{\nabla}} \mathcolor{gray}{\bar{\nabla}}$的转置需要指定(哪)两个维度。此外,点积 $\cdot$ 也可视为对应元素积=哈达玛积 $\odot$ 后,再对所有元素求和。}:
\begin{subequations} \label{eq:Div-DB}
\begin{align}
	\textcolor{Maroon}{\text{Coulomb's}} \to \textcolor{Maroon}{\text{Gauss's law for electricity}}\text{:}&\hspace{0.5em} \mathcolor{gray}{\bar{\nabla} \cdot} \bar{D}^{\;\!\mathcolor{gray}{t}}_{\;\!\mathcolor{gray}{z}} \hspace{-3.8em} &&= {\rho}^{\;\!\mathcolor{gray}{t}}_{\;\!\textcolor{Maroon}{\text{f}}\mathcolor{gray}{z}} \label{eq:Div-D} \\ 
	&\hspace{0.5em} \mathcolor{gray}{\nabla^i} D^{\;\!\mathcolor{gray}{t}}_{\;\! i\mathcolor{gray}{z}} \hspace{-3.8em} &&= {\rho}^{\;\!\mathcolor{gray}{t}}_{\;\!\textcolor{Maroon}{\text{f}}\mathcolor{gray}{z}}~, \label{eq:div-D} \\
	\textcolor{Maroon}{\text{Biot-Savart}} \to \textcolor{Maroon}{\text{Gauss's law for magnetism}}\text{:}&\hspace{0.5em} \mathcolor{gray}{\bar{\nabla} \cdot} \bar{B}^{\;\!\mathcolor{gray}{t}}_{\;\!\mathcolor{gray}{z}} \hspace{-3.8em} &&= {\kappa}^{\;\!\mathcolor{gray}{t}}_{\;\!\textcolor{Maroon}{\text{f}}\mathcolor{gray}{z}} \label{eq:Div-Bk} \\
	&\hspace{0.5em} \mathcolor{gray}{\nabla^i} B^{\;\!\mathcolor{gray}{t}}_{\;\! i\mathcolor{gray}{z}} \hspace{-3.8em} &&= {\kappa}^{\;\!\mathcolor{gray}{t}}_{\;\!\textcolor{Maroon}{\text{f}}\mathcolor{gray}{z}}~. \label{eq:div-Bk}
\end{align}
\end{subequations}
其中,为数学形式对称(以方便引入相对论效应和检验其协变性\cite{lakhtakiaCovariancesInvariancesMaxwell1995,chen-zhuChenZhuxieUndergraduate_courses2024}),和物理上不排除可能存在的磁单极子,除自由电(荷/流)源${\rho}^{\;\!\mathcolor{gray}{t}}_{\;\!\textcolor{Maroon}{\text{f}}\mathcolor{gray}{z}}, \bar{J}^{\;\!\mathcolor{gray}{t}}_{\;\!\textcolor{Maroon}{\text{f}}\mathcolor{gray}{z}}$外,还添加了自由磁(荷/流)源${\kappa}^{\;\!\mathcolor{gray}{t}}_{\;\!\textcolor{Maroon}{\text{f}}\mathcolor{gray}{z}}, \bar{K}^{\;\!\mathcolor{gray}{t}}_{\;\!\textcolor{Maroon}{\text{f}}\mathcolor{gray}{z}}$\cite{lakhtakiaCovariancesInvariancesMaxwell1995}。这 4 个自由源(体密度)项,满足 2 个连续性假设\cite{mackayElectromagneticAnisotropyBianisotropy2019,lakhtakiaCovariancesInvariancesMaxwell1995,chen-zhuChenZhuxieUndergraduate_courses2024}:
\begin{subequations} \label{eq:Div-em-f}
	%\abovedisplayskip=0pt
	%\belowdisplayskip=0pt
\begin{align}
	\textcolor{Maroon}{\text{Continuity for free electric charge}}\text{:}&\hspace{0.5em} \mathcolor{gray}{\bar{\nabla} \cdot} \bar{J}^{\;\!\mathcolor{gray}{t}}_{\;\!\textcolor{Maroon}{\text{f}}\mathcolor{gray}{z}} + \mathcolor{gray}{\nabla^t} {\rho}^{\;\!\mathcolor{gray}{t}}_{\;\!\textcolor{Maroon}{\text{f}}\mathcolor{gray}{z}} \hspace{-5.3em} &&= 0 \label{eq:Div-e-f} \\ 
	&\hspace{0.5em} \mathcolor{gray}{\nabla^i} J^{\;\!\mathcolor{gray}{t}}_{\;\!\textcolor{Maroon}{\text{f}} i\mathcolor{gray}{z}} + \mathcolor{gray}{\nabla^t} {\rho}^{\;\!\mathcolor{gray}{t}}_{\;\!\textcolor{Maroon}{\text{f}}\mathcolor{gray}{z}} \hspace{-5.3em} &&= 0~, \label{eq:div-e-f} \\ 
	\textcolor{Maroon}{\text{Continuity for free magnetic charge}}\text{:}&\hspace{0.5em} \mathcolor{gray}{\bar{\nabla} \cdot} \bar{K}^{\;\!\mathcolor{gray}{t}}_{\;\!\textcolor{Maroon}{\text{f}}\mathcolor{gray}{z}} + \mathcolor{gray}{\nabla^t} {\kappa}^{\;\!\mathcolor{gray}{t}}_{\;\!\textcolor{Maroon}{\text{f}}\mathcolor{gray}{z}} \hspace{-5.3em} &&= 0 \label{eq:Div-m-f} \\
	&\hspace{0.5em} \mathcolor{gray}{\nabla^i} K^{\;\!\mathcolor{gray}{t}}_{\;\!\textcolor{Maroon}{\text{f}} i\mathcolor{gray}{z}} + \mathcolor{gray}{\nabla^t} {\kappa}^{\;\!\mathcolor{gray}{t}}_{\;\!\textcolor{Maroon}{\text{f}}\mathcolor{gray}{z}} \hspace{-5.3em} &&= 0~. \label{eq:div-m-f}
\end{align}
\end{subequations}
注意,对旋度 \bref{eq:Curl-EH} 两边取散度($\mathcolor{gray}{\bar{\nabla} \cdot}$),连续性 \bref{eq:Div-em-f} 可导出散度 \bref{eq:Div-DB},反之亦然\cite{lakhtakiaGenesisPostConstraint2004}。因此,2 条散度方程均不是必需的,可将其视为冗余\Footnote{并且不应简单地仅根据$\bar{D}^{\;\!\mathcolor{gray}{t}}_{\;\!\mathcolor{gray}{z}}, \bar{B}^{\;\!\mathcolor{gray}{t}}_{\;\!\mathcolor{gray}{z}}$的横向性,而将二者视为基本场\cite{quesadaPhotonPairsNonlinear2022,berryOpticalSingularitiesBianisotropic2005}。但从场能量体密度变化率$\bar{E}^{\;\!\mathcolor{gray}{t}}_{\;\!\mathcolor{gray}{z}} \mathbb{d}\bar{D}^{\;\!\mathcolor{gray}{t}}_{\;\!\mathcolor{gray}{z}} +  \bar{H}^{\;\!\mathcolor{gray}{t}}_{\;\!\mathcolor{gray}{z}} \mathbb{d}\bar{B}^{\;\!\mathcolor{gray}{t}}_{\;\!\mathcolor{gray}{z}}$中含有 2 个旋度\bref{eq:Curl-EH}中对$\bar{D}^{\;\!\mathcolor{gray}{t}}_{\;\!\mathcolor{gray}{z}}, \bar{B}^{\;\!\mathcolor{gray}{t}}_{\;\!\mathcolor{gray}{z}}$的微分\cite{chen-zhuChenZhuxieUndergraduate_courses2024}、方便引入适用于非线性量子光学的正确的哈密顿量\cite{quesadaPhotonPairsNonlinear2022},将$\bar{D}^{\;\!\mathcolor{gray}{t}}_{\;\!\mathcolor{gray}{z}}, \bar{B}^{\;\!\mathcolor{gray}{t}}_{\;\!\mathcolor{gray}{z}}$视为基本场也有一定道理?但如此一来,电非线性 $\bar{E}^{\;\!\mathcolor{gray}{t}}_{\;\!\mathcolor{gray}{z}} \left( \bar{D}^{\;\!\mathcolor{gray}{t}}_{\;\!\mathcolor{gray}{z}} \right) \slashed{\propto} \bar{D}^{\;\!\mathcolor{gray}{t}}_{\;\!\mathcolor{gray}{z}}$ 的显式表达式有悖常理。}。

\clearpage
%\XGap{-10em}
\vspace*{-8.5em}

\marginLeft[-2.5em]{ssec:EBpJ}\subsection{$\bar{E},\bar{B}$ 基本场、${\rho}, \bar{J}$ 总源连续}\label{ssec:EBpJ}

现代电磁学/电动力学追根溯源地\Footnote{从 $\bar{E}^{\;\!\mathcolor{gray}{t}}_{\;\!\mathcolor{gray}{z}}, \bar{B}^{\;\!\mathcolor{gray}{t}}_{\;\!\mathcolor{gray}{z}}$ 出发又回到 $\bar{E}^{\;\!\mathcolor{gray}{t}}_{\;\!\mathcolor{gray}{z}}, \bar{B}^{\;\!\mathcolor{gray}{t}}_{\;\!\mathcolor{gray}{z}}$,先升格又降格 $\bar{E}^{\;\!\mathcolor{gray}{t}}_{\;\!\mathcolor{gray}{z}}, \bar{B}^{\;\!\mathcolor{gray}{t}}_{\;\!\mathcolor{gray}{z}} \to \bar{D}^{\;\!\mathcolor{gray}{t}}_{\;\!\mathcolor{gray}{z}}, \bar{H}^{\;\!\mathcolor{gray}{t}}_{\;\!\mathcolor{gray}{z}} \to \bar{E}^{\;\!\mathcolor{gray}{t}}_{\;\!\mathcolor{gray}{z}}, \bar{B}^{\;\!\mathcolor{gray}{t}}_{\;\!\mathcolor{gray}{z}}$,绕了一个大弯。},将 $\bar{E}^{\;\!\mathcolor{gray}{t}}_{\;\!\mathcolor{gray}{z}}, \bar{B}^{\;\!\mathcolor{gray}{t}}_{\;\!\mathcolor{gray}{z}}$\Footnote{$\bar{D}^{\;\!\mathcolor{gray}{t}}_{\;\!\mathcolor{gray}{z}},\bar{H}^{\;\!\mathcolor{gray}{t}}_{\;\!\mathcolor{gray}{z}}$作为总/裸场$\bar{E}^{\;\!\mathcolor{gray}{t}}_{\;\!\mathcolor{gray}{z}},\bar{B}^{\;\!\mathcolor{gray}{t}}_{\;\!\mathcolor{gray}{z}}$分别加上(通过\bref{eq:Div-D})或减去(通过\bref{eq:Curl-H})束缚源所响应产生的(极化或磁化)场后的辅助场,分别代表自由/裸源${\rho}^{\;\!\mathcolor{gray}{t}}_{\;\!\textcolor{Maroon}{\text{f}}\mathcolor{gray}{z}}, \bar{J}^{\;\!\mathcolor{gray}{t}}_{\;\!\textcolor{Maroon}{\text{f}}\mathcolor{gray}{z}}$所对应的“净留/残余场”。$\bar{E}^{\;\!\mathcolor{gray}{t}}_{\;\!\mathcolor{gray}{z}},\bar{B}^{\;\!\mathcolor{gray}{t}}_{\;\!\mathcolor{gray}{z}}$是基本场,出于下述原因:其起源是微观且明确的、可直接测量、包含了所有的束缚和自由源产生的场、洛伦兹力公式(普适至相对论情形)、\bref{eq:Curl-EK,eq:Div-Bk}的无源特性及其导出的标矢势和四维势矢量、四维二阶电磁场张量\cite{chen-zhuChenZhuxieUndergraduate_courses2024}、无矛盾地推导和适用 Post 约束\cite{lakhtakiaGenesisPostConstraint2004};同时也方便原子物理中对拉莫尔进动、史特恩—盖拉赫实验、塞曼效应的表述\cite{chen-zhuChenZhuxieUndergraduate_courses2024},以及量子电动力学中对磁光材料的拉氏量的处理\cite{nelsonLagrangianTreatmentMagnetic1994}。——但是,选择$\bar{B}^{\;\!\mathcolor{gray}{t}}_{\;\!\mathcolor{gray}{z}}$而不是$\bar{H}^{\;\!\mathcolor{gray}{t}}_{\;\!\mathcolor{gray}{z}}$将不方便(准静)磁学,如铁磁性物质的磁滞回线的表述\cite{hillionBasicFieldElectromagnetism1996}。但此时将外场写作$\bar{B}_0$而不是$\bar{H}$似乎即可解决:正如电光效应的外加准静电场$\bar{E}_0$一样。}而不是 $\bar{E}^{\;\!\mathcolor{gray}{t}}_{\;\!\mathcolor{gray}{z}}, \bar{H}^{\;\!\mathcolor{gray}{t}}_{\;\!\mathcolor{gray}{z}}$\Footnote{相对论或手性的情形下,将$\bar{E}^{\;\!\mathcolor{gray}{t}}_{\;\!\mathcolor{gray}{z}}, \bar{H}^{\;\!\mathcolor{gray}{t}}_{\;\!\mathcolor{gray}{z}}$而不是$\bar{E}^{\;\!\mathcolor{gray}{t}}_{\;\!\mathcolor{gray}{z}}, \bar{B}^{\;\!\mathcolor{gray}{t}}_{\;\!\mathcolor{gray}{z}}$作为本构关系的基本场,可能更有优势\cite{hillionBasicFieldElectromagnetism1996,lakhtakiaGenesisPostConstraint2004};此外,对于边界条件,进可采用$\bar{E}^{\;\!\mathcolor{gray}{t}}_{\;\!\mathcolor{gray}{z}},\bar{H}^{\;\!\mathcolor{gray}{t}}_{\;\!\mathcolor{gray}{z}}$切向连续边界条件\cite{mcleodVectorFourierOptics2014},退可四维时空傅立叶变换\cite{chenWavevectorspaceMethodWave1993,chenWavePropagationExciton1993,nelsonDerivingTransmissionReflection1995}。还允许不关注微观起源\cite{eimerlQuantumElectrodynamicsOptical1988,nelsonMechanismsDispersionCrystalline1989,boydNonlinearOptics2019,loudonPropagationElectromagneticEnergy1997,laxLinearNonlinearElectrodynamics1971},电场的非局域一阶波矢色散也可直接放进本构关系而无需额外处理\cite{berryOpticalSingularitiesBianisotropic2005}。但可能没法处理材料表面积累电荷(如铁电体的$\textcolor{Maroon}{+\symup{c}}$ 面)、表面电流\cite{chen-zhuChenZhuxieUndergraduate_courses2024}、表面光学活性\cite{nelsonMechanismsDispersionCrystalline1989},尽管没有使用到任何散度方程/横向约束,已经很有吸引力了\cite{eimerlQuantumElectrodynamicsOptical1988,berryOpticalSingularitiesBirefringent2003,berryOpticalSingularitiesBianisotropic2005}。}视为基本场\cite{hillionBasicFieldElectromagnetism1996,lakhtakiaGenesisPostConstraint2004,nelsonDerivingTransmissionReflection1995,hehlGentleIntroductionFoundations2000}。因此,2 个旋度方程 \bref{eq:Curl-EK}、\bref{eq:Curl-H} 应等效地写做:
\begin{subequations} \label{eq:Curl-EB}
	%\abovedisplayskip=0pt
	%\belowdisplayskip=0pt
\begin{align}
	\textcolor{Maroon}{\text{法拉第电磁感应定律}}\text{:}~~~~~~~ \mathcolor{gray}{\bar{\nabla} \times} \bar{E}^{\;\!\mathcolor{gray}{t}}_{\;\!\mathcolor{gray}{z}} + \mathcolor{gray}{\nabla^t} \bar{B}^{\;\!\mathcolor{gray}{t}} &= \bar{0} \label{eq:Curl-E} \\ 
	\epsilon^{\hphantom{ij}k}_{i\mathcolor{gray}{j}} \mathcolor{gray}{\nabla^j} E^{\;\!\mathcolor{gray}{t}}_{\;\! k\mathcolor{gray}{z}} + \mathcolor{gray}{\nabla^t} B^{\;\!\mathcolor{gray}{t}}_{\;\! i\mathcolor{gray}{z}} &= 0~, \label{eq:curl-E} \\
	\textcolor{Maroon}{\text{安倍环路定律}}\text{:}~~~~~~~~~ {\symup{\mu}}_0^{-1} \mathcolor{gray}{\bar{\nabla} \times} \bar{B}^{\;\!\mathcolor{gray}{t}}_{\;\!\mathcolor{gray}{z}} - {\symup{\varepsilon}}_0 \mathcolor{gray}{\nabla^t} \bar{E}^{\;\!\mathcolor{gray}{t}}_{\;\!\mathcolor{gray}{z}} &= \bar{J}^{\;\!\mathcolor{gray}{t}}_{\;\!\mathcolor{gray}{z}} \label{eq:Curl-B} \\ 
	{\symup{\mu}}_0^{-1} \epsilon^{\hphantom{ij}k}_{i\mathcolor{gray}{j}} \mathcolor{gray}{\nabla^j} B^{\;\!\mathcolor{gray}{t}}_{\;\! k\mathcolor{gray}{z}} - {\symup{\varepsilon}}_0 \mathcolor{gray}{\nabla^t} E^{\;\!\mathcolor{gray}{t}}_{\;\! i\mathcolor{gray}{z}} &= J^{\;\!\mathcolor{gray}{t}}_{\;\! i\mathcolor{gray}{z}}~. \label{eq:curl-B}
\end{align}
\end{subequations}
相应地,2 个散度方程 \bref{eq:Div-D}、\bref{eq:Curl-B} 返璞归真为:
\begin{subequations} \label{eq:Div-EB}
\begin{align}
	\textcolor{Maroon}{\text{高斯定律(电)}}\text{:}~~~~~~~ {\symup{\varepsilon}}_0 \mathcolor{gray}{\bar{\nabla} \cdot} \bar{E}^{\;\!\mathcolor{gray}{t}}_{\;\!\mathcolor{gray}{z}} &= {\rho}^{\;\!\mathcolor{gray}{t}}_{\;\!\mathcolor{gray}{z}} \label{eq:Div-E} \\ 
	{\symup{\varepsilon}}_0 \mathcolor{gray}{\nabla^i} E^{\;\!\mathcolor{gray}{t}}_{\;\! i\mathcolor{gray}{z}} &= {\rho}^{\;\!\mathcolor{gray}{t}}_{\;\!\mathcolor{gray}{z}}~, \label{eq:div-E} \\ 
	\textcolor{Maroon}{\text{高斯定律(磁)}}\text{:}~~~~~~~~~~ \mathcolor{gray}{\bar{\nabla} \cdot} \bar{B}^{\;\!\mathcolor{gray}{t}}_{\;\!\mathcolor{gray}{z}} &= 0 \label{eq:Div-B} \\ 
	\mathcolor{gray}{\nabla^i} B^{\;\!\mathcolor{gray}{t}}_{\;\! i\mathcolor{gray}{z}} &= 0~. \label{eq:div-B} 
\end{align}
\end{subequations}
同理,总电荷 ${\rho}^{\;\!\mathcolor{gray}{t}}_{\;\!\mathcolor{gray}{z}}$ 守恒方程从自由电荷 ${\rho}^{\;\!\mathcolor{gray}{t}}_{\;\!\textcolor{Maroon}{\text{f}}\mathcolor{gray}{z}}$ 所满足的连续性 \bref{eq:Div-e-f} 升级为:
\begin{align}
	%\abovedisplayskip=0pt
	%\belowdisplayskip=0pt
	\textcolor{Maroon}{\text{电荷守恒定律}}\text{:}~~~~~~~ \mathcolor{gray}{\bar{\nabla} \cdot} \bar{J}^{\;\!\mathcolor{gray}{t}}_{\;\!\mathcolor{gray}{z}} + \mathcolor{gray}{\nabla^t} {\rho}^{\;\!\mathcolor{gray}{t}}_{\;\!\mathcolor{gray}{z}} &= 0 \label{eq:Div-e} \\ 
	\mathcolor{gray}{\nabla^i} J^{\;\!\mathcolor{gray}{t}}_{\;\! i\mathcolor{gray}{z}} + \mathcolor{gray}{\nabla^t} {\rho}^{\;\!\mathcolor{gray}{t}}_{\;\!\mathcolor{gray}{z}} &= 0~. \label{eq:div-e} 
\end{align}
该 \bref{eq:Div-e} 即电荷 $Q$ 守恒,与磁流 $\varPhi$ 守恒(即法拉第电磁感应定律 \bref{eq:Curl-E}、安倍-麦克斯韦环路定律 \bref{eq:Curl-B})一起\cite{hehlSpacetimeMetricLocal2006},经受住了大量的实验考验\cite{hehlGentleIntroductionFoundations2000},因此二者都是基本的物理定律\cite{hehlRecentDevelopmentsPremetric2006,hehlFOUNDATIONSCLASSICALELECTRODYNAMICS}。

下 \bref{fig:EHDB} 中,2 条红边(连接了顶点 $D,H$、$E,B$)、2 条蓝边(连接了顶点 $E,H$、$D,B$)上,对应颜色(红/橘色、蓝色)的文字,分别给出了选择 4 种电磁场组合对 $\{ E,H$、$E,B$、$D,B$、$D,H \}$ 中的每一对作为基本场的理由。

\begin{figure}[htbp!]
	\centering
	\includegraphics[width=0.9\textwidth]{D:/C2D/Desktop/article_fig/phd_thesis_fig/chapter-02/本构关系与边界条件-single-page.pdf}
	\backcaption{-0.6em}{$E,B$ 作为基本场的理由(大量的红/橘色文字理由所在的边),远比其他 3 种组合(2 条蓝色文字边 $E,H$ 和 $D,B$,以及 1 条少量红/橘色文字对边 $D,H$)更丰富。}{fig:EHDB}
\end{figure}

\vspace*{-5.5em}

\marginLeft[-1.2em]{ssec:PMQN}\subsection{$\bar{P},\bar{M}$ 极磁化、$\bar{\bar{Q}},\bar{\bar{N}}$ 多极理论}\label{ssec:PMQN}

在高斯定律(电) \bref{eq:Div-E} 和电荷守恒定律 \bref{eq:Div-e} 中,总电荷源 ${\rho}^{\;\!\mathcolor{gray}{t}}_{\;\!\mathcolor{gray}{z}}$ 分为自由电荷源 ${\rho}^{\;\!\mathcolor{gray}{t}}_{\;\!\textcolor{Maroon}{\text{f}}\mathcolor{gray}{z}}$ 和束缚电荷源 ${\rho}^{\;\!\mathcolor{gray}{t}}_{\;\!\textcolor{Maroon}{\text{b}}\mathcolor{gray}{z}}$\cite{langeMultipoleTheoryHehl2015,raabMultipoleTheoryElectromagnetism2004},即:
\begin{align} \label{eq:p=f+b}
	%\abovedisplayskip=0pt
	%\belowdisplayskip=0pt
	{\rho}^{\;\!\mathcolor{gray}{t}}_{\;\!\mathcolor{gray}{z}} = {\rho}^{\;\!\mathcolor{gray}{t}}_{\;\!\textcolor{Maroon}{\text{f}}\mathcolor{gray}{z}} + {\rho}^{\;\!\mathcolor{gray}{t}}_{\;\!\textcolor{Maroon}{\text{b}}\mathcolor{gray}{z}}~,
\end{align}
其中,对动态电荷源分布所产生的标势场 $\phi^{\;\!\mathcolor{gray}{t}}_{\;\!\mathcolor{gray}{z}}$,在场点 $\mathcolor{gray}{\bar{r}}$ 邻域(无须在源分布体系的平衡态附近),进行三元泰勒展开,得束缚电荷体密度 ${\rho}^{\;\!\mathcolor{gray}{t}}_{\;\!\textcolor{Maroon}{\text{b}}\mathcolor{gray}{z}}$ 的表达式\cite{raabMultipoleTheoryElectromagnetism2004,delangeTranslationalInvariancePost2012,chen-zhuChenZhuxieUndergraduate_courses2024}:
\begin{subequations}
\begin{align}
	{\rho}^{\;\!\mathcolor{gray}{t}}_{\;\!\textcolor{Maroon}{\text{b}}\mathcolor{gray}{z}} &= - \mathcolor{gray}{\bar{\nabla} \cdot} \left( \bar{P}^{\;\!\mathcolor{gray}{t}}_{\;\!\mathcolor{gray}{z}} - \frac{1}{2!} \mathcolor{gray}{\bar{\nabla} \cdot} \bar{\bar{Q}}^{\;\!\mathcolor{gray}{t}}_{\;\!\mathcolor{gray}{z}} + \frac{1}{3!} \mathcolor{gray}{\bar{\nabla}} \mathcolor{gray}{\bar{\nabla} \colon}\! \bar{\bar{\bar{O}}}^{\;\!\mathcolor{gray}{t}}_{\;\!\mathcolor{gray}{z}} - \cdots \right) \label{eq:P-b} \\
	&= - \mathcolor{gray}{\nabla^i} \left( P^{\;\!\mathcolor{gray}{t}}_{\;\! i\mathcolor{gray}{z}} - \frac{1}{2!} \mathcolor{gray}{\nabla^j} Q^{\;\!\mathcolor{gray}{t}}_{\;\! ij\mathcolor{gray}{z}} + \frac{1}{3!} \mathcolor{gray}{\nabla^j} \mathcolor{gray}{\nabla^k} O^{\;\!\mathcolor{gray}{t}}_{\;\! ijk\mathcolor{gray}{z}} - \cdots \right)~. \label{eq:p-b}
\end{align}
\end{subequations}

同样,在安倍环路定律 \bref{eq:Curl-B} 和 电荷守恒定律 \bref{eq:Div-e} 中,总电流源 $\bar{J}^{\;\!\mathcolor{gray}{t}}_{\;\!\mathcolor{gray}{z}}$ 内包含了自由项 $\bar{J}^{\;\!\mathcolor{gray}{t}}_{\;\!\textcolor{Maroon}{\text{f}}\mathcolor{gray}{z}}$ 和束缚项 $\bar{J}^{\;\!\mathcolor{gray}{t}}_{\;\!\textcolor{Maroon}{\text{b}}\mathcolor{gray}{z}}$\cite{langeMultipoleTheoryHehl2015,raabMultipoleTheoryElectromagnetism2004},而束缚源 $\bar{J}^{\;\!\mathcolor{gray}{t}}_{\;\!\textcolor{Maroon}{\text{b}}\mathcolor{gray}{z}}$ 又由电源 $\bar{J}^{\;\!\mathcolor{gray}{t}}_{\;\!\textcolor{Maroon}{\text{e}}\mathcolor{gray}{z}}$ 和磁源 $\bar{J}^{\;\!\mathcolor{gray}{t}}_{\;\!\textcolor{Maroon}{\text{m}}\mathcolor{gray}{z}}$ 构成,即:
\begin{subequations} \label{eq:j-fbem}
	\abovedisplayskip=-5pt
	\belowdisplayskip=10pt
\begin{align}
	\bar{J}^{\;\!\mathcolor{gray}{t}}_{\;\!\mathcolor{gray}{z}} &= \bar{J}^{\;\!\mathcolor{gray}{t}}_{\;\!\textcolor{Maroon}{\text{f}}\mathcolor{gray}{z}} + \bar{J}^{\;\!\mathcolor{gray}{t}}_{\;\!\textcolor{Maroon}{\text{b}}\mathcolor{gray}{z}}~, \label{eq:j=f+b} \\ \bar{J}^{\;\!\mathcolor{gray}{t}}_{\;\!\textcolor{Maroon}{\text{b}}\mathcolor{gray}{z}} &= \bar{J}^{\;\!\mathcolor{gray}{t}}_{\;\!\textcolor{Maroon}{\text{e}}\mathcolor{gray}{z}} + \bar{J}^{\;\!\mathcolor{gray}{t}}_{\;\!\textcolor{Maroon}{\text{m}}\mathcolor{gray}{z}}~. \label{eq:b=e+m}
\end{align}
\end{subequations}
其中,考虑有限区域内连续分布的时变电流源所产生的矢势场 $\bar{A}^{\;\!\mathcolor{gray}{t}}_{\;\!\mathcolor{gray}{z}}$,等价于源全集中于其荷心时,在原点处的永久/自发多极矩和(受到外场如 $\bar{E}^{\;\!\mathcolor{gray}{t}}_{\;\!\mathcolor{gray}{z}}$ 和/或 $\bar{B}^{\;\!\mathcolor{gray}{t}}_{\;\!\mathcolor{gray}{z}}$、应力场 $\bar{T}^{\;\!\mathcolor{gray}{t}}_{\;\!\mathcolor{gray}{z}}$ 等后反作用出的)感应多极矩(的各项多极展开之和)之和,朝心外某一场点激发的矢势场\cite{raabMultipoleTheoryElectromagnetism2004,delangeTranslationalInvariancePost2012,chen-zhuChenZhuxieUndergraduate_courses2024},可分别得到束缚电源 $\bar{J}^{\;\!\mathcolor{gray}{t}}_{\;\!\textcolor{Maroon}{\text{e}}\mathcolor{gray}{z}}$ 和磁源 $\bar{J}^{\;\!\mathcolor{gray}{t}}_{\;\!\textcolor{Maroon}{\text{m}}\mathcolor{gray}{z}}$ 体密度:
\begin{subequations} \label{eq:J-em}
\begin{align}
	\bar{J}^{\;\!\mathcolor{gray}{t}}_{\;\!\textcolor{Maroon}{\text{e}}\mathcolor{gray}{z}} &= \mathcolor{gray}{\nabla^t} \left( \bar{P}^{\;\!\mathcolor{gray}{t}}_{\;\!\mathcolor{gray}{z}} - \frac{1}{2!} \mathcolor{gray}{\bar{\nabla} \cdot} \bar{\bar{Q}}^{\;\!\mathcolor{gray}{t}}_{\;\!\mathcolor{gray}{z}} + \frac{1}{3!} \mathcolor{gray}{\bar{\nabla}} \mathcolor{gray}{\bar{\nabla} \colon}\! \bar{\bar{\bar{O}}}^{\;\!\mathcolor{gray}{t}}_{\;\!\mathcolor{gray}{z}} - \cdots \right) \label{eq:J-e} \\ 
	&= \mathcolor{gray}{\nabla^t} \left( P^{\;\!\mathcolor{gray}{t}}_{\;\! i\mathcolor{gray}{z}} - \frac{1}{2!} \mathcolor{gray}{\nabla^j} Q^{\;\!\mathcolor{gray}{t}}_{\;\! ij\mathcolor{gray}{z}} + \frac{1}{3!} \mathcolor{gray}{\nabla^j} \mathcolor{gray}{\nabla^k} O^{\;\!\mathcolor{gray}{t}}_{\;\! ijk\mathcolor{gray}{z}} - \cdots \right) \hat{\symup{e}}^i~, \label{eq:j-e} \\ 
	\bar{J}^{\;\!\mathcolor{gray}{t}}_{\;\!\textcolor{Maroon}{\text{m}}\mathcolor{gray}{z}} &= \mathcolor{gray}{\bar{\nabla} \times} \left( \bar{M}^{\;\!\mathcolor{gray}{t}}_{\;\!\mathcolor{gray}{z}} - \frac{1}{2!} \mathcolor{gray}{\bar{\nabla} \cdot} \bar{\bar{N}}^{\;\!\mathcolor{gray}{t}}_{\;\!\mathcolor{gray}{z}} + \cdots \right) \label{eq:J-m} \\ 
	&= \epsilon^{\hphantom{ij}k}_{i\mathcolor{gray}{j}} \mathcolor{gray}{\nabla^j} \left( M^{\;\!\mathcolor{gray}{t}}_{\;\! k\mathcolor{gray}{z}} - \frac{1}{2!} \mathcolor{gray}{\nabla^l} N^{\;\!\mathcolor{gray}{t}}_{\;\! kl\mathcolor{gray}{z}} + \cdots \right) \hat{\symup{e}}^i~. \label{eq:j-m}
\end{align}
\end{subequations}
其中,$\bar{P}^{\;\!\mathcolor{gray}{t}}_{\;\!\mathcolor{gray}{z}}, \bar{M}^{\;\!\mathcolor{gray}{t}}_{\;\!\mathcolor{gray}{z}}$ 为电/磁偶极矩,$\bar{\bar{Q}}^{\;\!\mathcolor{gray}{t}}_{\;\!\mathcolor{gray}{z}}, \bar{\bar{N}}^{\;\!\mathcolor{gray}{t}}_{\;\!\mathcolor{gray}{z}}$ 为电/磁四极矩、$\bar{\bar{\bar{O}}}^{\;\!\mathcolor{gray}{t}}_{\;\!\mathcolor{gray}{z}}$ 为电八极矩。

然而,在强度上,多极矩的正确分组应是\cite{grahamMultipoleSolutionMacroscopic2000}\Footnote{从同阶磁矩弱于电矩,以及电子受到的电场力是洛伦兹力的c$\big/v$倍的角度,物质 和 CCD(电荷耦合器件)对电场的响应比对同等量级的磁场的响应大(至少对于高频电磁场=光的相互作用)。然而,电四/磁偶极矩的强度和其造成的影响,不一定比电偶极矩的小\cite{raabMultipoleTheoryElectromagnetism2004},包括其表面效应、响应的线性和非线性行为\cite{bethuneOpticalQuadrupoleSumfrequency1976}。}:
\begin{subequations}
	\abovedisplayskip=7pt
	\belowdisplayskip=8pt
\begin{align}
	\bar{P}^{\;\!\mathcolor{gray}{t}}_{\;\!\mathcolor{gray}{z}} \hspace{1em} &\ll \hspace{1em} \bar{\bar{Q}}^{\;\!\mathcolor{gray}{t}}_{\;\!\mathcolor{gray}{z}}, \bar{M}^{\;\!\mathcolor{gray}{t}}_{\;\!\mathcolor{gray}{z}} \hspace{-2.5em} &&\ll \hspace{1em} \bar{\bar{\bar{O}}}^{\;\!\mathcolor{gray}{t}}_{\;\!\mathcolor{gray}{z}}, \bar{\bar{N}}^{\;\!\mathcolor{gray}{t}}_{\;\!\mathcolor{gray}{z}} \label{eq:PQMON} \\ 
	\textcolor{Maroon}{\text{电偶极矩}} \hspace{1em} &\ll \hspace{1em} \textcolor{Maroon}{\text{电偶、磁四极矩}} \hspace{-2.5em} &&\ll \hspace{1em} \textcolor{Maroon}{\text{电八、磁四极矩}} ~. 
\end{align}
\end{subequations}
相应地将束缚电流源 \bref{eq:b=e+m} 展开并整理为:
\begin{subequations}
\begin{align}
\bar{J}^{\;\!\mathcolor{gray}{t}}_{\;\!\textcolor{Maroon}{\text{b}}\mathcolor{gray}{z}} = \mathcolor{gray}{\nabla^t} \bar{P}^{\;\!\mathcolor{gray}{t}}_{\;\!\mathcolor{gray}{z}} &- \left( \frac{1}{2!} \mathcolor{gray}{\bar{\nabla} \cdot} \mathcolor{gray}{\nabla^t} \bar{\bar{Q}}^{\;\!\mathcolor{gray}{t}}_{\;\!\mathcolor{gray}{z}} - \mathcolor{gray}{\bar{\nabla} \times} \bar{M}^{\;\!\mathcolor{gray}{t}}_{\;\!\mathcolor{gray}{z}} \right) \label{eq:J-b} \\ &+ \left[ \frac{1}{3!} \mathcolor{gray}{\bar{\nabla}} \mathcolor{gray}{\bar{\nabla} \colon}\! \mathcolor{gray}{\nabla^t} \bar{\bar{\bar{O}}}^{\;\!\mathcolor{gray}{t}}_{\;\!\mathcolor{gray}{z}} - \frac{1}{2!} \mathcolor{gray}{\bar{\nabla} \times} \left( \mathcolor{gray}{\bar{\nabla} \cdot}  \bar{\bar{N}}^{\;\!\mathcolor{gray}{t}}_{\;\!\mathcolor{gray}{z}} \right) \right] - \cdots~, \\
J^{\;\!\mathcolor{gray}{t}}_{\;\!\textcolor{Maroon}{\text{b}} i \mathcolor{gray}{z}} = \mathcolor{gray}{\nabla^t} P^{\;\!\mathcolor{gray}{t}}_{\;\! i\mathcolor{gray}{z}} &- \mathcolor{gray}{\nabla^j} \left( \frac{1}{2!} \mathcolor{gray}{\nabla^t} Q^{\;\!\mathcolor{gray}{t}}_{\;\! ij\mathcolor{gray}{z}} - \epsilon^{\hphantom{ij}k}_{i\mathcolor{gray}{j}} M^{\;\!\mathcolor{gray}{t}}_{\;\! k\mathcolor{gray}{z}} \right) \label{eq:j-b} \\ &+ \mathcolor{gray}{\nabla^j} \mathcolor{gray}{\nabla^l} \left[ \frac{1}{3!}  \mathcolor{gray}{\nabla^t} O^{\;\!\mathcolor{gray}{t}}_{\;\! ijl\mathcolor{gray}{z}} - \frac{1}{2!} \epsilon^{\hphantom{ij}k}_{i\mathcolor{gray}{j}} N^{\;\!\mathcolor{gray}{t}}_{\;\! kl\mathcolor{gray}{z}} \right] - \cdots~.
\end{align}
\end{subequations}

\marginLeft[-2.5em]{ssec:DH}\subsection{$\bar{D},\bar{H}$ 辅助场、$\bar{D},\bar{H}$ 非唯一性}\label{ssec:DHEB-boundary}

将束缚电荷源 \bref{eq:P-b}、电流源 \bref{eq:J-b} 分别代入总电荷源 \bref{eq:p=f+b}、电流源 \bref{eq:j=f+b},所得结果再代入基本场 $\bar{E}^{\;\!\mathcolor{gray}{t}}_{\;\!\mathcolor{gray}{z}}$ 的散度 \bref{eq:Div-E}、$\bar{B}^{\;\!\mathcolor{gray}{t}}_{\;\!\mathcolor{gray}{z}}$ 的旋度 \bref{eq:Curl-B},移项并对比 辅助场 $\bar{D}^{\;\!\mathcolor{gray}{t}}_{\;\!\mathcolor{gray}{z}}$ 的散度 \bref{eq:Div-D}、$\bar{H}^{\;\!\mathcolor{gray}{t}}_{\;\!\mathcolor{gray}{z}}$ 的旋度 \bref{eq:Curl-H},得辅助场 $\bar{D}^{\;\!\mathcolor{gray}{t}}_{\;\!\mathcolor{gray}{z}}, \bar{H}^{\;\!\mathcolor{gray}{t}}_{\;\!\mathcolor{gray}{z}}$ 关于基本场 $\bar{E}^{\;\!\mathcolor{gray}{t}}_{\;\!\mathcolor{gray}{z}}, \bar{B}^{\;\!\mathcolor{gray}{t}}_{\;\!\mathcolor{gray}{z}}$ 和源分布 $\bar{P}^{\;\!\mathcolor{gray}{t}}_{\;\!\mathcolor{gray}{z}},\bar{\bar{Q}}^{\;\!\mathcolor{gray}{t}}_{\;\!\mathcolor{gray}{z}},\bar{\bar{\bar{O}}}^{\;\!\mathcolor{gray}{t}}_{\;\!\mathcolor{gray}{z}} ; \bar{M}^{\;\!\mathcolor{gray}{t}}_{\;\!\mathcolor{gray}{z}}, \bar{\bar{N}}^{\;\!\mathcolor{gray}{t}}_{\;\!\mathcolor{gray}{z}}$ 的函数\cite{OriginDependenceMaterial,langeCompletionMultipoleTheory2003,raabTransformedMultipoleTheory2005,grahamMultipoleSolutionMacroscopic2000}:
\begin{subequations}
\begin{align}
	\bar{D}^{\;\!\mathcolor{gray}{t}}_{\;\!\mathcolor{gray}{z}} &= {\symup{\varepsilon}}_0 \bar{E}^{\;\!\mathcolor{gray}{t}}_{\;\!\mathcolor{gray}{z}} + \bar{P}^{\;\!\mathcolor{gray}{t}}_{\;\!\mathcolor{gray}{z}} - \frac{1}{2!} \mathcolor{gray}{\bar{\nabla} \cdot} \bar{\bar{Q}}^{\;\!\mathcolor{gray}{t}}_{\;\!\mathcolor{gray}{z}} + \frac{1}{3!} \mathcolor{gray}{\bar{\nabla}} \mathcolor{gray}{\bar{\nabla} \colon}\! \bar{\bar{\bar{O}}}^{\;\!\mathcolor{gray}{t}}_{\;\!\mathcolor{gray}{z}} - \cdots \label{eq:D} \\
	&= \left( {\symup{\varepsilon}}_0 E^{\;\!\mathcolor{gray}{t}}_{\;\! i\mathcolor{gray}{z}} + P^{\;\!\mathcolor{gray}{t}}_{\;\! i\mathcolor{gray}{z}} - \frac{1}{2!} \mathcolor{gray}{\nabla^j} Q^{\;\!\mathcolor{gray}{t}}_{\;\! ij\mathcolor{gray}{z}} + \frac{1}{3!} \mathcolor{gray}{\nabla^j} \mathcolor{gray}{\nabla^k} O^{\;\!\mathcolor{gray}{t}}_{\;\! ijk\mathcolor{gray}{z}} - \cdots \right) \hat{\symup{e}}^i~, \label{eq:d-pre} \\
	\bar{H}^{\;\!\mathcolor{gray}{t}}_{\;\!\mathcolor{gray}{z}} &= {\symup{\mu}}_0^{-1} \bar{B}^{\;\!\mathcolor{gray}{t}}_{\;\!\mathcolor{gray}{z}} - \bar{M}^{\;\!\mathcolor{gray}{t}}_{\;\!\mathcolor{gray}{z}} + \frac{1}{2!} \mathcolor{gray}{\bar{\nabla} \cdot} \bar{\bar{N}}^{\;\!\mathcolor{gray}{t}}_{\;\!\mathcolor{gray}{z}} - \cdots \label{eq:H} \\
	&= \left( {\symup{\mu}}_0^{-1} B^{\;\!\mathcolor{gray}{t}}_{\;\! i\mathcolor{gray}{z}} - M^{\;\!\mathcolor{gray}{t}}_{\;\! i\mathcolor{gray}{z}} + \frac{1}{2!} \mathcolor{gray}{\nabla^j} N^{\;\!\mathcolor{gray}{t}}_{\;\! ij\mathcolor{gray}{z}} - \cdots \right) \hat{\symup{e}}^i~. \label{eq:h-pre}
\end{align}
\end{subequations}
注意,此时组成 辅助场 $\bar{D}^{\;\!\mathcolor{gray}{t}}_{\;\!\mathcolor{gray}{z}}, \bar{H}^{\;\!\mathcolor{gray}{t}}_{\;\!\mathcolor{gray}{z}}$ 的介质/束缚源部分 $\bar{P}^{\;\!\mathcolor{gray}{t}}_{\;\!\mathcolor{gray}{z}},\bar{\bar{Q}}^{\;\!\mathcolor{gray}{t}}_{\;\!\mathcolor{gray}{z}},\bar{\bar{\bar{O}}}^{\;\!\mathcolor{gray}{t}}_{\;\!\mathcolor{gray}{z}} ; \bar{M}^{\;\!\mathcolor{gray}{t}}_{\;\!\mathcolor{gray}{z}}, \bar{\bar{N}}^{\;\!\mathcolor{gray}{t}}_{\;\!\mathcolor{gray}{z}}$ 暂时不写作 基本场 $\bar{E}^{\;\!\mathcolor{gray}{t}}_{\;\!\mathcolor{gray}{z}}, \bar{B}^{\;\!\mathcolor{gray}{t}}_{\;\!\mathcolor{gray}{z}}$ 或任何其他(引力、应力)场的显示函数。

由于矢量场 $\bar{A}^{\;\!\mathcolor{gray}{t}}_{\;\!\mathcolor{gray}{z}}$ 的旋度无源、标量场 $\phi^{\;\!\mathcolor{gray}{t}}_{\;\!\mathcolor{gray}{z}}$ 的梯度无旋,\bref{eq:D} 中的 $\bar{D}^{\;\!\mathcolor{gray}{t}}_{\;\!\mathcolor{gray}{z}}$、\bref{eq:H} 中的 $\bar{H}^{\;\!\mathcolor{gray}{t}}_{\;\!\mathcolor{gray}{z}}$ 可以分别加上 $\mathcolor{gray}{\bar{\nabla} \times} \bar{A}^{\;\!\mathcolor{gray}{t}}_{\;\!\mathcolor{gray}{z}}$、$\mathcolor{gray}{\bar{\nabla}} \phi^{\;\!\mathcolor{gray}{t}}_{\;\!\mathcolor{gray}{z}}$ 而不影响 \bref{eq:Div-D}、\bref{eq:Curl-H} 的成立,因此辅助场 $\bar{D}^{\;\!\mathcolor{gray}{t}}_{\;\!\mathcolor{gray}{z}}$ $~\left(~ + \mathcolor{gray}{\bar{\nabla} \times} \bar{A}^{\;\!\mathcolor{gray}{t}}_{\;\!\mathcolor{gray}{z}} ~\right)~$ 和 $\bar{H}^{\;\!\mathcolor{gray}{t}}_{\;\!\mathcolor{gray}{z}}$ $~\left(~ + \mathcolor{gray}{\bar{\nabla}} \phi^{\;\!\mathcolor{gray}{t}}_{\;\!\mathcolor{gray}{z}} ~\right)~$ 都不是唯一确定的\Footnote{类似标势与矢势也需要某种“规范”(如库伦/洛伦兹规范)才能确定一样。多极理论中的这种“规范”为:源的宏观表达式在坐标原点的平移下需保持不变,也称本构张量的原点独立性。}。—— 这一额外的自由度可以被用于确定在微观上与坐标原点无关的平移不变源 $\bar{P}^{\;\!\mathcolor{gray}{t}}_{\;\!\mathcolor{gray}{z}},\bar{\bar{Q}}^{\;\!\mathcolor{gray}{t}}_{\;\!\mathcolor{gray}{z}},\bar{\bar{\bar{O}}}^{\;\!\mathcolor{gray}{t}}_{\;\!\mathcolor{gray}{z}} ; \bar{M}^{\;\!\mathcolor{gray}{t}}_{\;\!\mathcolor{gray}{z}}, \bar{\bar{N}}^{\;\!\mathcolor{gray}{t}}_{\;\!\mathcolor{gray}{z}}$ (关于基本场 $\bar{E}^{\;\!\mathcolor{gray}{t}}_{\;\!\mathcolor{gray}{z}}, \bar{B}^{\;\!\mathcolor{gray}{t}}_{\;\!\mathcolor{gray}{z}}$ 的本构关系)的唯一确定且正确的宏观表达式\cite{welterTranslationallyInvariantSemiclassical2013,delangeTranslationalInvariancePost2012,langeTransitionMicroscopicMacroscopic2012,langeMultipoleTheoryHehl2015,raabCommentOriginDependence2010a,OriginindependentCalculationQuadrupole}\Footnote{微观表达式却不是唯一确定的\cite{OriginDependenceMaterial}。}。

\marginLeft[-2.5em]{ssec:step-delta}\subsection{${\rho}, \bar{J}$ 表面源、${\mathbb{1}},\delta$ 场源展开}\label{ssec:step-delta}

考虑在 $\mathcolor{gray}{z} = \mathcolor{gray}{0}$ 处面接触的两个半无限介质 \textcolor{Maroon}{0} 和介质 \textcolor{Maroon}{1},输入场$=$泵浦的能流方向从介质 \textcolor{Maroon}{0} $\to$ 介质 \textcolor{Maroon}{1},且与接触面内法向夹成锐角。在上述初始条件下,电荷/流源 ${\rho}^{\;\!\mathcolor{gray}{t}}_{\;\!\mathcolor{gray}{z}}, \bar{J}^{\;\!\mathcolor{gray}{t}}_{\;\!\mathcolor{gray}{z}}$ 中的每一个\Footnote{是将源 ${\rho}^{\;\!\mathcolor{gray}{t}}_{\;\!\mathcolor{gray}{z}}, \bar{J}^{\;\!\mathcolor{gray}{t}}_{\;\!\mathcolor{gray}{z}}$ 本身,而不是将构成源的底层元素们——束缚电/磁多极矩强度 $\bar{P}^{\;\!\mathcolor{gray}{t}}_{\;\!\mathcolor{gray}{z}},\bar{\bar{Q}}^{\;\!\mathcolor{gray}{t}}_{\;\!\mathcolor{gray}{z}},\bar{\bar{\bar{O}}}^{\;\!\mathcolor{gray}{t}}_{\;\!\mathcolor{gray}{z}} ; \bar{M}^{\;\!\mathcolor{gray}{t}}_{\;\!\mathcolor{gray}{z}}, \bar{\bar{N}}^{\;\!\mathcolor{gray}{t}}_{\;\!\mathcolor{gray}{z}}$,展开成 ${\mathbb{1}},\delta$ 的函数。—— 若将后者展开,则 ${\rho}^{\;\!\mathcolor{gray}{t}}_{\;\!\mathcolor{gray}{z}}$ 中也将产生 $\delta$ 项,并与 $\bar{J}^{\;\!\mathcolor{gray}{t}}_{\;\!\mathcolor{gray}{z}}$ 中的同层次 $\delta$ 项抵消,无法提供信息。} 都可以写成 2 个阶跃函数 / \textcolor{Maroon}{Heaviside} 单位函数
\begin{align} \label{eq:u}
	\mathbb{1}_{\mathcolor{gray}{z}} := \mathbb{1} \left( \mathcolor{gray}{z} \right) = ~\left\{~ \begin{aligned} 
		&1 &&, \mathcolor{gray}{z \mathcolor{black}{>} 0} \\ 
		&\textcolor{Maroon}{\text{undefined}} &&, \mathcolor{gray}{z \mathcolor{black}{=} 0} \\
		&0 &&, \mathcolor{gray}{z \mathcolor{black}{<} 0} \end{aligned}\right. ~,
\end{align}
之和:即使用
\begin{subequations} \label{eq:u-01}
	\abovedisplayskip=-5pt
	\belowdisplayskip=10pt
\begin{align}
	\leftindex_{\textcolor{Maroon}{1}} {\mathbb{1}}_{\mathcolor{gray}{z}} &= {\mathbb{1}}_{\mathcolor{gray}{z}} ~, \label{eq:u-1} \\ 
	\leftindex_{\textcolor{Maroon}{0}} {\mathbb{1}}_{\mathcolor{gray}{z}} &= {\mathbb{1}}_{ - \mathcolor{gray}{z} } ~, \label{eq:u-0}
\end{align}
\end{subequations}
将电荷/流源 $\rho_{\;\!\textcolor{Maroon}{\text{b}}}, J_{\;\!\textcolor{Maroon}{\text{b}}}$ 中的每一个(均暂记为 $X$)暂表达为:
\abovedisplayskip=8pt
\belowdisplayskip=10pt
\begin{align} \label{eq:X-01}
	X_{\mathcolor{gray}{z}} = \leftindex_{\textcolor{Maroon}{\mathsfit{z}}} {\mathbb{1}}_{\mathcolor{gray}{z}} \leftindex^{\textcolor{Maroon}{\mathsfit{z}}} X_{\mathcolor{gray}{z}} ~~~~, \text{其中} ~~~ \textcolor{Maroon}{\mathsfit{z}} = \textcolor{Maroon}{0,1} ~,
\end{align}
然后将二者代入由 \bref{eq:div-e,eq:div-e-f} 构建的束缚电荷 ${Q}^{\;\!\mathcolor{gray}{t}}_{\;\!\textcolor{Maroon}{\text{b}}\mathcolor{gray}{z}}$ 守恒\Footnote{束缚电荷 ${Q}^{\;\!\mathcolor{gray}{t}}_{\;\!\textcolor{Maroon}{\text{b}}\mathcolor{gray}{z}}$ 与 自由电荷 ${Q}^{\;\!\mathcolor{gray}{t}}_{\;\!\textcolor{Maroon}{\text{f}}\mathcolor{gray}{z}}$ 可以相互转换,因此各自单独而言在最广义上不(一定)守恒:通过价带/杂质能级到导带的跃迁,比如物质对光子的线性、非线性吸收并产生光电压/流所对应的内/外光伏/电效应\cite{boydNonlinearOptics2019};因此 \bref{eq:div-e-b} 和 \bref{eq:div-e-f} 均不是定律,相比总电荷 ${Q}^{\;\!\mathcolor{gray}{t}}_{\;\!\mathcolor{gray}{z}}$ 守恒 \bref{eq:div-e} 而言。在边界条件伊始,暂时不考虑摩擦起电、外光电效应、激光加工之强场激发等离子体\cite{boydNonlinearOptics2019,dengTheoryElectrodynamicResponse2020}等非平衡/含驰豫现象,因此暂认为束缚、自由电荷是分别独立守恒的。}中\cite{grahamMultipoleSolutionMacroscopic2000,delangeElectromagneticBoundaryConditions2013}:
\begin{subequations}
	\abovedisplayskip=-5pt
	\belowdisplayskip=12pt
\begin{align}
	\mathcolor{gray}{\nabla^i} J^{\;\!\mathcolor{gray}{t}}_{\;\!\textcolor{Maroon}{\text{b}} i\mathcolor{gray}{z}} &+ \mathcolor{gray}{\nabla^t} {\rho}^{\;\!\mathcolor{gray}{t}}_{\;\!\textcolor{Maroon}{\text{b}}\mathcolor{gray}{z}} &&\hspace{-2em}= 0 \label{eq:div-e-b} \\ 
	\mathcolor{gray}{\nabla^i} \left( \leftindex_{\textcolor{Maroon}{\mathsfit{z}}} {\mathbb{1}}_{\mathcolor{gray}{z}} \leftindex^{\textcolor{Maroon}{\mathsfit{z}}} J^{\;\!\mathcolor{gray}{t}}_{\;\!\textcolor{Maroon}{\text{b}} i\mathcolor{gray}{z}} \right) &+ \mathcolor{gray}{\nabla^t} \left( \leftindex_{\textcolor{Maroon}{\mathsfit{z}}} {\mathbb{1}}_{\mathcolor{gray}{z}} \leftindex^{\textcolor{Maroon}{\mathsfit{z}}} {\rho}^{\;\!\mathcolor{gray}{t}}_{\;\!\textcolor{Maroon}{\text{b}}\mathcolor{gray}{z}} \right) &&\hspace{-2em}= 0 \label{eq:div-e-b-01} \\ 
	\leftindex_{\textcolor{Maroon}{\mathsfit{z}}} {\mathbb{1}}_{\mathcolor{gray}{z}} \left( \mathcolor{gray}{\nabla_x} \leftindex^{\textcolor{Maroon}{\mathsfit{z}}} J^{\;\!\mathcolor{gray}{t}}_{\;\!\textcolor{Maroon}{\text{b}} \symup{x} \mathcolor{gray}{z}} + \mathcolor{gray}{\nabla_y} \leftindex^{\textcolor{Maroon}{\mathsfit{z}}} J^{\;\!\mathcolor{gray}{t}}_{\;\!\textcolor{Maroon}{\text{b}} \symup{y} \mathcolor{gray}{z}} \right) + \mathcolor{gray}{\nabla_z} \left( \leftindex_{\textcolor{Maroon}{\mathsfit{z}}} {\mathbb{1}}_{\mathcolor{gray}{z}} \leftindex^{\textcolor{Maroon}{\mathsfit{z}}} J^{\;\!\mathcolor{gray}{t}}_{\;\!\textcolor{Maroon}{\text{b}} \symup{z} \mathcolor{gray}{z}} \right) &+ \leftindex_{\textcolor{Maroon}{\mathsfit{z}}} {\mathbb{1}}_{\mathcolor{gray}{z}} \mathcolor{gray}{\nabla^t} \leftindex^{\textcolor{Maroon}{\mathsfit{z}}} {\rho}^{\;\!\mathcolor{gray}{t}}_{\;\!\textcolor{Maroon}{\text{b}}\mathcolor{gray}{z}} &&\hspace{-2em}= 0~, \label{eq:div-e-b-01'}
\end{align}
\end{subequations}
注意到,在空域上对 $\leftindex_{\textcolor{Maroon}{\mathsfit{z}}} {\mathbb{1}}_{\mathcolor{gray}{z}} \leftindex^{\textcolor{Maroon}{\mathsfit{z}}} X_{\mathcolor{gray}{z}}$ 求 $\mathcolor{gray}{z}$ 向一阶或高阶偏导 $\mathcolor{gray}{\nabla_z}$,即
\begin{subequations}
	\abovedisplayskip=8pt
	\belowdisplayskip=10pt
\begin{align}
	\mathcolor{gray}{\nabla_z} \left( \leftindex_{\textcolor{Maroon}{\mathsfit{z}}} {\mathbb{1}}_{\mathcolor{gray}{z}} \leftindex^{\textcolor{Maroon}{\mathsfit{z}}} X_{\mathcolor{gray}{z}} \right) &= \leftindex_{\textcolor{Maroon}{\mathsfit{z}}} \;\! \delta_{\mathcolor{gray}{z}} \leftindex^{\textcolor{Maroon}{\mathsfit{z}}} X_{\mathcolor{gray}{z}} + \leftindex_{\textcolor{Maroon}{\mathsfit{z}}} {\mathbb{1}}_{\mathcolor{gray}{z}} \left( \mathcolor{gray}{\nabla_z} \leftindex^{\textcolor{Maroon}{\mathsfit{z}}} X_{\mathcolor{gray}{z}} \right) \label{eq:partial_z-step} \\ 
	\mathcolor{gray}{\nabla_z^2} \left( \leftindex_{\textcolor{Maroon}{\mathsfit{z}}} {\mathbb{1}}_{\mathcolor{gray}{z}} \leftindex^{\textcolor{Maroon}{\mathsfit{z}}} X_{\mathcolor{gray}{z}} \right) &= \leftindex_{\textcolor{Maroon}{\mathsfit{z}}} \;\! \delta'_{\mathcolor{gray}{z}} \leftindex^{\textcolor{Maroon}{\mathsfit{z}}} X_{\mathcolor{gray}{z}} + 2 \leftindex_{\textcolor{Maroon}{\mathsfit{z}}} \;\! \delta_{\mathcolor{gray}{z}} \left( \mathcolor{gray}{\nabla_z} \leftindex^{\textcolor{Maroon}{\mathsfit{z}}} X_{\mathcolor{gray}{z}} \right) + \leftindex_{\textcolor{Maroon}{\mathsfit{z}}} {\mathbb{1}}_{\mathcolor{gray}{z}} \left( \mathcolor{gray}{\nabla_z^2} \leftindex^{\textcolor{Maroon}{\mathsfit{z}}} X_{\mathcolor{gray}{z}} \right) ~. \label{eq:partial_z^2-step}
\end{align}
\end{subequations}
会产生(带有单位 $\mathcolor{gray}{\symup{m}^{-1}}$ 的)$\delta_{\mathcolor{gray}{z}}$ 函数
\begin{subequations}
	\abovedisplayskip=7pt
	%\belowdisplayskip=0pt
\begin{align}
	\leftindex_{\textcolor{Maroon}{1}} \;\! \delta_{\mathcolor{gray}{z}} &= \mathcolor{gray}{\nabla_z} \leftindex_{\textcolor{Maroon}{1}} {\mathbb{1}}_{\mathcolor{gray}{z}} = {\mathbb{1}'}_{\mathcolor{gray}{z}} = \delta_{\mathcolor{gray}{z}} ~, \label{eq:delta-1} \\ 
	\leftindex_{\textcolor{Maroon}{0}} \;\! \delta_{\mathcolor{gray}{z}} &= \leftindex_{\textcolor{Maroon}{0}} {\mathbb{1}'}_{\mathcolor{gray}{z}} = - \leftindex_{\textcolor{Maroon}{1}} {\mathbb{1}'}_{\mathcolor{gray}{z}} = - \delta_{\mathcolor{gray}{z}} ~. \label{eq:delta-0}
\end{align}
\end{subequations}
及其(带有单位 $\mathcolor{gray}{\symup{m}^{-2}}$ 的)导数 $\delta'_{\mathcolor{gray}{z}}$ 等。—— 具体到 \bref{eq:div-e-b-01} 的情况,即对 $\bar{J}^{\;\!\mathcolor{gray}{t}}_{\;\!\textcolor{Maroon}{\text{b}}\mathcolor{gray}{z}}$ 的 $\mathcolor{gray}{z}$ 向散度 $\mathcolor{gray}{\nabla_z} \left( \leftindex_{\textcolor{Maroon}{\mathsfit{z}}} {\mathbb{1}}_{\mathcolor{gray}{z}} \leftindex^{\textcolor{Maroon}{\mathsfit{z}}} J^{\;\!\mathcolor{gray}{t}}_{\;\!\textcolor{Maroon}{\text{b}} \symup{z} \mathcolor{gray}{z}} \right)$,将产生 $\leftindex_{\textcolor{Maroon}{\mathsfit{z}}} \;\! \delta_{\mathcolor{gray}{z}} \leftindex^{\textcolor{Maroon}{\mathsfit{z}}} J^{\;\!\mathcolor{gray}{t}}_{\;\!\textcolor{Maroon}{\text{b}} \symup{z} \mathcolor{gray}{z}}$ 项(如 \bref{eq:partial_z-step} 所示),该项只能通过 ${\rho}^{\;\!\mathcolor{gray}{t}}_{\;\!\textcolor{Maroon}{\text{b}}\mathcolor{gray}{z}}$ 中也原始地就含有对应的表面 $\delta_{\mathcolor{gray}{z}}$ 项(如铁电体 $\textcolor{Maroon}{-\symup{c}}$ 面的表面束缚电子 ${\sigma}^{\;\!\mathcolor{gray}{t}}_{\;\!\textcolor{Maroon}{\text{b}}\mathcolor{gray}{z}}$)以吸收/消除掉,否则将违反 \bref{eq:div-e-b}\Footnote{这归根结底是因为 —— 在 \bref{eq:div-e-b} 中,对 ${\rho}^{\;\!\mathcolor{gray}{t}}_{\;\!\textcolor{Maroon}{\text{b}}\mathcolor{gray}{z}}$ 只有时间偏导 $\mathcolor{gray}{\nabla^t}$,然而对 $\bar{J}^{\;\!\mathcolor{gray}{t}}_{\;\!\textcolor{Maroon}{\text{b}}\mathcolor{gray}{z}}$ 却有空域偏导 $\mathcolor{gray}{\bar{\nabla}}$,特别是对 $\mathcolor{gray}{z}$ 的偏导 $\mathcolor{gray}{\nabla_z}$:这种不对称性最终会强制拓展 ${\rho}^{\;\!\mathcolor{gray}{t}}_{\;\!\textcolor{Maroon}{\text{b}}\mathcolor{gray}{z}}$ 的 \bref{eq:X-01} 并为之带来下一阶不连续/表面项。}。

此外,不仅 \bref{eq:X-01} 需要为 ${\rho}^{\;\!\mathcolor{gray}{t}}_{\;\!\textcolor{Maroon}{\text{b}}\mathcolor{gray}{z}}$ 拓展,以适应对 $\bar{J}^{\;\!\mathcolor{gray}{t}}_{\;\!\textcolor{Maroon}{\text{b}}\mathcolor{gray}{z}}$ 的 \bref{eq:X-01} 展开的一阶 $\mathcolor{gray}{z}$ 向偏导所生成的 $\delta_{\mathcolor{gray}{z}}$ 函数 —— $\bar{J}^{\;\!\mathcolor{gray}{t}}_{\;\!\textcolor{Maroon}{\text{b}}\mathcolor{gray}{z}}$ 本身还有表面电流(如金属/导体),相应地 $\bar{J}^{\;\!\mathcolor{gray}{t}}_{\;\!\textcolor{Maroon}{\text{b}}\mathcolor{gray}{z}}$ 的原始 \bref{eq:X-01} 也应该添加一个 $\delta_{\mathcolor{gray}{z}}$ 函数;那么为满足 \bref{eq:X-01},对应 ${\rho}^{\;\!\mathcolor{gray}{t}}_{\;\!\textcolor{Maroon}{\text{b}}\mathcolor{gray}{z}}$ 的表达式会继续出现 $\delta'_{\mathcolor{gray}{z}}$ 项。因此需要同时将 ${\rho}^{\;\!\mathcolor{gray}{t}}_{\;\!\textcolor{Maroon}{\text{b}}\mathcolor{gray}{z}},\bar{J}^{\;\!\mathcolor{gray}{t}}_{\;\!\textcolor{Maroon}{\text{b}}\mathcolor{gray}{z}}$ 分别拓展 \bref{eq:X-01} 至:
\begin{subequations} \label{eq:e-b-01}
	\abovedisplayskip=-3pt
	\belowdisplayskip=12pt
\begin{align}
	J^{\;\!\mathcolor{gray}{t}}_{\;\!\textcolor{Maroon}{\text{b}} i\mathcolor{gray}{z}} &= &&\hspace{-1em}\leftindex_{\textcolor{Maroon}{\mathsfit{z}}} {\mathbb{1}}_{\mathcolor{gray}{z}} \leftindex^{\textcolor{Maroon}{\mathsfit{z}}} J^{\;\!\mathcolor{gray}{t}}_{\;\!\textcolor{Maroon}{\text{b}} i\mathcolor{gray}{z}} &&\hspace{-1em}+ \leftindex_{\textcolor{Maroon}{\mathsfit{z}}} \;\! \delta_{\mathcolor{gray}{z}} \leftindex^{\textcolor{Maroon}{\mathsfit{z}}}
	{\mathcal{K}}^{\;\!\mathcolor{gray}{t}}_{\;\!\textcolor{Maroon}{\text{b}} i\symup{z}\mathcolor{gray}{z}} \left[ - \symup{sgn} \left( \leftindex_{\textcolor{Maroon}{\mathsfit{z}}} \;\! \delta_{\mathcolor{gray}{z}} \right) \right] &&\hspace{-1em} \\ 
	&= &&\hspace{-1em}\leftindex_{\textcolor{Maroon}{\mathsfit{z}}} {\mathbb{1}}_{\mathcolor{gray}{z}} \leftindex^{\textcolor{Maroon}{\mathsfit{z}}} J^{\;\!\mathcolor{gray}{t}}_{\;\!\textcolor{Maroon}{\text{b}} i\mathcolor{gray}{z}} &&\hspace{-1em}- \delta_{\mathcolor{gray}{z}} \leftindex_{\textcolor{Maroon}{\mathsfit{z}}} \;\! \leftindex^{\textcolor{Maroon}{\mathsfit{z}}}
	{\mathcal{K}}^{\;\!\mathcolor{gray}{t}}_{\;\!\textcolor{Maroon}{\text{b}} i\symup{z}\mathcolor{gray}{z}} &&\hspace{-1em} \label{eq:j-b-01} \\
	{\rho}^{\;\!\mathcolor{gray}{t}}_{\;\!\textcolor{Maroon}{\text{b}}\mathcolor{gray}{z}} &= &&\hspace{-1em}\leftindex_{\textcolor{Maroon}{\mathsfit{z}}} {\mathbb{1}}_{\mathcolor{gray}{z}} \leftindex^{\textcolor{Maroon}{\mathsfit{z}}} {\rho}^{\;\!\mathcolor{gray}{t}}_{\;\!\textcolor{Maroon}{\text{b}}\mathcolor{gray}{z}} &&\hspace{-1em}+ \leftindex_{\textcolor{Maroon}{\mathsfit{z}}} \;\! \delta_{\mathcolor{gray}{z}} \leftindex^{\textcolor{Maroon}{\mathsfit{z}}} \;\! {\mathcal{P}}^{\;\!\mathcolor{gray}{t}}_{\;\!\textcolor{Maroon}{\text{b}} \symup{z} \mathcolor{gray}{z}} \left[ - \symup{sgn} \left( \leftindex_{\textcolor{Maroon}{\mathsfit{z}}} \;\! \delta_{\mathcolor{gray}{z}} \right) \right] &&\hspace{-1em}+ \leftindex_{\textcolor{Maroon}{\mathsfit{z}}} \;\! \delta'_{\mathcolor{gray}{z}} \leftindex^{\textcolor{Maroon}{\mathsfit{z}}} \;\! {\mathcal{Q}}^{\;\!\mathcolor{gray}{t}}_{\;\!\textcolor{Maroon}{\text{b}} \symup{z} \symup{z} \mathcolor{gray}{z}} \left[ - \symup{sgn} \left( \leftindex_{\textcolor{Maroon}{\mathsfit{z}}} \;\! \delta'_{\mathcolor{gray}{z}} \right) \right]^2 \\ 
	&= &&\hspace{-1em}\leftindex_{\textcolor{Maroon}{\mathsfit{z}}} {\mathbb{1}}_{\mathcolor{gray}{z}} \leftindex^{\textcolor{Maroon}{\mathsfit{z}}} {\rho}^{\;\!\mathcolor{gray}{t}}_{\;\!\textcolor{Maroon}{\text{b}}\mathcolor{gray}{z}} &&\hspace{-1em}- \delta_{\mathcolor{gray}{z}} \leftindex_{\textcolor{Maroon}{\mathsfit{z}}} \;\! \leftindex^{\textcolor{Maroon}{\mathsfit{z}}} \;\! {\mathcal{P}}^{\;\!\mathcolor{gray}{t}}_{\;\!\textcolor{Maroon}{\text{b}} \symup{z} \mathcolor{gray}{z}} &&\hspace{-1em}+ \leftindex_{\textcolor{Maroon}{\mathsfit{z}}} \;\! \delta'_{\mathcolor{gray}{z}} \leftindex^{\textcolor{Maroon}{\mathsfit{z}}} \;\! {\mathcal{Q}}^{\;\!\mathcolor{gray}{t}}_{\;\!\textcolor{Maroon}{\text{b}} \symup{z} \symup{z} \mathcolor{gray}{z}} \label{eq:p-b-01}
\end{align}
\end{subequations}
其中,定义了 介质 $\textcolor{Maroon}{\mathsfit{z}}$ 在 $\mathcolor{gray}{z} = \mathcolor{gray}{0}$ 面上的表面单位外法向量 $\leftindex^{\textcolor{Maroon}{\mathsfit{z}}} {\hat{n}}_{\mathcolor{gray}{z}}$ 与 $\hat{\symup{e}}_{\mathcolor{gray}{z}}$ 的点积
\begin{align} \label{eq:surf-normal-dot}
	- \symup{sgn} \left( \leftindex_{\textcolor{Maroon}{\mathsfit{z}}} \;\! \delta_{\mathcolor{gray}{z}} \right) = - \symup{sgn} \left( \leftindex_{\textcolor{Maroon}{\mathsfit{z}}} \;\! \delta'_{\mathcolor{gray}{z}} \right) = \cdots = \leftindex^{\textcolor{Maroon}{\mathsfit{z}}} {\hat{n}}_{\mathcolor{gray}{z}} \cdot \hat{\symup{e}}_{\mathcolor{gray}{z}}~.
\end{align}
为了一步到位,\bref{eq:j-b-01,eq:p-b-01} 还可以进一步拓展至高一阶项:
\begin{subequations} \label{eq:e-b-01'}
\begin{align}
	J^{\;\!\mathcolor{gray}{t}}_{\;\!\textcolor{Maroon}{\text{b}} i\mathcolor{gray}{z}} &= &&\hspace{-2em}\leftindex_{\textcolor{Maroon}{\mathsfit{z}}} {\mathbb{1}}_{\mathcolor{gray}{z}} \leftindex^{\textcolor{Maroon}{\mathsfit{z}}} J^{\;\!\mathcolor{gray}{t}}_{\;\!\textcolor{Maroon}{\text{b}} i\mathcolor{gray}{z}} &&\hspace{-2em}- \delta_{\mathcolor{gray}{z}} \leftindex_{\textcolor{Maroon}{\mathsfit{z}}} \;\! \leftindex^{\textcolor{Maroon}{\mathsfit{z}}}
	{\mathcal{K}}^{\;\!\mathcolor{gray}{t}}_{\;\!\textcolor{Maroon}{\text{b}} i\symup{z}\mathcolor{gray}{z}} &&\hspace{-2em}+ \leftindex_{\textcolor{Maroon}{\mathsfit{z}}} \;\! \delta'_{\mathcolor{gray}{z}} \leftindex^{\textcolor{Maroon}{\mathsfit{z}}} \;\! {\mathcal{L}}^{\;\!\mathcolor{gray}{t}}_{\;\!\textcolor{Maroon}{\text{b}} i\symup{z} \symup{z} \mathcolor{gray}{z}} &&\hspace{-2em} \label{eq:j-b-01'} \\
	{\rho}^{\;\!\mathcolor{gray}{t}}_{\;\!\textcolor{Maroon}{\text{b}}\mathcolor{gray}{z}} &= &&\hspace{-2em}\leftindex_{\textcolor{Maroon}{\mathsfit{z}}} {\mathbb{1}}_{\mathcolor{gray}{z}} \leftindex^{\textcolor{Maroon}{\mathsfit{z}}} {\rho}^{\;\!\mathcolor{gray}{t}}_{\;\!\textcolor{Maroon}{\text{b}}\mathcolor{gray}{z}} &&\hspace{-2em}- \delta_{\mathcolor{gray}{z}} \leftindex_{\textcolor{Maroon}{\mathsfit{z}}} \;\! \leftindex^{\textcolor{Maroon}{\mathsfit{z}}} \;\! {\mathcal{P}}^{\;\!\mathcolor{gray}{t}}_{\;\!\textcolor{Maroon}{\text{b}} \symup{z} \mathcolor{gray}{z}} &&\hspace{-2em}+ \leftindex_{\textcolor{Maroon}{\mathsfit{z}}} \;\! \delta'_{\mathcolor{gray}{z}} \leftindex^{\textcolor{Maroon}{\mathsfit{z}}} \;\! {\mathcal{Q}}^{\;\!\mathcolor{gray}{t}}_{\;\!\textcolor{Maroon}{\text{b}} \symup{z} \symup{z} \mathcolor{gray}{z}} &&\hspace{-2em}- \delta''_{\mathcolor{gray}{z}} \leftindex_{\textcolor{Maroon}{\mathsfit{z}}} \;\! \leftindex^{\textcolor{Maroon}{\mathsfit{z}}} \;\! {\mathcal{O}}^{\;\!\mathcolor{gray}{t}}_{\;\!\textcolor{Maroon}{\text{b}} \symup{z} \symup{z} \symup{z} \mathcolor{gray}{z}} \label{eq:p-b-01'}
\end{align}
\end{subequations}
现将 \bref{eq:j-b-01',eq:p-b-01'} 代入 \bref{eq:div-e-b} 中\Footnote{利用 \bref{eq:div-e-b-01'} 和 \bref{eq:partial_z-step} 所导出的:$\leftindex_{\textcolor{Maroon}{\mathsfit{z}}} \;\! \delta_{\mathcolor{gray}{z}} \leftindex^{\textcolor{Maroon}{\mathsfit{z}}} J^{\;\!\mathcolor{gray}{t}}_{\;\!\textcolor{Maroon}{\text{b}} \symup{z} \mathcolor{gray}{z}} + \leftindex_{\textcolor{Maroon}{\mathsfit{z}}} {\mathbb{1}}_{\mathcolor{gray}{z}} \left( \mathcolor{gray}{\nabla^i} \leftindex^{\textcolor{Maroon}{\mathsfit{z}}} J^{\;\!\mathcolor{gray}{t}}_{\;\!\textcolor{Maroon}{\text{b}} i\mathcolor{gray}{z}} + \mathcolor{gray}{\nabla^t} \leftindex^{\textcolor{Maroon}{\mathsfit{z}}} {\rho}^{\;\!\mathcolor{gray}{t}}_{\;\!\textcolor{Maroon}{\text{b}}\mathcolor{gray}{z}} \right) = 0$。},同介质 $\textcolor{Maroon}{\mathsfit{z}}$ 内(另一侧介质是真空时也须成立),所有同阶次的奇异项必须分别(分层次)为零,因此有
\begin{subequations}
\begin{align}
	{\delta}_{\mathcolor{gray}{z}} ~\textcolor{Maroon}{\text{项}}:&\hspace{1.0em} + \leftindex^{\textcolor{Maroon}{\mathsfit{z}}} {\delta}_{\mathcolor{gray}{z}} \leftindex^{\textcolor{Maroon}{\mathsfit{z}}} J^{\;\!\mathcolor{gray}{t}}_{\;\!\textcolor{Maroon}{\text{b}} \symup{z}\mathcolor{gray}{z}} &&\hspace{-1.5em}- \hphantom{j} \delta_{\mathcolor{gray}{z}} \left( \mathcolor{gray}{\nabla^i} \leftindex^{\textcolor{Maroon}{\mathsfit{z}}}
	{\mathcal{K}}^{\;\!\mathcolor{gray}{t}}_{\;\!\textcolor{Maroon}{\text{b}} i\symup{z}\mathcolor{gray}{z}} \right. &&\hspace{-1.5em}+ \left. \mathcolor{gray}{\nabla^t} \leftindex^{\textcolor{Maroon}{\mathsfit{z}}} \;\! {\mathcal{P}}^{\;\!\mathcolor{gray}{t}}_{\;\!\textcolor{Maroon}{\text{b}} \symup{z} \mathcolor{gray}{z}} \right) &&\hspace{-1.5em}= 0~, \label{eq:div-e-b-01-delta} \\
	{\delta}'_{\mathcolor{gray}{z}} ~\textcolor{Maroon}{\text{项}}:&\hspace{1.0em} - \hphantom{j} \delta'_{\mathcolor{gray}{z}} \leftindex^{\textcolor{Maroon}{\mathsfit{z}}}
	{\mathcal{K}}^{\;\!\mathcolor{gray}{t}}_{\;\!\textcolor{Maroon}{\text{b}} \symup{z} \symup{z}\mathcolor{gray}{z}} &&\hspace{-1.5em}+ \leftindex^{\textcolor{Maroon}{\mathsfit{z}}} \delta'_{\mathcolor{gray}{z}} \left( \mathcolor{gray}{\nabla^i} \leftindex^{\textcolor{Maroon}{\mathsfit{z}}} \;\! {\mathcal{L}}^{\;\!\mathcolor{gray}{t}}_{\;\!\textcolor{Maroon}{\text{b}} i\symup{z} \symup{z} \mathcolor{gray}{z}} \right. &&\hspace{-1.5em}+ \left. \mathcolor{gray}{\nabla^t} \leftindex^{\textcolor{Maroon}{\mathsfit{z}}} \;\! {\mathcal{Q}}^{\;\!\mathcolor{gray}{t}}_{\;\!\textcolor{Maroon}{\text{b}} \symup{z} \symup{z} \mathcolor{gray}{z}} \right) &&\hspace{-1.5em}= 0~, \label{eq:div-e-b-01-delta'} \\
	{\delta}''_{\mathcolor{gray}{z}} ~\textcolor{Maroon}{\text{项}}:&\hspace{1.0em} + \leftindex^{\textcolor{Maroon}{\mathsfit{z}}} {\delta}''_{\mathcolor{gray}{z}} \leftindex^{\textcolor{Maroon}{\mathsfit{z}}} \;\! {\mathcal{L}}^{\;\!\mathcolor{gray}{t}}_{\;\!\textcolor{Maroon}{\text{b}} \symup{z} \symup{z} \symup{z} \mathcolor{gray}{z}} &&\hspace{-1.5em}- \hphantom{j} \delta''_{\mathcolor{gray}{z}} \left( \mathcolor{gray}{\nabla^i} \leftindex^{\textcolor{Maroon}{\mathsfit{z}}} \cdots \right. &&\hspace{-1.5em}+ \left. \mathcolor{gray}{\nabla^t} \leftindex^{\textcolor{Maroon}{\mathsfit{z}}} \;\! {\mathcal{O}}^{\;\!\mathcolor{gray}{t}}_{\;\!\textcolor{Maroon}{\text{b}} \symup{z} \symup{z} \symup{z} \mathcolor{gray}{z}} \right) &&\hspace{-1.5em}= 0~, \label{eq:div-e-b-01-delta''}
\end{align}
\end{subequations}
经典的多极理论给出 \bref{eq:j-b-01',eq:p-b-01'} 中的各体/表面电荷/流项:
\begin{subequations}
\begin{align}
	&{\mathcal{O}}^{\;\!\mathcolor{gray}{t}}_{\;\!\textcolor{Maroon}{\text{b}} ijk \mathcolor{gray}{z}} \hspace{-1em}&&=\hspace{0.2em} +~ \frac{1}{3!}~ O^{\;\!\mathcolor{gray}{t}}_{\;\! ijk\mathcolor{gray}{z}} &&\hspace{-0.7em}- \cdots~, &&\hspace{0.3em} {\mathcal{L}}^{\;\!\mathcolor{gray}{t}}_{\;\!\textcolor{Maroon}{\text{m}} ijl \mathcolor{gray}{z}} \hspace{-1em}&&=\hspace{0.2em} +~ \frac{1}{2!}~ \epsilon^{\hphantom{ij}k}_{i\mathcolor{gray}{j}} N^{\;\!\mathcolor{gray}{t}}_{\;\! kl\mathcolor{gray}{z}} &&\hspace{-0.7em}- \cdots~, \label{eq:Ob-Lm} \\
	&{\mathcal{Q}}^{\;\!\mathcolor{gray}{t}}_{\;\!\textcolor{Maroon}{\text{b}} ij \mathcolor{gray}{z}} \hspace{-1em}&&=\hspace{0.2em} -~ \frac{1}{2!}~ Q^{\;\!\mathcolor{gray}{t}}_{\;\! ij\mathcolor{gray}{z}} &&\hspace{-0.7em}+ \mathcolor{gray}{\nabla^k} {\mathcal{O}}^{\;\!\mathcolor{gray}{t}}_{\;\!\textcolor{Maroon}{\text{b}} ijk \mathcolor{gray}{z}}~,  &&\hspace{0.3em} {\mathcal{K}}^{\;\!\mathcolor{gray}{t}}_{\;\!\textcolor{Maroon}{\text{m}} ij \mathcolor{gray}{z}} \hspace{-1em}&&=\hspace{0.2em} -~ \hphantom{\frac{1}{2!}}~ \epsilon^{\hphantom{ij}k}_{i\mathcolor{gray}{j}} M^{\;\!\mathcolor{gray}{t}}_{\;\! k\mathcolor{gray}{z}} &&\hspace{-0.7em}+ \mathcolor{gray}{\nabla^l} {\mathcal{L}}^{\;\!\mathcolor{gray}{t}}_{\;\!\textcolor{Maroon}{\text{m}} ijl \mathcolor{gray}{z}}~, \label{eq:Qb-Km} \\
	&{\mathcal{P}}^{\;\!\mathcolor{gray}{t}}_{\;\!\textcolor{Maroon}{\text{b}} i \mathcolor{gray}{z}} \hspace{-1em}&&=\hspace{0.2em} ~ \hphantom{+\frac{1}{2!}}~ P^{\;\!\mathcolor{gray}{t}}_{\;\! i\mathcolor{gray}{z}} &&\hspace{-0.7em}+ \mathcolor{gray}{\nabla^j} {\mathcal{Q}}^{\;\!\mathcolor{gray}{t}}_{\;\!\textcolor{Maroon}{\text{b}} ij \mathcolor{gray}{z}}~, &&\hspace{0.3em} {J}^{\;\!\mathcolor{gray}{t}}_{\;\!\textcolor{Maroon}{\text{m}} i \mathcolor{gray}{z}} \hspace{-1em}&&=\hspace{0.2em} &&\hspace{-0.7em}- \mathcolor{gray}{\nabla^j} {\mathcal{K}}^{\;\!\mathcolor{gray}{t}}_{\;\!\textcolor{Maroon}{\text{m}} ij \mathcolor{gray}{z}}~, \label{eq:Pb-Jm} \\
	&{\rho}^{\;\!\mathcolor{gray}{t}}_{\;\!\textcolor{Maroon}{\text{b}} \mathcolor{gray}{z}} \hspace{-1em}&&=\hspace{0.2em} &&\hspace{-0.7em}- \mathcolor{gray}{\nabla^i} {\mathcal{P}}^{\;\!\mathcolor{gray}{t}}_{\;\!\textcolor{Maroon}{\text{b}} i \mathcolor{gray}{z}}~, &&\hspace{0.3em} \hspace{-1em}&&\hspace{0.2em} &&\hspace{-0.7em} \label{eq:pb-Jb} \\
	&{J}^{\;\!\mathcolor{gray}{t}}_{\;\!\textcolor{Maroon}{\text{e}} i \mathcolor{gray}{z}} \hspace{-1em}&&=\hspace{0.2em} &&\hspace{-0.7em}+ \mathcolor{gray}{\nabla^t} {\mathcal{P}}^{\;\!\mathcolor{gray}{t}}_{\;\!\textcolor{Maroon}{\text{b}} i \mathcolor{gray}{z}}~, &&\hspace{0.3em} {J}^{\;\!\mathcolor{gray}{t}}_{\;\!\textcolor{Maroon}{\text{b}} i \mathcolor{gray}{z}} \hspace{-1em}&&=\hspace{0.2em} ~ \hphantom{-\frac{1}{2!} \epsilon^{\hphantom{ij}k}_{i\mathcolor{gray}{j}}} {J}^{\;\!\mathcolor{gray}{t}}_{\;\!\textcolor{Maroon}{\text{e}} i \mathcolor{gray}{z}} &&\hspace{-0.7em}+ {J}^{\;\!\mathcolor{gray}{t}}_{\;\!\textcolor{Maroon}{\text{m}} i \mathcolor{gray}{z}}~, \label{eq:Je-Jb} \\
	&{\mathcal{K}}^{\;\!\mathcolor{gray}{t}}_{\;\!\textcolor{Maroon}{\text{e}} ij \mathcolor{gray}{z}} \hspace{-1em}&&=\hspace{0.2em} &&\hspace{-0.7em}- \mathcolor{gray}{\nabla^t} {\mathcal{Q}}^{\;\!\mathcolor{gray}{t}}_{\;\!\textcolor{Maroon}{\text{b}} ij \mathcolor{gray}{z}}~, &&\hspace{0.3em} {\mathcal{K}}^{\;\!\mathcolor{gray}{t}}_{\;\!\textcolor{Maroon}{\text{b}} ij \mathcolor{gray}{z}} \hspace{-1em}&&=\hspace{0.2em} ~ \hphantom{-\frac{1}{2!} \epsilon^{\hphantom{ij}k}_{i\mathcolor{gray}{j}}} {\mathcal{K}}^{\;\!\mathcolor{gray}{t}}_{\;\!\textcolor{Maroon}{\text{e}} ij \mathcolor{gray}{z}} &&\hspace{-0.7em}+ {\mathcal{K}}^{\;\!\mathcolor{gray}{t}}_{\;\!\textcolor{Maroon}{\text{m}} ij \mathcolor{gray}{z}}~, \label{eq:Ke-Kb} \\
	&{\mathcal{L}}^{\;\!\mathcolor{gray}{t}}_{\;\!\textcolor{Maroon}{\text{e}} ijl \mathcolor{gray}{z}} \hspace{-1em}&&=\hspace{0.2em} &&\hspace{-0.7em}- \mathcolor{gray}{\nabla^t} {\mathcal{O}}^{\;\!\mathcolor{gray}{t}}_{\;\!\textcolor{Maroon}{\text{b}} ijl \mathcolor{gray}{z}}~, &&\hspace{0.3em} {\mathcal{L}}^{\;\!\mathcolor{gray}{t}}_{\;\!\textcolor{Maroon}{\text{b}} ijl \mathcolor{gray}{z}} \hspace{-1em}&&=\hspace{0.2em} ~ \hphantom{-\frac{1}{2!} \epsilon^{\hphantom{ij}k}_{i\mathcolor{gray}{j}}} {\mathcal{L}}^{\;\!\mathcolor{gray}{t}}_{\;\!\textcolor{Maroon}{\text{e}} ijl \mathcolor{gray}{z}} &&\hspace{-0.7em}+ {\mathcal{L}}^{\;\!\mathcolor{gray}{t}}_{\;\!\textcolor{Maroon}{\text{m}} ijl \mathcolor{gray}{z}}~, \label{eq:Le-Lb}
\end{align}
\end{subequations}
然而可以验证,经典多极理论不满足每层奇异项的要求,以最末层次 ${\delta}''_{\mathcolor{gray}{z}}$ 为例,当只考虑到电八/磁四级别(忽略更高阶矩)时:
\begin{align} \label{eq:Lb-zzz}
	\leftindex^{\textcolor{Maroon}{\mathsfit{z}}} \;\! {\mathcal{L}}^{\;\!\mathcolor{gray}{t}}_{\;\!\textcolor{Maroon}{\text{b}} \symup{z} \symup{z} \symup{z} \mathcolor{gray}{z}} = \symup{sgn} \left( \leftindex^{\textcolor{Maroon}{\mathsfit{z}}} \delta''_{\mathcolor{gray}{z}} \right) \mathcolor{gray}{\nabla^t} \leftindex^{\textcolor{Maroon}{\mathsfit{z}}} \;\! {\mathcal{O}}^{\;\!\mathcolor{gray}{t}}_{\;\!\textcolor{Maroon}{\text{b}} \symup{z} \symup{z} \symup{z} \mathcolor{gray}{z}}
\end{align}
无论在介质 \textcolor{Maroon}{1} 中 $\leftindex^{\textcolor{Maroon}{1}} \;\! {\mathcal{L}}^{\;\!\mathcolor{gray}{t}}_{\;\!\textcolor{Maroon}{\text{b}} \symup{z} \symup{z} \symup{z} \mathcolor{gray}{z}} = \mathcolor{gray}{\nabla^t} \leftindex^{\textcolor{Maroon}{\mathsfit{z}}} \;\! {\mathcal{O}}^{\;\!\mathcolor{gray}{t}}_{\;\!\textcolor{Maroon}{\text{b}} \symup{z} \symup{z} \symup{z} \mathcolor{gray}{z}}$ 还是在介质 \textcolor{Maroon}{2} 中 $\leftindex^{\textcolor{Maroon}{2}} \;\! {\mathcal{L}}^{\;\!\mathcolor{gray}{t}}_{\;\!\textcolor{Maroon}{\text{b}} \symup{z} \symup{z} \symup{z} \mathcolor{gray}{z}} = - \mathcolor{gray}{\nabla^t} \leftindex^{\textcolor{Maroon}{\mathsfit{z}}} \;\! {\mathcal{O}}^{\;\!\mathcolor{gray}{t}}_{\;\!\textcolor{Maroon}{\text{b}} \symup{z} \symup{z} \symup{z} \mathcolor{gray}{z}}$,都不等于经典理论的预言:${\mathcal{L}}^{\;\!\mathcolor{gray}{t}}_{\;\!\textcolor{Maroon}{\text{e}} \symup{z} \symup{z} \symup{z} \mathcolor{gray}{z}} = - \mathcolor{gray}{\nabla^t} \leftindex^{\textcolor{Maroon}{\mathsfit{z}}} \;\! {\mathcal{O}}^{\;\!\mathcolor{gray}{t}}_{\;\!\textcolor{Maroon}{\text{b}} \symup{z} \symup{z} \symup{z} \mathcolor{gray}{z}} + \frac{1}{2!}~ \epsilon^{\hphantom{zz}k}_{z\mathcolor{gray}{z}} N^{\;\!\mathcolor{gray}{t}}_{\;\! kz\mathcolor{gray}{z}}$。

%{\mathcal{L}}^{\;\!\mathcolor{gray}{t}}_{\;\!\textcolor{Maroon}{\text{e}} ijk \mathcolor{gray}{z}} \hspace{-1em}&&= - {\mathcal{O}}^{\;\!\mathcolor{gray}{t}}_{\;\!\textcolor{Maroon}{\text{b}} ijk \mathcolor{gray}{z}} \hspace{1em}

在不考虑上述边值关系 \bref{eq:div-e-b-01-delta,eq:div-e-b-01-delta',eq:div-e-b-01-delta''} 的情况下,

而这些额外项只能期望通过。

\bref{eq:p-b,eq:j-b} ,并将所得结果代入,并省略电八/磁四以上的项

然后将它们 5 个代入 \bref{eq:d-pre,eq:h-pre} 并省略电八/磁四以上的项\Footnote{也可以,甚至原则上更应该 代入 \bref{eq:p-b,eq:j-e,eq:j-m,eq:j-b} 中,接着再代入 \bref{eq:div-E,eq:curl-B},但略有点迂回/不够直接,就没这么做。}:
\begin{subequations}
\begin{align}
	\bar{D}^{\;\!\mathcolor{gray}{t}}_{\;\! i\mathcolor{gray}{z}} &= {\symup{\varepsilon}}_0 E^{\;\!\mathcolor{gray}{t}}_{\;\! i\mathcolor{gray}{z}} + P^{\;\!\mathcolor{gray}{t}}_{\;\! i\mathcolor{gray}{z}} - \frac{1}{2!} \mathcolor{gray}{\nabla^j} Q^{\;\!\mathcolor{gray}{t}}_{\;\! ij\mathcolor{gray}{z}} + \frac{1}{3!} \mathcolor{gray}{\nabla^j} \mathcolor{gray}{\nabla^k} O^{\;\!\mathcolor{gray}{t}}_{\;\! ijk\mathcolor{gray}{z}} \label{eq:d} \\ 
	&= {\symup{\varepsilon}}_0 E^{\;\!\mathcolor{gray}{t}}_{\;\! i\mathcolor{gray}{z}} + \leftindex_{\textcolor{Maroon}{\mathsfit{z}}} {\mathbb{1}}_{\mathcolor{gray}{z}} \leftindex^{\textcolor{Maroon}{\mathsfit{z}}} P^{\;\!\mathcolor{gray}{t}}_{\;\! i\mathcolor{gray}{z}} - \frac{1}{2!} \mathcolor{gray}{\nabla^j} \left( \leftindex_{\textcolor{Maroon}{\mathsfit{z}}} {\mathbb{1}}_{\mathcolor{gray}{z}} \leftindex^{\textcolor{Maroon}{\mathsfit{z}}} Q^{\;\!\mathcolor{gray}{t}}_{\;\! ij\mathcolor{gray}{z}} \right) + \frac{1}{3!} \mathcolor{gray}{\nabla^j} \mathcolor{gray}{\nabla^k} \left( \leftindex_{\textcolor{Maroon}{\mathsfit{z}}} {\mathbb{1}}_{\mathcolor{gray}{z}} \leftindex^{\textcolor{Maroon}{\mathsfit{z}}} O^{\;\!\mathcolor{gray}{t}}_{\;\! ijk\mathcolor{gray}{z}} \right) ~, \label{eq:d-01} \\
	\bar{H}^{\;\!\mathcolor{gray}{t}}_{\;\! i\mathcolor{gray}{z}} &= {\symup{\mu}}_0^{-1} B^{\;\!\mathcolor{gray}{t}}_{\;\! i\mathcolor{gray}{z}} - M^{\;\!\mathcolor{gray}{t}}_{\;\! i\mathcolor{gray}{z}} + \frac{1}{2!} \mathcolor{gray}{\nabla^j} N^{\;\!\mathcolor{gray}{t}}_{\;\! ij\mathcolor{gray}{z}} \label{eq:h} \\
	&= {\symup{\mu}}_0^{-1} B^{\;\!\mathcolor{gray}{t}}_{\;\! i\mathcolor{gray}{z}} - \leftindex_{\textcolor{Maroon}{\mathsfit{z}}} {\mathbb{1}}_{\mathcolor{gray}{z}} \leftindex^{\textcolor{Maroon}{\mathsfit{z}}} M^{\;\!\mathcolor{gray}{t}}_{\;\! i\mathcolor{gray}{z}} + \frac{1}{2!} \mathcolor{gray}{\nabla^j} \left( \leftindex_{\textcolor{Maroon}{\mathsfit{z}}} {\mathbb{1}}_{\mathcolor{gray}{z}} \leftindex^{\textcolor{Maroon}{\mathsfit{z}}} N^{\;\!\mathcolor{gray}{t}}_{\;\! ij\mathcolor{gray}{z}} \right) ~. \label{eq:h-01}
\end{align}
\end{subequations}
接着将上述 2 者分别代入 \bref{eq:curl-EK,eq:curl-H},得
\begin{subequations} \label{eq:curl-EH}
%	\abovedisplayskip=0pt
	%\belowdisplayskip=0pt
	\small
\begin{align}
	\epsilon^{\hphantom{ij}k}_{i\mathcolor{gray}{j}} \mathcolor{gray}{\nabla^j} E^{\;\!\mathcolor{gray}{t}}_{\;\! k\mathcolor{gray}{z}} + \mathcolor{gray}{\nabla^t} B^{\;\!\mathcolor{gray}{t}}_{\;\! i\mathcolor{gray}{z}} &= 0~, \label{eq:curl-E'} \\
	\epsilon^{\hphantom{ij}k}_{i\mathcolor{gray}{j}} \mathcolor{gray}{\nabla^j} \left[ {\symup{\mu}}_0^{-1} B^{\;\!\mathcolor{gray}{t}}_{\;\! k\mathcolor{gray}{z}} - \leftindex_{\textcolor{Maroon}{\mathsfit{z}}} {\mathbb{1}}_{\mathcolor{gray}{z}} \leftindex^{\textcolor{Maroon}{\mathsfit{z}}} M^{\;\!\mathcolor{gray}{t}}_{\;\! k\mathcolor{gray}{z}} + \frac{1}{2!} \mathcolor{gray}{\nabla^l} \left( \leftindex_{\textcolor{Maroon}{\mathsfit{z}}} {\mathbb{1}}_{\mathcolor{gray}{z}} \leftindex^{\textcolor{Maroon}{\mathsfit{z}}} N^{\;\!\mathcolor{gray}{t}}_{\;\! kl\mathcolor{gray}{z}} \right) \right] & \label{eq:curl-H'} \\ - \mathcolor{gray}{\nabla^t} \left[ {\symup{\varepsilon}}_0 E^{\;\!\mathcolor{gray}{t}}_{\;\! i\mathcolor{gray}{z}} + \leftindex_{\textcolor{Maroon}{\mathsfit{z}}} {\mathbb{1}}_{\mathcolor{gray}{z}} \leftindex^{\textcolor{Maroon}{\mathsfit{z}}} P^{\;\!\mathcolor{gray}{t}}_{\;\! i\mathcolor{gray}{z}} - \frac{1}{2!} \mathcolor{gray}{\nabla^j} \left( \leftindex_{\textcolor{Maroon}{\mathsfit{z}}} {\mathbb{1}}_{\mathcolor{gray}{z}} \leftindex^{\textcolor{Maroon}{\mathsfit{z}}} Q^{\;\!\mathcolor{gray}{t}}_{\;\! ij\mathcolor{gray}{z}} \right) + \frac{1}{3!} \mathcolor{gray}{\nabla^j} \mathcolor{gray}{\nabla^k} \left( \leftindex_{\textcolor{Maroon}{\mathsfit{z}}} {\mathbb{1}}_{\mathcolor{gray}{z}} \leftindex^{\textcolor{Maroon}{\mathsfit{z}}} O^{\;\!\mathcolor{gray}{t}}_{\;\! ijk\mathcolor{gray}{z}} \right) \right] &= J^{\;\!\mathcolor{gray}{t}}_{\;\! \textcolor{Maroon}{\text{f}} i\mathcolor{gray}{z}}~, \\
	\mathcolor{gray}{\nabla^i} \left[ {\symup{\varepsilon}}_0 E^{\;\!\mathcolor{gray}{t}}_{\;\! i\mathcolor{gray}{z}} + \leftindex_{\textcolor{Maroon}{\mathsfit{z}}} {\mathbb{1}}_{\mathcolor{gray}{z}} \leftindex^{\textcolor{Maroon}{\mathsfit{z}}} P^{\;\!\mathcolor{gray}{t}}_{\;\! i\mathcolor{gray}{z}} - \frac{1}{2!} \mathcolor{gray}{\nabla^j} \left( \leftindex_{\textcolor{Maroon}{\mathsfit{z}}} {\mathbb{1}}_{\mathcolor{gray}{z}} \leftindex^{\textcolor{Maroon}{\mathsfit{z}}} Q^{\;\!\mathcolor{gray}{t}}_{\;\! ij\mathcolor{gray}{z}} \right) + \frac{1}{3!} \mathcolor{gray}{\nabla^j} \mathcolor{gray}{\nabla^k} \left( \leftindex_{\textcolor{Maroon}{\mathsfit{z}}} {\mathbb{1}}_{\mathcolor{gray}{z}} \leftindex^{\textcolor{Maroon}{\mathsfit{z}}} O^{\;\!\mathcolor{gray}{t}}_{\;\! ijk\mathcolor{gray}{z}} \right) \right] &= {\rho}^{\;\!\mathcolor{gray}{t}}_{\;\!\textcolor{Maroon}{\text{f}}\mathcolor{gray}{z}}~, \label{eq:div-D'} \\
	\mathcolor{gray}{\nabla^i} B^{\;\!\mathcolor{gray}{t}}_{\;\! i\mathcolor{gray}{z}} &= 0~, \label{eq:div-B'} \\
	\mathcolor{gray}{\nabla^i} J^{\;\!\mathcolor{gray}{t}}_{\;\!\textcolor{Maroon}{\text{f}} i\mathcolor{gray}{z}} + \mathcolor{gray}{\nabla^t} {\rho}^{\;\!\mathcolor{gray}{t}}_{\;\!\textcolor{Maroon}{\text{f}}\mathcolor{gray}{z}} &= 0~. \label{eq:div-e-f'}
\end{align}
\end{subequations}
注意到,在空域上对 

而这些额外项只能期望通过 $\bar{E}^{\;\!\mathcolor{gray}{t}}_{\;\!\mathcolor{gray}{z}},\bar{B}^{\;\!\mathcolor{gray}{t}}_{\;\!\mathcolor{gray}{z}}$ 中也含有的对应项来消除,以满足 \bref{eq:div-D',eq:curl-H'}。



\marginLeft[-2.5em]{ssec:EB-boundary}\subsection{$\bar{E},\bar{B}$ 表面场、$\bar{E},\bar{B}$ 边界条件}\label{ssec:EB-boundary}

\marginLeft[-2.5em]{ssec:PMQN-nonlinear}\subsection{$\bar{P},\bar{M}$ 非线性、$\bar{\bar{Q}},\bar{\bar{N}}$ 的非线性}\label{ssec:PMQN-nonlinear}

分别给磁感应场$\bar{B}^{\;\!\mathcolor{gray}{t}}_{\;\!\mathcolor{gray}{z}}$(而非磁场$\bar{H}^{\;\!\mathcolor{gray}{t}}_{\;\!\mathcolor{gray}{z}}$),自由电流源$\bar{J}^{\;\!\mathcolor{gray}{t}}_{\;\!\textcolor{Maroon}{\text{f}}\mathcolor{gray}{z}}$ 以及电位移场$\bar{D}^{\;\!\mathcolor{gray}{t}}_{\;\!\mathcolor{gray}{z}}$,定义了如下 3 个本构关系(\textcolor{Maroon}{\text{Constitutive Relation}} = \textcolor{Maroon}{\text{CR}})。

其一,磁场 $\bar{H}^{\;\!\mathcolor{gray}{t}}_{\;\!\mathcolor{gray}{z}}$ 的本构关系,考虑$\bar{M}^{\;\!\mathcolor{gray}{t}}_{\;\!\mathcolor{gray}{z}}$\Footnote{磁化强度。其在微观上来源于:分子电流(电子轨道运动)产生的磁矩和电子自旋磁矩的矢量和\cite{nelsonLagrangianTreatmentMagnetic1994},细究还将有电子与原子核、电子的自旋轨道耦合、电子电子间相互作用(多电子产生原子磁矩)、原子与原子间(晶体场)的相互作用\cite{chen-zhuChenZhuxieUndergraduate_courses2024}。磁的起源看上去可以是纯电的\cite{lakhtakiaGenesisPostConstraint2004}。}仅由磁偶极矩\Footnote{即只考虑最低阶磁多极矩 = 不考虑磁四极矩及以上\cite{nelsonLagrangianTreatmentMagnetic1994},因为对于受到电磁场的影响后,反过来产生电磁场的电子而言,其受到的电场力是洛伦兹力的c$\big/v$倍\cite{boydNonlinearOptics2019}。此外,(磁/电)多极矩与(磁/电)非线性是互相独立的——即任何阶的多极矩,都有自己的线性项和非线性项\cite{chen-zhuChenZhuxieUndergraduate_courses2024},这些项与其它阶多极矩的任何项都无关。}贡献时,定义为
\begin{subequations} \label{eq:cr-b}
\begin{align}
	\textcolor{Maroon}{\text{CR for magnetism}}\text{:}&\hspace{0.5em} \bar{B}^{\;\!\mathcolor{gray}{t}}_{\;\!\mathcolor{gray}{z}} \hspace{-0.7em} &&={\symup{\mu}}_0 \left( \bar{H}^{\;\!\mathcolor{gray}{t}}_{\;\!\mathcolor{gray}{z}} + \bar{M}^{\;\!\mathcolor{gray}{t}}_{\;\!\mathcolor{gray}{z}} \right) = {\symup{\mu}}_0 \left\{ \bar{\bar{\delta}}^{\;\!\mathcolor{gray}{t}}~\widetilde *~\bar{H}^{\;\!\mathcolor{gray}{t}}_{\;\!\mathcolor{gray}{z}} + \bar{M}^{\;\!\mathcolor{gray}{t}}_{\;\!\mathcolor{gray}{z}} \right\} \label{cr-b1} \\ 
	& &&\xrightarrow[]{\bar{M}^{\;\!\mathcolor{gray}{t}}_{\;\!\mathcolor{gray}{z}} = \bar{M}^{\;\!\textcolor{Maroon}{\text{(1)}} \mathcolor{gray}{t}}_{\;\!\mathcolor{gray}{z}} + \bar{M}^{\;\!\textcolor{Maroon}{\text{NL}}, \mathcolor{gray}{t}}_{\;\!\mathcolor{gray}{z}}} {\symup{\mu}}_0 \left\{ \left[ \bar{\bar{\delta}}^{\;\!\mathcolor{gray}{t}}~\widetilde *~\bar{H}^{\;\!\mathcolor{gray}{t}}_{\;\!\mathcolor{gray}{z}} + \bar{M}^{\;\!\textcolor{Maroon}{\text{(1)}} \mathcolor{gray}{t}}_{\;\!\mathcolor{gray}{z}} \right] + \bar{M}^{\;\!\textcolor{Maroon}{\text{NL}}, \mathcolor{gray}{t}}_{\;\!\mathcolor{gray}{z}} \right\} \label{cr-b2} \\ 
	& &&\xrightarrow[\displaystyle{ \bar{\bar{\mu}}^{\;\!\textcolor{Maroon}{\text{(1)}} \mathcolor{gray}{t}}_{\;\!\textcolor{Maroon}{\text{r}}\mathcolor{gray}{z}} := \bar{\bar{\delta}}^{\;\!\mathcolor{gray}{t}} + \bar{\bar{\chi}}^{\;\!\textcolor{Maroon}{\text{(1)}}\mathcolor{gray}{t}}_{\;\!\textcolor{Maroon}{\text{m}} \mathcolor{gray}{z}}}]{\displaystyle{\bar{M}^{\;\!\textcolor{Maroon}{\text{(1)}} \mathcolor{gray}{t}}_{\;\!\mathcolor{gray}{z}} := \bar{\bar{\chi}}^{\;\!\textcolor{Maroon}{\text{(1)}}\mathcolor{gray}{t}}_{\;\!\textcolor{Maroon}{\text{m}} \mathcolor{gray}{z}} ~\widetilde *~\bar{H}^{\;\!\mathcolor{gray}{t}}_{\;\!\mathcolor{gray}{z}}}} {\symup{\mu}}_0 \left\{ \bar{\bar{\mu}}^{\;\!\textcolor{Maroon}{\text{(1)}} \mathcolor{gray}{t}}_{\;\!\textcolor{Maroon}{\text{r}}\mathcolor{gray}{z}}~\widetilde *~\bar{H}^{\;\!\mathcolor{gray}{t}}_{\;\!\mathcolor{gray}{z}} + \bar{M}^{\;\!\textcolor{Maroon}{\text{NL}}, \mathcolor{gray}{t}}_{\;\!\mathcolor{gray}{z}} \right\} \label{cr-b3} \\ 
	& &&= \bar{\bar{\mu}}^{\;\!\textcolor{Maroon}{\text{(1)}} \mathcolor{gray}{t}}_{\;\!\mathcolor{gray}{z}}~\widetilde *~\bar{H}^{\;\!\mathcolor{gray}{t}}_{\;\!\mathcolor{gray}{z}} + {\symup{\mu}}_0 \bar{M}^{\;\!\textcolor{Maroon}{\text{NL}}, \mathcolor{gray}{t}}_{\;\!\mathcolor{gray}{z}} =: \bar{B}^{\;\!\textcolor{Maroon}{\text{(1)}} \mathcolor{gray}{t}}_{\;\!\mathcolor{gray}{z}} + \bar{B}^{\;\!\textcolor{Maroon}{\text{NL}}, \mathcolor{gray}{t}}_{\;\!\mathcolor{gray}{z}}~, \label{cr-b4}
\end{align}
\end{subequations}
其中,磁通量密度场 $\bar{B}^{\;\!\mathcolor{gray}{t}}_{\;\!\mathcolor{gray}{z}}$(直接/显示地)关于磁场 $\bar{H}^{\;\!\mathcolor{gray}{t}}_{\;\!\mathcolor{gray}{z}}$\Footnote{磁非线性,如郎之万顺磁性理论\cite{chen-zhuChenZhuxieUndergraduate_courses2024}中 $M \propto$ 郎之万函数 $\mathcal{L} \left( \alpha \right) = \coth \left( \alpha \right) - 1 / \alpha$(其中 $\alpha \propto H_{\textcolor{Maroon}{\text{ex}}}$)或其量子化修正之布里渊函数,铁磁体\cite{chen-zhuChenZhuxieUndergraduate_courses2024}或超导体\cite{wenBriefIntroductionFlux2021}中的磁滞现象等(每个时刻$t$,这些场量都是准静态$\Omega \to 0$的)。}、电场 $\bar{E}^{\;\!\mathcolor{gray}{t}}_{\;\!\mathcolor{gray}{z}}$\Footnote{双各向异性,其中的电$\to$磁耦合(如果 $\bar{B}^{\;\!\mathcolor{gray}{t}}_{\;\!\mathcolor{gray}{z}}$ 中的该部分只是 $\bar{E}^{\;\!\mathcolor{gray}{t}}_{\;\!\mathcolor{gray}{z}}$ 的线性函数,则也可归结到线性项中)。}、应力 $\bar{T}^{\;\!\mathcolor{gray}{t}}_{\;\!\mathcolor{gray}{z}}$\Footnote{正逆压磁/磁致伸缩/磁弹效应(这里未作区分)。}等其他场量(即含空 $\mathcolor{gray}{\bar{r}}$ 的物理量)\Footnote{$\bar{B}^{\;\!\mathcolor{gray}{t}}_{\;\!\mathcolor{gray}{z}},\bar{M}^{\;\!\mathcolor{gray}{t}}_{\;\!\mathcolor{gray}{z}}$ 以及各阶 $\bar{\bar{\mu}}^{\;\!\textcolor{Maroon}{\text{(1)}} \mathcolor{gray}{t}}_{\;\!\mathcolor{gray}{z}},\bar{\bar{\bar{\mu}}}^{\;\!\textcolor{Maroon}{\text{(2)}} \mathcolor{gray}{t}}_{\;\!\mathcolor{gray}{z}},\cdots$ 已经是关于温度$T$、波长$\lambda$(或 角频率$\omega$、时间$t$)等(非)场量的函数。}的非线性函数项,悉数包含在 $\bar{B}^{\;\!\textcolor{Maroon}{\text{NL}}, \mathcolor{gray}{t}}_{\;\!\mathcolor{gray}{z}} = {\symup{\mu}}_0 \bar{M}^{\;\!\textcolor{Maroon}{\text{NL}}, \mathcolor{gray}{t}}_{\;\!\mathcolor{gray}{z}}$ 内;剩余的线性项,放在 $\bar{B}^{\;\!\textcolor{Maroon}{\text{(1)}} \mathcolor{gray}{t}}_{\;\!\mathcolor{gray}{z}} = \bar{\bar{\mu}}^{\;\!\textcolor{Maroon}{\text{(1)}} \mathcolor{gray}{t}}_{\;\!\mathcolor{gray}{z}}~\widetilde *~\bar{H}^{\;\!\mathcolor{gray}{t}}_{\;\!\mathcolor{gray}{z}}$ 中。

其二,自由电流源$\bar{J}^{\;\!\mathcolor{gray}{t}}_{\;\!\textcolor{Maroon}{\text{f}}\mathcolor{gray}{z}}$的本构关系,包含欧姆定律的线性部分(漂移项) $\bar{J}^{\;\!\textcolor{Maroon}{\text{(1)}} \mathcolor{gray}{t}}_{\;\!\textcolor{Maroon}{\text{f}}\mathcolor{gray}{z}} = \bar{\bar{\sigma}}^{\;\!\textcolor{Maroon}{\text{(1)}}\mathcolor{gray}{t}}_{\;\!\mathcolor{gray}{z}}~\widetilde *~\bar{E}^{\;\!\mathcolor{gray}{t}}_{\;\!\mathcolor{gray}{z}}$\Footnote{可以由 Drude 模型描述,定量解释一阶电导率$\bar{\bar{\sigma}}^{\;\!\textcolor{Maroon}{\text{(1)}}\mathcolor{gray}{t}}_{\;\!\mathcolor{gray}{z}}$的起源。},及$\bar{J}^{\;\!\mathcolor{gray}{t}}_{\;\!\textcolor{Maroon}{\text{f}}\mathcolor{gray}{z}}$分别关于电场 $\bar{E}^{\;\!\mathcolor{gray}{t}}_{\;\!\mathcolor{gray}{z}}$\Footnote{欧姆定律中的电非线性部分,比如二/三极管的伏安特性曲线\cite{chen-zhuChenZhuxieUndergraduate_courses2024}(尽管输入/输出or自/因变量,即$\bar{E}^{\;\!\mathcolor{gray}{t}}_{\;\!\mathcolor{gray}{z}}$和$\bar{J}^{\;\!\mathcolor{gray}{t}}_{\;\!\textcolor{Maroon}{\text{f}}\mathcolor{gray}{z}}$,一般均在直流或低频$\Omega$,非交流且不在光波段 opt)。}、磁感应场$\bar{B}^{\;\!\mathcolor{gray}{t}}_{\;\!\mathcolor{gray}{z}}$\Footnote{磁感应场$\bar{B}^{\;\!\mathcolor{gray}{t}}_{\;\!\mathcolor{gray}{z}}$所带来的(电场力以外的)洛伦兹力$\left( \bar{J}^{\;\!\mathcolor{gray}{t}}_{\;\!\textcolor{Maroon}{\text{f}}\mathcolor{gray}{z}} + \dot{\bar{P}}^{\;\!\mathcolor{gray}{t}}_{\;\!\mathcolor{gray}{z}} + \mathcolor{gray}{\bar{\nabla} \times} \bar{M}^{\;\!\mathcolor{gray}{t}}_{\;\!\mathcolor{gray}{z}} \right) \times \bar{B}^{\;\!\mathcolor{gray}{t}}_{\;\!\mathcolor{gray}{z}}$\cite{mackayElectromagneticAnisotropyBianisotropy2019,chen-zhuChenZhuxieUndergraduate_courses2024},会影响导/价带电子的运动(速度)$\bar{v}^{\;\!\mathcolor{gray}{t}}_{\;\!\textcolor{Maroon}{\text{f}}\mathcolor{gray}{z}}$,进而全局地影响自由电流$\bar{J}^{\;\!\mathcolor{gray}{t}}_{\;\!\textcolor{Maroon}{\text{f}}\mathcolor{gray}{z}} = {\rho}^{\;\!\mathcolor{gray}{t}}_{\;\!\textcolor{Maroon}{\text{f}}\mathcolor{gray}{z}} \bar{v}^{\;\!\mathcolor{gray}{t}}_{\;\!\textcolor{Maroon}{\text{f}}\mathcolor{gray}{z}}$和(束缚)电(偶)极化强度$\bar{P}^{\;\!\mathcolor{gray}{t}}_{\;\!\mathcolor{gray}{z}}$,包括它们的线性和非线性项\cite{boydNonlinearOptics2019}。对于强场/超快非线性光学,相对论效应使得电磁场是个统一的整体,动生(而不仅是外加)的$\bar{B}^{\;\!\mathcolor{gray}{t}}_{\;\!\mathcolor{gray}{z}}$还将带来额外的影响。磁场$\bar{H}^{\;\!\mathcolor{gray}{t}}_{\;\!\mathcolor{gray}{z}}$对分子/磁化电流体密度$\mathcolor{gray}{\bar{\nabla} \times} \bar{M}^{\;\!\mathcolor{gray}{t}}_{\;\!\mathcolor{gray}{z}}$产生的影响已包含在$\bar{M}^{\;\!\mathcolor{gray}{t}}_{\;\!\mathcolor{gray}{z}}$中了。}、导带电子浓度(数密度)梯度场$\mathcolor{gray}{\bar{\nabla}} {\rho}^{\;\!\mathcolor{gray}{t}}_{\;\!\textcolor{Maroon}{\text{f}}\mathcolor{gray}{z}}$\Footnote{在光折变效应中,作为$\bar{J}^{\;\!\mathcolor{gray}{t}}_{\;\!\textcolor{Maroon}{\text{f}}\mathcolor{gray}{z}}$中的扩散项\cite{boydNonlinearOptics2019}。${\rho}^{\;\!\mathcolor{gray}{t}}_{\;\!\textcolor{Maroon}{\text{f}}\mathcolor{gray}{z}}, \bar{J}^{\;\!\mathcolor{gray}{t}}_{\;\!\textcolor{Maroon}{\text{f}}\mathcolor{gray}{z}}$之间还应满足\bref{eq:Div-e-f}以及$\bar{J}^{\;\!\mathcolor{gray}{t}}_{\;\!\textcolor{Maroon}{\text{f}}\mathcolor{gray}{z}} = {\rho}^{\;\!\mathcolor{gray}{t}}_{\;\!\textcolor{Maroon}{\text{f}}\mathcolor{gray}{z}} \bar{v}^{\;\!\mathcolor{gray}{t}}_{\;\!\textcolor{Maroon}{\text{f}}\mathcolor{gray}{z}}$\cite{chen-zhuChenZhuxieUndergraduate_courses2024}。}和光伏电流场$\propto \lvert \bar{E}^{\;\!\mathcolor{gray}{t}}_{\;\!\mathcolor{gray}{z}} \rvert^2 \hat{c}$\Footnote{与光电导效应并列,属于内光电效应;也可能在光折变效应的$\bar{J}^{\;\!\mathcolor{gray}{t}}_{\;\!\textcolor{Maroon}{\text{f}}\mathcolor{gray}{z}}$中扮演一份角色,特别是沿着一些各向异性晶体的光轴$\hat{c}$产生电势差和内建电场\cite{boydNonlinearOptics2019}(尽管一般也只影响直流或低频$\Omega$的$\bar{J}^{\;\!\mathcolor{gray}{t}}_{\;\!\textcolor{Maroon}{\text{f}}\mathcolor{gray}{z}}$;但$\bar{J}^{\;\!\mathcolor{gray}{t}}_{\;\!\textcolor{Maroon}{\text{f}}\mathcolor{gray}{z}}$会通过影响光波段的介电常数,进而影响光波段的光强$\lvert \bar{E}^{\;\!\mathcolor{gray}{t}}_{\;\!\mathcolor{gray}{z}} \rvert^2$及$\bar{J}^{\;\!\mathcolor{gray}{t}}_{\;\!\textcolor{Maroon}{\text{f}}\mathcolor{gray}{z}}$自己的重新分布);该二阶的带耦合的非线性,看上去很像非线性极化率$\bar{P}^{\;\!\textcolor{Maroon}{\text{(2)}} \mathcolor{gray}{t}}_{\;\!\mathcolor{gray}{z}}$中的光整流项,但其频率比 THz 低,且只服务于自由电流。—— 以至该项可作为差频合并至$\bar{J}^{\;\!\mathcolor{gray}{t}}_{\;\!\textcolor{Maroon}{\text{f}}\mathcolor{gray}{z}}$关于$\bar{E}^{\;\!\mathcolor{gray}{t}}_{\;\!\mathcolor{gray}{z}}$的二阶非线性$\bar{J}^{\;\!\textcolor{Maroon}{\text{(2)}} \mathcolor{gray}{t}}_{\;\!\textcolor{Maroon}{\text{f}}\mathcolor{gray}{z}}$中去?}等其他场量的非线性项 $\bar{J}^{\;\!\textcolor{Maroon}{\text{NL}}, \mathcolor{gray}{t}}_{\;\!\textcolor{Maroon}{\text{f}}\mathcolor{gray}{z}}$:
\begin{align} \label{eq:cr-j}
	\textcolor{Maroon}{\text{Ohm's law}}\text{:}\hspace{0.5em} \bar{J}^{\;\!\mathcolor{gray}{t}}_{\;\!\textcolor{Maroon}{\text{f}}\mathcolor{gray}{z}} = \bar{\bar{\sigma}}^{\;\!\textcolor{Maroon}{\text{(1)}}\mathcolor{gray}{t}}_{\;\!\mathcolor{gray}{z}}~\widetilde *~\bar{E}^{\;\!\mathcolor{gray}{t}}_{\;\!\mathcolor{gray}{z}} + \bar{J}^{\;\!\textcolor{Maroon}{\text{NL}}, \mathcolor{gray}{t}}_{\;\!\textcolor{Maroon}{\text{f}}\mathcolor{gray}{z}} =: \bar{J}^{\;\!\textcolor{Maroon}{\text{(1)}} \mathcolor{gray}{t}}_{\;\!\textcolor{Maroon}{\text{f}}\mathcolor{gray}{z}} + \bar{J}^{\;\!\textcolor{Maroon}{\text{NL}}, \mathcolor{gray}{t}}_{\;\!\textcolor{Maroon}{\text{f}}\mathcolor{gray}{z}}~,
\end{align}

其三,电位移场$\bar{D}^{\;\!\mathcolor{gray}{t}}_{\;\!\mathcolor{gray}{z}}$的本构关系,当$\bar{P}^{\;\!\mathcolor{gray}{t}}_{\;\!\mathcolor{gray}{z}}$只由电偶极矩\Footnote{不考虑电四极矩及以上。但电四极化强度场$\bar{\bar{Q}}^{\;\!\mathcolor{gray}{t}}_{\;\!\mathcolor{gray}{z}}$(的等效电偶极化强度场$\bar{P}^{\;\!\mathcolor{gray}{t}}_{\;\!\textcolor{Maroon}{\text{Q}}\mathcolor{gray}{z}} = - \mathcolor{gray}{\bar{\nabla} \cdot} \bar{\bar{Q}}^{\;\!\mathcolor{gray}{t}}_{\;\!\mathcolor{gray}{z}}$)\cite{chen-zhuChenZhuxieUndergraduate_courses2024}在有些效应中不可忽视且起关键作用:如其对线性晶体光学中的光学活性的贡献\cite{nelsonDerivingTransmissionReflection1995},以及非线性光学中基于$\bar{\bar{Q}}^{\;\!\mathcolor{gray}{t}}_{\;\!\mathcolor{gray}{z}}$的二阶和频\cite{bethuneOpticalQuadrupoleSumfrequency1976}。电四极子对光与物质相互作用的贡献,还会打破$\bar{D}^{\;\!\mathcolor{gray}{t}}_{\;\!\mathcolor{gray}{z}}$法向连续和$\bar{H}^{\;\!\mathcolor{gray}{t}}_{\;\!\mathcolor{gray}{z}}$切向连续边界条件,并与洛伦兹力的定义、(由唯二的无源齐次\cite{lakhtakiaGenesisPostConstraint2004}微分方程\bref{eq:Curl-EK,eq:Div-Bk}导出的)电磁场标/矢势\cite{chen-zhuChenZhuxieUndergraduate_courses2024}等一起,使$\bar{E}^{\;\!\mathcolor{gray}{t}}_{\;\!\mathcolor{gray}{z}},\bar{B}^{\;\!\mathcolor{gray}{t}}_{\;\!\mathcolor{gray}{z}}$而不是$\bar{E}^{\;\!\mathcolor{gray}{t}}_{\;\!\mathcolor{gray}{z}},\bar{H}^{\;\!\mathcolor{gray}{t}}_{\;\!\mathcolor{gray}{z}}$成为基本场\cite{nelsonDerivingTransmissionReflection1995},对应地,坡印亭矢量也需要修正为$\bar{E}^{\;\!\mathcolor{gray}{t}}_{\;\!\mathcolor{gray}{z}} \times \bar{B}^{\;\!\mathcolor{gray}{t}}_{\;\!\mathcolor{gray}{z}} \big/ {\symup{\mu}}_0$\cite{nelsonGeneralizingPoyntingVector1996,loudonPropagationElectromagneticEnergy1997,richterPoyntingsTheoremEnergy2008}而不是$\bar{E}^{\;\!\mathcolor{gray}{t}}_{\;\!\mathcolor{gray}{z}} \times \bar{H}^{\;\!\mathcolor{gray}{t}}_{\;\!\mathcolor{gray}{z}}$。}构成时,定义为
\begin{subequations} \label{eq:cr-d}
\begin{align}
	\textcolor{Maroon}{\text{CR for electricity}}\text{:}&\hspace{0.5em} \bar{D}^{\;\!\mathcolor{gray}{t}}_{\;\!\mathcolor{gray}{z}} \hspace{-2.0em} &&= {\symup{\varepsilon}}_0 \bar{E}^{\;\!\mathcolor{gray}{t}}_{\;\!\mathcolor{gray}{z}} + \bar{P}^{\;\!\mathcolor{gray}{t}}_{\;\!\mathcolor{gray}{z}} = {\symup{\varepsilon}}_0 \bar{\bar{\delta}}^{\;\!\mathcolor{gray}{t}}~\widetilde *~\bar{E}^{\;\!\mathcolor{gray}{t}}_{\;\!\mathcolor{gray}{z}} + \bar{P}^{\;\!\mathcolor{gray}{t}}_{\;\!\mathcolor{gray}{z}} \label{cr-d1} \\ 
	& &&\xrightarrow[]{\bar{P}^{\;\!\mathcolor{gray}{t}}_{\;\!\mathcolor{gray}{z}} = \bar{P}^{\;\!\textcolor{Maroon}{\text{(1)}} \mathcolor{gray}{t}}_{\;\!\mathcolor{gray}{z}} + \bar{P}^{\;\!\textcolor{Maroon}{\text{NL}}, \mathcolor{gray}{t}}_{\;\!\mathcolor{gray}{z}} + } \left[ {\symup{\varepsilon}}_0 \bar{\bar{\delta}}^{\;\!\mathcolor{gray}{t}}~\widetilde *~\bar{E}^{\;\!\mathcolor{gray}{t}}_{\;\!\mathcolor{gray}{z}} + \bar{P}^{\;\!\textcolor{Maroon}{\text{(1)}} \mathcolor{gray}{t}}_{\;\!\mathcolor{gray}{z}} \right] + \bar{P}^{\;\!\textcolor{Maroon}{\text{NL}}, \mathcolor{gray}{t}}_{\;\!\mathcolor{gray}{z}} \label{cr-d2} \\ 
	& &&\xrightarrow[\displaystyle{ \bar{\bar{\varepsilon}}^{\;\!\textcolor{Maroon}{\text{(1)}} \mathcolor{gray}{t}}_{\;\!\textcolor{Maroon}{\text{r}}\mathcolor{gray}{z}} := \bar{\bar{\delta}}^{\;\!\mathcolor{gray}{t}} + \bar{\bar{\chi}}^{\;\!\textcolor{Maroon}{\text{(1)}}\mathcolor{gray}{t}}_{\;\!\textcolor{Maroon}{\text{f}} \mathcolor{gray}{z}}}]{\displaystyle{\bar{P}^{\;\!\textcolor{Maroon}{\text{(1)}} \mathcolor{gray}{t}}_{\;\!\mathcolor{gray}{z}} := \bar{\bar{\chi}}^{\;\!\textcolor{Maroon}{\text{(1)}}\mathcolor{gray}{t}}_{\;\!\textcolor{Maroon}{\text{f}} \mathcolor{gray}{z}} ~\widetilde *~\bar{E}^{\;\!\mathcolor{gray}{t}}_{\;\!\mathcolor{gray}{z}}}} {\symup{\varepsilon}}_0 \bar{\bar{\varepsilon}}^{\;\!\textcolor{Maroon}{\text{(1)}} \mathcolor{gray}{t}}_{\;\!\textcolor{Maroon}{\text{r}}\mathcolor{gray}{z}}~\widetilde *~\bar{E}^{\;\!\mathcolor{gray}{t}}_{\;\!\mathcolor{gray}{z}} + \bar{P}^{\;\!\textcolor{Maroon}{\text{NL}}, \mathcolor{gray}{t}}_{\;\!\mathcolor{gray}{z}} \label{cr-d3} \\ 
	& &&= \bar{\bar{\varepsilon}}^{\;\!\textcolor{Maroon}{\text{(1)}} \mathcolor{gray}{t}}_{\;\!\mathcolor{gray}{z}}~\widetilde *~\bar{E}^{\;\!\mathcolor{gray}{t}}_{\;\!\mathcolor{gray}{z}} + \bar{P}^{\;\!\textcolor{Maroon}{\text{NL}}, \mathcolor{gray}{t}}_{\;\!\mathcolor{gray}{z}} =: \bar{D}^{\;\!\textcolor{Maroon}{\text{(1)}} \mathcolor{gray}{t}}_{\;\!\mathcolor{gray}{z}} + \bar{D}^{\;\!\textcolor{Maroon}{\text{NL}}, \mathcolor{gray}{t}}_{\;\!\mathcolor{gray}{z}}~, \label{cr-d4}
\end{align}
\end{subequations}
关于其组成成分,电位移场 $\bar{D}^{\;\!\mathcolor{gray}{t}}_{\;\!\mathcolor{gray}{z}}$(直接/显示地)关于电场 $\bar{E}^{\;\!\mathcolor{gray}{t}}_{\;\!\mathcolor{gray}{z}}$\Footnote{电非线性,包括高频段的(非)共振非线性、低频低温\cite{lakhtakiaGenesisPostConstraint2004}段的铁电体/畴的电滞现象等。}、磁场 $\bar{H}^{\;\!\mathcolor{gray}{t}}_{\;\!\mathcolor{gray}{z}}$\Footnote{双各向异性中的磁$\to$电耦合(如果 $\bar{D}^{\;\!\mathcolor{gray}{t}}_{\;\!\mathcolor{gray}{z}}$ 中的该部分只是 $\bar{H}^{\;\!\mathcolor{gray}{t}}_{\;\!\mathcolor{gray}{z}}$ 的线性函数,则也可归结到线性项中)。}、应力 $\bar{T}^{\;\!\mathcolor{gray}{t}}_{\;\!\mathcolor{gray}{z}}$\Footnote{正逆压磁/磁致伸缩/磁弹效应(这里未作区分)。}等其他场量的非线性函数项,均由 $\bar{D}^{\;\!\textcolor{Maroon}{\text{NL}}, \mathcolor{gray}{t}}_{\;\!\mathcolor{gray}{z}} = {\symup{\mu}}_0 \bar{M}^{\;\!\textcolor{Maroon}{\text{NL}}, \mathcolor{gray}{t}}_{\;\!\mathcolor{gray}{z}}$ 贡献;剩余的线性项,由 $\bar{B}^{\;\!\textcolor{Maroon}{\text{(1)}} \mathcolor{gray}{t}}_{\;\!\mathcolor{gray}{z}} = \bar{\bar{\mu}}^{\;\!\textcolor{Maroon}{\text{(1)}} \mathcolor{gray}{t}}_{\;\!\mathcolor{gray}{z}}~\widetilde *~\bar{H}^{\;\!\mathcolor{gray}{t}}_{\;\!\mathcolor{gray}{z}}$ 表示。

