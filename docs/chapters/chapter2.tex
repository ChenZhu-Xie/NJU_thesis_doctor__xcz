\marginLeft{chap:maxwell}\chapter{晶体中的(非)线性光学过程}\label{chap:maxwell}

相对观察者静止\Footnote{否则标/矢/张量场的2个实自变量:时间$\textcolor{gray}{t} \in \mathbb{R}^{\textcolor{Maroon}{(0)}}_1$和空间$\textcolor{gray}{\bar{r}} \in \bar{\mathbb{R}}^{\textcolor{Maroon}{(1)}}_3$将在其他参考系(见\bref{uwav})的度量下相互混合,即$\textcolor{gray}{\uwav{t}} \left( \textcolor{gray}{t}, \textcolor{gray}{\bar{r}} \right), \textcolor{gray}{\bar{\uwav{r}}} \left( \textcolor{gray}{t}, \textcolor{gray}{\bar{r}} \right)$,并形成统一的$1+3$维黎曼时空/微分流形$\textcolor{gray}{\uwav{\bar{x}}} \left( \textcolor{gray}{\bar{x}} \right) \in \bar{\mathbb{R}}^{\textcolor{Maroon}{(1)}}_4$;接着,定义四维位移矢量$\textcolor{gray}{\bar{x}}$对四维标量固有时$\textcolor{gray}{\uo{t}}$(见\bref{uo})的导数:四维速度矢量$\bar{u}_{\;\!\textcolor{gray}{\bar{x}}} := \mathbb{d} \textcolor{gray}{\bar{x}} \big/ \mathbb{d} \textcolor{gray}{\uo{t}} \in \bar{\mathbb{C}}^{\textcolor{Maroon}{(1)}}_4 ( \bar{\mathbb{R}}^{\textcolor{Maroon}{(1)}}_4 )$,并将总电流 $\bar{J}^{\;\!\textcolor{gray}{t}}_{\;\!\textcolor{gray}{z}} = {\rho}^{\;\!\textcolor{gray}{t}}_{\;\!\textcolor{gray}{z}} \bar{v}^{\;\!\textcolor{gray}{t}}_{\;\!\textcolor{gray}{z}}$ 与总电荷${\rho}^{\;\!\textcolor{gray}{t}}_{\;\!\textcolor{gray}{z}}$合并以扩展至四维$\bar{J}_{\;\! \textcolor{gray}{\bar{x}}} = \uo{\rho}_{\;\! \textcolor{gray}{\bar{\uo{x}}}} \bar{u}_{\;\! \textcolor{gray}{\bar{x}}}  \in \bar{\mathbb{C}}^{\textcolor{Maroon}{(1)}}_4 ( \bar{\mathbb{R}}^{\textcolor{Maroon}{(1)}}_4 )$\cite{lakhtakiaCovariancesInvariancesMaxwell1995,chen-zhuChenZhuxieUndergraduate_courses2024};此外,2个基本场 $\bar{E}_{\;\!\textcolor{gray}{\bar{x}}}$ 与 $\bar{B}_{\;\!\textcolor{gray}{\bar{x}}}$ 也将相互耦合,即$\uwav{\bar{E}}_{\;\!\textcolor{gray}{\uwav{\bar{x}}}} \left( \bar{E}_{\;\!\textcolor{gray}{\bar{x}}},\bar{B}_{\;\!\textcolor{gray}{\bar{x}}} \right), \uwav{\bar{B}}_{\;\!\textcolor{gray}{\uwav{\bar{x}}}} \left( \bar{E}_{\;\!\textcolor{gray}{\bar{x}}},\bar{B}_{\;\!\textcolor{gray}{\bar{x}}} \right)$,并成为一个整体:2阶4维电磁张量场 $\uwav{\bar{\bar{F}}}_{\;\!\textcolor{gray}{\uwav{\bar{x}}}} ( \bar{\bar{F}}_{\;\!\textcolor{gray}{\bar{x}}} ) \in \bar{\mathbb{C}}^{\textcolor{Maroon}{(2)}}_{\left[4 \times 4\right]} ( \bar{\mathbb{R}}^{\textcolor{Maroon}{(1)}}_4 )$\cite{lakhtakiaCovariancesInvariancesMaxwell1995,berryOpticalSingularitiesBianisotropic2005,chen-zhuChenZhuxieUndergraduate_courses2024},以致材料的本构关系将呈现固有双各向异性$\uwav{\bar{D}}_{\;\!\textcolor{gray}{\uwav{\bar{x}}}} \left[ \uwav{\bar{E}}_{\;\!\textcolor{gray}{\uwav{\bar{x}}}} \left( \bar{E}_{\;\!\textcolor{gray}{\bar{x}}},\bar{B}_{\;\!\textcolor{gray}{\bar{x}}} \right) \right], \uwav{\bar{H}}_{\;\!\textcolor{gray}{\uwav{\bar{x}}}} \left[ \uwav{\bar{B}}_{\;\!\textcolor{gray}{\uwav{\bar{x}}}} \left( \bar{E}_{\;\!\textcolor{gray}{\bar{x}}},\bar{B}_{\;\!\textcolor{gray}{\bar{x}}} \right) \right]$或$\uwav{\bar{\bar{G}}}_{\;\!\textcolor{gray}{\uwav{\bar{x}}}} [ \uwav{\bar{\bar{F}}}_{\;\!\textcolor{gray}{\uwav{\bar{x}}}} ( \bar{\bar{F}}_{\;\!\textcolor{gray}{\bar{x}}} ) ] = \uwav{\bar{\bar{G}}}_{\;\!\textcolor{gray}{\uwav{\bar{x}}}} [ \bar{\bar{G}}_{\;\!\textcolor{gray}{\bar{x}}} ( \bar{\bar{F}}_{\;\!\textcolor{gray}{\bar{x}}} ) ]$\cite{langeMultipoleTheoryHehl2015,hehlLinearMediaClassical2005,mackayElectromagneticAnisotropyBianisotropy2019,mackayModernAnalyticalElectromagnetic2020,lakhtakiaCovariancesInvariancesMaxwell1995}。此外,${\rho}^{\;\!\textcolor{gray}{t}}_{\;\!\textcolor{gray}{z}}$对应的束缚/自由=价带/导带电子的(有效)运动质量(或相对论速度)也会变大,并因此同时对线性和非线性极化、磁化、自由电流$\bar{J}^{\;\!\textcolor{gray}{t}}_{\;\!\textcolor{Maroon}{\text{f}}\textcolor{gray}{z}} = {\rho}^{\;\!\textcolor{gray}{t}}_{\;\!\textcolor{Maroon}{\text{f}}\textcolor{gray}{z}} \bar{v}^{\;\!\textcolor{gray}{t}}_{\;\!\textcolor{Maroon}{\text{f}}\textcolor{gray}{z}}$产生额外影响。——上述场景可发生在超快和强场泵浦(及其通过多光子/隧穿电离产生的等离子体)中\cite{boydNonlinearOptics2019}、相对材料运动的坐标系下,甚至一直在发生在某些材料内(库伦力与电子轨道:金的颜色\cite{boydNonlinearOptics2019}、汞常温液体、铅酸电池的额外电压和稳定性\cite{boydNonlinearOptics2019};狄拉克方程与电子自旋、旋轨耦合:大部分磁效应),以至于可能(必)需要引入(非线性)相对论电动力学。} 的三维空间$\textcolor{gray}{\bar{r}}$ 坐标系下,在可能存在非零的自由电荷源和自由电流源${\rho}_{\;\!\textcolor{Maroon}{\text{f}}} \left( \textcolor{gray}{\bar{r}}, \textcolor{gray}{t} \right), \bar{J}_{\;\!\textcolor{Maroon}{\text{f}}} \left( \textcolor{gray}{\bar{r}}, \textcolor{gray}{t} \right)$\Footnote{对于符号约定,比如下标 $\textcolor{Maroon}{\text{f}}$ 的\textcolor{Maroon}{褐红色}及其含义 `\textcolor{Maroon}{free}',其定义见\bref{Maroon};此外,在 \bref{1bar} 中还约定:总使用 1 条上短横线 $\bar{~}$(而不是粗体)来表示矢量(如 $\bar{J}_{\;\!\textcolor{Maroon}{\text{f}}}$),以区别于 \bref{0bar} 中定义的无上短横线的标量(如 ${\rho}_{\;\!\textcolor{Maroon}{\text{f}}}$);粗体在本文中另有其含义,不用于表示矢量,见。} 的一般电磁介质内部,4 个空域时变\Footnote{指复矢量场 $\bar{E}^{\;\!\textcolor{gray}{t}}_{\;\!\textcolor{gray}{z}}, \bar{H}^{\;\!\textcolor{gray}{t}}_{\;\!\textcolor{gray}{z}}, \bar{D}^{\;\!\textcolor{gray}{t}}_{\;\!\textcolor{gray}{z}}, \bar{B}^{\;\!\textcolor{gray}{t}}_{\;\!\textcolor{gray}{z}}$ 均是四维时空 $\textcolor{gray}{\bar{r}}, \textcolor{gray}{t}$ 的函数,且因此一般意义上是复色 $\left\{ \omega \in \mathbb{R} \right\}$ 的复场 $\in \bar{\mathbb{C}}^{\textcolor{Maroon}{(1)}}_3 ( \bar{\mathbb{R}}^{\textcolor{Maroon}{(1)}}_3 )$(认为这四者必须为实场\cite{boydNonlinearOptics2019}也没关系:它们对$t$的傅立叶变换所得到的$\pm \omega$单色子波是复共轭的,以至于正负频率的对应复子波求和后会消掉虚部,只剩下实部的余弦$\cos$实子波,因此在正/倒空间中的总/子场均是有物理意义的实场),属于在视觉上占主导的因变量,并用黑色(见 \bref{black})的 \textit{斜体 oblique}(见 \bref{oblique})表示;相对地,自变量用视觉和含义上均更次要的\textcolor{gray}{灰色}来表示,见 \bref{gray}。}复色场 
$\bar{E}^{\;\!\textcolor{gray}{t}}_{\;\!\textcolor{gray}{z}}, \bar{H}^{\;\!\textcolor{gray}{t}}_{\;\!\textcolor{gray}{z}}, \bar{D}^{\;\!\textcolor{gray}{t}}_{\;\!\textcolor{gray}{z}}, \bar{B}^{\;\!\textcolor{gray}{t}}_{\;\!\textcolor{gray}{z}}$\Footnote{由于傅立叶光学一般运行在平行平面间,约定下述表示相互等价:$\bar{E} \left( \textcolor{gray}{\bar{r}}, \textcolor{gray}{t} \right) = \bar{E}^{\;\!\textcolor{gray}{t}}_{\;\!\textcolor{gray}{\bar{r}}} = \bar{E}^{\;\!\textcolor{gray}{t}}_{\;\!\textcolor{gray}{z}} \left( \textcolor{gray}{\bar{\rho}} \right)$,并因此经常省略面内自变量 $\textcolor{gray}{\bar{\rho}}$,以只写作朝 $\textcolor{Maroon}{+\symup{z}}$ 轴传播距离 $\textcolor{gray}{z}$ 的函数 $\bar{E}^{\;\!\textcolor{gray}{t}}_{\;\!\textcolor{gray}{z}} := \bar{E}^{\;\!\textcolor{gray}{t}}_{\;\!\textcolor{gray}{z}} \left( \textcolor{gray}{\bar{\rho}} \right)$。—— 同样的规则也适用于其他场量(如 ${\rho}_{\;\!\textcolor{Maroon}{\text{f}}} \left( \textcolor{gray}{\bar{r}}, \textcolor{gray}{t} \right) \to {\rho}^{\;\!\textcolor{gray}{t}}_{\;\!\textcolor{Maroon}{\text{f}}\textcolor{gray}{z}}$),且适用于空间频率域,见。},满足微分形式\Footnote{尽管是微分形式,仍然处于(相对的)宏观层面:典型的光波长 $1$um 是原子特征尺寸 $1\text{\r{A}} = 0.1$nm 的 $10^4$ 倍,因此\bref{eq:maxwell-eh,eq:maxwell-db,eq:continuity-pj}涉及的所有物理量均是空间平均后的结果\cite{mackayElectromagneticAnisotropyBianisotropy2019};如果要引入(非)线性极化强度/率随考虑区域尺度的缩放,则需要 \textcolor{Maroon}{Clausius-Mossotti equation} 或 \textcolor{Maroon}{Lorentz-Lorenz law} 的局域场修正\cite{boydNonlinearOptics2019}。}的麦氏方程组的 2 个旋度假设:
\begin{subequations} \label{eq:maxwell-eh}
\begin{align}
	\textcolor{Maroon}{\text{Faraday's law of electromagnetic induction}}\text{:}&\hspace{0.5em} \bar{\nabla} \times \bar{E}^{\;\!\textcolor{gray}{t}}_{\;\!\textcolor{gray}{z}} \hspace{-1.2em} &&= - \bar{K}^{\;\!\textcolor{gray}{t}}_{\;\!\textcolor{Maroon}{\text{f}}\textcolor{gray}{z}} - \frac{\partial \bar{B}^{\;\!\textcolor{gray}{t}}_{\;\!\textcolor{gray}{z}}}{\partial t}~, \label{eq:maxwell-e} \\ \textcolor{Maroon}{\text{Jefimenko's}} \to \textcolor{Maroon}{\text{Amp\`{e}re-Maxwell circuital law}}\text{:}&\hspace{0.5em} \bar{\nabla} \times \bar{H}^{\;\!\textcolor{gray}{t}}_{\;\!\textcolor{gray}{z}} \hspace{-1.2em} &&= \bar{J}^{\;\!\textcolor{gray}{t}}_{\;\!\textcolor{Maroon}{\text{f}}\textcolor{gray}{z}} + \frac{\partial \bar{D}^{\;\!\textcolor{gray}{t}}_{\;\!\textcolor{gray}{z}}}{\partial t}~, \label{eq:maxwell-h}
\end{align}
\end{subequations}
以及 2 个散度假设:
\begin{subequations} \label{eq:maxwell-db}
\begin{align}
	\textcolor{Maroon}{\text{Coulomb's}} \to \textcolor{Maroon}{\text{Gauss's law for electricity}}\text{:}&\hspace{0.5em} \bar{\nabla} \cdot \bar{D}^{\;\!\textcolor{gray}{t}}_{\;\!\textcolor{gray}{z}} \hspace{-3.2em} &&= {\rho}^{\;\!\textcolor{gray}{t}}_{\;\!\textcolor{Maroon}{\text{f}}\textcolor{gray}{z}}~, \label{eq:maxwell-d} \\ \textcolor{Maroon}{\text{Biot-Savart}} \to \textcolor{Maroon}{\text{Gauss's law for magnetism}}\text{:}&\hspace{0.5em} \bar{\nabla} \cdot \bar{B}^{\;\!\textcolor{gray}{t}}_{\;\!\textcolor{gray}{z}} \hspace{-3.2em} &&= {\kappa}^{\;\!\textcolor{gray}{t}}_{\;\!\textcolor{Maroon}{\text{f}}\textcolor{gray}{z}}~. \label{eq:maxwell-b}
\end{align}
\end{subequations}
其中,为数学形式上的对称(以方便引入相对论效应和检验其协变性\cite{lakhtakiaCovariancesInvariancesMaxwell1995,chen-zhuChenZhuxieUndergraduate_courses2024}),和物理上不排除可能存在的磁单极子,除自由电(荷/流)源${\rho}^{\;\!\textcolor{gray}{t}}_{\;\!\textcolor{Maroon}{\text{f}}\textcolor{gray}{z}}, \bar{J}^{\;\!\textcolor{gray}{t}}_{\;\!\textcolor{Maroon}{\text{f}}\textcolor{gray}{z}}$外,还添加了自由磁源${\kappa}^{\;\!\textcolor{gray}{t}}_{\;\!\textcolor{Maroon}{\text{f}}\textcolor{gray}{z}}, \bar{K}^{\;\!\textcolor{gray}{t}}_{\;\!\textcolor{Maroon}{\text{f}}\textcolor{gray}{z}}$\cite{lakhtakiaCovariancesInvariancesMaxwell1995}。这 4 个自由源(体密度)项,满足2个连续性假设\cite{mackayElectromagneticAnisotropyBianisotropy2019,lakhtakiaCovariancesInvariancesMaxwell1995,chen-zhuChenZhuxieUndergraduate_courses2024}:
\begin{subequations} \label{eq:continuity-pj}
\begin{align}
	\textcolor{Maroon}{\text{Continuity for electric free source}}\text{:}&\hspace{0.5em} \bar{\nabla} \cdot \bar{J}^{\;\!\textcolor{gray}{t}}_{\;\!\textcolor{Maroon}{\text{f}}\textcolor{gray}{z}} + \frac{\partial {\rho}^{\;\!\textcolor{gray}{t}}_{\;\!\textcolor{Maroon}{\text{f}}\textcolor{gray}{z}}}{\partial t} \hspace{-5.2em} &&= 0~, \label{eq:continuity-e} \\ \textcolor{Maroon}{\text{Continuity for magnetic free source}}\text{:}&\hspace{0.5em} \bar{\nabla} \cdot \bar{K}^{\;\!\textcolor{gray}{t}}_{\;\!\textcolor{Maroon}{\text{f}}\textcolor{gray}{z}} + \frac{\partial {\kappa}^{\;\!\textcolor{gray}{t}}_{\;\!\textcolor{Maroon}{\text{f}}\textcolor{gray}{z}}}{\partial t} \hspace{-5.2em} &&= 0~. \label{eq:continuity-m}
\end{align}
\end{subequations}
注意,对旋度 \bref{eq:maxwell-eh} 两边取散度($\bar{\nabla} \cdot$),连续性 \bref{eq:continuity-pj} 可导出散度 \bref{eq:maxwell-db},反之亦然\cite{lakhtakiaGenesisPostConstraint2004}。因此,2 条散度方程均不是必需的,可将其视为冗余\Footnote{并且不应简单地仅根据$\bar{D}^{\;\!\textcolor{gray}{t}}_{\;\!\textcolor{gray}{z}}, \bar{B}^{\;\!\textcolor{gray}{t}}_{\;\!\textcolor{gray}{z}}$的横向性,而将二者视为基本场\cite{quesadaPhotonPairsNonlinear2022,berryOpticalSingularitiesBianisotropic2005}。但从场能量体密度变化率$\bar{E}^{\;\!\textcolor{gray}{t}}_{\;\!\textcolor{gray}{z}} \mathbb{d}\bar{D}^{\;\!\textcolor{gray}{t}}_{\;\!\textcolor{gray}{z}} +  \bar{H}^{\;\!\textcolor{gray}{t}}_{\;\!\textcolor{gray}{z}} \mathbb{d}\bar{B}^{\;\!\textcolor{gray}{t}}_{\;\!\textcolor{gray}{z}}$中含有 2 个旋度\bref{eq:maxwell-eh}中对$\bar{D}^{\;\!\textcolor{gray}{t}}_{\;\!\textcolor{gray}{z}}, \bar{B}^{\;\!\textcolor{gray}{t}}_{\;\!\textcolor{gray}{z}}$的微分、方便引入适用于非线性量子光学的正确的哈密顿量\cite{quesadaPhotonPairsNonlinear2022},或者从更便利、自然和优雅地描述天然/法拉第旋光效应的角度\cite{berryOpticalSingularitiesBianisotropic2005},将$\bar{D}^{\;\!\textcolor{gray}{t}}_{\;\!\textcolor{gray}{z}}, \bar{B}^{\;\!\textcolor{gray}{t}}_{\;\!\textcolor{gray}{z}}$视为基本场也有一定道理?}。

现代电磁学/电动力学将$\bar{E}^{\;\!\textcolor{gray}{t}}_{\;\!\textcolor{gray}{z}}, \bar{B}^{\;\!\textcolor{gray}{t}}_{\;\!\textcolor{gray}{z}}$视为基本场\cite{hillionBasicFieldElectromagnetism1996,lakhtakiaGenesisPostConstraint2004,nelsonDerivingTransmissionReflection1995}\Footnote{$\bar{D}^{\;\!\textcolor{gray}{t}}_{\;\!\textcolor{gray}{z}},\bar{H}^{\;\!\textcolor{gray}{t}}_{\;\!\textcolor{gray}{z}}$只是$\bar{E}^{\;\!\textcolor{gray}{t}}_{\;\!\textcolor{gray}{z}},\bar{B}^{\;\!\textcolor{gray}{t}}_{\;\!\textcolor{gray}{z}}$分别加上($\bar{P}^{\;\!\textcolor{gray}{t}}_{\;\!\textcolor{gray}{z}}$)或减去($\bar{M}^{\;\!\textcolor{gray}{t}}_{\;\!\textcolor{gray}{z}}$)束缚源所产生的场后的辅助场,代表自由源所对应的场。$\bar{E}^{\;\!\textcolor{gray}{t}}_{\;\!\textcolor{gray}{z}},\bar{B}^{\;\!\textcolor{gray}{t}}_{\;\!\textcolor{gray}{z}}$是总场和基本场,出于下述原因:其起源是微观且明确的、可直接测量、包含了所有的束缚和自由源产生的场、洛伦兹力公式(普适至相对论情形)、\bref{eq:maxwell-e,eq:maxwell-b}的无源特性及其导出的标矢势和四维势矢量、四维二阶电磁场张量\cite{chen-zhuChenZhuxieUndergraduate_courses2024}、无矛盾地推导和适用 Post 约束\cite{lakhtakiaGenesisPostConstraint2004};同时也方便原子物理中对拉莫尔进动、史特恩—盖拉赫实验、塞曼效应的表述\cite{chen-zhuChenZhuxieUndergraduate_courses2024},以及量子电动力学中对磁光材料的拉氏量的处理\cite{nelsonLagrangianTreatmentMagnetic1994}。——但是,选择$\bar{B}^{\;\!\textcolor{gray}{t}}_{\;\!\textcolor{gray}{z}}$而不是$\bar{H}^{\;\!\textcolor{gray}{t}}_{\;\!\textcolor{gray}{z}}$将不方便(准静)磁学,如铁磁性物质的磁滞回线的表述\cite{hillionBasicFieldElectromagnetism1996}。}。但本文将$\bar{E}^{\;\!\textcolor{gray}{t}}_{\;\!\textcolor{gray}{z}}, \bar{H}^{\;\!\textcolor{gray}{t}}_{\;\!\textcolor{gray}{z}}$视为基本场\Footnote{相对论或手性的情形下,将$\bar{E}^{\;\!\textcolor{gray}{t}}_{\;\!\textcolor{gray}{z}}, \bar{H}^{\;\!\textcolor{gray}{t}}_{\;\!\textcolor{gray}{z}}$而不是$\bar{E}^{\;\!\textcolor{gray}{t}}_{\;\!\textcolor{gray}{z}}, \bar{B}^{\;\!\textcolor{gray}{t}}_{\;\!\textcolor{gray}{z}}$作为本构关系的基本场,可能更有优势\cite{hillionBasicFieldElectromagnetism1996,lakhtakiaGenesisPostConstraint2004};此外,对于边界条件,进可采用$\bar{E}^{\;\!\textcolor{gray}{t}}_{\;\!\textcolor{gray}{z}},\bar{H}^{\;\!\textcolor{gray}{t}}_{\;\!\textcolor{gray}{z}}$切向连续边界条件,退可四维时空傅立叶变换\cite{chenWavevectorspaceMethodWave1993,chenWavePropagationExciton1993,nelsonDerivingTransmissionReflection1995}。还允许不关注微观起源\cite{eimerlQuantumElectrodynamicsOptical1988,nelsonMechanismsDispersionCrystalline1989,boydNonlinearOptics2019,loudonPropagationElectromagneticEnergy1997,laxLinearNonlinearElectrodynamics1971},电场的非局域一阶波矢色散也可直接放进本构关系而无需额外处理\cite{berryOpticalSingularitiesBianisotropic2005}。但可能没法处理材料表面积累电荷(如铁电体的$\textcolor{Maroon}{+\symup{c}}$ 面)、表面电流\cite{chen-zhuChenZhuxieUndergraduate_courses2024}、表面光学活性\cite{nelsonMechanismsDispersionCrystalline1989},尽管没有使用到任何散度方程/横向约束,已经很有吸引力了\cite{eimerlQuantumElectrodynamicsOptical1988,berryOpticalSingularitiesBirefringent2003,berryOpticalSingularitiesBianisotropic2005}。},并分别给磁感应场$\bar{B}^{\;\!\textcolor{gray}{t}}_{\;\!\textcolor{gray}{z}}$(而非磁场$\bar{H}^{\;\!\textcolor{gray}{t}}_{\;\!\textcolor{gray}{z}}$),自由电流源$\bar{J}^{\;\!\textcolor{gray}{t}}_{\;\!\textcolor{Maroon}{\text{f}}\textcolor{gray}{z}}$ 以及电位移场$\bar{D}^{\;\!\textcolor{gray}{t}}_{\;\!\textcolor{gray}{z}}$,定义了如下 3 个本构关系(\textcolor{Maroon}{\text{Constitutive Relation}} = \textcolor{Maroon}{\text{CR}})。

其一,磁场 $\bar{H}^{\;\!\textcolor{gray}{t}}_{\;\!\textcolor{gray}{z}}$ 的本构关系,考虑$\bar{M}^{\;\!\textcolor{gray}{t}}_{\;\!\textcolor{gray}{z}}$\Footnote{磁化强度。其在微观上来源于:分子电流(电子轨道运动)产生的磁矩和电子自旋磁矩的矢量和\cite{nelsonLagrangianTreatmentMagnetic1994},细究还将有电子与原子核、电子的自旋轨道耦合、电子电子间相互作用(多电子产生原子磁矩)、原子与原子间(晶体场)的相互作用\cite{chen-zhuChenZhuxieUndergraduate_courses2024}。磁的起源看上去可以是纯电的\cite{lakhtakiaGenesisPostConstraint2004}。}仅由磁偶极矩\Footnote{即只考虑最低阶磁多极矩 = 不考虑磁四极矩及以上\cite{nelsonLagrangianTreatmentMagnetic1994},因为对于受到电磁场的影响后,反过来产生电磁场的电子而言,其受到的电场力是洛伦兹力的c$\big/v$倍\cite{boydNonlinearOptics2019}。此外,(磁/电)多极矩与(磁/电)非线性是互相独立的——即任何阶的多极矩,都有自己的线性项和非线性项\cite{chen-zhuChenZhuxieUndergraduate_courses2024},这些项与其它阶多极矩的任何项都无关。}贡献时,定义为
\begin{subequations} \label{eq:cr-b}
\begin{align}
	\textcolor{Maroon}{\text{CR for magnetism}}\text{:}&\hspace{0.5em} \bar{B}^{\;\!\textcolor{gray}{t}}_{\;\!\textcolor{gray}{z}} \hspace{-0.7em} &&={\symup{\mu}}_0 \left( \bar{H}^{\;\!\textcolor{gray}{t}}_{\;\!\textcolor{gray}{z}} + \bar{M}^{\;\!\textcolor{gray}{t}}_{\;\!\textcolor{gray}{z}} \right) = {\symup{\mu}}_0 \left\{ \bar{\bar{\delta}}^{\;\!\textcolor{gray}{t}}~\widetilde *~\bar{H}^{\;\!\textcolor{gray}{t}}_{\;\!\textcolor{gray}{z}} + \bar{M}^{\;\!\textcolor{gray}{t}}_{\;\!\textcolor{gray}{z}} \right\} \label{cr-b1} \\ & &&\xrightarrow[]{\bar{M}^{\;\!\textcolor{gray}{t}}_{\;\!\textcolor{gray}{z}} = \bar{M}^{\;\!\textcolor{Maroon}{\text{(1)}} \textcolor{gray}{t}}_{\;\!\textcolor{gray}{z}} + \bar{M}^{\;\!\textcolor{Maroon}{\text{NL}}, \textcolor{gray}{t}}_{\;\!\textcolor{gray}{z}}} {\symup{\mu}}_0 \left\{ \left[ \bar{\bar{\delta}}^{\;\!\textcolor{gray}{t}}~\widetilde *~\bar{H}^{\;\!\textcolor{gray}{t}}_{\;\!\textcolor{gray}{z}} + \bar{M}^{\;\!\textcolor{Maroon}{\text{(1)}} \textcolor{gray}{t}}_{\;\!\textcolor{gray}{z}} \right] + \bar{M}^{\;\!\textcolor{Maroon}{\text{NL}}, \textcolor{gray}{t}}_{\;\!\textcolor{gray}{z}} \right\} \label{cr-b2} \\ & &&\xrightarrow[\displaystyle{ \bar{\bar{\mu}}^{\;\!\textcolor{Maroon}{\text{(1)}} \textcolor{gray}{t}}_{\;\!\textcolor{Maroon}{\text{r}}\textcolor{gray}{z}} := \bar{\bar{\delta}}^{\;\!\textcolor{gray}{t}} + \bar{\bar{\chi}}^{\;\!\textcolor{Maroon}{\text{(1)}}\textcolor{gray}{t}}_{\;\!\textcolor{Maroon}{\text{m}} \textcolor{gray}{z}}}]{\displaystyle{\bar{M}^{\;\!\textcolor{Maroon}{\text{(1)}} \textcolor{gray}{t}}_{\;\!\textcolor{gray}{z}} := \bar{\bar{\chi}}^{\;\!\textcolor{Maroon}{\text{(1)}}\textcolor{gray}{t}}_{\;\!\textcolor{Maroon}{\text{m}} \textcolor{gray}{z}} ~\widetilde *~\bar{H}^{\;\!\textcolor{gray}{t}}_{\;\!\textcolor{gray}{z}}}} {\symup{\mu}}_0 \left\{ \bar{\bar{\mu}}^{\;\!\textcolor{Maroon}{\text{(1)}} \textcolor{gray}{t}}_{\;\!\textcolor{Maroon}{\text{r}}\textcolor{gray}{z}}~\widetilde *~\bar{H}^{\;\!\textcolor{gray}{t}}_{\;\!\textcolor{gray}{z}} + \bar{M}^{\;\!\textcolor{Maroon}{\text{NL}}, \textcolor{gray}{t}}_{\;\!\textcolor{gray}{z}} \right\} \label{cr-b3} \\ & &&= \bar{\bar{\mu}}^{\;\!\textcolor{Maroon}{\text{(1)}} \textcolor{gray}{t}}_{\;\!\textcolor{gray}{z}}~\widetilde *~\bar{H}^{\;\!\textcolor{gray}{t}}_{\;\!\textcolor{gray}{z}} + {\symup{\mu}}_0 \bar{M}^{\;\!\textcolor{Maroon}{\text{NL}}, \textcolor{gray}{t}}_{\;\!\textcolor{gray}{z}} =: \bar{B}^{\;\!\textcolor{Maroon}{\text{(1)}} \textcolor{gray}{t}}_{\;\!\textcolor{gray}{z}} + \bar{B}^{\;\!\textcolor{Maroon}{\text{NL}}, \textcolor{gray}{t}}_{\;\!\textcolor{gray}{z}}~, \label{cr-b4}
\end{align}
\end{subequations}
其中,磁通量密度场 $\bar{B}^{\;\!\textcolor{gray}{t}}_{\;\!\textcolor{gray}{z}}$(直接/显示地)关于磁场 $\bar{H}^{\;\!\textcolor{gray}{t}}_{\;\!\textcolor{gray}{z}}$\Footnote{磁非线性,如郎之万顺磁性理论\cite{chen-zhuChenZhuxieUndergraduate_courses2024}中 $M \propto$ 郎之万函数 $\mathcal{L} \left( \alpha \right) = \coth \left( \alpha \right) - 1 / \alpha$(其中 $\alpha \propto H_{\textcolor{Maroon}{\text{ex}}}$)或其量子化修正之布里渊函数,铁磁体\cite{chen-zhuChenZhuxieUndergraduate_courses2024}或超导体\cite{wenBriefIntroductionFlux2021}中的磁滞现象等(每个时刻$t$,这些场量都是准静态$\Omega \to 0$的)。}、电场 $\bar{E}^{\;\!\textcolor{gray}{t}}_{\;\!\textcolor{gray}{z}}$\Footnote{双各向异性,其中的电$\to$磁耦合(如果 $\bar{B}^{\;\!\textcolor{gray}{t}}_{\;\!\textcolor{gray}{z}}$ 中的该部分只是 $\bar{E}^{\;\!\textcolor{gray}{t}}_{\;\!\textcolor{gray}{z}}$ 的线性函数,则也可归结到线性项中)。}、应力 $\bar{T}^{\;\!\textcolor{gray}{t}}_{\;\!\textcolor{gray}{z}}$\Footnote{正逆压磁/磁致伸缩/磁弹效应(这里未作区分)。}等其他场量(即含空 $\textcolor{gray}{\bar{r}}$ 的物理量)\Footnote{$\bar{B}^{\;\!\textcolor{gray}{t}}_{\;\!\textcolor{gray}{z}},\bar{M}^{\;\!\textcolor{gray}{t}}_{\;\!\textcolor{gray}{z}}$ 以及各阶 $\bar{\bar{\mu}}^{\;\!\textcolor{Maroon}{\text{(1)}} \textcolor{gray}{t}}_{\;\!\textcolor{gray}{z}},\bar{\bar{\bar{\mu}}}^{\;\!\textcolor{Maroon}{\text{(2)}} \textcolor{gray}{t}}_{\;\!\textcolor{gray}{z}},\cdots$ 已经是关于温度$T$、波长$\lambda$(或 角频率$\omega$、时间$t$)等(非)场量的函数。}的非线性函数项,悉数包含在 $\bar{B}^{\;\!\textcolor{Maroon}{\text{NL}}, \textcolor{gray}{t}}_{\;\!\textcolor{gray}{z}} = {\symup{\mu}}_0 \bar{M}^{\;\!\textcolor{Maroon}{\text{NL}}, \textcolor{gray}{t}}_{\;\!\textcolor{gray}{z}}$ 内;剩余的线性项,放在 $\bar{B}^{\;\!\textcolor{Maroon}{\text{(1)}} \textcolor{gray}{t}}_{\;\!\textcolor{gray}{z}} = \bar{\bar{\mu}}^{\;\!\textcolor{Maroon}{\text{(1)}} \textcolor{gray}{t}}_{\;\!\textcolor{gray}{z}}~\widetilde *~\bar{H}^{\;\!\textcolor{gray}{t}}_{\;\!\textcolor{gray}{z}}$ 中。

其二,自由电流源$\bar{J}^{\;\!\textcolor{gray}{t}}_{\;\!\textcolor{Maroon}{\text{f}}\textcolor{gray}{z}}$的本构关系,包含欧姆定律的线性部分(漂移项) $\bar{J}^{\;\!\textcolor{Maroon}{\text{(1)}} \textcolor{gray}{t}}_{\;\!\textcolor{Maroon}{\text{f}}\textcolor{gray}{z}} = \bar{\bar{\sigma}}^{\;\!\textcolor{Maroon}{\text{(1)}}\textcolor{gray}{t}}_{\;\!\textcolor{gray}{z}}~\widetilde *~\bar{E}^{\;\!\textcolor{gray}{t}}_{\;\!\textcolor{gray}{z}}$\Footnote{可以由 Drude 模型描述,定量解释一阶电导率$\bar{\bar{\sigma}}^{\;\!\textcolor{Maroon}{\text{(1)}}\textcolor{gray}{t}}_{\;\!\textcolor{gray}{z}}$的起源。},及$\bar{J}^{\;\!\textcolor{gray}{t}}_{\;\!\textcolor{Maroon}{\text{f}}\textcolor{gray}{z}}$分别关于电场 $\bar{E}^{\;\!\textcolor{gray}{t}}_{\;\!\textcolor{gray}{z}}$\Footnote{欧姆定律中的电非线性部分,比如二/三极管的伏安特性曲线\cite{chen-zhuChenZhuxieUndergraduate_courses2024}(尽管输入/输出or自/因变量,即$\bar{E}^{\;\!\textcolor{gray}{t}}_{\;\!\textcolor{gray}{z}}$和$\bar{J}^{\;\!\textcolor{gray}{t}}_{\;\!\textcolor{Maroon}{\text{f}}\textcolor{gray}{z}}$,一般均在直流或低频$\Omega$,非交流且不在光波段 opt)。}、磁感应场$\bar{B}^{\;\!\textcolor{gray}{t}}_{\;\!\textcolor{gray}{z}}$\Footnote{磁感应场$\bar{B}^{\;\!\textcolor{gray}{t}}_{\;\!\textcolor{gray}{z}}$所带来的(电场力以外的)洛伦兹力$\left( \bar{J}^{\;\!\textcolor{gray}{t}}_{\;\!\textcolor{Maroon}{\text{f}}\textcolor{gray}{z}} + \dot{\bar{P}}^{\;\!\textcolor{gray}{t}}_{\;\!\textcolor{gray}{z}} + \bar{\nabla} \times \bar{M}^{\;\!\textcolor{gray}{t}}_{\;\!\textcolor{gray}{z}} \right) \times \bar{B}^{\;\!\textcolor{gray}{t}}_{\;\!\textcolor{gray}{z}}$\cite{mackayElectromagneticAnisotropyBianisotropy2019,chen-zhuChenZhuxieUndergraduate_courses2024},会影响导/价带电子的运动(速度)$\bar{v}^{\;\!\textcolor{gray}{t}}_{\;\!\textcolor{Maroon}{\text{f}}\textcolor{gray}{z}}$,进而全局地影响自由电流$\bar{J}^{\;\!\textcolor{gray}{t}}_{\;\!\textcolor{Maroon}{\text{f}}\textcolor{gray}{z}} = {\rho}^{\;\!\textcolor{gray}{t}}_{\;\!\textcolor{Maroon}{\text{f}}\textcolor{gray}{z}} \bar{v}^{\;\!\textcolor{gray}{t}}_{\;\!\textcolor{Maroon}{\text{f}}\textcolor{gray}{z}}$和(束缚)电(偶)极化强度$\bar{P}^{\;\!\textcolor{gray}{t}}_{\;\!\textcolor{gray}{z}}$,包括它们的线性和非线性项\cite{boydNonlinearOptics2019}。对于强场/超快非线性光学,相对论效应使得电磁场是个统一的整体,动生(而不仅是外加)的$\bar{B}^{\;\!\textcolor{gray}{t}}_{\;\!\textcolor{gray}{z}}$还将带来额外的影响。磁场$\bar{H}^{\;\!\textcolor{gray}{t}}_{\;\!\textcolor{gray}{z}}$对分子/磁化电流体密度$\bar{\nabla} \times \bar{M}^{\;\!\textcolor{gray}{t}}_{\;\!\textcolor{gray}{z}}$产生的影响已包含在$\bar{M}^{\;\!\textcolor{gray}{t}}_{\;\!\textcolor{gray}{z}}$中了。}、导带电子浓度(数密度)梯度场$\bar{\nabla} {\rho}^{\;\!\textcolor{gray}{t}}_{\;\!\textcolor{Maroon}{\text{f}}\textcolor{gray}{z}}$\Footnote{在光折变效应中,作为$\bar{J}^{\;\!\textcolor{gray}{t}}_{\;\!\textcolor{Maroon}{\text{f}}\textcolor{gray}{z}}$中的扩散项\cite{boydNonlinearOptics2019}。${\rho}^{\;\!\textcolor{gray}{t}}_{\;\!\textcolor{Maroon}{\text{f}}\textcolor{gray}{z}}, \bar{J}^{\;\!\textcolor{gray}{t}}_{\;\!\textcolor{Maroon}{\text{f}}\textcolor{gray}{z}}$之间还应满足\bref{eq:continuity-e}以及$\bar{J}^{\;\!\textcolor{gray}{t}}_{\;\!\textcolor{Maroon}{\text{f}}\textcolor{gray}{z}} = {\rho}^{\;\!\textcolor{gray}{t}}_{\;\!\textcolor{Maroon}{\text{f}}\textcolor{gray}{z}} \bar{v}^{\;\!\textcolor{gray}{t}}_{\;\!\textcolor{Maroon}{\text{f}}\textcolor{gray}{z}}$\cite{chen-zhuChenZhuxieUndergraduate_courses2024}。}和光伏电流场$\propto \lvert \bar{E}^{\;\!\textcolor{gray}{t}}_{\;\!\textcolor{gray}{z}} \rvert^2 \hat{c}$\Footnote{与光电导效应并列,属于内光电效应;也可能在光折变效应的$\bar{J}^{\;\!\textcolor{gray}{t}}_{\;\!\textcolor{Maroon}{\text{f}}\textcolor{gray}{z}}$中扮演一份角色,特别是沿着一些各向异性晶体的光轴$\hat{c}$产生电势差和内建电场\cite{boydNonlinearOptics2019}(尽管一般也只影响直流或低频$\Omega$的$\bar{J}^{\;\!\textcolor{gray}{t}}_{\;\!\textcolor{Maroon}{\text{f}}\textcolor{gray}{z}}$;但$\bar{J}^{\;\!\textcolor{gray}{t}}_{\;\!\textcolor{Maroon}{\text{f}}\textcolor{gray}{z}}$会通过影响光波段的介电常数,进而影响光波段的光强$\lvert \bar{E}^{\;\!\textcolor{gray}{t}}_{\;\!\textcolor{gray}{z}} \rvert^2$及$\bar{J}^{\;\!\textcolor{gray}{t}}_{\;\!\textcolor{Maroon}{\text{f}}\textcolor{gray}{z}}$自己的重新分布);该二阶的带耦合的非线性,看上去很像非线性极化率$\bar{P}^{\;\!\textcolor{Maroon}{\text{(2)}} \textcolor{gray}{t}}_{\;\!\textcolor{gray}{z}}$中的光整流项,但其频率比 THz 低,且只服务于自由电流。—— 以至该项可作为差频合并至$\bar{J}^{\;\!\textcolor{gray}{t}}_{\;\!\textcolor{Maroon}{\text{f}}\textcolor{gray}{z}}$关于$\bar{E}^{\;\!\textcolor{gray}{t}}_{\;\!\textcolor{gray}{z}}$的二阶非线性$\bar{J}^{\;\!\textcolor{Maroon}{\text{(2)}} \textcolor{gray}{t}}_{\;\!\textcolor{Maroon}{\text{f}}\textcolor{gray}{z}}$中去?}等其他场量的非线性项 $\bar{J}^{\;\!\textcolor{Maroon}{\text{NL}}, \textcolor{gray}{t}}_{\;\!\textcolor{Maroon}{\text{f}}\textcolor{gray}{z}}$:
\begin{equation} \label{eq:cr-j}
	\textcolor{Maroon}{\text{Ohm's law}}\text{:}\hspace{0.5em} \bar{J}^{\;\!\textcolor{gray}{t}}_{\;\!\textcolor{Maroon}{\text{f}}\textcolor{gray}{z}} = \bar{\bar{\sigma}}^{\;\!\textcolor{Maroon}{\text{(1)}}\textcolor{gray}{t}}_{\;\!\textcolor{gray}{z}}~\widetilde *~\bar{E}^{\;\!\textcolor{gray}{t}}_{\;\!\textcolor{gray}{z}} + \bar{J}^{\;\!\textcolor{Maroon}{\text{NL}}, \textcolor{gray}{t}}_{\;\!\textcolor{Maroon}{\text{f}}\textcolor{gray}{z}} =: \bar{J}^{\;\!\textcolor{Maroon}{\text{(1)}} \textcolor{gray}{t}}_{\;\!\textcolor{Maroon}{\text{f}}\textcolor{gray}{z}} + \bar{J}^{\;\!\textcolor{Maroon}{\text{NL}}, \textcolor{gray}{t}}_{\;\!\textcolor{Maroon}{\text{f}}\textcolor{gray}{z}}~,
\end{equation}

其三,电位移场$\bar{D}^{\;\!\textcolor{gray}{t}}_{\;\!\textcolor{gray}{z}}$的本构关系,当$\bar{P}^{\;\!\textcolor{gray}{t}}_{\;\!\textcolor{gray}{z}}$只由电偶极矩\Footnote{不考虑电四极矩及以上。但电四极化强度场$\bar{\bar{Q}}^{\;\!\textcolor{gray}{t}}_{\;\!\textcolor{gray}{z}}$(的等效电偶极化强度场$\bar{P}^{\;\!\textcolor{gray}{t}}_{\;\!\textcolor{Maroon}{\text{Q}}\textcolor{gray}{z}} = - \bar{\nabla} \cdot \bar{\bar{Q}}^{\;\!\textcolor{gray}{t}}_{\;\!\textcolor{gray}{z}}$)\cite{chen-zhuChenZhuxieUndergraduate_courses2024}在有些效应中不可忽视且起关键作用:如其对线性晶体光学中的光学活性的贡献\cite{nelsonDerivingTransmissionReflection1995},以及非线性光学中基于$\bar{\bar{Q}}^{\;\!\textcolor{gray}{t}}_{\;\!\textcolor{gray}{z}}$的二阶和频\cite{bethuneOpticalQuadrupoleSumfrequency1976}。电四极子对光与物质相互作用的贡献,还会打破$\bar{D}^{\;\!\textcolor{gray}{t}}_{\;\!\textcolor{gray}{z}}$法向连续和$\bar{H}^{\;\!\textcolor{gray}{t}}_{\;\!\textcolor{gray}{z}}$切向连续边界条件,并与洛伦兹力的定义、(由唯二的无源齐次\cite{lakhtakiaGenesisPostConstraint2004}微分方程\bref{eq:maxwell-e,eq:maxwell-b}导出的)电磁场标/矢势\cite{chen-zhuChenZhuxieUndergraduate_courses2024}等一起,使$\bar{E}^{\;\!\textcolor{gray}{t}}_{\;\!\textcolor{gray}{z}},\bar{B}^{\;\!\textcolor{gray}{t}}_{\;\!\textcolor{gray}{z}}$而不是$\bar{E}^{\;\!\textcolor{gray}{t}}_{\;\!\textcolor{gray}{z}},\bar{H}^{\;\!\textcolor{gray}{t}}_{\;\!\textcolor{gray}{z}}$成为基本场\cite{nelsonDerivingTransmissionReflection1995},对应地,坡印亭矢量也需要修正为$\bar{E}^{\;\!\textcolor{gray}{t}}_{\;\!\textcolor{gray}{z}} \times \bar{B}^{\;\!\textcolor{gray}{t}}_{\;\!\textcolor{gray}{z}} \big/ {\symup{\mu}}_0$\cite{nelsonGeneralizingPoyntingVector1996,loudonPropagationElectromagneticEnergy1997,richterPoyntingsTheoremEnergy2008}而不是$\bar{E}^{\;\!\textcolor{gray}{t}}_{\;\!\textcolor{gray}{z}} \times \bar{H}^{\;\!\textcolor{gray}{t}}_{\;\!\textcolor{gray}{z}}$。}构成时,定义为
\begin{subequations} \label{eq:cr-d}
\begin{align}
	\textcolor{Maroon}{\text{CR for electricity}}\text{:}&\hspace{0.5em} \bar{D}^{\;\!\textcolor{gray}{t}}_{\;\!\textcolor{gray}{z}} \hspace{-2.0em} &&= {\symup{\varepsilon}}_0 \bar{E}^{\;\!\textcolor{gray}{t}}_{\;\!\textcolor{gray}{z}} + \bar{P}^{\;\!\textcolor{gray}{t}}_{\;\!\textcolor{gray}{z}} = {\symup{\varepsilon}}_0 \bar{\bar{\delta}}^{\;\!\textcolor{gray}{t}}~\widetilde *~\bar{E}^{\;\!\textcolor{gray}{t}}_{\;\!\textcolor{gray}{z}} + \bar{P}^{\;\!\textcolor{gray}{t}}_{\;\!\textcolor{gray}{z}} \label{cr-d1} \\ & &&\xrightarrow[]{\bar{P}^{\;\!\textcolor{gray}{t}}_{\;\!\textcolor{gray}{z}} = \bar{P}^{\;\!\textcolor{Maroon}{\text{(1)}} \textcolor{gray}{t}}_{\;\!\textcolor{gray}{z}} + \bar{P}^{\;\!\textcolor{Maroon}{\text{NL}}, \textcolor{gray}{t}}_{\;\!\textcolor{gray}{z}} + } \left[ {\symup{\varepsilon}}_0 \bar{\bar{\delta}}^{\;\!\textcolor{gray}{t}}~\widetilde *~\bar{E}^{\;\!\textcolor{gray}{t}}_{\;\!\textcolor{gray}{z}} + \bar{P}^{\;\!\textcolor{Maroon}{\text{(1)}} \textcolor{gray}{t}}_{\;\!\textcolor{gray}{z}} \right] + \bar{P}^{\;\!\textcolor{Maroon}{\text{NL}}, \textcolor{gray}{t}}_{\;\!\textcolor{gray}{z}} \label{cr-d2} \\ & &&\xrightarrow[\displaystyle{ \bar{\bar{\varepsilon}}^{\;\!\textcolor{Maroon}{\text{(1)}} \textcolor{gray}{t}}_{\;\!\textcolor{Maroon}{\text{r}}\textcolor{gray}{z}} := \bar{\bar{\delta}}^{\;\!\textcolor{gray}{t}} + \bar{\bar{\chi}}^{\;\!\textcolor{Maroon}{\text{(1)}}\textcolor{gray}{t}}_{\;\!\textcolor{Maroon}{\text{f}} \textcolor{gray}{z}}}]{\displaystyle{\bar{P}^{\;\!\textcolor{Maroon}{\text{(1)}} \textcolor{gray}{t}}_{\;\!\textcolor{gray}{z}} := \bar{\bar{\chi}}^{\;\!\textcolor{Maroon}{\text{(1)}}\textcolor{gray}{t}}_{\;\!\textcolor{Maroon}{\text{f}} \textcolor{gray}{z}} ~\widetilde *~\bar{E}^{\;\!\textcolor{gray}{t}}_{\;\!\textcolor{gray}{z}}}} {\symup{\varepsilon}}_0 \bar{\bar{\varepsilon}}^{\;\!\textcolor{Maroon}{\text{(1)}} \textcolor{gray}{t}}_{\;\!\textcolor{Maroon}{\text{r}}\textcolor{gray}{z}}~\widetilde *~\bar{E}^{\;\!\textcolor{gray}{t}}_{\;\!\textcolor{gray}{z}} + \bar{P}^{\;\!\textcolor{Maroon}{\text{NL}}, \textcolor{gray}{t}}_{\;\!\textcolor{gray}{z}} \label{cr-d3} \\ & &&= \bar{\bar{\varepsilon}}^{\;\!\textcolor{Maroon}{\text{(1)}} \textcolor{gray}{t}}_{\;\!\textcolor{gray}{z}}~\widetilde *~\bar{E}^{\;\!\textcolor{gray}{t}}_{\;\!\textcolor{gray}{z}} + \bar{P}^{\;\!\textcolor{Maroon}{\text{NL}}, \textcolor{gray}{t}}_{\;\!\textcolor{gray}{z}} =: \bar{D}^{\;\!\textcolor{Maroon}{\text{(1)}} \textcolor{gray}{t}}_{\;\!\textcolor{gray}{z}} + \bar{D}^{\;\!\textcolor{Maroon}{\text{NL}}, \textcolor{gray}{t}}_{\;\!\textcolor{gray}{z}}~, \label{cr-d4}
\end{align}
\end{subequations}
关于其组成成分,电位移场 $\bar{D}^{\;\!\textcolor{gray}{t}}_{\;\!\textcolor{gray}{z}}$(直接/显示地)关于电场 $\bar{E}^{\;\!\textcolor{gray}{t}}_{\;\!\textcolor{gray}{z}}$\Footnote{电非线性,包括高频段的(非)共振非线性、低频低温\cite{lakhtakiaGenesisPostConstraint2004}段的铁电体/畴的电滞现象等。}、磁场 $\bar{H}^{\;\!\textcolor{gray}{t}}_{\;\!\textcolor{gray}{z}}$\Footnote{双各向异性中的磁$\to$电耦合(如果 $\bar{D}^{\;\!\textcolor{gray}{t}}_{\;\!\textcolor{gray}{z}}$ 中的该部分只是 $\bar{H}^{\;\!\textcolor{gray}{t}}_{\;\!\textcolor{gray}{z}}$ 的线性函数,则也可归结到线性项中)。}、应力 $\bar{T}^{\;\!\textcolor{gray}{t}}_{\;\!\textcolor{gray}{z}}$\Footnote{正逆压磁/磁致伸缩/磁弹效应(这里未作区分)。}等其他场量的非线性函数项,均由 $\bar{D}^{\;\!\textcolor{Maroon}{\text{NL}}, \textcolor{gray}{t}}_{\;\!\textcolor{gray}{z}} = {\symup{\mu}}_0 \bar{M}^{\;\!\textcolor{Maroon}{\text{NL}}, \textcolor{gray}{t}}_{\;\!\textcolor{gray}{z}}$ 贡献;剩余的线性项,由 $\bar{B}^{\;\!\textcolor{Maroon}{\text{(1)}} \textcolor{gray}{t}}_{\;\!\textcolor{gray}{z}} = \bar{\bar{\mu}}^{\;\!\textcolor{Maroon}{\text{(1)}} \textcolor{gray}{t}}_{\;\!\textcolor{gray}{z}}~\widetilde *~\bar{H}^{\;\!\textcolor{gray}{t}}_{\;\!\textcolor{gray}{z}}$ 表示。


