\marginLeft[-2.4em]{chap:N/LCO}\chapter{晶体中的(非)线性光学过程}\label{chap:N/LCO}

\vspace*{-8.5em}

\marginLeft[-2.4em]{sec:maxwell}\section{\textcolor{Maroon}{Maxwell-Lorentz-Heaviside} 方程组:$\text{场} = f(\text{源})$}\label{sec:maxwell}

该节从经典/现代的多极理论视角,查看场关于源的 \textcolor{Maroon}{Maxwell-Lorentz-Heaviside} 方程组、源的守恒律、场的连续性、场/源的表面/体分布。——对应“物质告诉时空怎么弯曲,时空告诉物质怎么运动”的前半句:即“源告诉场怎么弯曲”。

\vspace*{-5.0em}

\marginLeft[-1.4em]{ssec:EHpJf}\subsection{$\bar{E},\bar{H}$ 双旋度、${\rho}_{\;\!\textcolor{Maroon}{\text{f}}}, \bar{J}_{\;\!\textcolor{Maroon}{\text{f}}}$ 裸源连续}\label{ssec:EHpJf}

相对观察者静止\Footnote{否则标/矢/张量场的2个实自变量:时间$\mathcolor{gray}{t} \in \mathcolor{gray}{\mathbb{R}_1}$和空间$\mathcolor{gray}{\bar{r}} \in \mathcolor{gray}{\bar{\mathbb{R}}^{\textcolor{Maroon}{(1)}}_3}$将在其他参考系(见\bref{uwav})的度量下相互混合,即$\mathcolor{gray}{\uwav{t}} \left( \mathcolor{gray}{t}, \mathcolor{gray}{\bar{r}} \right), \mathcolor{gray}{\uwav{\bar{r}}} \left( \mathcolor{gray}{t}, \mathcolor{gray}{\bar{r}} \right)$,并形成统一的$1+3$维黎曼时空/微分流形$\mathcolor{gray}{\uwav{\bar{x}}} \left( \mathcolor{gray}{\bar{x}} \right) \in \mathcolor{gray}{\bar{\mathbb{R}}_4}$;接着,定义四维位移矢量$\mathcolor{gray}{\bar{x}}$对四维标量固有时$\mathcolor{gray}{\uo{t}}$(见\bref{uo})的导数:四维速度矢量$\bar{u}_{\;\!\mathcolor{gray}{\bar{x}}} := \mathbb{d} \mathcolor{gray}{\bar{x}} \big/ \mathbb{d} \mathcolor{gray}{\uo{t}} \in \bar{\mathbb{C}}_4 ( \mathcolor{gray}{\bar{\mathbb{R}}_4} )$,并将总电流 $\bar{J}^{\;\!\mathcolor{gray}{t}}_{\;\!\mathcolor{gray}{z}} = {\rho}^{\;\!\mathcolor{gray}{t}}_{\;\!\mathcolor{gray}{z}} \bar{v}^{\;\!\mathcolor{gray}{t}}_{\;\!\mathcolor{gray}{z}}$ 与总电荷${\rho}^{\;\!\mathcolor{gray}{t}}_{\;\!\mathcolor{gray}{z}}$合并以扩展至四维$\bar{J}_{\;\! \mathcolor{gray}{\bar{x}}} = \uo{\rho}_{\;\! \mathcolor{gray}{\uo{\bar{x}}}} \bar{u}_{\;\! \mathcolor{gray}{\bar{x}}}  \in \bar{\mathbb{C}}_4 ( \mathcolor{gray}{\bar{\mathbb{R}}_4} )$\cite{lakhtakiaCovariancesInvariancesMaxwell1995,chen-zhuChenZhuxieUndergraduate_courses2024};此外,2个基本场 $\bar{E}_{\;\!\mathcolor{gray}{\bar{x}}}$ 与 $\bar{B}_{\;\!\mathcolor{gray}{\bar{x}}}$ 也将相互耦合,即$\uwav{\bar{E}}_{\;\!\mathcolor{gray}{\uwav{\bar{x}}}} \left( \bar{E}_{\;\!\mathcolor{gray}{\bar{x}}},\bar{B}_{\;\!\mathcolor{gray}{\bar{x}}} \right), \uwav{\bar{B}}_{\;\!\mathcolor{gray}{\uwav{\bar{x}}}} \left( \bar{E}_{\;\!\mathcolor{gray}{\bar{x}}},\bar{B}_{\;\!\mathcolor{gray}{\bar{x}}} \right)$,并成为一个整体:2阶4维电磁张量场 $\uwav{\bar{\bar{F}}}_{\;\!\mathcolor{gray}{\uwav{\bar{x}}}} ( \bar{\bar{F}}_{\;\!\mathcolor{gray}{\bar{x}}} ) \in \bar{\bar{\mathbb{C}}}_{\left[4 \times 4\right]} ( \bar{\mathbb{R}}_4 )$\cite{lakhtakiaCovariancesInvariancesMaxwell1995,berryOpticalSingularitiesBianisotropic2005,chen-zhuChenZhuxieUndergraduate_courses2024},以致材料的本构关系将呈现固有双各向异性$\uwav{\bar{D}}_{\;\!\mathcolor{gray}{\uwav{\bar{x}}}} [ \uwav{\bar{E}}_{\;\!\mathcolor{gray}{\uwav{\bar{x}}}} \left( \bar{E}_{\;\!\mathcolor{gray}{\bar{x}}},\bar{B}_{\;\!\mathcolor{gray}{\bar{x}}} \right) ], \uwav{\bar{H}}_{\;\!\mathcolor{gray}{\uwav{\bar{x}}}} [ \uwav{\bar{B}}_{\;\!\mathcolor{gray}{\uwav{\bar{x}}}} \left( \bar{E}_{\;\!\mathcolor{gray}{\bar{x}}},\bar{B}_{\;\!\mathcolor{gray}{\bar{x}}} \right) ]$或$\uwav{\bar{\bar{G}}}_{\;\!\mathcolor{gray}{\uwav{\bar{x}}}} [ \uwav{\bar{\bar{F}}}_{\;\!\mathcolor{gray}{\uwav{\bar{x}}}} ( \bar{\bar{F}}_{\;\!\mathcolor{gray}{\bar{x}}} ) ] = \uwav{\bar{\bar{G}}}_{\;\!\mathcolor{gray}{\uwav{\bar{x}}}} [ \bar{\bar{G}}_{\;\!\mathcolor{gray}{\bar{x}}} ( \bar{\bar{F}}_{\;\!\mathcolor{gray}{\bar{x}}} ) ]$\cite{langeMultipoleTheoryHehl2015,hehlLinearMediaClassical2005,mackayElectromagneticAnisotropyBianisotropy2019,mackayModernAnalyticalElectromagnetic2020,lakhtakiaCovariancesInvariancesMaxwell1995}。此外,${\rho}^{\;\!\mathcolor{gray}{t}}_{\;\!\mathcolor{gray}{z}}$对应的束缚/自由=价带/导带电子的(有效)运动质量(或相对论速度)也会变大,运动轨迹改变\cite{boydNonlinearOptics2019},并因此同时对线性和非线性极化、磁化、自由电流$\bar{J}^{\;\!\mathcolor{gray}{t}}_{\;\!\textcolor{Maroon}{\text{f}}\mathcolor{gray}{z}} = {\rho}^{\;\!\mathcolor{gray}{t}}_{\;\!\textcolor{Maroon}{\text{f}}\mathcolor{gray}{z}} \bar{v}^{\;\!\mathcolor{gray}{t}}_{\;\!\textcolor{Maroon}{\text{f}}\mathcolor{gray}{z}}$产生额外影响。——上述场景可发生在超快和强场泵浦(及其通过多光子/隧穿电离产生的等离子体)中\cite{boydNonlinearOptics2019}、相对材料运动的坐标系下,甚至一直在发生在所有材料内(核库伦力导致的电子轨道运动:金的颜色\cite{boydNonlinearOptics2019}、汞常温液体\cite{pyykkoRelativisticEffectsChemistry2012}、铅酸电池的额外电压和稳定性\cite{ahujaRelativityLeadacidBattery2011};狄拉克方程描述的电子自旋运动、自旋-轨道、自旋-自旋耦合\cite{pyykkoRelativisticEffectsChemistry2012}:许多磁效应\cite{chen-zhuChenZhuxieUndergraduate_courses2024}),以至于可能(必)需要引入相对论(量子)电动力学——电子一直在材料内运动,尽管材料本身相对观察者的参考系静止。此外,所有的磁效应的本质和来源,似乎都可以通过纯电的方式解释,见 \bref{ssec:EBpJ,ssec:PMQN}。} 的 3 维空间$\mathcolor{gray}{\bar{r}}$ 坐标系下,在可能存在非零的时 $\mathcolor{gray}{t}$ 变自由电荷源和自由电流源${\rho}_{\;\!\textcolor{Maroon}{\text{f}}} \left( \mathcolor{gray}{\bar{r}}, \mathcolor{gray}{t} \right), \bar{J}_{\;\!\textcolor{Maroon}{\text{f}}} \left( \mathcolor{gray}{\bar{r}}, \mathcolor{gray}{t} \right)$\Footnote{对于符号约定,比如下标 $\textcolor{Maroon}{\text{f}}$ 的\textcolor{Maroon}{褐红色}及其含义 `\textcolor{Maroon}{free}',其定义见\bref{Maroon};此外,在 \bref{1bar} 中还约定:总使用 1 条上短横线 $\bar{~}$(而不是粗体)来表示矢量(如 $\bar{J}_{\;\!\textcolor{Maroon}{\text{f}}}$),以区别于 \bref{0bar} 中定义的无上短横线的标量(如 ${\rho}_{\;\!\textcolor{Maroon}{\text{f}}}$);粗体在本文中另有其含义,不用于表示矢量,见\bref{bold}。} 的,相对静止的一般电磁介质内部,4 个空域时变\Footnote{指复矢量场 $\bar{E}^{\;\!\mathcolor{gray}{t}}_{\;\!\mathcolor{gray}{z}}, \bar{H}^{\;\!\mathcolor{gray}{t}}_{\;\!\mathcolor{gray}{z}}, \bar{D}^{\;\!\mathcolor{gray}{t}}_{\;\!\mathcolor{gray}{z}}, \bar{B}^{\;\!\mathcolor{gray}{t}}_{\;\!\mathcolor{gray}{z}}$ 均是四维时空 $\mathcolor{gray}{\bar{r}}, \mathcolor{gray}{t}$ 的函数,且因此一般意义上是复色 $\left\{ \mathcolor{gray}{\omega} \in \mathcolor{gray}{\mathbb{R}} \right\}$ 的复场 $\in \bar{\mathbb{C}}_3 ( \mathcolor{gray}{\bar{\mathbb{R}}_{3+1}} )$(认为这四者必须为实场\cite{boydNonlinearOptics2019}也没关系:它们对$\mathcolor{gray}{t}$的傅立叶变换\bref{eq:FT-tw}所得到的$\pm \omega$单色子波是复共轭的,以至于正负频率的对应复子波求和后会消掉虚部,只剩下实部的余弦$\cos$实子波,因此在正/倒空间中的总/子场均是有物理意义的实场),属于在视觉上占主导的因变量,并用黑色(见 \bref{black})的 \textit{斜体 oblique}(见 \bref{oblique})表示;相对地,自变量用视觉和含义上均更次要的\textcolor{gray}{灰色}来表示,见 \bref{gray}。}复色场 
$\bar{E}^{\;\!\mathcolor{gray}{t}}_{\;\!\mathcolor{gray}{z}}, \bar{H}^{\;\!\mathcolor{gray}{t}}_{\;\!\mathcolor{gray}{z}}, \bar{D}^{\;\!\mathcolor{gray}{t}}_{\;\!\mathcolor{gray}{z}}, \bar{B}^{\;\!\mathcolor{gray}{t}}_{\;\!\mathcolor{gray}{z}}$\Footnote{由于傅立叶光学一般运行在平行平面间,约定下述表示相互等价:$\bar{E} \left( \mathcolor{gray}{\bar{r}}, \mathcolor{gray}{t} \right) = \bar{E}^{\;\!\mathcolor{gray}{t}}_{\;\!\mathcolor{gray}{\bar{r}}} = \bar{E}^{\;\!\mathcolor{gray}{t}}_{\;\!\mathcolor{gray}{z}} \left( \mathcolor{gray}{\bar{\rho}} \right)$,并因此经常省略面内自变量 $\mathcolor{gray}{\bar{\rho}}$,以只写作朝 $\textcolor{Maroon}{+\symup{z}}$ 轴传播距离 $\mathcolor{gray}{z}$ 的函数 $\bar{E}^{\;\!\mathcolor{gray}{t}}_{\;\!\mathcolor{gray}{z}} := \bar{E}^{\;\!\mathcolor{gray}{t}}_{\;\!\mathcolor{gray}{z}} \left( \mathcolor{gray}{\bar{\rho}} \right)$。—— 同样的规则也适用于其他场量(如 ${\rho}_{\;\!\textcolor{Maroon}{\text{f}}} \left( \mathcolor{gray}{\bar{r}}, \mathcolor{gray}{t} \right) \to {\rho}^{\;\!\mathcolor{gray}{t}}_{\;\!\textcolor{Maroon}{\text{f}}\mathcolor{gray}{z}}$),且适用于空间频率域,见。},满足微分形式\Footnote{尽管是微分形式,仍然处于(相对的)宏观层面:典型的光波长 $1$um 是原子特征尺寸 $1\text{\r{A}} = 0.1$nm 的 $10^4$ 倍,因此\bref{eq:Curl-EH,eq:Div-DB,eq:Div-em-f}涉及的所有物理量均是空间平均后的结果\cite{mackayElectromagneticAnisotropyBianisotropy2019};如果要引入(非)线性极/磁化强度/率随考虑区域尺度的缩放,则需要 \textcolor{Maroon}{Clausius-Mossotti equation} 或 \textcolor{Maroon}{Lorentz-Lorenz law} 的局域场修正\cite{boydNonlinearOptics2019}。}的麦氏方程组的 2 个旋度假设\Footnote{使用爱因斯坦求和约定(2 个相同指标相遇则主体求和,但 2 个 同侧角标不适用\cite{frankelGeometryPhysicsIntroduction2011}),定义了空域 3 维向量微分算子 $\mathcolor{gray}{\bar{\nabla}} := \mathcolor{gray}{\bar{\nabla}_{\bar{r}}} := \hat{\symup{e}}_{\symup{\iota}} \mathcolor{gray}{\nabla^\iota}$,其中 $\symup{\iota} = \symup{x,y,z}$,且 $\hat{\symup{e}}_{\symup{x}},\hat{\symup{e}}_{\symup{y}},\hat{\symup{e}}_{\symup{z}}$ 分别为 $\symup{x,y,z}$ 方向的单位定矢(“定矢$\hat{\symup{e}}$”也“直体”:见 \bref{upright})。对于旋度\&叉乘运算,还可写成轴矢量构成的反对称二阶张量的矩阵乘积:$\mathcolor{gray}{\bar{\nabla} \times} \bar{v} = \mathcolor{gray}{\bar{\nabla}^\times \cdot} \bar{v}$\cite{ossikovskiConstitutiveRelationsOptically2021},并且矩阵乘积可以省略 `$\cdot$' 而写为 $\mathcolor{gray}{\bar{\nabla}^\times} \bar{v} = \mathcolor{gray}{\bar{\nabla}^\times \cdot} \bar{v}$。此外,定义了 \textcolor{Maroon}{\text{Levi-Civita symbol}} $\epsilon^{\hphantom{\symup{\iota}\hat{1}}\hat{2}}_{\symup{\iota}\mathcolor{gray}{\hat{1}}}$、空域偏导算符 $\mathcolor{gray}{\nabla^{\hat{1}}} := \mathcolor{gray}{ \partial \mathcolor{black}{\underline{~~}} \big/ \partial \mathcolor{gray}{j} }$(其中,黑色的 $i,j,k$ 取遍 $\symup{x},\symup{y},\symup{z}$、灰色的 $\mathcolor{gray}{i},\mathcolor{gray}{j},\mathcolor{gray}{k}$ 取遍 $\mathcolor{gray}{x},\mathcolor{gray}{y},\mathcolor{gray}{z}$)、时域偏导算符 $\mathcolor{gray}{\nabla^t} \underline{~~} := \mathcolor{gray}{\partial \mathcolor{black}{\underline{~~}} \big/ \partial t}$。}:
\begin{subequations} \label{eq:Curl-EH}
	\abovedisplayskip=-8pt
\begin{align}
	\textcolor{Maroon}{\text{Faraday's law of electromagnetic induction}}\text{:}&\hspace{1.2em} \mathcolor{gray}{\bar{\nabla} \times} \bar{E}^{\;\!\mathcolor{gray}{t}}_{\;\!\mathcolor{gray}{z}} + \mathcolor{gray}{\nabla^t} \bar{B}^{\;\!\mathcolor{gray}{t}}_{\;\!\mathcolor{gray}{z}} \hspace{-0.9em} &&= - \bar{K}^{\;\!\mathcolor{gray}{t}}_{\;\!\textcolor{Maroon}{\text{f}}\mathcolor{gray}{z}} \label{eq:Curl-EK} \\ 
	&\epsilon^{\hphantom{\symup{\iota}\hat{1}}\hat{2}}_{\symup{\iota}\mathcolor{gray}{\hat{1}}} \mathcolor{gray}{\nabla^{\hat{1}}} E^{\;\!\mathcolor{gray}{t}}_{\;\! \hat{2}\mathcolor{gray}{z}} + \mathcolor{gray}{\nabla^t} B^{\;\!\mathcolor{gray}{t}}_{\;\! \symup{\iota}\mathcolor{gray}{z}} \hspace{-0.9em} &&= - K^{\;\!\mathcolor{gray}{t}}_{\;\! \textcolor{Maroon}{\text{f}} \symup{\iota}\mathcolor{gray}{z}}~, \label{eq:curl-EK} \\ 
	\textcolor{Maroon}{\text{Jefimenko's}} \to \textcolor{Maroon}{\text{Amp\`{e}re-Maxwell circuital law}}\text{:}&\hspace{1.2em} \mathcolor{gray}{\bar{\nabla} \times} \bar{H}^{\;\!\mathcolor{gray}{t}}_{\;\!\mathcolor{gray}{z}} - \mathcolor{gray}{\nabla^t} \bar{D}^{\;\!\mathcolor{gray}{t}}_{\;\!\mathcolor{gray}{z}} \hspace{-0.9em} &&= \bar{J}^{\;\!\mathcolor{gray}{t}}_{\;\!\textcolor{Maroon}{\text{f}}\mathcolor{gray}{z}} \label{eq:Curl-H} \\ 
	&\epsilon^{\hphantom{\symup{\iota}\hat{1}}\hat{2}}_{\symup{\iota}\mathcolor{gray}{\hat{1}}} \mathcolor{gray}{\nabla^{\hat{1}}} H^{\;\!\mathcolor{gray}{t}}_{\;\! \hat{2}\mathcolor{gray}{z}} - \mathcolor{gray}{\nabla^t} D^{\;\!\mathcolor{gray}{t}}_{\;\! \symup{\iota}\mathcolor{gray}{z}} \hspace{-0.9em} &&= J^{\;\!\mathcolor{gray}{t}}_{\;\! \textcolor{Maroon}{\text{f}} \symup{\iota}\mathcolor{gray}{z}}~. \label{eq:curl-H}
\end{align}
\end{subequations}
以及 2 个散度假设\Footnote{对于散度\&点积,也可以使用“行向量·列向量”的矩阵乘法,如 $\mathcolor{gray}{\bar{\nabla}^\intercal} \bar{v} = \mathcolor{gray}{\bar{\nabla} \cdot} \bar{v}$。此外,高阶散度/点积(如 \bref{eq:P-b} 中的 $\mathcolor{gray}{\bar{\nabla}} \mathcolor{gray}{\bar{\nabla} \colon}\! \bar{\bar{Q}}^{\;\!\mathcolor{gray}{t}}_{\;\!\mathcolor{gray}{z}}$)原则上也可用 $\mathcolor{gray}{\bar{\nabla} \cdot} ( \mathcolor{gray}{\bar{\nabla} \cdot} \bar{\bar{Q}}^{\;\!\mathcolor{gray}{t}}_{\;\!\mathcolor{gray}{z}} ) \neq \mathcolor{gray}{( \bar{\nabla} \cdot \bar{\nabla} )} \bar{\bar{Q}}^{\;\!\mathcolor{gray}{t}}_{\;\!\mathcolor{gray}{z}} = \mathcolor{gray}{\bar{\nabla}^2} \bar{\bar{Q}}^{\;\!\mathcolor{gray}{t}}_{\;\!\mathcolor{gray}{z}} = \mathcolor{gray}{\nabla^2} \bar{\bar{Q}}^{\;\!\mathcolor{gray}{t}}_{\;\!\mathcolor{gray}{z}}$ 或 $\mathcolor{gray}{\left( \bar{\nabla} \bar{\nabla} \right)^\intercal} \! \bar{\bar{Q}}^{\;\!\mathcolor{gray}{t}}_{\;\!\mathcolor{gray}{z}} = \mathcolor{gray}{\bar{\nabla}} ( \mathcolor{gray}{\bar{\nabla}^\intercal} \bar{\bar{Q}}^{\;\!\mathcolor{gray}{t}}_{\;\!\mathcolor{gray}{z}} )^\intercal$ 表示,但对于后者中高阶张量$\mathcolor{gray}{\bar{\nabla}} \mathcolor{gray}{\bar{\nabla}}$的转置需要指定(哪)两个维度。此外,点积 $\cdot$ 也可视为对应元素积=哈达玛积 $\odot$ 后,再对所有元素求和。}:
\begin{subequations} \label{eq:Div-DB}
\begin{align}
	\textcolor{Maroon}{\text{Coulomb's}} \to \textcolor{Maroon}{\text{Gauss's law for electricity}}\text{:}&\hspace{0.5em} \mathcolor{gray}{\bar{\nabla} \cdot} \bar{D}^{\;\!\mathcolor{gray}{t}}_{\;\!\mathcolor{gray}{z}} \hspace{-3.8em} &&= {\rho}^{\;\!\mathcolor{gray}{t}}_{\;\!\textcolor{Maroon}{\text{f}}\mathcolor{gray}{z}} \label{eq:Div-D} \\ 
	&\hspace{0.5em} \mathcolor{gray}{\nabla^\iota} D^{\;\!\mathcolor{gray}{t}}_{\;\! \symup{\iota}\mathcolor{gray}{z}} \hspace{-3.8em} &&= {\rho}^{\;\!\mathcolor{gray}{t}}_{\;\!\textcolor{Maroon}{\text{f}}\mathcolor{gray}{z}}~, \label{eq:div-D} \\
	\textcolor{Maroon}{\text{Biot-Savart}} \to \textcolor{Maroon}{\text{Gauss's law for magnetism}}\text{:}&\hspace{0.5em} \mathcolor{gray}{\bar{\nabla} \cdot} \bar{B}^{\;\!\mathcolor{gray}{t}}_{\;\!\mathcolor{gray}{z}} \hspace{-3.8em} &&= {\kappa}^{\;\!\mathcolor{gray}{t}}_{\;\!\textcolor{Maroon}{\text{f}}\mathcolor{gray}{z}} \label{eq:Div-Bk} \\
	&\hspace{0.5em} \mathcolor{gray}{\nabla^\iota} B^{\;\!\mathcolor{gray}{t}}_{\;\! \symup{\iota}\mathcolor{gray}{z}} \hspace{-3.8em} &&= {\kappa}^{\;\!\mathcolor{gray}{t}}_{\;\!\textcolor{Maroon}{\text{f}}\mathcolor{gray}{z}}~. \label{eq:div-Bk}
\end{align}
\end{subequations}
其中,为数学形式对称(以方便引入相对论效应和检验其协变性\cite{lakhtakiaCovariancesInvariancesMaxwell1995,chen-zhuChenZhuxieUndergraduate_courses2024}),和物理上不排除可能存在的磁单极子,除自由电(荷/流)源${\rho}^{\;\!\mathcolor{gray}{t}}_{\;\!\textcolor{Maroon}{\text{f}}\mathcolor{gray}{z}}, \bar{J}^{\;\!\mathcolor{gray}{t}}_{\;\!\textcolor{Maroon}{\text{f}}\mathcolor{gray}{z}}$外,还添加了自由磁(荷/流)源${\kappa}^{\;\!\mathcolor{gray}{t}}_{\;\!\textcolor{Maroon}{\text{f}}\mathcolor{gray}{z}}, \bar{K}^{\;\!\mathcolor{gray}{t}}_{\;\!\textcolor{Maroon}{\text{f}}\mathcolor{gray}{z}}$\cite{lakhtakiaCovariancesInvariancesMaxwell1995}。这 4 个自由源(体密度)项,满足 2 个连续性假设\cite{mackayElectromagneticAnisotropyBianisotropy2019,lakhtakiaCovariancesInvariancesMaxwell1995,chen-zhuChenZhuxieUndergraduate_courses2024}:
\begin{subequations} \label{eq:Div-em-f}
\begin{align}
	\textcolor{Maroon}{\text{Continuity for free electric charge}}\text{:}&\hspace{0.5em} \mathcolor{gray}{\bar{\nabla} \cdot} \bar{J}^{\;\!\mathcolor{gray}{t}}_{\;\!\textcolor{Maroon}{\text{f}}\mathcolor{gray}{z}} + \mathcolor{gray}{\nabla^t} {\rho}^{\;\!\mathcolor{gray}{t}}_{\;\!\textcolor{Maroon}{\text{f}}\mathcolor{gray}{z}} \hspace{-5.3em} &&= 0 \label{eq:Div-e-f} \\ 
	&\hspace{0.5em} \mathcolor{gray}{\nabla^\iota} J^{\;\!\mathcolor{gray}{t}}_{\;\!\textcolor{Maroon}{\text{f}} \symup{\iota}\mathcolor{gray}{z}} + \mathcolor{gray}{\nabla^t} {\rho}^{\;\!\mathcolor{gray}{t}}_{\;\!\textcolor{Maroon}{\text{f}}\mathcolor{gray}{z}} \hspace{-5.3em} &&= 0~, \label{eq:div-e-f} \\ 
	\textcolor{Maroon}{\text{Continuity for free magnetic charge}}\text{:}&\hspace{0.5em} \mathcolor{gray}{\bar{\nabla} \cdot} \bar{K}^{\;\!\mathcolor{gray}{t}}_{\;\!\textcolor{Maroon}{\text{f}}\mathcolor{gray}{z}} + \mathcolor{gray}{\nabla^t} {\kappa}^{\;\!\mathcolor{gray}{t}}_{\;\!\textcolor{Maroon}{\text{f}}\mathcolor{gray}{z}} \hspace{-5.3em} &&= 0 \label{eq:Div-m-f} \\
	&\hspace{0.5em} \mathcolor{gray}{\nabla^\iota} K^{\;\!\mathcolor{gray}{t}}_{\;\!\textcolor{Maroon}{\text{f}} \symup{\iota}\mathcolor{gray}{z}} + \mathcolor{gray}{\nabla^t} {\kappa}^{\;\!\mathcolor{gray}{t}}_{\;\!\textcolor{Maroon}{\text{f}}\mathcolor{gray}{z}} \hspace{-5.3em} &&= 0~. \label{eq:div-m-f}
\end{align}
\end{subequations}
注意,对旋度 \bref{eq:Curl-EH} 两边取散度($\mathcolor{gray}{\bar{\nabla} \cdot}$),连续性 \bref{eq:Div-em-f} 可导出散度 \bref{eq:Div-DB},反之亦然\cite{lakhtakiaGenesisPostConstraint2004}。因此,2 条散度方程均不是必需的,可将其视为冗余\Footnote{并且不应简单地仅根据$\bar{D}^{\;\!\mathcolor{gray}{t}}_{\;\!\mathcolor{gray}{z}}, \bar{B}^{\;\!\mathcolor{gray}{t}}_{\;\!\mathcolor{gray}{z}}$的横向性,而将二者视为基本场\cite{quesadaPhotonPairsNonlinear2022,berryOpticalSingularitiesBianisotropic2005}。但从场能量体密度变化率$\bar{E}^{\;\!\mathcolor{gray}{t}}_{\;\!\mathcolor{gray}{z}} \mathbb{d}\bar{D}^{\;\!\mathcolor{gray}{t}}_{\;\!\mathcolor{gray}{z}} +  \bar{H}^{\;\!\mathcolor{gray}{t}}_{\;\!\mathcolor{gray}{z}} \mathbb{d}\bar{B}^{\;\!\mathcolor{gray}{t}}_{\;\!\mathcolor{gray}{z}}$中含有 2 个旋度\bref{eq:Curl-EH}中对$\bar{D}^{\;\!\mathcolor{gray}{t}}_{\;\!\mathcolor{gray}{z}}, \bar{B}^{\;\!\mathcolor{gray}{t}}_{\;\!\mathcolor{gray}{z}}$的微分\cite{chen-zhuChenZhuxieUndergraduate_courses2024}、方便引入适用于非线性量子光学的正确的哈密顿量\cite{quesadaPhotonPairsNonlinear2022},将$\bar{D}^{\;\!\mathcolor{gray}{t}}_{\;\!\mathcolor{gray}{z}}, \bar{B}^{\;\!\mathcolor{gray}{t}}_{\;\!\mathcolor{gray}{z}}$视为基本场也有一定道理?但如此一来,电非线性 $\bar{E}^{\;\!\mathcolor{gray}{t}}_{\;\!\mathcolor{gray}{z}} \left( \bar{D}^{\;\!\mathcolor{gray}{t}}_{\;\!\mathcolor{gray}{z}} \right) \slashed{\propto} \bar{D}^{\;\!\mathcolor{gray}{t}}_{\;\!\mathcolor{gray}{z}}$ 的显式表达式有悖常理。}。

\clearpage
%\XGap{-10em}
\vspace*{-8.5em}

\marginLeft[-2.4em]{ssec:EBpJ}\subsection{$\bar{E},\bar{B}$ 基本场、${\rho}, \bar{J}$ 总源连续}\label{ssec:EBpJ}

现代电磁学/电动力学追根溯源地\Footnote{从 $\bar{E}^{\;\!\mathcolor{gray}{t}}_{\;\!\mathcolor{gray}{z}}, \bar{B}^{\;\!\mathcolor{gray}{t}}_{\;\!\mathcolor{gray}{z}}$ 出发又回到 $\bar{E}^{\;\!\mathcolor{gray}{t}}_{\;\!\mathcolor{gray}{z}}, \bar{B}^{\;\!\mathcolor{gray}{t}}_{\;\!\mathcolor{gray}{z}}$,先升格又降格 $\bar{E}^{\;\!\mathcolor{gray}{t}}_{\;\!\mathcolor{gray}{z}}, \bar{B}^{\;\!\mathcolor{gray}{t}}_{\;\!\mathcolor{gray}{z}} \to \bar{D}^{\;\!\mathcolor{gray}{t}}_{\;\!\mathcolor{gray}{z}}, \bar{H}^{\;\!\mathcolor{gray}{t}}_{\;\!\mathcolor{gray}{z}} \to \bar{E}^{\;\!\mathcolor{gray}{t}}_{\;\!\mathcolor{gray}{z}}, \bar{B}^{\;\!\mathcolor{gray}{t}}_{\;\!\mathcolor{gray}{z}}$,绕了一个大弯。},将 $\bar{E}^{\;\!\mathcolor{gray}{t}}_{\;\!\mathcolor{gray}{z}}, \bar{B}^{\;\!\mathcolor{gray}{t}}_{\;\!\mathcolor{gray}{z}}$\Footnote{$\bar{D}^{\;\!\mathcolor{gray}{t}}_{\;\!\mathcolor{gray}{z}},\bar{H}^{\;\!\mathcolor{gray}{t}}_{\;\!\mathcolor{gray}{z}}$作为总/裸场$\bar{E}^{\;\!\mathcolor{gray}{t}}_{\;\!\mathcolor{gray}{z}},\bar{B}^{\;\!\mathcolor{gray}{t}}_{\;\!\mathcolor{gray}{z}}$分别加上(通过\bref{eq:Div-D})或减去(通过\bref{eq:Curl-H})束缚源所响应产生的(极化或磁化)场后的辅助场,分别代表自由/裸源${\rho}^{\;\!\mathcolor{gray}{t}}_{\;\!\textcolor{Maroon}{\text{f}}\mathcolor{gray}{z}}, \bar{J}^{\;\!\mathcolor{gray}{t}}_{\;\!\textcolor{Maroon}{\text{f}}\mathcolor{gray}{z}}$所对应的“净留/残余场”。$\bar{E}^{\;\!\mathcolor{gray}{t}}_{\;\!\mathcolor{gray}{z}},\bar{B}^{\;\!\mathcolor{gray}{t}}_{\;\!\mathcolor{gray}{z}}$是基本场,出于下述原因:其起源是微观且明确的、可直接测量、包含了所有的束缚和自由源产生的场、洛伦兹力公式(普适至相对论情形)、\bref{eq:Curl-EK,eq:Div-Bk}的无源特性及其导出的标矢势和四维势矢量、四维二阶电磁场张量\cite{chen-zhuChenZhuxieUndergraduate_courses2024}、无矛盾地推导和适用 Post 约束\cite{lakhtakiaGenesisPostConstraint2004};同时也方便原子物理中对拉莫尔进动、史特恩—盖拉赫实验、塞曼效应的表述\cite{chen-zhuChenZhuxieUndergraduate_courses2024},以及量子电动力学中对磁光材料的拉氏量的处理\cite{nelsonLagrangianTreatmentMagnetic1994}。——但是,选择$\bar{B}^{\;\!\mathcolor{gray}{t}}_{\;\!\mathcolor{gray}{z}}$而不是$\bar{H}^{\;\!\mathcolor{gray}{t}}_{\;\!\mathcolor{gray}{z}}$将不方便(准静)磁学,如铁磁性物质的磁滞回线的表述\cite{hillionBasicFieldElectromagnetism1996}。但此时将外场写作$\bar{B}_0$而不是$\bar{H}$似乎即可解决:正如电光效应的外加准静电场$\bar{E}_0$一样。}而不是 $\bar{E}^{\;\!\mathcolor{gray}{t}}_{\;\!\mathcolor{gray}{z}}, \bar{H}^{\;\!\mathcolor{gray}{t}}_{\;\!\mathcolor{gray}{z}}$\Footnote{相对论或手性的情形下,将$\bar{E}^{\;\!\mathcolor{gray}{t}}_{\;\!\mathcolor{gray}{z}}, \bar{H}^{\;\!\mathcolor{gray}{t}}_{\;\!\mathcolor{gray}{z}}$而不是$\bar{E}^{\;\!\mathcolor{gray}{t}}_{\;\!\mathcolor{gray}{z}}, \bar{B}^{\;\!\mathcolor{gray}{t}}_{\;\!\mathcolor{gray}{z}}$作为本构关系的基本场,可能更有优势\cite{hillionBasicFieldElectromagnetism1996,lakhtakiaGenesisPostConstraint2004};此外,对于边界条件,进可采用$\bar{E}^{\;\!\mathcolor{gray}{t}}_{\;\!\mathcolor{gray}{z}},\bar{H}^{\;\!\mathcolor{gray}{t}}_{\;\!\mathcolor{gray}{z}}$切向连续边界条件\cite{mcleodVectorFourierOptics2014},退可四维时空傅立叶变换\cite{chenWavevectorspaceMethodWave1993,chenWavePropagationExciton1993,nelsonDerivingTransmissionReflection1995}。还允许不关注微观起源\cite{eimerlQuantumElectrodynamicsOptical1988,nelsonMechanismsDispersionCrystalline1989,boydNonlinearOptics2019,loudonPropagationElectromagneticEnergy1997,laxLinearNonlinearElectrodynamics1971},电场的非局域一阶波矢色散也可直接放进本构关系而无需额外处理\cite{berryOpticalSingularitiesBianisotropic2005}。但可能没法处理材料表面积累电荷(如铁电体的$\textcolor{Maroon}{+\symup{c}}$ 面)、表面电流\cite{chen-zhuChenZhuxieUndergraduate_courses2024}、表面光学活性\cite{nelsonMechanismsDispersionCrystalline1989}、许多磁致光学现象\cite{raabMultipoleTheoryElectromagnetism2004},尽管没有使用到任何散度方程/横向约束,已经很有吸引力了\cite{eimerlQuantumElectrodynamicsOptical1988,berryOpticalSingularitiesBirefringent2003,berryOpticalSingularitiesBianisotropic2005}。}视为基本场\cite{hillionBasicFieldElectromagnetism1996,lakhtakiaGenesisPostConstraint2004,nelsonDerivingTransmissionReflection1995,hehlGentleIntroductionFoundations2000}。因此,2 个旋度方程 \bref{eq:Curl-EK}、\bref{eq:Curl-H} 应等效地写做:
\begin{subequations} \label{eq:Curl-EB}
\begin{align}
	\textcolor{Maroon}{\text{法拉第电磁感应定律}}\text{:}~~~~~~~ \mathcolor{gray}{\bar{\nabla} \times} \bar{E}^{\;\!\mathcolor{gray}{t}}_{\;\!\mathcolor{gray}{z}} + \mathcolor{gray}{\nabla^t} \bar{B}^{\;\!\mathcolor{gray}{t}} &= \bar{0} \label{eq:Curl-E} \\ 
	\epsilon^{\hphantom{\symup{\iota}\hat{1}}\hat{2}}_{\symup{\iota}\mathcolor{gray}{\hat{1}}} \mathcolor{gray}{\nabla^{\hat{1}}} E^{\;\!\mathcolor{gray}{t}}_{\;\! \hat{2}\mathcolor{gray}{z}} + \mathcolor{gray}{\nabla^t} B^{\;\!\mathcolor{gray}{t}}_{\;\! \symup{\iota}\mathcolor{gray}{z}} &= 0~, \label{eq:curl-E} \\
	\textcolor{Maroon}{\text{安倍环路定律}}\text{:}~~~~~~~~~ {\symup{\varepsilon}}_0^{-1} \mathcolor{gray}{\bar{\nabla} \times} \bar{B}^{\;\!\mathcolor{gray}{t}}_{\;\!\mathcolor{gray}{z}} - {\symup{\varepsilon}}_0 \mathcolor{gray}{\nabla^t} \bar{E}^{\;\!\mathcolor{gray}{t}}_{\;\!\mathcolor{gray}{z}} &= \bar{J}^{\;\!\mathcolor{gray}{t}}_{\;\!\mathcolor{gray}{z}} \label{eq:Curl-B} \\ 
	{\symup{\varepsilon}}_0^{-1} \epsilon^{\hphantom{\symup{\iota}\hat{1}}\hat{2}}_{\symup{\iota}\mathcolor{gray}{\hat{1}}} \mathcolor{gray}{\nabla^{\hat{1}}} B^{\;\!\mathcolor{gray}{t}}_{\;\! \hat{2}\mathcolor{gray}{z}} - {\symup{\varepsilon}}_0 \mathcolor{gray}{\nabla^t} E^{\;\!\mathcolor{gray}{t}}_{\;\! \symup{\iota}\mathcolor{gray}{z}} &= J^{\;\!\mathcolor{gray}{t}}_{\;\! \symup{\iota}\mathcolor{gray}{z}}~. \label{eq:curl-B}
\end{align}
\end{subequations}
相应地,2 个散度方程 \bref{eq:Div-D}、\bref{eq:Curl-B} 返璞归真为:
\begin{subequations} \label{eq:Div-EB}
\begin{align}
	\textcolor{Maroon}{\text{高斯定律(电)}}\text{:}~~~~~~~ {\symup{\varepsilon}}_0 \mathcolor{gray}{\bar{\nabla} \cdot} \bar{E}^{\;\!\mathcolor{gray}{t}}_{\;\!\mathcolor{gray}{z}} &= {\rho}^{\;\!\mathcolor{gray}{t}}_{\;\!\mathcolor{gray}{z}} \label{eq:Div-E} \\ 
	{\symup{\varepsilon}}_0 \mathcolor{gray}{\nabla^\iota} E^{\;\!\mathcolor{gray}{t}}_{\;\! \symup{\iota}\mathcolor{gray}{z}} &= {\rho}^{\;\!\mathcolor{gray}{t}}_{\;\!\mathcolor{gray}{z}}~, \label{eq:div-E} \\ 
	\textcolor{Maroon}{\text{高斯定律(磁)}}\text{:}~~~~~~~~~~ \mathcolor{gray}{\bar{\nabla} \cdot} \bar{B}^{\;\!\mathcolor{gray}{t}}_{\;\!\mathcolor{gray}{z}} &= 0 \label{eq:Div-B} \\ 
	\mathcolor{gray}{\nabla^\iota} B^{\;\!\mathcolor{gray}{t}}_{\;\! \symup{\iota}\mathcolor{gray}{z}} &= 0~. \label{eq:div-B} 
\end{align}
\end{subequations}
同理,总电荷 ${\rho}^{\;\!\mathcolor{gray}{t}}_{\;\!\mathcolor{gray}{z}}$ 守恒方程从自由电荷 ${\rho}^{\;\!\mathcolor{gray}{t}}_{\;\!\textcolor{Maroon}{\text{f}}\mathcolor{gray}{z}}$ 所满足的连续性 \bref{eq:Div-e-f} 升级为:
\begin{align}
	\textcolor{Maroon}{\text{电荷守恒定律}}\text{:}~~~~~~~ \mathcolor{gray}{\bar{\nabla} \cdot} \bar{J}^{\;\!\mathcolor{gray}{t}}_{\;\!\mathcolor{gray}{z}} + \mathcolor{gray}{\nabla^t} {\rho}^{\;\!\mathcolor{gray}{t}}_{\;\!\mathcolor{gray}{z}} &= 0 \label{eq:Div-e} \\ 
	\mathcolor{gray}{\nabla^\iota} J^{\;\!\mathcolor{gray}{t}}_{\;\! \symup{\iota}\mathcolor{gray}{z}} + \mathcolor{gray}{\nabla^t} {\rho}^{\;\!\mathcolor{gray}{t}}_{\;\!\mathcolor{gray}{z}} &= 0~. \label{eq:div-e} 
\end{align}
该 \bref{eq:Div-e} 即电荷 $Q$ 守恒,与磁流 $\varPhi$ 守恒(即法拉第电磁感应定律 \bref{eq:Curl-E}、安倍-麦克斯韦环路定律 \bref{eq:Curl-B})一起\cite{hehlSpacetimeMetricLocal2006},经受住了大量的实验考验\cite{hehlGentleIntroductionFoundations2000},因此二者都是基本的物理定律\cite{hehlRecentDevelopmentsPremetric2006,hehlFOUNDATIONSCLASSICALELECTRODYNAMICS}。

下 \bref{fig:EHDB} 中,2 条红边(连接了顶点 $D,H$、$E,B$)、2 条蓝边(连接了顶点 $E,H$、$D,B$)上,对应颜色(红/橘色、蓝色)的文字,分别给出了选择 4 种电磁场组合对 $\{ E,H$、$E,B$、$D,B$、$D,H \}$ 中的每一对作为基本场的理由。

\begin{figure}[htbp!]
	\centering
	\includegraphics[width=0.9\textwidth]{D:/C2D/Desktop/article_fig/phd_thesis_fig/chapter-02/本构关系与边界条件-single-page.pdf}
	\backcaption{-0.7em}{$E,B$ 作为基本场的理由(大量的红/橘色文字理由所在的边),远比其他 3 种组合(2 条蓝色文字边 $E,H$ 和 $D,B$,以及 1 条少量红/橘色文字对边 $D,H$)更丰富。}{fig:EHDB}
\end{figure}

\vspace*{-5.5em}

\marginLeft[-1.5em]{ssec:PMQN}\subsection{$\bar{P},\bar{M}$ 极磁化、$\bar{\bar{Q}},\bar{\bar{N}}$ 多极理论}\label{ssec:PMQN}

在高斯定律(电) \bref{eq:Div-E} 和电荷守恒定律 \bref{eq:Div-e} 中,总电荷源 ${\rho}^{\;\!\mathcolor{gray}{t}}_{\;\!\mathcolor{gray}{z}}$ 分为自由电荷源 ${\rho}^{\;\!\mathcolor{gray}{t}}_{\;\!\textcolor{Maroon}{\text{f}}\mathcolor{gray}{z}}$ 和束缚电荷源 ${\rho}^{\;\!\mathcolor{gray}{t}}_{\;\!\textcolor{Maroon}{\text{b}}\mathcolor{gray}{z}}$\cite{langeMultipoleTheoryHehl2015,raabMultipoleTheoryElectromagnetism2004},即:
\abovedisplayskip=10pt
\belowdisplayskip=10pt
\begin{align} \label{eq:p=f+b}
	{\rho}^{\;\!\mathcolor{gray}{t}}_{\;\!\mathcolor{gray}{z}} = {\rho}^{\;\!\mathcolor{gray}{t}}_{\;\!\textcolor{Maroon}{\text{f}}\mathcolor{gray}{z}} + {\rho}^{\;\!\mathcolor{gray}{t}}_{\;\!\textcolor{Maroon}{\text{b}}\mathcolor{gray}{z}}~,
\end{align}
其中,对动态电荷源分布所产生的标势场 $\phi^{\;\!\mathcolor{gray}{t}}_{\;\!\mathcolor{gray}{z}}$,在场点 $\mathcolor{gray}{\bar{r}}$ 邻域(无须在源分布体系的平衡态附近),进行三元泰勒展开\Footnote{注意:本文将多极、非局域、非线性展开式的所有系数,均纳入物理量本身中,以降低心智负担。},得束缚电荷体密度 ${\rho}^{\;\!\mathcolor{gray}{t}}_{\;\!\textcolor{Maroon}{\text{b}}\mathcolor{gray}{z}}$ 的表达式\cite{raabMultipoleTheoryElectromagnetism2004,delangeTranslationalInvariancePost2012,chen-zhuChenZhuxieUndergraduate_courses2024}:
\begin{subequations}
\begin{align}
	{\rho}^{\;\!\mathcolor{gray}{t}}_{\;\!\textcolor{Maroon}{\text{b}}\mathcolor{gray}{z}} &= -\hspace{0.2em} \mathcolor{gray}{\bar{\nabla} \cdot} \left( \bar{P}^{\;\!\mathcolor{gray}{t}}_{\;\!\mathcolor{gray}{z}} - \mathcolor{gray}{\bar{\nabla} \cdot} \bar{\bar{Q}}^{\;\!\mathcolor{gray}{t}}_{\;\!\mathcolor{gray}{z}} + \mathcolor{gray}{\bar{\nabla}} \mathcolor{gray}{\bar{\nabla} \colon}\! \bar{\bar{\bar{O}}}^{\;\!\mathcolor{gray}{t}}_{\;\!\mathcolor{gray}{z}} - \cdots \right) \label{eq:P-b} \\
	&= -\hspace{0.2em} \mathcolor{gray}{\nabla^\iota} \left( P^{\;\!\mathcolor{gray}{t}}_{\;\! \symup{\iota}\mathcolor{gray}{z}} - \mathcolor{gray}{\nabla^{\hat{1}}} Q^{\;\!\mathcolor{gray}{t}}_{\;\! \symup{\iota}\hat{1}\mathcolor{gray}{z}} + \mathcolor{gray}{\nabla^{\hat{1}}} \mathcolor{gray}{\nabla^{\hat{2}}} O^{\;\!\mathcolor{gray}{t}}_{\;\! \symup{\iota}\hat{1}\hat{2}\mathcolor{gray}{z}} - \cdots \right)~. \label{eq:p-b}
\end{align}
\end{subequations}

同样,在安倍环路定律 \bref{eq:Curl-B} 和 电荷守恒定律 \bref{eq:Div-e} 中,总电流源 $\bar{J}^{\;\!\mathcolor{gray}{t}}_{\;\!\mathcolor{gray}{z}}$ 内包含了自由项 $\bar{J}^{\;\!\mathcolor{gray}{t}}_{\;\!\textcolor{Maroon}{\text{f}}\mathcolor{gray}{z}}$ 和束缚项 $\bar{J}^{\;\!\mathcolor{gray}{t}}_{\;\!\textcolor{Maroon}{\text{b}}\mathcolor{gray}{z}}$\cite{langeMultipoleTheoryHehl2015,raabMultipoleTheoryElectromagnetism2004},而束缚源 $\bar{J}^{\;\!\mathcolor{gray}{t}}_{\;\!\textcolor{Maroon}{\text{b}}\mathcolor{gray}{z}}$ 又由电源 $\bar{J}^{\;\!\mathcolor{gray}{t}}_{\;\!\textcolor{Maroon}{\text{e}}\mathcolor{gray}{z}}$ 和磁源 $\bar{J}^{\;\!\mathcolor{gray}{t}}_{\;\!\textcolor{Maroon}{\text{m}}\mathcolor{gray}{z}}$ 构成,即:
\begin{subequations} \label{eq:j-fbem}
	\abovedisplayskip=-2pt
%	\belowdisplayskip=10pt
\begin{align}
	\bar{J}^{\;\!\mathcolor{gray}{t}}_{\;\!\mathcolor{gray}{z}} &= \bar{J}^{\;\!\mathcolor{gray}{t}}_{\;\!\textcolor{Maroon}{\text{f}}\mathcolor{gray}{z}} + \bar{J}^{\;\!\mathcolor{gray}{t}}_{\;\!\textcolor{Maroon}{\text{b}}\mathcolor{gray}{z}}~, \label{eq:j=f+b} \\ \bar{J}^{\;\!\mathcolor{gray}{t}}_{\;\!\textcolor{Maroon}{\text{b}}\mathcolor{gray}{z}} &= \bar{J}^{\;\!\mathcolor{gray}{t}}_{\;\!\textcolor{Maroon}{\text{e}}\mathcolor{gray}{z}} + \bar{J}^{\;\!\mathcolor{gray}{t}}_{\;\!\textcolor{Maroon}{\text{m}}\mathcolor{gray}{z}}~. \label{eq:b=e+m}
\end{align}
\end{subequations}
其中,考虑有限区域内连续分布的时变电流源所产生的矢势场 $\bar{A}^{\;\!\mathcolor{gray}{t}}_{\;\!\mathcolor{gray}{z}}$,等价于源全集中于其荷心时,在原点处的永久/自发多极矩和(受到外场如 $\bar{E}^{\;\!\mathcolor{gray}{t}}_{\;\!\mathcolor{gray}{z}}$ 和/或 $\bar{B}^{\;\!\mathcolor{gray}{t}}_{\;\!\mathcolor{gray}{z}}$、应力场 $\bar{T}^{\;\!\mathcolor{gray}{t}}_{\;\!\mathcolor{gray}{z}}$ 等后反作用出的)感应多极矩(的各项多极展开之和)之和,朝心外某一场点激发的矢势场\cite{raabMultipoleTheoryElectromagnetism2004,delangeTranslationalInvariancePost2012,chen-zhuChenZhuxieUndergraduate_courses2024},可分别得到束缚电源 $\bar{J}^{\;\!\mathcolor{gray}{t}}_{\;\!\textcolor{Maroon}{\text{e}}\mathcolor{gray}{z}}$ 和磁源 $\bar{J}^{\;\!\mathcolor{gray}{t}}_{\;\!\textcolor{Maroon}{\text{m}}\mathcolor{gray}{z}}$ 体密度:
\begin{subequations} \label{eq:J-em}
\begin{align}
	\bar{J}^{\;\!\mathcolor{gray}{t}}_{\;\!\textcolor{Maroon}{\text{e}}\mathcolor{gray}{z}} &= \mathcolor{gray}{\nabla^t} \left( \bar{P}^{\;\!\mathcolor{gray}{t}}_{\;\!\mathcolor{gray}{z}} - \mathcolor{gray}{\bar{\nabla} \cdot} \bar{\bar{Q}}^{\;\!\mathcolor{gray}{t}}_{\;\!\mathcolor{gray}{z}} + \mathcolor{gray}{\bar{\nabla}} \mathcolor{gray}{\bar{\nabla} \colon}\! \bar{\bar{\bar{O}}}^{\;\!\mathcolor{gray}{t}}_{\;\!\mathcolor{gray}{z}} - \cdots \right) \label{eq:J-e} \\ 
	&= \mathcolor{gray}{\nabla^t} \left( P^{\;\!\mathcolor{gray}{t}}_{\;\! \symup{\iota}\mathcolor{gray}{z}} - \mathcolor{gray}{\nabla^{\hat{1}}} Q^{\;\!\mathcolor{gray}{t}}_{\;\! \symup{\iota}\hat{1}\mathcolor{gray}{z}} + \mathcolor{gray}{\nabla^{\hat{1}}} \mathcolor{gray}{\nabla^{\hat{2}}} O^{\;\!\mathcolor{gray}{t}}_{\;\! \symup{\iota}\hat{1}\hat{2}\mathcolor{gray}{z}} - \cdots \right) \hat{\symup{e}}^i~, \label{eq:j-e} \\ 
	\bar{J}^{\;\!\mathcolor{gray}{t}}_{\;\!\textcolor{Maroon}{\text{m}}\mathcolor{gray}{z}} &= \mathcolor{gray}{\bar{\nabla} \times} \left( \bar{M}^{\;\!\mathcolor{gray}{t}}_{\;\!\mathcolor{gray}{z}} - \mathcolor{gray}{\bar{\nabla} \cdot} \bar{\bar{N}}^{\;\!\mathcolor{gray}{t}}_{\;\!\mathcolor{gray}{z}} + \cdots \right) \label{eq:J-m} \\ 
	&= \epsilon^{\hphantom{\symup{\iota}\hat{1}}\hat{2}}_{\symup{\iota}\mathcolor{gray}{\hat{1}}} \mathcolor{gray}{\nabla^{\hat{1}}} \left( M^{\;\!\mathcolor{gray}{t}}_{\;\! \hat{2}\mathcolor{gray}{z}} - \mathcolor{gray}{\nabla^{\hat{3}}} N^{\;\!\mathcolor{gray}{t}}_{\;\! \hat{2}\hat{3} \mathcolor{gray}{z}} + \cdots \right) \hat{\symup{e}}^i~. \label{eq:j-m}
\end{align}
\end{subequations}
其中,$\bar{P}^{\;\!\mathcolor{gray}{t}}_{\;\!\mathcolor{gray}{z}}, \bar{M}^{\;\!\mathcolor{gray}{t}}_{\;\!\mathcolor{gray}{z}}$ 为电/磁偶极矩,$\bar{\bar{Q}}^{\;\!\mathcolor{gray}{t}}_{\;\!\mathcolor{gray}{z}}, \bar{\bar{N}}^{\;\!\mathcolor{gray}{t}}_{\;\!\mathcolor{gray}{z}}$ 为电/磁四极矩、$\bar{\bar{\bar{O}}}^{\;\!\mathcolor{gray}{t}}_{\;\!\mathcolor{gray}{z}}$ 为电八极矩。

然而,在强度\Footnote{相邻阶多极矩的强度比大约为 $10^{-6} = \left( 2\pi \big/ 1\symup{\mu}\text{m} \cdot 1\text{\r{A}} \right)^2$。}上,多极矩的正确分组应是\cite{grahamMultipoleSolutionMacroscopic2000}\Footnote{从同阶磁矩弱于电矩、电子受到的电场力是洛伦兹力的c$\big/v$倍的角度,物质 和 CCD(电荷耦合器件)对电场的响应比对同等量级的磁场的响应大(至少对于高频电磁场=光的相互作用)。然而,电四/磁偶极矩的强度和其造成的影响,不一定比电偶极矩的小\cite{raabMultipoleTheoryElectromagnetism2004,OriginDependenceMaterial},包括其表面效应、响应的线性和非线性行为\cite{bethuneOpticalQuadrupoleSumfrequency1976}。}:
\begin{subequations}
%	\abovedisplayskip=8pt
%	\belowdisplayskip=8pt
\begin{align}
	\bar{P}^{\;\!\mathcolor{gray}{t}}_{\;\!\mathcolor{gray}{z}} \hspace{1em} &\ll \hspace{1em} \bar{\bar{Q}}^{\;\!\mathcolor{gray}{t}}_{\;\!\mathcolor{gray}{z}}, \bar{M}^{\;\!\mathcolor{gray}{t}}_{\;\!\mathcolor{gray}{z}} \hspace{-2.5em} &&\ll \hspace{1em} \bar{\bar{\bar{O}}}^{\;\!\mathcolor{gray}{t}}_{\;\!\mathcolor{gray}{z}}, \bar{\bar{N}}^{\;\!\mathcolor{gray}{t}}_{\;\!\mathcolor{gray}{z}} \label{eq:PQMON} \\ 
	\textcolor{Maroon}{\text{电偶极矩}} \hspace{1em} &\ll \hspace{1em} \textcolor{Maroon}{\text{电偶、磁四极矩}} \hspace{-2.5em} &&\ll \hspace{1em} \textcolor{Maroon}{\text{电八、磁四极矩}} ~. 
\end{align}
\end{subequations}
相应地将束缚电流源 \bref{eq:b=e+m} 展开并整理为:
\begin{subequations}
%	\abovedisplayskip=6pt
\begin{align}
\bar{J}^{\;\!\mathcolor{gray}{t}}_{\;\!\textcolor{Maroon}{\text{b}}\mathcolor{gray}{z}} = \mathcolor{gray}{\nabla^t} \bar{P}^{\;\!\mathcolor{gray}{t}}_{\;\!\mathcolor{gray}{z}} &- \left( \mathcolor{gray}{\bar{\nabla} \cdot} \mathcolor{gray}{\nabla^t} \bar{\bar{Q}}^{\;\!\mathcolor{gray}{t}}_{\;\!\mathcolor{gray}{z}} - \mathcolor{gray}{\bar{\nabla} \times} \bar{M}^{\;\!\mathcolor{gray}{t}}_{\;\!\mathcolor{gray}{z}} \right) \label{eq:J-b} \\ &+ \left[ \mathcolor{gray}{\bar{\nabla}} \mathcolor{gray}{\bar{\nabla} \colon}\! \mathcolor{gray}{\nabla^t} \bar{\bar{\bar{O}}}^{\;\!\mathcolor{gray}{t}}_{\;\!\mathcolor{gray}{z}} - \mathcolor{gray}{\bar{\nabla} \times} \left( \mathcolor{gray}{\bar{\nabla} \cdot}  \bar{\bar{N}}^{\;\!\mathcolor{gray}{t}}_{\;\!\mathcolor{gray}{z}} \right) \right] - \cdots~, \\
J^{\;\!\mathcolor{gray}{t}}_{\;\!\textcolor{Maroon}{\text{b}} \symup{\iota} \mathcolor{gray}{z}} = \mathcolor{gray}{\nabla^t} P^{\;\!\mathcolor{gray}{t}}_{\;\! \symup{\iota}\mathcolor{gray}{z}} &- \mathcolor{gray}{\nabla^{\hat{1}}} \left( \mathcolor{gray}{\nabla^t} Q^{\;\!\mathcolor{gray}{t}}_{\;\! \symup{\iota}\hat{1}\mathcolor{gray}{z}} - \epsilon^{\hphantom{\symup{\iota}\hat{1}}\hat{2}}_{\symup{\iota}\mathcolor{gray}{\hat{1}}} M^{\;\!\mathcolor{gray}{t}}_{\;\! \hat{2}\mathcolor{gray}{z}} \right) \label{eq:j-b} \\ &+ \mathcolor{gray}{\nabla^{\hat{1}}} \mathcolor{gray}{\nabla^{\hat{3}}} \left[  \mathcolor{gray}{\nabla^t} O^{\;\!\mathcolor{gray}{t}}_{\;\! \symup{\iota}\hat{1}\hat{3} \mathcolor{gray}{z}} - \epsilon^{\hphantom{\symup{\iota}\hat{1}}\hat{2}}_{\symup{\iota}\mathcolor{gray}{\hat{1}}} N^{\;\!\mathcolor{gray}{t}}_{\;\! \hat{2}\hat{3} \mathcolor{gray}{z}} \right] - \cdots~.
\end{align}
\end{subequations}

\clearpage
%\XGap{-10em}
\vspace*{-6.5em}

\marginLeft[-1.3em]{ssec:step-delta}\subsection{${\rho}, \bar{J}$ 表面源、${\mathbb{1}},\delta$ 场源展开}\label{ssec:step-delta}

考虑在 $\mathcolor{gray}{z} = \mathcolor{gray}{0}$ 处面接触的两个半无限介质 \textcolor{Maroon}{0} 和介质 \textcolor{Maroon}{1},输入场$=$泵浦的能流方向从介质 \textcolor{Maroon}{0} $\to$ 介质 \textcolor{Maroon}{1},且与接触面内法向夹成锐角。在上述初始条件下,电荷/流源 ${\rho}^{\;\!\mathcolor{gray}{t}}_{\;\!\mathcolor{gray}{z}}, \bar{J}^{\;\!\mathcolor{gray}{t}}_{\;\!\mathcolor{gray}{z}}$ 中的每一个\Footnote{是将源 ${\rho}^{\;\!\mathcolor{gray}{t}}_{\;\!\mathcolor{gray}{z}}, \bar{J}^{\;\!\mathcolor{gray}{t}}_{\;\!\mathcolor{gray}{z}}$ 本身,而不是将构成源的底层元素们——束缚电/磁多极矩强度 $\bar{P}^{\;\!\mathcolor{gray}{t}}_{\;\!\mathcolor{gray}{z}},\bar{\bar{Q}}^{\;\!\mathcolor{gray}{t}}_{\;\!\mathcolor{gray}{z}},\bar{\bar{\bar{O}}}^{\;\!\mathcolor{gray}{t}}_{\;\!\mathcolor{gray}{z}} ; \bar{M}^{\;\!\mathcolor{gray}{t}}_{\;\!\mathcolor{gray}{z}}, \bar{\bar{N}}^{\;\!\mathcolor{gray}{t}}_{\;\!\mathcolor{gray}{z}}$,展开成 ${\mathbb{1}},\delta$ 的函数。—— 若将后者展开,则 ${\rho}^{\;\!\mathcolor{gray}{t}}_{\;\!\mathcolor{gray}{z}}$ 中也将产生 $\delta$ 项,并与 $\bar{J}^{\;\!\mathcolor{gray}{t}}_{\;\!\mathcolor{gray}{z}}$ 中的同层次 $\delta$ 项抵消,无法提供信息。} 都可以写成 2 个阶跃函数 / \textcolor{Maroon}{Heaviside} 单位函数
\begin{align} \label{eq:u}
	\mathbb{1}_{\mathcolor{gray}{z}} := \mathbb{1} \left( \mathcolor{gray}{z} \right) = ~\left\{~ \begin{aligned} 
		&1 &&, \mathcolor{gray}{z \mathcolor{black}{>} 0} \\ 
		&\textcolor{Maroon}{\text{undefined}} &&, \mathcolor{gray}{z \mathcolor{black}{=} 0} \\
		&0 &&, \mathcolor{gray}{z \mathcolor{black}{<} 0} \end{aligned}\right. ~,
\end{align}
之和:即使用
\begin{subequations} \label{eq:u-01}
	\abovedisplayskip=0pt
%	\belowdisplayskip=10pt
\begin{align}
	\leftindex_{\textcolor{Maroon}{1}} {\mathbb{1}}_{\mathcolor{gray}{z}} &= {\mathbb{1}}_{\mathcolor{gray}{z}} ~, \label{eq:u-1} \\ 
	\leftindex_{\textcolor{Maroon}{0}} {\mathbb{1}}_{\mathcolor{gray}{z}} &= {\mathbb{1}}_{ - \mathcolor{gray}{z} } ~, \label{eq:u-0}
\end{align}
\end{subequations}
将电荷/流源 $\rho_{\;\!\textcolor{Maroon}{\text{b}}}, J_{\;\!\textcolor{Maroon}{\text{b}}}$ 中的每一个(均暂记为 $X$)暂表达为:
%\abovedisplayskip=8pt
%\belowdisplayskip=10pt
\begin{align} \label{eq:X-01}
	X_{\mathcolor{gray}{z}} = \leftindex_{\textcolor{Maroon}{\mathsfit{z}}} {\mathbb{1}}_{\mathcolor{gray}{z}} \leftindex^{\textcolor{Maroon}{\mathsfit{z}}} X_{\mathcolor{gray}{z}} ~~~~, \text{其中} ~~~ \textcolor{Maroon}{\mathsfit{z}} = \textcolor{Maroon}{0,1} ~,
\end{align}
然后将二者代入由 \bref{eq:div-e,eq:div-e-f} 构建的束缚电荷 ${Q}^{\;\!\mathcolor{gray}{t}}_{\;\!\textcolor{Maroon}{\text{b}}\mathcolor{gray}{z}}$ 守恒\Footnote{束缚电荷 ${Q}^{\;\!\mathcolor{gray}{t}}_{\;\!\textcolor{Maroon}{\text{b}}\mathcolor{gray}{z}}$ 与 自由电荷 ${Q}^{\;\!\mathcolor{gray}{t}}_{\;\!\textcolor{Maroon}{\text{f}}\mathcolor{gray}{z}}$ 可以相互转换,因此各自单独而言在最广义上不(一定)守恒:通过价带/杂质能级到导带的跃迁,比如物质对光子的线性、非线性吸收并产生光电压/流所对应的内/外光伏/电效应\cite{boydNonlinearOptics2019};因此 \bref{eq:div-e-b} 和 \bref{eq:div-e-f} 均不是定律,相比总电荷 ${Q}^{\;\!\mathcolor{gray}{t}}_{\;\!\mathcolor{gray}{z}}$ 守恒 \bref{eq:div-e} 而言\cite{markelExternalInducedFree2018}。此外,外加/固有、感应/永久电荷,均可处于束缚/自由态,3 组概念互相独立\cite{markelExternalInducedFree2018}。这里暂时不考虑摩擦起电、外光电效应、激光加工之强场激发等离子体\cite{boydNonlinearOptics2019,dengTheoryElectrodynamicResponse2020}等非平衡/含驰豫现象,因此暂认为束缚、自由电荷分别独立守恒。}中\cite{grahamMultipoleSolutionMacroscopic2000,delangeElectromagneticBoundaryConditions2013}:
\begin{subequations}
%	\abovedisplayskip=8pt
%	\belowdisplayskip=12pt
\begin{align}
	\mathcolor{gray}{\nabla^\iota} J^{\;\!\mathcolor{gray}{t}}_{\;\!\textcolor{Maroon}{\text{b}} \symup{\iota}\mathcolor{gray}{z}} &+ \mathcolor{gray}{\nabla^t} {\rho}^{\;\!\mathcolor{gray}{t}}_{\;\!\textcolor{Maroon}{\text{b}}\mathcolor{gray}{z}} &&\hspace{-2em}= 0 \label{eq:div-e-b} \\ 
	\mathcolor{gray}{\nabla^\iota} \left( \leftindex_{\textcolor{Maroon}{\mathsfit{z}}} {\mathbb{1}}_{\mathcolor{gray}{z}} \leftindex^{\textcolor{Maroon}{\mathsfit{z}}} \;\! J^{\;\!\mathcolor{gray}{t}}_{\;\!\textcolor{Maroon}{\text{b}} \symup{\iota}\mathcolor{gray}{z}} \right) &+ \mathcolor{gray}{\nabla^t} \left( \leftindex_{\textcolor{Maroon}{\mathsfit{z}}} {\mathbb{1}}_{\mathcolor{gray}{z}} \leftindex^{\textcolor{Maroon}{\mathsfit{z}}} {\rho}^{\;\!\mathcolor{gray}{t}}_{\;\!\textcolor{Maroon}{\text{b}}\mathcolor{gray}{z}} \right) &&\hspace{-2em}= 0 \label{eq:div-e-b-01} \\ 
	\leftindex_{\textcolor{Maroon}{\mathsfit{z}}} {\mathbb{1}}_{\mathcolor{gray}{z}} \left( \mathcolor{gray}{\nabla_x} \leftindex^{\textcolor{Maroon}{\mathsfit{z}}} \;\! J^{\;\!\mathcolor{gray}{t}}_{\;\!\textcolor{Maroon}{\text{b}} \symup{x} \mathcolor{gray}{z}} + \mathcolor{gray}{\nabla_y} \leftindex^{\textcolor{Maroon}{\mathsfit{z}}} \;\! J^{\;\!\mathcolor{gray}{t}}_{\;\!\textcolor{Maroon}{\text{b}} \symup{y} \mathcolor{gray}{z}} \right) + \mathcolor{gray}{\nabla_z} \left( \leftindex_{\textcolor{Maroon}{\mathsfit{z}}} {\mathbb{1}}_{\mathcolor{gray}{z}} \leftindex^{\textcolor{Maroon}{\mathsfit{z}}} \;\! J^{\;\!\mathcolor{gray}{t}}_{\;\!\textcolor{Maroon}{\text{b}} \symup{z} \mathcolor{gray}{z}} \right) &+ \leftindex_{\textcolor{Maroon}{\mathsfit{z}}} {\mathbb{1}}_{\mathcolor{gray}{z}} \mathcolor{gray}{\nabla^t} \leftindex^{\textcolor{Maroon}{\mathsfit{z}}} {\rho}^{\;\!\mathcolor{gray}{t}}_{\;\!\textcolor{Maroon}{\text{b}}\mathcolor{gray}{z}} &&\hspace{-2em}= 0~, \label{eq:div-e-b-01'}
\end{align}
\end{subequations}
注意到,在空域上对 $\leftindex_{\textcolor{Maroon}{\mathsfit{z}}} {\mathbb{1}}_{\mathcolor{gray}{z}} \leftindex^{\textcolor{Maroon}{\mathsfit{z}}} X_{\mathcolor{gray}{z}}$ 求 $\mathcolor{gray}{z}$ 向一阶或高阶偏导 $\mathcolor{gray}{\nabla_z}$,即
\begin{subequations}
%	\abovedisplayskip=12pt
%	\belowdisplayskip=10pt
\begin{align}
	\mathcolor{gray}{\nabla_z} \left( \leftindex_{\textcolor{Maroon}{\mathsfit{z}}} {\mathbb{1}}_{\mathcolor{gray}{z}} \leftindex^{\textcolor{Maroon}{\mathsfit{z}}} X_{\mathcolor{gray}{z}} \right) &= \leftindex_{\textcolor{Maroon}{\mathsfit{z}}} \;\! \delta_{\mathcolor{gray}{z}} \leftindex^{\textcolor{Maroon}{\mathsfit{z}}} X_{\mathcolor{gray}{z}} + \leftindex_{\textcolor{Maroon}{\mathsfit{z}}} {\mathbb{1}}_{\mathcolor{gray}{z}} \left( \mathcolor{gray}{\nabla_z} \leftindex^{\textcolor{Maroon}{\mathsfit{z}}} X_{\mathcolor{gray}{z}} \right) ~, \label{eq:partial_z-step} \\ 
	\mathcolor{gray}{\nabla_z^2} \left( \leftindex_{\textcolor{Maroon}{\mathsfit{z}}} {\mathbb{1}}_{\mathcolor{gray}{z}} \leftindex^{\textcolor{Maroon}{\mathsfit{z}}} X_{\mathcolor{gray}{z}} \right) &= \leftindex_{\textcolor{Maroon}{\mathsfit{z}}} \;\! \delta'_{\mathcolor{gray}{z}} \leftindex^{\textcolor{Maroon}{\mathsfit{z}}} X_{\mathcolor{gray}{z}} + 2 \leftindex_{\textcolor{Maroon}{\mathsfit{z}}} \;\! \delta_{\mathcolor{gray}{z}} \left( \mathcolor{gray}{\nabla_z} \leftindex^{\textcolor{Maroon}{\mathsfit{z}}} X_{\mathcolor{gray}{z}} \right) + \leftindex_{\textcolor{Maroon}{\mathsfit{z}}} {\mathbb{1}}_{\mathcolor{gray}{z}} \left( \mathcolor{gray}{\nabla_z^2} \leftindex^{\textcolor{Maroon}{\mathsfit{z}}} X_{\mathcolor{gray}{z}} \right) ~, \label{eq:partial_z^2-step}
\end{align}
\end{subequations}
会产生(带有单位 $\mathcolor{gray}{\symup{m}^{-1}}$ 的)$\delta_{\mathcolor{gray}{z}}$ 函数
\begin{subequations}
%	\abovedisplayskip=7pt
	%\belowdisplayskip=0pt
\begin{align}
	\leftindex_{\textcolor{Maroon}{1}} \;\! \delta_{\mathcolor{gray}{z}} &= \mathcolor{gray}{\nabla_z} \leftindex_{\textcolor{Maroon}{1}} {\mathbb{1}}_{\mathcolor{gray}{z}} = {\mathbb{1}'}_{\mathcolor{gray}{z}} = \delta_{\mathcolor{gray}{z}} ~, \label{eq:delta-1} \\ 
	\leftindex_{\textcolor{Maroon}{0}} \;\! \delta_{\mathcolor{gray}{z}} &= \leftindex_{\textcolor{Maroon}{0}} {\mathbb{1}'}_{\mathcolor{gray}{z}} = - \leftindex_{\textcolor{Maroon}{1}} {\mathbb{1}'}_{\mathcolor{gray}{z}} = -\hspace{0.2em} \delta_{\mathcolor{gray}{z}} ~, \label{eq:delta-0}
\end{align}
\end{subequations}
及其导数 $\delta'_{\mathcolor{gray}{z}}$(含单位 $\mathcolor{gray}{\symup{m}^{-2}}$)等。—— 具体到 \bref{eq:div-e-b-01} 的情况,即对 $\bar{J}^{\;\!\mathcolor{gray}{t}}_{\;\!\textcolor{Maroon}{\text{b}}\mathcolor{gray}{z}}$ 的 $\mathcolor{gray}{z}$ 向散度 $\mathcolor{gray}{\nabla_z} \left( \leftindex_{\textcolor{Maroon}{\mathsfit{z}}} {\mathbb{1}}_{\mathcolor{gray}{z}} \leftindex^{\textcolor{Maroon}{\mathsfit{z}}} \;\! J^{\;\!\mathcolor{gray}{t}}_{\;\!\textcolor{Maroon}{\text{b}} \symup{z} \mathcolor{gray}{z}} \right)$,将产生 \bref{eq:partial_z-step} 中的 $\leftindex_{\textcolor{Maroon}{\mathsfit{z}}} \;\! \delta_{\mathcolor{gray}{z}} \leftindex^{\textcolor{Maroon}{\mathsfit{z}}} \;\! J^{\;\!\mathcolor{gray}{t}}_{\;\!\textcolor{Maroon}{\text{b}} \symup{z} \mathcolor{gray}{z}}$ 项,该项只能通过 ${\rho}^{\;\!\mathcolor{gray}{t}}_{\;\!\textcolor{Maroon}{\text{b}}\mathcolor{gray}{z}}$ 中也原始地就含有对应的表面 $\delta_{\mathcolor{gray}{z}}$ 项(如铁电体 $\textcolor{Maroon}{-\symup{c}}$ 面薄层内的束缚电子 ${\sigma}^{\;\!\mathcolor{gray}{t}}_{\;\!\textcolor{Maroon}{\text{b}}\mathcolor{gray}{z}} = {\mathcal{P}}^{\;\!\mathcolor{gray}{t}}_{\;\!\textcolor{Maroon}{\text{b}} \symup{z} \mathcolor{gray}{z}}$;它在单位上比 ${\rho}^{\;\!\mathcolor{gray}{t}}_{\;\!\textcolor{Maroon}{\text{b}}\mathcolor{gray}{z}}$ 多个 $\mathcolor{gray}{\symup{m}}$)以吸收/消除掉 \cite{grahamMultipoleSolutionMacroscopic2000},否则将违反 \bref{eq:div-e-b}\Footnote{这归根结底是因为 —— 在 \bref{eq:div-e-b} 中,对 ${\rho}^{\;\!\mathcolor{gray}{t}}_{\;\!\textcolor{Maroon}{\text{b}}\mathcolor{gray}{z}}$ 只有时间偏导 $\mathcolor{gray}{\nabla^t}$,然而对 $\bar{J}^{\;\!\mathcolor{gray}{t}}_{\;\!\textcolor{Maroon}{\text{b}}\mathcolor{gray}{z}}$ 却有空域偏导 $\mathcolor{gray}{\bar{\nabla}}$,特别是对 $\mathcolor{gray}{z}$ 的偏导 $\mathcolor{gray}{\nabla_z}$:这种不对称性最终会强制拓展 ${\rho}^{\;\!\mathcolor{gray}{t}}_{\;\!\textcolor{Maroon}{\text{b}}\mathcolor{gray}{z}}$ 的 \bref{eq:X-01} 并为之带来下一阶不连续/表面项。}。

此外,不仅 \bref{eq:X-01} 需要为 ${\rho}^{\;\!\mathcolor{gray}{t}}_{\;\!\textcolor{Maroon}{\text{b}}\mathcolor{gray}{z}}$ 拓展,以适应对 $\bar{J}^{\;\!\mathcolor{gray}{t}}_{\;\!\textcolor{Maroon}{\text{b}}\mathcolor{gray}{z}}$ 的 \bref{eq:X-01} 展开的一阶 $\mathcolor{gray}{z}$ 向偏导所生成的 $\delta_{\mathcolor{gray}{z}}$ 函数 —— $\bar{J}^{\;\!\mathcolor{gray}{t}}_{\;\!\textcolor{Maroon}{\text{b}}\mathcolor{gray}{z}}$ 本身还有表面电流(如磁光材料),相应地 $\bar{J}^{\;\!\mathcolor{gray}{t}}_{\;\!\textcolor{Maroon}{\text{b}}\mathcolor{gray}{z}}$ 的原始 \bref{eq:X-01} 也应该添加一个 $\delta_{\mathcolor{gray}{z}}$ 函数;那么为满足 \bref{eq:X-01},对应 ${\rho}^{\;\!\mathcolor{gray}{t}}_{\;\!\textcolor{Maroon}{\text{b}}\mathcolor{gray}{z}}$ 的表达式会继续出现 $\delta'_{\mathcolor{gray}{z}}$ 项。因此需要同时将 ${\rho}^{\;\!\mathcolor{gray}{t}}_{\;\!\textcolor{Maroon}{\text{b}}\mathcolor{gray}{z}},\bar{J}^{\;\!\mathcolor{gray}{t}}_{\;\!\textcolor{Maroon}{\text{b}}\mathcolor{gray}{z}}$ 分别拓展 \bref{eq:X-01} 至:
\begin{subequations} \label{eq:e-b-01}
%	\abovedisplayskip=-3pt
%	\belowdisplayskip=12pt
\begin{align}
	J^{\;\!\mathcolor{gray}{t}}_{\;\!\textcolor{Maroon}{\text{b}} \symup{\iota}\mathcolor{gray}{z}} &= &&\hspace{-4.5em}\leftindex_{\textcolor{Maroon}{\mathsfit{z}}} {\mathbb{1}}_{\mathcolor{gray}{z}} \leftindex^{\textcolor{Maroon}{\mathsfit{z}}} \;\! J^{\;\!\mathcolor{gray}{t}}_{\;\!\textcolor{Maroon}{\text{b}} \symup{\iota}\mathcolor{gray}{z}} &&\hspace{-4.5em}+ \delta_{\mathcolor{gray}{z}} \leftindex_{\textcolor{Maroon}{\mathsfit{z}}} \;\! \leftindex^{\textcolor{Maroon}{\mathsfit{z}}}
	{\mathcal{K}}^{\;\!\mathcolor{gray}{t}}_{\;\!\textcolor{Maroon}{\text{b}} \symup{\iota}\symup{z}\mathcolor{gray}{z}} \leftindex^{\textcolor{Maroon}{\mathsfit{z}}} \;\! n_{\mathcolor{gray}{z}} &&\hspace{-4.5em} \\ 
	&= &&\hspace{-4.5em}\leftindex_{\textcolor{Maroon}{\mathsfit{z}}} {\mathbb{1}}_{\mathcolor{gray}{z}} \leftindex^{\textcolor{Maroon}{\mathsfit{z}}} \;\! J^{\;\!\mathcolor{gray}{t}}_{\;\!\textcolor{Maroon}{\text{b}} \symup{\iota}\mathcolor{gray}{z}} &&\hspace{-4.5em}- \leftindex_{\textcolor{Maroon}{\mathsfit{z}}} \;\! \delta_{\mathcolor{gray}{z}} \leftindex^{\textcolor{Maroon}{\mathsfit{z}}}
	{\mathcal{K}}^{\;\!\mathcolor{gray}{t}}_{\;\!\textcolor{Maroon}{\text{b}} \symup{\iota}\symup{z}\mathcolor{gray}{z}} ~, &&\hspace{-4.5em} \label{eq:j-b-01} \\
	{\rho}^{\;\!\mathcolor{gray}{t}}_{\;\!\textcolor{Maroon}{\text{b}}\mathcolor{gray}{z}} &= &&\hspace{-4.5em}\leftindex_{\textcolor{Maroon}{\mathsfit{z}}} {\mathbb{1}}_{\mathcolor{gray}{z}} \leftindex^{\textcolor{Maroon}{\mathsfit{z}}} {\rho}^{\;\!\mathcolor{gray}{t}}_{\;\!\textcolor{Maroon}{\text{b}}\mathcolor{gray}{z}} &&\hspace{-4.5em}+ \delta_{\mathcolor{gray}{z}} \leftindex_{\textcolor{Maroon}{\mathsfit{z}}} \;\! \leftindex^{\textcolor{Maroon}{\mathsfit{z}}} \;\! {\mathcal{P}}^{\;\!\mathcolor{gray}{t}}_{\;\!\textcolor{Maroon}{\text{b}} \symup{z} \mathcolor{gray}{z}} \leftindex^{\textcolor{Maroon}{\mathsfit{z}}} \;\! n_{\mathcolor{gray}{z}} &&\hspace{-4.5em}+ \delta'_{\mathcolor{gray}{z}} \leftindex_{\textcolor{Maroon}{\mathsfit{z}}} \;\! \leftindex^{\textcolor{Maroon}{\mathsfit{z}}} \;\! {\mathcal{Q}}^{\;\!\mathcolor{gray}{t}}_{\;\!\textcolor{Maroon}{\text{b}} \symup{z} \symup{z} \mathcolor{gray}{z}} \leftindex^{\textcolor{Maroon}{\mathsfit{z}}} \;\! n_{\mathcolor{gray}{z}} \\ 
	&= &&\hspace{-4.5em}\leftindex_{\textcolor{Maroon}{\mathsfit{z}}} {\mathbb{1}}_{\mathcolor{gray}{z}} \leftindex^{\textcolor{Maroon}{\mathsfit{z}}} {\rho}^{\;\!\mathcolor{gray}{t}}_{\;\!\textcolor{Maroon}{\text{b}}\mathcolor{gray}{z}} &&\hspace{-4.5em}- \leftindex_{\textcolor{Maroon}{\mathsfit{z}}} \;\! \delta_{\mathcolor{gray}{z}} \leftindex^{\textcolor{Maroon}{\mathsfit{z}}} \;\! {\mathcal{P}}^{\;\!\mathcolor{gray}{t}}_{\;\!\textcolor{Maroon}{\text{b}} \symup{z} \mathcolor{gray}{z}} &&\hspace{-4.5em}- \leftindex_{\textcolor{Maroon}{\mathsfit{z}}} \;\! \delta'_{\mathcolor{gray}{z}} \leftindex^{\textcolor{Maroon}{\mathsfit{z}}} \;\! {\mathcal{Q}}^{\;\!\mathcolor{gray}{t}}_{\;\!\textcolor{Maroon}{\text{b}} \symup{z} \symup{z} \mathcolor{gray}{z}} ~, \label{eq:p-b-01}
\end{align}
\end{subequations}
其中,定义了 介质 $\textcolor{Maroon}{\mathsfit{z}}$ 在 $\mathcolor{gray}{z} = \mathcolor{gray}{0}$ 面上的表面单位外法向量 $\leftindex^{\textcolor{Maroon}{\mathsfit{z}}} {\hat{n}}_{\mathcolor{gray}{z}}$ 与 $\hat{\symup{e}}_{\mathcolor{gray}{z}}$ 的点积
\begin{align} \label{eq:surf-normal-dot}
	\leftindex^{\textcolor{Maroon}{\mathsfit{z}}} \;\! n_{\mathcolor{gray}{z}} := - \symup{sgn} \left( \leftindex^{\textcolor{Maroon}{\mathsfit{z}}} \;\! \delta_{\mathcolor{gray}{z}} \right) = - \symup{sgn} \left( \leftindex^{\textcolor{Maroon}{\mathsfit{z}}} \;\! \delta'_{\mathcolor{gray}{z}} \right) = \cdots = \leftindex^{\textcolor{Maroon}{\mathsfit{z}}} {\hat{n}}_{\mathcolor{gray}{z}} \cdot \hat{\symup{e}}_{\mathcolor{gray}{z}}~.
\end{align}
为一步到位,\bref{eq:j-b-01,eq:p-b-01} 还可以进一步拓展至高一阶项:
\begin{subequations} \label{eq:e-b-01'}
\begin{align}
	J^{\;\!\mathcolor{gray}{t}}_{\;\!\textcolor{Maroon}{\text{b}} \symup{\iota}\mathcolor{gray}{z}} &= &&\hspace{-2em}\leftindex_{\textcolor{Maroon}{\mathsfit{z}}} {\mathbb{1}}_{\mathcolor{gray}{z}} \leftindex^{\textcolor{Maroon}{\mathsfit{z}}} \;\! J^{\;\!\mathcolor{gray}{t}}_{\;\!\textcolor{Maroon}{\text{b}} \symup{\iota}\mathcolor{gray}{z}} &&\hspace{-2em}- \leftindex_{\textcolor{Maroon}{\mathsfit{z}}} \;\! \delta_{\mathcolor{gray}{z}} \leftindex^{\textcolor{Maroon}{\mathsfit{z}}}
	{\mathcal{K}}^{\;\!\mathcolor{gray}{t}}_{\;\!\textcolor{Maroon}{\text{b}} \symup{\iota}\symup{z}\mathcolor{gray}{z}} &&\hspace{-2em}- \leftindex_{\textcolor{Maroon}{\mathsfit{z}}} \;\! \delta'_{\mathcolor{gray}{z}} \leftindex^{\textcolor{Maroon}{\mathsfit{z}}} \;\! {\mathcal{L}}^{\;\!\mathcolor{gray}{t}}_{\;\!\textcolor{Maroon}{\text{b}} \symup{\iota}\symup{z} \symup{z} \mathcolor{gray}{z}} ~, &&\hspace{-2em} \label{eq:j-b-01'} \\
	{\rho}^{\;\!\mathcolor{gray}{t}}_{\;\!\textcolor{Maroon}{\text{b}}\mathcolor{gray}{z}} &= &&\hspace{-2em}\leftindex_{\textcolor{Maroon}{\mathsfit{z}}} {\mathbb{1}}_{\mathcolor{gray}{z}} \leftindex^{\textcolor{Maroon}{\mathsfit{z}}} {\rho}^{\;\!\mathcolor{gray}{t}}_{\;\!\textcolor{Maroon}{\text{b}}\mathcolor{gray}{z}} &&\hspace{-2em}- \leftindex_{\textcolor{Maroon}{\mathsfit{z}}} \;\! \delta_{\mathcolor{gray}{z}} \leftindex^{\textcolor{Maroon}{\mathsfit{z}}} \;\! {\mathcal{P}}^{\;\!\mathcolor{gray}{t}}_{\;\!\textcolor{Maroon}{\text{b}} \symup{z} \mathcolor{gray}{z}} &&\hspace{-2em}- \leftindex_{\textcolor{Maroon}{\mathsfit{z}}} \;\! \delta'_{\mathcolor{gray}{z}} \leftindex^{\textcolor{Maroon}{\mathsfit{z}}} \;\! {\mathcal{Q}}^{\;\!\mathcolor{gray}{t}}_{\;\!\textcolor{Maroon}{\text{b}} \symup{z} \symup{z} \mathcolor{gray}{z}} &&\hspace{-2em}- \leftindex_{\textcolor{Maroon}{\mathsfit{z}}} \;\! \delta''_{\mathcolor{gray}{z}} \leftindex^{\textcolor{Maroon}{\mathsfit{z}}} \;\! {\mathcal{O}}^{\;\!\mathcolor{gray}{t}}_{\;\!\textcolor{Maroon}{\text{b}} \symup{z} \symup{z} \symup{z} \mathcolor{gray}{z}} ~, \label{eq:p-b-01'}
\end{align}
\end{subequations}
现将 \bref{eq:j-b-01',eq:p-b-01'} 代入 \bref{eq:div-e-b} 中\Footnote{利用 \bref{eq:div-e-b-01'} 和 \bref{eq:partial_z-step} 所导出的:$\mathcolor{gray}{\nabla^\iota} \left( \leftindex_{\textcolor{Maroon}{\mathsfit{z}}} {\mathbb{1}}_{\mathcolor{gray}{z}} \leftindex^{\textcolor{Maroon}{\mathsfit{z}}} \;\! J^{\;\!\mathcolor{gray}{t}}_{\;\!\textcolor{Maroon}{\text{b}} \symup{\iota}\mathcolor{gray}{z}} \right) = \leftindex_{\textcolor{Maroon}{\mathsfit{z}}} \;\! \delta_{\mathcolor{gray}{z}} \leftindex^{\textcolor{Maroon}{\mathsfit{z}}} \;\! J^{\;\!\mathcolor{gray}{t}}_{\;\!\textcolor{Maroon}{\text{b}} \symup{z} \mathcolor{gray}{z}} + \leftindex_{\textcolor{Maroon}{\mathsfit{z}}} {\mathbb{1}}_{\mathcolor{gray}{z}} \mathcolor{gray}{\nabla^\iota} \leftindex^{\textcolor{Maroon}{\mathsfit{z}}} \;\! J^{\;\!\mathcolor{gray}{t}}_{\;\!\textcolor{Maroon}{\text{b}} \symup{\iota}\mathcolor{gray}{z}}$。},同介质 $\textcolor{Maroon}{\mathsfit{z}}$ 内(另一侧介质是真空时也须成立),各同阶次内的奇异项必须分别(分层次)为零,因此有
\begin{subequations} \label{eq:div-e-b-01-deltas}
\begin{align}
	{\delta}_{\mathcolor{gray}{z}} ~\textcolor{Maroon}{\text{项}}:&\hspace{1.0em} + \leftindex^{\textcolor{Maroon}{\mathsfit{z}}} {\delta}_{\mathcolor{gray}{z}} \leftindex^{\textcolor{Maroon}{\mathsfit{z}}} \;\! J^{\;\!\mathcolor{gray}{t}}_{\;\!\textcolor{Maroon}{\text{b}} \symup{z}\mathcolor{gray}{z}} &&\hspace{-1.5em}- \leftindex^{\textcolor{Maroon}{\mathsfit{z}}} {\delta}_{\mathcolor{gray}{z}} \left( \mathcolor{gray}{\nabla^\iota} \leftindex^{\textcolor{Maroon}{\mathsfit{z}}}
	{\mathcal{K}}^{\;\!\mathcolor{gray}{t}}_{\;\!\textcolor{Maroon}{\text{b}} \symup{\iota}\symup{z}\mathcolor{gray}{z}} \right. &&\hspace{-1.5em}+ \left. \mathcolor{gray}{\nabla^t} \leftindex^{\textcolor{Maroon}{\mathsfit{z}}} \;\! {\mathcal{P}}^{\;\!\mathcolor{gray}{t}}_{\;\!\textcolor{Maroon}{\text{b}} \symup{z} \mathcolor{gray}{z}} \right) &&\hspace{-1.5em}= 0~, \label{eq:div-e-b-01-delta} \\
	{\delta}'_{\mathcolor{gray}{z}} ~\textcolor{Maroon}{\text{项}}:&\hspace{1.0em} - \leftindex^{\textcolor{Maroon}{\mathsfit{z}}} \delta'_{\mathcolor{gray}{z}} \leftindex^{\textcolor{Maroon}{\mathsfit{z}}}
	{\mathcal{K}}^{\;\!\mathcolor{gray}{t}}_{\;\!\textcolor{Maroon}{\text{b}} \symup{z} \symup{z}\mathcolor{gray}{z}} &&\hspace{-1.5em}- \leftindex^{\textcolor{Maroon}{\mathsfit{z}}} \delta'_{\mathcolor{gray}{z}} \left( \mathcolor{gray}{\nabla^\iota} \leftindex^{\textcolor{Maroon}{\mathsfit{z}}} \;\! {\mathcal{L}}^{\;\!\mathcolor{gray}{t}}_{\;\!\textcolor{Maroon}{\text{b}} \symup{\iota}\symup{z} \symup{z} \mathcolor{gray}{z}} \right. &&\hspace{-1.5em}+ \left. \mathcolor{gray}{\nabla^t} \leftindex^{\textcolor{Maroon}{\mathsfit{z}}} \;\! {\mathcal{Q}}^{\;\!\mathcolor{gray}{t}}_{\;\!\textcolor{Maroon}{\text{b}} \symup{z} \symup{z} \mathcolor{gray}{z}} \right) &&\hspace{-1.5em}= 0~, \label{eq:div-e-b-01-delta'} \\
	{\delta}''_{\mathcolor{gray}{z}} ~\textcolor{Maroon}{\text{项}}:&\hspace{1.0em} - \leftindex^{\textcolor{Maroon}{\mathsfit{z}}} {\delta}''_{\mathcolor{gray}{z}} \leftindex^{\textcolor{Maroon}{\mathsfit{z}}} \;\! {\mathcal{L}}^{\;\!\mathcolor{gray}{t}}_{\;\!\textcolor{Maroon}{\text{b}} \symup{z} \symup{z} \symup{z} \mathcolor{gray}{z}} &&\hspace{-1.5em}- \leftindex^{\textcolor{Maroon}{\mathsfit{z}}} \delta''_{\mathcolor{gray}{z}} \left( \mathcolor{gray}{\nabla^\iota} \leftindex^{\textcolor{Maroon}{\mathsfit{z}}} \;\! \cdots \right. &&\hspace{-1.5em}+ \left. \mathcolor{gray}{\nabla^t} \leftindex^{\textcolor{Maroon}{\mathsfit{z}}} \;\! {\mathcal{O}}^{\;\!\mathcolor{gray}{t}}_{\;\!\textcolor{Maroon}{\text{b}} \symup{z} \symup{z} \symup{z} \mathcolor{gray}{z}} \right) &&\hspace{-1.5em}= 0~, \label{eq:div-e-b-01-delta''}
\end{align}
\end{subequations}
即有 下述关系,与介质无关地 成立(实际上利用到了 \bref{eq:Intdeltasum=0})\Footnote{由于 2 个下标相同,看上去 \bref{eq:div-e-b-01-delta'-conclusion} 中的 $\mathcolor{gray}{\nabla^\iota} {\mathcal{L}}^{\;\!\mathcolor{gray}{t}}_{\;\!\textcolor{Maroon}{\text{m}} \symup{\iota} \symup{z} \symup{z} \mathcolor{gray}{z}} = 0$ 但实则不然;\bref{eq:div-e-b-01-delta''-conclusion} 只考虑到电八/磁四级(忽略更高阶矩)。}:
\begin{subequations} \label{eq:div-e-b-01-delta-conclusions}
\begin{align}
	{\delta}_{\mathcolor{gray}{z}} ~\textcolor{Maroon}{\text{项}}:&\hspace{1.0em}  J^{\;\!\mathcolor{gray}{t}}_{\;\!\textcolor{Maroon}{\text{b}} \symup{z} \mathcolor{gray}{0}} \hspace{-1.7em}&&=\hspace{0.2em} + \hspace{0.2em} \left( \mathcolor{gray}{\nabla^\iota} {\mathcal{K}}^{\;\!\mathcolor{gray}{t}}_{\;\!\textcolor{Maroon}{\text{b}} \symup{\iota}\symup{z}\mathcolor{gray}{0}} \hspace{-2.5em}\right. &&\hspace{0.8em}+ \left. \mathcolor{gray}{\nabla^t} {\mathcal{P}}^{\;\!\mathcolor{gray}{t}}_{\;\!\textcolor{Maroon}{\text{b}} \symup{z} \mathcolor{gray}{0}} \right) \hspace{-1.6em}&&=+\hspace{0.2em} \mathcolor{gray}{\nabla^t} {\mathcal{P}}^{\;\!\mathcolor{gray}{t}}_{\;\!\textcolor{Maroon}{\text{b}} \symup{z} \mathcolor{gray}{0}} + \mathcolor{gray}{\nabla^\iota} {\mathcal{K}}^{\;\!\mathcolor{gray}{t}}_{\;\!\textcolor{Maroon}{\text{b}} \symup{\iota}\symup{z}\mathcolor{gray}{0}}~, \label{eq:div-e-b-01-delta-conclusion} \\
	{\delta}'_{\mathcolor{gray}{z}} ~\textcolor{Maroon}{\text{项}}:&\hspace{1.0em}
	{\mathcal{K}}^{\;\!\mathcolor{gray}{t}}_{\;\!\textcolor{Maroon}{\text{b}} \symup{z} \symup{z}\mathcolor{gray}{0}} \hspace{-1.7em}&&=\hspace{0.2em} - \hspace{0.2em} \left( \mathcolor{gray}{\nabla^\iota} {\mathcal{L}}^{\;\!\mathcolor{gray}{t}}_{\;\!\textcolor{Maroon}{\text{b}} \symup{\iota}\symup{z} \symup{z} \mathcolor{gray}{0}} \hspace{-2.5em}\right. &&\hspace{0.8em}+ \left. \mathcolor{gray}{\nabla^t} {\mathcal{Q}}^{\;\!\mathcolor{gray}{t}}_{\;\!\textcolor{Maroon}{\text{b}} \symup{z} \symup{z} \mathcolor{gray}{0}} \right) \hspace{-1.6em}&&=-\hspace{0.2em} \mathcolor{gray}{\nabla^t} {\mathcal{Q}}^{\;\!\mathcolor{gray}{t}}_{\;\!\textcolor{Maroon}{\text{b}} \symup{z} \symup{z} \mathcolor{gray}{0}} - \mathcolor{gray}{\nabla^\iota} {\mathcal{L}}^{\;\!\mathcolor{gray}{t}}_{\;\!\textcolor{Maroon}{\text{b}} \symup{\iota}\symup{z} \symup{z} \mathcolor{gray}{0}}~, \label{eq:div-e-b-01-delta'-conclusion} \\
	{\delta}''_{\mathcolor{gray}{z}} ~\textcolor{Maroon}{\text{项}}:&\hspace{1.0em} {\mathcal{L}}^{\;\!\mathcolor{gray}{t}}_{\;\!\textcolor{Maroon}{\text{b}} \symup{z} \symup{z} \symup{z} \mathcolor{gray}{0}} \hspace{-1.7em}&&=\hspace{0.2em} - \hspace{0.2em} \left( \mathcolor{gray}{\nabla^\iota} \cdots \hspace{-2.5em}\right. &&\hspace{0.8em}+ \left. \mathcolor{gray}{\nabla^t} {\mathcal{O}}^{\;\!\mathcolor{gray}{t}}_{\;\!\textcolor{Maroon}{\text{b}} \symup{z} \symup{z} \symup{z} \mathcolor{gray}{0}} \right) \hspace{-1.6em}&&=-\hspace{0.2em} \mathcolor{gray}{\nabla^t} {\mathcal{O}}^{\;\!\mathcolor{gray}{t}}_{\;\!\textcolor{Maroon}{\text{b}} \symup{z} \symup{z} \symup{z} \mathcolor{gray}{0}} - \mathcolor{gray}{\nabla^\iota} \cdots~. \label{eq:div-e-b-01-delta''-conclusion}
\end{align}
\end{subequations}
经典的多极理论 \cite{raabMultipoleTheoryElectromagnetism2004} 给出 \bref{eq:j-b-01',eq:p-b-01'} 中的各体/表面电荷/流项:
\begin{subequations} \label{eq:multipole}
\begin{align}
	&{\mathcal{O}}^{\;\!\mathcolor{gray}{t}}_{\;\!\textcolor{Maroon}{\text{b}} \symup{\iota}\hat{1}\hat{2} \mathcolor{gray}{z}} \hspace{-1em}&&=\hspace{0.2em} +~ O^{\;\!\mathcolor{gray}{t}}_{\;\! \symup{\iota}\hat{1}\hat{2}\mathcolor{gray}{z}} &&\hspace{-1.1em}- \cdots~, &&\hspace{0.3em} {\mathcal{L}}^{\;\!\mathcolor{gray}{t}}_{\;\!\textcolor{Maroon}{\text{m}} \symup{\iota}\hat{1}\hat{3} \mathcolor{gray}{z}} \hspace{-1em}&&=\hspace{0.2em} +~ \epsilon^{\hphantom{\symup{\iota}\hat{1}}\hat{2}}_{\symup{\iota}\mathcolor{gray}{\hat{1}}} N^{\;\!\mathcolor{gray}{t}}_{\;\! \hat{2}\hat{3} \mathcolor{gray}{z}} &&\hspace{-1.1em}- \cdots~, \label{eq:Ob-Lm} \\
	&{\mathcal{Q}}^{\;\!\mathcolor{gray}{t}}_{\;\!\textcolor{Maroon}{\text{b}} \symup{\iota}\hat{1} \mathcolor{gray}{z}} \hspace{-1em}&&=\hspace{0.2em} -~ Q^{\;\!\mathcolor{gray}{t}}_{\;\! \symup{\iota}\hat{1}\mathcolor{gray}{z}} &&\hspace{-1.1em}+ \mathcolor{gray}{\nabla^{\hat{2}}} {\mathcal{O}}^{\;\!\mathcolor{gray}{t}}_{\;\!\textcolor{Maroon}{\text{b}} \symup{\iota}\hat{1}\hat{2} \mathcolor{gray}{z}}~,  &&\hspace{0.3em} {\mathcal{K}}^{\;\!\mathcolor{gray}{t}}_{\;\!\textcolor{Maroon}{\text{m}} \symup{\iota}\hat{1} \mathcolor{gray}{z}} \hspace{-1em}&&=\hspace{0.2em} -~ \epsilon^{\hphantom{\symup{\iota}\hat{1}}\hat{2}}_{\symup{\iota}\mathcolor{gray}{\hat{1}}} M^{\;\!\mathcolor{gray}{t}}_{\;\! \hat{2}\mathcolor{gray}{z}} &&\hspace{-1.1em}+ \mathcolor{gray}{\nabla^{\hat{3}}} {\mathcal{L}}^{\;\!\mathcolor{gray}{t}}_{\;\!\textcolor{Maroon}{\text{m}} \symup{\iota}\hat{1}\hat{3} \mathcolor{gray}{z}}~, \label{eq:Qb-Km} \\
	&{\mathcal{P}}^{\;\!\mathcolor{gray}{t}}_{\;\!\textcolor{Maroon}{\text{b}} \symup{\iota} \mathcolor{gray}{z}} \hspace{-1em}&&=\hspace{0.2em} \hphantom{+}~ P^{\;\!\mathcolor{gray}{t}}_{\;\! \symup{\iota}\mathcolor{gray}{z}} &&\hspace{-1.1em}+ \mathcolor{gray}{\nabla^{\hat{1}}} {\mathcal{Q}}^{\;\!\mathcolor{gray}{t}}_{\;\!\textcolor{Maroon}{\text{b}} \symup{\iota}\hat{1} \mathcolor{gray}{z}}~, &&\hspace{0.3em} {J}^{\;\!\mathcolor{gray}{t}}_{\;\!\textcolor{Maroon}{\text{m}} \symup{\iota} \mathcolor{gray}{z}} \hspace{-1em}&&=\hspace{0.2em} &&\hspace{-1.1em}- \mathcolor{gray}{\nabla^{\hat{1}}} {\mathcal{K}}^{\;\!\mathcolor{gray}{t}}_{\;\!\textcolor{Maroon}{\text{m}} \symup{\iota}\hat{1} \mathcolor{gray}{z}}~, \label{eq:Pb-Jm} \\
	&{\rho}^{\;\!\mathcolor{gray}{t}}_{\;\!\textcolor{Maroon}{\text{b}} \mathcolor{gray}{z}} \hspace{-1em}&&=\hspace{0.2em} &&\hspace{-1.1em}- \mathcolor{gray}{\nabla^\iota} {\mathcal{P}}^{\;\!\mathcolor{gray}{t}}_{\;\!\textcolor{Maroon}{\text{b}} \symup{\iota} \mathcolor{gray}{z}}~, &&\hspace{0.3em} \hspace{-1em}&& &&\hspace{-1.1em} \label{eq:pb} \\
	&{J}^{\;\!\mathcolor{gray}{t}}_{\;\!\textcolor{Maroon}{\text{e}} \symup{\iota} \mathcolor{gray}{z}} \hspace{-1em}&&=\hspace{0.2em} &&\hspace{-1.1em}+ \mathcolor{gray}{\nabla^t} {\mathcal{P}}^{\;\!\mathcolor{gray}{t}}_{\;\!\textcolor{Maroon}{\text{b}} \symup{\iota} \mathcolor{gray}{z}}~, &&\hspace{0.3em} {J}^{\;\!\mathcolor{gray}{t}}_{\;\!\textcolor{Maroon}{\text{b}} \symup{\iota} \mathcolor{gray}{z}} \hspace{-1em}&&=\hspace{0.2em} ~ \hphantom{- \epsilon^{\hphantom{\symup{\iota}\hat{1}}\hat{2}}_{\symup{\iota}\mathcolor{gray}{\hat{1}}}} {J}^{\;\!\mathcolor{gray}{t}}_{\;\!\textcolor{Maroon}{\text{e}} \symup{\iota} \mathcolor{gray}{z}} &&\hspace{-1.1em}+ {J}^{\;\!\mathcolor{gray}{t}}_{\;\!\textcolor{Maroon}{\text{m}} \symup{\iota} \mathcolor{gray}{z}}~, \label{eq:Je-Jb} \\
	&{\mathcal{K}}^{\;\!\mathcolor{gray}{t}}_{\;\!\textcolor{Maroon}{\text{e}} \symup{\iota}\hat{1} \mathcolor{gray}{z}} \hspace{-1em}&&=\hspace{0.2em} &&\hspace{-1.1em}- \mathcolor{gray}{\nabla^t} {\mathcal{Q}}^{\;\!\mathcolor{gray}{t}}_{\;\!\textcolor{Maroon}{\text{b}} \symup{\iota}\hat{1} \mathcolor{gray}{z}}~, &&\hspace{0.3em} {\mathcal{K}}^{\;\!\mathcolor{gray}{t}}_{\;\!\textcolor{Maroon}{\text{b}} \symup{\iota}\hat{1} \mathcolor{gray}{z}} \hspace{-1em}&&=\hspace{0.2em} ~ \hphantom{- \epsilon^{\hphantom{\symup{\iota}\hat{1}}\hat{2}}_{\symup{\iota}\mathcolor{gray}{\hat{1}}}} {\mathcal{K}}^{\;\!\mathcolor{gray}{t}}_{\;\!\textcolor{Maroon}{\text{e}} \symup{\iota}\hat{1} \mathcolor{gray}{z}} &&\hspace{-1.1em}+ {\mathcal{K}}^{\;\!\mathcolor{gray}{t}}_{\;\!\textcolor{Maroon}{\text{m}} \symup{\iota}\hat{1} \mathcolor{gray}{z}}~, \label{eq:Ke-Kb} \\
	&{\mathcal{L}}^{\;\!\mathcolor{gray}{t}}_{\;\!\textcolor{Maroon}{\text{e}} \symup{\iota}\hat{1}\hat{3} \mathcolor{gray}{z}} \hspace{-1em}&&=\hspace{0.2em} &&\hspace{-1.1em}- \mathcolor{gray}{\nabla^t} {\mathcal{O}}^{\;\!\mathcolor{gray}{t}}_{\;\!\textcolor{Maroon}{\text{b}} \symup{\iota}\hat{1}\hat{3} \mathcolor{gray}{z}}~, &&\hspace{0.3em} {\mathcal{L}}^{\;\!\mathcolor{gray}{t}}_{\;\!\textcolor{Maroon}{\text{b}} \symup{\iota}\hat{1}\hat{3} \mathcolor{gray}{z}} \hspace{-1em}&&=\hspace{0.2em} ~ \hphantom{- \epsilon^{\hphantom{\symup{\iota}\hat{1}}\hat{2}}_{\symup{\iota}\mathcolor{gray}{\hat{1}}}} {\mathcal{L}}^{\;\!\mathcolor{gray}{t}}_{\;\!\textcolor{Maroon}{\text{e}} \symup{\iota}\hat{1}\hat{3} \mathcolor{gray}{z}} &&\hspace{-1.1em}+ {\mathcal{L}}^{\;\!\mathcolor{gray}{t}}_{\;\!\textcolor{Maroon}{\text{m}} \symup{\iota}\hat{1}\hat{3} \mathcolor{gray}{z}}~, \label{eq:Le-Lb}
\end{align}
\end{subequations}
可以验证,上述经典多极理论 \bref{eq:multipole} 并不能自动将每一阶/级奇异项消除,即无法自动满足 \bref{eq:div-e-b-01-delta-conclusions} 中的 3 个奇异层次 ${\delta}_{\mathcolor{gray}{z}},{\delta}'_{\mathcolor{gray}{z}},{\delta}''_{\mathcolor{gray}{z}}$ 对应的方程。

为验证这 2 种理论的一致性,将 \bref{eq:Le-Lb,eq:Ke-Kb,eq:Je-Jb} 展开
\begin{subequations} \label{eq:JKL}
\begin{align}
	{J}^{\;\!\mathcolor{gray}{t}}_{\;\!\textcolor{Maroon}{\text{b}} \symup{\iota} \mathcolor{gray}{z}} &=+\hspace{0.2em} \mathcolor{gray}{\nabla^t} {\mathcal{P}}^{\;\!\mathcolor{gray}{t}}_{\;\!\textcolor{Maroon}{\text{b}} \symup{\iota} \mathcolor{gray}{z}} - \mathcolor{gray}{\nabla^{\hat{1}}} {\mathcal{K}}^{\;\!\mathcolor{gray}{t}}_{\;\!\textcolor{Maroon}{\text{m}} \symup{\iota}\hat{1} \mathcolor{gray}{z}}~, \label{eq:Jb} \\
	{\mathcal{K}}^{\;\!\mathcolor{gray}{t}}_{\;\!\textcolor{Maroon}{\text{b}} \symup{\iota}\hat{1} \mathcolor{gray}{z}} &= -\hspace{0.2em} \mathcolor{gray}{\nabla^t} {\mathcal{Q}}^{\;\!\mathcolor{gray}{t}}_{\;\!\textcolor{Maroon}{\text{b}} \symup{\iota}\hat{1} \mathcolor{gray}{z}} + {\mathcal{K}}^{\;\!\mathcolor{gray}{t}}_{\;\!\textcolor{Maroon}{\text{m}} \symup{\iota}\hat{1} \mathcolor{gray}{z}}~, \label{eq:Kb} \\
	{\mathcal{L}}^{\;\!\mathcolor{gray}{t}}_{\;\!\textcolor{Maroon}{\text{b}} \symup{\iota}\hat{1}\hat{3} \mathcolor{gray}{z}} &= -\hspace{0.2em} \mathcolor{gray}{\nabla^t} {\mathcal{O}}^{\;\!\mathcolor{gray}{t}}_{\;\!\textcolor{Maroon}{\text{b}} \symup{\iota}\hat{1}\hat{3} \mathcolor{gray}{z}} + \epsilon^{\hphantom{\symup{\iota}\hat{1}}\hat{2}}_{\symup{\iota}\mathcolor{gray}{\hat{1}}} N^{\;\!\mathcolor{gray}{t}}_{\;\! \hat{2}\hat{3} \mathcolor{gray}{z}}~, \label{eq:Lb}
\end{align}
\end{subequations}
上述 {\one} 经典多极理论 \bref{eq:Lb} 的预言:${\mathcal{L}}^{\;\!\mathcolor{gray}{t}}_{\;\!\textcolor{Maroon}{\text{b}} \symup{z} \symup{z} \symup{z} \mathcolor{gray}{z}} = - \mathcolor{gray}{\nabla^t} {\mathcal{O}}^{\;\!\mathcolor{gray}{t}}_{\;\!\textcolor{Maroon}{\text{b}} \symup{z} \symup{z} \symup{z} \mathcolor{gray}{z}} + \epsilon^{\hphantom{\symup{z} \symup{z}}\hat{2}}_{\symup{z} \mathcolor{gray}{\symup{z}}} N^{\;\!\mathcolor{gray}{t}}_{\;\! \hat{2}\symup{z}\mathcolor{gray}{z}} = - \mathcolor{gray}{\nabla^t} {\mathcal{O}}^{\;\!\mathcolor{gray}{t}}_{\;\!\textcolor{Maroon}{\text{b}} \symup{z} \symup{z} \symup{z} \mathcolor{gray}{z}}$,恰好等于奇异边界条件 \bref{eq:div-e-b-01-delta''-conclusion} 所给出的 ${\mathcal{L}}^{\;\!\mathcolor{gray}{t}}_{\;\!\textcolor{Maroon}{\text{b}} \symup{z} \symup{z} \symup{z} \mathcolor{gray}{0}} = - \mathcolor{gray}{\nabla^t} {\mathcal{O}}^{\;\!\mathcolor{gray}{t}}_{\;\!\textcolor{Maroon}{\text{b}} \symup{z} \symup{z} \symup{z} \mathcolor{gray}{0}}$;{\two} 经典多极理论 \bref{eq:Kb} 的预言:${\mathcal{K}}^{\;\!\mathcolor{gray}{t}}_{\;\!\textcolor{Maroon}{\text{b}} \symup{z} \symup{z} \mathcolor{gray}{z}} = - \mathcolor{gray}{\nabla^t} {\mathcal{Q}}^{\;\!\mathcolor{gray}{t}}_{\;\!\textcolor{Maroon}{\text{b}} \symup{z} \symup{z} \mathcolor{gray}{z}} + {\mathcal{K}}^{\;\!\mathcolor{gray}{t}}_{\;\!\textcolor{Maroon}{\text{m}} \symup{z} \symup{z} \mathcolor{gray}{z}} = - \mathcolor{gray}{\nabla^t} {\mathcal{Q}}^{\;\!\mathcolor{gray}{t}}_{\;\!\textcolor{Maroon}{\text{b}} \symup{z} \symup{z} \mathcolor{gray}{z}}$,暂不等于奇异边界条件 \bref{eq:div-e-b-01-delta'-conclusion} 所导出的 ${\mathcal{K}}^{\;\!\mathcolor{gray}{t}}_{\;\!\textcolor{Maroon}{\text{b}} \symup{z} \symup{z} \mathcolor{gray}{0}} = - \left( \mathcolor{gray}{\nabla^t} {\mathcal{Q}}^{\;\!\mathcolor{gray}{t}}_{\;\!\textcolor{Maroon}{\text{b}} \symup{z} \symup{z} \mathcolor{gray}{0}} + \mathcolor{gray}{\nabla^\iota} {\mathcal{L}}^{\;\!\mathcolor{gray}{t}}_{\;\!\textcolor{Maroon}{\text{b}} \symup{\iota}\symup{z} \symup{z} \mathcolor{gray}{0}} \right)$;{\three} 经典多极理论 \bref{eq:Jb} 的预言:${J}^{\;\!\mathcolor{gray}{t}}_{\;\!\textcolor{Maroon}{\text{b}} \symup{z} \mathcolor{gray}{z}} = \mathcolor{gray}{\nabla^t} {\mathcal{P}}^{\;\!\mathcolor{gray}{t}}_{\;\!\textcolor{Maroon}{\text{b}} \symup{z} \mathcolor{gray}{z}} - \mathcolor{gray}{\nabla^{\hat{1}}} {\mathcal{K}}^{\;\!\mathcolor{gray}{t}}_{\;\!\textcolor{Maroon}{\text{m}} \symup{z} \hat{1} \mathcolor{gray}{z}} = \mathcolor{gray}{\nabla^t} {\mathcal{P}}^{\;\!\mathcolor{gray}{t}}_{\;\!\textcolor{Maroon}{\text{b}} \symup{z} \mathcolor{gray}{z}} + \mathcolor{gray}{\nabla^\iota} {\mathcal{K}}^{\;\!\mathcolor{gray}{t}}_{\;\!\textcolor{Maroon}{\text{m}} \symup{\iota} \symup{z} \mathcolor{gray}{z}}$,也暂不等于奇异边界条件 \bref{eq:div-e-b-01-delta-conclusion} 所导出的 ${J}^{\;\!\mathcolor{gray}{t}}_{\;\!\textcolor{Maroon}{\text{b}} \symup{z} \mathcolor{gray}{0}} = \mathcolor{gray}{\nabla^t} {\mathcal{P}}^{\;\!\mathcolor{gray}{t}}_{\;\!\textcolor{Maroon}{\text{b}} \symup{z} \mathcolor{gray}{0}} + \mathcolor{gray}{\nabla^\iota} {\mathcal{K}}^{\;\!\mathcolor{gray}{t}}_{\;\!\textcolor{Maroon}{\text{b}} \symup{\iota}\symup{z}\mathcolor{gray}{0}}$。

可见,奇异边界条件的预测结果,似乎总比经典多极理论的预测结果,多出至少一项极化电流表面/体密度项:比如在 ${\delta}_{\mathcolor{gray}{z}}$ 层次即 ${J}^{\;\!\mathcolor{gray}{t}}_{\;\!\textcolor{Maroon}{\text{b}} \symup{z} \mathcolor{gray}{z}}$ 中,会多出一项 $\mathcolor{gray}{\nabla^\iota} {\mathcal{K}}^{\;\!\mathcolor{gray}{t}}_{\;\!\textcolor{Maroon}{\text{e}} \symup{\iota} \symup{z}\mathcolor{gray}{0}} = - \mathcolor{gray}{\nabla^\iota} \mathcolor{gray}{\nabla^t} {\mathcal{Q}}^{\;\!\mathcolor{gray}{t}}_{\;\!\textcolor{Maroon}{\text{b}} \symup{\iota} \symup{z} \mathcolor{gray}{z}}$ 并与 $\mathcolor{gray}{\nabla^t} {\mathcal{P}}^{\;\!\mathcolor{gray}{t}}_{\;\!\textcolor{Maroon}{\text{b}} \symup{z} \mathcolor{gray}{z}}$ 中的对应项 $\mathcolor{gray}{\nabla^t} \mathcolor{gray}{\nabla^{\hat{2}}} {\mathcal{Q}}^{\;\!\mathcolor{gray}{t}}_{\;\!\textcolor{Maroon}{\text{b}} \symup{z}\hat{2} \mathcolor{gray}{z}}$ 约掉(由于多极矩的置换对称性\cite{raabMultipoleTheoryElectromagnetism2004});又比如会在 ${\delta}'_{\mathcolor{gray}{z}}$ 层次即 ${K}^{\;\!\mathcolor{gray}{t}}_{\;\!\textcolor{Maroon}{\text{b}} \symup{z} \symup{z} \mathcolor{gray}{z}}$ 中,会多出一项 $\mathcolor{gray}{\nabla^\iota} {\mathcal{L}}^{\;\!\mathcolor{gray}{t}}_{\;\!\textcolor{Maroon}{\text{e}} \symup{\iota} \symup{z} \symup{z} \mathcolor{gray}{0}} = - \mathcolor{gray}{\nabla^\iota} \mathcolor{gray}{\nabla^t} {\mathcal{O}}^{\;\!\mathcolor{gray}{t}}_{\;\!\textcolor{Maroon}{\text{b}} \symup{\iota}\symup{z} \symup{z} \mathcolor{gray}{0}}$ 并与 $\mathcolor{gray}{\nabla^t} {\mathcal{Q}}^{\;\!\mathcolor{gray}{t}}_{\;\!\textcolor{Maroon}{\text{b}} \symup{z} \symup{z} \mathcolor{gray}{z}}$ 中的对应项 $\mathcolor{gray}{\nabla^t} \mathcolor{gray}{\nabla^{\hat{2}}} {\mathcal{O}}^{\;\!\mathcolor{gray}{t}}_{\;\!\textcolor{Maroon}{\text{b}} \symup{z} \symup{z} \hat{2} \mathcolor{gray}{0}}$ 约掉(同样由于多极矩的置换对称性)。此外,在 ${\delta}'_{\mathcolor{gray}{z}}$ 层次即 ${K}^{\;\!\mathcolor{gray}{t}}_{\;\!\textcolor{Maroon}{\text{b}} \symup{z} \symup{z} \mathcolor{gray}{z}}$ 中,还会多出一项 $\mathcolor{gray}{\nabla^\iota} {\mathcal{L}}^{\;\!\mathcolor{gray}{t}}_{\;\!\textcolor{Maroon}{\text{m}} \symup{\iota} \symup{z} \symup{z} \mathcolor{gray}{z}} = \epsilon^{\hphantom{\symup{\iota}\hat{1}}\hat{2}}_{i \mathcolor{gray}{\symup{z}}} \mathcolor{gray}{\nabla^\iota} N^{\;\!\mathcolor{gray}{t}}_{\;\! \hat{2}\symup{z}\mathcolor{gray}{z}}$。

因此,不禁会像爱因斯坦一样发问:哪个理论错了?—— 答案是奇异边界条件正确,经典多极理论部分错误 —— 后者的磁化表面电流 ${\mathcal{K}}^{\;\!\mathcolor{gray}{t}}_{\;\!\textcolor{Maroon}{\text{m}} \symup{\iota}\hat{1} \mathcolor{gray}{z}},{\mathcal{L}}^{\;\!\mathcolor{gray}{t}}_{\;\!\textcolor{Maroon}{\text{m}} \symup{\iota}\hat{1}\hat{2} \mathcolor{gray}{z}}$,以及极化表面电荷/流项 ${\mathcal{P}}^{\;\!\mathcolor{gray}{t}}_{\;\!\textcolor{Maroon}{\text{b}} \symup{\iota} \mathcolor{gray}{z}},{\mathcal{Q}}^{\;\!\mathcolor{gray}{t}}_{\;\!\textcolor{Maroon}{\text{b}} \symup{\iota}\hat{1} \mathcolor{gray}{z}},{\mathcal{O}}^{\;\!\mathcolor{gray}{t}}_{\;\!\textcolor{Maroon}{\text{b}} \symup{\iota}\hat{1}\hat{2} \mathcolor{gray}{z}};{\mathcal{K}}^{\;\!\mathcolor{gray}{t}}_{\;\!\textcolor{Maroon}{\text{e}} \symup{\iota}\hat{1} \mathcolor{gray}{z}},{\mathcal{L}}^{\;\!\mathcolor{gray}{t}}_{\;\!\textcolor{Maroon}{\text{e}} \symup{\iota}\hat{1}\hat{2} \mathcolor{gray}{z}}$ 错了\Footnote{经典多极理论给出的表面项错了,以及这样修改它们的原因,见 \textcolor{Maroon}{O.L. DE LANGE} 等人的著作\cite{grahamMultipoleSolutionMacroscopic2000,raabMultipoleTheoryElectromagnetism2004,delangeElectromagneticBoundaryConditions2013}。}(但极/磁化体电荷/流项 ${\rho}^{\;\!\mathcolor{gray}{t}}_{\;\!\textcolor{Maroon}{\text{b}} \mathcolor{gray}{z}};{J}^{\;\!\mathcolor{gray}{t}}_{\;\!\textcolor{Maroon}{\text{e}} \symup{\iota} \mathcolor{gray}{z}},{J}^{\;\!\mathcolor{gray}{t}}_{\;\!\textcolor{Maroon}{\text{m}} \symup{\iota} \mathcolor{gray}{z}}$ 均正确)。接着,在假设奇异边界条件是正确的条件下,修正多极理论。为满足奇异边界条件 \bref{eq:div-e-b-01-delta-conclusions},多极理论必须满足以下层次关系:
\begin{subequations} \label{eq:multipole-JKL}
\begin{align}
	&{J}^{\;\!\mathcolor{gray}{t}}_{\;\!\textcolor{Maroon}{\text{e}} \symup{z} \mathcolor{gray}{z}} \hspace{-2.5em}&&=\hspace{0.2em} &&\hspace{-2.5em}+ \mathcolor{gray}{\nabla^t} {\mathcal{P}}^{\;\!\mathcolor{gray}{t}}_{\;\! \symup{z} \mathcolor{gray}{z}} + \mathcolor{gray}{\nabla^{\hat{1}}} {\mathcal{K}}^{\;\!\mathcolor{gray}{t}}_{\;\!\textcolor{Maroon}{\text{E}} \hat{1} \symup{z}\mathcolor{gray}{z}}~, &&\hspace{0.3em} {J}^{\;\!\mathcolor{gray}{t}}_{\;\!\textcolor{Maroon}{\text{m}} \symup{z} \mathcolor{gray}{z}} \hspace{-2.5em}&&=\hspace{0.2em} +~ \mathcolor{gray}{\nabla^{\hat{1}}} {\mathcal{K}}^{\;\!\mathcolor{gray}{t}}_{\;\!\textcolor{Maroon}{\text{M}} \hat{1} \symup{z} \mathcolor{gray}{z}}~, \label{eq:Je-JM} \\
	&{\mathcal{K}}^{\;\!\mathcolor{gray}{t}}_{\;\!\textcolor{Maroon}{\text{E}} \symup{z}\symup{z} \mathcolor{gray}{z}} \hspace{-2.5em}&&=\hspace{0.2em} &&\hspace{-2.5em}- \mathcolor{gray}{\nabla^t} {\mathcal{Q}}^{\;\!\mathcolor{gray}{t}}_{\;\! \symup{z}\symup{z} \mathcolor{gray}{z}} - \mathcolor{gray}{\nabla^{\hat{3}}} {\mathcal{L}}^{\;\!\mathcolor{gray}{t}}_{\;\!\textcolor{Maroon}{\text{E}} \hat{3} \symup{z}\symup{z}\mathcolor{gray}{z}}~, &&\hspace{0.3em} {\mathcal{K}}^{\;\!\mathcolor{gray}{t}}_{\;\!\textcolor{Maroon}{\text{M}} \symup{z}\symup{z} \mathcolor{gray}{z}} \hspace{-2.5em}&&=\hspace{0.2em} -~ \mathcolor{gray}{\nabla^{\hat{3}}} {\mathcal{L}}^{\;\!\mathcolor{gray}{t}}_{\;\!\textcolor{Maroon}{\text{M}} \hat{3} \symup{z}\symup{z} \mathcolor{gray}{z}}~, \label{eq:Ke-KM} \\
	&{\mathcal{L}}^{\;\!\mathcolor{gray}{t}}_{\;\!\textcolor{Maroon}{\text{E}} \symup{z}\symup{z}\symup{z} \mathcolor{gray}{z}} \hspace{-2.5em}&&=\hspace{0.2em} &&\hspace{-2.5em}- \mathcolor{gray}{\nabla^t} {\mathcal{O}}^{\;\!\mathcolor{gray}{t}}_{\;\! \symup{z}\symup{z}\symup{z} \mathcolor{gray}{z}} - \mathcolor{gray}{\nabla^m} \cdots~, &&\hspace{0.3em} {\mathcal{L}}^{\;\!\mathcolor{gray}{t}}_{\;\!\textcolor{Maroon}{\text{M}} \symup{z}\symup{z}\symup{z} \mathcolor{gray}{z}} \hspace{-2.5em}&&=\hspace{0.2em} -~ \mathcolor{gray}{\nabla^m} \cdots~, \label{eq:Le-LM}
\end{align}
\end{subequations}
其中,角标 $\textcolor{Maroon}{\text{E}},\textcolor{Maroon}{\text{M}}$ 是更新了的 $\textcolor{Maroon}{\text{e}},\textcolor{Maroon}{\text{m}}$。无角标 $\textcolor{Maroon}{\text{b}}$ 的 $\mathcal{P},\mathcal{Q},\mathcal{O}$ 分别是更新了的 $\mathcal{P}_{\;\!\textcolor{Maroon}{\text{b}}},\mathcal{Q}_{\;\!\textcolor{Maroon}{\text{b}}},\mathcal{O}_{\;\!\textcolor{Maroon}{\text{b}}}$。所有的这些修正量,重新适用于 \bref{eq:e-b-01',eq:div-e-b-01-deltas,eq:div-e-b-01-delta-conclusions}。

为保持极化体电荷/流公式 ${\rho}^{\;\!\mathcolor{gray}{t}}_{\;\!\textcolor{Maroon}{\text{b}} \mathcolor{gray}{z}};{J}^{\;\!\mathcolor{gray}{t}}_{\;\!\textcolor{Maroon}{\text{b}} \symup{\iota} \mathcolor{gray}{z}},{J}^{\;\!\mathcolor{gray}{t}}_{\;\!\textcolor{Maroon}{\text{e}} \symup{\iota} \mathcolor{gray}{z}},{J}^{\;\!\mathcolor{gray}{t}}_{\;\!\textcolor{Maroon}{\text{m}} \symup{\iota} \mathcolor{gray}{z}}$ 不变的同时,满足电荷守恒 \bref{eq:div-e-b},经典多极理论 \bref{eq:multipole},需要修正并进化为现代奇异版本:
\begin{subequations} \label{eq:multipole-modify}
\begin{align}
	&{\mathcal{O}}^{\;\!\mathcolor{gray}{t}}_{\;\! \symup{\iota}\hat{1}\hat{2} \mathcolor{gray}{z}} \hspace{-2.0em}&&=\hspace{0.2em} {\mathcal{O}}^{\;\!\mathcolor{gray}{t}}_{\;\!\textcolor{Maroon}{\text{b}} \symup{\iota}\hat{1}\hat{2} \mathcolor{gray}{z}} &&\hspace{-2.0em}+ \cdots~, &&\hspace{0.3em} {\mathcal{L}}^{\;\!\mathcolor{gray}{t}}_{\;\!\textcolor{Maroon}{\text{M}} \symup{\iota}\hat{1}\hat{3} \mathcolor{gray}{z}} \hspace{-2.0em}&&=\hspace{0.2em} {\mathcal{L}}^{\;\!\mathcolor{gray}{t}}_{\;\!\textcolor{Maroon}{\text{m}} \symup{\iota}\hat{1}\hat{3} \mathcolor{gray}{z}} &&\hspace{-2.0em}- \cdots~, \label{eq:Ob-LM} \\
	&{\mathcal{Q}}^{\;\!\mathcolor{gray}{t}}_{\;\! \symup{\iota}\hat{1} \mathcolor{gray}{z}} \hspace{-2.0em}&&=\hspace{0.2em} {\mathcal{Q}}^{\;\!\mathcolor{gray}{t}}_{\;\!\textcolor{Maroon}{\text{b}} \symup{\iota}\hat{1} \mathcolor{gray}{z}} &&\hspace{-2.0em}+ \mathcolor{gray}{\nabla^{\hat{2}}} {\mathcal{O}}^{\;\!\mathcolor{gray}{t}}_{\;\!\textcolor{Maroon}{\text{b}} \symup{\iota}\hat{1}\hat{2} \mathcolor{gray}{z}}~,  &&\hspace{0.3em} {\mathcal{K}}^{\;\!\mathcolor{gray}{t}}_{\;\!\textcolor{Maroon}{\text{M}} \symup{\iota}\hat{1} \mathcolor{gray}{z}} \hspace{-2.0em}&&=\hspace{0.2em} {\mathcal{K}}^{\;\!\mathcolor{gray}{t}}_{\;\!\textcolor{Maroon}{\text{m}} \symup{\iota}\hat{1} \mathcolor{gray}{z}} &&\hspace{-2.0em}- \mathcolor{gray}{\nabla^{\hat{3}}} {\mathcal{L}}^{\;\!\mathcolor{gray}{t}}_{\;\!\textcolor{Maroon}{\text{m}} l\symup{\iota}\hat{1} \mathcolor{gray}{z}}~, \label{eq:Qb-KM} \\
	&{\mathcal{P}}^{\;\!\mathcolor{gray}{t}}_{\;\! \symup{\iota} \mathcolor{gray}{z}} \hspace{-2.0em}&&=\hspace{0.2em} {\mathcal{P}}^{\;\!\mathcolor{gray}{t}}_{\;\!\textcolor{Maroon}{\text{b}} \symup{\iota} \mathcolor{gray}{z}} &&\hspace{-2.0em}+ \mathcolor{gray}{\nabla^{\hat{1}}} {\mathcal{Q}}^{\;\!\mathcolor{gray}{t}}_{\;\!\textcolor{Maroon}{\text{b}} \symup{\iota}\hat{1} \mathcolor{gray}{z}}~, &&\hspace{0.3em} {\mathcal{Q}}^{\;\!\mathcolor{gray}{t}}_{\;\!\textcolor{Maroon}{\text{b}} \symup{\iota}\hat{1} \mathcolor{gray}{z}} \hspace{-2.0em}&&=\hspace{0.2em} {\mathcal{Q}}^{\;\!\mathcolor{gray}{t}}_{\;\!\textcolor{Maroon}{\text{b}} \symup{\iota}\hat{1} \mathcolor{gray}{z}} &&\hspace{-2.0em}+ \mathcolor{gray}{\nabla^{\hat{2}}} {\mathcal{O}}^{\;\!\mathcolor{gray}{t}}_{\;\!\textcolor{Maroon}{\text{b}} \symup{\iota}\hat{1}\hat{2} \mathcolor{gray}{z}}~, \label{eq:Pb-QB} \\
	&{\mathcal{K}}^{\;\!\mathcolor{gray}{t}}_{\;\!\textcolor{Maroon}{\text{E}} \symup{\iota}\hat{1} \mathcolor{gray}{z}} \hspace{-2.0em}&&=\hspace{0.2em} &&\hspace{-2.0em}- \mathcolor{gray}{\nabla^t} {\mathcal{Q}}^{\;\!\mathcolor{gray}{t}}_{\;\!\textcolor{Maroon}{\text{b}} \symup{\iota}\hat{1} \mathcolor{gray}{z}}~, &&\hspace{0.3em} {\mathcal{K}}^{\;\!\mathcolor{gray}{t}}_{\;\! \symup{\iota}\hat{1} \mathcolor{gray}{z}} \hspace{-2.0em}&&=\hspace{0.2em} {\mathcal{K}}^{\;\!\mathcolor{gray}{t}}_{\;\!\textcolor{Maroon}{\text{E}} \symup{\iota}\hat{1} \mathcolor{gray}{z}} &&\hspace{-2.0em}+ {\mathcal{K}}^{\;\!\mathcolor{gray}{t}}_{\;\!\textcolor{Maroon}{\text{M}} \symup{\iota}\hat{1} \mathcolor{gray}{z}}~, \label{eq:KE-Kb} \\
	&{\mathcal{L}}^{\;\!\mathcolor{gray}{t}}_{\;\!\textcolor{Maroon}{\text{E}} \symup{\iota}\hat{1}\hat{3} \mathcolor{gray}{z}} \hspace{-2.0em}&&=\hspace{0.2em} &&\hspace{-2.0em}- \mathcolor{gray}{\nabla^t} {\mathcal{O}}^{\;\!\mathcolor{gray}{t}}_{\;\!\textcolor{Maroon}{\text{b}} \symup{\iota}\hat{1}\hat{3} \mathcolor{gray}{z}}~, &&\hspace{0.3em} {\mathcal{L}}^{\;\!\mathcolor{gray}{t}}_{\;\! \symup{\iota}\hat{1}\hat{3} \mathcolor{gray}{z}} \hspace{-2.0em}&&=\hspace{0.2em} {\mathcal{L}}^{\;\!\mathcolor{gray}{t}}_{\;\!\textcolor{Maroon}{\text{E}} \symup{\iota}\hat{1}\hat{3} \mathcolor{gray}{z}} &&\hspace{-2.0em}+ {\mathcal{L}}^{\;\!\mathcolor{gray}{t}}_{\;\!\textcolor{Maroon}{\text{M}} \symup{\iota}\hat{1}\hat{3} \mathcolor{gray}{z}}~, \label{eq:LE-Lb}
\end{align}
\end{subequations}
可以验证,上述修正后的多极理论 \bref{eq:multipole-modify} 满足\Footnote{提示,$\mathcolor{gray}{\nabla^\iota} \mathcolor{gray}{\nabla^{\hat{3}}} {\mathcal{L}}^{\;\!\mathcolor{gray}{t}}_{\;\!\textcolor{Maroon}{\text{m}} l\symup{\iota}\hat{1} \mathcolor{gray}{z}} = \mathcolor{gray}{\nabla^{\hat{1}}} \mathcolor{gray}{\nabla^\iota} {\mathcal{L}}^{\;\!\mathcolor{gray}{t}}_{\;\!\textcolor{Maroon}{\text{m}} \symup{\iota}\hat{1}\hat{3} \mathcolor{gray}{z}} = 0$,但不是每项都为零,而是爱因斯坦求和后,两两抵消,一共三对。}奇异展开 \bref{eq:e-b-01'} 和 束缚电荷守恒 \bref{eq:div-e-b} 所导出的 奇异边界条件 \bref{eq:div-e-b-01-delta-conclusions} 即 \bref{eq:multipole-JKL}。此外,上述修正后的多极理论 \bref{eq:multipole-modify} 也可以从经典多极理论自身(即回溯 \bref{eq:multipole} 的来源)出发\cite{raabMultipoleTheoryElectromagnetism2004,delangeElectromagneticBoundaryConditions2013} 以找到问题所在,并修补之以得到正果。可以验证,上述 \bref{eq:multipole-modify} 与文献\cite{delangeElectromagneticBoundaryConditions2013}的主要结果的绝大部分一致,除了与其公式 (6) 的最末一项符号相反。此问题应归结为文献\cite{delangeElectromagneticBoundaryConditions2013}的作者将那处的 $-$ 误打错成 $+$。

需要强调的是,\bref{eq:e-b-01'} 中的每一层次的奇异项,都含有低阶至高阶的多极矩(体密度/表面有限厚度的薄层内的面密度等)\Footnote{其中越高阶的多极矩,越需要更多的散度来降张量的阶,以保持在同一层次的奇异表面/体项内,各多极矩项的总张量阶次相同。};此外,各奇异项每多一个分量(张量每升一阶),就在单位上多乘一个 $\mathcolor{gray}{\symup{m}}$;配合相应的带有单位 $\mathcolor{gray}{\symup{m}^{-1}},\mathcolor{gray}{\symup{m}^{-2}},\cdots$ 的 $\delta_{\mathcolor{gray}{z}}$ 函数及其导数,\bref{eq:e-b-01'} 中各奇异项的(总)单位才保持不变。

总之,以上通过对比修正后的经典多极理论 \bref{eq:multipole-modify} 所导出的各阶表面电流密度 \bref{eq:multipole-JKL} 的 $\symup{z}$ 、 $\symup{z} \symup{z}$ 、 $\symup{z} \symup{z} \symup{z}$ 分量,与奇异边界条件 \bref{eq:e-b-01'} 配合 \bref{eq:div-e-b-01-deltas} 所要求的 \bref{eq:div-e-b-01-delta-conclusions},两个来源独立的假说却各自最终获得了相同的结果,因此同时验证了 \bref{eq:e-b-01'} 、 \bref{eq:multipole} 和 \bref{eq:multipole-modify} 的正确性。

此外,\bref{eq:e-b-01'} 的 另一点 显而易见的 正确性是,当 介质 \textcolor{Maroon}{1} 与介质 \textcolor{Maroon}{2} 的晶格结构和取向全同时(尽管它们处于不同的两个半空间 $\leftindex_{\textcolor{Maroon}{1}} {\mathbb{1}}_{\mathcolor{gray}{z}}$、$\leftindex_{\textcolor{Maroon}{2}} {\mathbb{1}}_{\mathcolor{gray}{z}}$),即当 \bref{eq:multipole} 中的各项(与 $\textcolor{Maroon}{\mathsfit{z}}$ 无关地;或者退一步只需要在 $\mathcolor{gray}{z \mathcolor{black}{=} 0}$ 处/左右连续即可)在介质 \textcolor{Maroon}{1} 与介质 \textcolor{Maroon}{2} 中完全一致时,\bref{eq:e-b-01'} 中的所有奇异/表面/接触项会自动约去并消失,甚至不需要 \bref{eq:div-e-b-01-delta-conclusions} 及多极矩理论(给出的显式 \bref{eq:multipole} 和 \bref{eq:multipole-modify})成立。—— 这说明奇异项确实对应表面项,即只存在于体表面,而不是体内部。

如果自由电流/荷 ${Q}^{\;\!\mathcolor{gray}{t}}_{\;\!\textcolor{Maroon}{\text{f}}\mathcolor{gray}{z}}$ 也单独守恒(但可能从束缚电源 ${Q}^{\;\!\mathcolor{gray}{t}}_{\;\!\textcolor{Maroon}{\text{b}}\mathcolor{gray}{z}}$ 转换过来),那么原则上 ${\rho}^{\;\!\mathcolor{gray}{t}}_{\;\!\textcolor{Maroon}{\text{f}}\mathcolor{gray}{z}},\bar{J}^{\;\!\mathcolor{gray}{t}}_{\;\!\textcolor{Maroon}{\text{f}}\mathcolor{gray}{z}}$ 也应需要以 \bref{eq:e-b-01'} 的形式展开,且各奇异层次间满足 \bref{eq:div-e-b-01-delta-conclusions},以遵循自由电荷守恒 \bref{eq:div-e-f};但对应的 \bref{eq:e-b-01'} 中的各阶自由体/表面电荷/流密度,不再有经典多极理论的显式 \bref{eq:multipole}。因此,对于自由电流/荷 ${Q}^{\;\!\mathcolor{gray}{t}}_{\;\!\textcolor{Maroon}{\text{f}}\mathcolor{gray}{z}}$,原则上只需要将该节中的每条公式中各变量的下角标 \textcolor{Maroon}{\text{b}} 替换为 \textcolor{Maroon}{\text{f}} 即可。为清晰起见,给出 \bref{eq:e-b-01'} 的自由电荷版本:
\begin{subequations} \label{eq:e-f-01}
\begin{align}
	J^{\;\!\mathcolor{gray}{t}}_{\;\!\textcolor{Maroon}{\text{f}} \symup{\iota}\mathcolor{gray}{z}} &= \leftindex_{\textcolor{Maroon}{\mathsfit{z}}} {\mathbb{1}}_{\mathcolor{gray}{z}} \leftindex^{\textcolor{Maroon}{\mathsfit{z}}} \;\! J^{\;\!\mathcolor{gray}{t}}_{\;\!\textcolor{Maroon}{\text{f}} \symup{\iota}\mathcolor{gray}{z}} + \delta_{\mathcolor{gray}{z}} \leftindex_{\textcolor{Maroon}{\mathsfit{z}}} \;\! \leftindex^{\textcolor{Maroon}{\mathsfit{z}}} \;\!
	{\alpha}^{\;\!\mathcolor{gray}{t}}_{\;\!\textcolor{Maroon}{\text{f}} \symup{\iota}\mathcolor{gray}{z}} ~, \label{eq:j-f-01} \\
	{\rho}^{\;\!\mathcolor{gray}{t}}_{\;\!\textcolor{Maroon}{\text{f}}\mathcolor{gray}{z}} &= \leftindex_{\textcolor{Maroon}{\mathsfit{z}}} {\mathbb{1}}_{\mathcolor{gray}{z}} \leftindex^{\textcolor{Maroon}{\mathsfit{z}}} {\rho}^{\;\!\mathcolor{gray}{t}}_{\;\!\textcolor{Maroon}{\text{f}}\mathcolor{gray}{z}} + \delta_{\mathcolor{gray}{z}} \leftindex_{\textcolor{Maroon}{\mathsfit{z}}} \;\! \leftindex^{\textcolor{Maroon}{\mathsfit{z}}} \;\! {\sigma}^{\;\!\mathcolor{gray}{t}}_{\;\!\textcolor{Maroon}{\text{f}} \mathcolor{gray}{z}} ~, \label{eq:p-f-01}
\end{align}
\end{subequations}
注意,由于(额外的)自由电荷不会在两个接触端面自动抵消(而是代数求和),且会因库伦力和在导带中“自由”而快速移动并最终平衡,因此上 \bref{eq:e-f-01} 与 \bref{eq:e-b-01'} 在 $\delta_{\mathcolor{gray}{z}}$ 函数的正负上略有不同。因此, \bref{eq:div-e-b-01-delta-conclusions} 的自由电荷版应写作
\begin{align} \label{eq:div-e-f-01-delta-conclusions}
	{\delta}_{\mathcolor{gray}{z}} ~\textcolor{Maroon}{\text{项}}:  \leftindex^{\textcolor{Maroon}{\mathsfit{z}}} \;\! J^{\;\!\mathcolor{gray}{t}}_{\;\!\textcolor{Maroon}{\text{f}} \symup{z} \mathcolor{gray}{0}} = \leftindex^{\textcolor{Maroon}{\mathsfit{z}}} \;\! n_{\mathcolor{gray}{z}} \left( \mathcolor{gray}{\nabla^t} {\sigma}^{\;\!\mathcolor{gray}{t}}_{\;\!\textcolor{Maroon}{\text{f}} \mathcolor{gray}{0}} + \mathcolor{gray}{\nabla^\iota} {\alpha}^{\;\!\mathcolor{gray}{t}}_{\;\!\textcolor{Maroon}{\text{f}} \symup{\iota}\mathcolor{gray}{0}} \right)~.
\end{align}
注意,该 \bref{eq:div-e-f-01-delta-conclusions} 没有更高阶的 ${\delta}_{\mathcolor{gray}{z}}$ 函数项:自由电流/荷 ${Q}^{\;\!\mathcolor{gray}{t}}_{\;\!\textcolor{Maroon}{\text{f}}\mathcolor{gray}{z}}$ 不需要 \bref{eq:div-e-b-01-delta'-conclusion} 及以上的奇异层次 \cite{dengTheoryElectrodynamicResponse2020,delangeElectromagneticBoundaryConditions2013}。此外,\bref{eq:div-e-f-01-delta-conclusions} 变得与材料表面法向的取向有关。由于无多重含义和字母冲突,表面自由电荷/流 ${\sigma}_{\;\!\textcolor{Maroon}{\text{f}}},{\alpha}_{\;\!\textcolor{Maroon}{\text{f}} \symup{\iota}}$ 均可省略下标 $\textcolor{Maroon}{\text{f}}$ 写作 ${\sigma},{\alpha}_{\;\! \symup{\iota}}$。

然而,为了减少符号数量,以及后续的形式统一,总可以将 $\leftindex^{\textcolor{Maroon}{\mathsfit{z}}} \;\!
{\alpha}^{\;\!\mathcolor{gray}{t}}_{\;\! \symup{\iota}\mathcolor{gray}{z}}, \leftindex^{\textcolor{Maroon}{\mathsfit{z}}} \;\! {\sigma}^{\;\!\mathcolor{gray}{t}}_{\;\! \mathcolor{gray}{z}}$ 重新定义为 $-\leftindex^{\textcolor{Maroon}{\mathsfit{z}}} \;\! n_{\mathcolor{gray}{z}} \leftindex^{\textcolor{Maroon}{\mathsfit{z}}} \;\!
{\alpha}^{\;\!\mathcolor{gray}{t}}_{\;\! \symup{\iota}\mathcolor{gray}{z}}, -\leftindex^{\textcolor{Maroon}{\mathsfit{z}}} \;\! n_{\mathcolor{gray}{z}} \leftindex^{\textcolor{Maroon}{\mathsfit{z}}} \;\! {\sigma}^{\;\!\mathcolor{gray}{t}}_{\;\! \mathcolor{gray}{z}}$,然后对 \bref{eq:e-f-01} 两边乘以 $1 = \left( - \leftindex^{\textcolor{Maroon}{\mathsfit{z}}} \;\! n_{\mathcolor{gray}{z}} \right) \left( - \leftindex^{\textcolor{Maroon}{\mathsfit{z}}} \;\! n_{\mathcolor{gray}{z}} \right)$,即得到新定义的 $\leftindex^{\textcolor{Maroon}{\mathsfit{z}}} \;\!
{\alpha}^{\;\!\mathcolor{gray}{t}}_{\;\! \symup{\iota}\mathcolor{gray}{z}}, \leftindex^{\textcolor{Maroon}{\mathsfit{z}}} \;\! {\sigma}^{\;\!\mathcolor{gray}{t}}_{\;\! \mathcolor{gray}{z}}$ 所满足的、与 \bref{eq:e-b-01'} 类似的形式
\begin{subequations} \label{eq:e-f-01'}
\begin{align}
	J^{\;\!\mathcolor{gray}{t}}_{\;\!\textcolor{Maroon}{\text{f}} \symup{\iota}\mathcolor{gray}{z}} &= \leftindex_{\textcolor{Maroon}{\mathsfit{z}}} {\mathbb{1}}_{\mathcolor{gray}{z}} \leftindex^{\textcolor{Maroon}{\mathsfit{z}}} \;\! J^{\;\!\mathcolor{gray}{t}}_{\;\!\textcolor{Maroon}{\text{f}} \symup{\iota}\mathcolor{gray}{z}} + \leftindex_{\textcolor{Maroon}{\mathsfit{z}}} \;\! \delta_{\mathcolor{gray}{z}} \leftindex^{\textcolor{Maroon}{\mathsfit{z}}} \;\!
	{\alpha}^{\;\!\mathcolor{gray}{t}}_{\;\! \symup{\iota}\mathcolor{gray}{z}} ~, \label{eq:j-f-01'} \\
	{\rho}^{\;\!\mathcolor{gray}{t}}_{\;\!\textcolor{Maroon}{\text{f}} \mathcolor{gray}{z}} &= \leftindex_{\textcolor{Maroon}{\mathsfit{z}}} {\mathbb{1}}_{\mathcolor{gray}{z}} \leftindex^{\textcolor{Maroon}{\mathsfit{z}}} {\rho}^{\;\!\mathcolor{gray}{t}}_{\;\!\textcolor{Maroon}{\text{f}} \mathcolor{gray}{z}} + \leftindex_{\textcolor{Maroon}{\mathsfit{z}}} \;\! \delta_{\mathcolor{gray}{z}} \leftindex^{\textcolor{Maroon}{\mathsfit{z}}} \;\! {\sigma}^{\;\!\mathcolor{gray}{t}}_{\;\! \mathcolor{gray}{z}} ~, \label{eq:p-f-01'}
\end{align}
\end{subequations}
相应地,重新定义的 $\leftindex^{\textcolor{Maroon}{\mathsfit{z}}} \;\!
{\alpha}^{\;\!\mathcolor{gray}{t}}_{\;\! \symup{\iota}\mathcolor{gray}{z}}, \leftindex^{\textcolor{Maroon}{\mathsfit{z}}} \;\! {\sigma}^{\;\!\mathcolor{gray}{t}}_{\;\! \mathcolor{gray}{z}}$ 所满足的 \bref{eq:div-e-f-01-delta-conclusions} 变为
\abovedisplayskip=8pt
\belowdisplayskip=8pt
\begin{align} \label{eq:div-e-f-01-delta-conclusions'}
	{\delta}_{\mathcolor{gray}{z}} ~\textcolor{Maroon}{\text{项}}:  J^{\;\!\mathcolor{gray}{t}}_{\;\!\textcolor{Maroon}{\text{f}} \symup{z} \mathcolor{gray}{0}} = - \left( \mathcolor{gray}{\nabla^t} {\sigma}^{\;\!\mathcolor{gray}{t}}_{\;\! \mathcolor{gray}{0}} + \mathcolor{gray}{\nabla^\iota} {\alpha}^{\;\!\mathcolor{gray}{t}}_{\;\! \symup{\iota}\mathcolor{gray}{0}} \right)~.
\end{align}
\bref{eq:e-f-01'} 相比 \bref{eq:e-f-01},通过给 ${\delta}_{\mathcolor{gray}{z}}$ 函数引入 $\textcolor{Maroon}{\mathsfit{z}}$ 为代价,消除了 \bref{eq:div-e-f-01-delta-conclusions} 中的 $\textcolor{Maroon}{\mathsfit{z}}$ 以成为 \bref{eq:div-e-f-01-delta-conclusions'}。

事实上,对于“各阶奇异项之和为零”
\abovedisplayskip=8pt
\begin{align} \label{eq:deltasum=0}
	{\delta}_{\mathcolor{gray}{z}} f_{\mathcolor{gray}{z}} + {\delta}'_{\mathcolor{gray}{z}} g_{\mathcolor{gray}{z}} + 
	{\delta}''_{\mathcolor{gray}{z}} h_{\mathcolor{gray}{z}} = 0 ~,
\end{align}
有除“各阶奇异项分别为零”即 \bref{eq:div-e-b-01-deltas} 以外的结论存在。对 \bref{eq:deltasum=0}、$\mathcolor{gray}{z}$ 乘以之、$\mathcolor{gray}{z^2}$ 乘以之,分别沿 $\mathcolor{gray}{z}$ 做跨越过 $\mathcolor{gray}{z} = \mathcolor{gray}{0}$ 定积分,使用分部积分法\cite{delangeElectromagneticBoundaryConditions2013},有
\begin{subequations} \label{eq:Intdeltasum=0}
	\belowdisplayskip=8pt
\begin{align}
	f_{\mathcolor{gray}{0}} - g'_{\mathcolor{gray}{0}} + h''_{\mathcolor{gray}{0}} &= 0~, \hspace{1.2em} \Longrightarrow \hspace{1em} f_{\mathcolor{gray}{0}} = g'_{\mathcolor{gray}{0}} = h''_{\mathcolor{gray}{0}} = 0~, \label{eq:intdeltasum=0} \\
	- g_{\mathcolor{gray}{0}} + h'_{\mathcolor{gray}{0}} &= 0~, \hspace{1.2em} \Longrightarrow \hspace{1em} \hphantom{f_{\mathcolor{gray}{0}} =}~\;\! g_{\mathcolor{gray}{0}} = h'_{\mathcolor{gray}{0}} = 0~, \label{eq:intzdeltasum=0} \\
	h_{\mathcolor{gray}{0}} &= 0~, \hspace{1.2em} \Longrightarrow \hspace{1em} \hphantom{f_{\mathcolor{gray}{0}} = g_{\mathcolor{gray}{0}} =}~\;\! h_{\mathcolor{gray}{0}} = 0~, \label{eq:intzzdeltasum=0}
\end{align}
\end{subequations}
此外,起源互相独立的三个函数 $f_{\mathcolor{gray}{z}}$、$g_{\mathcolor{gray}{z}}$、$h_{\mathcolor{gray}{z}}$ 对 $\mathcolor{gray}{x,y}$ 的各阶导数 $\mathcolor{gray}{\nabla_x^n},\mathcolor{gray}{\nabla_y^n}$ 也满足 \bref{eq:deltasum=0},因此它们也满足上述三个 \bref{eq:Intdeltasum=0}。上述所有结论在被验证是否适用于束缚电源 ${Q}^{\;\!\mathcolor{gray}{t}}_{\;\!\textcolor{Maroon}{\text{b}}\mathcolor{gray}{z}}$ 时,都可以用 \bref{eq:multipole-modify} 朝着 \bref{eq:JKL} 的进一步展开来证明。

\vspace*{-4.5em}

\marginLeft[-1.3em]{ssec:EB-boundary}\subsection{$\bar{E},\bar{B}$ 表面场、$\bar{E},\bar{B}$ 边界条件}\label{ssec:EB-boundary}

当 \bref{ssec:PMQN} 的束缚源 \bref{eq:p-b,eq:j-b}、自由电源 \bref{eq:div-e-f},被上面 \bref{ssec:step-delta} 升级为 \bref{eq:e-b-01',eq:e-f-01'} 时,\bref{ssec:EBpJ} 中的 \textcolor{Maroon}{Maxwell-Lorentz-Heaviside} \bref{eq:curl-E,eq:div-B,eq:div-E,eq:curl-B} 升级为
\begin{subequations} \label{eq:curl-EB}
	%	\abovedisplayskip=0pt
	%\belowdisplayskip=0pt
	\small
\begin{align}
	\epsilon^{\hphantom{\symup{\iota}\hat{1}}\hat{2}}_{\symup{\iota}\mathcolor{gray}{\hat{1}}} &\mathcolor{gray}{\nabla^{\hat{1}}} E^{\;\!\mathcolor{gray}{t}}_{\;\! \hat{2}\mathcolor{gray}{z}} + \mathcolor{gray}{\nabla^t} B^{\;\!\mathcolor{gray}{t}}_{\;\! \symup{\iota}\mathcolor{gray}{z}} = 0~, \label{eq:curl-E'} \\
	{\symup{c}}^2 \epsilon^{\hphantom{\symup{\iota}\hat{1}}\hat{2}}_{\symup{\iota}\mathcolor{gray}{\hat{1}}} &\mathcolor{gray}{\nabla^{\hat{1}}} B^{\;\!\mathcolor{gray}{t}}_{\;\! \hat{2}\mathcolor{gray}{z}} - \mathcolor{gray}{\nabla^t} E^{\;\!\mathcolor{gray}{t}}_{\;\! \symup{\iota}\mathcolor{gray}{z}} = \left[ \leftindex_{\textcolor{Maroon}{\mathsfit{z}}} {\mathbb{1}}_{\mathcolor{gray}{z}} \left( \leftindex^{\textcolor{Maroon}{\mathsfit{z}}} \;\! J^{\;\!\mathcolor{gray}{t}}_{\;\!\textcolor{Maroon}{\text{b}} \symup{\iota}\mathcolor{gray}{z}} + \leftindex^{\textcolor{Maroon}{\mathsfit{z}}} \;\! J^{\;\!\mathcolor{gray}{t}}_{\;\!\textcolor{Maroon}{\text{f}} \symup{\iota}\mathcolor{gray}{z}} \right) - \leftindex_{\textcolor{Maroon}{\mathsfit{z}}} \;\! \delta_{\mathcolor{gray}{z}} \left( \leftindex^{\textcolor{Maroon}{\mathsfit{z}}}
	{\mathcal{K}}^{\;\!\mathcolor{gray}{t}}_{\;\! \symup{\iota}\symup{z}\mathcolor{gray}{z}} - \leftindex^{\textcolor{Maroon}{\mathsfit{z}}} \;\!
	{\alpha}^{\;\!\mathcolor{gray}{t}}_{\;\! \symup{\iota}\mathcolor{gray}{z}} \right) - \leftindex_{\textcolor{Maroon}{\mathsfit{z}}} \;\! \delta'_{\mathcolor{gray}{z}} \leftindex^{\textcolor{Maroon}{\mathsfit{z}}} \;\! {\mathcal{L}}^{\;\!\mathcolor{gray}{t}}_{\;\! \symup{\iota}\symup{z} \symup{z} \mathcolor{gray}{z}} \right] \big/ {\symup{\varepsilon}}_0 ~, \label{eq:curl-B'} \\
	{\symup{\varepsilon}}_0 &\mathcolor{gray}{\nabla^\iota} E^{\;\!\mathcolor{gray}{t}}_{\;\! \symup{\iota}\mathcolor{gray}{z}} =  \leftindex_{\textcolor{Maroon}{\mathsfit{z}}} {\mathbb{1}}_{\mathcolor{gray}{z}} \left( \leftindex^{\textcolor{Maroon}{\mathsfit{z}}}  {\rho}^{\;\!\mathcolor{gray}{t}}_{\;\!\textcolor{Maroon}{\text{b}}\mathcolor{gray}{z}} + \leftindex^{\textcolor{Maroon}{\mathsfit{z}}} {\rho}^{\;\!\mathcolor{gray}{t}}_{\;\!\textcolor{Maroon}{\text{f}}\mathcolor{gray}{z}} \right) - \leftindex_{\textcolor{Maroon}{\mathsfit{z}}} \;\! \delta_{\mathcolor{gray}{z}} \left( \leftindex^{\textcolor{Maroon}{\mathsfit{z}}} \;\! {\mathcal{P}}^{\;\!\mathcolor{gray}{t}}_{\;\! \symup{z} \mathcolor{gray}{z}} - \leftindex^{\textcolor{Maroon}{\mathsfit{z}}} \;\! {\sigma}^{\;\!\mathcolor{gray}{t}}_{\;\! \mathcolor{gray}{z}} \right) - \leftindex_{\textcolor{Maroon}{\mathsfit{z}}} \;\! \delta'_{\mathcolor{gray}{z}} \leftindex^{\textcolor{Maroon}{\mathsfit{z}}} \;\! {\mathcal{Q}}^{\;\!\mathcolor{gray}{t}}_{\;\! \symup{z} \symup{z} \mathcolor{gray}{z}} - \leftindex_{\textcolor{Maroon}{\mathsfit{z}}} \;\! \delta''_{\mathcolor{gray}{z}} \leftindex^{\textcolor{Maroon}{\mathsfit{z}}} \;\! {\mathcal{O}}^{\;\!\mathcolor{gray}{t}}_{\;\! \symup{z} \symup{z} \symup{z} \mathcolor{gray}{z}}~, \label{eq:div-E'} \\
	&\mathcolor{gray}{\nabla^\iota} B^{\;\!\mathcolor{gray}{t}}_{\;\! \symup{\iota}\mathcolor{gray}{z}} = 0~. \label{eq:div-B'}
\end{align}
\end{subequations}
上述 \bref{eq:curl-B',eq:div-E'} 要想成立,意味着基本场 $\bar{E}^{\;\!\mathcolor{gray}{t}}_{\;\!\mathcolor{gray}{z}}, \bar{B}^{\;\!\mathcolor{gray}{t}}_{\;\!\mathcolor{gray}{z}}$ 也需要被展开为与(束缚)奇异源 \bref{eq:e-b-01'} 类似的\Footnote{不展开为奇异自由源 \bref{eq:e-f-01} 的类似物,是因为由表面束缚源产生的表面场,也应在同介质内抵消 —— 即因果同形。此外,正如 \bref{eq:e-f-01'} 一样,总可以重定义 \bref{eq:EB-01} 为任何形式,它们都是等价的。},到至少低一层次的奇异场:
\begin{subequations} \label{eq:EB-01}
	\belowdisplayskip=14pt
\begin{align}
	\hphantom{xxxxxxxxx} E^{\;\!\mathcolor{gray}{t}}_{\;\! \symup{\iota}\mathcolor{gray}{z}} &= &&\hspace{-2.5em}\leftindex_{\textcolor{Maroon}{\mathsfit{z}}} {\mathbb{1}}_{\mathcolor{gray}{z}} \leftindex^{\textcolor{Maroon}{\mathsfit{z}}} \;\! E^{\;\!\mathcolor{gray}{t}}_{\;\! \symup{\iota}\mathcolor{gray}{z}} &&\hspace{-2.5em}- \leftindex_{\textcolor{Maroon}{\mathsfit{z}}} \;\! \delta_{\mathcolor{gray}{z}} \leftindex^{\textcolor{Maroon}{\mathsfit{z}}} \;\!
	{\mathcal{E}}^{\;\!\textcolor{Maroon}{(1)}\mathcolor{gray}{t}}_{\;\! \symup{\iota}\mathcolor{gray}{z}} &&\hspace{-2.5em}- \leftindex_{\textcolor{Maroon}{\mathsfit{z}}} \;\! \delta'_{\mathcolor{gray}{z}} \leftindex^{\textcolor{Maroon}{\mathsfit{z}}} \;\! {\mathcal{E}}^{\;\!\textcolor{Maroon}{(2)}\mathcolor{gray}{t}}_{\;\! \symup{\iota}\mathcolor{gray}{z}} ~,&& \label{eq:E-01} \\
	\hphantom{xxxxxxxxx} B^{\;\!\mathcolor{gray}{t}}_{\;\! \symup{\iota}\mathcolor{gray}{z}} &= &&\hspace{-2.5em}\leftindex_{\textcolor{Maroon}{\mathsfit{z}}} {\mathbb{1}}_{\mathcolor{gray}{z}} \leftindex^{\textcolor{Maroon}{\mathsfit{z}}} \;\! B^{\;\!\mathcolor{gray}{t}}_{\;\! \symup{\iota}\mathcolor{gray}{z}} &&\hspace{-2.5em}- \leftindex_{\textcolor{Maroon}{\mathsfit{z}}} \;\! \delta_{\mathcolor{gray}{z}} \leftindex^{\textcolor{Maroon}{\mathsfit{z}}} \;\!
	{\mathcal{B}}^{\;\!\textcolor{Maroon}{(1)}\mathcolor{gray}{t}}_{\;\! \symup{\iota}\mathcolor{gray}{z}} &&\hspace{-2.5em}- \leftindex_{\textcolor{Maroon}{\mathsfit{z}}} \;\! \delta'_{\mathcolor{gray}{z}} \leftindex^{\textcolor{Maroon}{\mathsfit{z}}} \;\! {\mathcal{B}}^{\;\!\textcolor{Maroon}{(2)}\mathcolor{gray}{t}}_{\;\! \symup{\iota}\mathcolor{gray}{z}} ~,&& \label{eq:B-01}
\end{align}
\end{subequations}
将上述 \bref{eq:EB-01} 带入 \bref{eq:curl-EB} 的四个方程中,得到 \bref{eq:div-e-b-01-deltas} 的类似物
%\clearpage
%\vspace*{-3.5em}
\begin{subequations} \label{eq:curl-EB-01}
	%	\abovedisplayskip=0pt
	\belowdisplayskip=0pt
	\footnotesize
\begin{align}
	\epsilon^{\hphantom{\symup{iz}}\hat{2}}_{\symup{\iota}\mathcolor{gray}{\symup{z}}} \left( \leftindex_{\textcolor{Maroon}{\mathsfit{z}}} \;\! \delta_{\mathcolor{gray}{z}} \leftindex^{\textcolor{Maroon}{\mathsfit{z}}} E^{\;\!\mathcolor{gray}{t}}_{\;\! \hat{2}\mathcolor{gray}{z}} - \leftindex_{\textcolor{Maroon}{\mathsfit{z}}} \;\! \delta'_{\mathcolor{gray}{z}} \leftindex^{\textcolor{Maroon}{\mathsfit{z}}} \;\!
	{\mathcal{E}}^{\;\!\textcolor{Maroon}{(1)}\mathcolor{gray}{t}}_{\;\! \hat{2}\mathcolor{gray}{z}} \right) &- \epsilon^{\hphantom{\symup{\iota}\hat{1}}\hat{2}}_{\symup{\iota}\mathcolor{gray}{\hat{1}}} \left( \leftindex_{\textcolor{Maroon}{\mathsfit{z}}} \;\! \delta_{\mathcolor{gray}{z}} \mathcolor{gray}{\nabla^{\hat{1}}} \leftindex^{\textcolor{Maroon}{\mathsfit{z}}} \;\!
	{\mathcal{E}}^{\;\!\textcolor{Maroon}{(1)}\mathcolor{gray}{t}}_{\;\! \hat{2}\mathcolor{gray}{z}} + \leftindex_{\textcolor{Maroon}{\mathsfit{z}}} \;\! \delta'_{\mathcolor{gray}{z}} \mathcolor{gray}{\nabla^{\hat{1}}} \leftindex^{\textcolor{Maroon}{\mathsfit{z}}} \;\!
	{\mathcal{E}}^{\;\!\textcolor{Maroon}{(2)}\mathcolor{gray}{t}}_{\;\! \hat{2}\mathcolor{gray}{z}} \right) - \mathcolor{gray}{\nabla^t} \left( \leftindex_{\textcolor{Maroon}{\mathsfit{z}}} \;\! \delta_{\mathcolor{gray}{z}} \leftindex^{\textcolor{Maroon}{\mathsfit{z}}}
	{\mathcal{B}}^{\;\!\textcolor{Maroon}{(1)}\mathcolor{gray}{t}}_{\;\! \symup{\iota}\mathcolor{gray}{z}} + \leftindex_{\textcolor{Maroon}{\mathsfit{z}}} \;\! \delta'_{\mathcolor{gray}{z}} \leftindex^{\textcolor{Maroon}{\mathsfit{z}}} \;\! {\mathcal{B}}^{\;\!\textcolor{Maroon}{(2)}\mathcolor{gray}{t}}_{\;\! \symup{\iota}\mathcolor{gray}{z}} \right) \label{eq:curl-E-01} \\ &= \epsilon^{\hphantom{\symup{iz}}\hat{2}}_{\symup{\iota}\mathcolor{gray}{\symup{z}}} \leftindex_{\textcolor{Maroon}{\mathsfit{z}}} \;\! \delta''_{\mathcolor{gray}{z}} \leftindex^{\textcolor{Maroon}{\mathsfit{z}}} \;\!
	{\mathcal{E}}^{\;\!\textcolor{Maroon}{(2)}\mathcolor{gray}{t}}_{\;\! \hat{2}\mathcolor{gray}{z}}~, \\ \epsilon^{\hphantom{\symup{iz}}\hat{2}}_{\symup{\iota}\mathcolor{gray}{\symup{z}}} \left( \leftindex_{\textcolor{Maroon}{\mathsfit{z}}} \;\! \delta_{\mathcolor{gray}{z}} \leftindex^{\textcolor{Maroon}{\mathsfit{z}}} B^{\;\!\mathcolor{gray}{t}}_{\;\! \hat{2}\mathcolor{gray}{z}} - \leftindex_{\textcolor{Maroon}{\mathsfit{z}}} \;\! \delta'_{\mathcolor{gray}{z}} \leftindex^{\textcolor{Maroon}{\mathsfit{z}}} \;\!
	{\mathcal{B}}^{\;\!\textcolor{Maroon}{(1)}\mathcolor{gray}{t}}_{\;\! \hat{2}\mathcolor{gray}{z}} \right) &- \epsilon^{\hphantom{\symup{\iota}\hat{1}}\hat{2}}_{\symup{\iota}\mathcolor{gray}{\hat{1}}} \left( \leftindex_{\textcolor{Maroon}{\mathsfit{z}}} \;\! \delta_{\mathcolor{gray}{z}} \mathcolor{gray}{\nabla^{\hat{1}}} \leftindex^{\textcolor{Maroon}{\mathsfit{z}}} \;\!
	{\mathcal{B}}^{\;\!\textcolor{Maroon}{(1)}\mathcolor{gray}{t}}_{\;\! \hat{2}\mathcolor{gray}{z}} + \leftindex_{\textcolor{Maroon}{\mathsfit{z}}} \;\! \delta'_{\mathcolor{gray}{z}} \mathcolor{gray}{\nabla^{\hat{1}}} \leftindex^{\textcolor{Maroon}{\mathsfit{z}}} \;\!
	{\mathcal{B}}^{\;\!\textcolor{Maroon}{(2)}\mathcolor{gray}{t}}_{\;\! \hat{2}\mathcolor{gray}{z}} \right) + \mathcolor{gray}{\nabla^t} \left( \leftindex_{\textcolor{Maroon}{\mathsfit{z}}} \;\! \delta_{\mathcolor{gray}{z}} \leftindex^{\textcolor{Maroon}{\mathsfit{z}}}
	{\mathcal{E}}^{\;\!\textcolor{Maroon}{(1)}\mathcolor{gray}{t}}_{\;\! \symup{\iota}\mathcolor{gray}{z}} + \leftindex_{\textcolor{Maroon}{\mathsfit{z}}} \;\! \delta'_{\mathcolor{gray}{z}} \leftindex^{\textcolor{Maroon}{\mathsfit{z}}} \;\! {\mathcal{E}}^{\;\!\textcolor{Maroon}{(2)}\mathcolor{gray}{t}}_{\;\! \symup{\iota}\mathcolor{gray}{z}} \right) \big/ {\symup{c}}^2 \label{eq:curl-B-01} \\ &= - {\symup{\varepsilon}}_0 \left[ \leftindex_{\textcolor{Maroon}{\mathsfit{z}}} \;\! \delta_{\mathcolor{gray}{z}} \left( \leftindex^{\textcolor{Maroon}{\mathsfit{z}}}
	{\mathcal{K}}^{\;\!\mathcolor{gray}{t}}_{\;\! \symup{\iota}\symup{z}\mathcolor{gray}{z}} - \leftindex^{\textcolor{Maroon}{\mathsfit{z}}} \;\!
	{\alpha}^{\;\!\mathcolor{gray}{t}}_{\;\! \symup{\iota}\mathcolor{gray}{z}} \right) + \leftindex_{\textcolor{Maroon}{\mathsfit{z}}} \;\! \delta'_{\mathcolor{gray}{z}} \leftindex^{\textcolor{Maroon}{\mathsfit{z}}} \;\! {\mathcal{L}}^{\;\!\mathcolor{gray}{t}}_{\;\! \symup{\iota}\symup{z} \symup{z} \mathcolor{gray}{z}} \right] + \epsilon^{\hphantom{\symup{iz}}\hat{2}}_{\symup{\iota}\mathcolor{gray}{\symup{z}}} \leftindex_{\textcolor{Maroon}{\mathsfit{z}}} \;\! \delta''_{\mathcolor{gray}{z}} \leftindex^{\textcolor{Maroon}{\mathsfit{z}}} \;\!
	{\mathcal{B}}^{\;\!\textcolor{Maroon}{(2)}\mathcolor{gray}{t}}_{\;\! \hat{2}\mathcolor{gray}{z}} ~, \\
	\left( \leftindex_{\textcolor{Maroon}{\mathsfit{z}}} \;\! \delta_{\mathcolor{gray}{z}} \leftindex^{\textcolor{Maroon}{\mathsfit{z}}} \;\! E^{\;\!\mathcolor{gray}{t}}_{\;\! \symup{z} \mathcolor{gray}{z}} - \leftindex_{\textcolor{Maroon}{\mathsfit{z}}} \;\! \delta'_{\mathcolor{gray}{z}} \leftindex^{\textcolor{Maroon}{\mathsfit{z}}} \;\!
	{\mathcal{E}}^{\;\!\textcolor{Maroon}{(1)}\mathcolor{gray}{t}}_{\;\! \symup{z} \mathcolor{gray}{z}} \right. &- \left. \leftindex_{\textcolor{Maroon}{\mathsfit{z}}} \;\! \delta''_{\mathcolor{gray}{z}} \leftindex^{\textcolor{Maroon}{\mathsfit{z}}} \;\! {\mathcal{E}}^{\;\!\textcolor{Maroon}{(2)}\mathcolor{gray}{t}}_{\;\! \symup{z} \mathcolor{gray}{z}} \right) - \left( \leftindex_{\textcolor{Maroon}{\mathsfit{z}}} \;\! \delta_{\mathcolor{gray}{z}} \mathcolor{gray}{\nabla^\iota} \leftindex^{\textcolor{Maroon}{\mathsfit{z}}} \;\!
	{\mathcal{E}}^{\;\!\textcolor{Maroon}{(1)}\mathcolor{gray}{t}}_{\;\! \symup{\iota}\mathcolor{gray}{z}} + \leftindex_{\textcolor{Maroon}{\mathsfit{z}}} \;\! \delta'_{\mathcolor{gray}{z}} \mathcolor{gray}{\nabla^\iota} \leftindex^{\textcolor{Maroon}{\mathsfit{z}}} \;\!
	{\mathcal{E}}^{\;\!\textcolor{Maroon}{(2)}\mathcolor{gray}{t}}_{\;\! \symup{\iota}\mathcolor{gray}{z}} \right) \label{eq:div-E-01} \\ &= - {\symup{\varepsilon}}_0^{-1} \left[ \leftindex_{\textcolor{Maroon}{\mathsfit{z}}} \;\! \delta_{\mathcolor{gray}{z}} \left( \leftindex^{\textcolor{Maroon}{\mathsfit{z}}} \;\! {\mathcal{P}}^{\;\!\mathcolor{gray}{t}}_{\;\! \symup{z} \mathcolor{gray}{z}} - \leftindex^{\textcolor{Maroon}{\mathsfit{z}}} \;\! {\sigma}^{\;\!\mathcolor{gray}{t}}_{\;\! \mathcolor{gray}{z}} \right) + \leftindex_{\textcolor{Maroon}{\mathsfit{z}}} \;\! \delta'_{\mathcolor{gray}{z}} \leftindex^{\textcolor{Maroon}{\mathsfit{z}}} \;\! {\mathcal{Q}}^{\;\!\mathcolor{gray}{t}}_{\;\! \symup{z} \symup{z} \mathcolor{gray}{z}} + \leftindex_{\textcolor{Maroon}{\mathsfit{z}}} \;\! \delta''_{\mathcolor{gray}{z}} \leftindex^{\textcolor{Maroon}{\mathsfit{z}}} \;\! {\mathcal{O}}^{\;\!\mathcolor{gray}{t}}_{\;\! \symup{z} \symup{z} \symup{z} \mathcolor{gray}{z}} \right]~, \\
	\left( \leftindex_{\textcolor{Maroon}{\mathsfit{z}}} \;\! \delta_{\mathcolor{gray}{z}} \leftindex^{\textcolor{Maroon}{\mathsfit{z}}} \;\! B^{\;\!\mathcolor{gray}{t}}_{\;\! \symup{z} \mathcolor{gray}{z}} - \leftindex_{\textcolor{Maroon}{\mathsfit{z}}} \;\! \delta'_{\mathcolor{gray}{z}} \leftindex^{\textcolor{Maroon}{\mathsfit{z}}} \;\!
	{\mathcal{B}}^{\;\!\textcolor{Maroon}{(1)}\mathcolor{gray}{t}}_{\;\! \symup{z} \mathcolor{gray}{z}} \right. &- \left. \leftindex_{\textcolor{Maroon}{\mathsfit{z}}} \;\! \delta''_{\mathcolor{gray}{z}} \leftindex^{\textcolor{Maroon}{\mathsfit{z}}} \;\! {\mathcal{B}}^{\;\!\textcolor{Maroon}{(2)}\mathcolor{gray}{t}}_{\;\! \symup{z} \mathcolor{gray}{z}} \right) - \left( \leftindex_{\textcolor{Maroon}{\mathsfit{z}}} \;\! \delta_{\mathcolor{gray}{z}} \mathcolor{gray}{\nabla^\iota} \leftindex^{\textcolor{Maroon}{\mathsfit{z}}} \;\!
	{\mathcal{B}}^{\;\!\textcolor{Maroon}{(1)}\mathcolor{gray}{t}}_{\;\! \symup{\iota}\mathcolor{gray}{z}} + \leftindex_{\textcolor{Maroon}{\mathsfit{z}}} \;\! \delta'_{\mathcolor{gray}{z}} \mathcolor{gray}{\nabla^\iota} \leftindex^{\textcolor{Maroon}{\mathsfit{z}}} \;\!
	{\mathcal{B}}^{\;\!\textcolor{Maroon}{(2)}\mathcolor{gray}{t}}_{\;\! \symup{\iota}\mathcolor{gray}{z}} \right) \label{eq:div-B-01} \\ &= 0~. 
\end{align}
\end{subequations}
合并同层次奇异项(通过 \bref{eq:Intdeltasum=0}),得到 \bref{eq:div-e-b-01-delta-conclusions} 的对应版本
\begin{subequations} \label{eq:curl-EB-01-deltas}
\begin{align}
	{\delta}_{\mathcolor{gray}{z}} ~\textcolor{Maroon}{\text{项}}:&\hspace{1.0em}  \epsilon^{\hphantom{\symup{iz}}\hat{2}}_{\symup{\iota}\mathcolor{gray}{\symup{z}}} E^{\;\!\mathcolor{gray}{t}}_{\;\! \hat{2}\mathcolor{gray}{0}} - \epsilon^{\hphantom{\symup{\iota}\hat{1}}\hat{2}}_{\symup{\iota}\mathcolor{gray}{\hat{1}}} \mathcolor{gray}{\nabla^{\hat{1}}} 
	{\mathcal{E}}^{\;\!\textcolor{Maroon}{(1)}\mathcolor{gray}{t}}_{\;\! \hat{2}\mathcolor{gray}{0}} - \mathcolor{gray}{\nabla^t} 
	{\mathcal{B}}^{\;\!\textcolor{Maroon}{(1)}\mathcolor{gray}{t}}_{\;\! \symup{\iota}\mathcolor{gray}{0}} \hspace{-1.8em}&&=\hspace{0.2em} 0~, \hspace{-2.5em} &&\hspace{-1.3em} \label{eq:curl-EB-01-delta} \\
	&\hspace{1.0em} \epsilon^{\hphantom{\symup{iz}}\hat{2}}_{\symup{\iota}\mathcolor{gray}{\symup{z}}} B^{\;\!\mathcolor{gray}{t}}_{\;\! \hat{2}\mathcolor{gray}{0}} - \epsilon^{\hphantom{\symup{\iota}\hat{1}}\hat{2}}_{\symup{\iota}\mathcolor{gray}{\hat{1}}} \mathcolor{gray}{\nabla^{\hat{1}}} 
	{\mathcal{B}}^{\;\!\textcolor{Maroon}{(1)}\mathcolor{gray}{t}}_{\;\! \hat{2}\mathcolor{gray}{0}} + \mathcolor{gray}{\nabla^t} 
	{\mathcal{E}}^{\;\!\textcolor{Maroon}{(1)}\mathcolor{gray}{t}}_{\;\! \symup{\iota}\mathcolor{gray}{0}} \big/ {\symup{c}}^2 \hspace{-1.8em}&&=\hspace{0.2em} -\hspace{0.2em} {\symup{\varepsilon}}_0 \left( 
	{\mathcal{K}}^{\;\!\mathcolor{gray}{t}}_{\;\! \symup{\iota}\symup{z}\mathcolor{gray}{0}} \hspace{-2.5em}\right. &&\hspace{-1.5em}- \left. 
	{\alpha}^{\;\!\mathcolor{gray}{t}}_{\;\! \symup{\iota}\mathcolor{gray}{0}} \right)~, \label{eq:curl-EB-01-delta2} \\
	&\hspace{1.0em} \hphantom{\epsilon^{\hphantom{\symup{iz}}\hat{2}}_{\symup{\iota}\mathcolor{gray}{\symup{z}}}} E^{\;\!\mathcolor{gray}{t}}_{\;\! \symup{z} \mathcolor{gray}{0}} - \hphantom{\epsilon^{\hphantom{\symup{\iota}\hat{1}}\hat{2}}_{\symup{\iota}\mathcolor{gray}{\hat{1}}}} \mathcolor{gray}{\nabla^\iota} 
	{\mathcal{E}}^{\;\!\textcolor{Maroon}{(1)}\mathcolor{gray}{t}}_{\;\! \symup{\iota}\mathcolor{gray}{0}} \hspace{-1.8em}&&=\hspace{0.2em} -\hspace{0.2em} {\symup{\varepsilon}}_0^{-1} \left( {\mathcal{P}}^{\;\!\mathcolor{gray}{t}}_{\;\! \symup{z} \mathcolor{gray}{0}} \hspace{-2.5em}\right. &&\hspace{-1.5em}- \left. {\sigma}^{\;\!\mathcolor{gray}{t}}_{\;\! \mathcolor{gray}{0}} \right)~, \label{eq:curl-EB-01-delta3} \\ 
	&\hspace{1.0em} \hphantom{\epsilon^{\hphantom{\symup{iz}}\hat{2}}_{\symup{\iota}\mathcolor{gray}{\symup{z}}}} B^{\;\!\mathcolor{gray}{t}}_{\;\! \symup{z} \mathcolor{gray}{0}} - \hphantom{\epsilon^{\hphantom{\symup{\iota}\hat{1}}\hat{2}}_{\symup{\iota}\mathcolor{gray}{\hat{1}}}} \mathcolor{gray}{\nabla^\iota} 
	{\mathcal{B}}^{\;\!\textcolor{Maroon}{(1)}\mathcolor{gray}{t}}_{\;\! \symup{\iota}\mathcolor{gray}{0}} \hspace{-1.8em}&&=\hspace{0.2em} 0~, \hspace{-2.5em} &&\hspace{-1.3em} \label{eq:curl-EB-01-delta4} \\[0.7em]
	{\delta}'_{\mathcolor{gray}{z}} ~\textcolor{Maroon}{\text{项}}:&\hspace{1.0em}  \epsilon^{\hphantom{\symup{iz}}\hat{2}}_{\symup{\iota}\mathcolor{gray}{\symup{z}}} {\mathcal{E}}^{\;\!\textcolor{Maroon}{(1)}\mathcolor{gray}{t}}_{\;\! \hat{2}\mathcolor{gray}{0}} + \epsilon^{\hphantom{\symup{\iota}\hat{1}}\hat{2}}_{\symup{\iota}\mathcolor{gray}{\hat{1}}} \mathcolor{gray}{\nabla^{\hat{1}}} 
	{\mathcal{E}}^{\;\!\textcolor{Maroon}{(2)}\mathcolor{gray}{t}}_{\;\! \hat{2}\mathcolor{gray}{0}} + \mathcolor{gray}{\nabla^t} 
	{\mathcal{B}}^{\;\!\textcolor{Maroon}{(2)}\mathcolor{gray}{t}}_{\;\! \symup{\iota}\mathcolor{gray}{0}} \hspace{-1.8em}&&=\hspace{0.2em} 0~, \hspace{-2.5em} &&\hspace{-1.3em} \label{eq:curl-EB-01-delta'} \\
	&\hspace{1.0em} \epsilon^{\hphantom{\symup{iz}}\hat{2}}_{\symup{\iota}\mathcolor{gray}{\symup{z}}} {\mathcal{B}}^{\;\!\textcolor{Maroon}{(1)}\mathcolor{gray}{t}}_{\;\! \hat{2}\mathcolor{gray}{0}} + \epsilon^{\hphantom{\symup{\iota}\hat{1}}\hat{2}}_{\symup{\iota}\mathcolor{gray}{\hat{1}}} \mathcolor{gray}{\nabla^{\hat{1}}} 
	{\mathcal{B}}^{\;\!\textcolor{Maroon}{(2)}\mathcolor{gray}{t}}_{\;\! \hat{2}\mathcolor{gray}{0}} - \mathcolor{gray}{\nabla^t} 
	{\mathcal{E}}^{\;\!\textcolor{Maroon}{(2)}\mathcolor{gray}{t}}_{\;\! \symup{\iota}\mathcolor{gray}{0}} \big/ {\symup{c}}^2 \hspace{-1.8em}&&=\hspace{0.2em} {\symup{\varepsilon}}_0 {\mathcal{L}}^{\;\!\mathcolor{gray}{t}}_{\;\! \symup{\iota}\symup{z} \symup{z} \mathcolor{gray}{0}}~, \hspace{-2.5em} &&\hspace{-1.3em} \label{eq:curl-EB-01-delta'2} \\
	&\hspace{1.0em} \hphantom{\epsilon^{\hphantom{\symup{iz}}\hat{2}}_{\symup{\iota}\mathcolor{gray}{\symup{z}}}} {\mathcal{E}}^{\;\!\textcolor{Maroon}{(1)}\mathcolor{gray}{t}}_{\;\! \symup{z} \mathcolor{gray}{0}} + \hphantom{\epsilon^{\hphantom{\symup{\iota}\hat{1}}\hat{2}}_{\symup{\iota}\mathcolor{gray}{\hat{1}}}} \mathcolor{gray}{\nabla^\iota} 
	{\mathcal{E}}^{\;\!\textcolor{Maroon}{(2)}\mathcolor{gray}{t}}_{\;\! \symup{\iota}\mathcolor{gray}{0}} \hspace{-1.8em}&&=\hspace{0.2em} {\symup{\varepsilon}}_0^{-1} {\mathcal{Q}}^{\;\!\mathcolor{gray}{t}}_{\;\! \symup{z} \symup{z} \mathcolor{gray}{0}}~, \hspace{-2.5em} &&\hspace{-1.3em} \label{eq:curl-EB-01-delta'3} \\ 
	&\hspace{1.0em} \hphantom{\epsilon^{\hphantom{\symup{iz}}\hat{2}}_{\symup{\iota}\mathcolor{gray}{\symup{z}}}} {\mathcal{B}}^{\;\!\textcolor{Maroon}{(1)}\mathcolor{gray}{t}}_{\;\! \symup{z} \mathcolor{gray}{0}} + \hphantom{\epsilon^{\hphantom{\symup{\iota}\hat{1}}\hat{2}}_{\symup{\iota}\mathcolor{gray}{\hat{1}}}} \mathcolor{gray}{\nabla^\iota} 
	{\mathcal{B}}^{\;\!\textcolor{Maroon}{(2)}\mathcolor{gray}{t}}_{\;\! \symup{\iota}\mathcolor{gray}{0}} \hspace{-1.8em}&&=\hspace{0.2em} 0~, \hspace{-2.5em} &&\hspace{-1.3em} \label{eq:curl-EB-01-delta'4} \\[0.7em]
	{\delta}''_{\mathcolor{gray}{z}} ~\textcolor{Maroon}{\text{项}}:&\hspace{1.0em} \epsilon^{\hphantom{\symup{iz}}\hat{2}}_{\symup{\iota}\mathcolor{gray}{\symup{z}}} 
	{\mathcal{E}}^{\;\!\textcolor{Maroon}{(2)}\mathcolor{gray}{t}}_{\;\! \hat{2}\mathcolor{gray}{0}} \hspace{-1.8em}&&=\hspace{0.2em} 0~, &&\hspace{-1.3em} \label{eq:curl-EB-01-delta''} \\
	&\hspace{1.0em} \epsilon^{\hphantom{\symup{iz}}\hat{2}}_{\symup{\iota}\mathcolor{gray}{\symup{z}}} 
	{\mathcal{B}}^{\;\!\textcolor{Maroon}{(2)}\mathcolor{gray}{t}}_{\;\! \hat{2}\mathcolor{gray}{0}} \hspace{-1.8em}&&=\hspace{0.2em} 0~, &&\hspace{-1.3em} \label{eq:curl-EB-01-delta''2} \\
	&\hspace{1.0em} \hphantom{\epsilon^{\hphantom{\symup{iz}}\hat{2}}_{\symup{\iota}\mathcolor{gray}{\symup{z}}}} 
	{\mathcal{E}}^{\;\!\textcolor{Maroon}{(2)}\mathcolor{gray}{t}}_{\;\! \symup{z} \mathcolor{gray}{0}} \hspace{-1.8em}&&=\hspace{0.2em} {\symup{\varepsilon}}_0^{-1} {\mathcal{O}}^{\;\!\mathcolor{gray}{t}}_{\;\! \symup{z} \symup{z} \symup{z} \mathcolor{gray}{0}}~, &&\hspace{-1.3em} \label{eq:curl-EB-01-delta''3} \\
	&\hspace{1.0em} \hphantom{\epsilon^{\hphantom{\symup{iz}}\hat{2}}_{\symup{\iota}\mathcolor{gray}{\symup{z}}}} 
	{\mathcal{B}}^{\;\!\textcolor{Maroon}{(2)}\mathcolor{gray}{t}}_{\;\! \symup{z} \mathcolor{gray}{0}} \hspace{-1.8em}&&=\hspace{0.2em} 0~. &&\hspace{-1.3em} \label{eq:curl-EB-01-delta''4}
\end{align}
\end{subequations}
能满足 \bref{eq:curl-EB-01-delta''2,eq:curl-EB-01-delta''4} 的
$\bar{\mathcal{B}}^{\;\!\textcolor{Maroon}{(2)}\mathcolor{gray}{t}}_{\;\!  \mathcolor{gray}{0}} = \begin{pmatrix} 0, & 0, & 0 \end{pmatrix}^\top$,能满足 \bref{eq:curl-EB-01-delta'',eq:curl-EB-01-delta''3} 的 $\bar{\mathcal{E}}^{\;\!\textcolor{Maroon}{(2)}\mathcolor{gray}{t}}_{\;\!  \mathcolor{gray}{0}} = {\symup{\varepsilon}}_0^{-1} \begin{pmatrix} 0, & 0, & {\mathcal{O}}^{\;\!\mathcolor{gray}{t}}_{\;\! \symup{z} \symup{z} \symup{z} \mathcolor{gray}{0}} \end{pmatrix}^\top$;将 $\bar{\mathcal{B}}^{\;\!\textcolor{Maroon}{(2)}\mathcolor{gray}{t}}_{\;\!  \mathcolor{gray}{0}},\bar{\mathcal{E}}^{\;\!\textcolor{Maroon}{(2)}\mathcolor{gray}{t}}_{\;\! \mathcolor{gray}{0}}$ 代入 \bref{eq:curl-EB-01-delta'2,eq:curl-EB-01-delta'4} 可得 $\bar{\mathcal{B}}^{\;\!\textcolor{Maroon}{(1)}\mathcolor{gray}{t}}_{\;\!  \mathcolor{gray}{0}} = {\symup{\varepsilon}}_0 \begin{pmatrix} {\mathcal{L}}^{\;\!\mathcolor{gray}{t}}_{\;\! \symup{y} \symup{z} \symup{z} \mathcolor{gray}{0}}, & - {\mathcal{L}}^{\;\!\mathcolor{gray}{t}}_{\;\! \symup{x} \symup{z} \symup{z} \mathcolor{gray}{0}}, & 0 \end{pmatrix}^\top$,然后再代入 \bref{eq:curl-EB-01-delta',eq:curl-EB-01-delta'3},得到 $\bar{\mathcal{E}}^{\;\!\textcolor{Maroon}{(1)}\mathcolor{gray}{t}}_{\;\!  \mathcolor{gray}{0}} = {\symup{\varepsilon}}_0^{-1} \left( \mathcolor{gray}{\nabla_x} \mathcal{E}^{\;\!\textcolor{Maroon}{(2)}\mathcolor{gray}{t}}_{\;\! \symup{z} \mathcolor{gray}{0}}, \right.$ $\left. \mathcolor{gray}{\nabla_y} \mathcal{E}^{\;\!\textcolor{Maroon}{(2)}\mathcolor{gray}{t}}_{\;\! \symup{z} \mathcolor{gray}{0}},~~ {\mathcal{Q}}^{\;\!\mathcolor{gray}{t}}_{\;\! \symup{z} \symup{z} \mathcolor{gray}{0}} - \mathcolor{gray}{\nabla_z} \mathcal{E}^{\;\!\textcolor{Maroon}{(2)}\mathcolor{gray}{t}}_{\;\! \symup{z} \mathcolor{gray}{0}} \right)^\top$,即 $\bar{\mathcal{E}}^{\;\!\textcolor{Maroon}{(1)}\mathcolor{gray}{t}}_{\;\!  \mathcolor{gray}{0}} = {\symup{\varepsilon}}_0^{-1} \begin{pmatrix} \mathcolor{gray}{\nabla_x} {\mathcal{O}}^{\;\!\mathcolor{gray}{t}}_{\;\! \symup{z} \symup{z} \symup{z} \mathcolor{gray}{0}}, & \mathcolor{gray}{\nabla_y} {\mathcal{O}}^{\;\!\mathcolor{gray}{t}}_{\;\! \symup{z} \symup{z} \symup{z} \mathcolor{gray}{0}}, & {\mathcal{Q}}^{\;\!\mathcolor{gray}{t}}_{\;\! \symup{z} \symup{z} \mathcolor{gray}{0}} - \mathcolor{gray}{\nabla_z} {\mathcal{O}}^{\;\!\mathcolor{gray}{t}}_{\;\! \symup{z} \symup{z} \symup{z} \mathcolor{gray}{0}} \end{pmatrix}^\top$。

将上一段各非零量展开至 \bref{eq:multipole} 层次,即有
\begin{subequations} \label{eq:EB^(2-1)_0}
\begin{align}
	{\mathcal{E}}^{\;\!\textcolor{Maroon}{(2)}\mathcolor{gray}{t}}_{\;\! \symup{z} \mathcolor{gray}{0}} &= {\symup{\varepsilon}}_0^{-1} {\mathcal{O}}^{\;\!\mathcolor{gray}{t}}_{\;\!\textcolor{Maroon}{\text{b}}\symup{z} \symup{z} \symup{z} \mathcolor{gray}{0}}~; \label{eq:E^(2)_z0} \\[0.7em]
	{\mathcal{B}}^{\;\!\textcolor{Maroon}{(1)}\mathcolor{gray}{t}}_{\;\! \symup{x} \mathcolor{gray}{0}} &= {\symup{\varepsilon}}_0 \left( -\hspace{0.2em} \mathcolor{gray}{\nabla^t}
	{\mathcal{O}}^{\;\!\mathcolor{gray}{t}}_{\;\!\textcolor{Maroon}{\text{b}}\symup{y}\symup{z}\symup{z}\mathcolor{gray}{0}} + {\mathcal{L}}^{\;\!\mathcolor{gray}{t}}_{\;\!\textcolor{Maroon}{\text{m}}\symup{y}\symup{z}\symup{z}\mathcolor{gray}{0}} \right)~, \label{eq:B^(1)_x0} \\
	{\mathcal{B}}^{\;\!\textcolor{Maroon}{(1)}\mathcolor{gray}{t}}_{\;\! \symup{y} \mathcolor{gray}{0}} &= {\symup{\varepsilon}}_0 \left( +\hspace{0.2em} \mathcolor{gray}{\nabla^t}
	{\mathcal{O}}^{\;\!\mathcolor{gray}{t}}_{\;\!\textcolor{Maroon}{\text{b}}\symup{x}\symup{z}\symup{z}\mathcolor{gray}{0}} - {\mathcal{L}}^{\;\!\mathcolor{gray}{t}}_{\;\!\textcolor{Maroon}{\text{m}}\symup{x}\symup{z}\symup{z}\mathcolor{gray}{0}} \right)~; \label{eq:B^(1)_y0} \\[0.7em]
	{\mathcal{E}}^{\;\!\textcolor{Maroon}{(1)}\mathcolor{gray}{t}}_{\;\! \symup{x} \mathcolor{gray}{0}} &= {\symup{\varepsilon}}_0^{-1} \mathcolor{gray}{\nabla_x} {\mathcal{O}}^{\;\!\mathcolor{gray}{t}}_{\;\!\textcolor{Maroon}{\text{b}}\symup{z} \symup{z} \symup{z} \mathcolor{gray}{0}}~, \label{eq:E^(1)_x0} \\
	{\mathcal{E}}^{\;\!\textcolor{Maroon}{(1)}\mathcolor{gray}{t}}_{\;\! \symup{y} \mathcolor{gray}{0}} &= {\symup{\varepsilon}}_0^{-1} \mathcolor{gray}{\nabla_y} {\mathcal{O}}^{\;\!\mathcolor{gray}{t}}_{\;\!\textcolor{Maroon}{\text{b}}\symup{z} \symup{z} \symup{z} \mathcolor{gray}{0}}~, \label{eq:E^(1)_y0} \\
	{\mathcal{E}}^{\;\!\textcolor{Maroon}{(1)}\mathcolor{gray}{t}}_{\;\! \symup{z} \mathcolor{gray}{0}} &= {\symup{\varepsilon}}_0^{-1} \left( {\mathcal{Q}}^{\;\!\mathcolor{gray}{t}}_{\;\!\textcolor{Maroon}{\text{b}}\symup{z} \symup{z} \mathcolor{gray}{0}} + 2 \mathcolor{gray}{\nabla^{\hat{2}}} {\mathcal{O}}^{\;\!\mathcolor{gray}{t}}_{\;\!\textcolor{Maroon}{\text{b}}\symup{z} \symup{z} \hat{2} \mathcolor{gray}{0}} - \mathcolor{gray}{\nabla_z} {\mathcal{O}}^{\;\!\mathcolor{gray}{t}}_{\;\!\textcolor{Maroon}{\text{b}}\symup{z} \symup{z} \symup{z} \mathcolor{gray}{0}} \right)~, \label{eq:E^(1)_z0}
\end{align}
\end{subequations}

接着,展开 \bref{eq:curl-EB-01-delta2,eq:curl-EB-01-delta4},得到
\begin{subequations} \label{eq:B_0}
\begin{align}
	B^{\;\!\mathcolor{gray}{t}}_{\;\! \symup{x}\mathcolor{gray}{0}} &= \mathcolor{gray}{\nabla_z} 
	{\mathcal{B}}^{\;\!\textcolor{Maroon}{(1)}\mathcolor{gray}{t}}_{\;\! \symup{x} \mathcolor{gray}{0}} - \mathcolor{gray}{\nabla^t} 
	{\mathcal{E}}^{\;\!\textcolor{Maroon}{(1)}\mathcolor{gray}{t}}_{\;\! \symup{y} \mathcolor{gray}{0}} \big/ {\symup{c}}^2 - {\symup{\varepsilon}}_0 \left( 
	{\mathcal{K}}^{\;\!\mathcolor{gray}{t}}_{\;\! \symup{y}\symup{z}\mathcolor{gray}{0}} - 
	{\alpha}^{\;\!\mathcolor{gray}{t}}_{\;\! \symup{y}\mathcolor{gray}{0}} \right)~, \label{eq:B_x0} \\
	B^{\;\!\mathcolor{gray}{t}}_{\;\! \symup{y}\mathcolor{gray}{0}} &= \mathcolor{gray}{\nabla_z} 
	{\mathcal{B}}^{\;\!\textcolor{Maroon}{(1)}\mathcolor{gray}{t}}_{\;\! \symup{y} \mathcolor{gray}{0}} + \mathcolor{gray}{\nabla^t} 
	{\mathcal{E}}^{\;\!\textcolor{Maroon}{(1)}\mathcolor{gray}{t}}_{\;\! \symup{x} \mathcolor{gray}{0}} \big/ {\symup{c}}^2 + {\symup{\varepsilon}}_0 \left( 
	{\mathcal{K}}^{\;\!\mathcolor{gray}{t}}_{\;\! \symup{x}\symup{z}\mathcolor{gray}{0}} - 
	{\alpha}^{\;\!\mathcolor{gray}{t}}_{\;\! \symup{x}\mathcolor{gray}{0}} \right)~, \label{eq:B_y0} \\
	B^{\;\!\mathcolor{gray}{t}}_{\;\! \symup{z} \mathcolor{gray}{0}} &= \mathcolor{gray}{\nabla^\iota} 
	{\mathcal{B}}^{\;\!\textcolor{Maroon}{(1)}\mathcolor{gray}{t}}_{\;\! \symup{\iota}\mathcolor{gray}{0}} = {\symup{\varepsilon}}_0 \left( \mathcolor{gray}{\nabla_x} {\mathcal{L}}^{\;\!\mathcolor{gray}{t}}_{\;\! \symup{y}\symup{z}\symup{z}\mathcolor{gray}{0}} - \mathcolor{gray}{\nabla_y}
	{\mathcal{L}}^{\;\!\mathcolor{gray}{t}}_{\;\! \symup{x}\symup{z}\symup{z}\mathcolor{gray}{0}} \right)~, \label{eq:B_z0}
\end{align}
\end{subequations}
同样,展开 \bref{eq:curl-EB-01-delta,eq:curl-EB-01-delta3},得到
\begin{subequations} \label{eq:E_0}
\begin{align}
	E^{\;\!\mathcolor{gray}{t}}_{\;\! \symup{x}\mathcolor{gray}{0}} &= \mathcolor{gray}{\nabla_z}
	{\mathcal{E}}^{\;\!\textcolor{Maroon}{(1)}\mathcolor{gray}{t}}_{\;\! \symup{x} \mathcolor{gray}{0}} - \mathcolor{gray}{\nabla_x}
	{\mathcal{E}}^{\;\!\textcolor{Maroon}{(1)}\mathcolor{gray}{t}}_{\;\! \symup{z} \mathcolor{gray}{0}} + \mathcolor{gray}{\nabla^t}
	{\mathcal{B}}^{\;\!\textcolor{Maroon}{(1)}\mathcolor{gray}{t}}_{\;\! \symup{y} \mathcolor{gray}{0}}~, \label{eq:E_x0} \\
	E^{\;\!\mathcolor{gray}{t}}_{\;\! \symup{y}\mathcolor{gray}{0}} &= \mathcolor{gray}{\nabla_z}
	{\mathcal{E}}^{\;\!\textcolor{Maroon}{(1)}\mathcolor{gray}{t}}_{\;\! \symup{y} \mathcolor{gray}{0}} - \mathcolor{gray}{\nabla_y}
	{\mathcal{E}}^{\;\!\textcolor{Maroon}{(1)}\mathcolor{gray}{t}}_{\;\! \symup{z} \mathcolor{gray}{0}} - \mathcolor{gray}{\nabla^t}
	{\mathcal{B}}^{\;\!\textcolor{Maroon}{(1)}\mathcolor{gray}{t}}_{\;\! \symup{x} \mathcolor{gray}{0}}~, \label{eq:E_y0} \\
	E^{\;\!\mathcolor{gray}{t}}_{\;\! \symup{z} \mathcolor{gray}{0}} &= \mathcolor{gray}{\nabla^\iota} 
	{\mathcal{E}}^{\;\!\textcolor{Maroon}{(1)}\mathcolor{gray}{t}}_{\;\! \symup{\iota} \mathcolor{gray}{0}} - {\symup{\varepsilon}}_0^{-1} \left( {\mathcal{P}}^{\;\!\mathcolor{gray}{t}}_{\;\! \symup{z} \mathcolor{gray}{0}} - {\sigma}^{\;\!\mathcolor{gray}{t}}_{\;\! \mathcolor{gray}{0}} \right)~. \label{eq:E_z0}
\end{align}
\end{subequations}
%注意,如果表面自由电荷/流 $\leftindex^{\textcolor{Maroon}{\mathsfit{z}}} \;\! {\sigma},\leftindex^{\textcolor{Maroon}{\mathsfit{z}}} \;\! {\alpha}_{\;\! \symup{\iota}}$ 都可忽略,则所有体/表面场的三分量均与所处介质 $\textcolor{Maroon}{\mathsfit{z}}$(的种类/位置)无关;否则如果自由电荷不可忽略,则只有 $\leftindex^{\textcolor{Maroon}{\mathsfit{z}}} \;\! B^{\;\!\mathcolor{gray}{t}}_{\;\! \symup{x}\mathcolor{gray}{0}}, \leftindex^{\textcolor{Maroon}{\mathsfit{z}}} \;\! B^{\;\!\mathcolor{gray}{t}}_{\;\! \symup{y}\mathcolor{gray}{0}}; \leftindex^{\textcolor{Maroon}{\mathsfit{z}}} \;\! E^{\;\!\mathcolor{gray}{t}}_{\;\! \symup{z}\mathcolor{gray}{0}}$ 与所处介质 $\textcolor{Maroon}{\mathsfit{z}}$(的种类/位置)有关。

接着,将前述所得的 $\bar{\mathcal{B}}^{\;\!\textcolor{Maroon}{(1)}\mathcolor{gray}{t}}_{\;\!  \mathcolor{gray}{0}},\bar{\mathcal{E}}^{\;\!\textcolor{Maroon}{(1)}\mathcolor{gray}{t}}_{\;\! \mathcolor{gray}{0}}$ 三分量表达式代入 \bref{eq:B_0,eq:E_0},并像 \bref{eq:EB^(2-1)_0} 一样,展开至 \bref{eq:multipole} 层次,得 $\bar{B}^{\;\!\mathcolor{gray}{t}}_{\;\! \mathcolor{gray}{0}}$ 的三分量
\begin{subequations} \label{eq:B_0'}
\begin{align}
	B^{\;\!\mathcolor{gray}{t}}_{\;\! \symup{x}\mathcolor{gray}{0}} = &\hphantom{+} {\symup{\varepsilon}}_0 \left[ \mathcolor{gray}{\nabla^t} \left(  {\mathcal{Q}}^{\;\!\mathcolor{gray}{t}}_{\;\!\textcolor{Maroon}{\text{b}} \symup{y} \symup{z} \mathcolor{gray}{0}} + \mathcolor{gray}{\nabla_x}  {\mathcal{O}}^{\;\!\mathcolor{gray}{t}}_{\;\!\textcolor{Maroon}{\text{b}} \symup{y} \symup{z} \symup{x} \mathcolor{gray}{0}} + \mathcolor{gray}{\nabla_y}  {\mathcal{O}}^{\;\!\mathcolor{gray}{t}}_{\;\!\textcolor{Maroon}{\text{b}} \symup{y} \symup{z} \symup{y} \mathcolor{gray}{0}} - \mathcolor{gray}{\nabla_y}  {\mathcal{O}}^{\;\!\mathcolor{gray}{t}}_{\;\!\textcolor{Maroon}{\text{b}} \symup{z} \symup{z} \symup{z} \mathcolor{gray}{0}} \right) + 
	{\alpha}^{\;\!\mathcolor{gray}{t}}_{\;\! \symup{y}\mathcolor{gray}{0}} \right. \label{eq:B_x0'} \\ & \left. + M^{\;\!\mathcolor{gray}{t}}_{\;\! \symup{x}\mathcolor{gray}{0}} +
	\mathcolor{gray}{\nabla^{\hat{3}}} \left(  {\mathcal{L}}^{\;\!\mathcolor{gray}{t}}_{\;\!\textcolor{Maroon}{\text{m}}\hat{3} \symup{y} \symup{z} \mathcolor{gray}{0}} -  {\mathcal{L}}^{\;\!\mathcolor{gray}{t}}_{\;\!\textcolor{Maroon}{\text{m}}\symup{y} \symup{z} \hat{3} \mathcolor{gray}{0}} \right) + \mathcolor{gray}{\nabla_z}  {\mathcal{L}}^{\;\!\mathcolor{gray}{t}}_{\;\!\textcolor{Maroon}{\text{m}}\symup{y} \symup{z} \symup{z} \mathcolor{gray}{0}} \right]~, \\
	B^{\;\!\mathcolor{gray}{t}}_{\;\! \symup{y}\mathcolor{gray}{0}} = &- {\symup{\varepsilon}}_0 \left[ \mathcolor{gray}{\nabla^t} \left(  {\mathcal{Q}}^{\;\!\mathcolor{gray}{t}}_{\;\!\textcolor{Maroon}{\text{b}} \symup{x} \symup{z} \mathcolor{gray}{0}} + \mathcolor{gray}{\nabla_x}  {\mathcal{O}}^{\;\!\mathcolor{gray}{t}}_{\;\!\textcolor{Maroon}{\text{b}} \symup{x} \symup{z} \symup{x} \mathcolor{gray}{0}} + \mathcolor{gray}{\nabla_y}  {\mathcal{O}}^{\;\!\mathcolor{gray}{t}}_{\;\!\textcolor{Maroon}{\text{b}} \symup{x} \symup{z} \symup{y} \mathcolor{gray}{0}} - \mathcolor{gray}{\nabla_x}  {\mathcal{O}}^{\;\!\mathcolor{gray}{t}}_{\;\!\textcolor{Maroon}{\text{b}} \symup{z} \symup{z} \symup{z} \mathcolor{gray}{0}} \right) + 
	{\alpha}^{\;\!\mathcolor{gray}{t}}_{\;\! \symup{x}\mathcolor{gray}{0}} \right. \label{eq:B_y0'} \\ & \left. - M^{\;\!\mathcolor{gray}{t}}_{\;\! \symup{y}\mathcolor{gray}{0}} +
	\mathcolor{gray}{\nabla^{\hat{3}}} \left(  {\mathcal{L}}^{\;\!\mathcolor{gray}{t}}_{\;\!\textcolor{Maroon}{\text{m}}\hat{3} \symup{x} \symup{z} \mathcolor{gray}{0}} -  {\mathcal{L}}^{\;\!\mathcolor{gray}{t}}_{\;\!\textcolor{Maroon}{\text{m}}\symup{x} \symup{z} \hat{3} \mathcolor{gray}{0}} \right) + \mathcolor{gray}{\nabla_z}  {\mathcal{L}}^{\;\!\mathcolor{gray}{t}}_{\;\!\textcolor{Maroon}{\text{m}}\symup{x} \symup{z} \symup{z} \mathcolor{gray}{0}} \right]~, \\
	B^{\;\!\mathcolor{gray}{t}}_{\;\! \symup{z} \mathcolor{gray}{0}} = &\hphantom{+} {\symup{\varepsilon}}_0 \left[ \mathcolor{gray}{\nabla^t} \left( \mathcolor{gray}{\nabla_y}
	{\mathcal{O}}^{\;\!\mathcolor{gray}{t}}_{\;\!\textcolor{Maroon}{\text{b}}\symup{x}\symup{z}\symup{z}\mathcolor{gray}{0}} - \mathcolor{gray}{\nabla_x}
	{\mathcal{O}}^{\;\!\mathcolor{gray}{t}}_{\;\!\textcolor{Maroon}{\text{b}}\symup{y}\symup{z}\symup{z}\mathcolor{gray}{0}} \right) + \left( \mathcolor{gray}{\nabla_x} {\mathcal{L}}^{\;\!\mathcolor{gray}{t}}_{\;\!\textcolor{Maroon}{\text{m}}\symup{y}\symup{z}\symup{z}\mathcolor{gray}{0}} - \mathcolor{gray}{\nabla_y}
	{\mathcal{L}}^{\;\!\mathcolor{gray}{t}}_{\;\!\textcolor{Maroon}{\text{m}}\symup{x}\symup{z}\symup{z}\mathcolor{gray}{0}} \right) \right]~, \label{eq:B_z0'}
\end{align}
\end{subequations}
以及 $\bar{E}^{\;\!\mathcolor{gray}{t}}_{\;\! \mathcolor{gray}{0}}$ 的三分量
\begin{subequations} \label{eq:E_0'}
\begin{align}
	E^{\;\!\mathcolor{gray}{t}}_{\;\! \symup{x}\mathcolor{gray}{0}} = &- {\symup{\varepsilon}}_0^{-1} \mathcolor{gray}{\nabla_x} \left[ {\mathcal{Q}}^{\;\!\mathcolor{gray}{t}}_{\;\!\textcolor{Maroon}{\text{b}}\symup{z} \symup{z} \mathcolor{gray}{0}} + 2 \left( \mathcolor{gray}{\nabla_x} {\mathcal{O}}^{\;\!\mathcolor{gray}{t}}_{\;\!\textcolor{Maroon}{\text{b}}\symup{z} \symup{z} \symup{x} \mathcolor{gray}{0}} + \mathcolor{gray}{\nabla_y}  {\mathcal{O}}^{\;\!\mathcolor{gray}{t}}_{\;\!\textcolor{Maroon}{\text{b}}\symup{z} \symup{z} \symup{y} \mathcolor{gray}{0}} \right) \right] \label{eq:E_x0'} \\ &+ {\symup{\varepsilon}}_0 \mathcolor{gray}{\nabla^t} \left( \mathcolor{gray}{\nabla^t} {\mathcal{O}}^{\;\!\mathcolor{gray}{t}}_{\;\!\textcolor{Maroon}{\text{b}}\symup{x} \symup{z} \symup{z} \mathcolor{gray}{0}} - {\mathcal{L}}^{\;\!\mathcolor{gray}{t}}_{\;\!\textcolor{Maroon}{\text{m}}\symup{x} \symup{z} \symup{z} \mathcolor{gray}{0}} \right) ~, \\ E^{\;\!\mathcolor{gray}{t}}_{\;\! \symup{y}\mathcolor{gray}{0}} = &- {\symup{\varepsilon}}_0^{-1} \mathcolor{gray}{\nabla_y} \left[ {\mathcal{Q}}^{\;\!\mathcolor{gray}{t}}_{\;\!\textcolor{Maroon}{\text{b}}\symup{z} \symup{z} \mathcolor{gray}{0}} + 2 \left( \mathcolor{gray}{\nabla_x} {\mathcal{O}}^{\;\!\mathcolor{gray}{t}}_{\;\!\textcolor{Maroon}{\text{b}}\symup{z} \symup{z} \symup{x} \mathcolor{gray}{0}} + \mathcolor{gray}{\nabla_y}  {\mathcal{O}}^{\;\!\mathcolor{gray}{t}}_{\;\!\textcolor{Maroon}{\text{b}}\symup{z} \symup{z} \symup{y} \mathcolor{gray}{0}} \right) \right] \label{eq:E_y0'} \\ &+ {\symup{\varepsilon}}_0 \mathcolor{gray}{\nabla^t} \left( \mathcolor{gray}{\nabla^t} {\mathcal{O}}^{\;\!\mathcolor{gray}{t}}_{\;\!\textcolor{Maroon}{\text{b}}\symup{y} \symup{z} \symup{z} \mathcolor{gray}{0}} - {\mathcal{L}}^{\;\!\mathcolor{gray}{t}}_{\;\!\textcolor{Maroon}{\text{m}}\symup{y} \symup{z} \symup{z} \mathcolor{gray}{0}} \right) ~, \\
	E^{\;\!\mathcolor{gray}{t}}_{\;\! \symup{z}\mathcolor{gray}{0}} = &- {\symup{\varepsilon}}_0^{-1} \left[ {\mathcal{P}}^{\;\!\mathcolor{gray}{t}}_{\;\!\textcolor{Maroon}{\text{b}} \symup{z} \mathcolor{gray}{0}} + \left( \mathcolor{gray}{\nabla_x} {\mathcal{Q}}^{\;\!\mathcolor{gray}{t}}_{\;\!\textcolor{Maroon}{\text{b}} \symup{z} \symup{x} \mathcolor{gray}{0}} + \mathcolor{gray}{\nabla_y} {\mathcal{Q}}^{\;\!\mathcolor{gray}{t}}_{\;\!\textcolor{Maroon}{\text{b}} \symup{z} \symup{y} \mathcolor{gray}{0}} \right) - {\sigma}^{\;\!\mathcolor{gray}{t}}_{\;\! \mathcolor{gray}{0}} \right. \label{eq:E_z0'} \\ & \left. +\hspace{0.2em} 2 \mathcolor{gray}{\nabla_x} \mathcolor{gray}{\nabla_y} {\mathcal{O}}^{\;\!\mathcolor{gray}{t}}_{\;\!\textcolor{Maroon}{\text{b}} \symup{x} \symup{y} \symup{z} \mathcolor{gray}{0}} + \left( \mathcolor{gray}{\nabla_x^2} {\mathcal{O}}^{\;\!\mathcolor{gray}{t}}_{\;\!\textcolor{Maroon}{\text{b}} \symup{z} \symup{x} \symup{x} \mathcolor{gray}{0}} + \mathcolor{gray}{\nabla_y^2} {\mathcal{O}}^{\;\!\mathcolor{gray}{t}}_{\;\!\textcolor{Maroon}{\text{b}} \symup{z} \symup{y} \symup{y} \mathcolor{gray}{0}} \right) - \left( \mathcolor{gray}{\nabla_x^2} + \mathcolor{gray}{\nabla_y^2} \right) {\mathcal{O}}^{\;\!\mathcolor{gray}{t}}_{\;\!\textcolor{Maroon}{\text{b}} \symup{z} \symup{z} \symup{z} \mathcolor{gray}{0}} \right] ~.
\end{align}
\end{subequations}

现像 \bref{ssec:PMQN,ssec:DH-boundary} 一样,将 \bref{eq:EB^(2-1)_0,eq:B_0',eq:E_0'} 再展开至 \bref{eq:multipole} 的下一层次,直至各项全是裸多极矩 $\bar{P}^{\;\!\mathcolor{gray}{t}}_{\;\!\mathcolor{gray}{z}},\bar{\bar{Q}}^{\;\!\mathcolor{gray}{t}}_{\;\!\mathcolor{gray}{z}},\bar{\bar{\bar{O}}}^{\;\!\mathcolor{gray}{t}}_{\;\!\mathcolor{gray}{z}} ; \bar{M}^{\;\!\mathcolor{gray}{t}}_{\;\!\mathcolor{gray}{z}}, \bar{\bar{N}}^{\;\!\mathcolor{gray}{t}}_{\;\!\mathcolor{gray}{z}}$ 形式\Footnote{$B^{\;\!\mathcolor{gray}{t}}_{\;\! \symup{x}\mathcolor{gray}{0}},B^{\;\!\mathcolor{gray}{t}}_{\;\! \symup{y}\mathcolor{gray}{0}}$ 中的 $\mathcolor{gray}{\nabla_z}  {\mathcal{L}}^{\;\!\mathcolor{gray}{t}}_{\;\!\textcolor{Maroon}{\text{m}}\symup{y} \symup{z} \symup{z} \mathcolor{gray}{0}},\mathcolor{gray}{\nabla_z}  {\mathcal{L}}^{\;\!\mathcolor{gray}{t}}_{\;\!\textcolor{Maroon}{\text{m}}\symup{x} \symup{z} \symup{z} \mathcolor{gray}{0}}$ 总会被 $\mathcolor{gray}{\nabla^{\hat{3}}}  {\mathcal{L}}^{\;\!\mathcolor{gray}{t}}_{\;\!\textcolor{Maroon}{\text{m}}\hat{3} \symup{y} \symup{z} \mathcolor{gray}{0}},\mathcolor{gray}{\nabla^{\hat{3}}}  {\mathcal{L}}^{\;\!\mathcolor{gray}{t}}_{\;\!\textcolor{Maroon}{\text{m}}\hat{3} \symup{x} \symup{z} \mathcolor{gray}{0}}$ 中的某项抵消;$E^{\;\!\mathcolor{gray}{t}}_{\;\! \symup{z}\mathcolor{gray}{0}}$ 中的 ${\mathcal{P}}^{\;\!\mathcolor{gray}{t}}_{\;\!\textcolor{Maroon}{\text{b}} \symup{z} \mathcolor{gray}{0}}$ 会分裂出很多项,并分别与其后续的所有子项合并同类项,且产生新项。在得到 \bref{eq:EB^(2-0)_0=} 的过程中,已经使用到了多极矩的各分量(脚标)的置换对称性。}
\begin{subequations} \label{eq:EB^(2-0)_0=}
%	\belowdisplayskip=15pt
\begin{align}
	{\mathcal{E}}^{\;\!\textcolor{Maroon}{(2)}\mathcolor{gray}{t}}_{\;\! \symup{z} \mathcolor{gray}{0}} = &\hphantom{+} {\symup{\varepsilon}}_0^{-1} O^{\;\!\mathcolor{gray}{t}}_{\;\! \symup{z}\symup{z}\symup{z}\mathcolor{gray}{0}}~; \label{eq:E^(2)_z0=} \\[1.5em]
	{\mathcal{B}}^{\;\!\textcolor{Maroon}{(1)}\mathcolor{gray}{t}}_{\;\! \symup{x} \mathcolor{gray}{0}} = &- {\symup{\varepsilon}}_0 \left( \mathcolor{gray}{\nabla^t}
	O^{\;\!\mathcolor{gray}{t}}_{\;\!\symup{y}\symup{z}\symup{z}\mathcolor{gray}{0}} - N^{\;\!\mathcolor{gray}{t}}_{\;\!\symup{x}\symup{z}\mathcolor{gray}{0}} \right)~, \label{eq:B^(1)_x0=} \\
	{\mathcal{B}}^{\;\!\textcolor{Maroon}{(1)}\mathcolor{gray}{t}}_{\;\! \symup{y} \mathcolor{gray}{0}} = &\hphantom{+} {\symup{\varepsilon}}_0 \left( \mathcolor{gray}{\nabla^t}
	O^{\;\!\mathcolor{gray}{t}}_{\;\!\symup{x}\symup{z}\symup{z}\mathcolor{gray}{0}} + N^{\;\!\mathcolor{gray}{t}}_{\;\!\symup{y}\symup{z}\mathcolor{gray}{0}} \right)~; \label{eq:B^(1)_y0=} \\[1.5em]
	{\mathcal{E}}^{\;\!\textcolor{Maroon}{(1)}\mathcolor{gray}{t}}_{\;\! \symup{x} \mathcolor{gray}{0}} = &\hphantom{+} {\symup{\varepsilon}}_0^{-1} \mathcolor{gray}{\nabla_x} O^{\;\!\mathcolor{gray}{t}}_{\;\! \symup{z}\symup{z}\symup{z}\mathcolor{gray}{0}}~, \label{eq:E^(1)_x0=} \\
	{\mathcal{E}}^{\;\!\textcolor{Maroon}{(1)}\mathcolor{gray}{t}}_{\;\! \symup{y} \mathcolor{gray}{0}} = &\hphantom{+} {\symup{\varepsilon}}_0^{-1} \mathcolor{gray}{\nabla_y} O^{\;\!\mathcolor{gray}{t}}_{\;\! \symup{z}\symup{z}\symup{z}\mathcolor{gray}{0}}~, \label{eq:E^(1)_y0=} \\
	{\mathcal{E}}^{\;\!\textcolor{Maroon}{(1)}\mathcolor{gray}{t}}_{\;\! \symup{z} \mathcolor{gray}{0}} = &- {\symup{\varepsilon}}_0^{-1} \left( Q^{\;\!\mathcolor{gray}{t}}_{\;\! \symup{z} \symup{z} \mathcolor{gray}{0}} - 3~ \mathcolor{gray}{\nabla^{\hat{2}}} O^{\;\!\mathcolor{gray}{t}}_{\;\! \symup{z} \symup{z} \hat{2} \mathcolor{gray}{0}} + \mathcolor{gray}{\nabla_z} O^{\;\!\mathcolor{gray}{t}}_{\;\! \symup{z} \symup{z} \symup{z} \mathcolor{gray}{0}} \right)~; \label{eq:E^(1)_z0=} \\[1.5em]
	B^{\;\!\mathcolor{gray}{t}}_{\;\! \symup{x}\mathcolor{gray}{0}} = &\hphantom{+} {\symup{\varepsilon}}_0 \left[ \mathcolor{gray}{\nabla^t} \left( 2~ \mathcolor{gray}{\nabla_x} O^{\;\!\mathcolor{gray}{t}}_{\;\! \symup{y} \symup{z} \symup{x} \mathcolor{gray}{0}} + 2~ \mathcolor{gray}{\nabla_y} O^{\;\!\mathcolor{gray}{t}}_{\;\! \symup{y} \symup{z} \symup{y} \mathcolor{gray}{0}} + \mathcolor{gray}{\nabla_z} O^{\;\!\mathcolor{gray}{t}}_{\;\! \symup{y} \symup{z} \symup{z} \mathcolor{gray}{0}} - \mathcolor{gray}{\nabla_y} O^{\;\!\mathcolor{gray}{t}}_{\;\! \symup{z} \symup{z} \symup{z} \mathcolor{gray}{0}} \right) \right. \label{eq:B_x0=} \\ & \left. + \left( M^{\;\!\mathcolor{gray}{t}}_{\;\! \symup{x}\mathcolor{gray}{0}} - \mathcolor{gray}{\nabla^t} Q^{\;\!\mathcolor{gray}{t}}_{\;\! \symup{y} \symup{z} \mathcolor{gray}{0}} \right) +
	\left( \mathcolor{gray}{\nabla_x} N^{\;\!\mathcolor{gray}{t}}_{\;\!\symup{z} \symup{z} \mathcolor{gray}{0}} - \mathcolor{gray}{\nabla^{\hat{3}}} N^{\;\!\mathcolor{gray}{t}}_{\;\!\symup{x} \hat{3} \mathcolor{gray}{0}} \right) + 
	{\alpha}^{\;\!\mathcolor{gray}{t}}_{\;\! \symup{y}\mathcolor{gray}{0}} \right]~, \\
	B^{\;\!\mathcolor{gray}{t}}_{\;\! \symup{y}\mathcolor{gray}{0}} = &\hphantom{+} {\symup{\varepsilon}}_0 \left[ - \mathcolor{gray}{\nabla^t} \left( 2~ \mathcolor{gray}{\nabla_x} O^{\;\!\mathcolor{gray}{t}}_{\;\! \symup{x} \symup{z} \symup{x} \mathcolor{gray}{0}} + 2~ \mathcolor{gray}{\nabla_y}  O^{\;\!\mathcolor{gray}{t}}_{\;\! \symup{x} \symup{z} \symup{y} \mathcolor{gray}{0}} + \mathcolor{gray}{\nabla_z}  O^{\;\!\mathcolor{gray}{t}}_{\;\! \symup{x} \symup{z} \symup{z} \mathcolor{gray}{0}} - \mathcolor{gray}{\nabla_x}  O^{\;\!\mathcolor{gray}{t}}_{\;\! \symup{z} \symup{z} \symup{z} \mathcolor{gray}{0}} \right) \right. \label{eq:B_y0=} \\ & \left. + \left( M^{\;\!\mathcolor{gray}{t}}_{\;\! \symup{y}\mathcolor{gray}{0}} + \mathcolor{gray}{\nabla^t} Q^{\;\!\mathcolor{gray}{t}}_{\;\! \symup{x} \symup{z} \mathcolor{gray}{0}} \right) +
	\left( \mathcolor{gray}{\nabla_y} N^{\;\!\mathcolor{gray}{t}}_{\;\! \symup{z} \symup{z} \mathcolor{gray}{0}} - \mathcolor{gray}{\nabla^{\hat{3}}} N^{\;\!\mathcolor{gray}{t}}_{\;\! \symup{y} \hat{3} \mathcolor{gray}{0}} \right) - 
	{\alpha}^{\;\!\mathcolor{gray}{t}}_{\;\! \symup{x}\mathcolor{gray}{0}} \right]~, \\
	B^{\;\!\mathcolor{gray}{t}}_{\;\! \symup{z} \mathcolor{gray}{0}} = &\hphantom{+} {\symup{\varepsilon}}_0 \left[ \mathcolor{gray}{\nabla^t} \left( \mathcolor{gray}{\nabla_y}
	O^{\;\!\mathcolor{gray}{t}}_{\;\! \symup{x}\symup{z}\symup{z} \mathcolor{gray}{0}} - \mathcolor{gray}{\nabla_x}
	O^{\;\!\mathcolor{gray}{t}}_{\;\! \symup{y}\symup{z}\symup{z}\mathcolor{gray}{0}} \right) + \left( \mathcolor{gray}{\nabla_x}
	N^{\;\!\mathcolor{gray}{t}}_{\;\! \symup{x}\symup{z} \mathcolor{gray}{0}} + \mathcolor{gray}{\nabla_y} N^{\;\!\mathcolor{gray}{t}}_{\;\! \symup{y}\symup{z} \mathcolor{gray}{0}} \right) \right]~; \label{eq:B_z0=} \\[1.7em]
	E^{\;\!\mathcolor{gray}{t}}_{\;\! \symup{x}\mathcolor{gray}{0}} = &\hphantom{+} {\symup{\varepsilon}}_0^{-1} \mathcolor{gray}{\nabla_x} \left[ Q^{\;\!\mathcolor{gray}{t}}_{\;\! \symup{z} \symup{z} \mathcolor{gray}{0}} - 3~ \mathcolor{gray}{\nabla_x} O^{\;\!\mathcolor{gray}{t}}_{\;\! \symup{z} \symup{z} \symup{x} \mathcolor{gray}{0}} - 3~ \mathcolor{gray}{\nabla_y}  O^{\;\!\mathcolor{gray}{t}}_{\;\! \symup{z} \symup{z} \symup{y} \mathcolor{gray}{0}} - \mathcolor{gray}{\nabla_z}  O^{\;\!\mathcolor{gray}{t}}_{\;\! \symup{z} \symup{z} \symup{z} \mathcolor{gray}{0}} \right] \label{eq:E_x0=} \\ &+ {\symup{\varepsilon}}_0 \mathcolor{gray}{\nabla^t} \left( \mathcolor{gray}{\nabla^t} O^{\;\!\mathcolor{gray}{t}}_{\;\!\symup{x} \symup{z} \symup{z} \mathcolor{gray}{0}} + N^{\;\!\mathcolor{gray}{t}}_{\;\! \symup{y} \symup{z} \mathcolor{gray}{0}} \right) ~, \\ E^{\;\!\mathcolor{gray}{t}}_{\;\! \symup{y}\mathcolor{gray}{0}} = &\hphantom{+} {\symup{\varepsilon}}_0^{-1} \mathcolor{gray}{\nabla_y} \left[ Q^{\;\!\mathcolor{gray}{t}}_{\;\! \symup{z} \symup{z} \mathcolor{gray}{0}} - 3~ \mathcolor{gray}{\nabla_x} O^{\;\!\mathcolor{gray}{t}}_{\;\! \symup{z} \symup{z} \symup{x} \mathcolor{gray}{0}} - 3~ \mathcolor{gray}{\nabla_y}  O^{\;\!\mathcolor{gray}{t}}_{\;\! \symup{z} \symup{z} \symup{y} \mathcolor{gray}{0}} - \mathcolor{gray}{\nabla_z}  O^{\;\!\mathcolor{gray}{t}}_{\;\! \symup{z} \symup{z} \symup{z} \mathcolor{gray}{0}} \right] \label{eq:E_y0=} \\ &+ {\symup{\varepsilon}}_0 \mathcolor{gray}{\nabla^t} \left( \mathcolor{gray}{\nabla^t} O^{\;\!\mathcolor{gray}{t}}_{\;\! \symup{y} \symup{z} \symup{z} \mathcolor{gray}{0}} - N^{\;\!\mathcolor{gray}{t}}_{\;\! \symup{x} \symup{z} \mathcolor{gray}{0}} \right) ~, \\
	E^{\;\!\mathcolor{gray}{t}}_{\;\! \symup{z}\mathcolor{gray}{0}} = &- {\symup{\varepsilon}}_0^{-1} \left[ P^{\;\!\mathcolor{gray}{t}}_{\;\! \symup{z} \mathcolor{gray}{0}} - 2~ \mathcolor{gray}{\nabla_x} Q^{\;\!\mathcolor{gray}{t}}_{\;\! \symup{z} \symup{x} \mathcolor{gray}{0}} - 2~ \mathcolor{gray}{\nabla_y} Q^{\;\!\mathcolor{gray}{t}}_{\;\! \symup{z} \symup{y} \mathcolor{gray}{0}} - \mathcolor{gray}{\nabla_z} Q^{\;\!\mathcolor{gray}{t}}_{\;\! \symup{z} \symup{z} \mathcolor{gray}{0}} - {\sigma}^{\;\!\mathcolor{gray}{t}}_{\;\! \mathcolor{gray}{0}} \right. \label{eq:E_z0=} \\ & + 3~ \mathcolor{gray}{\nabla_z} \left( \mathcolor{gray}{\nabla_x} O^{\;\!\mathcolor{gray}{t}}_{\;\! \symup{z} \symup{z} \symup{x} \mathcolor{gray}{0}} + \mathcolor{gray}{\nabla_y} O^{\;\!\mathcolor{gray}{t}}_{\;\! \symup{z} \symup{z} \symup{y} \mathcolor{gray}{0}} \right) + 6~ \mathcolor{gray}{\nabla_x} \mathcolor{gray}{\nabla_y} O^{\;\!\mathcolor{gray}{t}}_{\;\! \symup{x} \symup{y} \symup{z} \mathcolor{gray}{0}} \\ & \left. + 3 \left( \mathcolor{gray}{\nabla_x^2} O^{\;\!\mathcolor{gray}{t}}_{\;\! \symup{z} \symup{x} \symup{x} \mathcolor{gray}{0}} + \mathcolor{gray}{\nabla_y^2} O^{\;\!\mathcolor{gray}{t}}_{\;\! \symup{z} \symup{y} \symup{y} \mathcolor{gray}{0}} \right) + \left( \mathcolor{gray}{\nabla_z^2} - \mathcolor{gray}{\nabla_x^2} - \mathcolor{gray}{\nabla_y^2} \right) O^{\;\!\mathcolor{gray}{t}}_{\;\! \symup{z} \symup{z} \symup{z} \mathcolor{gray}{0}} \right] ~.
\end{align}
\end{subequations}
上述 \bref{eq:EB^(2-0)_0=} 与文献\cite{delangeElectromagneticBoundaryConditions2013}的主要结果的绝大部分一致,除了 \bref{eq:E^(2)_z0=,eq:B^(1)_x0=,eq:B^(1)_y0=,eq:E^(1)_x0=,eq:E^(1)_y0=} 反号。这些分量异号的原因,主要应归结为:本文对 \bref{eq:e-b-01'} 的定义与文献\cite{delangeElectromagneticBoundaryConditions2013}相同,但本文对 \bref{eq:e-b-01',eq:e-f-01',eq:EB-01} 的定义却均与文献\cite{delangeElectromagneticBoundaryConditions2013}不同。此外,这里暂不涉及 2 种介质,而是单侧半无限介质,另一侧真空。

再利用多极矩 $\bar{P}^{\;\!\mathcolor{gray}{t}}_{\;\!\mathcolor{gray}{z}},\bar{\bar{Q}}^{\;\!\mathcolor{gray}{t}}_{\;\!\mathcolor{gray}{z}},\bar{\bar{\bar{O}}}^{\;\!\mathcolor{gray}{t}}_{\;\!\mathcolor{gray}{z}} ; \bar{M}^{\;\!\mathcolor{gray}{t}}_{\;\!\mathcolor{gray}{z}}, \bar{\bar{N}}^{\;\!\mathcolor{gray}{t}}_{\;\!\mathcolor{gray}{z}}$ 的(脚标中)各分量的置换对称性,将上述 \bref{eq:EB^(2-0)_0=} 各张量的分量以从左到右 xyz 排列,且重排各项至电磁场匹配(电/磁的首项总对应电/磁项),并将整体分组为电磁两组。

对于电场,其(在宏观阶跃边界 $\mathcolor{gray}{z \mathcolor{black}{=} 0}$ 附近的)非零分量包含
\begin{subequations} \label{eq:E^(2-0)_0==}
\begin{align}
	{\mathcal{E}}^{\;\!\textcolor{Maroon}{(2)}\mathcolor{gray}{t}}_{\;\! \symup{z} \mathcolor{gray}{0}} = &\hphantom{+} {\symup{\varepsilon}}_0^{-1} O^{\;\!\mathcolor{gray}{t}}_{\;\! \symup{z}\symup{z}\symup{z}\mathcolor{gray}{0}}~; \label{eq:E^(2)_z0==} \\[0.7em]
	{\mathcal{E}}^{\;\!\textcolor{Maroon}{(1)}\mathcolor{gray}{t}}_{\;\! \symup{x} \mathcolor{gray}{0}} = &\hphantom{+} {\symup{\varepsilon}}_0^{-1} \mathcolor{gray}{\nabla_x} O^{\;\!\mathcolor{gray}{t}}_{\;\! \symup{z}\symup{z}\symup{z}\mathcolor{gray}{0}}~, \label{eq:E^(1)_x0==} \\
	{\mathcal{E}}^{\;\!\textcolor{Maroon}{(1)}\mathcolor{gray}{t}}_{\;\! \symup{y} \mathcolor{gray}{0}} = &\hphantom{+} {\symup{\varepsilon}}_0^{-1} \mathcolor{gray}{\nabla_y} O^{\;\!\mathcolor{gray}{t}}_{\;\! \symup{z}\symup{z}\symup{z}\mathcolor{gray}{0}}~, \label{eq:E^(1)_y0==} \\
	{\mathcal{E}}^{\;\!\textcolor{Maroon}{(1)}\mathcolor{gray}{t}}_{\;\! \symup{z} \mathcolor{gray}{0}} = &- {\symup{\varepsilon}}_0^{-1} \left( Q^{\;\!\mathcolor{gray}{t}}_{\;\! \symup{z} \symup{z} \mathcolor{gray}{0}} - 3 \mathcolor{gray}{\nabla^\iota} O^{\;\!\mathcolor{gray}{t}}_{\;\! \symup{\iota} \symup{z} \symup{z} \mathcolor{gray}{0}} + \mathcolor{gray}{\nabla_z} O^{\;\!\mathcolor{gray}{t}}_{\;\! \symup{z} \symup{z} \symup{z} \mathcolor{gray}{0}} \right)~; \label{eq:E^(1)_z0==} \\[0.7em]
	E^{\;\!\mathcolor{gray}{t}}_{\;\! \symup{x}\mathcolor{gray}{0}} = &\hphantom{+} {\symup{\varepsilon}}_0^{-1} \mathcolor{gray}{\nabla_x} \left[ Q^{\;\!\mathcolor{gray}{t}}_{\;\! \symup{z} \symup{z} \mathcolor{gray}{0}} - 3~ \mathcolor{gray}{\nabla_x} O^{\;\!\mathcolor{gray}{t}}_{\;\! \symup{x} \symup{z} \symup{z} \mathcolor{gray}{0}} - 3~ \mathcolor{gray}{\nabla_y}  O^{\;\!\mathcolor{gray}{t}}_{\;\! \symup{y} \symup{z} \symup{z} \mathcolor{gray}{0}} - \mathcolor{gray}{\nabla_z}  O^{\;\!\mathcolor{gray}{t}}_{\;\! \symup{z} \symup{z} \symup{z} \mathcolor{gray}{0}} \right] \label{eq:E_x0==} \\ &+ {\symup{\varepsilon}}_0 \mathcolor{gray}{\nabla^t} \left( \mathcolor{gray}{\nabla^t} O^{\;\!\mathcolor{gray}{t}}_{\;\!\symup{x} \symup{z} \symup{z} \mathcolor{gray}{0}} + N^{\;\!\mathcolor{gray}{t}}_{\;\! \symup{y} \symup{z} \mathcolor{gray}{0}} \right) ~, \\ E^{\;\!\mathcolor{gray}{t}}_{\;\! \symup{y}\mathcolor{gray}{0}} = &\hphantom{+} {\symup{\varepsilon}}_0^{-1} \mathcolor{gray}{\nabla_y} \left[ Q^{\;\!\mathcolor{gray}{t}}_{\;\! \symup{z} \symup{z} \mathcolor{gray}{0}} - 3~ \mathcolor{gray}{\nabla_x} O^{\;\!\mathcolor{gray}{t}}_{\;\! \symup{x} \symup{z} \symup{z} \mathcolor{gray}{0}} - 3~ \mathcolor{gray}{\nabla_y}  O^{\;\!\mathcolor{gray}{t}}_{\;\! \symup{y} \symup{z} \symup{z} \mathcolor{gray}{0}} - \mathcolor{gray}{\nabla_z}  O^{\;\!\mathcolor{gray}{t}}_{\;\! \symup{z} \symup{z} \symup{z} \mathcolor{gray}{0}} \right] \label{eq:E_y0==} \\ &+ {\symup{\varepsilon}}_0 \mathcolor{gray}{\nabla^t} \left( \mathcolor{gray}{\nabla^t} O^{\;\!\mathcolor{gray}{t}}_{\;\! \symup{y} \symup{z} \symup{z} \mathcolor{gray}{0}} - N^{\;\!\mathcolor{gray}{t}}_{\;\! \symup{x} \symup{z} \mathcolor{gray}{0}} \right) ~, \\
	E^{\;\!\mathcolor{gray}{t}}_{\;\! \symup{z} \mathcolor{gray}{0}} = &- {\symup{\varepsilon}}_0^{-1} \left[ \left( P^{\;\!\mathcolor{gray}{t}}_{\;\! \symup{z} \mathcolor{gray}{0}} - {\sigma}^{\;\!\mathcolor{gray}{t}}_{\;\! \mathcolor{gray}{0}} \right) - 2~ \mathcolor{gray}{\nabla_x} Q^{\;\!\mathcolor{gray}{t}}_{\;\! \symup{x} \symup{z} \mathcolor{gray}{0}} - 2~ \mathcolor{gray}{\nabla_y} Q^{\;\!\mathcolor{gray}{t}}_{\;\! \symup{y} \symup{z} \mathcolor{gray}{0}} - \mathcolor{gray}{\nabla_z} Q^{\;\!\mathcolor{gray}{t}}_{\;\! \symup{z} \symup{z} \mathcolor{gray}{0}} \right. \label{eq:E_z0==} \\ & + 3~ \mathcolor{gray}{\nabla_z} \left( \mathcolor{gray}{\nabla_x} O^{\;\!\mathcolor{gray}{t}}_{\;\! \symup{x} \symup{z} \symup{z} \mathcolor{gray}{0}} + \mathcolor{gray}{\nabla_y} O^{\;\!\mathcolor{gray}{t}}_{\;\! \symup{y} \symup{z} \symup{z} \mathcolor{gray}{0}} \right) + 6~ \mathcolor{gray}{\nabla_x} \mathcolor{gray}{\nabla_y} O^{\;\!\mathcolor{gray}{t}}_{\;\! \symup{x} \symup{y} \symup{z} \mathcolor{gray}{0}} \\ & \left. + 3 \left( \mathcolor{gray}{\nabla_x^2} O^{\;\!\mathcolor{gray}{t}}_{\;\! \symup{x} \symup{x} \symup{z} \mathcolor{gray}{0}} + \mathcolor{gray}{\nabla_y^2} O^{\;\!\mathcolor{gray}{t}}_{\;\! \symup{y} \symup{y} \symup{z} \mathcolor{gray}{0}} \right) + \left( \mathcolor{gray}{\nabla_z^2} - \mathcolor{gray}{\nabla_x^2} - \mathcolor{gray}{\nabla_y^2} \right) O^{\;\!\mathcolor{gray}{t}}_{\;\! \symup{z} \symup{z} \symup{z} \mathcolor{gray}{0}} \right] ~.
\end{align}
\end{subequations}
对于磁感应场,其(在宏观阶跃边界 $\mathcolor{gray}{z \mathcolor{black}{=} 0}$ 附近的)非零分量为
\begin{subequations} \label{eq:B^(2-0)_0==}
\begin{align}
	{\mathcal{B}}^{\;\!\textcolor{Maroon}{(1)}\mathcolor{gray}{t}}_{\;\! \symup{x} \mathcolor{gray}{0}} = &\hphantom{+} {\symup{\varepsilon}}_0 \left( N^{\;\!\mathcolor{gray}{t}}_{\;\!\symup{x}\symup{z}\mathcolor{gray}{0}} - \mathcolor{gray}{\nabla^t}
	O^{\;\!\mathcolor{gray}{t}}_{\;\!\symup{y}\symup{z}\symup{z}\mathcolor{gray}{0}} \right)~, \label{eq:B^(1)_x0==} \\
	{\mathcal{B}}^{\;\!\textcolor{Maroon}{(1)}\mathcolor{gray}{t}}_{\;\! \symup{y} \mathcolor{gray}{0}} = &\hphantom{+} {\symup{\varepsilon}}_0 \left( N^{\;\!\mathcolor{gray}{t}}_{\;\!\symup{y}\symup{z}\mathcolor{gray}{0}} + \mathcolor{gray}{\nabla^t}
	O^{\;\!\mathcolor{gray}{t}}_{\;\!\symup{x}\symup{z}\symup{z}\mathcolor{gray}{0}} \right)~; \label{eq:B^(1)_y0==} \\[1.0em]
	B^{\;\!\mathcolor{gray}{t}}_{\;\! \symup{x}\mathcolor{gray}{0}} = &\hphantom{+} {\symup{\varepsilon}}_0 \left[ M^{\;\!\mathcolor{gray}{t}}_{\;\! \symup{x}\mathcolor{gray}{0}} - \left( \mathcolor{gray}{\nabla^t} Q^{\;\!\mathcolor{gray}{t}}_{\;\! \symup{y} \symup{z} \mathcolor{gray}{0}} - 
	{\alpha}^{\;\!\mathcolor{gray}{t}}_{\;\! \symup{y}\mathcolor{gray}{0}} \right) +
	\left( \mathcolor{gray}{\nabla_x} N^{\;\!\mathcolor{gray}{t}}_{\;\!\symup{z} \symup{z} \mathcolor{gray}{0}} - \mathcolor{gray}{\nabla^\iota} N^{\;\!\mathcolor{gray}{t}}_{\;\! \symup{\iota} \symup{x} \mathcolor{gray}{0}} \right) \right. \label{eq:B_x0==} \\ & \left. + \mathcolor{gray}{\nabla^t} \left( 2~ \mathcolor{gray}{\nabla_x} O^{\;\!\mathcolor{gray}{t}}_{\;\! \symup{x} \symup{y} \symup{z} \mathcolor{gray}{0}} + 2~ \mathcolor{gray}{\nabla_y} O^{\;\!\mathcolor{gray}{t}}_{\;\! \symup{y} \symup{y} \symup{z} \mathcolor{gray}{0}} + \mathcolor{gray}{\nabla_z} O^{\;\!\mathcolor{gray}{t}}_{\;\! \symup{y} \symup{z} \symup{z} \mathcolor{gray}{0}} - \mathcolor{gray}{\nabla_y} O^{\;\!\mathcolor{gray}{t}}_{\;\! \symup{z} \symup{z} \symup{z} \mathcolor{gray}{0}} \right) \right]~, \\
	B^{\;\!\mathcolor{gray}{t}}_{\;\! \symup{y}\mathcolor{gray}{0}} = &\hphantom{+} {\symup{\varepsilon}}_0 \left[ M^{\;\!\mathcolor{gray}{t}}_{\;\! \symup{y}\mathcolor{gray}{0}} + \left( \mathcolor{gray}{\nabla^t} Q^{\;\!\mathcolor{gray}{t}}_{\;\! \symup{x} \symup{z} \mathcolor{gray}{0}} -
	{\alpha}^{\;\!\mathcolor{gray}{t}}_{\;\! \symup{x}\mathcolor{gray}{0}} \right) +
	\left( \mathcolor{gray}{\nabla_y} N^{\;\!\mathcolor{gray}{t}}_{\;\! \symup{z} \symup{z} \mathcolor{gray}{0}} - \mathcolor{gray}{\nabla^\iota} N^{\;\!\mathcolor{gray}{t}}_{\;\! \symup{\iota} \symup{y} \mathcolor{gray}{0}} \right) \right. \label{eq:B_y0==} \\ & \left. - \mathcolor{gray}{\nabla^t} \left( 2~ \mathcolor{gray}{\nabla_x} O^{\;\!\mathcolor{gray}{t}}_{\;\! \symup{x} \symup{x} \symup{z} \mathcolor{gray}{0}} + 2~ \mathcolor{gray}{\nabla_y}  O^{\;\!\mathcolor{gray}{t}}_{\;\! \symup{x} \symup{y} \symup{z} \mathcolor{gray}{0}} + \mathcolor{gray}{\nabla_z}  O^{\;\!\mathcolor{gray}{t}}_{\;\! \symup{x} \symup{z} \symup{z} \mathcolor{gray}{0}} - \mathcolor{gray}{\nabla_x}  O^{\;\!\mathcolor{gray}{t}}_{\;\! \symup{z} \symup{z} \symup{z} \mathcolor{gray}{0}} \right) \right]~, \\
	B^{\;\!\mathcolor{gray}{t}}_{\;\! \symup{z} \mathcolor{gray}{0}} = &\hphantom{+} {\symup{\varepsilon}}_0 \left[ \left( \mathcolor{gray}{\nabla_x}
	N^{\;\!\mathcolor{gray}{t}}_{\;\! \symup{x}\symup{z} \mathcolor{gray}{0}} + \mathcolor{gray}{\nabla_y} N^{\;\!\mathcolor{gray}{t}}_{\;\! \symup{y}\symup{z} \mathcolor{gray}{0}} \right) + \mathcolor{gray}{\nabla^t} \left( \mathcolor{gray}{\nabla_y}
	O^{\;\!\mathcolor{gray}{t}}_{\;\! \symup{x}\symup{z}\symup{z} \mathcolor{gray}{0}} - \mathcolor{gray}{\nabla_x}
	O^{\;\!\mathcolor{gray}{t}}_{\;\! \symup{y}\symup{z}\symup{z}\mathcolor{gray}{0}} \right) \right]~. \label{eq:B_z0==}
\end{align}
\end{subequations}
注意,上述 \bref{eq:E^(2-0)_0==} 与 \bref{eq:B^(2-0)_0==} 并不是对边界条件的完整描述。二者是通过直接沿用 \bref{eq:e-b-01'} $\to$ \bref{eq:div-e-b-01-deltas} 的方法得到的 —— 然而原则上不能这样做:因为(电磁)源可以因其所在的一侧是半无限真空而为零,但(电磁)场不会因真空而消失:真空中也允许(电磁)场的存在。因此,原则上从 \bref{eq:curl-EB-01-deltas} 处开始,所有公式中的每一个源/场量 $X$,都需要继承并添补上 \bref{eq:curl-EB-01} 中的左上角介质标记 $\textcolor{Maroon}{\mathsfit{z}}$ 并对两侧半无限介质(在接触面 $\mathcolor{gray}{z \mathcolor{black}{=} 0}$ 处)乘以 $\leftindex_{\textcolor{Maroon}{\mathsfit{z}}} \;\! \delta_{\mathcolor{gray}{z}}$(或其他对应的 $\leftindex_{\textcolor{Maroon}{\mathsfit{z}}} \;\! \delta'_{\mathcolor{gray}{z}},\leftindex_{\textcolor{Maroon}{\mathsfit{z}}} \;\! \delta''_{\mathcolor{gray}{z}}$)并求和以成为 $\leftindex_{\textcolor{Maroon}{\mathsfit{z}}} \;\! \delta_{\mathcolor{gray}{z}} \leftindex^{\textcolor{Maroon}{\mathsfit{z}}} X$(或 $\leftindex_{\textcolor{Maroon}{\mathsfit{z}}} \;\! \delta'_{\mathcolor{gray}{z}} \leftindex^{\textcolor{Maroon}{\mathsfit{z}}} X,\leftindex_{\textcolor{Maroon}{\mathsfit{z}}} \;\! \delta''_{\mathcolor{gray}{z}} \leftindex^{\textcolor{Maroon}{\mathsfit{z}}} X$)。—— 这一点从 \bref{eq:E_x0==,eq:E_y0==,eq:E_z0==} 和 \bref{eq:B_x0==,eq:B_y0==,eq:B_z0==} 如果不加上左上角介质标记 $\textcolor{Maroon}{\mathsfit{z}}$,则与 \bref{eq:EB-01} 冲突\Footnote{具有相同符号,但含义却不同。多对一的满射是不允许的。}的角度,也能看出。

总之,须对两侧介质 $\textcolor{Maroon}{\mathsfit{z}} = \textcolor{Maroon}{0,1}$ 中的 \bref{eq:E^(2-0)_0==} 与 \bref{eq:B^(2-0)_0==} 乘以 $\leftindex_{\textcolor{Maroon}{\mathsfit{z}}} \;\! \delta_{\mathcolor{gray}{z}}$(或 $\leftindex_{\textcolor{Maroon}{\mathsfit{z}}} \;\! \delta'_{\mathcolor{gray}{z}},\leftindex_{\textcolor{Maroon}{\mathsfit{z}}} \;\! \delta''_{\mathcolor{gray}{z}}$)并求和,即将其形式还原为 \bref{eq:curl-EB-01} 后,才是完整的边界条件。以代表 \bref{eq:E^(2-0)_0==} 和 \bref{eq:B^(2-0)_0==} 的场=源多项式,即 $Y^{\;\!\mathcolor{gray}{t}}_{\mathcolor{gray}{0}} = \leftindex_{\textcolor{Maroon}{i}} \;\! X^{\;\!\textcolor{Maroon}{(i)}\mathcolor{gray}{t}}_{\mathcolor{gray}{0}}$ 为例,它须升级为
\begin{subequations} \label{eq:2BC}
\begin{align}
	\leftindex_{\textcolor{Maroon}{\mathsfit{z}}} \;\! n_{\mathcolor{gray}{z}} \leftindex^{\textcolor{Maroon}{\mathsfit{z}}} \;\! Y^{\;\!\mathcolor{gray}{t}}_{\mathcolor{gray}{0}} &= \leftindex_{\textcolor{Maroon}{\mathsfit{z}}} \;\! n_{\mathcolor{gray}{z}} \leftindex^{\textcolor{Maroon}{\mathsfit{z}}}_{\textcolor{Maroon}{i}} \;\! X^{\;\!\textcolor{Maroon}{(i)}\mathcolor{gray}{t}}_{\mathcolor{gray}{0}} \label{eq:^zY_0} \\ \leftindex^{\textcolor{Maroon}{\mathsfit{0}}} \;\! Y^{\;\!\mathcolor{gray}{t}}_{\mathcolor{gray}{0}} - \leftindex^{\textcolor{Maroon}{\mathsfit{1}}} \;\! Y^{\;\!\mathcolor{gray}{t}}_{\mathcolor{gray}{0}} &= \leftindex^{\textcolor{Maroon}{\mathsfit{0}}}_{\textcolor{Maroon}{i}} \;\! X^{\;\!\textcolor{Maroon}{(i)}\mathcolor{gray}{t}}_{\mathcolor{gray}{0}} - \leftindex^{\textcolor{Maroon}{\mathsfit{1}}}_{\textcolor{Maroon}{i}} \;\! X^{\;\!\textcolor{Maroon}{(i)}\mathcolor{gray}{t}}_{\mathcolor{gray}{0}} \label{eq:^0Y_0-^1Y_0} \\ \leftindex^{\textcolor{Maroon}{\mathsfit{1}}} \;\! Y^{\;\!\mathcolor{gray}{t}}_{\mathcolor{gray}{0}} &= \leftindex^{\textcolor{Maroon}{\mathsfit{0}}} \;\! Y^{\;\!\mathcolor{gray}{t}}_{\mathcolor{gray}{0}} + \left( \leftindex^{\textcolor{Maroon}{\mathsfit{1}}}_{\textcolor{Maroon}{i}} \;\! X^{\;\!\textcolor{Maroon}{(i)}\mathcolor{gray}{t}}_{\mathcolor{gray}{0}} - \leftindex^{\textcolor{Maroon}{\mathsfit{0}}}_{\textcolor{Maroon}{i}} \;\! X^{\;\!\textcolor{Maroon}{(i)}\mathcolor{gray}{t}}_{\mathcolor{gray}{0}} \right) ~, \label{eq:^1Y_0}
\end{align}
\end{subequations}
有了上述双侧半无限介质的边界条件 \bref{eq:^1Y_0} 之后,才能谈论单侧半无限介质的边界条件。当入射侧介质 \textcolor{Maroon}{0} 为真空时,(电磁)源$\leftindex^{\textcolor{Maroon}{\mathsfit{0}}}_{\textcolor{Maroon}{i}} \;\! X^{\;\!\textcolor{Maroon}{(i)}\mathcolor{gray}{t}}_{\mathcolor{gray}{0}} = 0$ 但(电磁)场$\leftindex^{\textcolor{Maroon}{\mathsfit{0}}} \;\! Y^{\;\!\mathcolor{gray}{t}}_{\mathcolor{gray}{0}} \neq 0$,于是 \bref{eq:^1Y_0} 变为
\begin{align} \label{eq:1BC}
	\leftindex^{\textcolor{Maroon}{\mathsfit{1}}} \;\! Y^{\;\!\mathcolor{gray}{t}}_{\mathcolor{gray}{0}} = \leftindex^{\textcolor{Maroon}{\mathsfit{0}}} \;\! Y^{\;\!\mathcolor{gray}{t}}_{\mathcolor{gray}{0}} + \leftindex^{\textcolor{Maroon}{\mathsfit{1}}}_{\textcolor{Maroon}{i}} \;\! X^{\;\!\textcolor{Maroon}{(i)}\mathcolor{gray}{t}}_{\mathcolor{gray}{0}} ~.
\end{align}

上述 \bref{eq:1BC} 或 \bref{eq:2BC},结合 \bref{eq:E^(2-0)_0==} 和 \bref{eq:B^(2-0)_0==},给出了以下深刻结论:{\one} 由于 \bref{eq:E_x0==} 和 \bref{eq:E_y0==} 中的第一项,当展开至电四-磁偶极矩 $\bar{\bar{Q}}^{\;\!\mathcolor{gray}{t}}_{\;\!\mathcolor{gray}{z}}, \bar{M}^{\;\!\mathcolor{gray}{t}}_{\;\!\mathcolor{gray}{z}}$ 层次及以下时,基本场之一的电场 $\bar{E}^{\;\!\mathcolor{gray}{t}}_{\;\!\mathcolor{gray}{z}}$ 的切向分量 $\bar{E}^{\;\!\mathcolor{gray}{t}}_{\;\! \symup{\rho} \mathcolor{gray}{z}} := \begin{pmatrix} E^{\;\!\mathcolor{gray}{t}}_{\;\! \symup{x} \mathcolor{gray}{z}}, & E^{\;\!\mathcolor{gray}{t}}_{\;\! \symup{y} \mathcolor{gray}{z}} \end{pmatrix}^\top$(在接触面 $\mathcolor{gray}{z \mathcolor{black}{=} 0}$ 的左右两侧介质 $\textcolor{Maroon}{\mathsfit{z}} = \textcolor{Maroon}{0,1}$ 中)不连续,即 $\leftindex^{\textcolor{Maroon}{\mathsfit{1}}} {\bar{E}}^{\;\!\mathcolor{gray}{t}}_{\;\! \symup{\rho} \mathcolor{gray}{0}} \neq \leftindex^{\textcolor{Maroon}{\mathsfit{0}}} {\bar{E}}^{\;\!\mathcolor{gray}{t}}_{\;\! \symup{\rho} \mathcolor{gray}{0}}$;{\two} 从 \bref{eq:B_z0==} 可见,在电八-磁四极矩 $\bar{\bar{\bar{O}}}^{\;\!\mathcolor{gray}{t}}_{\;\!\mathcolor{gray}{z}}, \bar{\bar{N}}^{\;\!\mathcolor{gray}{t}}_{\;\!\mathcolor{gray}{z}}$ 层次及以下,基本场之一的磁感应场 $\bar{B}^{\;\!\mathcolor{gray}{t}}_{\;\!\mathcolor{gray}{z}}$ 的法向分量 $B^{\;\!\mathcolor{gray}{t}}_{\;\! \symup{z} \mathcolor{gray}{z}}$(在 $\mathcolor{gray}{z \mathcolor{black}{=} 0^+}$ 和 $\mathcolor{gray}{z \mathcolor{black}{=} 0^-}$ 处的值)不连续,即 $B^{\;\!\mathcolor{gray}{t}}_{\;\! \symup{z} \mathcolor{gray}{0^+}} \neq B^{\;\!\mathcolor{gray}{t}}_{\;\! \symup{z} \mathcolor{gray}{0^-}}$。—— 注意,原则上这里的 $\bar{E}^{\;\!\mathcolor{gray}{t}}_{\;\! \symup{\rho} \mathcolor{gray}{z}},B^{\;\!\mathcolor{gray}{t}}_{\;\! \symup{z} \mathcolor{gray}{z}}$ 也指 \bref{eq:EB-01} 等号右侧多项式中的第一项,而不是等号左侧的总场,因此也需要对它们添加左上角介质标记 $\textcolor{Maroon}{\mathsfit{z}}$。然而由于这里已经处于两侧介质内部,而不是恰好在接触面上,那么一旦 $\mathcolor{gray}{z} \neq \mathcolor{gray}{0}$,则这里的 $\bar{E}^{\;\!\mathcolor{gray}{t}}_{\;\! \symup{\rho} \mathcolor{gray}{z}},B^{\;\!\mathcolor{gray}{t}}_{\;\! \symup{z} \mathcolor{gray}{z}}$ 直接指代总场 \bref{eq:EB-01} 也没问题,因为此时二者相等。

值得注意的是,该 \bref{ssec:EB-boundary} 从 \bref{eq:curl-EB-01-deltas} 开始直到此处,只给出了 \bref{eq:EB-01} 中等号右侧各多项式中的场量在接触面 $\mathcolor{gray}{z \mathcolor{black}{=} 0}$ 附近的关于源的多极展开式和连续性关系。在深入材料内部的其他 $\mathcolor{gray}{z}$ 两侧的基本场 $\bar{E}^{\;\!\mathcolor{gray}{t}}_{\;\!\mathcolor{gray}{z}}, \bar{B}^{\;\!\mathcolor{gray}{t}}_{\;\!\mathcolor{gray}{z}}$ 的连续性关系,也可以套用该理论:只需将那里的 $\mathcolor{gray}{z}$ 视为新的 $\mathcolor{gray}{0}$ 即可。一般而言,在源 $\bar{P}^{\;\!\mathcolor{gray}{t}}_{\;\!\mathcolor{gray}{z}},\bar{\bar{Q}}^{\;\!\mathcolor{gray}{t}}_{\;\!\mathcolor{gray}{z}},\bar{\bar{\bar{O}}}^{\;\!\mathcolor{gray}{t}}_{\;\!\mathcolor{gray}{z}} ; \bar{M}^{\;\!\mathcolor{gray}{t}}_{\;\!\mathcolor{gray}{z}}, \bar{\bar{N}}^{\;\!\mathcolor{gray}{t}}_{\;\!\mathcolor{gray}{z}};{\rho}^{\;\!\mathcolor{gray}{t}}_{\;\!\textcolor{Maroon}{\text{f}}\mathcolor{gray}{z}}, \bar{J}^{\;\!\mathcolor{gray}{t}}_{\;\!\textcolor{Maroon}{\text{f}}\mathcolor{gray}{z}}$ 连续性分布的材料内部,即使材料/源(的 $\mathcolor{gray}{\bar{r}}$ 分布)不是均匀的,场 $\bar{E}^{\;\!\mathcolor{gray}{t}}_{\;\!\mathcolor{gray}{z}}, \bar{B}^{\;\!\mathcolor{gray}{t}}_{\;\!\mathcolor{gray}{z}}$(的分布)也是连续的。然而反过来,即使材料/源是均匀(分布)的,场 $\bar{E}^{\;\!\mathcolor{gray}{t}}_{\;\!\mathcolor{gray}{z}}, \bar{B}^{\;\!\mathcolor{gray}{t}}_{\;\!\mathcolor{gray}{z}}$ 也不是均匀(分布)的。

\bref{ssec:step-delta,ssec:EB-boundary,ssec:DH-boundary} 的理论用于解决宏观上突变的边界条件问题。以至于该理论可以提供 $\mathcolor{gray}{z \mathcolor{black}{=} 0}$ 处基本场 $\bar{E}^{\;\!\mathcolor{gray}{t}}_{\;\!\mathcolor{gray}{z}}, \bar{B}^{\;\!\mathcolor{gray}{t}}_{\;\!\mathcolor{gray}{z}}$ 的连续性关系和(场关于源的)值,以及深入材料内部的其他 $\mathcolor{gray}{z}$ 处的连续性关系,但无法给出在其他 $\mathcolor{gray}{z}$ 处的值。材料内部基本场 $\bar{E}^{\;\!\mathcolor{gray}{t}}_{\;\!\mathcolor{gray}{z}}, \bar{B}^{\;\!\mathcolor{gray}{t}}_{\;\!\mathcolor{gray}{z}}$ 的值及其动力学过程,需要进一步知晓本构关系的显示表达式(即材料/源关于场的响应函数)\cite{raabMultipoleTheoryElectromagnetism2004},并耦合之以联立求解有限体积内连续分布\cite{landauCHAPTERXIELECTROMAGNETIC1984}的 \textcolor{Maroon}{Maxwell-Lorentz-Heaviside} \bref{eq:curl-E,eq:div-B,eq:div-E,eq:curl-B}。

\clearpage
\vspace*{-8.0em}

\marginLeft[-2.4em]{ssec:DH-boundary}\subsection{$\bar{D},\bar{H}$ 辅助场、$\bar{D},\bar{H}$ 非唯一性}\label{ssec:DH-boundary}

当 \bref{ssec:PMQN} 的束缚源 \bref{eq:p-b,eq:j-b}、自由电源 \bref{eq:div-e-f},被 \bref{ssec:step-delta} 升级为 \bref{eq:e-b-01',eq:e-f-01'} 的同时,\bref{ssec:EBpJ} 的基本场 $\bar{E}^{\;\!\mathcolor{gray}{t}}_{\;\!\mathcolor{gray}{z}}, \bar{B}^{\;\!\mathcolor{gray}{t}}_{\;\!\mathcolor{gray}{z}}$ 也被上面 \bref{ssec:EB-boundary} 升级为 \bref{eq:EB-01}。此时,\bref{ssec:EHpJf} 中的 \textcolor{Maroon}{Maxwell-Lorentz-Heaviside} \bref{eq:curl-EK,eq:div-Bk,eq:div-D,eq:curl-H} 升级为 \bref{eq:curl-EB} 的类似物
\begin{subequations} \label{eq:curl-EH}
\begin{align}
	\epsilon^{\hphantom{\symup{\iota}\hat{1}}\hat{2}}_{\symup{\iota}\mathcolor{gray}{\hat{1}}} \mathcolor{gray}{\nabla^{\hat{1}}} E^{\;\!\mathcolor{gray}{t}}_{\;\! \hat{2}\mathcolor{gray}{z}} + \mathcolor{gray}{\nabla^t} B^{\;\!\mathcolor{gray}{t}}_{\;\! \symup{\iota}\mathcolor{gray}{z}} &= 0~, \label{eq:curl-E'duplicate} \\
	\epsilon^{\hphantom{\symup{\iota}\hat{1}}\hat{2}}_{\symup{\iota}\mathcolor{gray}{\hat{1}}} \mathcolor{gray}{\nabla^{\hat{1}}} H^{\;\!\mathcolor{gray}{t}}_{\;\! \hat{2}\mathcolor{gray}{z}} - \mathcolor{gray}{\nabla^t} D^{\;\!\mathcolor{gray}{t}}_{\;\! \symup{\iota}\mathcolor{gray}{z}} &= \leftindex_{\textcolor{Maroon}{\mathsfit{z}}} {\mathbb{1}}_{\mathcolor{gray}{z}} \leftindex^{\textcolor{Maroon}{\mathsfit{z}}} \;\! J^{\;\!\mathcolor{gray}{t}}_{\;\!\textcolor{Maroon}{\text{f}} \symup{\iota}\mathcolor{gray}{z}} + \leftindex_{\textcolor{Maroon}{\mathsfit{z}}} \;\! \delta_{\mathcolor{gray}{z}} \leftindex^{\textcolor{Maroon}{\mathsfit{z}}} \;\!
	{\alpha}^{\;\!\mathcolor{gray}{t}}_{\;\! \symup{\iota}\mathcolor{gray}{z}} ~, \label{eq:curl-H'} \\
	\mathcolor{gray}{\nabla^\iota} D^{\;\!\mathcolor{gray}{t}}_{\;\! \symup{\iota}\mathcolor{gray}{z}} &= \leftindex_{\textcolor{Maroon}{\mathsfit{z}}} {\mathbb{1}}_{\mathcolor{gray}{z}} \leftindex^{\textcolor{Maroon}{\mathsfit{z}}} {\rho}^{\;\!\mathcolor{gray}{t}}_{\;\!\textcolor{Maroon}{\text{f}}\mathcolor{gray}{z}} + \leftindex_{\textcolor{Maroon}{\mathsfit{z}}} \;\! \delta_{\mathcolor{gray}{z}} \leftindex^{\textcolor{Maroon}{\mathsfit{z}}} \;\! {\sigma}^{\;\!\mathcolor{gray}{t}}_{\;\! \mathcolor{gray}{z}}~, \label{eq:div-D'} \\
	\mathcolor{gray}{\nabla^\iota} B^{\;\!\mathcolor{gray}{t}}_{\;\! \symup{\iota}\mathcolor{gray}{z}} &= 0~. \label{eq:div-B'duplicate}
\end{align}
\end{subequations}
其中,$\bar{E}^{\;\!\mathcolor{gray}{t}}_{\;\!\mathcolor{gray}{z}}, \bar{B}^{\;\!\mathcolor{gray}{t}}_{\;\!\mathcolor{gray}{z}}$ 为 \bref{eq:EB-01} 等号左侧的总场,$\bar{D}^{\;\!\mathcolor{gray}{t}}_{\;\!\mathcolor{gray}{z}}, \bar{H}^{\;\!\mathcolor{gray}{t}}_{\;\!\mathcolor{gray}{z}}$ 也具有相同的奇异场层次结构,但尚未展开成 \bref{eq:EB-01} 的形式。将上述 \bref{eq:div-D'} 减去 \bref{eq:div-E'},\bref{eq:curl-H'} 减去 \bref{eq:curl-B'} 乘以 ${\symup{\varepsilon}}_0$,得到
\begin{subequations} \label{eq:curl-EH-EB}
	\small
\begin{align}
	&\mathcolor{gray}{\nabla^\iota} \left( D^{\;\!\mathcolor{gray}{t}}_{\;\! \symup{\iota}\mathcolor{gray}{z}} - {\symup{\varepsilon}}_0 E^{\;\!\mathcolor{gray}{t}}_{\;\! \symup{\iota}\mathcolor{gray}{z}} \right) = - \left( \leftindex_{\textcolor{Maroon}{\mathsfit{z}}} {\mathbb{1}}_{\mathcolor{gray}{z}} \leftindex^{\textcolor{Maroon}{\mathsfit{z}}}  {\rho}^{\;\!\mathcolor{gray}{t}}_{\;\!\textcolor{Maroon}{\text{b}}\mathcolor{gray}{z}} - \leftindex_{\textcolor{Maroon}{\mathsfit{z}}} \;\! \delta_{\mathcolor{gray}{z}} \leftindex^{\textcolor{Maroon}{\mathsfit{z}}} \;\! {\mathcal{P}}^{\;\!\mathcolor{gray}{t}}_{\;\! \symup{z} \mathcolor{gray}{z}} - \leftindex_{\textcolor{Maroon}{\mathsfit{z}}} \;\! \delta'_{\mathcolor{gray}{z}} \leftindex^{\textcolor{Maroon}{\mathsfit{z}}} \;\! {\mathcal{Q}}^{\;\!\mathcolor{gray}{t}}_{\;\! \symup{z} \symup{z} \mathcolor{gray}{z}} - \leftindex_{\textcolor{Maroon}{\mathsfit{z}}} \;\! \delta''_{\mathcolor{gray}{z}} \leftindex^{\textcolor{Maroon}{\mathsfit{z}}} \;\! {\mathcal{O}}^{\;\!\mathcolor{gray}{t}}_{\;\! \symup{z} \symup{z} \symup{z} \mathcolor{gray}{z}} \right) ~, \label{eq:div-D'-E'} \\
	\epsilon^{\hphantom{\symup{\iota}\hat{1}}\hat{2}}_{\symup{\iota}\mathcolor{gray}{\hat{1}}} &\mathcolor{gray}{\nabla^{\hat{1}}} \left( H^{\;\!\mathcolor{gray}{t}}_{\;\! \hat{2}\mathcolor{gray}{z}} - {\symup{\varepsilon}}_0^{-1} B^{\;\!\mathcolor{gray}{t}}_{\;\! \hat{2}\mathcolor{gray}{z}} \right) - \mathcolor{gray}{\nabla^t} \left( D^{\;\!\mathcolor{gray}{t}}_{\;\! \symup{\iota}\mathcolor{gray}{z}} - {\symup{\varepsilon}}_0 E^{\;\!\mathcolor{gray}{t}}_{\;\! \symup{\iota}\mathcolor{gray}{z}} \right) = - \left( \leftindex_{\textcolor{Maroon}{\mathsfit{z}}} {\mathbb{1}}_{\mathcolor{gray}{z}} \leftindex^{\textcolor{Maroon}{\mathsfit{z}}} \;\! J^{\;\!\mathcolor{gray}{t}}_{\;\!\textcolor{Maroon}{\text{b}} \symup{\iota}\mathcolor{gray}{z}} - \leftindex_{\textcolor{Maroon}{\mathsfit{z}}} \;\! \delta_{\mathcolor{gray}{z}} \leftindex^{\textcolor{Maroon}{\mathsfit{z}}}
	{\mathcal{K}}^{\;\!\mathcolor{gray}{t}}_{\;\! \symup{\iota}\symup{z}\mathcolor{gray}{z}} - \leftindex_{\textcolor{Maroon}{\mathsfit{z}}} \;\! \delta'_{\mathcolor{gray}{z}} \leftindex^{\textcolor{Maroon}{\mathsfit{z}}} \;\! {\mathcal{L}}^{\;\!\mathcolor{gray}{t}}_{\;\! \symup{\iota}\symup{z} \symup{z} \mathcolor{gray}{z}} \right) ~, \label{eq:curl-H'-B'}
\end{align}
\end{subequations}
该 \bref{eq:curl-EH-EB} 暗示 $\bar{D}^{\;\!\mathcolor{gray}{t}}_{\;\!\mathcolor{gray}{z}}, \bar{H}^{\;\!\mathcolor{gray}{t}}_{\;\!\mathcolor{gray}{z}}$ 具有以下类似但不同于 \bref{eq:EB-01} 的奇异场层次\Footnote{\bref{eq:DH-01} 可视为教科书式定义下的辅助场 $\bar{D}^{\;\!\mathcolor{gray}{t}}_{\;\!\mathcolor{gray}{z}}, \bar{H}^{\;\!\mathcolor{gray}{t}}_{\;\!\mathcolor{gray}{z}}$ 同时朝多极形式和奇异结构的扩展。}
\begin{subequations} \label{eq:DH-01}
\begin{align}
	\hphantom{xxxxx} D^{\;\!\mathcolor{gray}{t}}_{\;\! \symup{\iota}\mathcolor{gray}{z}} &= \hspace{0.2em} {\symup{\varepsilon}}_0 &&\hspace{-2.7em} E^{\;\!\mathcolor{gray}{t}}_{\;\! \symup{\iota}\mathcolor{gray}{z}} \hspace{0.5em} + &&\hspace{-2.5em}\leftindex_{\textcolor{Maroon}{\mathsfit{z}}} {\mathbb{1}}_{\mathcolor{gray}{z}} \leftindex^{\textcolor{Maroon}{\mathsfit{z}}} \;\! D^{\;\!\mathcolor{gray}{t}}_{\;\! \symup{\iota}\mathcolor{gray}{z}} &&\hspace{-2.5em}- \leftindex_{\textcolor{Maroon}{\mathsfit{z}}} \;\! \delta_{\mathcolor{gray}{z}} \leftindex^{\textcolor{Maroon}{\mathsfit{z}}} \;\!
	{\mathcal{D}}^{\;\!\textcolor{Maroon}{(1)}\mathcolor{gray}{t}}_{\;\! \symup{\iota}\mathcolor{gray}{z}} &&\hspace{-2.5em}- \leftindex_{\textcolor{Maroon}{\mathsfit{z}}} \;\! \delta'_{\mathcolor{gray}{z}} \leftindex^{\textcolor{Maroon}{\mathsfit{z}}} \;\! {\mathcal{D}}^{\;\!\textcolor{Maroon}{(2)}\mathcolor{gray}{t}}_{\;\! \symup{\iota}\mathcolor{gray}{z}} ~, \label{eq:D-01} \\
	\hphantom{xxxxx} H^{\;\!\mathcolor{gray}{t}}_{\;\! \symup{\iota}\mathcolor{gray}{z}} &= \hspace{0.2em} {\symup{\varepsilon}}_0^{-1} &&\hspace{-2.7em} B^{\;\!\mathcolor{gray}{t}}_{\;\! \symup{\iota}\mathcolor{gray}{z}} \hspace{0.5em} + &&\hspace{-2.5em}\leftindex_{\textcolor{Maroon}{\mathsfit{z}}} {\mathbb{1}}_{\mathcolor{gray}{z}} \leftindex^{\textcolor{Maroon}{\mathsfit{z}}} \;\! H^{\;\!\mathcolor{gray}{t}}_{\;\! \symup{\iota}\mathcolor{gray}{z}} &&\hspace{-2.5em}- \leftindex_{\textcolor{Maroon}{\mathsfit{z}}} \;\! \delta_{\mathcolor{gray}{z}} \leftindex^{\textcolor{Maroon}{\mathsfit{z}}} \;\!
	{\mathcal{H}}^{\;\!\textcolor{Maroon}{(1)}\mathcolor{gray}{t}}_{\;\! \symup{\iota}\mathcolor{gray}{z}} &&\hspace{-2.5em}~, \label{eq:H-01}
\end{align}
\end{subequations}
其中 $\bar{E}^{\;\!\mathcolor{gray}{t}}_{\;\!\mathcolor{gray}{z}}, \bar{B}^{\;\!\mathcolor{gray}{t}}_{\;\!\mathcolor{gray}{z}}$ 为 \bref{eq:EB-01} 中具有奇异多项式的总场。将上述 \bref{eq:DH-01} 代入 \bref{eq:curl-EH-EB},得到类似但不同于 \bref{eq:curl-EB-01} 的形式
\begin{subequations} \label{eq:curl-DH-01}
	\footnotesize
\begin{align}
	\left( \leftindex_{\textcolor{Maroon}{\mathsfit{z}}} \;\! \delta_{\mathcolor{gray}{z}} \leftindex^{\textcolor{Maroon}{\mathsfit{z}}} \;\! D^{\;\!\mathcolor{gray}{t}}_{\;\! \symup{z} \mathcolor{gray}{z}} - \leftindex_{\textcolor{Maroon}{\mathsfit{z}}} \;\! \delta'_{\mathcolor{gray}{z}} \leftindex^{\textcolor{Maroon}{\mathsfit{z}}} \;\!
	{\mathcal{D}}^{\;\!\textcolor{Maroon}{(1)}\mathcolor{gray}{t}}_{\;\! \symup{z} \mathcolor{gray}{z}} \right. &- \left. \leftindex_{\textcolor{Maroon}{\mathsfit{z}}} \;\! \delta''_{\mathcolor{gray}{z}} \leftindex^{\textcolor{Maroon}{\mathsfit{z}}} \;\! {\mathcal{D}}^{\;\!\textcolor{Maroon}{(2)}\mathcolor{gray}{t}}_{\;\! \symup{z} \mathcolor{gray}{z}} \right) + \left( \leftindex_{\textcolor{Maroon}{\mathsfit{z}}} {\mathbb{1}}_{\mathcolor{gray}{z}} \mathcolor{gray}{\nabla^\iota} \leftindex^{\textcolor{Maroon}{\mathsfit{z}}} \;\! D^{\;\!\mathcolor{gray}{t}}_{\;\! \symup{\iota}\mathcolor{gray}{z}} - \leftindex_{\textcolor{Maroon}{\mathsfit{z}}} \;\! \delta_{\mathcolor{gray}{z}} \mathcolor{gray}{\nabla^\iota} \leftindex^{\textcolor{Maroon}{\mathsfit{z}}} \;\!
	{\mathcal{D}}^{\;\!\textcolor{Maroon}{(1)}\mathcolor{gray}{t}}_{\;\! \symup{\iota}\mathcolor{gray}{z}} - \leftindex_{\textcolor{Maroon}{\mathsfit{z}}} \;\! \delta'_{\mathcolor{gray}{z}} \mathcolor{gray}{\nabla^\iota} \leftindex^{\textcolor{Maroon}{\mathsfit{z}}} \;\!
	{\mathcal{D}}^{\;\!\textcolor{Maroon}{(2)}\mathcolor{gray}{t}}_{\;\! \symup{\iota}\mathcolor{gray}{z}} \right) \label{eq:div-D-01} \\ &= - \left( \leftindex_{\textcolor{Maroon}{\mathsfit{z}}} {\mathbb{1}}_{\mathcolor{gray}{z}} \leftindex^{\textcolor{Maroon}{\mathsfit{z}}}  {\rho}^{\;\!\mathcolor{gray}{t}}_{\;\!\textcolor{Maroon}{\text{b}}\mathcolor{gray}{z}} - \leftindex_{\textcolor{Maroon}{\mathsfit{z}}} \;\! \delta_{\mathcolor{gray}{z}} \leftindex^{\textcolor{Maroon}{\mathsfit{z}}} \;\! {\mathcal{P}}^{\;\!\mathcolor{gray}{t}}_{\;\! \symup{z} \mathcolor{gray}{z}} - \leftindex_{\textcolor{Maroon}{\mathsfit{z}}} \;\! \delta'_{\mathcolor{gray}{z}} \leftindex^{\textcolor{Maroon}{\mathsfit{z}}} \;\! {\mathcal{Q}}^{\;\!\mathcolor{gray}{t}}_{\;\! \symup{z} \symup{z} \mathcolor{gray}{z}} - \leftindex_{\textcolor{Maroon}{\mathsfit{z}}} \;\! \delta''_{\mathcolor{gray}{z}} \leftindex^{\textcolor{Maroon}{\mathsfit{z}}} \;\! {\mathcal{O}}^{\;\!\mathcolor{gray}{t}}_{\;\! \symup{z} \symup{z} \symup{z} \mathcolor{gray}{z}} \right)~, \\
	\epsilon^{\hphantom{\symup{iz}}\hat{2}}_{\symup{\iota}\mathcolor{gray}{\symup{z}}} \left( \leftindex_{\textcolor{Maroon}{\mathsfit{z}}} \;\! \delta_{\mathcolor{gray}{z}} \leftindex^{\textcolor{Maroon}{\mathsfit{z}}} H^{\;\!\mathcolor{gray}{t}}_{\;\! \hat{2}\mathcolor{gray}{z}} - \leftindex_{\textcolor{Maroon}{\mathsfit{z}}} \;\! \delta'_{\mathcolor{gray}{z}} \leftindex^{\textcolor{Maroon}{\mathsfit{z}}} \;\!
	{\mathcal{H}}^{\;\!\textcolor{Maroon}{(1)}\mathcolor{gray}{t}}_{\;\! \hat{2}\mathcolor{gray}{z}} \right) &+ \epsilon^{\hphantom{\symup{\iota}\hat{1}}\hat{2}}_{\symup{\iota}\mathcolor{gray}{\hat{1}}} \left( \leftindex_{\textcolor{Maroon}{\mathsfit{z}}} {\mathbb{1}}_{\mathcolor{gray}{z}} \mathcolor{gray}{\nabla^{\hat{1}}} \leftindex^{\textcolor{Maroon}{\mathsfit{z}}} \;\! H^{\;\!\mathcolor{gray}{t}}_{\;\! \hat{2}\mathcolor{gray}{z}} - \leftindex_{\textcolor{Maroon}{\mathsfit{z}}} \;\! \delta_{\mathcolor{gray}{z}} \mathcolor{gray}{\nabla^{\hat{1}}} \leftindex^{\textcolor{Maroon}{\mathsfit{z}}} \;\!
	{\mathcal{H}}^{\;\!\textcolor{Maroon}{(1)}\mathcolor{gray}{t}}_{\;\! \hat{2}\mathcolor{gray}{z}} \right) + \mathcolor{gray}{\nabla^t} \left( \leftindex_{\textcolor{Maroon}{\mathsfit{z}}} \;\! \delta_{\mathcolor{gray}{z}} \leftindex^{\textcolor{Maroon}{\mathsfit{z}}}
	{\mathcal{D}}^{\;\!\textcolor{Maroon}{(1)}\mathcolor{gray}{t}}_{\;\! \symup{\iota}\mathcolor{gray}{z}} + \leftindex_{\textcolor{Maroon}{\mathsfit{z}}} \;\! \delta'_{\mathcolor{gray}{z}} \leftindex^{\textcolor{Maroon}{\mathsfit{z}}} \;\! {\mathcal{D}}^{\;\!\textcolor{Maroon}{(2)}\mathcolor{gray}{t}}_{\;\! \symup{\iota}\mathcolor{gray}{z}} \right) \label{eq:curl-H-01} \\ &= - \left( \leftindex_{\textcolor{Maroon}{\mathsfit{z}}} {\mathbb{1}}_{\mathcolor{gray}{z}} \leftindex^{\textcolor{Maroon}{\mathsfit{z}}} \;\! J^{\;\!\mathcolor{gray}{t}}_{\;\!\textcolor{Maroon}{\text{b}} \symup{\iota}\mathcolor{gray}{z}} - \leftindex_{\textcolor{Maroon}{\mathsfit{z}}} \;\! \delta_{\mathcolor{gray}{z}} \leftindex^{\textcolor{Maroon}{\mathsfit{z}}}
	{\mathcal{K}}^{\;\!\mathcolor{gray}{t}}_{\;\! \symup{\iota}\symup{z}\mathcolor{gray}{z}} - \leftindex_{\textcolor{Maroon}{\mathsfit{z}}} \;\! \delta'_{\mathcolor{gray}{z}} \leftindex^{\textcolor{Maroon}{\mathsfit{z}}} \;\! {\mathcal{L}}^{\;\!\mathcolor{gray}{t}}_{\;\! \symup{\iota}\symup{z} \symup{z} \mathcolor{gray}{z}} \right) + \mathcolor{gray}{\nabla^t} \leftindex_{\textcolor{Maroon}{\mathsfit{z}}} {\mathbb{1}}_{\mathcolor{gray}{z}} \leftindex^{\textcolor{Maroon}{\mathsfit{z}}} \;\! D^{\;\!\mathcolor{gray}{t}}_{\;\! \symup{\iota}\mathcolor{gray}{z}} ~,
\end{align}
\end{subequations}
合并同层次奇异项(通过 \bref{eq:Intdeltasum=0})和额外的 ${\mathbb{1}}_{\mathcolor{gray}{z}} ~\textcolor{Maroon}{\text{项}}$,得 \bref{eq:curl-EB-01-deltas} 的对应版本
\begin{subequations} \label{eq:curl-DH-01-deltas}
\begin{align}
	{\mathbb{1}}_{\mathcolor{gray}{z}} ~\textcolor{Maroon}{\text{项}}:&\hspace{1.0em}  \hphantom{\epsilon^{\hphantom{\symup{iz}}\hat{2}}_{\symup{\iota}\mathcolor{gray}{\symup{z}}} H^{\;\!\mathcolor{gray}{t}}_{\;\! \hat{2}\mathcolor{gray}{0}} - \epsilon^{\hphantom{\symup{\iota}\hat{1}}\hat{2}}_{\symup{\iota}\mathcolor{gray}{\hat{1}}}} \mathcolor{gray}{\nabla^\iota} \;\! D^{\;\!\mathcolor{gray}{t}}_{\;\! \symup{\iota}\mathcolor{gray}{z}} \hspace{-3.8em}&&=\hspace{0.2em} -\hspace{0.2em} {\rho}^{\;\!\mathcolor{gray}{t}}_{\;\!\textcolor{Maroon}{\text{b}}\mathcolor{gray}{z}}~,  \label{eq:curl-DH-01-one} \\
	&\hspace{1.0em} \hphantom{\epsilon^{\hphantom{\symup{iz}}\hat{2}}_{\symup{\iota}\mathcolor{gray}{\symup{z}}} H^{\;\!\mathcolor{gray}{t}}_{\;\! \hat{2}\mathcolor{gray}{0}} -}\;\!\;\!\;\!\;\! \epsilon^{\hphantom{\symup{\iota}\hat{1}}\hat{2}}_{\symup{\iota}\mathcolor{gray}{\hat{1}}} \mathcolor{gray}{\nabla^{\hat{1}}} H^{\;\!\mathcolor{gray}{t}}_{\;\! \hat{2}\mathcolor{gray}{z}} - \mathcolor{gray}{\nabla^t} D^{\;\!\mathcolor{gray}{t}}_{\;\! \symup{\iota}\mathcolor{gray}{z}} \hspace{-3.8em}&&=\hspace{0.2em} -\hspace{0.2em} J^{\;\!\mathcolor{gray}{t}}_{\;\!\textcolor{Maroon}{\text{b}} \symup{\iota}\mathcolor{gray}{z}}~, \label{eq:curl-DH-01-one2} \\[0.7em]
	{\delta}_{\mathcolor{gray}{z}} ~\textcolor{Maroon}{\text{项}}:&\hspace{1.0em}  \hphantom{\epsilon^{\hphantom{\symup{iz}}\hat{2}}_{\symup{\iota}\mathcolor{gray}{\symup{z}}}}\;\! D^{\;\!\mathcolor{gray}{t}}_{\;\! \symup{z} \mathcolor{gray}{0}} - \hphantom{\epsilon^{\hphantom{\symup{\iota}\hat{1}}\hat{2}}_{\symup{\iota}\mathcolor{gray}{\hat{1}}}} \mathcolor{gray}{\nabla^\iota}
	{\mathcal{D}}^{\;\!\textcolor{Maroon}{(1)}\mathcolor{gray}{t}}_{\;\! \symup{\iota}\mathcolor{gray}{0}} \hspace{-3.8em}&&=\hspace{0.2em} {\mathcal{P}}^{\;\!\mathcolor{gray}{t}}_{\;\! \symup{z} \mathcolor{gray}{0}}~,  \label{eq:curl-DH-01-delta} \\
	&\hspace{1.0em} \epsilon^{\hphantom{\symup{iz}}\hat{2}}_{\symup{\iota}\mathcolor{gray}{\symup{z}}} H^{\;\!\mathcolor{gray}{t}}_{\;\! \hat{2}\mathcolor{gray}{0}} - \epsilon^{\hphantom{\symup{\iota}\hat{1}}\hat{2}}_{\symup{\iota}\mathcolor{gray}{\hat{1}}} \mathcolor{gray}{\nabla^{\hat{1}}} 
	{\mathcal{H}}^{\;\!\textcolor{Maroon}{(1)}\mathcolor{gray}{t}}_{\;\! \hat{2}\mathcolor{gray}{0}} + \mathcolor{gray}{\nabla^t} 
	{\mathcal{D}}^{\;\!\textcolor{Maroon}{(1)}\mathcolor{gray}{t}}_{\;\! \symup{\iota}\mathcolor{gray}{0}} \hspace{-3.8em}&&=\hspace{0.2em} {\mathcal{K}}^{\;\!\mathcolor{gray}{t}}_{\;\! \symup{\iota}\symup{z}\mathcolor{gray}{0}}~, \label{eq:curl-DH-01-delta2} \\[0.7em]
	{\delta}'_{\mathcolor{gray}{z}} ~\textcolor{Maroon}{\text{项}}:&\hspace{1.0em}  \hphantom{\epsilon^{\hphantom{\symup{iz}}\hat{2}}_{\symup{\iota}\mathcolor{gray}{\symup{z}}}} {\mathcal{D}}^{\;\!\textcolor{Maroon}{(1)}\mathcolor{gray}{t}}_{\;\! \symup{z} \mathcolor{gray}{0}} + \hphantom{\epsilon^{\hphantom{\symup{\iota}\hat{1}}\hat{2}}_{\symup{\iota}\mathcolor{gray}{\hat{1}}}}\!\!\;\! \mathcolor{gray}{\nabla^\iota} 
	{\mathcal{D}}^{\;\!\textcolor{Maroon}{(2)}\mathcolor{gray}{t}}_{\;\! \symup{\iota}\mathcolor{gray}{0}} \hspace{-3.8em}&&=\hspace{0.2em} -\hspace{0.2em} {\mathcal{Q}}^{\;\!\mathcolor{gray}{t}}_{\;\! \symup{z} \symup{z} \mathcolor{gray}{0}}~,  \label{eq:curl-DH-01-delta'} \\
	&\hspace{1.0em} \epsilon^{\hphantom{\symup{iz}}\hat{2}}_{\symup{\iota}\mathcolor{gray}{\symup{z}}} {\mathcal{H}}^{\;\!\textcolor{Maroon}{(1)}\mathcolor{gray}{t}}_{\;\! \hat{2}\mathcolor{gray}{0}} \hphantom{\epsilon^{\hphantom{\symup{iz}}\hat{2}}_{\symup{\iota}\mathcolor{gray}{\symup{z}}} H^{\;\!\mathcolor{gray}{t}}_{\;\! \hat{2}\mathcolor{gray}{0}} - \epsilon^{\hphantom{\symup{\iota}\hat{1}}\hat{2}}_{\symup{\iota}\mathcolor{gray}{\hat{1}}}}\!\! - \mathcolor{gray}{\nabla^t} 
	{\mathcal{D}}^{\;\!\textcolor{Maroon}{(2)}\mathcolor{gray}{t}}_{\;\! \symup{\iota}\mathcolor{gray}{0}} \hspace{-3.8em}&&=\hspace{0.2em} -\hspace{0.2em} {\mathcal{L}}^{\;\!\mathcolor{gray}{t}}_{\;\! \symup{\iota}\symup{z} \symup{z} \mathcolor{gray}{0}}~,  \label{eq:curl-DH-01-delta'2} \\[0.7em]
	{\delta}''_{\mathcolor{gray}{z}} ~\textcolor{Maroon}{\text{项}}:&\hspace{1.0em} \hphantom{\epsilon^{\hphantom{\symup{iz}}\hat{2}}_{\symup{\iota}\mathcolor{gray}{\symup{z}}}} 
	{\mathcal{D}}^{\;\!\textcolor{Maroon}{(2)}\mathcolor{gray}{t}}_{\;\! \symup{z} \mathcolor{gray}{0}} \hspace{-3.8em}&&=\hspace{0.2em} -\hspace{0.2em} {\mathcal{O}}^{\;\!\mathcolor{gray}{t}}_{\;\! \symup{z} \symup{z} \symup{z} \mathcolor{gray}{0}}~, \label{eq:curl-DH-01-delta''}
\end{align}
\end{subequations}
其中,对比 \bref{eq:curl-DH-01-one,eq:pb} 得 $D^{\;\!\mathcolor{gray}{t}}_{\;\! \symup{\iota}\mathcolor{gray}{z}} = {\mathcal{P}}^{\;\!\mathcolor{gray}{t}}_{\;\!\textcolor{Maroon}{\text{b}} \symup{\iota} \mathcolor{gray}{z}}$;将其和 \bref{eq:Je-Jb} 代入 \bref{eq:curl-DH-01-one2} 得 $H^{\;\!\mathcolor{gray}{t}}_{\;\! \hat{2}\mathcolor{gray}{z}} = - M^{\;\!\mathcolor{gray}{t}}_{\;\! \hat{2} \mathcolor{gray}{z}} + \mathcolor{gray}{\nabla^{\hat{3}}} N^{\;\!\mathcolor{gray}{t}}_{\;\! \hat{2}\hat{3} \mathcolor{gray}{z}}$;继续将 $D^{\;\!\mathcolor{gray}{t}}_{\;\! \symup{\iota}\mathcolor{gray}{0}}$ 和 \bref{eq:Pb-QB} 代入 \bref{eq:curl-DH-01-delta},得到 ${\mathcal{D}}^{\;\!\textcolor{Maroon}{(1)}\mathcolor{gray}{t}}_{\;\! \symup{\iota}\mathcolor{gray}{0}} = - {\mathcal{Q}}^{\;\!\mathcolor{gray}{t}}_{\;\!\textcolor{Maroon}{\text{B}} \symup{z}\symup{\iota} \mathcolor{gray}{0}}$;将其和 $H^{\;\!\mathcolor{gray}{t}}_{\;\! \hat{2}\mathcolor{gray}{0}}$,以及 \bref{eq:KE-Kb} 代入 \bref{eq:curl-DH-01-delta2} 得 ${\mathcal{H}}^{\;\!\textcolor{Maroon}{(1)}\mathcolor{gray}{t}}_{\;\! \hat{2}\mathcolor{gray}{0}} = - N^{\;\!\mathcolor{gray}{t}}_{\;\! \hat{2}\symup{z}\mathcolor{gray}{0}}$;继续将 ${\mathcal{D}}^{\;\!\textcolor{Maroon}{(1)}\mathcolor{gray}{t}}_{\;\! \symup{\iota}\mathcolor{gray}{0}}$ 和 \bref{eq:Qb-KM} 代入 \bref{eq:curl-DH-01-delta'} 得 ${\mathcal{D}}^{\;\!\textcolor{Maroon}{(2)}\mathcolor{gray}{t}}_{\;\! \symup{\iota}\mathcolor{gray}{0}} = - {\mathcal{O}}^{\;\!\mathcolor{gray}{t}}_{\;\!\textcolor{Maroon}{\text{b}} \symup{z}\symup{z}\symup{\iota} \mathcolor{gray}{0}}$。可以验证上述得到的 ${\mathcal{H}}^{\;\!\textcolor{Maroon}{(1)}\mathcolor{gray}{t}}_{\;\! \hat{2}\mathcolor{gray}{0}}, {\mathcal{D}}^{\;\!\textcolor{Maroon}{(2)}\mathcolor{gray}{t}}_{\;\! \symup{\iota}\mathcolor{gray}{0}}$ 满足 \bref{eq:curl-DH-01-delta'2,eq:curl-DH-01-delta'',eq:LE-Lb}。

将上一段各量展开至 \bref{eq:multipole} 层次\Footnote{用到了多极矩脚标分量的置换对称性。}
\begin{subequations} \label{eq:DH^(2-0)_0}
\begin{align}
	D^{\;\!\mathcolor{gray}{t}}_{\;\! \symup{\iota}\mathcolor{gray}{z}} &= {\mathcal{P}}^{\;\!\mathcolor{gray}{t}}_{\;\!\textcolor{Maroon}{\text{b}} \symup{\iota} \mathcolor{gray}{z}}~, \label{eq:D^(0)} \\
	{\mathcal{D}}^{\;\!\textcolor{Maroon}{(1)}\mathcolor{gray}{t}}_{\;\! \symup{\iota}\mathcolor{gray}{0}} &= -\hspace{0.2em} {\mathcal{Q}}^{\;\!\mathcolor{gray}{t}}_{\;\!\textcolor{Maroon}{\text{b}} \symup{\iota}\symup{z} \mathcolor{gray}{0}} - \mathcolor{gray}{\nabla^{\hat{2}}} {\mathcal{O}}^{\;\!\mathcolor{gray}{t}}_{\;\!\textcolor{Maroon}{\text{b}} \symup{\iota}\symup{z} \hat{2} \mathcolor{gray}{0}}~, \label{eq:D^(1)_0} \\
	{\mathcal{D}}^{\;\!\textcolor{Maroon}{(2)}\mathcolor{gray}{t}}_{\;\! \symup{\iota}\mathcolor{gray}{0}} &= -\hspace{0.2em} {\mathcal{O}}^{\;\!\mathcolor{gray}{t}}_{\;\!\textcolor{Maroon}{\text{b}} \symup{\iota} \symup{z}\symup{z} \mathcolor{gray}{0}}~, \label{eq:D^(2)_0}
\end{align}
\end{subequations}
及其下一(裸多极矩)层次\cite{OriginDependenceMaterial,langeCompletionMultipoleTheory2003,raabTransformedMultipoleTheory2005,grahamMultipoleSolutionMacroscopic2000}:
\begin{subequations} \label{eq:DH^(2-0)_0=}
\begin{align}
	D^{\;\!\mathcolor{gray}{t}}_{\;\! \symup{\iota}\mathcolor{gray}{z}} &= P^{\;\!\mathcolor{gray}{t}}_{\;\! \symup{\iota}\mathcolor{gray}{z}} - \mathcolor{gray}{\nabla^{\hat{1}}} Q^{\;\!\mathcolor{gray}{t}}_{\;\! \symup{\iota}\hat{1}\mathcolor{gray}{z}} + \mathcolor{gray}{\nabla^{\hat{1}}} \mathcolor{gray}{\nabla^{\hat{2}}} O^{\;\!\mathcolor{gray}{t}}_{\;\! \symup{\iota}\hat{1}\hat{2}\mathcolor{gray}{z}} - \cdots~, \label{eq:D^(0)=} \\
	{\mathcal{D}}^{\;\!\textcolor{Maroon}{(1)}\mathcolor{gray}{t}}_{\;\! \symup{\iota}\mathcolor{gray}{0}} &= Q^{\;\!\mathcolor{gray}{t}}_{\;\! \symup{\iota}\symup{z}\mathcolor{gray}{z}} - 2~ \mathcolor{gray}{\nabla^{\hat{1}}} O^{\;\!\mathcolor{gray}{t}}_{\;\! \symup{\iota}\hat{1}\symup{z}\mathcolor{gray}{z}} + \cdots~, \label{eq:D^(1)_0=} \\
	{\mathcal{D}}^{\;\!\textcolor{Maroon}{(2)}\mathcolor{gray}{t}}_{\;\! \symup{\iota}\mathcolor{gray}{0}} &= -\hspace{0.2em} O^{\;\!\mathcolor{gray}{t}}_{\;\! \symup{\iota}\symup{z}\symup{z}\mathcolor{gray}{z}}~, \label{eq:D^(2)_0=} \\[0.7em]
	H^{\;\!\mathcolor{gray}{t}}_{\;\! \symup{\iota}\mathcolor{gray}{z}} &= -\hspace{0.2em} M^{\;\!\mathcolor{gray}{t}}_{\;\! \symup{\iota} \mathcolor{gray}{z}} + \mathcolor{gray}{\nabla^{\hat{1}}} N^{\;\!\mathcolor{gray}{t}}_{\;\! \symup{\iota}\hat{1}\mathcolor{gray}{z}}~, \label{eq:H^(0)=} \\
	{\mathcal{H}}^{\;\!\textcolor{Maroon}{(1)}\mathcolor{gray}{t}}_{\;\! \symup{\iota}\mathcolor{gray}{0}} &= -\hspace{0.2em} N^{\;\!\mathcolor{gray}{t}}_{\;\! \symup{\iota}\symup{z}\mathcolor{gray}{0}}~. \label{eq:H^(1)_0=}
\end{align}
\end{subequations}

若将 $\bar{D}^{\;\!\mathcolor{gray}{t}}_{\;\!\mathcolor{gray}{z}}, \bar{H}^{\;\!\mathcolor{gray}{t}}_{\;\!\mathcolor{gray}{z}}$ 展开成类似 \bref{eq:EB-01}(而不是 \bref{eq:DH-01})的奇异场层次,即
\begin{subequations} \label{eq:DH-01'}
\begin{align}
	\hphantom{xxxxx} D^{\;\!\mathcolor{gray}{t}}_{\;\! \symup{\iota}\mathcolor{gray}{z}} &= \hspace{0.2em} &&\hspace{-4.5em}\leftindex_{\textcolor{Maroon}{\mathsfit{z}}} {\mathbb{1}}_{\mathcolor{gray}{z}} \leftindex^{\textcolor{Maroon}{\mathsfit{z}}} \;\! D^{\;\!\mathcolor{gray}{t}}_{\;\! \symup{\iota}\mathcolor{gray}{z}} &&\hspace{-4.5em}- \leftindex_{\textcolor{Maroon}{\mathsfit{z}}} \;\! \delta_{\mathcolor{gray}{z}} \leftindex^{\textcolor{Maroon}{\mathsfit{z}}} \;\!
	{\mathcal{D}}^{\;\!\textcolor{Maroon}{(1)}\mathcolor{gray}{t}}_{\;\! \symup{\iota}\mathcolor{gray}{z}} &&\hspace{-4.5em}- \leftindex_{\textcolor{Maroon}{\mathsfit{z}}} \;\! \delta'_{\mathcolor{gray}{z}} \leftindex^{\textcolor{Maroon}{\mathsfit{z}}} \;\! {\mathcal{D}}^{\;\!\textcolor{Maroon}{(2)}\mathcolor{gray}{t}}_{\;\! \symup{\iota}\mathcolor{gray}{z}} ~, \label{eq:D-01'} \\
	\hphantom{xxxxx} H^{\;\!\mathcolor{gray}{t}}_{\;\! \symup{\iota}\mathcolor{gray}{z}} &= \hspace{0.2em} &&\hspace{-4.5em}\leftindex_{\textcolor{Maroon}{\mathsfit{z}}} {\mathbb{1}}_{\mathcolor{gray}{z}} \leftindex^{\textcolor{Maroon}{\mathsfit{z}}} \;\! H^{\;\!\mathcolor{gray}{t}}_{\;\! \symup{\iota}\mathcolor{gray}{z}} &&\hspace{-4.5em}- \leftindex_{\textcolor{Maroon}{\mathsfit{z}}} \;\! \delta_{\mathcolor{gray}{z}} \leftindex^{\textcolor{Maroon}{\mathsfit{z}}} \;\!
	{\mathcal{H}}^{\;\!\textcolor{Maroon}{(1)}\mathcolor{gray}{t}}_{\;\! \symup{\iota}\mathcolor{gray}{z}} &&\hspace{-4.5em}~, \label{eq:H-01'}
\end{align}
\end{subequations}
那么对于电位移场,其(在宏观阶跃边界 $\mathcolor{gray}{z \mathcolor{black}{=} 0}$ 附近的)非零分量有
\begin{subequations} \label{eq:D^(2-0)_0=}
\begin{align}
	{\mathcal{D}}^{\;\!\textcolor{Maroon}{(2)}\mathcolor{gray}{t}}_{\;\! \symup{x} \mathcolor{gray}{0}} = &- O^{\;\!\mathcolor{gray}{t}}_{\;\! \symup{x}\symup{z}\symup{z}\mathcolor{gray}{0}}~, \label{eq:D^(2)_x0=} \\
	{\mathcal{D}}^{\;\!\textcolor{Maroon}{(2)}\mathcolor{gray}{t}}_{\;\! \symup{y} \mathcolor{gray}{0}} = &- O^{\;\!\mathcolor{gray}{t}}_{\;\! \symup{y}\symup{z}\symup{z}\mathcolor{gray}{0}}~; \label{eq:D^(2)_y0=} \\[1.0em]
	{\mathcal{D}}^{\;\!\textcolor{Maroon}{(1)}\mathcolor{gray}{t}}_{\;\! \symup{x} \mathcolor{gray}{0}} = &\hphantom{+} Q^{\;\!\mathcolor{gray}{t}}_{\;\! \symup{x}\symup{z}\mathcolor{gray}{z}} - 2~ \mathcolor{gray}{\nabla^\iota} O^{\;\!\mathcolor{gray}{t}}_{\;\! \symup{\iota}\symup{x}\symup{z}\mathcolor{gray}{z}} + \mathcolor{gray}{\nabla_x} O^{\;\!\mathcolor{gray}{t}}_{\;\! \symup{z}\symup{z}\symup{z}\mathcolor{gray}{0}}~, \label{eq:D^(1)_x0=} \\
	{\mathcal{D}}^{\;\!\textcolor{Maroon}{(1)}\mathcolor{gray}{t}}_{\;\! \symup{y} \mathcolor{gray}{0}} = &\hphantom{+} Q^{\;\!\mathcolor{gray}{t}}_{\;\! \symup{y}\symup{z}\mathcolor{gray}{z}} - 2~ \mathcolor{gray}{\nabla^\iota} O^{\;\!\mathcolor{gray}{t}}_{\;\! \symup{\iota}\symup{y}\symup{z}\mathcolor{gray}{z}} + \mathcolor{gray}{\nabla_y} O^{\;\!\mathcolor{gray}{t}}_{\;\! \symup{z}\symup{z}\symup{z}\mathcolor{gray}{0}}~, \label{eq:D^(1)_y0=} \\
	{\mathcal{D}}^{\;\!\textcolor{Maroon}{(1)}\mathcolor{gray}{t}}_{\;\! \symup{z} \mathcolor{gray}{0}} = &\hphantom{+} \mathcolor{gray}{\nabla_x} O^{\;\!\mathcolor{gray}{t}}_{\;\! \symup{x} \symup{z} \symup{z} \mathcolor{gray}{0}} + \mathcolor{gray}{\nabla_y} O^{\;\!\mathcolor{gray}{t}}_{\;\! \symup{y} \symup{z} \symup{z} \mathcolor{gray}{0}}~; \label{eq:D^(1)_z0=} \\[1.0em]
	D^{\;\!\mathcolor{gray}{t}}_{\;\! \symup{x}\mathcolor{gray}{0}} = &\hphantom{+} P^{\;\!\mathcolor{gray}{t}}_{\;\! \symup{x}\mathcolor{gray}{z}} + \mathcolor{gray}{\nabla_x} \left[ Q^{\;\!\mathcolor{gray}{t}}_{\;\! \symup{z} \symup{z} \mathcolor{gray}{0}} - 3~ \mathcolor{gray}{\nabla_x} O^{\;\!\mathcolor{gray}{t}}_{\;\! \symup{x} \symup{z} \symup{z} \mathcolor{gray}{0}} - 3~ \mathcolor{gray}{\nabla_y}  O^{\;\!\mathcolor{gray}{t}}_{\;\! \symup{y} \symup{z} \symup{z}  \mathcolor{gray}{0}} - \mathcolor{gray}{\nabla_z}  O^{\;\!\mathcolor{gray}{t}}_{\;\! \symup{z} \symup{z} \symup{z} \mathcolor{gray}{0}} \right] \label{eq:D_x0=} \\ &- \mathcolor{gray}{\nabla^\iota} \left( Q^{\;\!\mathcolor{gray}{t}}_{\;\! \symup{\iota}\symup{x}\mathcolor{gray}{z}} - \mathcolor{gray}{\nabla^{\hat{1}}} O^{\;\!\mathcolor{gray}{t}}_{\;\! \symup{\iota}\hat{1}\symup{x}\mathcolor{gray}{z}} \right) + \mathcolor{gray}{\nabla^t} \left( \mathcolor{gray}{\nabla^t} O^{\;\!\mathcolor{gray}{t}}_{\;\!\symup{x} \symup{z} \symup{z} \mathcolor{gray}{0}} + N^{\;\!\mathcolor{gray}{t}}_{\;\! \symup{y} \symup{z} \mathcolor{gray}{0}} \right) \big/ \symup{c}^2 ~, \\ 
	D^{\;\!\mathcolor{gray}{t}}_{\;\! \symup{y}\mathcolor{gray}{0}} = &\hphantom{+} P^{\;\!\mathcolor{gray}{t}}_{\;\! \symup{y}\mathcolor{gray}{z}} + \mathcolor{gray}{\nabla_y} \left[ Q^{\;\!\mathcolor{gray}{t}}_{\;\! \symup{z} \symup{z} \mathcolor{gray}{0}} - 3~ \mathcolor{gray}{\nabla_x} O^{\;\!\mathcolor{gray}{t}}_{\;\! \symup{x} \symup{z} \symup{z} \mathcolor{gray}{0}} - 3~ \mathcolor{gray}{\nabla_y}  O^{\;\!\mathcolor{gray}{t}}_{\;\! \symup{y} \symup{z} \symup{z} \mathcolor{gray}{0}} - \mathcolor{gray}{\nabla_z}  O^{\;\!\mathcolor{gray}{t}}_{\;\! \symup{z} \symup{z} \symup{z} \mathcolor{gray}{0}} \right] \label{eq:D_y0=} \\ &- \mathcolor{gray}{\nabla^\iota} \left( Q^{\;\!\mathcolor{gray}{t}}_{\;\! \symup{\iota}\symup{y}\mathcolor{gray}{z}} - \mathcolor{gray}{\nabla^{\hat{1}}} O^{\;\!\mathcolor{gray}{t}}_{\;\! \symup{\iota}\hat{1}\symup{y}\mathcolor{gray}{z}} \right) + \mathcolor{gray}{\nabla^t} \left( \mathcolor{gray}{\nabla^t} O^{\;\!\mathcolor{gray}{t}}_{\;\! \symup{y} \symup{z} \symup{z} \mathcolor{gray}{0}} - N^{\;\!\mathcolor{gray}{t}}_{\;\! \symup{x} \symup{z} \mathcolor{gray}{0}} \right) \big/ \symup{c}^2 ~, \\
	D^{\;\!\mathcolor{gray}{t}}_{\;\! \symup{z} \mathcolor{gray}{0}} = &\hphantom{+} {\sigma}^{\;\!\mathcolor{gray}{t}}_{\;\! \mathcolor{gray}{0}} + \left( \mathcolor{gray}{\nabla_x} Q^{\;\!\mathcolor{gray}{t}}_{\;\! \symup{x} \symup{z} \mathcolor{gray}{0}} + \mathcolor{gray}{\nabla_y} Q^{\;\!\mathcolor{gray}{t}}_{\;\! \symup{y} \symup{z} \mathcolor{gray}{0}} \right) \label{eq:D_z0=} \\ & - \left[ \mathcolor{gray}{\nabla_z} \left( \mathcolor{gray}{\nabla_x} O^{\;\!\mathcolor{gray}{t}}_{\;\! \symup{x} \symup{z} \symup{z} \mathcolor{gray}{0}} + \mathcolor{gray}{\nabla_y} O^{\;\!\mathcolor{gray}{t}}_{\;\! \symup{y} \symup{z} \symup{z} \mathcolor{gray}{0}} \right) + 4~ \mathcolor{gray}{\nabla_x} \mathcolor{gray}{\nabla_y} O^{\;\!\mathcolor{gray}{t}}_{\;\! \symup{x} \symup{y} \symup{z} \mathcolor{gray}{0}} \right. \\ & \left. + 2 \left( \mathcolor{gray}{\nabla_x^2} O^{\;\!\mathcolor{gray}{t}}_{\;\! \symup{x} \symup{x} \symup{z} \mathcolor{gray}{0}} + \mathcolor{gray}{\nabla_y^2} O^{\;\!\mathcolor{gray}{t}}_{\;\! \symup{y} \symup{y} \symup{z} \mathcolor{gray}{0}} \right) - \left( \mathcolor{gray}{\nabla_x^2} + \mathcolor{gray}{\nabla_y^2} \right) O^{\;\!\mathcolor{gray}{t}}_{\;\! \symup{z} \symup{z} \symup{z} \mathcolor{gray}{0}} \right] ~.
\end{align}
\end{subequations}
对于磁场,其(在宏观阶跃边界 $\mathcolor{gray}{z \mathcolor{black}{=} 0}$ 附近的)非零分量有
\begin{subequations} \label{eq:H^(2-0)_0=}
\begin{align}
	{\mathcal{H}}^{\;\!\textcolor{Maroon}{(1)}\mathcolor{gray}{t}}_{\;\! \symup{x} \mathcolor{gray}{0}} = &- \mathcolor{gray}{\nabla^t}
	O^{\;\!\mathcolor{gray}{t}}_{\;\!\symup{y}\symup{z}\symup{z}\mathcolor{gray}{0}}~, \label{eq:H^(1)_x0=} \\
	{\mathcal{H}}^{\;\!\textcolor{Maroon}{(1)}\mathcolor{gray}{t}}_{\;\! \symup{y} \mathcolor{gray}{0}} = &\hphantom{+} \mathcolor{gray}{\nabla^t}
	O^{\;\!\mathcolor{gray}{t}}_{\;\!\symup{x}\symup{z}\symup{z}\mathcolor{gray}{0}}~; \label{eq:H^(1)_y0=} \\[1.0em]
	H^{\;\!\mathcolor{gray}{t}}_{\;\! \symup{x}\mathcolor{gray}{0}} = & - \left( \mathcolor{gray}{\nabla^t} Q^{\;\!\mathcolor{gray}{t}}_{\;\! \symup{y} \symup{z} \mathcolor{gray}{0}} - 
	{\alpha}^{\;\!\mathcolor{gray}{t}}_{\;\! \symup{y}\mathcolor{gray}{0}} \right) + \mathcolor{gray}{\nabla_x} N^{\;\!\mathcolor{gray}{t}}_{\;\!\symup{z} \symup{z} \mathcolor{gray}{0}} \label{eq:H_x0=} \\ & + \mathcolor{gray}{\nabla^t} \left( 2~ \mathcolor{gray}{\nabla_x} O^{\;\!\mathcolor{gray}{t}}_{\;\! \symup{x} \symup{y} \symup{z} \mathcolor{gray}{0}} + 2~ \mathcolor{gray}{\nabla_y}  O^{\;\!\mathcolor{gray}{t}}_{\;\! \symup{y} \symup{y} \symup{z} \mathcolor{gray}{0}} + \mathcolor{gray}{\nabla_z} O^{\;\!\mathcolor{gray}{t}}_{\;\! \symup{y} \symup{z} \symup{z} \mathcolor{gray}{0}} - \mathcolor{gray}{\nabla_y} O^{\;\!\mathcolor{gray}{t}}_{\;\! \symup{z} \symup{z} \symup{z} \mathcolor{gray}{0}} \right)~, \\
	H^{\;\!\mathcolor{gray}{t}}_{\;\! \symup{y}\mathcolor{gray}{0}} = &\hphantom{+} \left( \mathcolor{gray}{\nabla^t} Q^{\;\!\mathcolor{gray}{t}}_{\;\! \symup{x} \symup{z} \mathcolor{gray}{0}} -
	{\alpha}^{\;\!\mathcolor{gray}{t}}_{\;\! \symup{x}\mathcolor{gray}{0}} \right) + \mathcolor{gray}{\nabla_y} N^{\;\!\mathcolor{gray}{t}}_{\;\! \symup{z} \symup{z} \mathcolor{gray}{0}} \label{eq:H_y0=} \\ & - \mathcolor{gray}{\nabla^t} \left( 2~ \mathcolor{gray}{\nabla_x} O^{\;\!\mathcolor{gray}{t}}_{\;\! \symup{x} \symup{x} \symup{z} \mathcolor{gray}{0}} + 2~ \mathcolor{gray}{\nabla_y}  O^{\;\!\mathcolor{gray}{t}}_{\;\! \symup{x} \symup{y} \symup{z} \mathcolor{gray}{0}} + \mathcolor{gray}{\nabla_z}  O^{\;\!\mathcolor{gray}{t}}_{\;\! \symup{x} \symup{z} \symup{z} \mathcolor{gray}{0}} - \mathcolor{gray}{\nabla_x}  O^{\;\!\mathcolor{gray}{t}}_{\;\! \symup{z} \symup{z} \symup{z} \mathcolor{gray}{0}} \right)~, \\
	H^{\;\!\mathcolor{gray}{t}}_{\;\! \symup{z} \mathcolor{gray}{0}} = & - M^{\;\!\mathcolor{gray}{t}}_{\;\! \symup{\iota} \mathcolor{gray}{z}} + \left[ 2~ \mathcolor{gray}{\nabla_x}
	N^{\;\!\mathcolor{gray}{t}}_{\;\! \symup{x}\symup{z} \mathcolor{gray}{0}} + 2~ \mathcolor{gray}{\nabla_y} N^{\;\!\mathcolor{gray}{t}}_{\;\! \symup{y}\symup{z} \mathcolor{gray}{0}} + \mathcolor{gray}{\nabla_z} N^{\;\!\mathcolor{gray}{t}}_{\;\! \symup{z}\symup{z} \mathcolor{gray}{0}} \right. \label{eq:H_z0=} \\ & \left. + \mathcolor{gray}{\nabla^t} \left( \mathcolor{gray}{\nabla_y}
	O^{\;\!\mathcolor{gray}{t}}_{\;\! \symup{x}\symup{z}\symup{z} \mathcolor{gray}{0}} - \mathcolor{gray}{\nabla_x}
	O^{\;\!\mathcolor{gray}{t}}_{\;\! \symup{y}\symup{z}\symup{z}\mathcolor{gray}{0}} \right) \right]~.
\end{align}
\end{subequations}
同样,从 \bref{eq:curl-DH-01-delta} 开始到 \bref{eq:H^(2-0)_0=},所有公式\Footnote{除了 \bref{eq:D^(0),eq:D^(0)=,eq:H^(0)=} 这些体项,即 ${\mathbb{1}}_{\mathcolor{gray}{z}} ~\textcolor{Maroon}{\text{项}}$ 中远离接触面 $\mathcolor{gray}{z \mathcolor{black}{=} 0}$ 的地方的对应项。}中的每一个源/场量 $X$,都需要继承并添补上 \bref{eq:curl-DH-01} 中的左上角介质标记 $\textcolor{Maroon}{\mathsfit{z}}$ 并对两侧半无限介质(在接触面 $\mathcolor{gray}{z \mathcolor{black}{=} 0}$ 处)乘以 $\leftindex_{\textcolor{Maroon}{\mathsfit{z}}} \;\! \delta_{\mathcolor{gray}{z}}$(或其他对应的 $\leftindex_{\textcolor{Maroon}{\mathsfit{z}}} \;\! \delta'_{\mathcolor{gray}{z}},\leftindex_{\textcolor{Maroon}{\mathsfit{z}}} \;\! \delta''_{\mathcolor{gray}{z}}$)并求和以成为 $\leftindex_{\textcolor{Maroon}{\mathsfit{z}}} \;\! \delta_{\mathcolor{gray}{z}} \leftindex^{\textcolor{Maroon}{\mathsfit{z}}} X$(或 $\leftindex_{\textcolor{Maroon}{\mathsfit{z}}} \;\! \delta'_{\mathcolor{gray}{z}} \leftindex^{\textcolor{Maroon}{\mathsfit{z}}} X,\leftindex_{\textcolor{Maroon}{\mathsfit{z}}} \;\! \delta''_{\mathcolor{gray}{z}} \leftindex^{\textcolor{Maroon}{\mathsfit{z}}} X$),才是有效的约束,和完整的边界条件。

上述 \bref{eq:D_z0=,eq:H_x0=,eq:H_y0=},结合 \bref{eq:1BC} 或 \bref{eq:2BC} 给出了以下深刻结论:在电四-磁偶极矩 $\bar{\bar{Q}}^{\;\!\mathcolor{gray}{t}}_{\;\!\mathcolor{gray}{z}}, \bar{M}^{\;\!\mathcolor{gray}{t}}_{\;\!\mathcolor{gray}{z}}$ 层次及以下,由场$+$源构成的两个辅助场:电位移场 $\bar{D}^{\;\!\mathcolor{gray}{t}}_{\;\!\mathcolor{gray}{z}}$ 的法向分量 $D^{\;\!\mathcolor{gray}{t}}_{\;\! \symup{z} \mathcolor{gray}{z}}$、磁场 $\bar{H}^{\;\!\mathcolor{gray}{t}}_{\;\!\mathcolor{gray}{z}}$ 的切向分量 $\bar{H}^{\;\!\mathcolor{gray}{t}}_{\;\! \symup{\rho} \mathcolor{gray}{z}}$(在 $\mathcolor{gray}{z \mathcolor{black}{=} 0^+}$ 和 $\mathcolor{gray}{z \mathcolor{black}{=} 0^-}$ 处的值)均不连续,即有 $\leftindex^{\textcolor{Maroon}{1}} D^{\;\!\mathcolor{gray}{t}}_{\;\! \symup{z} \mathcolor{gray}{0^+}} \neq \leftindex^{\textcolor{Maroon}{0}} D^{\;\!\mathcolor{gray}{t}}_{\;\! \symup{z} \mathcolor{gray}{0^-}}$ 以及 $\leftindex^{\textcolor{Maroon}{1}} {\bar{H}}^{\;\!\mathcolor{gray}{t}}_{\;\! \symup{\rho} \mathcolor{gray}{0^+}} \neq \leftindex^{\textcolor{Maroon}{0}} {\bar{H}}^{\;\!\mathcolor{gray}{t}}_{\;\! \symup{\rho} \mathcolor{gray}{0^-}}$。

由于矢量场 $\bar{A}^{\;\!\mathcolor{gray}{t}}_{\;\!\mathcolor{gray}{z}}$ 的旋度无源、标量场 $\phi^{\;\!\mathcolor{gray}{t}}_{\;\!\mathcolor{gray}{z}}$ 的梯度无旋,\bref{eq:DH-01} 中的 $\bar{D}^{\;\!\mathcolor{gray}{t}}_{\;\!\mathcolor{gray}{z}}, \bar{H}^{\;\!\mathcolor{gray}{t}}_{\;\!\mathcolor{gray}{z}}$ 可以分别加上 $\mathcolor{gray}{\bar{\nabla} \times} \bar{A}^{\;\!\mathcolor{gray}{t}}_{\;\!\mathcolor{gray}{z}}$、$\mathcolor{gray}{\bar{\nabla}} \phi^{\;\!\mathcolor{gray}{t}}_{\;\!\mathcolor{gray}{z}}$ 而不影响 \bref{eq:Div-D}、\bref{eq:Curl-H} 的成立,因此辅助场 $\bar{D}^{\;\!\mathcolor{gray}{t}}_{\;\!\mathcolor{gray}{z}}$ $~\left(~ + \mathcolor{gray}{\bar{\nabla} \times} \bar{A}^{\;\!\mathcolor{gray}{t}}_{\;\!\mathcolor{gray}{z}} ~\right)~$ 和 $\bar{H}^{\;\!\mathcolor{gray}{t}}_{\;\!\mathcolor{gray}{z}}$ $~\left(~ + \mathcolor{gray}{\bar{\nabla}} \phi^{\;\!\mathcolor{gray}{t}}_{\;\!\mathcolor{gray}{z}} ~\right)~$ 都不是唯一确定的\Footnote{类似标势与矢势也需要某种“规范”(如库伦/洛伦兹规范)才能确定一样。多极理论中的这种“规范”为:源的宏观表达式在选择的坐标原点的平移下需保持不变,也称本构张量的原点独立性。}。—— 这一额外的自由度可以被用于确定在微观上与坐标原点无关的平移不变源 $\bar{P}^{\;\!\mathcolor{gray}{t}}_{\;\!\mathcolor{gray}{z}},\bar{\bar{Q}}^{\;\!\mathcolor{gray}{t}}_{\;\!\mathcolor{gray}{z}},\bar{\bar{\bar{O}}}^{\;\!\mathcolor{gray}{t}}_{\;\!\mathcolor{gray}{z}} ; \bar{M}^{\;\!\mathcolor{gray}{t}}_{\;\!\mathcolor{gray}{z}}, \bar{\bar{N}}^{\;\!\mathcolor{gray}{t}}_{\;\!\mathcolor{gray}{z}}$ (关于基本场 $\bar{E}^{\;\!\mathcolor{gray}{t}}_{\;\!\mathcolor{gray}{z}}, \bar{B}^{\;\!\mathcolor{gray}{t}}_{\;\!\mathcolor{gray}{z}}$ 的本构关系)的唯一确定且正确的宏观表达式\cite{welterTranslationallyInvariantSemiclassical2013,delangeTranslationalInvariancePost2012,langeTransitionMicroscopicMacroscopic2012,langeMultipoleTheoryHehl2015,raabCommentOriginDependence2010a,OriginindependentCalculationQuadrupole}\Footnote{微观表达式却不是唯一确定的\cite{OriginDependenceMaterial}。}。

源(动)是因,(生)场为果。在 ${\mathbb{1}}_{\mathcolor{gray}{z}} ~\textcolor{Maroon}{\text{项}}$ 中,$\bar{D}^{\;\!\mathcolor{gray}{t}}_{\;\!\mathcolor{gray}{z}}, \bar{H}^{\;\!\mathcolor{gray}{t}}_{\;\!\mathcolor{gray}{z}}$ 由基本场 $\bar{E}^{\;\!\mathcolor{gray}{t}}_{\;\!\mathcolor{gray}{z}}, \bar{B}^{\;\!\mathcolor{gray}{t}}_{\;\!\mathcolor{gray}{z}}$ 和束缚源 $\bar{P}^{\;\!\mathcolor{gray}{t}}_{\;\!\mathcolor{gray}{z}},\bar{\bar{Q}}^{\;\!\mathcolor{gray}{t}}_{\;\!\mathcolor{gray}{z}},\bar{\bar{\bar{O}}}^{\;\!\mathcolor{gray}{t}}_{\;\!\mathcolor{gray}{z}} ; \bar{M}^{\;\!\mathcolor{gray}{t}}_{\;\!\mathcolor{gray}{z}}, \bar{\bar{N}}^{\;\!\mathcolor{gray}{t}}_{\;\!\mathcolor{gray}{z}}$ 构成,因此称其为“辅助场”,其本质为基本场和束缚源的混合。相比而言,基本场 $\bar{E}^{\;\!\mathcolor{gray}{t}}_{\;\!\mathcolor{gray}{z}}, \bar{B}^{\;\!\mathcolor{gray}{t}}_{\;\!\mathcolor{gray}{z}}$ 在其 ${\mathbb{1}}_{\mathcolor{gray}{z}} ~\textcolor{Maroon}{\text{项}}$ 中的值,在远离\Footnote{在接近接触面 $\mathcolor{gray}{z \mathcolor{black}{=} 0}$ 的薄层内,基本场 $\bar{E}^{\;\!\mathcolor{gray}{t}}_{\;\!\mathcolor{gray}{z}}, \bar{B}^{\;\!\mathcolor{gray}{t}}_{\;\!\mathcolor{gray}{z}}$“在两种介质中的差”仍是束缚源 $\bar{P}^{\;\!\mathcolor{gray}{t}}_{\;\!\mathcolor{gray}{z}},\bar{\bar{Q}}^{\;\!\mathcolor{gray}{t}}_{\;\!\mathcolor{gray}{z}},\bar{\bar{\bar{O}}}^{\;\!\mathcolor{gray}{t}}_{\;\!\mathcolor{gray}{z}} ; \bar{M}^{\;\!\mathcolor{gray}{t}}_{\;\!\mathcolor{gray}{z}}, \bar{\bar{N}}^{\;\!\mathcolor{gray}{t}}_{\;\!\mathcolor{gray}{z}}$ 的显示多项式函数,如 \bref{eq:E_x0==,eq:E_y0==,eq:E_z0==} 和 \bref{eq:B_x0==,eq:B_y0==,eq:B_z0==} 所示。}接触面 $\mathcolor{gray}{z \mathcolor{black}{=} 0}$ 处,不是束缚源 $\bar{P}^{\;\!\mathcolor{gray}{t}}_{\;\!\mathcolor{gray}{z}},\bar{\bar{Q}}^{\;\!\mathcolor{gray}{t}}_{\;\!\mathcolor{gray}{z}},\bar{\bar{\bar{O}}}^{\;\!\mathcolor{gray}{t}}_{\;\!\mathcolor{gray}{z}} ; \bar{M}^{\;\!\mathcolor{gray}{t}}_{\;\!\mathcolor{gray}{z}}, \bar{\bar{N}}^{\;\!\mathcolor{gray}{t}}_{\;\!\mathcolor{gray}{z}}$ 的显示多项式函数,但仍因受到体区域 \textcolor{Maroon}{Maxwell-Lorentz-Heaviside} \bref{eq:curl-E,eq:div-B,eq:div-E,eq:curl-B}的约束,而是这些束缚源的隐式函数。总之,$\bar{D}^{\;\!\mathcolor{gray}{t}}_{\;\!\mathcolor{gray}{z}}, \bar{H}^{\;\!\mathcolor{gray}{t}}_{\;\!\mathcolor{gray}{z}};\bar{E}^{\;\!\mathcolor{gray}{t}}_{\;\!\mathcolor{gray}{z}}, \bar{B}^{\;\!\mathcolor{gray}{t}}_{\;\!\mathcolor{gray}{z}}$ 均直/间接地是 $\bar{P}^{\;\!\mathcolor{gray}{t}}_{\;\!\mathcolor{gray}{z}},\bar{\bar{Q}}^{\;\!\mathcolor{gray}{t}}_{\;\!\mathcolor{gray}{z}},\bar{\bar{\bar{O}}}^{\;\!\mathcolor{gray}{t}}_{\;\!\mathcolor{gray}{z}} ; \bar{M}^{\;\!\mathcolor{gray}{t}}_{\;\!\mathcolor{gray}{z}}, \bar{\bar{N}}^{\;\!\mathcolor{gray}{t}}_{\;\!\mathcolor{gray}{z}}$ 的函数。

反过来,\bref{sec:maxwell} 整体(直到 \bref{ssec:DH-boundary} 末的这里),直接通过多项式组成辅助场 $\bar{D}^{\;\!\mathcolor{gray}{t}}_{\;\!\mathcolor{gray}{z}}, \bar{H}^{\;\!\mathcolor{gray}{t}}_{\;\!\mathcolor{gray}{z}}$,以及间接通过 \textcolor{Maroon}{Maxwell-Lorentz-Heaviside} \bref{eq:curl-E,eq:div-B,eq:div-E,eq:curl-B} 控制基本场 $\bar{E}^{\;\!\mathcolor{gray}{t}}_{\;\!\mathcolor{gray}{z}}, \bar{B}^{\;\!\mathcolor{gray}{t}}_{\;\!\mathcolor{gray}{z}}$ 的介质/束缚源部分 $\bar{P}^{\;\!\mathcolor{gray}{t}}_{\;\!\mathcolor{gray}{z}},\bar{\bar{Q}}^{\;\!\mathcolor{gray}{t}}_{\;\!\mathcolor{gray}{z}},\bar{\bar{\bar{O}}}^{\;\!\mathcolor{gray}{t}}_{\;\!\mathcolor{gray}{z}} ; \bar{M}^{\;\!\mathcolor{gray}{t}}_{\;\!\mathcolor{gray}{z}}, \bar{\bar{N}}^{\;\!\mathcolor{gray}{t}}_{\;\!\mathcolor{gray}{z}}$,暂时不写作 基本场 $\bar{E}^{\;\!\mathcolor{gray}{t}}_{\;\!\mathcolor{gray}{z}}, \bar{B}^{\;\!\mathcolor{gray}{t}}_{\;\!\mathcolor{gray}{z}}$ 或任何其他(引力、应力、温度)场的显示函数。

\marginLeft[-2.4em]{sec:constitutive}\section{\textcolor{Maroon}{Constitutive Relations} 本构关系:$\text{源} = f(\text{场})$}\label{sec:constitutive}

该节从经典/现代的多极理论视角,查看源关于场的 \textcolor{Maroon}{Constitutive Relations} 本构关系、非线性。——对应“物质告诉时空怎么弯曲,时空告诉物质怎么运动”的后半句:即“场告诉源怎么运动”。

\vspace*{-4.0em}

\marginLeft[-2.4em]{ssec:PMQN-nonlinear}\subsection{$\bar{P},\bar{M}$ 的线性、$\bar{\bar{Q}},\bar{\bar{N}}$ 的非线性}\label{ssec:PMQN-nonlinear}

材料的束缚源 ${\rho}^{\;\!\mathcolor{gray}{t}}_{\;\!\textcolor{Maroon}{\text{b}}\mathcolor{gray}{z}}, \bar{J}^{\;\!\mathcolor{gray}{t}}_{\;\!\textcolor{Maroon}{\text{b}}\mathcolor{gray}{z}};\bar{P}^{\;\!\mathcolor{gray}{t}}_{\;\!\mathcolor{gray}{z}},\bar{\bar{Q}}^{\;\!\mathcolor{gray}{t}}_{\;\!\mathcolor{gray}{z}},\bar{\bar{\bar{O}}}^{\;\!\mathcolor{gray}{t}}_{\;\!\mathcolor{gray}{z}} ; \bar{M}^{\;\!\mathcolor{gray}{t}}_{\;\!\mathcolor{gray}{z}}, \bar{\bar{N}}^{\;\!\mathcolor{gray}{t}}_{\;\!\mathcolor{gray}{z}}$ 部分,可能存在不含时 $\mathcolor{gray}{t}$ 的永久/自发/固有(宏观)极/磁化多极矩 $\bar{P}^{\;\!\textcolor{Maroon}{(0)}}_{\;\!\mathcolor{gray}{z}},\bar{\bar{Q}}^{\;\!\textcolor{Maroon}{(0)}}_{\;\!\mathcolor{gray}{z}},\bar{\bar{\bar{O}}}^{\;\!\textcolor{Maroon}{(0)}}_{\;\!\mathcolor{gray}{z}} ; \bar{M}^{\;\!\textcolor{Maroon}{(0)}}_{\;\!\mathcolor{gray}{z}}, \bar{\bar{N}}^{\;\!\textcolor{Maroon}{(0)}}_{\;\!\mathcolor{gray}{z}}$。即没有被施加外场时,其结构自身就具有的极/磁化多极矩,对应材料\Footnote{这里的“材料”具体指:原/离子=多电子体系,或晶格 or 分子 or 化学键\cite{boydNonlinearOptics2019} or 基团=多原/离子体系。}波函数的哈密顿量 $\bar{H}_0$(的本征态中的基态)。典型地如非中心对称晶体结构的铁电/铁磁材料,具有永久电/磁偶极矩$\bar{P}^{\;\!\textcolor{Maroon}{(0)}}_{\;\!\mathcolor{gray}{z}}, \bar{M}^{\;\!\textcolor{Maroon}{(0)}}_{\;\!\mathcolor{gray}{z}}$。中心对称的有机材料中的苯环结构、反铁电/反铁磁材料等,具有永久电/磁四极矩$\bar{\bar{Q}}^{\;\!\textcolor{Maroon}{(0)}}_{\;\!\mathcolor{gray}{z}};\bar{\bar{N}}^{\;\!\textcolor{Maroon}{(0)}}_{\;\!\mathcolor{gray}{z}}$。液晶、超材料、复合材料、多铁材料等可以同时具有永久电、磁多极矩$\bar{P}^{\;\!\textcolor{Maroon}{(0)}}_{\;\!\mathcolor{gray}{z}},\bar{\bar{Q}}^{\;\!\textcolor{Maroon}{(0)}}_{\;\!\mathcolor{gray}{z}},\bar{\bar{\bar{O}}}^{\;\!\textcolor{Maroon}{(0)}}_{\;\!\mathcolor{gray}{z}} ; \bar{M}^{\;\!\textcolor{Maroon}{(0)}}_{\;\!\mathcolor{gray}{z}}, \bar{\bar{N}}^{\;\!\textcolor{Maroon}{(0)}}_{\;\!\mathcolor{gray}{z}}; \cdots$。

材料的束缚源和自由源 ${\rho}^{\;\!\mathcolor{gray}{t}}_{\;\!\textcolor{Maroon}{\text{b}}\mathcolor{gray}{z}}, \bar{J}^{\;\!\mathcolor{gray}{t}}_{\;\!\textcolor{Maroon}{\text{b}}\mathcolor{gray}{z}};{\rho}^{\;\!\mathcolor{gray}{t}}_{\;\!\textcolor{Maroon}{\text{f}}\mathcolor{gray}{z}}, \bar{J}^{\;\!\mathcolor{gray}{t}}_{\;\!\textcolor{Maroon}{\text{f}}\mathcolor{gray}{z}}$ 中的每一个,均可能会产生(含时 $\mathcolor{gray}{t}$ 或不含时 $\mathcolor{gray}{t}$ 的)感生/诱导/响应电荷/流\cite{markelExternalInducedFree2018,raabMultipoleTheoryElectromagnetism2004,tsukermanPolarizationArbitraryCharge2021a}和极/磁化多极矩。即与外场相互作用时,材料哈密顿量 $\bar{H} = \bar{H}_0 + \bar{V}$ 中多出的(由外场控制的)势能项 $\Delta \bar{H} = \bar{V}$\cite{boydNonlinearOptics2019,raabMultipoleTheoryElectromagnetism2004},所对应的电荷体系(相对于无外场时的)额外重新分布。值得注意的是,受场影响的带电体系重新分布后的终态(可能含时 $\mathcolor{gray}{t}$ 且不是热力学平衡态),减去原始分布的初态之差 $\Delta \bar{H}$,相对于初态 $\bar{H}_0$ 不一定是微扰 $\bar{H}'$\cite{boydNonlinearOptics2019},即可能有 $\bar{H}_0 \ll \Delta \bar{H}$,此时体现为材料对场的线性/非线性共振、(饱和)吸收,及其所可能导致的电离\cite{boydNonlinearOptics2019}、结构损伤、极化反转、材料改性\cite{xuFemtosecondLaserWriting2022,weiExperimentalDemonstrationThreedimensional2018,xuThreedimensionalNonlinearPhotonic2018,keren-zurNewDimensionNonlinear2018}、电光声强耦合、电滞/磁滞现象等。

当外/主场、驱动场是直/交流(对应不含时$\mathcolor{gray}{t}$/含时$\mathcolor{gray}{t}$、静态/动态)时,材料束缚/自由源也会因外场的存在(or 变化)而变化,即因果/迟滞地被外/主场驱动而产生回/反/响应。它们作为次波源,辐射出新/次场,并与外/主场叠加(成辅助场 $\bar{D}^{\;\!\mathcolor{gray}{t}}_{\;\!\mathcolor{gray}{z}}, \bar{H}^{\;\!\mathcolor{gray}{t}}_{\;\!\mathcolor{gray}{z}}$,如果外/主场是基本场 $\bar{E}^{\;\!\mathcolor{gray}{t}}_{\;\!\mathcolor{gray}{z}}, \bar{B}^{\;\!\mathcolor{gray}{t}}_{\;\!\mathcolor{gray}{z}}$ 的话)。不论外场是静态的还是动态的,束缚/自由源对外场的响应,都可能是外场及其导数的非线性函数,甚至包含与外场(的函数)在时域和/或空域上的卷积,对应材料的因果/推迟和非局域响应。

外场可以是电磁/引力/应力/温度场(它们都与“力”有关)。因此,电磁源 ${\rho}^{\;\!\mathcolor{gray}{t}}_{\;\!\mathcolor{gray}{z}}, \bar{J}^{\;\!\mathcolor{gray}{t}}_{\;\!\mathcolor{gray}{z}}$ 的自变(场)量是复杂的,许多因素都可以触发并更新源的分布——以至于$\text{源} = f(\text{场})$即本构关系本身,就涵盖了理论和应用物理学中的大面积主题\Footnote{超导/量子/固体/凝聚态/半导体/天体/等离子体物理、应用磁学/超声探测、流体力学、激光/电/探针/化学加工、材料/统计/计算物理/化学等。},使得本文既无法从细节上面面俱到,也无法从大局上给出其统一形式。

在可以同时扮演驱动源角色的众多主动场中,本文把重心落脚在基本场 $\bar{E}^{\;\!\mathcolor{gray}{t}}_{\;\!\mathcolor{gray}{z}}, \bar{B}^{\;\!\mathcolor{gray}{t}}_{\;\!\mathcolor{gray}{z}}$ 上,并适当忽视引力/应力/温度场等带来的多物理场耦合效应。即,在将场作为自变量方面,本文(在本构关系中)只将源 ${\rho}^{\;\!\mathcolor{gray}{t}}_{\;\!\mathcolor{gray}{z}}, \bar{J}^{\;\!\mathcolor{gray}{t}}_{\;\!\mathcolor{gray}{z}}$ 视为基本场 $\bar{E}^{\;\!\mathcolor{gray}{t}}_{\;\!\mathcolor{gray}{z}}, \bar{B}^{\;\!\mathcolor{gray}{t}}_{\;\!\mathcolor{gray}{z}}$ 的(线性、非线性)函数,使得至少底层变量只包含 $\bar{E}^{\;\!\mathcolor{gray}{t}}_{\;\!\mathcolor{gray}{z}}, \bar{B}^{\;\!\mathcolor{gray}{t}}_{\;\!\mathcolor{gray}{z}}$:即,中间过程可以还同时是热/声/应力/温度(梯度)场的函数,但这些中间场变量,也仅/纯粹由 $\bar{E}^{\;\!\mathcolor{gray}{t}}_{\;\!\mathcolor{gray}{z}}, \bar{B}^{\;\!\mathcolor{gray}{t}}_{\;\!\mathcolor{gray}{z}}$ 引起或导致,即共享相同的最多 2 个底层场量 $\bar{E}^{\;\!\mathcolor{gray}{t}}_{\;\!\mathcolor{gray}{z}}, \bar{B}^{\;\!\mathcolor{gray}{t}}_{\;\!\mathcolor{gray}{z}}$,也即最底层的驱动力源与 $\bar{E}^{\;\!\mathcolor{gray}{t}}_{\;\!\mathcolor{gray}{z}}, \bar{B}^{\;\!\mathcolor{gray}{t}}_{\;\!\mathcolor{gray}{z}}$ 以外的其他场量无关。

电磁源 ${\rho}^{\;\!\mathcolor{gray}{t}}_{\;\!\mathcolor{gray}{z}}, \bar{J}^{\;\!\mathcolor{gray}{t}}_{\;\!\mathcolor{gray}{z}}$ 中,束缚源 ${\rho}^{\;\!\mathcolor{gray}{t}}_{\;\!\textcolor{Maroon}{\text{b}}\mathcolor{gray}{z}}, \bar{J}^{\;\!\mathcolor{gray}{t}}_{\;\!\textcolor{Maroon}{\text{b}}\mathcolor{gray}{z}}$ 的各阶极化/磁化多极矩 $\bar{P}^{\;\!\mathcolor{gray}{t}}_{\;\!\mathcolor{gray}{z}},\bar{\bar{Q}}^{\;\!\mathcolor{gray}{t}}_{\;\!\mathcolor{gray}{z}},\bar{\bar{\bar{O}}}^{\;\!\mathcolor{gray}{t}}_{\;\!\mathcolor{gray}{z}} ; \bar{M}^{\;\!\mathcolor{gray}{t}}_{\;\!\mathcolor{gray}{z}}, \bar{\bar{N}}^{\;\!\mathcolor{gray}{t}}_{\;\!\mathcolor{gray}{z}}$ 均分别有自己的(关于 $\bar{E}^{\;\!\mathcolor{gray}{t}}_{\;\!\mathcolor{gray}{z}}, \bar{B}^{\;\!\mathcolor{gray}{t}}_{\;\!\mathcolor{gray}{z}}$ 的)线性、非线性(函数),即“多极矩的阶”与其“非线性的阶”是无关的,正如其也与(源的)“奇异层次的阶”无关一样(见 \bref{ssec:step-delta}):这是三个独立的自由度,三者两两无关。

在所有的束缚/自由源 ${\rho}^{\;\!\mathcolor{gray}{t}}_{\;\!\textcolor{Maroon}{\text{f}}\mathcolor{gray}{z}}, \bar{J}^{\;\!\mathcolor{gray}{t}}_{\;\!\textcolor{Maroon}{\text{f}}\mathcolor{gray}{z}};{\rho}^{\;\!\mathcolor{gray}{t}}_{\;\!\textcolor{Maroon}{\text{b}}\mathcolor{gray}{z}}, \bar{J}^{\;\!\mathcolor{gray}{t}}_{\;\!\textcolor{Maroon}{\text{b}}\mathcolor{gray}{z}};\bar{P}^{\;\!\mathcolor{gray}{t}}_{\;\!\mathcolor{gray}{z}},\bar{\bar{Q}}^{\;\!\mathcolor{gray}{t}}_{\;\!\mathcolor{gray}{z}},\bar{\bar{\bar{O}}}^{\;\!\mathcolor{gray}{t}}_{\;\!\mathcolor{gray}{z}} ; \bar{M}^{\;\!\mathcolor{gray}{t}}_{\;\!\mathcolor{gray}{z}}, \bar{\bar{N}}^{\;\!\mathcolor{gray}{t}}_{\;\!\mathcolor{gray}{z}}$ 中,该 \bref{ssec:PMQN-nonlinear} 适合从读者熟悉的感应\Footnote{永久 $\bar{P}^{\;\!\textcolor{Maroon}{(0)}}_{\;\!\mathcolor{gray}{z}}$ 对麦氏方程组的解的影响,与直流/静态的感应 $\bar{P}^{\;\!}_{\;\!\mathcolor{gray}{z}}$ 对外场的影响类似,因此无需特殊考虑 $\bar{P}^{\;\!\textcolor{Maroon}{(0)}}_{\;\!\mathcolor{gray}{z}}$。}电偶极矩体密度\Footnote{这里的体密度并不对应 ${\mathbb{1}}_{\mathcolor{gray}{z}} ~\textcolor{Maroon}{\text{项}}$ or 体项:只能针对源 ${\rho}^{\;\!\mathcolor{gray}{t}}_{\;\!\mathcolor{gray}{z}}, \bar{J}^{\;\!\mathcolor{gray}{t}}_{\;\!\mathcolor{gray}{z}}$ 或场 $\bar{E}^{\;\!\mathcolor{gray}{t}}_{\;\!\mathcolor{gray}{z}}, \bar{B}^{\;\!\mathcolor{gray}{t}}_{\;\!\mathcolor{gray}{z}};\bar{D}^{\;\!\mathcolor{gray}{t}}_{\;\!\mathcolor{gray}{z}}, \bar{H}^{\;\!\mathcolor{gray}{t}}_{\;\!\mathcolor{gray}{z}}$ 提奇异层次,多极矩本身没有“表面项”或“体项”,只是组成了源和场的“表面项”和“体项”,见 \bref{ssec:step-delta}。} $\bar{P}^{\;\!\mathcolor{gray}{t}}_{\;\!\mathcolor{gray}{z}}$(关于 $\bar{E}^{\;\!\mathcolor{gray}{t}}_{\;\!\mathcolor{gray}{z}}, \bar{B}^{\;\!\mathcolor{gray}{t}}_{\;\!\mathcolor{gray}{z}}$ 的) 线性、非线性(函数)写起。为包含共振、强场、等离子体、多物理场耦合所带来的非线性,$\bar{P}^{\;\!\mathcolor{gray}{t}}_{\;\!\mathcolor{gray}{z}} \left( \bar{E}^{\;\!\mathcolor{gray}{t}}_{\;\!\mathcolor{gray}{z}}, \bar{B}^{\;\!\mathcolor{gray}{t}}_{\;\!\mathcolor{gray}{z}} \right)$ 的表达式最多只能在数学形式上写作 $P^{\;\!\mathcolor{gray}{t}}_{\;\! \symup{\iota}\mathcolor{gray}{z}} \left( \mathcolor{gray}{\nabla^t}, \mathcolor{gray}{\nabla_{\hat{1}}} ; E^{\;\!\mathcolor{gray}{t}}_{\;\! \hat{2}\mathcolor{gray}{z}}, B^{\;\!\mathcolor{gray}{t}}_{\;\! \hat{3}\mathcolor{gray}{z}} \right)$ 而无法提供更多的微观或宏观层面的信息量;同时,$P^{\;\!\mathcolor{gray}{t}}_{\;\! \symup{\iota}\mathcolor{gray}{z}}$ 中的各线性、非线性项中的材料学常数(各阶极化率),即“系统状态”如原子能级、分子振/转动能级、电子云晶格场离子声子格波能带,也将同样与 $\bar{E}^{\;\!\mathcolor{gray}{t}}_{\;\!\mathcolor{gray}{z}}, \bar{B}^{\;\!\mathcolor{gray}{t}}_{\;\!\mathcolor{gray}{z}}$ 的各分量的各阶时空导数有关,并有自己的$\text{极化率} = f(\text{场})$本构关系。然而由于其线性项的各阶极化率本身就与 $\bar{E}^{\;\!\mathcolor{gray}{t}}_{\;\!\mathcolor{gray}{z}}, \bar{B}^{\;\!\mathcolor{gray}{t}}_{\;\!\mathcolor{gray}{z}}$ 有关,而使得该线性项总体(=极化率$\cdot$场)自带非线性,以至于抹去了剩余对应非线性项的存在意义,并到头来最终抹去了线性项和一阶极化率本身的存在意义\cite{boydNonlinearOptics2019},直至只剩下 $P^{\;\!\mathcolor{gray}{t}}_{\;\! \symup{\iota}\mathcolor{gray}{z}} \left( \mathcolor{gray}{\nabla^t}, \mathcolor{gray}{\nabla_{\hat{1}}} ; E^{\;\!\mathcolor{gray}{t}}_{\;\! \hat{2}\mathcolor{gray}{z}}, B^{\;\!\mathcolor{gray}{t}}_{\;\! \hat{3}\mathcolor{gray}{z}} \right)$ 表达式整体。

当然,以激光加工为例,各阶极化率(物质结构)是在激光加工过程中被$\text{极化率} = f(\text{场})$本构关系改变的,但对 $P^{\;\!\mathcolor{gray}{t}}_{\;\! \symup{\iota}\mathcolor{gray}{z}}$ 的评估是在激光加工结束后,再对改变后的晶格结构终态,所对应的 $P'^{\;\!\mathcolor{gray}{t}}_{\;\! \symup{\iota}\mathcolor{gray}{z}}$ 进行计算的——当这两个过程\Footnote{过程 1:晶体结构被强外场作用时/后发生永久改变,该过程晶格结构的热力学转变,关于强场的函数。过程 2:改变后的晶格结构所对应的 $P'^{\;\!\mathcolor{gray}{t}}_{\;\! \symup{\iota}\mathcolor{gray}{z}}$,关于弱场的函数。}分开时,可以独立对各阶极化率构建本构关系,否则不如直接对 $P^{\;\!\mathcolor{gray}{t}}_{\;\! \symup{\iota}\mathcolor{gray}{z}}$ 构建本构关系。此外,$P^{\;\!\mathcolor{gray}{t}}_{\;\! \symup{\iota}\mathcolor{gray}{z}} \left( \mathcolor{gray}{\nabla^t}, \mathcolor{gray}{\nabla_{\hat{1}}} ; E^{\;\!\mathcolor{gray}{t}}_{\;\! \hat{2}\mathcolor{gray}{z}}, B^{\;\!\mathcolor{gray}{t}}_{\;\! \hat{3}\mathcolor{gray}{z}} \right)$ 表达式本身还有个问题:它的空间色散/非局域项,只能通过空域偏导 $\mathcolor{gray}{\nabla_{\hat{1}}}$ 体现,无法通过空域卷积/对不同地点处的乘积求和体现。

上述材料在与外场作用后,即使撤去外场,材料的终态相对初态性质改变了的系统,属于非线性时变系统:对于不同时刻的相同驱动场输入,材料可能有不同的响应输出,对应的 $\text{源} = f(\text{场})$ 本构关系 $P^{\;\!\mathcolor{gray}{t}}_{\;\! \symup{\iota}\mathcolor{gray}{z}} \left( \mathcolor{gray}{\nabla^t}, \mathcolor{gray}{\nabla_{\hat{1}}} ; E^{\;\!\mathcolor{gray}{t}}_{\;\! \hat{2}\mathcolor{gray}{z}}, B^{\;\!\mathcolor{gray}{t}}_{\;\! \hat{3}\mathcolor{gray}{z}} \right)$ 以及 $\text{材料学常数} = f(\text{场})$ 本构关系,均没有一般的表达式。因此,为得到可分析的对象,本文主要考虑非线性时不变系统\cite{zalevskyOpticalImplementationSecondorder2001}\Footnote{系统的响应输出,与系统状态、输入时刻无关(时间平移不变),只与其他输入参数有关的非线性系统。},即对于不同时刻的相同驱动场输入,材料的响应输出是相同的系统,也即材料电子云分布/晶格场状态在撤去外场后等于施加外场前的系统\Footnote{但允许在外场持续作用时,材料常数的弹性强烈变化,如拉比振荡。然而一旦驱动场被撤走,材料常数/体系波函数必须复原(以迎接下一次驱动)。复原的时间既可以很短,如电磁诱导透明、受激拉曼散射;也可能很长甚至可调,如光折变效应;也可以根本不复原,如激光直写/雕刻或加工。}。然而即使只考虑非线性时不变\Footnote{变与不变,需要指定起终点/态。这里指的是材料受到驱动前(的平衡态)为始态,驱动源被撤销、并且材料恢复平衡态后(的态)为终态。}系统,强场、强耦合、非微扰的情况下,在材料与场的强相互作用的过程中,$P^{\;\!\mathcolor{gray}{t}}_{\;\! \symup{\iota}\mathcolor{gray}{z}} \left( \mathcolor{gray}{\nabla^t}, \mathcolor{gray}{\nabla_{\hat{1}}} ; E^{\;\!\mathcolor{gray}{t}}_{\;\! \hat{2}\mathcolor{gray}{z}}, B^{\;\!\mathcolor{gray}{t}}_{\;\! \hat{3}\mathcolor{gray}{z}} \right)$ 以及 $\text{材料学常数} = f(\text{场})$ 仍不一定能展成收敛的级数形式,典型地如非线性(如包含电致伸缩项的)压电方程、非线性(如包含太赫兹驱动场和阻尼/晶格振动耗散项的)黄昆方程\cite{wuqiangShouJiShengZiJiHuaJiYuanYuTaiHeZiGuangWuLiTeYao2024}、光学双稳态\cite{boydNonlinearOptics2019}等。同样,本文也不考虑上述无法用 Volterra 级数\cite{pintoExactVolterraseriesComputation1982,shenNonlinearOpticalSusceptibilities2001}描述的、材料学常数与外场(瞬时弹性)相关的非线性时不变系统。

本文将自身局限于所有可以用“材料学常数与外场无关的 Volterra 级数\cite{pintoExactVolterraseriesComputation1982,shenNonlinearOpticalSusceptibilities2001}”描述的非线性光学现象。此时 $P^{\;\!\mathcolor{gray}{t}}_{\;\! \symup{\iota}\mathcolor{gray}{z}} \left( \mathcolor{gray}{\nabla^t}, \mathcolor{gray}{\nabla_{\hat{1}}} ; E^{\;\!\mathcolor{gray}{t}}_{\;\! \hat{2}\mathcolor{gray}{z}}, B^{\;\!\mathcolor{gray}{t}}_{\;\! \hat{3}\mathcolor{gray}{z}} \right)$ 可以进一步写为\cite{teixeiraOpticalTransmissionModeling2013,andreasczylwikNonlinearSystemModeling1986,shenNonlinearOpticalSusceptibilities2001,zalevskyOpticalImplementationSecondorder2001,zhangNonlinearQuantumInputoutput2014}
\begin{subequations}
\begin{align}
	P^{\;\!\mathcolor{gray}{t}}_{\;\! \symup{\iota}\mathcolor{gray}{\bar{r}}} &= \mathcolor{gray}{\mathcal F_{\bar{x}}^{-1}} \left[ P^{\;\! \mathcolor{gray}{\omega}}_{\;\! \symup{\iota}\mathcolor{gray}{\bar{k}}} \right] = P^{\;\! \textcolor{Maroon}{(1)} \mathcolor{gray}{t}}_{\;\! \symup{\iota}\mathcolor{gray}{\bar{r}}} + P^{\;\! \textcolor{Maroon}{(2)} \mathcolor{gray}{t}}_{\;\! \symup{\iota}\mathcolor{gray}{\bar{r}}} + P^{\;\! \textcolor{Maroon}{(3)} \mathcolor{gray}{t}}_{\;\! \symup{\iota}\mathcolor{gray}{\bar{r}}} + \cdots \label{P_barx_nonlinear} \\ &= {\symup{\varepsilon_0}} \left\{ \chi^{\;\! \mathcolor{gray}{t} \hat{1}}_{\;\! \symup{\iota} \mathcolor{gray}{\bar{r}} \textcolor{Maroon}{(1)}} ~\mathcolor{gray}{\widetilde \circledast}~ E^{\;\!\mathcolor{gray}{t}}_{\;\! \hat{1} \mathcolor{gray}{\bar{r}}} + \chi^{\;\! \mathcolor{gray}{t} \hat{1} \hat{2}}_{\;\! \symup{\iota} \mathcolor{gray}{\bar{r}} \textcolor{Maroon}{(2)}}~\mathcolor{gray}{\widetilde \circledast}\left( E^{\;\!\mathcolor{gray}{t}}_{\;\! \hat{1} \mathcolor{gray}{\bar{r}}} E^{\;\!\mathcolor{gray}{t}}_{\;\! \hat{2} \mathcolor{gray}{\bar{r}}} \right) \right. \\ &+ \left. \chi^{\;\! \mathcolor{gray}{t} \hat{1} \hat{2} \hat{3}}_{\;\! \symup{\iota} \mathcolor{gray}{\bar{r}} \textcolor{Maroon}{(3)}}~\mathcolor{gray}{\widetilde \circledast}\left( E^{\;\!\mathcolor{gray}{t}}_{\;\! \hat{1} \mathcolor{gray}{\bar{r}}} E^{\;\!\mathcolor{gray}{t}}_{\;\! \hat{2} \mathcolor{gray}{\bar{r}}} E^{\;\!\mathcolor{gray}{t}}_{\;\! \hat{3} \mathcolor{gray}{\bar{r}}} \right) + \cdots \right\}~,
\end{align}
\end{subequations}
其中,$\mathcolor{gray}{\widetilde \circledast}$ 在这里定义为 $\mathcolor{gray}{\bar{x}} := \left( \mathcolor{gray}{t}, \mathcolor{gray}{\bar{r}} \right)$ 域的 4 维卷积算符\Footnote{严格地说,$\mathcolor{gray}{\bar{x}} := \left( \mathcolor{gray}{t},~ ^*{\mathcolor{gray}{\bar{r}}}^\top \right)^\top$ 还包含 \textit{*args} 解包 $^*\left( \cdot \right)$、转置 $^\top$ 等操作(的配合),以将其定义为标准列向量。};对应地,定义了 4 维时空域 $\mathcolor{gray}{\bar{x}} \in \mathcolor{gray}{\bar{\mathbb{R}}_{3+1}}$ 中的傅立叶正 $\mathcolor{gray}{\mathcal F_{\bar{\kappa}}}$、逆 $\mathcolor{gray}{\mathcal F_{\bar{x}}^{-1}}$ 变换对
\begin{subequations} \label{eq:FT-xkappa}
\begin{align}
	\mathcolor{gray}{\mathcal F_{\bar{\kappa}}} \left[ \cdot \right] &:= \mathcolor{gray}{\mathcal F_{\omega}^{-1}} \left[ \mathcolor{gray}{\mathcal F_{\bar{k}}} \left[ \cdot \right] \right] = \mathcolor{gray}{\mathcal F_{\bar{k}}} \left[ \mathcolor{gray}{\mathcal F_{\omega}^{-1}} \left[ \cdot \right] \right] ~, \label{eq:FT-kappa} \\
	\mathcolor{gray}{\mathcal F_{\bar{x}}^{-1}} \left[ \cdot \right] &:= \mathcolor{gray}{\mathcal F_{t}} \left[ \mathcolor{gray}{\mathcal F_{\bar{r}}^{-1}} \left[ \cdot \right] \right] = \mathcolor{gray}{\mathcal F_{\bar{r}}^{-1}} \left[ \mathcolor{gray}{\mathcal F_{t}} \left[ \cdot \right] \right] ~. \label{eq:IFT-x}
\end{align}
\end{subequations}
其中,定义了 1 维时域 $\mathcolor{gray}{t} \in \mathcolor{gray}{\mathbb{R}}$ 中的傅立叶正 $\mathcolor{gray}{\mathcal F_{t}}$、逆 $\mathcolor{gray}{\mathcal F_{\omega}^{-1}}$ 变换对\Footnote{时域傅立叶变换的核,需要与空域的共轭,这样正频率正波矢才对应前向行波\cite{mcleodVectorFourierOptics2014}。}
\begin{subequations} \label{eq:FT-tw}
\begin{align}
	\mathcolor{gray}{\mathcal F_{t}} \left[ \cdot \right] &:= \frac{ 1 }{ 2\pi } \mathcolor{gray}{\int_{-\infty}^{+\infty}} \cdot~ \mathbb{e}^{-\mathbb{i}\mathcolor{gray}{\omega} \mathcolor{gray}{t}} \mathbb{d}\mathcolor{gray}{\omega} ~, \label{eq:FT-t} \\
	\mathcolor{gray}{\mathcal F_{\omega}^{-1}} \left[ \cdot \right] &:= \hphantom{\frac{ 1 }{ 2\pi }} \mathcolor{gray}{\int_{-\infty}^{+\infty}} \cdot~ \mathbb{e}^{\mathbb{i}\mathcolor{gray}{\omega} \mathcolor{gray}{t}} \hphantom{^-} \mathbb{d}\mathcolor{gray}{t} ~. \label{eq:IFT-w}
\end{align}
\end{subequations}
以及 3 维空域 $\mathcolor{gray}{\bar{r}} \in \mathcolor{gray}{\bar{\mathbb{R}}_{3}}$ 中的傅立叶正 $\mathcolor{gray}{\mathcal F_{\bar{k}}}$、逆 $\mathcolor{gray}{\mathcal F_{\bar{r}}^{-1}}$ 变换对
\begin{subequations} \label{eq:FT-rk}
\begin{align}
	\mathcolor{gray}{\mathcal F_{\bar{k}}} \left[ \cdot \right] &:= \frac{ 1 }{ \left( 2\pi \right)^3 } \mathcolor{gray}{\iiint_{-\infty}^{+\infty}} \cdot~ \mathbb{e}^{-\mathbb{i}\mathcolor{gray}{\bar{k}} \cdot \mathcolor{gray}{\bar{r}}} \mathbb{d}\mathcolor{gray}{\bar{r}} ~, \label{eq:FT-k} \\
	\mathcolor{gray}{\mathcal F_{\bar{r}}^{-1}} \left[ \cdot \right] &:= \hphantom{\frac{ 1 }{ \left( 2\pi \right)^3 }} \mathcolor{gray}{\iiint_{-\infty}^{+\infty}} \cdot~ \mathbb{e}^{\mathbb{i}\mathcolor{gray}{\bar{k}} \cdot \mathcolor{gray}{\bar{r}}} \hphantom{^-} \mathbb{d}\mathcolor{gray}{\bar{k}} ~. \label{eq:IFT-r}
\end{align}
\end{subequations}
注意,这里的 $\mathcolor{gray}{\bar{k}} := \left( \mathcolor{gray}{k_{\symup{x}}},~ \mathcolor{gray}{k_{\symup{y}}},~ \mathcolor{gray}{k_{\symup{z}}} \right)^\top \asymp \left( \mathcolor{gray}{\bar{k}_{\symup{\rho}}},~ \mathcolor{gray}{k_{\symup{z}}} \right)$\Footnote{符号 $\asymp$ 表示:“含义相同,尽管形式(数学符号表达式)不同”,也即“指代同一对象”。} 不是与波长相关的波矢 $\bar{k}^{\mathcolor{gray}{\omega}} := \left( \mathcolor{gray}{k_{\symup{x}}},~ \mathcolor{gray}{k_{\symup{y}}},~ k_{\symup{z}}^{\mathcolor{gray}{\omega}} \right)^\top \asymp \left( \mathcolor{gray}{\bar{k}_{\symup{\rho}}},~ k_{\symup{z}}^{\mathcolor{gray}{\omega}} \right)$,它作为(场量的)自变量,是频率 $\mathcolor{gray}{\omega}$ 无关的(无色散)3 维倒空间的向径。它与同样作为自变量的 3 维实空间的向径 $\mathcolor{gray}{\bar{r}}$ 等价。

对 \bref{P_barx_nonlinear} 执行 4 维傅立叶(正)变换 $\mathcolor{gray}{\mathcal F_{\bar{\kappa}}}$ 即 \bref{eq:FT-kappa},得到
\begin{subequations}
\begin{align}
	P^{\;\!\mathcolor{gray}{\omega}}_{\;\! \symup{\iota}\mathcolor{gray}{\bar{k}}} &= \mathcolor{gray}{\mathcal F_{\bar{\kappa}}} \left[ P^{\;\! \mathcolor{gray}{t}}_{\;\! \symup{\iota}\mathcolor{gray}{\bar{r}}} \right] = P^{\;\! \textcolor{Maroon}{(1)} \mathcolor{gray}{\omega}}_{\;\! \symup{\iota}\mathcolor{gray}{\bar{k}}} + P^{\;\! \textcolor{Maroon}{(2)} \mathcolor{gray}{\omega}}_{\;\! \symup{\iota}\mathcolor{gray}{\bar{k}}} + P^{\;\! \textcolor{Maroon}{(3)} \mathcolor{gray}{\omega}}_{\;\! \symup{\iota}\mathcolor{gray}{\bar{k}}} + \cdots \\ &= {\symup{\varepsilon_0}} \left\{ \chi^{\;\! \mathcolor{gray}{\omega} \hat{1}}_{\;\! \symup{\iota} \mathcolor{gray}{\bar{k}} \textcolor{Maroon}{(1)}} \mathcolor{gray}{\mathcal F_{\bar{\kappa}}} \left[ E^{\;\!\mathcolor{gray}{t}}_{\;\! \hat{1} \mathcolor{gray}{\bar{r}}} \right] + \chi^{\;\! \mathcolor{gray}{\omega} \hat{1} \hat{2}}_{\;\! \symup{\iota} \mathcolor{gray}{\bar{k}} \textcolor{Maroon}{(2)}} \mathcolor{gray}{\mathcal F_{\bar{\kappa}}} \left[ E^{\;\!\mathcolor{gray}{t}}_{\;\! \hat{1} \mathcolor{gray}{\bar{r}}} E^{\;\!\mathcolor{gray}{t}}_{\;\! \hat{2} \mathcolor{gray}{\bar{r}}} \right] \right. \label{P_barkappa_nonlinear} \\ &+ \left. \chi^{\;\! \mathcolor{gray}{\omega} \hat{1} \hat{2} \hat{3}}_{\;\! \symup{\iota} \mathcolor{gray}{\bar{k}} \textcolor{Maroon}{(3)}} \mathcolor{gray}{\mathcal F_{\bar{\kappa}}} \left[ E^{\;\!\mathcolor{gray}{t}}_{\;\! \hat{1} \mathcolor{gray}{\bar{r}}} E^{\;\!\mathcolor{gray}{t}}_{\;\! \hat{2} \mathcolor{gray}{\bar{r}}} E^{\;\!\mathcolor{gray}{t}}_{\;\! \hat{3} \mathcolor{gray}{\bar{r}}} \right] + \cdots \right\}
	\\ &= {\symup{\varepsilon_0}} \left\{ \chi^{\;\! \mathcolor{gray}{\omega} \hat{1}}_{\;\! \symup{\iota} \mathcolor{gray}{\bar{k}} \textcolor{Maroon}{(1)}} E^{\;\!\mathcolor{gray}{\omega}}_{\;\! \hat{1} \mathcolor{gray}{\bar{k}}} + \chi^{\;\! \mathcolor{gray}{\omega} \hat{1} \hat{2}}_{\;\! \symup{\iota} \mathcolor{gray}{\bar{k}} \textcolor{Maroon}{(2)}} \left( E^{\;\!\mathcolor{gray}{\omega}}_{\;\! \hat{1} \mathcolor{gray}{\bar{k}}} ~\mathcolor{gray}{\widetilde \circledast}~ E^{\;\!\mathcolor{gray}{\omega}}_{\;\! \hat{2} \mathcolor{gray}{\bar{k}}} \right) \right. \\ &+ \left. \chi^{\;\! \mathcolor{gray}{\omega} \hat{1} \hat{2} \hat{3}}_{\;\! \symup{\iota} \mathcolor{gray}{\bar{k}} \textcolor{Maroon}{(3)}} \left( E^{\;\!\mathcolor{gray}{\omega}}_{\;\! \hat{1} \mathcolor{gray}{\bar{k}}} ~\mathcolor{gray}{\widetilde \circledast}~ E^{\;\!\mathcolor{gray}{\omega}}_{\;\! \hat{2} \mathcolor{gray}{\bar{k}}} ~\mathcolor{gray}{\widetilde \circledast}~ E^{\;\!\mathcolor{gray}{\omega}}_{\;\! \hat{3} \mathcolor{gray}{\bar{k}}} \right) + \cdots \right\}~,
\end{align}
\end{subequations}
注意到 $\chi^{\;\! \mathcolor{gray}{\omega} \hat{1}}_{\;\! \symup{\iota} \mathcolor{gray}{\bar{k}} \textcolor{Maroon}{(1)}}$ 与 $\mathcolor{gray}{\bar{k}}$ 有关,即材料常数除了有时间/角频率 $\mathcolor{gray}{\omega}$ 色散外,还有空间/波矢 $\mathcolor{gray}{\bar{k}}$ 色散,意味着介质的响应是非局域的:从正空间上讲,即长程相互作用:$\mathcolor{gray}{\bar{r}'}$ 处的外场可以影响 $\mathcolor{gray}{\bar{r}}$ 处的材料常数,见 \bref{P_barx_nonlinear};从倒空间上讲,即外场的梯度也起作用:材料常数在 $\mathcolor{gray}{\bar{r}}$ 处的值,不仅是同一地点的外场的函数,还是其在该 $\mathcolor{gray}{\bar{r}}$ 处的空间各阶导数 $\mathcolor{gray}{\nabla^n_{\hat{1}}}$ 的函数,即 $\chi^{\;\! \mathcolor{gray}{t} \hat{1}}_{\;\! \symup{\iota} \mathcolor{gray}{\bar{r}} \textcolor{Maroon}{(1)}} \left( \mathcolor{gray}{\nabla^t}, \mathcolor{gray}{\nabla_{\hat{2}}} ; E^{\;\!\mathcolor{gray}{t}}_{\;\! \hat{3}\mathcolor{gray}{\bar{r}}}, B^{\;\!\mathcolor{gray}{t}}_{\;\! \hat{4}\mathcolor{gray}{\bar{r}}} \right)$。—— 然而,在电偶极 $\bar{P}^{\;\!\mathcolor{gray}{t}}_{\;\!\mathcolor{gray}{z}}$ 近似下,一般认为材料常数与 $\mathcolor{gray}{\bar{k}}$ 无关,只是 $\mathcolor{gray}{\omega}$ 的函数,如大部分国内外教材中所定义的 $\chi^{\;\! \mathcolor{gray}{\omega} \hat{1}}_{\;\! \symup{\iota} \textcolor{Maroon}{(1)}} \asymp \chi^{ \textcolor{Maroon}{(1)} \mathcolor{gray}{\omega}}_{\;\! ij}$。这意味着 $\chi^{\;\! \mathcolor{gray}{\omega} \hat{1}}_{\;\! \symup{\iota} \mathcolor{gray}{\bar{k}} \textcolor{Maroon}{(1)}}$ 中的空间色散项(即含 $\mathcolor{gray}{\bar{k}}$ 项),应由电偶极矩 $\bar{P}^{\;\!\mathcolor{gray}{t}}_{\;\!\mathcolor{gray}{z}}$ 之外的电多极矩 $\bar{\bar{Q}}^{\;\!\mathcolor{gray}{t}}_{\;\!\mathcolor{gray}{z}},\bar{\bar{\bar{O}}}^{\;\!\mathcolor{gray}{t}}_{\;\!\mathcolor{gray}{z}}$ 贡献\cite{shenNonlinearOpticalSusceptibilities2001},以至于不应将该部分包含在电偶极化强度 $P^{\;\!\mathcolor{gray}{\omega}}_{\;\! \symup{\iota}\mathcolor{gray}{\bar{k}}}$ 的 \bref{P_barkappa_nonlinear} 之内(但后续的 \bref{P(1)_barkappa_nonlinear} 会打破该认识)。此外,\bref{P_barkappa_nonlinear} 无法给出 $\chi^{\;\! \mathcolor{gray}{\omega} \hat{1}}_{\;\! \symup{\iota} \mathcolor{gray}{\bar{k}} \textcolor{Maroon}{(1)}} \left( \mathcolor{gray}{\bar{k}} \right)$ 关于 $\mathcolor{gray}{\bar{k}}$ 的显式公式。

以上三点原因,迫使线性、非线性晶体光学,转而继续向多极理论寻求帮助。在半经典量子力学的框架下,考虑场和电荷分布体系间的相互作用能远小于体系自能的情况,从一、二、三阶微扰下的含时哈密顿量的微观起源的角度,多极理论建议将 $\mathcolor{gray}{\bar{\kappa}}$ 空间的材料常数 $\chi^{\;\! \mathcolor{gray}{\omega} \hat{1}}_{\;\! \symup{\iota} \mathcolor{gray}{\bar{k}} \textcolor{Maroon}{(1)}}, \chi^{\;\! \mathcolor{gray}{\omega} \hat{1} \hat{2}}_{\;\! \symup{\iota} \mathcolor{gray}{\bar{k}} \textcolor{Maroon}{(2)}}, \chi^{\;\! \mathcolor{gray}{\omega} \hat{1} \hat{2} \hat{3}}_{\;\! \symup{\iota} \mathcolor{gray}{\bar{k}} \textcolor{Maroon}{(3)}}$ 的空间色散部分剥离,并全部转移到场 $E^{\;\!\mathcolor{gray}{\omega}}_{\;\! \hat{n} \mathcolor{gray}{\bar{k}}}$ 中。以 \bref{P_barkappa_nonlinear} 的线性项 $P^{\;\! \textcolor{Maroon}{(1)} \mathcolor{gray}{\omega}}_{\;\! \symup{\iota}\mathcolor{gray}{\bar{k}}} = {\symup{\varepsilon_0}} \chi^{\;\! \mathcolor{gray}{\omega} \hat{1}}_{\;\! \symup{\iota} \mathcolor{gray}{\bar{k}} \textcolor{Maroon}{(1)}} E^{\;\!\mathcolor{gray}{\omega}}_{\;\! \hat{1} \mathcolor{gray}{\bar{k}}}$ 为例,材料常数 $\chi^{\;\! \mathcolor{gray}{\omega} \hat{1}}_{\;\! \symup{\iota} \mathcolor{gray}{\bar{k}} \textcolor{Maroon}{(1)}}$ 的空间/波矢 $\mathcolor{gray}{\bar{k}}$ 色散部分,可以如下地转移到场 $E^{\;\!\mathcolor{gray}{\omega}}_{\;\! \hat{n} \mathcolor{gray}{\bar{k}}}$ 中
\begin{subequations}
\begin{align}
	P^{\;\! \textcolor{Maroon}{(1)} \mathcolor{gray}{\omega}}_{\;\! \symup{\iota}\mathcolor{gray}{\bar{k}}} =&~ {\symup{\varepsilon_0}} \chi^{\;\! \mathcolor{gray}{\omega} \hat{1}}_{\;\! \symup{\iota} \mathcolor{gray}{\bar{k}} \textcolor{Maroon}{(1)}} E^{\;\!\mathcolor{gray}{\omega}}_{\;\! \hat{1} \mathcolor{gray}{\bar{k}}} \\ =&~ {\symup{\varepsilon_0}} \left\{ \chi^{\;\! \mathcolor{gray}{\omega} \hat{1}}_{\;\! \symup{\iota} \textcolor{Maroon}{(1)}} E^{\;\!\mathcolor{gray}{\omega}}_{\;\! \hat{1} \mathcolor{gray}{\bar{k}}} + \chi^{\;\! \mathcolor{gray}{\omega} \hat{1} \mathcolor{gray}{\hat{2}}}_{\;\! \symup{\iota} \textcolor{Maroon}{(1)}} E^{\;\!\mathcolor{gray}{\omega}}_{\;\! \mathcolor{gray}{\hat{2}} \hat{1} \mathcolor{gray}{\bar{k}}} + \chi^{\;\! \mathcolor{gray}{\omega} \hat{1} \mathcolor{gray}{\hat{2} \hat{3}}}_{\;\! \symup{\iota} \textcolor{Maroon}{(1)}} E^{\;\!\mathcolor{gray}{\omega}}_{\;\! \mathcolor{gray}{\hat{3} \hat{2}} \hat{1} \mathcolor{gray}{\bar{k}}} + \cdots \right\}  \label{P(1)_barkappa_nonlinear} \\ :=&~ {\symup{\varepsilon_0}} \left\{ \chi^{\;\! \mathcolor{gray}{\omega} \hat{1}}_{\;\! \symup{\iota} \textcolor{Maroon}{(1)}} E^{\;\!\mathcolor{gray}{\omega}}_{\;\! \hat{1} \mathcolor{gray}{\bar{k}}} + \chi^{\;\! \mathcolor{gray}{\omega} \hat{1} \mathcolor{gray}{\hat{2}}}_{\;\! \symup{\iota} \textcolor{Maroon}{(1)}} \left( \mathcolor{gray}{k_{\hat{2}}} E^{\;\!\mathcolor{gray}{\omega}}_{\;\hat{1} \mathcolor{gray}{\bar{k}}} \right) + \chi^{\;\! \mathcolor{gray}{\omega} \hat{1} \mathcolor{gray}{\hat{2} \hat{3}}}_{\;\! \symup{\iota} \textcolor{Maroon}{(1)}} \left( \mathcolor{gray}{k_{\hat{3}}} \mathcolor{gray}{k_{\hat{2}}} E^{\;\!\mathcolor{gray}{\omega}}_{\;\! \hat{1} \mathcolor{gray}{\bar{k}}} \right) + \cdots \right\}~,
\end{align}
\end{subequations}
从上述多极理论给出的结果可见,哪怕是电偶极矩 $\bar{P}^{\;\!\mathcolor{gray}{\omega}}_{\;\!\mathcolor{gray}{\bar{k}}}$(的线性部分),也含空间/波矢 $\mathcolor{gray}{\bar{k}}$ 色散(而不仅是电多极矩 $\bar{\bar{Q}}^{\;\!\mathcolor{gray}{\omega}}_{\;\!\mathcolor{gray}{\bar{k}}},\bar{\bar{\bar{O}}}^{\;\!\mathcolor{gray}{\omega}}_{\;\!\mathcolor{gray}{\bar{k}}}$ 的专属特性)。

需注意的是,\bref{P(1)_barkappa_nonlinear} 包含了以下可能的外场:交/直流电场 $E^{\;\!\mathcolor{gray}{t}}_{\;\! \hat{n} \mathcolor{gray}{\bar{r}}}$、交流电场的各 $\geq 1$ 阶时间导数 $\mathcolor{gray}{\nabla^t} E^{\;\!\mathcolor{gray}{t}}_{\;\! \hat{n} \mathcolor{gray}{\bar{r}}}$、交/直流电场的各 $\geq 1$ 阶空间导数 $\mathcolor{gray}{\nabla_{\hat{m}}} E^{\;\!\mathcolor{gray}{t}}_{\;\! \hat{n} \mathcolor{gray}{\bar{r}}}$;交流磁感应场 $B^{\;\!\mathcolor{gray}{t}}_{\;\! \hat{n} \mathcolor{gray}{\bar{r}}}$、交流磁感应场的各 $\geq 1$ 阶时间导数 $\mathcolor{gray}{\nabla^t} B^{\;\!\mathcolor{gray}{t}}_{\;\! \hat{n} \mathcolor{gray}{\bar{r}}}$、交流磁感应场的各 $\geq 1$ 阶空间导数 $\mathcolor{gray}{\nabla_{\hat{m}}} B^{\;\!\mathcolor{gray}{t}}_{\;\! \hat{n} \mathcolor{gray}{\bar{r}}}$,—— 但唯独不包含直流静磁感应场 $B^{\;\!\mathcolor{gray}{\text{dc}}}_{\;\! \hat{n} \mathcolor{gray}{\bar{r}}}$,及其各 $\geq 1$ 阶空间导数 $\mathcolor{gray}{\nabla_{\hat{m}}} B^{\;\!\mathcolor{gray}{\text{dc}}}_{\;\! \hat{n} \mathcolor{gray}{\bar{r}}}$。这是因为:叉积算子 $\epsilon^{\hphantom{\symup{\iota}\hat{m}}\hat{n}}_{\hat{l}\mathcolor{gray}{\hat{m}}}$ 可以包含在 $\chi^{\;\! \mathcolor{gray}{\omega} \hat{1} \hat{2}}_{\;\! \symup{\iota} \mathcolor{gray}{\bar{k}} \textcolor{Maroon}{(2)}}$ 及以上阶次的材料常数中,使得通过\textcolor{Maroon}{\text{法拉第电磁感应定律}} $\epsilon^{\hphantom{\symup{\iota}\hat{m}}\hat{n}}_{\hat{l}\mathcolor{gray}{\hat{m}}} \mathcolor{gray}{\nabla^{\hat{m}}} E^{\;\!\mathcolor{gray}{t}}_{\;\! \hat{n} \mathcolor{gray}{\bar{r}}} = - \mathcolor{gray}{\nabla^t} B^{\;\!\mathcolor{gray}{t}}_{\;\! \hat{l} \mathcolor{gray}{\bar{r}}}$,材料常数中的 $\epsilon^{\hphantom{\symup{\iota}\hat{m}}\hat{n}}_{\hat{l}\mathcolor{gray}{\hat{m}}}$ 与梯度算子 $\mathcolor{gray}{\nabla^{\hat{m}}}$、电场 $E^{\;\!\mathcolor{gray}{t}}_{\;\! \hat{n} \mathcolor{gray}{\bar{r}}}$ 三者的乘积能且只能转换为磁感应场的时间导数 $\mathcolor{gray}{\nabla^t} B^{\;\!\mathcolor{gray}{t}}_{\;\! \hat{l} \mathcolor{gray}{\bar{r}}}$。

此外,还需要提醒的是,空域偏导算子 $\mathcolor{gray}{\nabla^{\hat{m}}}$ 作用于单色时变场后,严格来说得到的不是 \bref{P(1)_barkappa_nonlinear} 中的 $\mathcolor{gray}{k_{\hat{m}}}$(即 $\mathcolor{gray}{k_{\symup{x}}},~ \mathcolor{gray}{k_{\symup{y}}},~ \mathcolor{gray}{k_{\symup{z}}}$ 中的某个),而是 $k^{\mathcolor{gray}{\omega}}_{\hat{m}}$(即 $\mathcolor{gray}{k_{\symup{x}}},~ \mathcolor{gray}{k_{\symup{y}}},~ k_{\symup{z}}^{\mathcolor{gray}{\omega}}$ 中的某个)。因此,严格来说多极理论与 Volterra 级数各自给出的非局域表达式不同,下文会再次更新 \bref{P_barx_nonlinear,P_barkappa_nonlinear,P(1)_barkappa_nonlinear} 所涉及的所有公式的定义,以将其彻底纳入多极理论框架内(将 Volterra 级数视为辅助解释,但不用之计算)。

接着,同样以 \bref{P_barkappa_nonlinear} 的非线性项 $P^{\;\! \textcolor{Maroon}{(2)} \mathcolor{gray}{\omega}}_{\;\! \symup{\iota}\mathcolor{gray}{\bar{k}}} = \chi^{\;\! \mathcolor{gray}{\omega} \hat{1} \hat{2}}_{\;\! \symup{\iota} \mathcolor{gray}{\bar{k}} \textcolor{Maroon}{(2)}} \left( E^{\;\!\mathcolor{gray}{\omega}}_{\;\! \hat{1} \mathcolor{gray}{\bar{k}}} ~\mathcolor{gray}{\widetilde \circledast}~ E^{\;\!\mathcolor{gray}{\omega}}_{\;\! \hat{2} \mathcolor{gray}{\bar{k}}} \right)$ 为例,材料常数 $\chi^{\;\! \mathcolor{gray}{\omega} \hat{1} \hat{2}}_{\;\! \symup{\iota} \mathcolor{gray}{\bar{k}} \textcolor{Maroon}{(2)}}$ 的空间/波矢 $\mathcolor{gray}{\bar{k}}$ 色散部分,可以如下地转移到场 $E^{\;\!\mathcolor{gray}{\omega}}_{\;\! \hat{n} \mathcolor{gray}{\bar{k}}}$ 中
\begin{subequations}
\begin{align}
	P^{\;\! \textcolor{Maroon}{(2)} \mathcolor{gray}{\omega}}_{\;\! \symup{\iota}\mathcolor{gray}{\bar{k}}} =&~ {\symup{\varepsilon_0}} \chi^{\;\! \mathcolor{gray}{\omega} \hat{1} \hat{2}}_{\;\! \symup{\iota} \mathcolor{gray}{\bar{k}} \textcolor{Maroon}{(2)}} \left( E^{\;\!\mathcolor{gray}{\omega}}_{\;\! \hat{1} \mathcolor{gray}{\bar{k}}} ~\mathcolor{gray}{\widetilde \circledast}~ E^{\;\!\mathcolor{gray}{\omega}}_{\;\! \hat{2} \mathcolor{gray}{\bar{k}}} \right) \\ =&~ {\symup{\varepsilon_0}} \left\{ \chi^{\;\! \mathcolor{gray}{\omega} \hat{1} \hat{2}}_{\;\! \symup{\iota} \textcolor{Maroon}{(2)}} \left( E^{\;\!\mathcolor{gray}{\omega}}_{\;\! \hat{1} \mathcolor{gray}{\bar{k}}} ~\mathcolor{gray}{\widetilde \circledast}~ E^{\;\!\mathcolor{gray}{\omega}}_{\;\! \hat{2} \mathcolor{gray}{\bar{k}}} \right) + \chi^{\;\! \mathcolor{gray}{\omega} \hat{1} \hat{2} \mathcolor{gray}{\hat{3}}}_{\;\! \symup{\iota} \mathcolor{gray}{\bar{k}} \textcolor{Maroon}{(2)}} \left( E^{\;\!\mathcolor{gray}{\omega}}_{\;\! \mathcolor{gray}{\hat{3}} \hat{1} \mathcolor{gray}{\bar{k}}} ~\mathcolor{gray}{\widetilde \circledast}~ E^{\;\!\mathcolor{gray}{\omega}}_{\;\! \hat{2} \mathcolor{gray}{\bar{k}}} \right) + \cdots \right\} \label{P_barkappa(2)_nonlinear}
	\\ :=&~ {\symup{\varepsilon_0}} \left\{ \chi^{\;\! \mathcolor{gray}{\omega} \hat{1} \hat{2}}_{\;\! \symup{\iota} \textcolor{Maroon}{(2)}} \left( E^{\;\!\mathcolor{gray}{\omega}}_{\;\! \hat{1} \mathcolor{gray}{\bar{k}}} ~\mathcolor{gray}{\widetilde \circledast}~ E^{\;\!\mathcolor{gray}{\omega}}_{\;\! \hat{2} \mathcolor{gray}{\bar{k}}} \right) + \chi^{\;\! \mathcolor{gray}{\omega} \hat{1} \hat{2} \mathcolor{gray}{\hat{3}}}_{\;\! \symup{\iota} \mathcolor{gray}{\bar{k}} \textcolor{Maroon}{(2)}} \left[ \left( \mathcolor{gray}{k_{\hat{3}}} E^{\;\!\mathcolor{gray}{\omega}}_{\;\! \hat{1} \mathcolor{gray}{\bar{k}}} \right) ~\mathcolor{gray}{\widetilde \circledast}~ E^{\;\!\mathcolor{gray}{\omega}}_{\;\! \hat{2} \mathcolor{gray}{\bar{k}}} \right] + \cdots \right\}~,
\end{align}
\end{subequations}
其中,通过内禀置换对称性\cite{boydNonlinearOptics2019}、卷积交换律,可以得出 $\chi^{\;\! \mathcolor{gray}{\omega} \hat{1} \hat{2} \mathcolor{gray}{\hat{3}}}_{\;\! \symup{\iota} \mathcolor{gray}{\bar{k}} \textcolor{Maroon}{(2)}} \left( E^{\;\!\mathcolor{gray}{\omega}}_{\;\! \mathcolor{gray}{\hat{3}} \hat{1} \mathcolor{gray}{\bar{k}}} ~\mathcolor{gray}{\widetilde \circledast}~ E^{\;\!\mathcolor{gray}{\omega}}_{\;\! \hat{2} \mathcolor{gray}{\bar{k}}} \right) = \chi^{\;\! \mathcolor{gray}{\omega} \hat{2} \hat{1} \mathcolor{gray}{\hat{3}}}_{\;\! \symup{\iota} \mathcolor{gray}{\bar{k}} \textcolor{Maroon}{(2)}} \left( E^{\;\!\mathcolor{gray}{\omega}}_{\;\! \hat{1} \mathcolor{gray}{\bar{k}}} ~\mathcolor{gray}{\widetilde \circledast}~ E^{\;\!\mathcolor{gray}{\omega}}_{\;\! \mathcolor{gray}{\hat{3}} \hat{2} \mathcolor{gray}{\bar{k}}} \right)$;另外,傅立叶变换的导数定理所导出的 $\mathcolor{gray}{\nabla_{\hat{3}}} \left( E^{\;\!\mathcolor{gray}{\omega}}_{\;\! \hat{1} \mathcolor{gray}{\bar{k}}} ~\mathcolor{gray}{\widetilde \circledast}~ E^{\;\!\mathcolor{gray}{\omega}}_{\;\! \hat{2} \mathcolor{gray}{\bar{k}}} \right) = E^{\;\!\mathcolor{gray}{\omega}}_{\;\! \mathcolor{gray}{\hat{3}} \hat{1} \mathcolor{gray}{\bar{k}}} ~\mathcolor{gray}{\widetilde \circledast}~ E^{\;\!\mathcolor{gray}{\omega}}_{\;\! \hat{2} \mathcolor{gray}{\bar{k}}} = E^{\;\!\mathcolor{gray}{\omega}}_{\;\! \hat{1} \mathcolor{gray}{\bar{k}}} ~\mathcolor{gray}{\widetilde \circledast}~ E^{\;\!\mathcolor{gray}{\omega}}_{\;\! \mathcolor{gray}{\hat{3}} \hat{2} \mathcolor{gray}{\bar{k}}}$\Footnote{同样出于 $\mathcolor{gray}{k_{\hat{m}}} \neq k^{\mathcolor{gray}{\omega}}_{\hat{m}}$ 的原因,这里并不严谨:须完全在多极理论框架内才有该结论;但不影响其成立。},可得 $\chi^{\;\! \mathcolor{gray}{\omega} \hat{1} \hat{2} \mathcolor{gray}{\hat{3}}}_{\;\! \symup{\iota} \mathcolor{gray}{\bar{k}} \textcolor{Maroon}{(2)}} \left( E^{\;\!\mathcolor{gray}{\omega}}_{\;\! \mathcolor{gray}{\hat{3}} \hat{1} \mathcolor{gray}{\bar{k}}} ~\mathcolor{gray}{\widetilde \circledast}~ E^{\;\!\mathcolor{gray}{\omega}}_{\;\! \hat{2} \mathcolor{gray}{\bar{k}}} \right) = \chi^{\;\! \mathcolor{gray}{\omega} \hat{1} \hat{2} \mathcolor{gray}{\hat{3}}}_{\;\! \symup{\iota} \mathcolor{gray}{\bar{k}} \textcolor{Maroon}{(2)}} \left( E^{\;\!\mathcolor{gray}{\omega}}_{\;\! \hat{1} \mathcolor{gray}{\bar{k}}} ~\mathcolor{gray}{\widetilde \circledast}~ E^{\;\!\mathcolor{gray}{\omega}}_{\;\! \mathcolor{gray}{\hat{3}} \hat{2} \mathcolor{gray}{\bar{k}}} \right)$;两相对比,即有 $\chi^{\;\! \mathcolor{gray}{\omega} \hat{1} \hat{2} \mathcolor{gray}{\hat{3}}}_{\;\! \symup{\iota} \mathcolor{gray}{\bar{k}} \textcolor{Maroon}{(2)}} = \chi^{\;\! \mathcolor{gray}{\omega} \hat{2} \hat{1} \mathcolor{gray}{\hat{3}}}_{\;\! \symup{\iota} \mathcolor{gray}{\bar{k}} \textcolor{Maroon}{(2)}}$。此外,可以看出,非线性项仍然可以是非局域的,因此(非)线性响应与(非)局域响应又成为一对(新的)相互独立的概念。

类似 \bref{P(1)_barkappa_nonlinear} 地,电四极矩 $Q^{\;\!\mathcolor{gray}{t}}_{\;\! \symup{\iota} \hat{1} \mathcolor{gray}{z}} \left( \mathcolor{gray}{\nabla^t}, \mathcolor{gray}{\nabla_{\hat{2}}} ; E^{\;\!\mathcolor{gray}{t}}_{\;\! \hat{3}\mathcolor{gray}{z}}, B^{\;\!\mathcolor{gray}{t}}_{\;\! \hat{4}\mathcolor{gray}{z}} \right)$ 的线性响应部分为
\begin{subequations}
\begin{align}
	Q^{\;\! \textcolor{Maroon}{(1)} \mathcolor{gray}{\omega}}_{\;\! \symup{\iota} \hat{1} \mathcolor{gray}{\bar{k}}} =&~ {\symup{\varepsilon_0}} \chi^{\;\! \mathcolor{gray}{\omega} \hat{2}}_{\;\! \symup{\iota} \hat{1} \mathcolor{gray}{\bar{k}} \textcolor{Maroon}{(1)}} E^{\;\!\mathcolor{gray}{\omega}}_{\;\! \hat{2} \mathcolor{gray}{\bar{k}}} \\ =&~ {\symup{\varepsilon_0}} \left\{ \chi^{\;\! \mathcolor{gray}{\omega} \hat{2}}_{\;\! \symup{\iota} \hat{1} \textcolor{Maroon}{(1)}} E^{\;\!\mathcolor{gray}{\omega}}_{\;\! \hat{2} \mathcolor{gray}{\bar{k}}} + \chi^{\;\! \mathcolor{gray}{\omega} \hat{1} \mathcolor{gray}{\hat{2}}}_{\;\! \symup{\iota} \textcolor{Maroon}{(1)}} E^{\;\!\mathcolor{gray}{\omega}}_{\;\! \mathcolor{gray}{\hat{2}} \hat{1} \mathcolor{gray}{\bar{k}}} + \chi^{\;\! \mathcolor{gray}{\omega} \hat{1} \mathcolor{gray}{\hat{2} \hat{3}}}_{\;\! \symup{\iota} \textcolor{Maroon}{(1)}} E^{\;\!\mathcolor{gray}{\omega}}_{\;\! \mathcolor{gray}{\hat{3} \hat{2}} \hat{1} \mathcolor{gray}{\bar{k}}} + \cdots \right\}  \label{Q(1)_barkappa_nonlinear} \\ :=&~ {\symup{\varepsilon_0}} \left\{ \chi^{\;\! \mathcolor{gray}{\omega} \hat{1}}_{\;\! \symup{\iota} \textcolor{Maroon}{(1)}} E^{\;\!\mathcolor{gray}{\omega}}_{\;\! \hat{1} \mathcolor{gray}{\bar{k}}} + \chi^{\;\! \mathcolor{gray}{\omega} \hat{1} \mathcolor{gray}{\hat{2}}}_{\;\! \symup{\iota} \textcolor{Maroon}{(1)}} \left( \mathcolor{gray}{k_{\hat{2}}} E^{\;\!\mathcolor{gray}{\omega}}_{\;\hat{1} \mathcolor{gray}{\bar{k}}} \right) + \chi^{\;\! \mathcolor{gray}{\omega} \hat{1} \mathcolor{gray}{\hat{2} \hat{3}}}_{\;\! \symup{\iota} \textcolor{Maroon}{(1)}} \left( \mathcolor{gray}{k_{\hat{3}}} \mathcolor{gray}{k_{\hat{2}}} E^{\;\!\mathcolor{gray}{\omega}}_{\;\! \hat{1} \mathcolor{gray}{\bar{k}}} \right) + \cdots \right\}~,
\end{align}
\end{subequations}

分别给磁感应场$\bar{B}^{\;\!\mathcolor{gray}{t}}_{\;\!\mathcolor{gray}{z}}$(而非磁场$\bar{H}^{\;\!\mathcolor{gray}{t}}_{\;\!\mathcolor{gray}{z}}$),自由电流源$\bar{J}^{\;\!\mathcolor{gray}{t}}_{\;\!\textcolor{Maroon}{\text{f}}\mathcolor{gray}{z}}$ 以及电位移场$\bar{D}^{\;\!\mathcolor{gray}{t}}_{\;\!\mathcolor{gray}{z}}$,定义了如下 3 个本构关系(\textcolor{Maroon}{\text{Constitutive Relation}} = \textcolor{Maroon}{\text{CR}})。

\begin{subequations}
\begin{align}
	P^{\;\!\mathcolor{gray}{t}}_{\;\! \symup{\iota}\mathcolor{gray}{z}} &= \mathcal F_t \left[ P^{\;\! \mathcolor{gray}{\omega}}_{\;\! \symup{\iota}\mathcolor{gray}{z}} \right] = P^{\;\! \textcolor{Maroon}{(1)} \mathcolor{gray}{t}}_{\;\! \symup{\iota}\mathcolor{gray}{z}} + P^{\;\! \textcolor{Maroon}{(2)} \mathcolor{gray}{t}}_{\;\! \symup{\iota}\mathcolor{gray}{z}} + P^{\;\! \textcolor{Maroon}{(3)} \mathcolor{gray}{t}}_{\;\! \symup{\iota}\mathcolor{gray}{z}} + \cdots \label{P_z_nonlinear} \\ &= {\symup{\varepsilon_0}} \left\{ \chi^{\;\! \mathcolor{gray}{t} \hat{1}}_{\;\! \symup{\iota} \mathcolor{gray}{z} \textcolor{Maroon}{(1)}} ~\widetilde *~ E^{\;\!\mathcolor{gray}{t}}_{\;\! \hat{1} \mathcolor{gray}{z}} + \chi^{\;\! \mathcolor{gray}{t} \hat{1} \hat{2}}_{\;\! \symup{\iota} \mathcolor{gray}{z} \textcolor{Maroon}{(2)}}~\widetilde *\left( E^{\;\!\mathcolor{gray}{t}}_{\;\! \hat{1} \mathcolor{gray}{z}} E^{\;\!\mathcolor{gray}{t}}_{\;\! \hat{2} \mathcolor{gray}{z}} \right) \right. \\ &+ \left. \chi^{\;\! \mathcolor{gray}{t} \hat{1} \hat{2} \hat{3}}_{\;\! \symup{\iota} \mathcolor{gray}{z} \textcolor{Maroon}{(3)}}~\widetilde *\left( E^{\;\!\mathcolor{gray}{t}}_{\;\! \hat{1} \mathcolor{gray}{z}} E^{\;\!\mathcolor{gray}{t}}_{\;\! \hat{2} \mathcolor{gray}{z}} E^{\;\!\mathcolor{gray}{t}}_{\;\! \hat{3} \mathcolor{gray}{z}} \right) + \cdots \right\}
\end{align}
\end{subequations}

其一,磁场 $\bar{H}^{\;\!\mathcolor{gray}{t}}_{\;\!\mathcolor{gray}{z}}$ 的本构关系,考虑$\bar{M}^{\;\!\mathcolor{gray}{t}}_{\;\!\mathcolor{gray}{z}}$\Footnote{磁化强度。其在微观上来源于:分子电流(电子轨道运动)产生的磁矩和电子自旋磁矩的矢量和\cite{nelsonLagrangianTreatmentMagnetic1994},细究还将有电子与原子核、电子的自旋轨道耦合、电子电子间相互作用(多电子产生原子磁矩)、原子与原子间(晶体场)的相互作用\cite{chen-zhuChenZhuxieUndergraduate_courses2024}。磁的起源看上去可以是纯电的\cite{lakhtakiaGenesisPostConstraint2004}。}仅由磁偶极矩\Footnote{即只考虑最低阶磁多极矩 = 不考虑磁四极矩及以上\cite{nelsonLagrangianTreatmentMagnetic1994},因为对于受到电磁场的影响后,反过来产生电磁场的电子而言,其受到的电场力是洛伦兹力的c$\big/v$倍\cite{boydNonlinearOptics2019}。此外,(磁/电)多极矩与(磁/电)非线性是互相独立的——即任何阶的多极矩,都有自己的线性项和非线性项\cite{chen-zhuChenZhuxieUndergraduate_courses2024},这些项与其它阶多极矩的任何项都无关。}贡献时,定义为
\begin{subequations} \label{eq:cr-b}
\begin{align}
	\textcolor{Maroon}{\text{CR for magnetism}}\text{:}&\hspace{0.5em} \bar{B}^{\;\!\mathcolor{gray}{t}}_{\;\!\mathcolor{gray}{z}} \hspace{-0.7em} &&={\symup{\varepsilon}}_0 \left( \bar{H}^{\;\!\mathcolor{gray}{t}}_{\;\!\mathcolor{gray}{z}} + \bar{M}^{\;\!\mathcolor{gray}{t}}_{\;\!\mathcolor{gray}{z}} \right) = {\symup{\varepsilon}}_0 \left\{ \bar{\bar{\delta}}^{\;\!\mathcolor{gray}{t}}~\widetilde *~\bar{H}^{\;\!\mathcolor{gray}{t}}_{\;\!\mathcolor{gray}{z}} + \bar{M}^{\;\!\mathcolor{gray}{t}}_{\;\!\mathcolor{gray}{z}} \right\} \label{cr-b1} \\ 
	& &&\xrightarrow[]{\bar{M}^{\;\!\mathcolor{gray}{t}}_{\;\!\mathcolor{gray}{z}} = \bar{M}^{\;\!\textcolor{Maroon}{\text{(1)}} \mathcolor{gray}{t}}_{\;\!\mathcolor{gray}{z}} + \bar{M}^{\;\!\textcolor{Maroon}{\text{NL}}, \mathcolor{gray}{t}}_{\;\!\mathcolor{gray}{z}}} {\symup{\varepsilon}}_0 \left\{ \left[ \bar{\bar{\delta}}^{\;\!\mathcolor{gray}{t}}~\widetilde *~\bar{H}^{\;\!\mathcolor{gray}{t}}_{\;\!\mathcolor{gray}{z}} + \bar{M}^{\;\!\textcolor{Maroon}{\text{(1)}} \mathcolor{gray}{t}}_{\;\!\mathcolor{gray}{z}} \right] + \bar{M}^{\;\!\textcolor{Maroon}{\text{NL}}, \mathcolor{gray}{t}}_{\;\!\mathcolor{gray}{z}} \right\} \label{cr-b2} \\ 
	& &&\xrightarrow[\displaystyle{ \bar{\bar{\mu}}^{\;\!\textcolor{Maroon}{\text{(1)}} \mathcolor{gray}{t}}_{\;\!\textcolor{Maroon}{\text{r}}\mathcolor{gray}{z}} := \bar{\bar{\delta}}^{\;\!\mathcolor{gray}{t}} + \bar{\bar{\chi}}^{\;\!\textcolor{Maroon}{\text{(1)}}\mathcolor{gray}{t}}_{\;\!\textcolor{Maroon}{\text{m}} \mathcolor{gray}{z}}}]{\displaystyle{\bar{M}^{\;\!\textcolor{Maroon}{\text{(1)}} \mathcolor{gray}{t}}_{\;\!\mathcolor{gray}{z}} := \bar{\bar{\chi}}^{\;\!\textcolor{Maroon}{\text{(1)}}\mathcolor{gray}{t}}_{\;\!\textcolor{Maroon}{\text{m}} \mathcolor{gray}{z}} ~\widetilde *~\bar{H}^{\;\!\mathcolor{gray}{t}}_{\;\!\mathcolor{gray}{z}}}} {\symup{\varepsilon}}_0 \left\{ \bar{\bar{\mu}}^{\;\!\textcolor{Maroon}{\text{(1)}} \mathcolor{gray}{t}}_{\;\!\textcolor{Maroon}{\text{r}}\mathcolor{gray}{z}}~\widetilde *~\bar{H}^{\;\!\mathcolor{gray}{t}}_{\;\!\mathcolor{gray}{z}} + \bar{M}^{\;\!\textcolor{Maroon}{\text{NL}}, \mathcolor{gray}{t}}_{\;\!\mathcolor{gray}{z}} \right\} \label{cr-b3} \\ 
	& &&= \bar{\bar{\mu}}^{\;\!\textcolor{Maroon}{\text{(1)}} \mathcolor{gray}{t}}_{\;\!\mathcolor{gray}{z}}~\widetilde *~\bar{H}^{\;\!\mathcolor{gray}{t}}_{\;\!\mathcolor{gray}{z}} + {\symup{\varepsilon}}_0 \bar{M}^{\;\!\textcolor{Maroon}{\text{NL}}, \mathcolor{gray}{t}}_{\;\!\mathcolor{gray}{z}} =: \bar{B}^{\;\!\textcolor{Maroon}{\text{(1)}} \mathcolor{gray}{t}}_{\;\!\mathcolor{gray}{z}} + \bar{B}^{\;\!\textcolor{Maroon}{\text{NL}}, \mathcolor{gray}{t}}_{\;\!\mathcolor{gray}{z}}~, \label{cr-b4}
\end{align}
\end{subequations}
其中,磁通量密度场 $\bar{B}^{\;\!\mathcolor{gray}{t}}_{\;\!\mathcolor{gray}{z}}$(直接/显示地)关于磁场 $\bar{H}^{\;\!\mathcolor{gray}{t}}_{\;\!\mathcolor{gray}{z}}$\Footnote{磁非线性,如郎之万顺磁性理论\cite{chen-zhuChenZhuxieUndergraduate_courses2024}中 $M \propto$ 郎之万函数 $\mathcal{L} \left( \alpha \right) = \coth \left( \alpha \right) - 1 / \alpha$(其中 $\alpha \propto H_{\textcolor{Maroon}{\text{ex}}}$)或其量子化修正之布里渊函数,铁磁体\cite{chen-zhuChenZhuxieUndergraduate_courses2024}或超导体\cite{wenBriefIntroductionFlux2021}中的磁滞现象等(每个时刻$t$,这些场量都是准静态$\Omega \to 0$的)。}、电场 $\bar{E}^{\;\!\mathcolor{gray}{t}}_{\;\!\mathcolor{gray}{z}}$\Footnote{双各向异性,其中的电$\to$磁耦合(如果 $\bar{B}^{\;\!\mathcolor{gray}{t}}_{\;\!\mathcolor{gray}{z}}$ 中的该部分只是 $\bar{E}^{\;\!\mathcolor{gray}{t}}_{\;\!\mathcolor{gray}{z}}$ 的线性函数,则也可归结到线性项中)。}、应力 $\bar{T}^{\;\!\mathcolor{gray}{t}}_{\;\!\mathcolor{gray}{z}}$\Footnote{正逆压磁/磁致伸缩/磁弹效应(这里未作区分)。}等其他场量(即含空 $\mathcolor{gray}{\bar{r}}$ 的物理量)\Footnote{$\bar{B}^{\;\!\mathcolor{gray}{t}}_{\;\!\mathcolor{gray}{z}},\bar{M}^{\;\!\mathcolor{gray}{t}}_{\;\!\mathcolor{gray}{z}}$ 以及各阶 $\bar{\bar{\mu}}^{\;\!\textcolor{Maroon}{\text{(1)}} \mathcolor{gray}{t}}_{\;\!\mathcolor{gray}{z}},\bar{\bar{\bar{\mu}}}^{\;\!\textcolor{Maroon}{\text{(2)}} \mathcolor{gray}{t}}_{\;\!\mathcolor{gray}{z}},\cdots$ 已经是关于温度$T$、波长$\lambda$(或 角频率$\omega$、时间$t$)等(非)场量的函数。}的非线性函数项,悉数包含在 $\bar{B}^{\;\!\textcolor{Maroon}{\text{NL}}, \mathcolor{gray}{t}}_{\;\!\mathcolor{gray}{z}} = {\symup{\varepsilon}}_0 \bar{M}^{\;\!\textcolor{Maroon}{\text{NL}}, \mathcolor{gray}{t}}_{\;\!\mathcolor{gray}{z}}$ 内;剩余的线性项,放在 $\bar{B}^{\;\!\textcolor{Maroon}{\text{(1)}} \mathcolor{gray}{t}}_{\;\!\mathcolor{gray}{z}} = \bar{\bar{\mu}}^{\;\!\textcolor{Maroon}{\text{(1)}} \mathcolor{gray}{t}}_{\;\!\mathcolor{gray}{z}}~\widetilde *~\bar{H}^{\;\!\mathcolor{gray}{t}}_{\;\!\mathcolor{gray}{z}}$ 中。

其二,自由电流源$\bar{J}^{\;\!\mathcolor{gray}{t}}_{\;\!\textcolor{Maroon}{\text{f}}\mathcolor{gray}{z}}$的本构关系,包含欧姆定律的线性部分(漂移项) $\bar{J}^{\;\!\textcolor{Maroon}{\text{(1)}} \mathcolor{gray}{t}}_{\;\!\textcolor{Maroon}{\text{f}}\mathcolor{gray}{z}} = \bar{\bar{\sigma}}^{\;\!\textcolor{Maroon}{\text{(1)}}\mathcolor{gray}{t}}_{\;\!\mathcolor{gray}{z}}~\widetilde *~\bar{E}^{\;\!\mathcolor{gray}{t}}_{\;\!\mathcolor{gray}{z}}$\Footnote{可以由 Drude 模型描述,定量解释一阶电导率$\bar{\bar{\sigma}}^{\;\!\textcolor{Maroon}{\text{(1)}}\mathcolor{gray}{t}}_{\;\!\mathcolor{gray}{z}}$的起源。},及$\bar{J}^{\;\!\mathcolor{gray}{t}}_{\;\!\textcolor{Maroon}{\text{f}}\mathcolor{gray}{z}}$分别关于电场 $\bar{E}^{\;\!\mathcolor{gray}{t}}_{\;\!\mathcolor{gray}{z}}$\Footnote{欧姆定律中的电非线性部分,比如二/三极管的伏安特性曲线\cite{chen-zhuChenZhuxieUndergraduate_courses2024}(尽管输入/输出or自/因变量,即$\bar{E}^{\;\!\mathcolor{gray}{t}}_{\;\!\mathcolor{gray}{z}}$和$\bar{J}^{\;\!\mathcolor{gray}{t}}_{\;\!\textcolor{Maroon}{\text{f}}\mathcolor{gray}{z}}$,一般均在直流或低频$\Omega$,非交流且不在光波段 opt)。}、磁感应场$\bar{B}^{\;\!\mathcolor{gray}{t}}_{\;\!\mathcolor{gray}{z}}$\Footnote{磁感应场$\bar{B}^{\;\!\mathcolor{gray}{t}}_{\;\!\mathcolor{gray}{z}}$所带来的(电场力以外的)洛伦兹力$\left( \bar{J}^{\;\!\mathcolor{gray}{t}}_{\;\!\textcolor{Maroon}{\text{f}}\mathcolor{gray}{z}} + \dot{\bar{P}}^{\;\!\mathcolor{gray}{t}}_{\;\!\mathcolor{gray}{z}} + \mathcolor{gray}{\bar{\nabla} \times} \bar{M}^{\;\!\mathcolor{gray}{t}}_{\;\!\mathcolor{gray}{z}} \right) \times \bar{B}^{\;\!\mathcolor{gray}{t}}_{\;\!\mathcolor{gray}{z}}$\cite{mackayElectromagneticAnisotropyBianisotropy2019,chen-zhuChenZhuxieUndergraduate_courses2024},会影响导/价带电子的运动(速度)$\bar{v}^{\;\!\mathcolor{gray}{t}}_{\;\!\textcolor{Maroon}{\text{f}}\mathcolor{gray}{z}}$,进而全局地影响自由电流$\bar{J}^{\;\!\mathcolor{gray}{t}}_{\;\!\textcolor{Maroon}{\text{f}}\mathcolor{gray}{z}} = {\rho}^{\;\!\mathcolor{gray}{t}}_{\;\!\textcolor{Maroon}{\text{f}}\mathcolor{gray}{z}} \bar{v}^{\;\!\mathcolor{gray}{t}}_{\;\!\textcolor{Maroon}{\text{f}}\mathcolor{gray}{z}}$和(束缚)电(偶)极化强度$\bar{P}^{\;\!\mathcolor{gray}{t}}_{\;\!\mathcolor{gray}{z}}$,包括它们的线性和非线性项\cite{boydNonlinearOptics2019}。对于强场/超快非线性光学,相对论效应使得电磁场是个统一的整体,动生(而不仅是外加)的$\bar{B}^{\;\!\mathcolor{gray}{t}}_{\;\!\mathcolor{gray}{z}}$还将带来额外的影响。磁场$\bar{H}^{\;\!\mathcolor{gray}{t}}_{\;\!\mathcolor{gray}{z}}$对分子/磁化电流体密度$\mathcolor{gray}{\bar{\nabla} \times} \bar{M}^{\;\!\mathcolor{gray}{t}}_{\;\!\mathcolor{gray}{z}}$产生的影响已包含在$\bar{M}^{\;\!\mathcolor{gray}{t}}_{\;\!\mathcolor{gray}{z}}$中了。}、导带电子浓度(数密度)梯度场$\mathcolor{gray}{\bar{\nabla}} {\rho}^{\;\!\mathcolor{gray}{t}}_{\;\!\textcolor{Maroon}{\text{f}}\mathcolor{gray}{z}}$\Footnote{在光折变效应中,作为$\bar{J}^{\;\!\mathcolor{gray}{t}}_{\;\!\textcolor{Maroon}{\text{f}}\mathcolor{gray}{z}}$中的扩散项\cite{boydNonlinearOptics2019}。${\rho}^{\;\!\mathcolor{gray}{t}}_{\;\!\textcolor{Maroon}{\text{f}}\mathcolor{gray}{z}}, \bar{J}^{\;\!\mathcolor{gray}{t}}_{\;\!\textcolor{Maroon}{\text{f}}\mathcolor{gray}{z}}$之间还应满足\bref{eq:Div-e-f}以及$\bar{J}^{\;\!\mathcolor{gray}{t}}_{\;\!\textcolor{Maroon}{\text{f}}\mathcolor{gray}{z}} = {\rho}^{\;\!\mathcolor{gray}{t}}_{\;\!\textcolor{Maroon}{\text{f}}\mathcolor{gray}{z}} \bar{v}^{\;\!\mathcolor{gray}{t}}_{\;\!\textcolor{Maroon}{\text{f}}\mathcolor{gray}{z}}$\cite{chen-zhuChenZhuxieUndergraduate_courses2024}。}和光伏电流场$\propto \lvert \bar{E}^{\;\!\mathcolor{gray}{t}}_{\;\!\mathcolor{gray}{z}} \rvert^2 \hat{c}$\Footnote{与光电导效应并列,属于内光电效应;也可能在光折变效应的$\bar{J}^{\;\!\mathcolor{gray}{t}}_{\;\!\textcolor{Maroon}{\text{f}}\mathcolor{gray}{z}}$中扮演一份角色,特别是沿着一些各向异性晶体的光轴$\hat{c}$产生电势差和内建电场\cite{boydNonlinearOptics2019}(尽管一般也只影响直流或低频$\Omega$的$\bar{J}^{\;\!\mathcolor{gray}{t}}_{\;\!\textcolor{Maroon}{\text{f}}\mathcolor{gray}{z}}$;但$\bar{J}^{\;\!\mathcolor{gray}{t}}_{\;\!\textcolor{Maroon}{\text{f}}\mathcolor{gray}{z}}$会通过影响光波段的介电常数,进而影响光波段的光强$\lvert \bar{E}^{\;\!\mathcolor{gray}{t}}_{\;\!\mathcolor{gray}{z}} \rvert^2$及$\bar{J}^{\;\!\mathcolor{gray}{t}}_{\;\!\textcolor{Maroon}{\text{f}}\mathcolor{gray}{z}}$自己的重新分布);该二阶的带耦合的非线性,看上去很像非线性极化率$\bar{P}^{\;\!\textcolor{Maroon}{\text{(2)}} \mathcolor{gray}{t}}_{\;\!\mathcolor{gray}{z}}$中的光整流项,但其频率比 THz 低,且只服务于自由电流。—— 以至该项可作为差频合并至$\bar{J}^{\;\!\mathcolor{gray}{t}}_{\;\!\textcolor{Maroon}{\text{f}}\mathcolor{gray}{z}}$关于$\bar{E}^{\;\!\mathcolor{gray}{t}}_{\;\!\mathcolor{gray}{z}}$的二阶非线性$\bar{J}^{\;\!\textcolor{Maroon}{\text{(2)}} \mathcolor{gray}{t}}_{\;\!\textcolor{Maroon}{\text{f}}\mathcolor{gray}{z}}$中去?}等其他场量的非线性项 $\bar{J}^{\;\!\textcolor{Maroon}{\text{NL}}, \mathcolor{gray}{t}}_{\;\!\textcolor{Maroon}{\text{f}}\mathcolor{gray}{z}}$:
\begin{align} \label{eq:cr-j}
	\textcolor{Maroon}{\text{Ohm's law}}\text{:}\hspace{0.5em} \bar{J}^{\;\!\mathcolor{gray}{t}}_{\;\!\textcolor{Maroon}{\text{f}}\mathcolor{gray}{z}} = \bar{\bar{\sigma}}^{\;\!\textcolor{Maroon}{\text{(1)}}\mathcolor{gray}{t}}_{\;\!\mathcolor{gray}{z}}~\widetilde *~\bar{E}^{\;\!\mathcolor{gray}{t}}_{\;\!\mathcolor{gray}{z}} + \bar{J}^{\;\!\textcolor{Maroon}{\text{NL}}, \mathcolor{gray}{t}}_{\;\!\textcolor{Maroon}{\text{f}}\mathcolor{gray}{z}} =: \bar{J}^{\;\!\textcolor{Maroon}{\text{(1)}} \mathcolor{gray}{t}}_{\;\!\textcolor{Maroon}{\text{f}}\mathcolor{gray}{z}} + \bar{J}^{\;\!\textcolor{Maroon}{\text{NL}}, \mathcolor{gray}{t}}_{\;\!\textcolor{Maroon}{\text{f}}\mathcolor{gray}{z}}~,
\end{align}

其三,电位移场$\bar{D}^{\;\!\mathcolor{gray}{t}}_{\;\!\mathcolor{gray}{z}}$的本构关系,当$\bar{P}^{\;\!\mathcolor{gray}{t}}_{\;\!\mathcolor{gray}{z}}$只由电偶极矩\Footnote{不考虑电四极矩及以上。但电四极化强度场$\bar{\bar{Q}}^{\;\!\mathcolor{gray}{t}}_{\;\!\mathcolor{gray}{z}}$(的等效电偶极化强度场$\bar{P}^{\;\!\mathcolor{gray}{t}}_{\;\!\textcolor{Maroon}{\text{Q}}\mathcolor{gray}{z}} = - \mathcolor{gray}{\bar{\nabla} \cdot} \bar{\bar{Q}}^{\;\!\mathcolor{gray}{t}}_{\;\!\mathcolor{gray}{z}}$)\cite{chen-zhuChenZhuxieUndergraduate_courses2024}在有些效应中不可忽视且起关键作用:如其对线性晶体光学中的光学活性的贡献\cite{nelsonDerivingTransmissionReflection1995},以及非线性光学中基于$\bar{\bar{Q}}^{\;\!\mathcolor{gray}{t}}_{\;\!\mathcolor{gray}{z}}$的二阶和频\cite{bethuneOpticalQuadrupoleSumfrequency1976}。电四极子对光与物质相互作用的贡献,还会打破$\bar{D}^{\;\!\mathcolor{gray}{t}}_{\;\!\mathcolor{gray}{z}}$法向连续和$\bar{H}^{\;\!\mathcolor{gray}{t}}_{\;\!\mathcolor{gray}{z}}$切向连续边界条件,并与洛伦兹力的定义、(由唯二的无源齐次\cite{lakhtakiaGenesisPostConstraint2004}微分方程\bref{eq:Curl-EK,eq:Div-Bk}导出的)电磁场标/矢势\cite{chen-zhuChenZhuxieUndergraduate_courses2024}等一起,使$\bar{E}^{\;\!\mathcolor{gray}{t}}_{\;\!\mathcolor{gray}{z}},\bar{B}^{\;\!\mathcolor{gray}{t}}_{\;\!\mathcolor{gray}{z}}$而不是$\bar{E}^{\;\!\mathcolor{gray}{t}}_{\;\!\mathcolor{gray}{z}},\bar{H}^{\;\!\mathcolor{gray}{t}}_{\;\!\mathcolor{gray}{z}}$成为基本场\cite{nelsonDerivingTransmissionReflection1995},对应地,坡印亭矢量也需要修正为$\bar{E}^{\;\!\mathcolor{gray}{t}}_{\;\!\mathcolor{gray}{z}} \times \bar{B}^{\;\!\mathcolor{gray}{t}}_{\;\!\mathcolor{gray}{z}} \big/ {\symup{\varepsilon}}_0$\cite{nelsonGeneralizingPoyntingVector1996,loudonPropagationElectromagneticEnergy1997,richterPoyntingsTheoremEnergy2008}而不是$\bar{E}^{\;\!\mathcolor{gray}{t}}_{\;\!\mathcolor{gray}{z}} \times \bar{H}^{\;\!\mathcolor{gray}{t}}_{\;\!\mathcolor{gray}{z}}$。}构成时,定义为
\begin{subequations} \label{eq:cr-d}
\begin{align}
	\textcolor{Maroon}{\text{CR for electricity}}\text{:}&\hspace{0.5em} \bar{D}^{\;\!\mathcolor{gray}{t}}_{\;\!\mathcolor{gray}{z}} \hspace{-2.0em} &&= {\symup{\varepsilon}}_0 \bar{E}^{\;\!\mathcolor{gray}{t}}_{\;\!\mathcolor{gray}{z}} + \bar{P}^{\;\!\mathcolor{gray}{t}}_{\;\!\mathcolor{gray}{z}} = {\symup{\varepsilon}}_0 \bar{\bar{\delta}}^{\;\!\mathcolor{gray}{t}}~\widetilde *~\bar{E}^{\;\!\mathcolor{gray}{t}}_{\;\!\mathcolor{gray}{z}} + \bar{P}^{\;\!\mathcolor{gray}{t}}_{\;\!\mathcolor{gray}{z}} \label{cr-d1} \\ 
	& &&\xrightarrow[]{\bar{P}^{\;\!\mathcolor{gray}{t}}_{\;\!\mathcolor{gray}{z}} = \bar{P}^{\;\!\textcolor{Maroon}{\text{(1)}} \mathcolor{gray}{t}}_{\;\!\mathcolor{gray}{z}} + \bar{P}^{\;\!\textcolor{Maroon}{\text{NL}}, \mathcolor{gray}{t}}_{\;\!\mathcolor{gray}{z}} + } \left[ {\symup{\varepsilon}}_0 \bar{\bar{\delta}}^{\;\!\mathcolor{gray}{t}}~\widetilde *~\bar{E}^{\;\!\mathcolor{gray}{t}}_{\;\!\mathcolor{gray}{z}} + \bar{P}^{\;\!\textcolor{Maroon}{\text{(1)}} \mathcolor{gray}{t}}_{\;\!\mathcolor{gray}{z}} \right] + \bar{P}^{\;\!\textcolor{Maroon}{\text{NL}}, \mathcolor{gray}{t}}_{\;\!\mathcolor{gray}{z}} \label{cr-d2} \\ 
	& &&\xrightarrow[\displaystyle{ \bar{\bar{\varepsilon}}^{\;\!\textcolor{Maroon}{\text{(1)}} \mathcolor{gray}{t}}_{\;\!\textcolor{Maroon}{\text{r}}\mathcolor{gray}{z}} := \bar{\bar{\delta}}^{\;\!\mathcolor{gray}{t}} + \bar{\bar{\chi}}^{\;\!\textcolor{Maroon}{\text{(1)}}\mathcolor{gray}{t}}_{\;\!\textcolor{Maroon}{\text{f}} \mathcolor{gray}{z}}}]{\displaystyle{\bar{P}^{\;\!\textcolor{Maroon}{\text{(1)}} \mathcolor{gray}{t}}_{\;\!\mathcolor{gray}{z}} := \bar{\bar{\chi}}^{\;\!\textcolor{Maroon}{\text{(1)}}\mathcolor{gray}{t}}_{\;\!\textcolor{Maroon}{\text{f}} \mathcolor{gray}{z}} ~\widetilde *~\bar{E}^{\;\!\mathcolor{gray}{t}}_{\;\!\mathcolor{gray}{z}}}} {\symup{\varepsilon}}_0 \bar{\bar{\varepsilon}}^{\;\!\textcolor{Maroon}{\text{(1)}} \mathcolor{gray}{t}}_{\;\!\textcolor{Maroon}{\text{r}}\mathcolor{gray}{z}}~\widetilde *~\bar{E}^{\;\!\mathcolor{gray}{t}}_{\;\!\mathcolor{gray}{z}} + \bar{P}^{\;\!\textcolor{Maroon}{\text{NL}}, \mathcolor{gray}{t}}_{\;\!\mathcolor{gray}{z}} \label{cr-d3} \\ 
	& &&= \bar{\bar{\varepsilon}}^{\;\!\textcolor{Maroon}{\text{(1)}} \mathcolor{gray}{t}}_{\;\!\mathcolor{gray}{z}}~\widetilde *~\bar{E}^{\;\!\mathcolor{gray}{t}}_{\;\!\mathcolor{gray}{z}} + \bar{P}^{\;\!\textcolor{Maroon}{\text{NL}}, \mathcolor{gray}{t}}_{\;\!\mathcolor{gray}{z}} =: \bar{D}^{\;\!\textcolor{Maroon}{\text{(1)}} \mathcolor{gray}{t}}_{\;\!\mathcolor{gray}{z}} + \bar{D}^{\;\!\textcolor{Maroon}{\text{NL}}, \mathcolor{gray}{t}}_{\;\!\mathcolor{gray}{z}}~, \label{cr-d4}
\end{align}
\end{subequations}
关于其组成成分,电位移场 $\bar{D}^{\;\!\mathcolor{gray}{t}}_{\;\!\mathcolor{gray}{z}}$(直接/显示地)关于电场 $\bar{E}^{\;\!\mathcolor{gray}{t}}_{\;\!\mathcolor{gray}{z}}$\Footnote{电非线性,包括高频段的(非)共振非线性、低频低温\cite{lakhtakiaGenesisPostConstraint2004}段的铁电体/畴的电滞现象等。}、磁场 $\bar{H}^{\;\!\mathcolor{gray}{t}}_{\;\!\mathcolor{gray}{z}}$\Footnote{双各向异性中的磁$\to$电耦合(如果 $\bar{D}^{\;\!\mathcolor{gray}{t}}_{\;\!\mathcolor{gray}{z}}$ 中的该部分只是 $\bar{H}^{\;\!\mathcolor{gray}{t}}_{\;\!\mathcolor{gray}{z}}$ 的线性函数,则也可归结到线性项中)。}、应力 $\bar{T}^{\;\!\mathcolor{gray}{t}}_{\;\!\mathcolor{gray}{z}}$\Footnote{正逆压磁/磁致伸缩/磁弹效应(这里未作区分)。}等其他场量的非线性函数项,均由 $\bar{D}^{\;\!\textcolor{Maroon}{\text{NL}}, \mathcolor{gray}{t}}_{\;\!\mathcolor{gray}{z}} = {\symup{\varepsilon}}_0 \bar{M}^{\;\!\textcolor{Maroon}{\text{NL}}, \mathcolor{gray}{t}}_{\;\!\mathcolor{gray}{z}}$ 贡献;剩余的线性项,由 $\bar{B}^{\;\!\textcolor{Maroon}{\text{(1)}} \mathcolor{gray}{t}}_{\;\!\mathcolor{gray}{z}} = \bar{\bar{\mu}}^{\;\!\textcolor{Maroon}{\text{(1)}} \mathcolor{gray}{t}}_{\;\!\mathcolor{gray}{z}}~\widetilde *~\bar{H}^{\;\!\mathcolor{gray}{t}}_{\;\!\mathcolor{gray}{z}}$ 表示。

