% based on 北理工博士毕业论文 的 chapter2 https://tex.nju.edu.cn/project/user/10d0ed00-6f46-42fa-8a3a-18270214cfa3/a0ed1339-764d-4cc7-9ac4-d5d05b33da53

\chapter{晶体中的线性、非线性光学过程}

%\chapter{\protect\hyperlink{chap:\thechapter}{晶体中的线性、非线性光学过程的解析解}}
%\addtocontents{toc}{\protect\linkdest{chap:\thechapter}}
%\label{晶体中的线性、非线性光学过程的解析解}
%
%%\section{晶体中的电场混频方程组}
%\section{\protect\hyperlink{chap:\thesection}{晶体中的电场混频方程组}}
%\addtocontents{toc}{\protect\linkdest{chap:\thesection}}
%\label{晶体中的电场混频方程组}

晶体%\cref{eq:2-1,eq:r-2}
\bref{eq:r-2}

\begin{align} \label{eq:r-2}
	\left( k^{2}_{\omega} - \bar{k}^{\;\!\omega}\bar{k}^{\intercal}_{\omega} - k^{2}_{0\omega} \bar{\bar{\varepsilon}}^{\;\!\prime\omega}_{\mathrm{r} z} \right) \cdot \bar{g}^{\;\!\omega} = \bar{0}.
\end{align}

testsetse\cite{ossikovskiConstitutiveRelationsOptically2021}

%\begin{subequations} \label{eq:2-1}
%	\begin{align} 
%		\left\{\ \begin{aligned}\nabla \cdot \widetilde{\symbf D} &= {\widetilde \rho}_{\symup{f}}  \\\nabla \cdot \widetilde{\symbf B} &= 0 \\\nabla \times \widetilde{\symbf E} &= - \frac{\partial \widetilde{\symbf B}}{\partial t} \\ \nabla \times \widetilde{\symbf H} &= \widetilde{\symbf J}_{\symup{f}} + \frac{\partial \widetilde{\symbf D}}{\partial t} \end{aligned}\right. \xrightarrow[{\symup{inside\ material}}]{\left\{\ \begin{aligned}{\widetilde \rho}_{\symup{f}} &\rightarrow 0  \\ \widetilde{\symbf J}_{\symup{f}} &= {\widetilde \rho}_{\symup{f}} \cdot {\symbf v}_{\symup{f}} \end{aligned}\right.} \left\{\ \begin{aligned}\nabla \cdot \widetilde{\symbf D} &= 0 \\\nabla \cdot \widetilde{\symbf B} &= 0 \\\nabla \times \widetilde{\symbf E} &= - \frac{\partial \widetilde{\symbf B}}{\partial t} \\ \nabla \times \widetilde{\symbf H} &= \widetilde{\symbf J}_{\symup{f}} + \frac{\partial \widetilde{\symbf D}}{\partial t} \end{aligned}\right. ~,
%	\end{align}
%\end{subequations}

