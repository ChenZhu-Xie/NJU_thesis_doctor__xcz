%!TEX root = ..\njuthesis-sample.tex

% based on 北理工博士毕业论文 的 chapter1 https://tex.nju.edu.cn/project/user/10d0ed00-6f46-42fa-8a3a-18270214cfa3/a0ed1339-764d-4cc7-9ac4-d5d05b33da53

\marginLeft[-2.4em]{chap:tttttt}\chapter{前言}\label{chap:tttttt}

非线性光学,又名凝聚态光物理,属于凝聚态物理,因此继承了它的“众说纷纭”。由于量子多体系统各阶重整化导致的复杂涌现特性,每一个新的实现现象背后的机理可能没有单独的甚至确定的主导因素,即(可能)由多个因素级联/共同描述,以至于该领域自从诞生以来,就一沙一世界,群雄割据,体现在许多教科书的每一章,甚至都由不同的人撰写。这种局面既可以说是百花齐放,也可以说是比较“脏”。准确的用词我想应该是因多者异也,只能自下而上、推崇涌现论,从小模型到大模型,先入为主地:先知实验现象后建模去拟合。与物理学中的自上而下的、数学主导、推崇还原论,从大模型到小模型,先建模后去预测实验现象的领域,形成鲜明的对比。

\marginLeft[-2.4em]{sec:ttt}\section{拆开白盒:从还原论的视角解析非线性晶体光学}\label{sec:ttt}

本博士毕业大论文,属于“脏脏”凝聚态光物理中的一股“清流”。即反大部分道而行之,从完全脱离实验现象的顶层设计开始,一砖一瓦地构建起一个统一的大数学模型,然后再将该大模型做物理层面的各种简化和近似,并应用于各具体的实验场景,以对比和预测实验现象,以检验模型的正确性。\textbf{它刺激就刺激在,至于最终符不符合实验现象,遵循天意。}最终由实验物理学家的 CCD 相机(掷骰子和投硬币般地)决定。构建整个大模型的数学砖块,甚至代码砖块中,任何一个小砖块的不正确,可能都会使得最终大模型,不符合实验现象。但好的理论工程师喜欢挑战人类的极限,敢于做一些串联起来成功概率无限接近于 0\% 的事。\textbf{幸运的是,随着观众阅读进度的延续,至本文的末尾,本模型会自证至越来越倾向于 100\% 正确。}以至于我相信本文将最终成为填补非线性光学的理论物理研究领域的空白的一段佳话。这也是我分享这段条完整逻辑叙事链的快乐所在:从 0\% 的无力回天,至 100\% 的风致天成,君不见黄河之水天上来,奔流到海不复回。蓦然回首,那人那事已在灯火阑珊处。

\marginLeft[-2.4em]{sec:tt}\section{文字和符号的色彩的存在意义:符号即将用尽}\label{sec:tt}

严格来说,本文只是一场旅行、体验或者修行。即展示如何无限接近目标的同时,证明无法达到该目标。比如刚开始,本文的第一版发现数学字符和角标,甚至角标的位置,都不够用了\Footnote{许多粒子物理学家所熟悉的境况。},于是加了第一种黑色以外的颜色。随着理论的纵深和横向发展,后来为数学字符加了共 4 种颜色。再后来,从数学字符的颜色,拓展到文字的 1 种颜色,和文字的 4 种颜色。于是形成了现在这个,从文字到数学符号,均五颜六色的版本。从这个意义上讲,这像自下而上的、无限修 bug 的程序员,虽然他声称他的作品内核理念出于自上而下的还原论,但实际上从形式上看,颜色架构没有从一开始就确定好,所以就理论的发展过程而言,还是属于自下而上、工蜂和青蛙式勤能补拙的涌现论。尽管理论的内核,从出发点上,确实是属于还原论的。

从“可望而不可即”的另一个角度来看,描述电磁源受平面波微扰而动态分布的现代多极理论给出,{\one} 传统电磁场切向连续边界条件不再成立、{\two} 线性晶体光学一般是关于波矢的 6 次方程、{\three} 对例外点处的简并特征向量的通用批量计算包似乎在理论上就已经完全不可行,这将给所有从事以及想从事(甚至体块而非微纳)非线性晶体光学严格计算的理论研究者,当头一棒:连线性晶体光学都做不好,谈何非线性?

这也是 Berry 的名言之“没有肤浅的学问,只有肤浅的人”所映射的现实(而不是具体的人) --- 即这个世界在任何尺度,对于任何对象,在任何层次,均是“一花一世界”的。但对于每一朵花,均只“可以肤浅地理解,但本质上是不可计算,即不可完全理解的”。那么其实做物理甚至非物理学的任何方向都行:因为反正没法(完全)理解 --- 无论是谁,也无论他/她的功夫如何,修炼和打磨到什么程度,最终结果永远将是求之不得,以至于含恨而终的(如果是以此为志向的/想当军官的好士兵)。受限于人类的躯壳和逻辑、计算、寿命的有限性。

此既是爱因斯坦的“我们可以描述宇宙,但无法描述一片弯曲的树叶”,也是他的“这个世界最不可理解的地方是它是可理解的”,以及他的“大统一梦”和梦的破碎 --- 甚至任何方向的“小精尖梦”都只能原则上接近而无法实现。

在无法表达所有信息的时候,为什么不选择只表达关键信息,而不表示剩余边缘信息?--- 因为本文选择贯彻“全部信息”到底,不省略每一个角标、不省略每一步推导、不省略每一个反向链接,一方面是我想看看最终的公式长什么样子,另一方面,每一个物理量与字符、颜色、字体、位置的组合一一对应,使得“没有歧义”,且因此“放之四海皆准”:即在 a 处的定义,和它在 b,c,d 处的定义,是一致的。--- 这允许管中窥豹:即从一个带有丰富角标的物理量中,看出它的“来历”,看出它的“去路”,透过表面上的冗余,直奔它引伸出的上下文环境,和曾经裹挟的所有历史信息、它的本质。有趣的是本文所有的一切,都是逻辑、逻辑具象化后的对象,全然自然演化的结果,是“每次加一条音轨”的结果。静观其变到最后,就自编织成了一首“交响乐”。甚至作者本人,到最后也被排除在这幅杰作之外。

\marginLeft[-2.4em]{sec:t}\section{图表、公式、章节、引文旁的反向链接?}\label{sec:t}

{\one} 随机漫步的起点。

{\two} VIP 自然浮现:引用数量越多,对象就越重要。

{\three} 逻辑的灯塔:将文章织成一张网可以从任何地方阅读的网,使得几乎每一个对象的任何一个链接,都可以最终导向任何一个其他的被链接对象。
